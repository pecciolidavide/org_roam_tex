% Created 2026-02-07 Sat 19:31
% Intended LaTeX compiler: pdflatex
\documentclass[10pt]{article}
%% CREATO CON ORG - EMACS
\newcommand{\use}[2][]{\usepackage[#1]{#2}}
% PACCHETTI FONDAMENTLAI
\use[utf8]{inputenc}
\use[T1]{fontenc}
\use{graphicx}
\use{longtable}
\use{wrapfig}
\use{rotating}
\use[normalem]{ulem}
\use{amsmath}
\use{amsthm}
\use{amssymb}

\use{eucal} % Cambia mathcal{...}

\use{capt-of}
\use[italian]{babel}
\use[babel]{csquotes}
% bib la TEX lo carica in automatico org-cite
\use{microtype}
\use{lmodern}
\use{subfig} % sottofigure
\use{multicol} % due colonne
\use{lipsum} % lorem ipsum
\use{color} % colori in latex
\use{parskip} % rimuove l'indentazione dei nuovi paragrafi %% Add parbox=false to all new tcolorbox
\use{centernot}
\use[outline]{contour}\contourlength{3pt}
\use{fancyhdr}
\use{layout}
\use[most]{tcolorbox} % Riquadri colorati
\use{ifthen} % IFTHEN
\use{geometry}

% pacchetti matematica
\use{yhmath}
\use{dsfont}
\use{mathrsfs}
\use{cancel} % semplificare
\use{polynom} %divisione tra polinomi
\use{forest} % grafi ad albero
\use{booktabs} % tabelle
\use{commath} %simboli e differenziali
\use{bm} %bold
\use[fulladjust]{marginnote} %to use marginnote for date notes
\use{arrayjobx}%array
\use[intlimits]{empheq} % Riquadri colorati attorno alle equazioni
\use{mathtools}
\use{circuitikz} % Disegnare i circuiti
\use{mathtools}
\use{stmaryrd} % [[ \llbracket ]] \rrbracket
\use{bussproofs} % dimostrazioni

%%%%%%%%%%%%%


%%%% QUIVER
\newcommand{\duepunti}{\,\mathchar\numexpr"6000+`:\relax\,}
% A TikZ style for curved arrows of a fixed height, due to AndréC.
\tikzset{curve/.style={settings={#1},to path={(\tikztostart)
    .. controls ($(\tikztostart)!\pv{pos}!(\tikztotarget)!\pv{height}!270:(\tikztotarget)$)
    and ($(\tikztostart)!1-\pv{pos}!(\tikztotarget)!\pv{height}!270:(\tikztotarget)$)
    .. (\tikztotarget)\tikztonodes}},
    settings/.code={\tikzset{quiver/.cd,#1}
        \def\pv##1{\pgfkeysvalueof{/tikz/quiver/##1}}},
    quiver/.cd,pos/.initial=0.35,height/.initial=0}

% TikZ arrowhead/tail styles.
\tikzset{tail reversed/.code={\pgfsetarrowsstart{tikzcd to}}}
\tikzset{2tail/.code={\pgfsetarrowsstart{Implies[reversed]}}}
\tikzset{2tail reversed/.code={\pgfsetarrowsstart{Implies}}}
% TikZ arrow styles.
\tikzset{no body/.style={/tikz/dash pattern=on 0 off 1mm}}
%%%%%%%%%%


%% DEFINIZIONI COMANDI MATEMATICI
\let\sin\relax %TOGLIE LA DEFINIZIONE SU "\sin"

% cambia la definizione di empty set
% ---
\let\oldemptyset\emptyset
% ---
% \let\emptyset\varnothing
% ---
% \let\emptyset\relax
% \newcommand{\emptyset}{\text{\textnormal{\O}}}
% ---

\DeclareMathOperator{\bounded}{bd}
\DeclareMathOperator{\sin}{sen}
\DeclareMathOperator{\epi}{Epi}
\DeclareMathOperator{\cl}{cl}
\DeclareMathOperator{\graph}{graph}
\DeclareMathOperator{\arcsec}{arcsec}
\DeclareMathOperator{\arccot}{arccot}
\DeclareMathOperator{\arccsc}{arccsc}
\DeclareMathOperator{\spettro}{Spettro}
\DeclareMathOperator{\nulls}{nullspace}
\DeclareMathOperator{\dom}{dom}
\DeclareMathOperator{\ar}{ar}
\DeclareMathOperator{\const}{Const}
\DeclareMathOperator{\fun}{Fun}
\DeclareMathOperator{\rel}{Rel}
\DeclareMathOperator{\altezza}{ht}
\let\det\relax %TOGLIE LA DEFINIZIONE SU "\det"
\DeclareMathOperator{\det}{det}
\DeclareMathOperator{\End}{End}
\DeclareMathOperator{\gl}{GL}
\def\Id{\mathrm{Id}}
\def\id{\mathrm{id}}
\DeclareMathOperator{\I}{\mathds{1}}
\DeclareMathOperator{\II}{II}
\DeclareMathOperator{\rank}{rank}
\DeclareMathOperator{\tr}{tr}
\DeclareMathOperator{\tc}{t.c.}
\DeclareMathOperator{\T}{T}
\DeclareMathOperator{\var}{Var}
\DeclareMathOperator{\cov}{Cov}
\DeclareMathOperator{\st}{st}
\DeclareMathOperator{\mon}{Mon}
\newcommand{\card}[1]{\left\vert #1 \right\vert}
\newcommand{\trasposta}[1]{\prescript{\text{T}}{}{#1}}
\newcommand{\1}{\mathds{1}}
\newcommand{\R}{\mathds{R}}
\newcommand{\diesis}{\#}
\newcommand{\bemolle}{\flat}
\newcommand{\nonstandard}[1]{\prescript{*}{}{#1}}
\newcommand{\starR}{\nonstandard{\R}}
\newcommand{\borel}{\mathscr{B}}
\newcommand{\lebesgue}[1]{\mathscr{L}\left(#1\right)}
\newcommand{\media}{\mathds{E}}
\newcommand{\K}{\mathds{K}}
\newcommand{\A}{\mathds{A}}
\newcommand{\Q}{\mathds{Q}}
\newcommand{\N}{\mathds{N}}
\newcommand{\C}{\mathds{C}}
\newcommand{\Z}{\mathds{Z}}
\newcommand{\qo}{\hspace{1em}\text{q.o.}\,}
\renewcommand{\tilde}[1]{\widetilde{#1}}
\renewcommand{\parallel}{\mathrel{/\mkern-5mu/}}
\newcommand{\parti}[2][]{\wp_{#1}(#2)}
\newcommand{\diff}[1]{\operatorname{d}_{#1}}
\let\oldvec\vec
\renewcommand{\vec}[1]{\overrightarrow{\vphantom{i}#1}}
\newcommand{\floor}[1]{\left\lfloor #1 \right\rfloor}
\newcommand{\cat}[1]{\mathbf{#1}}
\newcommand{\dfreccia}[1]{\xrightarrow{\ #1 \ }}
\newcommand{\sfreccia}[1]{\xleftarrow{\ #1 \ }}
\newcommand{\formalsum}[2]{{\sum_{#1}^{#2}}{\vphantom{\sum}}'}
\newcommand{\minim}[2]{\mu_{#1}\, \left(#2\right)}
\newcommand{\concat}{\null^{\frown}} % concatenazione di stringe
\newcommand{\godelcode}[1]{\langle\!\langle #1 \rangle\!\rangle}
\newcommand{\godeldec}[1]{(\!(#1)\!)}
\newcommand{\termcode}[1]{\ulcorner #1\urcorner}
\newcommand{\partialto}{\dashrightarrow}
\newcommand{\restricted}{\upharpoonright}
\newcommand{\embeds}{\precsim}
\newcommand{\surjects}{\twoheadrightarrow}
\newcommand{\equipotenti}{\asymp}
%% \newcommand{\dotplus}{\mathbin{\dot{+}}} %% A quanto pare esiste già
\newcommand{\bigdot}{\mathbin{\boldsymbol{\cdot}}}
\newcommand{\dotexp}[1]{^{.#1}}
\newcommand{\conv}{\mathbin{*}}
\newcommand{\convolution}[2]{(#1\conv #2)}
\newcommand{\nil}{\mathfrak{N}}
\newcommand{\divisore}{\mathrel{|}}
\newcommand{\simplesso}[1]{\mathrm{e}_{#1}}

\renewcommand{\iff}{\mathrel{\longleftrightarrow}} %% Notazione Logica.
\newcommand{\oldiff}{\mathrel{\Longleftrightarrow}}
\renewcommand{\implies}{\mathrel{\rightarrow}} %% Notazione Logica
\newcommand{\oldimplies}{\mathrel{\Longrightarrow}}
\renewcommand{\impliedby}{\mathrel{\leftarrow}} %% Notazione Logica
\newcommand{\oldimpliedby}{\mathrel{\Longleftarrow}}

\newcommand{\IFF}{\quad\Longleftrightarrow\quad}
\newcommand{\IMPLICA}{\quad\Longrightarrow\quad}


\renewcommand{\descriptionlabel}[1]{\hspace{\labelsep}\normalfont #1} % remove bold from description


%% Definizione di Divergenza di K-L

\DeclarePairedDelimiterX{\infdivx}[2]{(}{)}{%
  #1\;\delimsize\|\;#2%
}
\newcommand{\kldiv}{D_{KL}\infdivx}

%% Definizione di \dotminus

\makeatletter
\newcommand{\dotminus}{\mathbin{\text{\@dotminus}}}

\newcommand{\@dotminus}{%
  \ooalign{\hidewidth\raise1ex\hbox{.}\hidewidth\cr$\m@th-$\cr}%
}
\makeatother

%tramite i prossimi due comandi posso decidere come scrivere i logaritmi naturali in tutti i documenti: ho infatti eliminato qualsiasi differenza tra "ln" e "log": se si vuole qualcosa di diverso bisogna inserire manualmente il tutto
\let\ln\relax
\DeclareMathOperator{\ln}{ln}
\let\log\relax
\DeclareMathOperator{\log}{log}
%%%%%%

%% NUOVI COMANDI
\newcommand{\straniero}[1]{\textit{#1}} %parole straniere
\newcommand{\titolo}[1]{\textsc{#1}} %titoli
\newcommand{\qedd}{\tag*{$\blacksquare$}} %qed per ambienti matemastici
\renewcommand{\qedsymbol}{$\blacksquare$} %modifica colore qed
\newcommand{\ooverline}[1]{\overline{\overline{#1}}}
\newcommand{\circoletto}[1]{\left(#1\right)^{\text{o}}}
%
\newcommand{\qmatrice}[1]{\begin{pmatrix}
#1_{11} & \cdots & #1_{1n}\\
\vdots & \ddots & \vdots \\
#1_{m1} & \cdots & #1_{mn}
\end{pmatrix}}
%
\newcommand{\parentesi}[2]{%
\underset{#1}{\underbrace{#2}}%
}
%
\newcommand{\norma}[1]{% Norma
\left\lVert#1\right\rVert%
}
\newcommand{\scalare}[2]{% Scalare
\left\langle #1, #2\right\rangle
}
%%%%%

%% RESTRIZIONI
\newcommand{\referenze}[2]{
        \phantomsection{}#2\textsuperscript{\textcolor{blue}{\textbf{#1}}}
}

\let\restriction\relax

\def\restriction#1#2{\mathchoice
              {\setbox1\hbox{${\displaystyle #1}_{\scriptstyle #2}$}
              \restrictionaux{#1}{#2}}
              {\setbox1\hbox{${\textstyle #1}_{\scriptstyle #2}$}
              \restrictionaux{#1}{#2}}
              {\setbox1\hbox{${\scriptstyle #1}_{\scriptscriptstyle #2}$}
              \restrictionaux{#1}{#2}}
              {\setbox1\hbox{${\scriptscriptstyle #1}_{\scriptscriptstyle #2}$}
              \restrictionaux{#1}{#2}}}
\def\restrictionaux#1#2{{#1\,\smash{\vrule height .8\ht1 depth .85\dp1}}_{\,#2}}
%%%%%%%%%%%

%%% FORMATTAZIONE FOOTNOTEMARK

\def\footnotemarkformatting#1{[#1]}
\renewcommand{\thefootnote}{\footnotemarkformatting{\arabic{footnote}}}

%% SEZIONE GRAFICA
\use{tikz}
\usetikzlibrary{matrix, patterns, calc, decorations.pathreplacing, hobby, decorations.markings, decorations.pathmorphing, babel}
\use{tikz-3dplot}
\use{mathrsfs} %per geogebra
\use{tikz-cd}
\tikzset
{
  %surface/.style={fill=black!10, shading=ball,fill opacity=0.4},
  plane/.style={black,pattern=north east lines},
  curve/.style={black,line width=0.5mm},
  dritto/.style={decoration={markings,mark=at position 0.5 with {\arrow{Stealth}}}, postaction=decorate},
  rovescio/.style={decoration={markings,mark=at position 0.5 with {\arrow{Stealth[reversed]}}}, postaction=decorate}
}
\use{pgfplots} % stampare le funzioni
        \pgfplotsset{/pgf/number format/use comma,compat=1.15}
        %\pgfplotsset{compat=1.15} %per geogebra
        \usepgfplotslibrary{fillbetween, polar}
%%%%%%

%% CITAZIONI
\use{lineno}

\newcommand{\citazione}[1]{%
  \begin{quotation}
  \begin{linenumbers}
  \modulolinenumbers[5]
  \begingroup
  \setlength{\parindent}{0cm}
  \noindent #1
  \endgroup
  \end{linenumbers}
  \end{quotation}\setcounter{linenumber}{1}
  }
%%%%%%

%%%%%%%%%%%%%%%%%%%%%%%%%%%%%%%%%%%%%%%%%%%%
%%%%%%%%%%%%%%%%%%%%%%%%%%%%%%%%%%%%%%%%%%%%

%% AMS THM

\theoremstyle{definition}% default
\newtheorem{thm}{Teorema}[section]
\newtheorem{lem}[thm]{Lemma}
\newtheorem{prop}[thm]{Proposizione}
\newtheorem{cor}[thm]{Corollario}
\newtheorem{esempio}[thm]{Esempio}
\theoremstyle{plain}
\newtheorem{definizione}[thm]{Definizione}
\theoremstyle{remark}
\newtheorem*{oss}{Osservazione}


%%%%%%%%%%%%%%%%%%%%%%%%%%%%%%%%%%%%%%%%%%%%
%%%%%%%%%%%%%%%%%%%%%%%%%%%%%%%%%%%%%%%%%%%%

\use{hyperref}
\hypersetup{%
        pdfauthor={Davide Peccioli},
        pdfsubject={},
        allcolors=black,
        citecolor=black,
%	colorlinks=true,
        bookmarksopen=true}
\setcounter{secnumdepth}{0} % rimuove i numeri di sezione senza rimuovere le ref
\renewcommand{\href}[2]{\textcolor{blue}{#2}} % disabilita il comando href
\use{enotez} %
\setenotez{%
 mark-format = \footnotemarkformatting % Mette i numeri tra parentesi quadre%
}\let\footnote=\endnote % rende tutte le note a pié pagina come delle note a fine file 


\let\olddocument\document % modifico l'ambiende documenti per non dover stampare \printendnote
\let\oldenddocument\enddocument
\renewenvironment{document}%
{%
  \olddocument
}{%
  \printendnotes\oldenddocument
}
\renewcommand{\thethm}{\arabic{thm}}

\usepackage[hyperref]{biblatex}
\addbibresource{~/Documents/org/roam/bib/master.bib}
\author{Davide Peccioli}
\date{\today}
\title{Analisi non-standard}
\begin{document}

Sia \(\mathcal{L}\) il \href{20250130162057-linguaggio_del_prim_ordine.org}{linguaggio del prim'ordine} che contiene:
\begin{itemize}
\item per ogni \href{20250203161110-numeri_naturali_sono_ordinali.org}{\(n \in \omega\)} e per ogni \(X \subseteq \R^{n}\), un \href{20250130162057-linguaggio_del_prim_ordine.org}{simbolo di relazione} \(X\) di \href{20250130162057-linguaggio_del_prim_ordine.org}{arietà} \(n\);
\item per ogni \(n \in \omega\) e per ogni \(f:\R^{n}\to \R\), un \href{20250130162057-linguaggio_del_prim_ordine.org}{simbolo di funzione} \(f\) di \href{20250130162057-linguaggio_del_prim_ordine.org}{arietà} \(n\).
\end{itemize}

Il \uline{modello standard} dell'analisi \(\R\) è la \(\mathcal{L}\)-\href{20250131103035-struttura_del_prim_ordine.org}{struttura} di \href{20250131103035-struttura_del_prim_ordine.org}{dominio} \(\R\), e in cui ciascun simbolo di \(\mathcal{L}\) è intepretato nel modo naturale.
\section{Teorema}
\label{sec:org87742bc}

Esiste una \href{20250212102253-sottostruttura_elementare.org}{estensione elementare} di \(\R\), detta \(\starR\), tale che \(\R\neq \starR\).

Inoltre \(\starR\) è un \href{20250611115646-campo_ordinato.org}{campo ordinato}, e viene quindi detto \uline{campo degli iperreali}
\section{Notazione}
\label{sec:orga4cf965}

Se \(X\) è un simbolo di relazione di \(\mathcal{L}\), si denoterà con \(\nonstandard{X}\) la sua interpretazione in \(\starR\); se \(f\) è un simbolo di funzione di \(\mathcal{L}\), si denoterà con \(\nonstandard{f}\) la sua interpretazione in \(\starR\).

Quando sarà chiaro dal contesto, si ometterà l'asterisco.
\section{Numeri iperreali}
\label{sec:orgeba653f}
Gli elementi di \(\starR\) sono detti \uline{numeri iperreali}; gli elementi di \(R \subseteq \starR\) sono detti \uline{iperreali standard}, mentre gli elementi di \(\starR\setminus \R\) sono detti \uline{iperreali nonstandard}.

Sia \(c \in \starR\):
\begin{itemize}
\item \(c\) è detto \uline{infinitesimo} se per ogni \(\varepsilon \in \R\), \(\varepsilon>0\): \href{20250131122913-soddisfazione_di_una_formula.org}{\(\starR\vDash |c|<\varepsilon\)};
\item \(c\) è detto \uline{infinito} se per ogni \(k \in \R\): \(\starR\vdash k<|c|\), ed è detto \uline{finito} altrimenti.
\end{itemize}

Se \(c \in \starR\) è infinito, allora \(c^{-1}\) è infinitesimale.

Tutti i reali standard sono finiti, e \(0\) è l'unico infinitesimo standard.
\section{Proprietà}
\label{sec:orgd1ab96e}

\subsection{Lemma 1}
\label{sec:org84ac24a}

Gli infinitesimi sono chiusi rispetto a somma, prodotto e moltiplicazione per un numero reale.
\subsection{Lemma 2}
\label{sec:orga680872}

\begin{enumerate}
\item Esistono iperreali infiniti e infinitesimi diversi da \(0\).

\item Inoltre, per ogni iperreale finito \(c\) esiste un unico iperreale standard \(b\) tale che \(b-c\) è infinitesimo.
\end{enumerate}
\subsubsection{Dimostrazione}
\label{sec:orge0f426d}

2 implica 1; infatti preso \(c \in \starR\setminus\R\), allora ci sono due possibilità
\begin{itemize}
\item \(c\) è infinito, allora \(c^{-1}\) è infinitesimo; inoltre \(c^{-1}\neq 0\), poiché
\begin{equation*}
  \R\vDash \forall\,x\ (x\neq 0);
\end{equation*}
\item \(c\) è finito, e allora per 2. esiste un unico \(b \in \R\) tale che \(b-c\) sia infinitesimo; inoltre \(b-c\neq 0\), poiché
\begin{equation*}
  \R\vDash \forall\,x\, \forall\,y (x=y\iff x-y=0)
\end{equation*}
Inoltre, \((b-c)^{-1}\), che esiste poiché \(\R\vDash \forall\,x\ (x\neq 0 \implies \exists\,y\ (xy=1))\), è infinito.
\end{itemize}

Per 2., si ponga per \(c \in \starR\) finito:
\begin{equation*}
b\coloneqq \inf\set{a \in \R\mid c<a}.
\end{equation*}
\subsection{Campo degli iperreali non è archimedeo}
\label{sec:org89d127f}
L'esistenza di iperreali infiniti mostra che \(\starR\) non è un \href{20250320150051-gruppo_archimedeo.org}{campo archimedeo}.
\subsection{Caratterizzazione dei limiti per gli iperreali}
\label{sec:orgded6b86}
Sia \(f:\R\to \R\) e siano \(a,\ell \in \R\). Allora:
\begin{enumerate}
\item \(\lim_{x\to + \infty} f(x)=+\infty\) se e solo se \(\nonstandard{f}(c)\) è positivo e infinito per ogni \(c>0\) infinito.
\item \(\lim_{x\to+\infty} f(x) = \ell\) se e solo se \(\nonstandard{f}(c)\approx \ell\) per ogni \(c>0\) infinito.
\item \(\lim_{x\to a} f(x) = + \infty\) se e solo se per ogni \(a\neq c\approx a\) si ha che \(\nonstandard{f}(c)>0\) è infinito.
\item \(\lim_{x\to a}(fx) = \ell\) se e solo se per ogni \(a\neq c\approx a\), \(f(c)\approx \ell\).
\end{enumerate}
\subsection{Caratterizzazione funzioni continue tramite gli iperreali.}
\label{sec:org849a99b}
Sia \(f:\R\to \R\). Sono equivalenti:
\begin{enumerate}
\item \(f\) è \href{20250103103252-funzione_continua.org}{continua};
\item per ogni \(a,c \in \starR\) finiti, se \(a\approx c\) allora \(\nonstandard{f}(a)\approx \nonstandard{f}(c)\).
\end{enumerate}
\subsection{Caratterizzazione funzioni uniformemente contiune tramite gli iperreali}
\label{sec:org12b314c}
Sia \(f:\R\to \R\). Sono equivalenti:
\begin{enumerate}
\item \(f\) è \href{20250611135127-funzione_uniformemente_continua.org}{uniformemente continua};
\item per ogni \(a,c \in \starR\), se \(a\approx c\) allora \(\nonstandard{f}(a)\approx \nonstandard{f}(c)\).
\end{enumerate}
\section{Altre definizioni}
\label{sec:orgf83eab8}

\subsection{Monade di un iperreale}
\label{sec:org9beccf2}
In virtà del \ref{sec:org84ac24a}, è possibile definire una \href{20250113110148-relazione_di_equivalenza.org}{relazione di equivalenza} su \(\starR\): scriveremo \(a\approx b\) se \(|a-b|\) è infinitesimo. La \href{20250114100810-quoziente_rispetto_a_relazione_di_equivalenza.org}{classe di equivalenza} di \(c\) è chiamata \uline{monade di \(c\)}.

Per il \ref{sec:orga680872}, se \(c\) è un iperreale finito, allora vi è un unico numero reale nella monade di \(c\); questo è detto \uline{parte standard} di \(c\), ed è denotato con \(\operatorname{st}(c)\).
\end{document}
