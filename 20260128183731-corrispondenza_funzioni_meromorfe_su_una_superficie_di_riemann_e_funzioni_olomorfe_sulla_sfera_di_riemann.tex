% Created 2026-02-07 Sat 19:34
% Intended LaTeX compiler: pdflatex
\documentclass[10pt]{article}
%% CREATO CON ORG - EMACS
\newcommand{\use}[2][]{\usepackage[#1]{#2}}
% PACCHETTI FONDAMENTLAI
\use[utf8]{inputenc}
\use[T1]{fontenc}
\use{graphicx}
\use{longtable}
\use{wrapfig}
\use{rotating}
\use[normalem]{ulem}
\use{amsmath}
\use{amsthm}
\use{amssymb}

\use{eucal} % Cambia mathcal{...}

\use{capt-of}
\use[italian]{babel}
\use[babel]{csquotes}
% bib la TEX lo carica in automatico org-cite
\use{microtype}
\use{lmodern}
\use{subfig} % sottofigure
\use{multicol} % due colonne
\use{lipsum} % lorem ipsum
\use{color} % colori in latex
\use{parskip} % rimuove l'indentazione dei nuovi paragrafi %% Add parbox=false to all new tcolorbox
\use{centernot}
\use[outline]{contour}\contourlength{3pt}
\use{fancyhdr}
\use{layout}
\use[most]{tcolorbox} % Riquadri colorati
\use{ifthen} % IFTHEN
\use{geometry}

% pacchetti matematica
\use{yhmath}
\use{dsfont}
\use{mathrsfs}
\use{cancel} % semplificare
\use{polynom} %divisione tra polinomi
\use{forest} % grafi ad albero
\use{booktabs} % tabelle
\use{commath} %simboli e differenziali
\use{bm} %bold
\use[fulladjust]{marginnote} %to use marginnote for date notes
\use{arrayjobx}%array
\use[intlimits]{empheq} % Riquadri colorati attorno alle equazioni
\use{mathtools}
\use{circuitikz} % Disegnare i circuiti
\use{mathtools}
\use{stmaryrd} % [[ \llbracket ]] \rrbracket
\use{bussproofs} % dimostrazioni

%%%%%%%%%%%%%


%%%% QUIVER
\newcommand{\duepunti}{\,\mathchar\numexpr"6000+`:\relax\,}
% A TikZ style for curved arrows of a fixed height, due to AndréC.
\tikzset{curve/.style={settings={#1},to path={(\tikztostart)
    .. controls ($(\tikztostart)!\pv{pos}!(\tikztotarget)!\pv{height}!270:(\tikztotarget)$)
    and ($(\tikztostart)!1-\pv{pos}!(\tikztotarget)!\pv{height}!270:(\tikztotarget)$)
    .. (\tikztotarget)\tikztonodes}},
    settings/.code={\tikzset{quiver/.cd,#1}
        \def\pv##1{\pgfkeysvalueof{/tikz/quiver/##1}}},
    quiver/.cd,pos/.initial=0.35,height/.initial=0}

% TikZ arrowhead/tail styles.
\tikzset{tail reversed/.code={\pgfsetarrowsstart{tikzcd to}}}
\tikzset{2tail/.code={\pgfsetarrowsstart{Implies[reversed]}}}
\tikzset{2tail reversed/.code={\pgfsetarrowsstart{Implies}}}
% TikZ arrow styles.
\tikzset{no body/.style={/tikz/dash pattern=on 0 off 1mm}}
%%%%%%%%%%


%% DEFINIZIONI COMANDI MATEMATICI
\let\sin\relax %TOGLIE LA DEFINIZIONE SU "\sin"

% cambia la definizione di empty set
% ---
\let\oldemptyset\emptyset
% ---
% \let\emptyset\varnothing
% ---
% \let\emptyset\relax
% \newcommand{\emptyset}{\text{\textnormal{\O}}}
% ---

\DeclareMathOperator{\bounded}{bd}
\DeclareMathOperator{\sin}{sen}
\DeclareMathOperator{\epi}{Epi}
\DeclareMathOperator{\cl}{cl}
\DeclareMathOperator{\graph}{graph}
\DeclareMathOperator{\arcsec}{arcsec}
\DeclareMathOperator{\arccot}{arccot}
\DeclareMathOperator{\arccsc}{arccsc}
\DeclareMathOperator{\spettro}{Spettro}
\DeclareMathOperator{\nulls}{nullspace}
\DeclareMathOperator{\dom}{dom}
\DeclareMathOperator{\ar}{ar}
\DeclareMathOperator{\const}{Const}
\DeclareMathOperator{\fun}{Fun}
\DeclareMathOperator{\rel}{Rel}
\DeclareMathOperator{\altezza}{ht}
\let\det\relax %TOGLIE LA DEFINIZIONE SU "\det"
\DeclareMathOperator{\det}{det}
\DeclareMathOperator{\End}{End}
\DeclareMathOperator{\gl}{GL}
\def\Id{\mathrm{Id}}
\def\id{\mathrm{id}}
\DeclareMathOperator{\I}{\mathds{1}}
\DeclareMathOperator{\II}{II}
\DeclareMathOperator{\rank}{rank}
\DeclareMathOperator{\tr}{tr}
\DeclareMathOperator{\tc}{t.c.}
\DeclareMathOperator{\T}{T}
\DeclareMathOperator{\var}{Var}
\DeclareMathOperator{\cov}{Cov}
\DeclareMathOperator{\st}{st}
\DeclareMathOperator{\mon}{Mon}
\newcommand{\card}[1]{\left\vert #1 \right\vert}
\newcommand{\trasposta}[1]{\prescript{\text{T}}{}{#1}}
\newcommand{\1}{\mathds{1}}
\newcommand{\R}{\mathds{R}}
\newcommand{\diesis}{\#}
\newcommand{\bemolle}{\flat}
\newcommand{\nonstandard}[1]{\prescript{*}{}{#1}}
\newcommand{\starR}{\nonstandard{\R}}
\newcommand{\borel}{\mathscr{B}}
\newcommand{\lebesgue}[1]{\mathscr{L}\left(#1\right)}
\newcommand{\media}{\mathds{E}}
\newcommand{\K}{\mathds{K}}
\newcommand{\A}{\mathds{A}}
\newcommand{\Q}{\mathds{Q}}
\newcommand{\N}{\mathds{N}}
\newcommand{\C}{\mathds{C}}
\newcommand{\Z}{\mathds{Z}}
\newcommand{\qo}{\hspace{1em}\text{q.o.}\,}
\renewcommand{\tilde}[1]{\widetilde{#1}}
\renewcommand{\parallel}{\mathrel{/\mkern-5mu/}}
\newcommand{\parti}[2][]{\wp_{#1}(#2)}
\newcommand{\diff}[1]{\operatorname{d}_{#1}}
\let\oldvec\vec
\renewcommand{\vec}[1]{\overrightarrow{\vphantom{i}#1}}
\newcommand{\floor}[1]{\left\lfloor #1 \right\rfloor}
\newcommand{\cat}[1]{\mathbf{#1}}
\newcommand{\dfreccia}[1]{\xrightarrow{\ #1 \ }}
\newcommand{\sfreccia}[1]{\xleftarrow{\ #1 \ }}
\newcommand{\formalsum}[2]{{\sum_{#1}^{#2}}{\vphantom{\sum}}'}
\newcommand{\minim}[2]{\mu_{#1}\, \left(#2\right)}
\newcommand{\concat}{\null^{\frown}} % concatenazione di stringe
\newcommand{\godelcode}[1]{\langle\!\langle #1 \rangle\!\rangle}
\newcommand{\godeldec}[1]{(\!(#1)\!)}
\newcommand{\termcode}[1]{\ulcorner #1\urcorner}
\newcommand{\partialto}{\dashrightarrow}
\newcommand{\restricted}{\upharpoonright}
\newcommand{\embeds}{\precsim}
\newcommand{\surjects}{\twoheadrightarrow}
\newcommand{\equipotenti}{\asymp}
%% \newcommand{\dotplus}{\mathbin{\dot{+}}} %% A quanto pare esiste già
\newcommand{\bigdot}{\mathbin{\boldsymbol{\cdot}}}
\newcommand{\dotexp}[1]{^{.#1}}
\newcommand{\conv}{\mathbin{*}}
\newcommand{\convolution}[2]{(#1\conv #2)}
\newcommand{\nil}{\mathfrak{N}}
\newcommand{\divisore}{\mathrel{|}}
\newcommand{\simplesso}[1]{\mathrm{e}_{#1}}

\renewcommand{\iff}{\mathrel{\longleftrightarrow}} %% Notazione Logica.
\newcommand{\oldiff}{\mathrel{\Longleftrightarrow}}
\renewcommand{\implies}{\mathrel{\rightarrow}} %% Notazione Logica
\newcommand{\oldimplies}{\mathrel{\Longrightarrow}}
\renewcommand{\impliedby}{\mathrel{\leftarrow}} %% Notazione Logica
\newcommand{\oldimpliedby}{\mathrel{\Longleftarrow}}

\newcommand{\IFF}{\quad\Longleftrightarrow\quad}
\newcommand{\IMPLICA}{\quad\Longrightarrow\quad}


\renewcommand{\descriptionlabel}[1]{\hspace{\labelsep}\normalfont #1} % remove bold from description


%% Definizione di Divergenza di K-L

\DeclarePairedDelimiterX{\infdivx}[2]{(}{)}{%
  #1\;\delimsize\|\;#2%
}
\newcommand{\kldiv}{D_{KL}\infdivx}

%% Definizione di \dotminus

\makeatletter
\newcommand{\dotminus}{\mathbin{\text{\@dotminus}}}

\newcommand{\@dotminus}{%
  \ooalign{\hidewidth\raise1ex\hbox{.}\hidewidth\cr$\m@th-$\cr}%
}
\makeatother

%tramite i prossimi due comandi posso decidere come scrivere i logaritmi naturali in tutti i documenti: ho infatti eliminato qualsiasi differenza tra "ln" e "log": se si vuole qualcosa di diverso bisogna inserire manualmente il tutto
\let\ln\relax
\DeclareMathOperator{\ln}{ln}
\let\log\relax
\DeclareMathOperator{\log}{log}
%%%%%%

%% NUOVI COMANDI
\newcommand{\straniero}[1]{\textit{#1}} %parole straniere
\newcommand{\titolo}[1]{\textsc{#1}} %titoli
\newcommand{\qedd}{\tag*{$\blacksquare$}} %qed per ambienti matemastici
\renewcommand{\qedsymbol}{$\blacksquare$} %modifica colore qed
\newcommand{\ooverline}[1]{\overline{\overline{#1}}}
\newcommand{\circoletto}[1]{\left(#1\right)^{\text{o}}}
%
\newcommand{\qmatrice}[1]{\begin{pmatrix}
#1_{11} & \cdots & #1_{1n}\\
\vdots & \ddots & \vdots \\
#1_{m1} & \cdots & #1_{mn}
\end{pmatrix}}
%
\newcommand{\parentesi}[2]{%
\underset{#1}{\underbrace{#2}}%
}
%
\newcommand{\norma}[1]{% Norma
\left\lVert#1\right\rVert%
}
\newcommand{\scalare}[2]{% Scalare
\left\langle #1, #2\right\rangle
}
%%%%%

%% RESTRIZIONI
\newcommand{\referenze}[2]{
        \phantomsection{}#2\textsuperscript{\textcolor{blue}{\textbf{#1}}}
}

\let\restriction\relax

\def\restriction#1#2{\mathchoice
              {\setbox1\hbox{${\displaystyle #1}_{\scriptstyle #2}$}
              \restrictionaux{#1}{#2}}
              {\setbox1\hbox{${\textstyle #1}_{\scriptstyle #2}$}
              \restrictionaux{#1}{#2}}
              {\setbox1\hbox{${\scriptstyle #1}_{\scriptscriptstyle #2}$}
              \restrictionaux{#1}{#2}}
              {\setbox1\hbox{${\scriptscriptstyle #1}_{\scriptscriptstyle #2}$}
              \restrictionaux{#1}{#2}}}
\def\restrictionaux#1#2{{#1\,\smash{\vrule height .8\ht1 depth .85\dp1}}_{\,#2}}
%%%%%%%%%%%

%%% FORMATTAZIONE FOOTNOTEMARK

\def\footnotemarkformatting#1{[#1]}
\renewcommand{\thefootnote}{\footnotemarkformatting{\arabic{footnote}}}

%% SEZIONE GRAFICA
\use{tikz}
\usetikzlibrary{matrix, patterns, calc, decorations.pathreplacing, hobby, decorations.markings, decorations.pathmorphing, babel}
\use{tikz-3dplot}
\use{mathrsfs} %per geogebra
\use{tikz-cd}
\tikzset
{
  %surface/.style={fill=black!10, shading=ball,fill opacity=0.4},
  plane/.style={black,pattern=north east lines},
  curve/.style={black,line width=0.5mm},
  dritto/.style={decoration={markings,mark=at position 0.5 with {\arrow{Stealth}}}, postaction=decorate},
  rovescio/.style={decoration={markings,mark=at position 0.5 with {\arrow{Stealth[reversed]}}}, postaction=decorate}
}
\use{pgfplots} % stampare le funzioni
        \pgfplotsset{/pgf/number format/use comma,compat=1.15}
        %\pgfplotsset{compat=1.15} %per geogebra
        \usepgfplotslibrary{fillbetween, polar}
%%%%%%

%% CITAZIONI
\use{lineno}

\newcommand{\citazione}[1]{%
  \begin{quotation}
  \begin{linenumbers}
  \modulolinenumbers[5]
  \begingroup
  \setlength{\parindent}{0cm}
  \noindent #1
  \endgroup
  \end{linenumbers}
  \end{quotation}\setcounter{linenumber}{1}
  }
%%%%%%

%%%%%%%%%%%%%%%%%%%%%%%%%%%%%%%%%%%%%%%%%%%%
%%%%%%%%%%%%%%%%%%%%%%%%%%%%%%%%%%%%%%%%%%%%

%% AMS THM

\theoremstyle{definition}% default
\newtheorem{thm}{Teorema}[section]
\newtheorem{lem}[thm]{Lemma}
\newtheorem{prop}[thm]{Proposizione}
\newtheorem{cor}[thm]{Corollario}
\newtheorem{esempio}[thm]{Esempio}
\theoremstyle{plain}
\newtheorem{definizione}[thm]{Definizione}
\theoremstyle{remark}
\newtheorem*{oss}{Osservazione}


%%%%%%%%%%%%%%%%%%%%%%%%%%%%%%%%%%%%%%%%%%%%
%%%%%%%%%%%%%%%%%%%%%%%%%%%%%%%%%%%%%%%%%%%%

\use{hyperref}
\hypersetup{%
        pdfauthor={Davide Peccioli},
        pdfsubject={},
        allcolors=black,
        citecolor=black,
%	colorlinks=true,
        bookmarksopen=true}
\setcounter{secnumdepth}{0} % rimuove i numeri di sezione senza rimuovere le ref
\renewcommand{\href}[2]{\textcolor{blue}{#2}} % disabilita il comando href
\use{enotez} %
\setenotez{%
 mark-format = \footnotemarkformatting % Mette i numeri tra parentesi quadre%
}\let\footnote=\endnote % rende tutte le note a pié pagina come delle note a fine file 


\let\olddocument\document % modifico l'ambiende documenti per non dover stampare \printendnote
\let\oldenddocument\enddocument
\renewenvironment{document}%
{%
  \olddocument
}{%
  \printendnotes\oldenddocument
}
\renewcommand{\thethm}{\arabic{thm}}

\usepackage[hyperref]{biblatex}
\addbibresource{~/Documents/org/roam/bib/master.bib}
\author{Davide Peccioli}
\date{\today}
\title{}
\begin{document}

\section{Corrispondenza funzioni meromorfe su una superficie di Riemann e funzioni olomorfe sulla Sfera di Riemann}
\label{sec:org6dfc674}
Sia \(X\) una \href{20260127112828-superficie_di_riemann.org}{superficie di Riemann}, sia \(\C_{\infty}\) la \href{20260127112905-sfera_di_riemann.org}{Sfera di Riemann}, e sia \(\mathds{P}^{1}_{\C}\) il \href{20241231115051-spazio_proiettivo.org}{piano proiettivo} \href{20260127112924-esempi_fondamentali_di_varieta_complesse.org}{complesso}, \href{20260128144717-isomorfismo_tra_superfici_di_riemann.org}{biolomorfo} a \(\C_{\infty}\)\footnote{Vedi ``\href{20260128181135-sfera_di_riemann_biolomorfa_al_piano_proiettivo_complesso.org}{Sfera di Riemann biolomorfa al piano proiettivo complesso}''}.

\begin{prop}
C'è una corrispondenza biunivoca tra:
\begin{itemize}
\item le \href{20260128144105-funzione_meromorfa_su_una_superficie_di_riemann.org}{funzioni meromorfe su \(X\) a valori in \(\C\)}: \(\mathcal{M}_{X}(X)\)
\item le \href{20260128143822-funzione_olomorfa_su_una_superficie_di_riemann.org}{funzioni olomorfe} \(F:X\to \C_{\infty} \cong \mathds{P}^{1}_{\C}\) non costanti in \(\infty\).
\end{itemize}
\begin{equation*}
\set{\parbox{10em}{\centering%
	funzioni meromorfe \(f\) su \(X\) a valori in \(\C\)%
}} \xleftrightarrow{\hspace{1em} 1:1 \hspace{1em}} %
\set{\parbox{10em}{\centering%
	mappe olomorfe \(F:X\to \C_{\infty}\cong \mathds{P}^{1}_{\C}\) non costanti in \(\infty\)
}}.
\end{equation*}
\end{prop}
\begin{proof}
(\(\longrightarrow\)): Sia \(f \in \mathcal{M}_{X}(X)\): si definisce
\begin{align*}
F: X &\longrightarrow \C_{\infty}\\
x &\longmapsto \begin{cases}%
f(x) \in \C & \text{\(x\) non è un polo}\\%
\infty & \text{\(x\) è un polo}.%
\end{cases}
\end{align*}
o, equivalentemente,
\begin{align*}
F: X &\longrightarrow \mathds{P}^{1}_{\C}\\
x &\longmapsto \begin{cases}%
(f(x):1) \in \C & \text{\(x\) non è un polo}\\%
(1:0) & \text{\(x\) è un polo}.%
\end{cases}
\end{align*}

Verifichiamo che \(F\) è \href{20260128143822-funzione_olomorfa_su_una_superficie_di_riemann.org}{olomorfa}:
\begin{itemize}
\item Se \(x_{0} \in X\) non è un polo per \(f\), allora \(F(x_{0}) \in U_{1} = \set{z_{1}\neq 0} \in \mathds{P}^{1}_{\C}\), quindi componendo le carte:
\begin{equation*}
\begin{tikzcd}[ampersand replacement=\&,row sep=scriptsize]
        X \&\& {U_1 \subseteq \mathds{P}_{\C}^1} \&\& \C \\
        {x_0} \&\& {F(x_0) = (f(x)\duepunti 1)} \&\& {f(x)}
        \arrow["F", from=1-1, to=1-3]
        \arrow["{(x\duepunti y)\mapsto \frac{x}{y}}", from=1-3, to=1-5]
        \arrow[maps to, from=2-1, to=2-3]
        \arrow[maps to, from=2-3, to=2-5]
\end{tikzcd}
\end{equation*}
e si ottiene un mappa olomorfa per ipotesi.
\item Se \(x_{0} \in X\) è un polo per \(f\), allora in un intorno \(U_{x_{0}} \subseteq X\) di \(x_{0}\)\footnote{Siccome \(f\) è \href{20260128144105-funzione_meromorfa_su_una_superficie_di_riemann.org}{meromorfa}, allora l'insieme delle singolarità e degli zeri di \(f\) è \href{20260128123515-sottoinsieme_discreto.org}{discreto}, e pertanto è possibile richiedere che dentro ad \(U_{x_{0}}\) non ci siano altri zeri di \(f\).} si ha che
\begin{equation*}
  \frac{1}{f(x)} \eqqcolon g(x)
\end{equation*}
è olomorfa e nulla in \(x_{0}\)\footnote{Questo segue dalle proprietà dell'\href{20260128144433-ordine_di_una_funzione_meromorfa.org}{ordine delle funzioni meromorfe}.}. Quindi, per ogni \(x \in U_{x_{0}} \setminus\set{x_{0}}\):
\begin{equation*}
  F(x) = (f(x):1) = \left(\frac{1}{g(x)} : 1\right) = (1:g(x))
\end{equation*}
mentre per \(x=x_{0}\):
\begin{equation*}
  F(x) = (1:0) = (1:g(x_{0}))
\end{equation*}

Quindi, per ogni \(x \in U_{x_{0}}\), \(F(x) \in U_{0} \coloneqq \set{z_{0}\neq 0} \subseteq \mathds{P}^{1}_{\C}\), e si ottiene:
\begin{equation*}
\begin{tikzcd}[ampersand replacement=\&,row sep=scriptsize]
        {U_{x_0}\subseteq X} \&\& {U_0 \subseteq \mathds{P}_{\C}^1} \&\& \C \\
        x \&\& {F(x) = \big(1\duepunti g(x)\big)} \&\& {g(x)}
        \arrow["F", from=1-1, to=1-3]
        \arrow["{(x\duepunti y)\mapsto \frac{y}{x}}", from=1-3, to=1-5]
        \arrow[maps to, from=2-1, to=2-3]
        \arrow[maps to, from=2-3, to=2-5]
\end{tikzcd}
\end{equation*}
che è una funzione olomorfa.
\end{itemize}

Quindi \(F\) è olomorfa, ed è la funzione associata ad \(f \in \mathcal{M}_{X}(X)\).

(\(\longleftarrow\)): Sia \(F: X\to \mathds{P}^{1}_{\C}\) una funzione olomorfa non costante in \((1:0)\).

Sia \(S\coloneqq F^{-1}\left((1:0)\right)\): per il \href{20260128144016-principio_di_identita_per_funzioni_olomorfe_su_superfici_di_riemann.org}{principio di identità}, questo è un sottoinsieme discreto.

Inoltre, l'\href{20250202190147-immagine_punto_a_punto_di_due_classi.org}{immagine} \(F(X\setminus S) \subseteq U_{1} \coloneqq \set{z_{1}\neq 0} \subseteq \mathds{P}^{1}_{\C}\): quindi si definisce \(f: X\setminus S\to \C\) come segue:
\begin{equation*}
\begin{tikzcd}[ampersand replacement=\&,row sep=scriptsize]
	{X\setminus S} \&\& {U_1 \subseteq \mathds{P}_{\C}^1} \&\& \C
	\arrow["F", from=1-1, to=1-3]
	\arrow["f"', bend right=24pt, from=1-1, to=1-5]
	\arrow["{(z_0\duepunti z_1)\mapsto \frac{z_0}{z_1}}", from=1-3, to=1-5]
\end{tikzcd}
\end{equation*}

In particolare si ha che
\begin{equation*}
\restriction{F}{X\setminus S} = \big(f(x):1\big).
\end{equation*}

Verificare che:
\begin{enumerate}
\item \(f \in \mathcal{O}(X\setminus S)\);
\item \(f \in \mathcal{M}_{X}(X)\).
\end{enumerate}
\end{proof}

\begin{oss}
La biiezione si restringe a
\begin{equation*}
\set{\parbox{10em}{\centering%
	funzioni meromorfe \(f\) su \(X\) a valori in \(\C\) non costanti
}} \xleftrightarrow{\hspace{1em} 1:1 \hspace{1em}} %
\set{\parbox{10em}{\centering%
	mappe olomorfe \(F:X\to \C_{\infty}\cong \mathds{P}^{1}_{\C}\) non costanti
}}.
\end{equation*}
\end{oss}
\subsection{Esistenza funzioni meromorfe a valori complessi per superfici di Riemann compatte}
\label{sec:org5aecf8b}
\begin{prop}
Se \(X\) è una \href{20260127112828-superficie_di_riemann.org}{superficie di Riemann} \href{20250103163701-spazio_topologico_compatto.org}{compatta}, sono fatti equivalenti:
\begin{enumerate}
\item esiste \(f \in\mathcal{M}_{X}(M)\) \href{20260128144105-funzione_meromorfa_su_una_superficie_di_riemann.org}{funzione meromorfa} \textbf{non costante};
\item esiste \(F: X \to \mathds{P}^{1}_{\C}\) \href{20260128143822-funzione_olomorfa_su_una_superficie_di_riemann.org}{funzione olomorfa} e \href{20241213105600-funzione_suriettiva.org}{suriettiva}.
\end{enumerate}
\end{prop}

\begin{proof}
Questo segue banalmente da:
\begin{itemize}
\item \hyperref[sec:org6dfc674]{Corrispondenza funzioni meromorfe su una superficie di Riemann e funzioni olomorfe sulla Sfera di Riemann}
\item \href{20260128183856-funzione_olomorfa_tra_superfici_di_riemann_condominio_compatto.org}{Proprietà funzione olomorfa non costante tra superfici di Riemann con dominio compatto}
\end{itemize}
\end{proof}
\end{document}
