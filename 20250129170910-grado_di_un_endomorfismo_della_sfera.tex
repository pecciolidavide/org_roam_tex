% Created 2026-02-07 Sat 19:30
% Intended LaTeX compiler: pdflatex
\documentclass[10pt]{article}
%% CREATO CON ORG - EMACS
\newcommand{\use}[2][]{\usepackage[#1]{#2}}
% PACCHETTI FONDAMENTLAI
\use[utf8]{inputenc}
\use[T1]{fontenc}
\use{graphicx}
\use{longtable}
\use{wrapfig}
\use{rotating}
\use[normalem]{ulem}
\use{amsmath}
\use{amsthm}
\use{amssymb}

\use{eucal} % Cambia mathcal{...}

\use{capt-of}
\use[italian]{babel}
\use[babel]{csquotes}
% bib la TEX lo carica in automatico org-cite
\use{microtype}
\use{lmodern}
\use{subfig} % sottofigure
\use{multicol} % due colonne
\use{lipsum} % lorem ipsum
\use{color} % colori in latex
\use{parskip} % rimuove l'indentazione dei nuovi paragrafi %% Add parbox=false to all new tcolorbox
\use{centernot}
\use[outline]{contour}\contourlength{3pt}
\use{fancyhdr}
\use{layout}
\use[most]{tcolorbox} % Riquadri colorati
\use{ifthen} % IFTHEN
\use{geometry}

% pacchetti matematica
\use{yhmath}
\use{dsfont}
\use{mathrsfs}
\use{cancel} % semplificare
\use{polynom} %divisione tra polinomi
\use{forest} % grafi ad albero
\use{booktabs} % tabelle
\use{commath} %simboli e differenziali
\use{bm} %bold
\use[fulladjust]{marginnote} %to use marginnote for date notes
\use{arrayjobx}%array
\use[intlimits]{empheq} % Riquadri colorati attorno alle equazioni
\use{mathtools}
\use{circuitikz} % Disegnare i circuiti
\use{mathtools}
\use{stmaryrd} % [[ \llbracket ]] \rrbracket
\use{bussproofs} % dimostrazioni

%%%%%%%%%%%%%


%%%% QUIVER
\newcommand{\duepunti}{\,\mathchar\numexpr"6000+`:\relax\,}
% A TikZ style for curved arrows of a fixed height, due to AndréC.
\tikzset{curve/.style={settings={#1},to path={(\tikztostart)
    .. controls ($(\tikztostart)!\pv{pos}!(\tikztotarget)!\pv{height}!270:(\tikztotarget)$)
    and ($(\tikztostart)!1-\pv{pos}!(\tikztotarget)!\pv{height}!270:(\tikztotarget)$)
    .. (\tikztotarget)\tikztonodes}},
    settings/.code={\tikzset{quiver/.cd,#1}
        \def\pv##1{\pgfkeysvalueof{/tikz/quiver/##1}}},
    quiver/.cd,pos/.initial=0.35,height/.initial=0}

% TikZ arrowhead/tail styles.
\tikzset{tail reversed/.code={\pgfsetarrowsstart{tikzcd to}}}
\tikzset{2tail/.code={\pgfsetarrowsstart{Implies[reversed]}}}
\tikzset{2tail reversed/.code={\pgfsetarrowsstart{Implies}}}
% TikZ arrow styles.
\tikzset{no body/.style={/tikz/dash pattern=on 0 off 1mm}}
%%%%%%%%%%


%% DEFINIZIONI COMANDI MATEMATICI
\let\sin\relax %TOGLIE LA DEFINIZIONE SU "\sin"

% cambia la definizione di empty set
% ---
\let\oldemptyset\emptyset
% ---
% \let\emptyset\varnothing
% ---
% \let\emptyset\relax
% \newcommand{\emptyset}{\text{\textnormal{\O}}}
% ---

\DeclareMathOperator{\bounded}{bd}
\DeclareMathOperator{\sin}{sen}
\DeclareMathOperator{\epi}{Epi}
\DeclareMathOperator{\cl}{cl}
\DeclareMathOperator{\graph}{graph}
\DeclareMathOperator{\arcsec}{arcsec}
\DeclareMathOperator{\arccot}{arccot}
\DeclareMathOperator{\arccsc}{arccsc}
\DeclareMathOperator{\spettro}{Spettro}
\DeclareMathOperator{\nulls}{nullspace}
\DeclareMathOperator{\dom}{dom}
\DeclareMathOperator{\ar}{ar}
\DeclareMathOperator{\const}{Const}
\DeclareMathOperator{\fun}{Fun}
\DeclareMathOperator{\rel}{Rel}
\DeclareMathOperator{\altezza}{ht}
\let\det\relax %TOGLIE LA DEFINIZIONE SU "\det"
\DeclareMathOperator{\det}{det}
\DeclareMathOperator{\End}{End}
\DeclareMathOperator{\gl}{GL}
\def\Id{\mathrm{Id}}
\def\id{\mathrm{id}}
\DeclareMathOperator{\I}{\mathds{1}}
\DeclareMathOperator{\II}{II}
\DeclareMathOperator{\rank}{rank}
\DeclareMathOperator{\tr}{tr}
\DeclareMathOperator{\tc}{t.c.}
\DeclareMathOperator{\T}{T}
\DeclareMathOperator{\var}{Var}
\DeclareMathOperator{\cov}{Cov}
\DeclareMathOperator{\st}{st}
\DeclareMathOperator{\mon}{Mon}
\newcommand{\card}[1]{\left\vert #1 \right\vert}
\newcommand{\trasposta}[1]{\prescript{\text{T}}{}{#1}}
\newcommand{\1}{\mathds{1}}
\newcommand{\R}{\mathds{R}}
\newcommand{\diesis}{\#}
\newcommand{\bemolle}{\flat}
\newcommand{\nonstandard}[1]{\prescript{*}{}{#1}}
\newcommand{\starR}{\nonstandard{\R}}
\newcommand{\borel}{\mathscr{B}}
\newcommand{\lebesgue}[1]{\mathscr{L}\left(#1\right)}
\newcommand{\media}{\mathds{E}}
\newcommand{\K}{\mathds{K}}
\newcommand{\A}{\mathds{A}}
\newcommand{\Q}{\mathds{Q}}
\newcommand{\N}{\mathds{N}}
\newcommand{\C}{\mathds{C}}
\newcommand{\Z}{\mathds{Z}}
\newcommand{\qo}{\hspace{1em}\text{q.o.}\,}
\renewcommand{\tilde}[1]{\widetilde{#1}}
\renewcommand{\parallel}{\mathrel{/\mkern-5mu/}}
\newcommand{\parti}[2][]{\wp_{#1}(#2)}
\newcommand{\diff}[1]{\operatorname{d}_{#1}}
\let\oldvec\vec
\renewcommand{\vec}[1]{\overrightarrow{\vphantom{i}#1}}
\newcommand{\floor}[1]{\left\lfloor #1 \right\rfloor}
\newcommand{\cat}[1]{\mathbf{#1}}
\newcommand{\dfreccia}[1]{\xrightarrow{\ #1 \ }}
\newcommand{\sfreccia}[1]{\xleftarrow{\ #1 \ }}
\newcommand{\formalsum}[2]{{\sum_{#1}^{#2}}{\vphantom{\sum}}'}
\newcommand{\minim}[2]{\mu_{#1}\, \left(#2\right)}
\newcommand{\concat}{\null^{\frown}} % concatenazione di stringe
\newcommand{\godelcode}[1]{\langle\!\langle #1 \rangle\!\rangle}
\newcommand{\godeldec}[1]{(\!(#1)\!)}
\newcommand{\termcode}[1]{\ulcorner #1\urcorner}
\newcommand{\partialto}{\dashrightarrow}
\newcommand{\restricted}{\upharpoonright}
\newcommand{\embeds}{\precsim}
\newcommand{\surjects}{\twoheadrightarrow}
\newcommand{\equipotenti}{\asymp}
%% \newcommand{\dotplus}{\mathbin{\dot{+}}} %% A quanto pare esiste già
\newcommand{\bigdot}{\mathbin{\boldsymbol{\cdot}}}
\newcommand{\dotexp}[1]{^{.#1}}
\newcommand{\conv}{\mathbin{*}}
\newcommand{\convolution}[2]{(#1\conv #2)}
\newcommand{\nil}{\mathfrak{N}}
\newcommand{\divisore}{\mathrel{|}}
\newcommand{\simplesso}[1]{\mathrm{e}_{#1}}

\renewcommand{\iff}{\mathrel{\longleftrightarrow}} %% Notazione Logica.
\newcommand{\oldiff}{\mathrel{\Longleftrightarrow}}
\renewcommand{\implies}{\mathrel{\rightarrow}} %% Notazione Logica
\newcommand{\oldimplies}{\mathrel{\Longrightarrow}}
\renewcommand{\impliedby}{\mathrel{\leftarrow}} %% Notazione Logica
\newcommand{\oldimpliedby}{\mathrel{\Longleftarrow}}

\newcommand{\IFF}{\quad\Longleftrightarrow\quad}
\newcommand{\IMPLICA}{\quad\Longrightarrow\quad}


\renewcommand{\descriptionlabel}[1]{\hspace{\labelsep}\normalfont #1} % remove bold from description


%% Definizione di Divergenza di K-L

\DeclarePairedDelimiterX{\infdivx}[2]{(}{)}{%
  #1\;\delimsize\|\;#2%
}
\newcommand{\kldiv}{D_{KL}\infdivx}

%% Definizione di \dotminus

\makeatletter
\newcommand{\dotminus}{\mathbin{\text{\@dotminus}}}

\newcommand{\@dotminus}{%
  \ooalign{\hidewidth\raise1ex\hbox{.}\hidewidth\cr$\m@th-$\cr}%
}
\makeatother

%tramite i prossimi due comandi posso decidere come scrivere i logaritmi naturali in tutti i documenti: ho infatti eliminato qualsiasi differenza tra "ln" e "log": se si vuole qualcosa di diverso bisogna inserire manualmente il tutto
\let\ln\relax
\DeclareMathOperator{\ln}{ln}
\let\log\relax
\DeclareMathOperator{\log}{log}
%%%%%%

%% NUOVI COMANDI
\newcommand{\straniero}[1]{\textit{#1}} %parole straniere
\newcommand{\titolo}[1]{\textsc{#1}} %titoli
\newcommand{\qedd}{\tag*{$\blacksquare$}} %qed per ambienti matemastici
\renewcommand{\qedsymbol}{$\blacksquare$} %modifica colore qed
\newcommand{\ooverline}[1]{\overline{\overline{#1}}}
\newcommand{\circoletto}[1]{\left(#1\right)^{\text{o}}}
%
\newcommand{\qmatrice}[1]{\begin{pmatrix}
#1_{11} & \cdots & #1_{1n}\\
\vdots & \ddots & \vdots \\
#1_{m1} & \cdots & #1_{mn}
\end{pmatrix}}
%
\newcommand{\parentesi}[2]{%
\underset{#1}{\underbrace{#2}}%
}
%
\newcommand{\norma}[1]{% Norma
\left\lVert#1\right\rVert%
}
\newcommand{\scalare}[2]{% Scalare
\left\langle #1, #2\right\rangle
}
%%%%%

%% RESTRIZIONI
\newcommand{\referenze}[2]{
        \phantomsection{}#2\textsuperscript{\textcolor{blue}{\textbf{#1}}}
}

\let\restriction\relax

\def\restriction#1#2{\mathchoice
              {\setbox1\hbox{${\displaystyle #1}_{\scriptstyle #2}$}
              \restrictionaux{#1}{#2}}
              {\setbox1\hbox{${\textstyle #1}_{\scriptstyle #2}$}
              \restrictionaux{#1}{#2}}
              {\setbox1\hbox{${\scriptstyle #1}_{\scriptscriptstyle #2}$}
              \restrictionaux{#1}{#2}}
              {\setbox1\hbox{${\scriptscriptstyle #1}_{\scriptscriptstyle #2}$}
              \restrictionaux{#1}{#2}}}
\def\restrictionaux#1#2{{#1\,\smash{\vrule height .8\ht1 depth .85\dp1}}_{\,#2}}
%%%%%%%%%%%

%%% FORMATTAZIONE FOOTNOTEMARK

\def\footnotemarkformatting#1{[#1]}
\renewcommand{\thefootnote}{\footnotemarkformatting{\arabic{footnote}}}

%% SEZIONE GRAFICA
\use{tikz}
\usetikzlibrary{matrix, patterns, calc, decorations.pathreplacing, hobby, decorations.markings, decorations.pathmorphing, babel}
\use{tikz-3dplot}
\use{mathrsfs} %per geogebra
\use{tikz-cd}
\tikzset
{
  %surface/.style={fill=black!10, shading=ball,fill opacity=0.4},
  plane/.style={black,pattern=north east lines},
  curve/.style={black,line width=0.5mm},
  dritto/.style={decoration={markings,mark=at position 0.5 with {\arrow{Stealth}}}, postaction=decorate},
  rovescio/.style={decoration={markings,mark=at position 0.5 with {\arrow{Stealth[reversed]}}}, postaction=decorate}
}
\use{pgfplots} % stampare le funzioni
        \pgfplotsset{/pgf/number format/use comma,compat=1.15}
        %\pgfplotsset{compat=1.15} %per geogebra
        \usepgfplotslibrary{fillbetween, polar}
%%%%%%

%% CITAZIONI
\use{lineno}

\newcommand{\citazione}[1]{%
  \begin{quotation}
  \begin{linenumbers}
  \modulolinenumbers[5]
  \begingroup
  \setlength{\parindent}{0cm}
  \noindent #1
  \endgroup
  \end{linenumbers}
  \end{quotation}\setcounter{linenumber}{1}
  }
%%%%%%

%%%%%%%%%%%%%%%%%%%%%%%%%%%%%%%%%%%%%%%%%%%%
%%%%%%%%%%%%%%%%%%%%%%%%%%%%%%%%%%%%%%%%%%%%

%% AMS THM

\theoremstyle{definition}% default
\newtheorem{thm}{Teorema}[section]
\newtheorem{lem}[thm]{Lemma}
\newtheorem{prop}[thm]{Proposizione}
\newtheorem{cor}[thm]{Corollario}
\newtheorem{esempio}[thm]{Esempio}
\theoremstyle{plain}
\newtheorem{definizione}[thm]{Definizione}
\theoremstyle{remark}
\newtheorem*{oss}{Osservazione}


%%%%%%%%%%%%%%%%%%%%%%%%%%%%%%%%%%%%%%%%%%%%
%%%%%%%%%%%%%%%%%%%%%%%%%%%%%%%%%%%%%%%%%%%%

\use{hyperref}
\hypersetup{%
        pdfauthor={Davide Peccioli},
        pdfsubject={},
        allcolors=black,
        citecolor=black,
%	colorlinks=true,
        bookmarksopen=true}
\setcounter{secnumdepth}{0} % rimuove i numeri di sezione senza rimuovere le ref
\renewcommand{\href}[2]{\textcolor{blue}{#2}} % disabilita il comando href
\use{enotez} %
\setenotez{%
 mark-format = \footnotemarkformatting % Mette i numeri tra parentesi quadre%
}\let\footnote=\endnote % rende tutte le note a pié pagina come delle note a fine file 


\let\olddocument\document % modifico l'ambiende documenti per non dover stampare \printendnote
\let\oldenddocument\enddocument
\renewenvironment{document}%
{%
  \olddocument
}{%
  \printendnotes\oldenddocument
}
\renewcommand{\thethm}{\arabic{thm}}

\usepackage[hyperref]{biblatex}
\addbibresource{~/Documents/org/roam/bib/master.bib}
\author{Davide Peccioli}
\date{\today}
\title{Grado di un endomorfismo della sfera}
\begin{document}

Sia \(R=\Z\) un \href{20241219112842-pid.org}{PID} e sia \(\mathds{S}^{n}\) la \href{20250115150754-sfera_n_dimensionale.org}{sfera \(n\)-dimensionale}
\begin{definizione}
Sia \(f: \mathds{S}^{n} \longrightarrow \mathds{S}^{n}\) una \href{20250103103252-funzione_continua.org}{funzione continua}. Allora applicando il \href{20241205131958-funtore.org}{funtore} \href{20250123115927-funtore_di_omologia_singolare.org}{di omologia singolare}, questa induce un \href{20241206115416-morfismi_r_moduli.org}{morfismo}
\begin{equation*}
f_{\star}: H_{n}(\mathds{S}^{n}) \longrightarrow H_{n}(\mathds{S}^{n})
\end{equation*}
\href{20250127162702-calcolo_dell_omologia_singolare_della_sfera_e_dell_omologia_singolare_relativa_del_disco_rispetto_alla_sfera.org}{ovvero} \(f_{\star} : \Z \longrightarrow \Z\) \href{20241206115531-morfismo_di_gruppi.org}{morfismo di gruppi}, tale che \(f_{\star}(x) = m\cdot x\) per qualche \(m \in \Z\).
Si definisce il grado di \(f\) come
\begin{equation*}
\operatorname{deg}f\coloneqq m
\end{equation*}
\end{definizione}
\begin{lem}
Siano \(f,g: \mathds{S}^{n}\longrightarrow\mathds{S}^{n}\) \href{20250103103252-funzione_continua.org}{funzioni continue}.
\begin{enumerate}
\item Se \(f\sim g\) \href{20250121094654-omotopia_tra_funzioni_continue.org}{funzioni omotope}, allora \(\operatorname{deg} f= \operatorname{deg}g\);
\item Se \(f \sim \Id\), allora \(\operatorname{deg}f = 1\);
\item Se \(f\) è omotopa alla funzione costante, allora \(\operatorname{deg}f = 0\);
\item \(\operatorname{deg}(f\circ g) = \operatorname{deg}f\cdot \operatorname{deg}g\)
\end{enumerate}
\end{lem}

\begin{proof}
\begin{enumerate}
\item Se due funzioni sono omotope, allora \href{20250122155528-teorema_dell_invarianza_per_omotopia.org}{inducono la stessa mappa in omologia}.
\item Per funtorialità, \((\Id_{\mathcal{S}^{n}})_{\star} = \Id_{\Z}\) e pertanto \(\deg \Id = 1\).
\item Se \(g\) è costante, allora \(g\) fattorizza come segue:
\begin{equation*}
\begin{tikzcd}[ampersand replacement=\&]
        \& {\set{p}} \\
        {\mathds{S}^n} \&\& {\mathds{S}^n}
        \arrow[hook, from=1-2, to=2-3]
        \arrow["g", from=2-1, to=1-2]
        \arrow["g"', from=2-1, to=2-3]
\end{tikzcd}
\end{equation*}
e per funtorialità
\begin{equation*}
\begin{tikzcd}[ampersand replacement=\&]
        \& {H_n(\set{p}) = 0} \\
        {H_n(\mathds{S}^n)} \&\& {H_n(\mathds{S}^n)}
        \arrow[hook, from=1-2, to=2-3]
        \arrow["0", from=2-1, to=1-2]
        \arrow["g"', from=2-1, to=2-3]
\end{tikzcd}
\end{equation*}
ottenendo che \(g_{\star} = 0\), \(\deg g = 0\).
\item Applicando il \href{20241205131958-funtore.org}{funtore} \href{20250123115927-funtore_di_omologia_singolare.org}{di omologia singolare}:
\begin{equation*}
\begin{tikzcd}[ampersand replacement=\&,row sep=tiny]
        {\mathds{S}^n} \& {\mathds{S}^n} \& {\mathds{S}^n} \\
        \\
        \\
        \\
        {H_n(\mathds{S}^n)} \& {H_n(\mathds{S}^n)} \& {H_n(\mathds{S}^n)} \\
        x \& {\deg g\cdot x} \\
        \& y \& {\deg f \cdot y} \\
        x \&\& {(\deg g)\cdot(\deg f)\cdot x}
        \arrow["g", from=1-1, to=1-2]
        \arrow["f", from=1-2, to=1-3]
        \arrow[Rightarrow, from=1-2, to=5-2]
        \arrow["{g_\star}", from=5-1, to=5-2]
        \arrow["{f_\star}", from=5-2, to=5-3]
        \arrow[maps to, from=6-1, to=6-2]
        \arrow[maps to, from=7-2, to=7-3]
        \arrow["{(f\circ g)_{\star}}", maps to, from=8-1, to=8-3]
\end{tikzcd}
\end{equation*}
e pertanto la tesi.\qedhere
\end{enumerate}
\end{proof}
\section{Grado dell'endomorfismo di riflessione sulla sfera}
\label{sec:org92a8c5e}
\begin{prop}
Sia
\begin{align*}
r: \mathds{S}^{n} &\longrightarrow \mathds{S}^{n}\\
(x_{1},\dots,x_{n+1}) &\longmapsto (x_{1},\dots,x_{n},-x_{n+1})
\end{align*}
Allora \(\deg r = -1\)
\end{prop}

\begin{proof}
Consideriamo i due insiemi
\begin{align*}
A^{+} &= \set{(x_{0},\dots,x_{n}) \in \mathds{S}^{n} \mid x_{n}\ge 0 }\\
A^{-} &= \set{(x_{0},\dots,x_{n}) \in \mathds{S}^{n} \mid x_{n}\le 0 }\\
E &= A^{+} \cap A^{-}
\end{align*}
Allora \(E\) è \href{20250111142332-omeomorfismo.org}{omeomorfo} alla \href{20250115150754-sfera_n_dimensionale.org}{sfera \(\mathds{S}^{n-1}\)} e \(\restriction{r}{E} = \Id_{E}\).

Se consideriamo ora, per \(\varepsilon>0\):
\begin{align*}
A^{+}_{\varepsilon} &= \set{(x_{0},\dots,x_{n}) \in \mathds{S}^{n} \mid x_{n} > - \varepsilon }\\
A^{-} &= \set{(x_{0},\dots,x_{n}) \in \mathds{S}^{n} \mid x_{n}< \varepsilon }\\
B &= A^{+}_{\varepsilon} \cap A^{-}_{\varepsilon}
\end{align*}
allora \((\mathds{S}^{n}, A^{+}_{\varepsilon}, A^{-}_{\varepsilon}, B)\) soddisfa \href{20250128132648-teorema_di_mayer_vietoris.org}{Mayer-Vietoris}, e inoltre la mappa
\begin{equation*}
\Id: (\mathds{S}^{n}, A^{+}, A^{-}, E)\longrightarrow (\mathds{S}^{n}, A^{ +}_{\varepsilon}, A^{-}_{\vraepsilon}, B)
\end{equation*}
dà origine a tutti \href{20250122155727-retratto_di_deformazione_di_uno_spazio_topologico.org}{retratti di deformazione}.

\textbf{Si ripercorre la \href{20250128132648-teorema_di_mayer_vietoris.org}{dimostrazione del Teorema di Mayer-Vietoris}}.
Ottengo le due \href{20250120183640-sec_di_complessi_di_catene.org}{SEC}:\footnote{Vedi \href{20260203110150-complesso_di_catene_somma.org}{somma} e \href{20260204100611-somma_diretta_di_complessi_di_catene.org}{somma diretta} di \href{20250120163114-complesso_di_catene.org}{complessi di catene}.}
\begin{equation*}
\begin{tikzcd}[ampersand replacement=\&]
	0 \& {\mathcal{S}_{\bullet}(E)} \& {\mathcal{S}_{\bullet}(A^+)\oplus \mathcal{S}_{\bullet}(A^-)} \&\& {\mathcal{S}_{\bullet}(\mathds{S}^n)} \& 0 \\
	\\
	0 \& {\mathcal{S}_{\bullet}(E)} \& {\mathcal{S}_{\bullet}(A^-)\oplus \mathcal{S}_{\bullet}(A^+)} \&\& {\mathcal{S}_{\bullet}(\mathds{S}^n)} \& 0
	\arrow[from=1-1, to=1-2]
	\arrow[from=1-2, to=1-3]
	\arrow["{(j_1)_{\diesis} - (j_2)_{\diesis}}", from=1-3, to=1-5]
	\arrow[from=1-5, to=1-6]
	\arrow[from=3-1, to=3-2]
	\arrow[from=3-2, to=3-3]
	\arrow["{(j_2)_{\diesis} - (j_1)_{\diesis}}"', from=3-3, to=3-5]
	\arrow[from=3-5, to=3-6]
\end{tikzcd}
\end{equation*}
Inoltre \(r\) \href{20250122154136-funtorialita_dell_omologia_singolare.org}{induce} tutte le mappe\footnote{Vedi la \href{20260204100611-somma_diretta_di_complessi_di_catene.org}{somma diretta di morfismi}} che rendono il diagramma commutativo:
\begin{equation*}
\begin{tikzcd}[ampersand replacement=\&]
	0 \& {\mathcal{S}_{\bullet}(E)} \& {\mathcal{S}_{\bullet}(A^+)\oplus \mathcal{S}_{\bullet}(A^-)} \&\& {\mathcal{S}_{\bullet}(\mathds{S}^n)} \& 0 \\
	\\
	0 \& {\mathcal{S}_{\bullet}(E)} \& {\mathcal{S}_{\bullet}(A^-)\oplus \mathcal{S}_{\bullet}(A^+)} \&\& {\mathcal{S}_{\bullet}(\mathds{S}^n)} \& 0
	\arrow[from=1-1, to=1-2]
	\arrow[from=1-2, to=1-3]
	\arrow["{(\restriction{r}{E})_{\diesis}}"', from=1-2, to=3-2]
	\arrow["{(j_1)_{\diesis} - (j_2)_{\diesis}}", from=1-3, to=1-5]
	\arrow["{(\restriction{r}{A^+})_{\diesis}\oplus (\restriction{r}{A^-})_{\diesis}}", from=1-3, to=3-3]
	\arrow[from=1-5, to=1-6]
	\arrow["{(r)_{\diesis}}", from=1-5, to=3-5]
	\arrow[from=3-1, to=3-2]
	\arrow[from=3-2, to=3-3]
	\arrow["{(j_2)_{\diesis} - (j_1)_{\diesis}}"', from=3-3, to=3-5]
	\arrow[from=3-5, to=3-6]
\end{tikzcd}
\end{equation*}
e per \href{20250120165029-funtore_tra_chr_e_rmod.org}{funtorialità} + \href{20250120164938-zig_zag_lemma.org}{Zig-Zag Lemma}
\begin{equation*}
\begin{tikzcd}[ampersand replacement=\&,sep=scriptsize]
	{H_n(A^+)\oplus H_n (A^-)} \&\& {H_n(\mathds{S}^n)} \&\& {H_{n-1}(E)} \&\& {H_{n-1}(A^+)\oplus H_{n-1} (A^-)} \\
	\\
	{H_n(A^-)\oplus H_n (A^+)} \&\& {H_n(\mathds{S}^n)} \&\& {H_{n-1}(E)} \&\& {H_{n-1}(A^-)\oplus H_{n-1} (A^+)}
	\arrow["{j_1-j_2}", from=1-1, to=1-3]
	\arrow["{r_\star}"', from=1-1, to=3-1]
	\arrow["{\partial_\star}", from=1-3, to=1-5]
	\arrow["{r_\star}"', from=1-3, to=3-3]
	\arrow[from=1-5, to=1-7]
	\arrow["\Id", from=1-5, to=3-5]
	\arrow["{r_\star}"', from=1-7, to=3-7]
	\arrow["{j_2-j_1}"', from=3-1, to=3-3]
	\arrow["{\partial_\star'}"', from=3-3, to=3-5]
	\arrow[from=3-5, to=3-7]
\end{tikzcd}
\end{equation*}
Per il \href{20250120164938-zig_zag_lemma.org}{lemma}, \(\partial_{\star}' = -\partial_{\star}\).

Siccome \(A^{+}, A^{-}\) sono \href{20250122155640-spazio_topologico_contraibile.org}{contraibili}, \href{20250122155700-spazio_topologico_contraibile_e_aciclico.org}{allora}
\begin{equation*}
H_{n}(A^{ +}) = H_{n}(A^{-}) = 0 = H_{n-1}(A^{ +}) = H_{{n-1}}(A^{-})
\end{equation*}
e \href{20250120130155-caratterizzazione_di_alcune_successioni_esatte_di_r_moduli.org}{quindi} \(\partial_{\star}\) e \(\partial_{\star}'\) sono isomorfismi:
\begin{align*}
r_{\star} &= (\partial_{\star}')^{-1} \circ \Id \circ \partial_{\star} = (\partial_{\star}')^{-1} \circ \Id \circ (- \partial_{\star}') \\
&= - (\partial_{\star}')^{-1}\circ (\partial_{\star}') = - \Id.\qedhere
\end{align*}
\end{proof}
\subsection{Grado della mappa antipodale sulla sfera}
\label{sec:org56e8108}
\begin{cor}
La mappa antipodale
\begin{align*}
a: \mathds{S}^{n} &\longrightarrow \mathds{S}^{n}\\
x &\longmapsto -x
\end{align*}
ha grado \(\deg a = (-1)^{n+1}\).
\end{cor}
\end{document}
