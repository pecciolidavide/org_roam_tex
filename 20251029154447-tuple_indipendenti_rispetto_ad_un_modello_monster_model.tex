% Created 2026-02-07 Sat 19:35
% Intended LaTeX compiler: pdflatex
\documentclass[10pt]{article}
%% CREATO CON ORG - EMACS
\newcommand{\use}[2][]{\usepackage[#1]{#2}}
% PACCHETTI FONDAMENTLAI
\use[utf8]{inputenc}
\use[T1]{fontenc}
\use{graphicx}
\use{longtable}
\use{wrapfig}
\use{rotating}
\use[normalem]{ulem}
\use{amsmath}
\use{amsthm}
\use{amssymb}

\use{eucal} % Cambia mathcal{...}

\use{capt-of}
\use[italian]{babel}
\use[babel]{csquotes}
% bib la TEX lo carica in automatico org-cite
\use{microtype}
\use{lmodern}
\use{subfig} % sottofigure
\use{multicol} % due colonne
\use{lipsum} % lorem ipsum
\use{color} % colori in latex
\use{parskip} % rimuove l'indentazione dei nuovi paragrafi %% Add parbox=false to all new tcolorbox
\use{centernot}
\use[outline]{contour}\contourlength{3pt}
\use{fancyhdr}
\use{layout}
\use[most]{tcolorbox} % Riquadri colorati
\use{ifthen} % IFTHEN
\use{geometry}

% pacchetti matematica
\use{yhmath}
\use{dsfont}
\use{mathrsfs}
\use{cancel} % semplificare
\use{polynom} %divisione tra polinomi
\use{forest} % grafi ad albero
\use{booktabs} % tabelle
\use{commath} %simboli e differenziali
\use{bm} %bold
\use[fulladjust]{marginnote} %to use marginnote for date notes
\use{arrayjobx}%array
\use[intlimits]{empheq} % Riquadri colorati attorno alle equazioni
\use{mathtools}
\use{circuitikz} % Disegnare i circuiti
\use{mathtools}
\use{stmaryrd} % [[ \llbracket ]] \rrbracket
\use{bussproofs} % dimostrazioni

%%%%%%%%%%%%%


%%%% QUIVER
\newcommand{\duepunti}{\,\mathchar\numexpr"6000+`:\relax\,}
% A TikZ style for curved arrows of a fixed height, due to AndréC.
\tikzset{curve/.style={settings={#1},to path={(\tikztostart)
    .. controls ($(\tikztostart)!\pv{pos}!(\tikztotarget)!\pv{height}!270:(\tikztotarget)$)
    and ($(\tikztostart)!1-\pv{pos}!(\tikztotarget)!\pv{height}!270:(\tikztotarget)$)
    .. (\tikztotarget)\tikztonodes}},
    settings/.code={\tikzset{quiver/.cd,#1}
        \def\pv##1{\pgfkeysvalueof{/tikz/quiver/##1}}},
    quiver/.cd,pos/.initial=0.35,height/.initial=0}

% TikZ arrowhead/tail styles.
\tikzset{tail reversed/.code={\pgfsetarrowsstart{tikzcd to}}}
\tikzset{2tail/.code={\pgfsetarrowsstart{Implies[reversed]}}}
\tikzset{2tail reversed/.code={\pgfsetarrowsstart{Implies}}}
% TikZ arrow styles.
\tikzset{no body/.style={/tikz/dash pattern=on 0 off 1mm}}
%%%%%%%%%%


%% DEFINIZIONI COMANDI MATEMATICI
\let\sin\relax %TOGLIE LA DEFINIZIONE SU "\sin"

% cambia la definizione di empty set
% ---
\let\oldemptyset\emptyset
% ---
% \let\emptyset\varnothing
% ---
% \let\emptyset\relax
% \newcommand{\emptyset}{\text{\textnormal{\O}}}
% ---

\DeclareMathOperator{\bounded}{bd}
\DeclareMathOperator{\sin}{sen}
\DeclareMathOperator{\epi}{Epi}
\DeclareMathOperator{\cl}{cl}
\DeclareMathOperator{\graph}{graph}
\DeclareMathOperator{\arcsec}{arcsec}
\DeclareMathOperator{\arccot}{arccot}
\DeclareMathOperator{\arccsc}{arccsc}
\DeclareMathOperator{\spettro}{Spettro}
\DeclareMathOperator{\nulls}{nullspace}
\DeclareMathOperator{\dom}{dom}
\DeclareMathOperator{\ar}{ar}
\DeclareMathOperator{\const}{Const}
\DeclareMathOperator{\fun}{Fun}
\DeclareMathOperator{\rel}{Rel}
\DeclareMathOperator{\altezza}{ht}
\let\det\relax %TOGLIE LA DEFINIZIONE SU "\det"
\DeclareMathOperator{\det}{det}
\DeclareMathOperator{\End}{End}
\DeclareMathOperator{\gl}{GL}
\def\Id{\mathrm{Id}}
\def\id{\mathrm{id}}
\DeclareMathOperator{\I}{\mathds{1}}
\DeclareMathOperator{\II}{II}
\DeclareMathOperator{\rank}{rank}
\DeclareMathOperator{\tr}{tr}
\DeclareMathOperator{\tc}{t.c.}
\DeclareMathOperator{\T}{T}
\DeclareMathOperator{\var}{Var}
\DeclareMathOperator{\cov}{Cov}
\DeclareMathOperator{\st}{st}
\DeclareMathOperator{\mon}{Mon}
\newcommand{\card}[1]{\left\vert #1 \right\vert}
\newcommand{\trasposta}[1]{\prescript{\text{T}}{}{#1}}
\newcommand{\1}{\mathds{1}}
\newcommand{\R}{\mathds{R}}
\newcommand{\diesis}{\#}
\newcommand{\bemolle}{\flat}
\newcommand{\nonstandard}[1]{\prescript{*}{}{#1}}
\newcommand{\starR}{\nonstandard{\R}}
\newcommand{\borel}{\mathscr{B}}
\newcommand{\lebesgue}[1]{\mathscr{L}\left(#1\right)}
\newcommand{\media}{\mathds{E}}
\newcommand{\K}{\mathds{K}}
\newcommand{\A}{\mathds{A}}
\newcommand{\Q}{\mathds{Q}}
\newcommand{\N}{\mathds{N}}
\newcommand{\C}{\mathds{C}}
\newcommand{\Z}{\mathds{Z}}
\newcommand{\qo}{\hspace{1em}\text{q.o.}\,}
\renewcommand{\tilde}[1]{\widetilde{#1}}
\renewcommand{\parallel}{\mathrel{/\mkern-5mu/}}
\newcommand{\parti}[2][]{\wp_{#1}(#2)}
\newcommand{\diff}[1]{\operatorname{d}_{#1}}
\let\oldvec\vec
\renewcommand{\vec}[1]{\overrightarrow{\vphantom{i}#1}}
\newcommand{\floor}[1]{\left\lfloor #1 \right\rfloor}
\newcommand{\cat}[1]{\mathbf{#1}}
\newcommand{\dfreccia}[1]{\xrightarrow{\ #1 \ }}
\newcommand{\sfreccia}[1]{\xleftarrow{\ #1 \ }}
\newcommand{\formalsum}[2]{{\sum_{#1}^{#2}}{\vphantom{\sum}}'}
\newcommand{\minim}[2]{\mu_{#1}\, \left(#2\right)}
\newcommand{\concat}{\null^{\frown}} % concatenazione di stringe
\newcommand{\godelcode}[1]{\langle\!\langle #1 \rangle\!\rangle}
\newcommand{\godeldec}[1]{(\!(#1)\!)}
\newcommand{\termcode}[1]{\ulcorner #1\urcorner}
\newcommand{\partialto}{\dashrightarrow}
\newcommand{\restricted}{\upharpoonright}
\newcommand{\embeds}{\precsim}
\newcommand{\surjects}{\twoheadrightarrow}
\newcommand{\equipotenti}{\asymp}
%% \newcommand{\dotplus}{\mathbin{\dot{+}}} %% A quanto pare esiste già
\newcommand{\bigdot}{\mathbin{\boldsymbol{\cdot}}}
\newcommand{\dotexp}[1]{^{.#1}}
\newcommand{\conv}{\mathbin{*}}
\newcommand{\convolution}[2]{(#1\conv #2)}
\newcommand{\nil}{\mathfrak{N}}
\newcommand{\divisore}{\mathrel{|}}
\newcommand{\simplesso}[1]{\mathrm{e}_{#1}}

\renewcommand{\iff}{\mathrel{\longleftrightarrow}} %% Notazione Logica.
\newcommand{\oldiff}{\mathrel{\Longleftrightarrow}}
\renewcommand{\implies}{\mathrel{\rightarrow}} %% Notazione Logica
\newcommand{\oldimplies}{\mathrel{\Longrightarrow}}
\renewcommand{\impliedby}{\mathrel{\leftarrow}} %% Notazione Logica
\newcommand{\oldimpliedby}{\mathrel{\Longleftarrow}}

\newcommand{\IFF}{\quad\Longleftrightarrow\quad}
\newcommand{\IMPLICA}{\quad\Longrightarrow\quad}


\renewcommand{\descriptionlabel}[1]{\hspace{\labelsep}\normalfont #1} % remove bold from description


%% Definizione di Divergenza di K-L

\DeclarePairedDelimiterX{\infdivx}[2]{(}{)}{%
  #1\;\delimsize\|\;#2%
}
\newcommand{\kldiv}{D_{KL}\infdivx}

%% Definizione di \dotminus

\makeatletter
\newcommand{\dotminus}{\mathbin{\text{\@dotminus}}}

\newcommand{\@dotminus}{%
  \ooalign{\hidewidth\raise1ex\hbox{.}\hidewidth\cr$\m@th-$\cr}%
}
\makeatother

%tramite i prossimi due comandi posso decidere come scrivere i logaritmi naturali in tutti i documenti: ho infatti eliminato qualsiasi differenza tra "ln" e "log": se si vuole qualcosa di diverso bisogna inserire manualmente il tutto
\let\ln\relax
\DeclareMathOperator{\ln}{ln}
\let\log\relax
\DeclareMathOperator{\log}{log}
%%%%%%

%% NUOVI COMANDI
\newcommand{\straniero}[1]{\textit{#1}} %parole straniere
\newcommand{\titolo}[1]{\textsc{#1}} %titoli
\newcommand{\qedd}{\tag*{$\blacksquare$}} %qed per ambienti matemastici
\renewcommand{\qedsymbol}{$\blacksquare$} %modifica colore qed
\newcommand{\ooverline}[1]{\overline{\overline{#1}}}
\newcommand{\circoletto}[1]{\left(#1\right)^{\text{o}}}
%
\newcommand{\qmatrice}[1]{\begin{pmatrix}
#1_{11} & \cdots & #1_{1n}\\
\vdots & \ddots & \vdots \\
#1_{m1} & \cdots & #1_{mn}
\end{pmatrix}}
%
\newcommand{\parentesi}[2]{%
\underset{#1}{\underbrace{#2}}%
}
%
\newcommand{\norma}[1]{% Norma
\left\lVert#1\right\rVert%
}
\newcommand{\scalare}[2]{% Scalare
\left\langle #1, #2\right\rangle
}
%%%%%

%% RESTRIZIONI
\newcommand{\referenze}[2]{
        \phantomsection{}#2\textsuperscript{\textcolor{blue}{\textbf{#1}}}
}

\let\restriction\relax

\def\restriction#1#2{\mathchoice
              {\setbox1\hbox{${\displaystyle #1}_{\scriptstyle #2}$}
              \restrictionaux{#1}{#2}}
              {\setbox1\hbox{${\textstyle #1}_{\scriptstyle #2}$}
              \restrictionaux{#1}{#2}}
              {\setbox1\hbox{${\scriptstyle #1}_{\scriptscriptstyle #2}$}
              \restrictionaux{#1}{#2}}
              {\setbox1\hbox{${\scriptscriptstyle #1}_{\scriptscriptstyle #2}$}
              \restrictionaux{#1}{#2}}}
\def\restrictionaux#1#2{{#1\,\smash{\vrule height .8\ht1 depth .85\dp1}}_{\,#2}}
%%%%%%%%%%%

%%% FORMATTAZIONE FOOTNOTEMARK

\def\footnotemarkformatting#1{[#1]}
\renewcommand{\thefootnote}{\footnotemarkformatting{\arabic{footnote}}}

%% SEZIONE GRAFICA
\use{tikz}
\usetikzlibrary{matrix, patterns, calc, decorations.pathreplacing, hobby, decorations.markings, decorations.pathmorphing, babel}
\use{tikz-3dplot}
\use{mathrsfs} %per geogebra
\use{tikz-cd}
\tikzset
{
  %surface/.style={fill=black!10, shading=ball,fill opacity=0.4},
  plane/.style={black,pattern=north east lines},
  curve/.style={black,line width=0.5mm},
  dritto/.style={decoration={markings,mark=at position 0.5 with {\arrow{Stealth}}}, postaction=decorate},
  rovescio/.style={decoration={markings,mark=at position 0.5 with {\arrow{Stealth[reversed]}}}, postaction=decorate}
}
\use{pgfplots} % stampare le funzioni
        \pgfplotsset{/pgf/number format/use comma,compat=1.15}
        %\pgfplotsset{compat=1.15} %per geogebra
        \usepgfplotslibrary{fillbetween, polar}
%%%%%%

%% CITAZIONI
\use{lineno}

\newcommand{\citazione}[1]{%
  \begin{quotation}
  \begin{linenumbers}
  \modulolinenumbers[5]
  \begingroup
  \setlength{\parindent}{0cm}
  \noindent #1
  \endgroup
  \end{linenumbers}
  \end{quotation}\setcounter{linenumber}{1}
  }
%%%%%%

%%%%%%%%%%%%%%%%%%%%%%%%%%%%%%%%%%%%%%%%%%%%
%%%%%%%%%%%%%%%%%%%%%%%%%%%%%%%%%%%%%%%%%%%%

%% AMS THM

\theoremstyle{definition}% default
\newtheorem{thm}{Teorema}[section]
\newtheorem{lem}[thm]{Lemma}
\newtheorem{prop}[thm]{Proposizione}
\newtheorem{cor}[thm]{Corollario}
\newtheorem{esempio}[thm]{Esempio}
\theoremstyle{plain}
\newtheorem{definizione}[thm]{Definizione}
\theoremstyle{remark}
\newtheorem*{oss}{Osservazione}


%%%%%%%%%%%%%%%%%%%%%%%%%%%%%%%%%%%%%%%%%%%%
%%%%%%%%%%%%%%%%%%%%%%%%%%%%%%%%%%%%%%%%%%%%

\use{hyperref}
\hypersetup{%
        pdfauthor={Davide Peccioli},
        pdfsubject={},
        allcolors=black,
        citecolor=black,
%	colorlinks=true,
        bookmarksopen=true}
\setcounter{secnumdepth}{0} % rimuove i numeri di sezione senza rimuovere le ref
\renewcommand{\href}[2]{\textcolor{blue}{#2}} % disabilita il comando href
\use{enotez} %
\setenotez{%
 mark-format = \footnotemarkformatting % Mette i numeri tra parentesi quadre%
}\let\footnote=\endnote % rende tutte le note a pié pagina come delle note a fine file 


\let\olddocument\document % modifico l'ambiende documenti per non dover stampare \printendnote
\let\oldenddocument\enddocument
\renewenvironment{document}%
{%
  \olddocument
}{%
  \printendnotes\oldenddocument
}
\renewcommand{\thethm}{\arabic{thm}}

\usepackage[hyperref]{biblatex}
\addbibresource{~/Documents/org/roam/bib/master.bib}
\author{Davide Peccioli}
\date{\today}
\title{}
\begin{document}

\def\D{\mathcal{D}}
\def\DD{\bm{\D}}
\def\U{\mathcal{U}}
\def\eq{{\rm eq}}
\def\Ueq{\U^\eq}
\def\L{\mathcal{L}}
\def\orbita{\mathcal{O}}
\def\Aut{\operatorname{Aut}}
\def\tc{\mid}
\def\tp{\operatorname{tp}}
\def\<{\langle}
\def\>{\rangle}
\def\b{\bm{b}}

\def\restricted#1{\,\mathord{\upharpoonright}{{\scriptstyle #1}}}
\def\equivalentover#1{\mathrel{\equiv_{ #1 }}}

%% NON FORKING
\def\nonforkSymbol{%
\mathbin{\raise1.8ex%
\rlap{\kern0.6ex\rule{0.6ex}{0.1ex}}%
\rlap{\kern1.1ex\rule{0.1ex}{1.9ex}}\raise-0.3ex\hbox{$\smile$}}}
\def\defaultnonforkmodel{M}
\def\nonfork{\nonforkSymbol}
\renewcommand{\nonfork}[1][\defaultnonforkmodel]{%
\mathrel{\nonforkSymbol_{#1}}}
\section{Relazione eredi-coeredi}
\label{sec:org0492a08}
Si utilizza la \href{20250612143636-notazione_teoria_dei_modelli.org}{Notazione della TEORIA DEI MODELLI}

Sia \(\L\) un \href{20250130162057-linguaggio_del_prim_ordine.org}{linguaggio}, \(T\) una \href{20250130114950-teoria_del_prim_ordine.org}{teoria} \href{20250131123151-teoria_completa.org}{completa} senza \href{20250131122945-modello_di_un_insieme_di_formule.org}{modelli} finiti e \(\U\) un \href{20250617095548-modello_lambda_saturo.org}{modello saturo} di \href{20241213101756-cardinalita.org}{cardinalità} \href{20250211123155-cardinale_limite_forte.org}{inaccessibile} \(\kappa>\card{\L}+ \omega\). \(\U\) è un \href{20250617102733-modello_mostro.org}{modello mostro}.

Sia \(M\preceq \U\) un \href{20250131103035-struttura_del_prim_ordine.org}{modello}.

\begin{definizione}
Siano \(a \in \mathcal{U}^{x}\), \(b \in \mathcal{U}^{z}\). \(a\) è \uline{indipendente da \(b\)} su \(M\), e si scrive
\begin{equation*}
a \nonfork b
\end{equation*}
se il \href{20250212164424-tipo_teoria_dei_modelli.org}{tipo} \href{20250212164424-tipo_teoria_dei_modelli.org}{\(\tp(a/M,b)\)} è \href{20250212164424-tipo_teoria_dei_modelli.org}{finitamente soddisfacibile in \(M\)}\footnote{Questa definizione è propria del \href{20250617102733-modello_mostro.org}{Modello MOSTRO}}.
\end{definizione}
\begin{oss}
LSASE:
\begin{enumerate}
\item \(a\nonfork b\);
\item per ogni \(\varphi(x;z) \in \mathcal{L}(M)\)
\begin{equation*}
\varphi(a;b) \IMPLICA \varphi(m;b)\text{ per qualche }m \in M^{x};
\end{equation*}
\item \(a\) appartiene alla \href{20250103144944-chiusura_topologica.org}{chiusura (topologica)} di \(M^{x}\) nella \href{20250617102733-modello_mostro.org}{\((M,b)\)-topologia di \(\U^{x}\)}.
\end{enumerate}
\end{oss}
\begin{oss}
Si definisce il tipo
\begin{equation*}
x\nonfork  b =
\set{\lnot \varphi(x,b) \mid
\varphi(x;z) \in \mathcal{L}(M),\,%
\varphi(M^{x},b)= \emptyset}
\end{equation*}
\href{20250212164424-tipo_teoria_dei_modelli.org}{soddisfatto} da tutti i seguenti elementi:
\begin{equation*}
a\vDash x\nonfork  b \IFF a\nonfork  b.
\end{equation*}

Invece \(a \nonfork z\) non è un tipo. Vedi Remark~14.7 e Esercizio~14.13 di \autocite{zambellaCrecheCourseModel2025a}.
\end{oss}
\begin{lem}
Valgono le seguenti proprietà.
\begin{enumerate}
\item Se \(a\nonfork  b\) allora \(fa\nonfork  fb\) per ogni \href{20251020151126-insieme_degli_automorfismi_di_una_struttura_del_prim_ordine.org}{\(f \in \operatorname{Aut}(\mathcal{U}/M)\)}. \hfill(invarianza)
\item \(a\nonfork b\) sse \(a_{0}\nonfork b_{0}\) per ogni \(a_{0} \subseteq a\), \(b_{0} \subseteq b\) sottotuple finite.
\hfill(carattere finito)
\item Se \(a\nonfork  b, c\) e \(c\nonfork  b\) allora \(a,c\nonfork  b\).
\hfill(transitività)
\item Se \(a\nonfork b\), per ogni \(c\) esiste \(a'\equivalentover{M,b}a\) e \(a'\nonfork b,c\).
\hfill(\emph{coheir extension})
\item Se \(a\nonfork b_{1},b_{2}\) e \(b_{1} \equivalentover{M} b_{2}\) allora \(b_{1}\equivalentover{M,a}b_{2}\).
\hfill(\emph{non-splitting})
\end{enumerate}
\label{lem:propfork}
\end{lem}
\begin{proof}
\begin{enumerate}
\item Se \(a \nonfork[A] b\) allora \(fa \nonfork[A] fb\) per ogni \(f \in \Aut(\U/A)\).

\emph{dim}.
Sia \(\varphi(x;z) \in \L(A)\) tale che \(\U \vDash \varphi(fa,fb)\);
allora \(\U \vDash \varphi(a;b)\) poiché \(f\) automorfismo.

Dunque, per ipotesi, \(\varphi(\U^{x}; b)\cap A^{x} \neq \emptyset\) e pertanto
\begin{equation*}
 \varphi(\U^{x}; fb) \cap A^{x}\neq \emptyset
\end{equation*}
poiché \(\varphi(\U^{x}; fb) = f[\varphi(\U^{x};b)]\) in quanto \(f \in \Aut(\U/A)\); quindi
\begin{equation*}
 \varphi(\U^{x}; fb) \cap A^{x}
 = f[\varphi(\U^{x};b)] \cap A^{x}
  = \varphi(\U^{x};b) \cap A^{x} \neq \emptyset.
\end{equation*}

\item \(a \nonfork[A] b\) sse \(a_{0} \nonfork[A] b_{0}\) per ogni \(a_{0} \subseteq a\), \(b_{0} \subseteq b\) finiti.

\emph{dim}.
(\(\Rightarrow\)):
Sia \(\varphi(x_{0};z_{0}) \in \L(A)\) tale che \(\U \vDash \varphi(a_{0};b_{0})\)
(con \(\card{x_{0}} = \card{a_{0}}, \card{z_{0}} = \card{b_{0}}\)).
Inoltre, \(\varphi(x_{0},z_{0}) = \varphi(x;z)\),
con \(\card{x} = \card{a}\) e \(\card{z} = \card{b}\), e pertanto
\begin{equation*}
 \U\vDash \varphi(a;b).
\end{equation*}
Per ipotesi, quindi, \(\varphi(\U^{x}; b)\cap A^{x} \neq \emptyset\) e dunque anche \(\varphi(\U^{x}; b_{0})\cap A^{x} \neq \emptyset\).
Proiettando un elemento di \(\varphi(\U^{x}; b_{0})\cap A^{x}\) in \(A^{x_{0}}\) si ottiene che
\(\varphi(\U^{x_{0}};b_{0}) \cap A^{x_{0}} \neq \emptyset\).

(\(\Leftarrow\)):
Sia \(\varphi(x;z) \in \L(A)\) tale che \(\U\vDash \varphi(a;b)\).

Siano \(a_{0} \subseteq a\), \(b_{0} \subseteq b\) le sottostringhe finite che compaiono in \(\varphi(a;b)\),
sicché si abbia
\begin{equation*}
 \U\vDash\varphi(a_{0},b_{0}).
\end{equation*}
Allora, per ipotesi, \(\varphi(\U^{x_{0}};b_{0})\cap A^{x_{0}} \neq \emptyset\).
Dunque \(\varphi(\U^{x_{0}};b)\cap A^{x_{0}} \neq \emptyset\).

In particolare, sia \(v_{0} \in A^{x_{0}}\) tale che \(\U\vDash \varphi(v_{0};b)\).
È possibile estendere naturalmente \(v_{0}\) ad un elemento \(v \in A^{x}\) tale che \(\U\vDash\varphi(v;b)\) e pertanto
\begin{equation*}
 \varphi(\U^{x};b)\cap A^{x} \neq \emptyset.
\end{equation*}

\item Se \(a \nonfork[A] c,b\) e \(c \nonfork[A] b\) allora \(a,c \nonfork[A] b\).

\emph{dim}.
Con (\emph{i}) si indica l'ipotesi \(a \nonfork[A] c,b\); con (\emph{ii}) l'ipotesi \(c \nonfork[A] b\).

Sia \(\varphi(x;y;z) \in \L(A)\) tale che \(\vDash \varphi(a;c;b)\).
Allora per (\emph{i}) si ha che
\begin{equation*}
 \varphi(\U^{x};c;b) \cap A^{x} \neq \emptyset
\end{equation*}
e quindi esiste \(a' \in A^{x}\) tale che \(\vDash \varphi(a';c;b)\).
Inoltre \(\varphi(a';y;z) \in \L(A)\).

Per (\emph{ii}) si ha quindi che
\begin{equation*}
 \varphi(a'; \U^{y}; b) \cap A^{y}\neq \emptyset
\end{equation*}
e dunque esiste \(c' \in \mathcal{A}^{y}\) tale che \(\vDash \varphi(a';c';b)\).

Pertanto \(\varphi(\U^{x};\U^{y};b)\cap A^{x;y}\neq \emptyset\) (ne è rappresentante \((a',c')\)) e quindi \(a,c \nonfork[A] b\).

\item Se \(a \nonfork[A] b\) allora esiste \(a' \equivalentover{A,b} a\) tale che \(a' \nonfork[A] b,c\)

\emph{dim}.
Per ipotesi \(\tp(a/A,b)\) è finitamente soddisfacibile in \(A\), allora (Prop.~14.3) esiste \(p(x)\) tipo globale finitamente soddisfacibile in \(A\) che lo estende.

 Allora un qualsiasi \(a'\vDash p(x)\restricted{A,b,c}\) soddisfa la tesi.
Tale \(a'\) esiste poiché \(p(x)\restricted{A,b,c}\) è un tipo piccolo.

\(a'\) soddisfa la tesi poiché \(\tp(a/A,b) \subseteq p(x)\restricted{A,b,c}\) e dunque \(a' \equivalentover{A,b} a\) ed inoltre, se \(\varphi(x;y;z) \in \L(A)\) è tale che
\begin{equation*}
 \U\vDash \varphi(a',b,c)
\end{equation*}
allora per completezza di \(p\) si ha
\(\varphi(x;b;c) \in p(x) \cap \L(M,b,c)  = p(x)\restricted{A,b,c}\):
per finita soddisfacibilità di \(p(x)\restricted{A,b,c}\) in \(A\) segue che \(\varphi(\U^{x};b;c) \cap A^{x} \neq \emptyset\) (e quindi \(a' \nonfork[A] b,c\)).

\item Se \(a \nonfork[A] b_{1},b_{2}\) e \(b_{1} \equivalentover{A} b_{2}\) allora \(b_{1} \equivalentover{A,a} b_{2}\).

\emph{dim}.
Sia \(p(x)\) tipo globale finitamente soddisfacibile in \(A\) che estende \(\tp(a/A,b_{1},b_{2})\).

Allora per ogni \(c \vDash p(x) \restricted{A,b_{1},b_{2}}\) si ha \(b_{1}\equivalentover{A,c} b_{2}\).

Infatti, sia \(\varphi(x;y) \in \L(A)\) tale che \(\varphi(c;b_{1})\) (ovvero \(\varphi(c;y) \in \tp(b_{1}/A,c)\)). Siccome \(\varphi(x;b_{1}) \in p(x)\)\footnote{Se per assurdo \(\varphi(x;b_{1}) \notin p(x)\) allora \(\lnot\varphi(x;b_{1}) \in p(x)\) (poiché \(p\) tipo completo) e pertanto \(\lnot\varphi(x;b_{1}) \in p(x) \restricted{A,b_{1},b_{2}}\). Quindi vale \(\varphi(c;b_{1})\) e \(\lnot\varphi(c;b_{1})\). Assurdo.} allora \(\varphi(x;b_{2}) \in p(x)\).

Se per assurdo \(\varphi(x;b_{2}) \notin p(x)\) allora \(\lnot\varphi(x;b_{2}) \in p(x)\). Per finita soddisfacibilità di \(p(x)\) in \(A\), esiste \(d \in A\) tale che
\begin{equation*}
 \varphi(d;b_{1}) \land \lnot\varphi(d,b_{2}).
\end{equation*}
Assurdo poiché \(b_{1} \equivalentover{A} b_{2}\).

Pertanto, dato che \(\varphi(x;b_{2}) \in p(x)\) allora \(\varphi(x;b_{2}) \in p(x) \restricted{A,b_{1},b_{2}}\) e dunque \(\varphi(c;b_{2})\) ovvero \(\varphi(c;y) \in \tp(b_{2}/A,c)\). Per simmetria quindi \(b_{1}\equivalentover{A,c} b_{2}\).

In particolare, se \(a \nonfork[A] b_{1},b_{2}\) allora \(a\vDash p(x)\restricted{A,b_{1},b_{2}} \supseteq \tp(a/A,b_{1},b_{2})\).\qedhere
\end{enumerate}
\end{proof}

\printbibliography
\end{document}
