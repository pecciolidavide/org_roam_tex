% Created 2026-02-07 Sat 19:31
% Intended LaTeX compiler: pdflatex
\documentclass[10pt]{article}
%% CREATO CON ORG - EMACS
\newcommand{\use}[2][]{\usepackage[#1]{#2}}
% PACCHETTI FONDAMENTLAI
\use[utf8]{inputenc}
\use[T1]{fontenc}
\use{graphicx}
\use{longtable}
\use{wrapfig}
\use{rotating}
\use[normalem]{ulem}
\use{amsmath}
\use{amsthm}
\use{amssymb}

\use{eucal} % Cambia mathcal{...}

\use{capt-of}
\use[italian]{babel}
\use[babel]{csquotes}
% bib la TEX lo carica in automatico org-cite
\use{microtype}
\use{lmodern}
\use{subfig} % sottofigure
\use{multicol} % due colonne
\use{lipsum} % lorem ipsum
\use{color} % colori in latex
\use{parskip} % rimuove l'indentazione dei nuovi paragrafi %% Add parbox=false to all new tcolorbox
\use{centernot}
\use[outline]{contour}\contourlength{3pt}
\use{fancyhdr}
\use{layout}
\use[most]{tcolorbox} % Riquadri colorati
\use{ifthen} % IFTHEN
\use{geometry}

% pacchetti matematica
\use{yhmath}
\use{dsfont}
\use{mathrsfs}
\use{cancel} % semplificare
\use{polynom} %divisione tra polinomi
\use{forest} % grafi ad albero
\use{booktabs} % tabelle
\use{commath} %simboli e differenziali
\use{bm} %bold
\use[fulladjust]{marginnote} %to use marginnote for date notes
\use{arrayjobx}%array
\use[intlimits]{empheq} % Riquadri colorati attorno alle equazioni
\use{mathtools}
\use{circuitikz} % Disegnare i circuiti
\use{mathtools}
\use{stmaryrd} % [[ \llbracket ]] \rrbracket
\use{bussproofs} % dimostrazioni

%%%%%%%%%%%%%


%%%% QUIVER
\newcommand{\duepunti}{\,\mathchar\numexpr"6000+`:\relax\,}
% A TikZ style for curved arrows of a fixed height, due to AndréC.
\tikzset{curve/.style={settings={#1},to path={(\tikztostart)
    .. controls ($(\tikztostart)!\pv{pos}!(\tikztotarget)!\pv{height}!270:(\tikztotarget)$)
    and ($(\tikztostart)!1-\pv{pos}!(\tikztotarget)!\pv{height}!270:(\tikztotarget)$)
    .. (\tikztotarget)\tikztonodes}},
    settings/.code={\tikzset{quiver/.cd,#1}
        \def\pv##1{\pgfkeysvalueof{/tikz/quiver/##1}}},
    quiver/.cd,pos/.initial=0.35,height/.initial=0}

% TikZ arrowhead/tail styles.
\tikzset{tail reversed/.code={\pgfsetarrowsstart{tikzcd to}}}
\tikzset{2tail/.code={\pgfsetarrowsstart{Implies[reversed]}}}
\tikzset{2tail reversed/.code={\pgfsetarrowsstart{Implies}}}
% TikZ arrow styles.
\tikzset{no body/.style={/tikz/dash pattern=on 0 off 1mm}}
%%%%%%%%%%


%% DEFINIZIONI COMANDI MATEMATICI
\let\sin\relax %TOGLIE LA DEFINIZIONE SU "\sin"

% cambia la definizione di empty set
% ---
\let\oldemptyset\emptyset
% ---
% \let\emptyset\varnothing
% ---
% \let\emptyset\relax
% \newcommand{\emptyset}{\text{\textnormal{\O}}}
% ---

\DeclareMathOperator{\bounded}{bd}
\DeclareMathOperator{\sin}{sen}
\DeclareMathOperator{\epi}{Epi}
\DeclareMathOperator{\cl}{cl}
\DeclareMathOperator{\graph}{graph}
\DeclareMathOperator{\arcsec}{arcsec}
\DeclareMathOperator{\arccot}{arccot}
\DeclareMathOperator{\arccsc}{arccsc}
\DeclareMathOperator{\spettro}{Spettro}
\DeclareMathOperator{\nulls}{nullspace}
\DeclareMathOperator{\dom}{dom}
\DeclareMathOperator{\ar}{ar}
\DeclareMathOperator{\const}{Const}
\DeclareMathOperator{\fun}{Fun}
\DeclareMathOperator{\rel}{Rel}
\DeclareMathOperator{\altezza}{ht}
\let\det\relax %TOGLIE LA DEFINIZIONE SU "\det"
\DeclareMathOperator{\det}{det}
\DeclareMathOperator{\End}{End}
\DeclareMathOperator{\gl}{GL}
\def\Id{\mathrm{Id}}
\def\id{\mathrm{id}}
\DeclareMathOperator{\I}{\mathds{1}}
\DeclareMathOperator{\II}{II}
\DeclareMathOperator{\rank}{rank}
\DeclareMathOperator{\tr}{tr}
\DeclareMathOperator{\tc}{t.c.}
\DeclareMathOperator{\T}{T}
\DeclareMathOperator{\var}{Var}
\DeclareMathOperator{\cov}{Cov}
\DeclareMathOperator{\st}{st}
\DeclareMathOperator{\mon}{Mon}
\newcommand{\card}[1]{\left\vert #1 \right\vert}
\newcommand{\trasposta}[1]{\prescript{\text{T}}{}{#1}}
\newcommand{\1}{\mathds{1}}
\newcommand{\R}{\mathds{R}}
\newcommand{\diesis}{\#}
\newcommand{\bemolle}{\flat}
\newcommand{\nonstandard}[1]{\prescript{*}{}{#1}}
\newcommand{\starR}{\nonstandard{\R}}
\newcommand{\borel}{\mathscr{B}}
\newcommand{\lebesgue}[1]{\mathscr{L}\left(#1\right)}
\newcommand{\media}{\mathds{E}}
\newcommand{\K}{\mathds{K}}
\newcommand{\A}{\mathds{A}}
\newcommand{\Q}{\mathds{Q}}
\newcommand{\N}{\mathds{N}}
\newcommand{\C}{\mathds{C}}
\newcommand{\Z}{\mathds{Z}}
\newcommand{\qo}{\hspace{1em}\text{q.o.}\,}
\renewcommand{\tilde}[1]{\widetilde{#1}}
\renewcommand{\parallel}{\mathrel{/\mkern-5mu/}}
\newcommand{\parti}[2][]{\wp_{#1}(#2)}
\newcommand{\diff}[1]{\operatorname{d}_{#1}}
\let\oldvec\vec
\renewcommand{\vec}[1]{\overrightarrow{\vphantom{i}#1}}
\newcommand{\floor}[1]{\left\lfloor #1 \right\rfloor}
\newcommand{\cat}[1]{\mathbf{#1}}
\newcommand{\dfreccia}[1]{\xrightarrow{\ #1 \ }}
\newcommand{\sfreccia}[1]{\xleftarrow{\ #1 \ }}
\newcommand{\formalsum}[2]{{\sum_{#1}^{#2}}{\vphantom{\sum}}'}
\newcommand{\minim}[2]{\mu_{#1}\, \left(#2\right)}
\newcommand{\concat}{\null^{\frown}} % concatenazione di stringe
\newcommand{\godelcode}[1]{\langle\!\langle #1 \rangle\!\rangle}
\newcommand{\godeldec}[1]{(\!(#1)\!)}
\newcommand{\termcode}[1]{\ulcorner #1\urcorner}
\newcommand{\partialto}{\dashrightarrow}
\newcommand{\restricted}{\upharpoonright}
\newcommand{\embeds}{\precsim}
\newcommand{\surjects}{\twoheadrightarrow}
\newcommand{\equipotenti}{\asymp}
%% \newcommand{\dotplus}{\mathbin{\dot{+}}} %% A quanto pare esiste già
\newcommand{\bigdot}{\mathbin{\boldsymbol{\cdot}}}
\newcommand{\dotexp}[1]{^{.#1}}
\newcommand{\conv}{\mathbin{*}}
\newcommand{\convolution}[2]{(#1\conv #2)}
\newcommand{\nil}{\mathfrak{N}}
\newcommand{\divisore}{\mathrel{|}}
\newcommand{\simplesso}[1]{\mathrm{e}_{#1}}

\renewcommand{\iff}{\mathrel{\longleftrightarrow}} %% Notazione Logica.
\newcommand{\oldiff}{\mathrel{\Longleftrightarrow}}
\renewcommand{\implies}{\mathrel{\rightarrow}} %% Notazione Logica
\newcommand{\oldimplies}{\mathrel{\Longrightarrow}}
\renewcommand{\impliedby}{\mathrel{\leftarrow}} %% Notazione Logica
\newcommand{\oldimpliedby}{\mathrel{\Longleftarrow}}

\newcommand{\IFF}{\quad\Longleftrightarrow\quad}
\newcommand{\IMPLICA}{\quad\Longrightarrow\quad}


\renewcommand{\descriptionlabel}[1]{\hspace{\labelsep}\normalfont #1} % remove bold from description


%% Definizione di Divergenza di K-L

\DeclarePairedDelimiterX{\infdivx}[2]{(}{)}{%
  #1\;\delimsize\|\;#2%
}
\newcommand{\kldiv}{D_{KL}\infdivx}

%% Definizione di \dotminus

\makeatletter
\newcommand{\dotminus}{\mathbin{\text{\@dotminus}}}

\newcommand{\@dotminus}{%
  \ooalign{\hidewidth\raise1ex\hbox{.}\hidewidth\cr$\m@th-$\cr}%
}
\makeatother

%tramite i prossimi due comandi posso decidere come scrivere i logaritmi naturali in tutti i documenti: ho infatti eliminato qualsiasi differenza tra "ln" e "log": se si vuole qualcosa di diverso bisogna inserire manualmente il tutto
\let\ln\relax
\DeclareMathOperator{\ln}{ln}
\let\log\relax
\DeclareMathOperator{\log}{log}
%%%%%%

%% NUOVI COMANDI
\newcommand{\straniero}[1]{\textit{#1}} %parole straniere
\newcommand{\titolo}[1]{\textsc{#1}} %titoli
\newcommand{\qedd}{\tag*{$\blacksquare$}} %qed per ambienti matemastici
\renewcommand{\qedsymbol}{$\blacksquare$} %modifica colore qed
\newcommand{\ooverline}[1]{\overline{\overline{#1}}}
\newcommand{\circoletto}[1]{\left(#1\right)^{\text{o}}}
%
\newcommand{\qmatrice}[1]{\begin{pmatrix}
#1_{11} & \cdots & #1_{1n}\\
\vdots & \ddots & \vdots \\
#1_{m1} & \cdots & #1_{mn}
\end{pmatrix}}
%
\newcommand{\parentesi}[2]{%
\underset{#1}{\underbrace{#2}}%
}
%
\newcommand{\norma}[1]{% Norma
\left\lVert#1\right\rVert%
}
\newcommand{\scalare}[2]{% Scalare
\left\langle #1, #2\right\rangle
}
%%%%%

%% RESTRIZIONI
\newcommand{\referenze}[2]{
        \phantomsection{}#2\textsuperscript{\textcolor{blue}{\textbf{#1}}}
}

\let\restriction\relax

\def\restriction#1#2{\mathchoice
              {\setbox1\hbox{${\displaystyle #1}_{\scriptstyle #2}$}
              \restrictionaux{#1}{#2}}
              {\setbox1\hbox{${\textstyle #1}_{\scriptstyle #2}$}
              \restrictionaux{#1}{#2}}
              {\setbox1\hbox{${\scriptstyle #1}_{\scriptscriptstyle #2}$}
              \restrictionaux{#1}{#2}}
              {\setbox1\hbox{${\scriptscriptstyle #1}_{\scriptscriptstyle #2}$}
              \restrictionaux{#1}{#2}}}
\def\restrictionaux#1#2{{#1\,\smash{\vrule height .8\ht1 depth .85\dp1}}_{\,#2}}
%%%%%%%%%%%

%%% FORMATTAZIONE FOOTNOTEMARK

\def\footnotemarkformatting#1{[#1]}
\renewcommand{\thefootnote}{\footnotemarkformatting{\arabic{footnote}}}

%% SEZIONE GRAFICA
\use{tikz}
\usetikzlibrary{matrix, patterns, calc, decorations.pathreplacing, hobby, decorations.markings, decorations.pathmorphing, babel}
\use{tikz-3dplot}
\use{mathrsfs} %per geogebra
\use{tikz-cd}
\tikzset
{
  %surface/.style={fill=black!10, shading=ball,fill opacity=0.4},
  plane/.style={black,pattern=north east lines},
  curve/.style={black,line width=0.5mm},
  dritto/.style={decoration={markings,mark=at position 0.5 with {\arrow{Stealth}}}, postaction=decorate},
  rovescio/.style={decoration={markings,mark=at position 0.5 with {\arrow{Stealth[reversed]}}}, postaction=decorate}
}
\use{pgfplots} % stampare le funzioni
        \pgfplotsset{/pgf/number format/use comma,compat=1.15}
        %\pgfplotsset{compat=1.15} %per geogebra
        \usepgfplotslibrary{fillbetween, polar}
%%%%%%

%% CITAZIONI
\use{lineno}

\newcommand{\citazione}[1]{%
  \begin{quotation}
  \begin{linenumbers}
  \modulolinenumbers[5]
  \begingroup
  \setlength{\parindent}{0cm}
  \noindent #1
  \endgroup
  \end{linenumbers}
  \end{quotation}\setcounter{linenumber}{1}
  }
%%%%%%

%%%%%%%%%%%%%%%%%%%%%%%%%%%%%%%%%%%%%%%%%%%%
%%%%%%%%%%%%%%%%%%%%%%%%%%%%%%%%%%%%%%%%%%%%

%% AMS THM

\theoremstyle{definition}% default
\newtheorem{thm}{Teorema}[section]
\newtheorem{lem}[thm]{Lemma}
\newtheorem{prop}[thm]{Proposizione}
\newtheorem{cor}[thm]{Corollario}
\newtheorem{esempio}[thm]{Esempio}
\theoremstyle{plain}
\newtheorem{definizione}[thm]{Definizione}
\theoremstyle{remark}
\newtheorem*{oss}{Osservazione}


%%%%%%%%%%%%%%%%%%%%%%%%%%%%%%%%%%%%%%%%%%%%
%%%%%%%%%%%%%%%%%%%%%%%%%%%%%%%%%%%%%%%%%%%%

\use{hyperref}
\hypersetup{%
        pdfauthor={Davide Peccioli},
        pdfsubject={},
        allcolors=black,
        citecolor=black,
%	colorlinks=true,
        bookmarksopen=true}
\setcounter{secnumdepth}{0} % rimuove i numeri di sezione senza rimuovere le ref
\renewcommand{\href}[2]{\textcolor{blue}{#2}} % disabilita il comando href
\use{enotez} %
\setenotez{%
 mark-format = \footnotemarkformatting % Mette i numeri tra parentesi quadre%
}\let\footnote=\endnote % rende tutte le note a pié pagina come delle note a fine file 


\let\olddocument\document % modifico l'ambiende documenti per non dover stampare \printendnote
\let\oldenddocument\enddocument
\renewenvironment{document}%
{%
  \olddocument
}{%
  \printendnotes\oldenddocument
}
\renewcommand{\thethm}{\arabic{thm}}

\usepackage[hyperref]{biblatex}
\addbibresource{~/Documents/org/roam/bib/master.bib}
\author{Davide Peccioli}
\date{\today}
\title{Ordinali}
\begin{document}

Contesto: \href{20250130104245-morse_kelly_set_theory.org}{Morse Kelly Set Theory}
\section{Definizione}
\label{sec:orga560c21}
Un \textbf{\textbf{ordinale}} è un \href{20250130104331-insieme_mk.org}{insieme} \href{20250203110714-classe_transitiva.org}{transitivo} tale che tutti i suoi elementi siano transitivi.

Gli ordinali sono generalmente indicati con una lettera greca minuscola, e
\begin{equation*}
\operatorname{Ord}
\end{equation*}
è la \href{20250130104320-classe_mk.org}{classe} degli ordinali
\section{Proprietà}
\label{sec:orgf98b966}
\begin{enumerate}
\item Se \(\alpha \in \operatorname{Ord}\) allora \(\alpha \subseteq \operatorname{Ord}\) (vedi \href{20250131155822-operazioni_insiemistiche_tra_classi_mk.org}{Sottoclasse MK}) e \(\operatorname{S}(\alpha) \in \operatorname{Ord}\) (vedi \href{20250202124648-successore_di_un_insieme_mk.org}{Successore di un insieme MK}).
\item Se \(x\) è un \href{20250130104331-insieme_mk.org}{insieme} di ordinali, allora (vedi \href{20250131155822-operazioni_insiemistiche_tra_classi_mk.org}{Classe Unione Generalizzata})
\begin{equation*}
 \bigcup x \in \operatorname{Ord}
\end{equation*}
\item La classe \(\operatorname{Ord}\) è \href{20250203110714-classe_transitiva.org}{transitiva}; infatti, se \(\beta \in \alpha \in \operatorname{Ord}\), allora \(\alpha \subseteq \operatorname{Ord}\) e quindi \(\beta \in \operatorname{Ord}\).
\item \(\operatorname{Ord}\) è una \href{20250130104320-classe_mk.org}{classe propria}.
\end{enumerate}
\section{Teorema}
\label{sec:org449084c}
Siano \(\alpha, \beta \in \operatorname{Ord}\) due \href{20250203111003-ordinali.org}{ordinali}. Allora vale esattamente una delle seguenti:
\begin{equation*}
\alpha \in \beta;\quad \alpha = \beta;\quad \beta \in \alpha
\end{equation*}
\subsection{Dimostrazione}
\label{sec:orgb83983b}
Per l'Axiom of foundation (vedi \href{20250131180704-nessun_insieme_appartiene_a_se_stesso.org}{Nessuna classe appartiene a se stessa}) necessariamente non possono essere vere due opzioni contemporaneamente.

Basta quindi dimostrare che almeno una delle opzioni è verificata, ovvero che
\begin{equation*}
A = \set{\alpha \in \operatorname{Ord}\,|\, \exists\, \beta \in \operatorname{Ord}\ \left(\alpha\notin \beta \,\land\, \alpha\neq\beta \,\land\, \beta\notin\alpha\right)}
\end{equation*}
è \href{20250131161811-insieme_vuoto_mk.org}{vuoto}.

Se per assurdo \(A\neq \emptyset\), allora, per l'Axiom of Foundation, esiste \(\overline{\alpha} \in A\) tale che
\begin{equation*}
\overline{\alpha}\cap A =\emptyset
\end{equation*}

Allora
\begin{equation*}
B \coloneqq \set{\beta \in \operatorname{Ord}\,|\, \beta\notin\overline{\alpha} \,\land\, \beta\neq\overline{\alpha} \,\land\, \overline{\alpha}\notin\beta}
\end{equation*}
è non vuota (poiché \(\overline{\alpha} \in A\), e dunque esiste \(\beta\) che soddisfa le condizoini di cui sopra) e pertanto, per l'Axiom of Foundation, esiste \(\overline{\beta} \in B\) tale che \(\overline{\beta}\cap B =\emptyset\).

Se \(\gamma \in \overline{\alpha}\), allora \(\gamma\notin A\) poiché \(\overline{\alpha}\cap A = \emptyset\) e quindi, in particolare
\begin{equation*}
\overline{\beta} \in \gamma \,\lor\, \overline{\beta} = \gamma \,\lor\, \gamma \in \overline{\beta}
\end{equation*}
(poiché se \(\gamma \notin A\) allora non esistono \(\beta \in \operatorname{Ord}\) tali che nessuna delle tre opzioni valga. Pertanto per \(\overline{\beta}\) deve necessariamente valere una delle tre opzioni)

Le prime due opzioni, siccome \(\overline{\alpha} \in A \subseteq \operatorname{Ord}\) e quindi \(\overline{\alpha}\) è \href{20250203110714-classe_transitiva.org}{transitivo}, implicano che \(\overline{\beta} \in\overline{\alpha}\). Questo è assurdo, poiché \(\overline{\beta} \in B\), e dunque \(\overline{\beta}\notin\overline{\alpha}\).

Pertanto \(\gamma \in \overline{\beta}\). Siccome \(\gamma \in \overline{\alpha}\) è arbitrario, otteniamo \(\overline{\alpha} \subseteq \overline{\beta}\).

Sia invece \(\gamma \in \overline{\beta}\). Allora \(\gamma \notin B\), poiché \(\overline{\beta}\cap B =\emptyset\) e in particolare vale una delle seguenti:
\begin{equation*}
\overline{\alpha} \in \gamma \,\lor\, \overline{\alpha} = \gamma \,\lor\, \gamma \in \overline{\alpha}
\end{equation*}
Le prime due opzioni, siccome \(\overline{\beta} \in \operatorname{Ord}\) è transitivo, implicano che \(\overline{\alpha} \in \overline{\beta}\). Questo è assurdo, poiché \(\overline{\beta} \in B\) e dunque \(\overline{\alpha} \notin \overline{\beta}\).
Pertanto \(\gamma \in \overline{\alpha}\). Siccome \(\gamma \in \overline{\beta}\) è arbitrario, otteniamo \(\overline{\beta} = \overline{\alpha}\).

Dunque \(\overline{\alpha}=\overline{\beta}\). Assurdo poiché \(\overline{\alpha} \in A\). Dunque \(A=\emptyset\).
\section{Corollario}
\label{sec:orgd9bb9e4}
Dunque \(\in\) è un \href{20250203104134-buon_ordine_mk.org}{buon ordine} \href{20250203101604-ordine.org}{stretto} su \(\operatorname{Ord}\).
\section{Notazione}
\label{sec:org2ac800a}
Se \(\alpha, \beta \in \operatorname{Ord}\) scriveremo:
\begin{itemize}
\item \(\alpha<\beta\) significa \(\alpha \in \beta\);
\item \(\alpha\le\beta\) significa \(\alpha \in \beta\) oppure \(\alpha=\beta\).
\end{itemize}

Se \(A \subseteq \operatorname{Ord}\), diciamo che un elemento è il minimo di \(A\) se è l'\href{20250203102516-massimo_e_minimo.org}{elemento \(\in\)-minimale di \(A\)}.

Scriveremo \(A \le \operatorname{Ord}\) per dire che
\begin{equation*}
A \in \operatorname{Ord} \,\lor\, A = \operatorname{Ord}
\end{equation*}
\section{Osservazione}
\label{sec:org6140322}
Ogni ordinale \(\alpha\) è dotato di un \href{20250203104134-buon_ordine_mk.org}{buon ordine}, in quanto \(\alpha \subseteq \operatorname{Ord}\). Se \(\beta \in \alpha\), allora \(\beta = \operatorname{pred}(\beta,\alpha;\le)\) (vedi \href{20250206120526-segmento_iniziale_per_un_ordine.org}{Insieme dei predecessori}).
\section{Proposizione}
\label{sec:org52fb019}
Siano \(\alpha, \beta\) due ordinali.

\begin{enumerate}
\item Se \(f:\alpha \longrightarrow\beta\) è una \href{20250203132953-funzione_monotona.org}{funzione crescente}, allora
\begin{equation*}
 \gamma \le f(\gamma )
\end{equation*}
per ogni \(\gamma \in \alpha\). Inoltre \(\alpha\le \beta\).
\item Se \(f:\alpha \longrightarrow\beta\) è un \href{20250203110432-isomorfismo_tra_ordini.org}{isomorfismo}, allora \(\alpha=\beta\) e \(f\) è l'identità.
\end{enumerate}
\section{Proprietà}
\label{sec:orgb6ca9e7}
Siano \(\alpha,\beta \in \operatorname{Ord}\) \href{20250203111003-ordinali.org}{ordinali}, e sia \(<\) l'\href{20250203101604-ordine.org}{ordine} \hyperref[sec:org449084c]{dato dall'inclusione}.

\begin{enumerate}
\item \(\alpha < \beta\) se e solo se \(\alpha \subset \beta\) (vedi \href{20250131155822-operazioni_insiemistiche_tra_classi_mk.org}{Sottoclasse MK}).
\item \(\alpha\le\beta\) se e solo se \(\alpha \subseteq \beta\).
\item \(\alpha < \beta\) se e solo se \(\operatorname{S}(\alpha) \le \beta\) (vedi \href{20250202124648-successore_di_un_insieme_mk.org}{Successore di un insieme MK}).
\item \(\alpha < \beta\) se e solo se \(\operatorname{S}(\alpha)< \operatorname{S}(\beta)\).
\item se \(x \subseteq \alpha\) allora \(\displaystyle \bigcup x = \alpha \,\lor\, \bigcup x < \alpha\). (vedi \href{20250131155822-operazioni_insiemistiche_tra_classi_mk.org}{Classe Unione Generalizzata})
\item \(\displaystyle \bigcup\left(\operatorname{S}(\alpha)\right) = \alpha\).
\item \(\displaystyle \alpha= \operatorname{S}\left(\bigcup \alpha\right)\) oppure \(\alpha=\bigcup\alpha\).
\item Sono fatti equivalenti
\begin{enumerate}
\item \(\bigcup \alpha = \alpha\)
\item \(\alpha=0\) oppure \(\alpha\) è \href{20250203161132-ordinale_limite.org}{ordinale limite} (vedi \href{20250202130045-insieme_dei_numeri_naturali_mk.org}{Insieme dei numeri naturali MK} e \href{20250203161110-numeri_naturali_sono_ordinali.org}{Ordinale omega})
\item \(\langle \alpha,<\rangle\) non ha \href{20250203102516-massimo_e_minimo.org}{massimo}.
\end{enumerate}
\item I tre punti precedenti indicano che: se \(\alpha\) è un ordinale successore, allora
\begin{equation*}
 \bigcup \operatorname{S}(\alpha) = \alpha = \operatorname{S}\left(\bigcup\alpha\right)
\end{equation*}
\item Inoltre, si ha che
\begin{equation*}
 \alpha= \bigcup \set{\operatorname{S}(\beta)\mid \beta<\alpha}
\end{equation*}
\end{enumerate}
\section{Caratterizzazione degli ordinali limite}
\label{sec:orgd7d0bc3}
Da questo lemma segue banalmente che \(\lambda\) è un \href{20250203161132-ordinale_limite.org}{ordinale limite} se e solo se
\begin{equation*}
\lambda=\bigcup\lambda>0
\end{equation*}
\section{Nessuna catena discendente di ordinali}
\label{sec:org362c64a}
Non esiste nessuna \href{20250102120836-catena.org}{catena discendente} di \href{20250203111003-ordinali.org}{ordinali}, ovvero
\begin{equation*}
\lnot\,\exists\,f\ \left(f:\N \longrightarrow \operatorname{Ord} \,\land\, \forall\, n \in \N\ \left(f(\operatorname{S}(n)) < f(n)\right)\right)
\end{equation*}
(vedi \href{20250202170607-classe_relazione_binaria.org}{Classe-Funzione}, \href{20250202130045-insieme_dei_numeri_naturali_mk.org}{Insieme dei numeri naturali MK}, \href{20250202124648-successore_di_un_insieme_mk.org}{Successore di un insieme MK})
\section{Intersezione di una classe di ordinali}
\label{sec:org3197a13}
Se \(\emptyset\neq A \subseteq \operatorname{Ord}\) (vedi \href{20250203111003-ordinali.org}{Ordinali}) è una \href{20250131155822-operazioni_insiemistiche_tra_classi_mk.org}{sottoclasse} non \href{20250131161811-insieme_vuoto_mk.org}{vuota}, allora
\begin{equation*}
\min A = \bigcap A
\end{equation*}
(vedi \href{20250203102516-massimo_e_minimo.org}{Elemento Minimo} e \hyperref[sec:org449084c]{Relazione d'ordine sugli ordinali} e \href{20250131155822-operazioni_insiemistiche_tra_classi_mk.org}{Classe Intersezione Generalizzata})
\section{Unione di una classe di ordinali}
\label{sec:orge007107}
Se \(A \subseteq \operatorname{Ord}\) (vedi \href{20250203111003-ordinali.org}{Ordinali}) è un \href{20250130104331-insieme_mk.org}{insieme}, allora \(\bigcup A = \operatorname{sup} A\) (vedi \href{20250131155822-operazioni_insiemistiche_tra_classi_mk.org}{Classe Unione Generalizzata} e \href{20250203102516-massimo_e_minimo.org}{Infimum e supremum} e \hyperref[sec:org449084c]{Relazione d'ordine sugli ordinali})
\end{document}
