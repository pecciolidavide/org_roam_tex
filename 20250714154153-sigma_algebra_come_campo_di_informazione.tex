% Created 2026-02-07 Sat 19:31
% Intended LaTeX compiler: pdflatex
\documentclass[10pt]{article}
%% CREATO CON ORG - EMACS
\newcommand{\use}[2][]{\usepackage[#1]{#2}}
% PACCHETTI FONDAMENTLAI
\use[utf8]{inputenc}
\use[T1]{fontenc}
\use{graphicx}
\use{longtable}
\use{wrapfig}
\use{rotating}
\use[normalem]{ulem}
\use{amsmath}
\use{amsthm}
\use{amssymb}

\use{eucal} % Cambia mathcal{...}

\use{capt-of}
\use[italian]{babel}
\use[babel]{csquotes}
% bib la TEX lo carica in automatico org-cite
\use{microtype}
\use{lmodern}
\use{subfig} % sottofigure
\use{multicol} % due colonne
\use{lipsum} % lorem ipsum
\use{color} % colori in latex
\use{parskip} % rimuove l'indentazione dei nuovi paragrafi %% Add parbox=false to all new tcolorbox
\use{centernot}
\use[outline]{contour}\contourlength{3pt}
\use{fancyhdr}
\use{layout}
\use[most]{tcolorbox} % Riquadri colorati
\use{ifthen} % IFTHEN
\use{geometry}

% pacchetti matematica
\use{yhmath}
\use{dsfont}
\use{mathrsfs}
\use{cancel} % semplificare
\use{polynom} %divisione tra polinomi
\use{forest} % grafi ad albero
\use{booktabs} % tabelle
\use{commath} %simboli e differenziali
\use{bm} %bold
\use[fulladjust]{marginnote} %to use marginnote for date notes
\use{arrayjobx}%array
\use[intlimits]{empheq} % Riquadri colorati attorno alle equazioni
\use{mathtools}
\use{circuitikz} % Disegnare i circuiti
\use{mathtools}
\use{stmaryrd} % [[ \llbracket ]] \rrbracket
\use{bussproofs} % dimostrazioni

%%%%%%%%%%%%%


%%%% QUIVER
\newcommand{\duepunti}{\,\mathchar\numexpr"6000+`:\relax\,}
% A TikZ style for curved arrows of a fixed height, due to AndréC.
\tikzset{curve/.style={settings={#1},to path={(\tikztostart)
    .. controls ($(\tikztostart)!\pv{pos}!(\tikztotarget)!\pv{height}!270:(\tikztotarget)$)
    and ($(\tikztostart)!1-\pv{pos}!(\tikztotarget)!\pv{height}!270:(\tikztotarget)$)
    .. (\tikztotarget)\tikztonodes}},
    settings/.code={\tikzset{quiver/.cd,#1}
        \def\pv##1{\pgfkeysvalueof{/tikz/quiver/##1}}},
    quiver/.cd,pos/.initial=0.35,height/.initial=0}

% TikZ arrowhead/tail styles.
\tikzset{tail reversed/.code={\pgfsetarrowsstart{tikzcd to}}}
\tikzset{2tail/.code={\pgfsetarrowsstart{Implies[reversed]}}}
\tikzset{2tail reversed/.code={\pgfsetarrowsstart{Implies}}}
% TikZ arrow styles.
\tikzset{no body/.style={/tikz/dash pattern=on 0 off 1mm}}
%%%%%%%%%%


%% DEFINIZIONI COMANDI MATEMATICI
\let\sin\relax %TOGLIE LA DEFINIZIONE SU "\sin"

% cambia la definizione di empty set
% ---
\let\oldemptyset\emptyset
% ---
% \let\emptyset\varnothing
% ---
% \let\emptyset\relax
% \newcommand{\emptyset}{\text{\textnormal{\O}}}
% ---

\DeclareMathOperator{\bounded}{bd}
\DeclareMathOperator{\sin}{sen}
\DeclareMathOperator{\epi}{Epi}
\DeclareMathOperator{\cl}{cl}
\DeclareMathOperator{\graph}{graph}
\DeclareMathOperator{\arcsec}{arcsec}
\DeclareMathOperator{\arccot}{arccot}
\DeclareMathOperator{\arccsc}{arccsc}
\DeclareMathOperator{\spettro}{Spettro}
\DeclareMathOperator{\nulls}{nullspace}
\DeclareMathOperator{\dom}{dom}
\DeclareMathOperator{\ar}{ar}
\DeclareMathOperator{\const}{Const}
\DeclareMathOperator{\fun}{Fun}
\DeclareMathOperator{\rel}{Rel}
\DeclareMathOperator{\altezza}{ht}
\let\det\relax %TOGLIE LA DEFINIZIONE SU "\det"
\DeclareMathOperator{\det}{det}
\DeclareMathOperator{\End}{End}
\DeclareMathOperator{\gl}{GL}
\def\Id{\mathrm{Id}}
\def\id{\mathrm{id}}
\DeclareMathOperator{\I}{\mathds{1}}
\DeclareMathOperator{\II}{II}
\DeclareMathOperator{\rank}{rank}
\DeclareMathOperator{\tr}{tr}
\DeclareMathOperator{\tc}{t.c.}
\DeclareMathOperator{\T}{T}
\DeclareMathOperator{\var}{Var}
\DeclareMathOperator{\cov}{Cov}
\DeclareMathOperator{\st}{st}
\DeclareMathOperator{\mon}{Mon}
\newcommand{\card}[1]{\left\vert #1 \right\vert}
\newcommand{\trasposta}[1]{\prescript{\text{T}}{}{#1}}
\newcommand{\1}{\mathds{1}}
\newcommand{\R}{\mathds{R}}
\newcommand{\diesis}{\#}
\newcommand{\bemolle}{\flat}
\newcommand{\nonstandard}[1]{\prescript{*}{}{#1}}
\newcommand{\starR}{\nonstandard{\R}}
\newcommand{\borel}{\mathscr{B}}
\newcommand{\lebesgue}[1]{\mathscr{L}\left(#1\right)}
\newcommand{\media}{\mathds{E}}
\newcommand{\K}{\mathds{K}}
\newcommand{\A}{\mathds{A}}
\newcommand{\Q}{\mathds{Q}}
\newcommand{\N}{\mathds{N}}
\newcommand{\C}{\mathds{C}}
\newcommand{\Z}{\mathds{Z}}
\newcommand{\qo}{\hspace{1em}\text{q.o.}\,}
\renewcommand{\tilde}[1]{\widetilde{#1}}
\renewcommand{\parallel}{\mathrel{/\mkern-5mu/}}
\newcommand{\parti}[2][]{\wp_{#1}(#2)}
\newcommand{\diff}[1]{\operatorname{d}_{#1}}
\let\oldvec\vec
\renewcommand{\vec}[1]{\overrightarrow{\vphantom{i}#1}}
\newcommand{\floor}[1]{\left\lfloor #1 \right\rfloor}
\newcommand{\cat}[1]{\mathbf{#1}}
\newcommand{\dfreccia}[1]{\xrightarrow{\ #1 \ }}
\newcommand{\sfreccia}[1]{\xleftarrow{\ #1 \ }}
\newcommand{\formalsum}[2]{{\sum_{#1}^{#2}}{\vphantom{\sum}}'}
\newcommand{\minim}[2]{\mu_{#1}\, \left(#2\right)}
\newcommand{\concat}{\null^{\frown}} % concatenazione di stringe
\newcommand{\godelcode}[1]{\langle\!\langle #1 \rangle\!\rangle}
\newcommand{\godeldec}[1]{(\!(#1)\!)}
\newcommand{\termcode}[1]{\ulcorner #1\urcorner}
\newcommand{\partialto}{\dashrightarrow}
\newcommand{\restricted}{\upharpoonright}
\newcommand{\embeds}{\precsim}
\newcommand{\surjects}{\twoheadrightarrow}
\newcommand{\equipotenti}{\asymp}
%% \newcommand{\dotplus}{\mathbin{\dot{+}}} %% A quanto pare esiste già
\newcommand{\bigdot}{\mathbin{\boldsymbol{\cdot}}}
\newcommand{\dotexp}[1]{^{.#1}}
\newcommand{\conv}{\mathbin{*}}
\newcommand{\convolution}[2]{(#1\conv #2)}
\newcommand{\nil}{\mathfrak{N}}
\newcommand{\divisore}{\mathrel{|}}
\newcommand{\simplesso}[1]{\mathrm{e}_{#1}}

\renewcommand{\iff}{\mathrel{\longleftrightarrow}} %% Notazione Logica.
\newcommand{\oldiff}{\mathrel{\Longleftrightarrow}}
\renewcommand{\implies}{\mathrel{\rightarrow}} %% Notazione Logica
\newcommand{\oldimplies}{\mathrel{\Longrightarrow}}
\renewcommand{\impliedby}{\mathrel{\leftarrow}} %% Notazione Logica
\newcommand{\oldimpliedby}{\mathrel{\Longleftarrow}}

\newcommand{\IFF}{\quad\Longleftrightarrow\quad}
\newcommand{\IMPLICA}{\quad\Longrightarrow\quad}


\renewcommand{\descriptionlabel}[1]{\hspace{\labelsep}\normalfont #1} % remove bold from description


%% Definizione di Divergenza di K-L

\DeclarePairedDelimiterX{\infdivx}[2]{(}{)}{%
  #1\;\delimsize\|\;#2%
}
\newcommand{\kldiv}{D_{KL}\infdivx}

%% Definizione di \dotminus

\makeatletter
\newcommand{\dotminus}{\mathbin{\text{\@dotminus}}}

\newcommand{\@dotminus}{%
  \ooalign{\hidewidth\raise1ex\hbox{.}\hidewidth\cr$\m@th-$\cr}%
}
\makeatother

%tramite i prossimi due comandi posso decidere come scrivere i logaritmi naturali in tutti i documenti: ho infatti eliminato qualsiasi differenza tra "ln" e "log": se si vuole qualcosa di diverso bisogna inserire manualmente il tutto
\let\ln\relax
\DeclareMathOperator{\ln}{ln}
\let\log\relax
\DeclareMathOperator{\log}{log}
%%%%%%

%% NUOVI COMANDI
\newcommand{\straniero}[1]{\textit{#1}} %parole straniere
\newcommand{\titolo}[1]{\textsc{#1}} %titoli
\newcommand{\qedd}{\tag*{$\blacksquare$}} %qed per ambienti matemastici
\renewcommand{\qedsymbol}{$\blacksquare$} %modifica colore qed
\newcommand{\ooverline}[1]{\overline{\overline{#1}}}
\newcommand{\circoletto}[1]{\left(#1\right)^{\text{o}}}
%
\newcommand{\qmatrice}[1]{\begin{pmatrix}
#1_{11} & \cdots & #1_{1n}\\
\vdots & \ddots & \vdots \\
#1_{m1} & \cdots & #1_{mn}
\end{pmatrix}}
%
\newcommand{\parentesi}[2]{%
\underset{#1}{\underbrace{#2}}%
}
%
\newcommand{\norma}[1]{% Norma
\left\lVert#1\right\rVert%
}
\newcommand{\scalare}[2]{% Scalare
\left\langle #1, #2\right\rangle
}
%%%%%

%% RESTRIZIONI
\newcommand{\referenze}[2]{
        \phantomsection{}#2\textsuperscript{\textcolor{blue}{\textbf{#1}}}
}

\let\restriction\relax

\def\restriction#1#2{\mathchoice
              {\setbox1\hbox{${\displaystyle #1}_{\scriptstyle #2}$}
              \restrictionaux{#1}{#2}}
              {\setbox1\hbox{${\textstyle #1}_{\scriptstyle #2}$}
              \restrictionaux{#1}{#2}}
              {\setbox1\hbox{${\scriptstyle #1}_{\scriptscriptstyle #2}$}
              \restrictionaux{#1}{#2}}
              {\setbox1\hbox{${\scriptscriptstyle #1}_{\scriptscriptstyle #2}$}
              \restrictionaux{#1}{#2}}}
\def\restrictionaux#1#2{{#1\,\smash{\vrule height .8\ht1 depth .85\dp1}}_{\,#2}}
%%%%%%%%%%%

%%% FORMATTAZIONE FOOTNOTEMARK

\def\footnotemarkformatting#1{[#1]}
\renewcommand{\thefootnote}{\footnotemarkformatting{\arabic{footnote}}}

%% SEZIONE GRAFICA
\use{tikz}
\usetikzlibrary{matrix, patterns, calc, decorations.pathreplacing, hobby, decorations.markings, decorations.pathmorphing, babel}
\use{tikz-3dplot}
\use{mathrsfs} %per geogebra
\use{tikz-cd}
\tikzset
{
  %surface/.style={fill=black!10, shading=ball,fill opacity=0.4},
  plane/.style={black,pattern=north east lines},
  curve/.style={black,line width=0.5mm},
  dritto/.style={decoration={markings,mark=at position 0.5 with {\arrow{Stealth}}}, postaction=decorate},
  rovescio/.style={decoration={markings,mark=at position 0.5 with {\arrow{Stealth[reversed]}}}, postaction=decorate}
}
\use{pgfplots} % stampare le funzioni
        \pgfplotsset{/pgf/number format/use comma,compat=1.15}
        %\pgfplotsset{compat=1.15} %per geogebra
        \usepgfplotslibrary{fillbetween, polar}
%%%%%%

%% CITAZIONI
\use{lineno}

\newcommand{\citazione}[1]{%
  \begin{quotation}
  \begin{linenumbers}
  \modulolinenumbers[5]
  \begingroup
  \setlength{\parindent}{0cm}
  \noindent #1
  \endgroup
  \end{linenumbers}
  \end{quotation}\setcounter{linenumber}{1}
  }
%%%%%%

%%%%%%%%%%%%%%%%%%%%%%%%%%%%%%%%%%%%%%%%%%%%
%%%%%%%%%%%%%%%%%%%%%%%%%%%%%%%%%%%%%%%%%%%%

%% AMS THM

\theoremstyle{definition}% default
\newtheorem{thm}{Teorema}[section]
\newtheorem{lem}[thm]{Lemma}
\newtheorem{prop}[thm]{Proposizione}
\newtheorem{cor}[thm]{Corollario}
\newtheorem{esempio}[thm]{Esempio}
\theoremstyle{plain}
\newtheorem{definizione}[thm]{Definizione}
\theoremstyle{remark}
\newtheorem*{oss}{Osservazione}


%%%%%%%%%%%%%%%%%%%%%%%%%%%%%%%%%%%%%%%%%%%%
%%%%%%%%%%%%%%%%%%%%%%%%%%%%%%%%%%%%%%%%%%%%

\use{hyperref}
\hypersetup{%
        pdfauthor={Davide Peccioli},
        pdfsubject={},
        allcolors=black,
        citecolor=black,
%	colorlinks=true,
        bookmarksopen=true}
\setcounter{secnumdepth}{0} % rimuove i numeri di sezione senza rimuovere le ref
\renewcommand{\href}[2]{\textcolor{blue}{#2}} % disabilita il comando href
\use{enotez} %
\setenotez{%
 mark-format = \footnotemarkformatting % Mette i numeri tra parentesi quadre%
}\let\footnote=\endnote % rende tutte le note a pié pagina come delle note a fine file 


\let\olddocument\document % modifico l'ambiende documenti per non dover stampare \printendnote
\let\oldenddocument\enddocument
\renewenvironment{document}%
{%
  \olddocument
}{%
  \printendnotes\oldenddocument
}
\renewcommand{\thethm}{\arabic{thm}}

\usepackage[hyperref]{biblatex}
\addbibresource{~/Documents/org/roam/bib/master.bib}
\author{Davide Peccioli}
\date{\today}
\title{}
\begin{document}

\section{Sigma-algebra come campo di informazione}
\label{sec:orgf69b43f}
FONTE: CHATGPT
\subsection{Contesto: \(\sigma\)-algebra come struttura di conoscenza}
\label{sec:orgbe8b057}

Sia \((\Omega, \mathcal{F}, \mathbb{P})\) uno \href{20250711175559-spazio_di_probabilita.org}{spazio di probabilità}.

\begin{itemize}
\item \(\Omega\) è lo spazio degli esiti: tutte le possibili realtà del mondo.
\item \(\mathcal{F}\) è una \href{20250526100313-sigma_algebra.org}{\(\sigma\)-algebra} di sottoinsiemi di \(\Omega\): rappresenta gli eventi osservabili o misurabili.
\item \(\mathbb{P}\) è una \href{20250704105515-misura_di_probabilita.org}{misura di probabilità} su \((\Omega, \mathcal{F})\).
\end{itemize}

Ogni sottoinsieme \(A \in \mathcal{F}\) rappresenta un evento che ha significato empirico: possiamo sapere se è accaduto o meno.
\subsection{La \(\sigma\)-algebra come insieme delle informazioni disponibili}
\label{sec:org25f6278}

Una \(\sigma\)-algebra \(\mathcal{G} \subseteq \mathcal{F}\) può essere vista come l'insieme delle informazioni che un osservatore possiede.

\begin{itemize}
\item Se \(\mathcal{G}\) è ``piccola'', l’osservatore sa poco.
\item Se \(\mathcal{G}\) è ``grande'', sa molto (fino a \(\mathcal{F}\), conoscenza completa).
\end{itemize}

Ogni evento \(A \in \mathcal{G}\) è un fatto che l’osservatore può distinguere.

Se \(A \notin \mathcal{G}\), allora l’osservatore non può sapere se \(A\) è accaduto.
\subsection{Esempio: partizionare lo spazio degli eventi}
\label{sec:org75e89f6}

Sia \(\Omega = \{ \omega_1, \omega_2, \omega_3, \omega_4 \}\), e l’osservatore sa solo se l’esito è nei primi due o negli ultimi due. Allora:

\[
\mathcal{G} = \{ \emptyset, \Omega, \{ \omega_1, \omega_2 \}, \{ \omega_3, \omega_4 \} \}
\]

Questa \(\sigma\)-algebra rappresenta una partizione dell’informazione.
\subsection{\(\sigma\)-algebra generata da una variabile aleatoria}
\label{sec:orga1d02bd}

Sia \(X : \Omega \to \mathbb{R}\) una \href{20250711175937-variabile_aleatoria.org}{variabile aleatoria}. La \href{20250714154501-sigma_algebra_generata_da_una_variabile_aleatoria.org}{\(\sigma\)-algebra generata da \(X\)} è:
\begin{equation*}
\sigma(X) = \{ X^{-1}(B) : B \in \mathcal{B}(\mathbb{R}) \}
\end{equation*}

Questa contiene tutti gli eventi determinabili in base al valore di \(X\), ed è l'informazione completa codificata da \(X\).
\subsubsection{Significato informativo}
\label{sec:orgbfe3328}

\begin{itemize}
\item \(\sigma(X)\) rappresenta tutta l'informazione contenuta in \(X\).
\item Se conosciamo \(X(\omega)\), allora possiamo sapere esattamente quali eventi in \(\sigma(X)\) si sono verificati.
\item Ogni evento \(A \in \sigma(X)\) è determinabile unicamente tramite \(X\).
\end{itemize}

Se \(A \notin \sigma(X)\), non possiamo sapere se \(A\) è accaduto conoscendo solo \(X\).
\subsubsection{Esempio esplicito}
\label{sec:orgc8d42a9}

Sia:

\begin{itemize}
\item \(\Omega = \{ \omega_1, \omega_2, \omega_3, \omega_4 \}\)
\item \(X : \Omega \to \mathbb{R}\) definita da:
\begin{itemize}
\item \(X(\omega_1) = 0\)
\item \(X(\omega_2) = 0\)
\item \(X(\omega_3) = 1\)
\item \(X(\omega_4) = 2\)
\end{itemize}
\end{itemize}

Allora \(X^{-1}(\{0\}) = \{ \omega_1, \omega_2 \}\), \(X^{-1}(\{1\}) = \{ \omega_3 \}\), \(X^{-1}(\{2\}) = \{ \omega_4 \}\)

Quindi: \(\sigma(X)\) è la\(\sigma\)-algebra generata dalla partizione \(\{ \{ \omega_1, \omega_2 \}, \{ \omega_3 \}, \{ \omega_4 \} \}\)

In particolare:
\begin{equation*}
\sigma(X) = \{ \emptyset, \Omega, \{ \omega_1, \omega_2 \}, \{ \omega_3 \}, \{ \omega_4 \}, \{ \omega_1, \omega_2, \omega_3 \}, \{ \omega_1, \omega_2, \omega_4 \}, \{ \omega_3, \omega_4 \} \}
\end{equation*}
\subsubsection{Interpretazione informativa}
\label{sec:org61cab36}

\begin{itemize}
\item Se \(X(\omega) = 0\), so solo che \(\omega \in \{ \omega_1, \omega_2 \}\)
\item Se \(X(\omega) = 1\), allora \(\omega = \omega_3\)
\item Se \(X(\omega) = 2\), allora \(\omega = \omega_4\)
\end{itemize}

Ogni evento in \(\sigma(X)\) è un’unione di blocchi della partizione indotta da \(X\).

Esempi:

\begin{itemize}
\item \(\{ \omega_1, \omega_2, \omega_4 \} \in \sigma(X)\), perché corrisponde a \(X \in \{ 0, 2 \}\)
\item \(\{ \omega_1, \omega_3 \} \notin \sigma(X)\), perché non è unione di intere fibre.
\end{itemize}
\subsubsection{Teorema: caratterizzazione di \(\sigma(X)\)}
\label{sec:orgb70a6fd}

\begin{itemize}
\item \(Y\) è \(\sigma(X)\)-misurabile \(\iff\) esiste \(f : \mathbb{R} \to \mathbb{R}\) tale che \(Y = f(X)\)
\item \(A \subseteq \Omega\) è in \(\sigma(X)\) \(\iff\) \(A = X^{-1}(B)\) per qualche Borel \(B \subseteq \mathbb{R}\)
\end{itemize}
\subsubsection{Conclusione}
\label{sec:orgfd6c869}

\(\sigma(X)\) è esattamente tutta e sola l’informazione che può essere dedotta conoscendo il valore di \(X(\omega)\).
\subsection{Variabili condizionate e informazione}
\label{sec:org5e1c486}

La probabilità condizionata \(\mathbb{E}[Y \mid \mathcal{G}]\) è la miglior stima di \(Y\) dato ciò che so in \(\mathcal{G}\).  
Quindi \(\mathcal{G}\) rappresenta l'informazione disponibile, e la condizionata è un modo per fare inferenza su variabili non note usando quella informazione.

Se \(Y\) è misurabile rispetto a \(\sigma(X)\), allora esiste una funzione \(f\) tale che \(Y = f(X)\), e in particolare:

\[
\mathbb{E}[Y \mid \sigma(X)] = g(X)
\]

per qualche funzione \(g\).
\subsection{Estensione al caso multidimensionale}
\label{sec:org1faf4b8}

Se \(X = (X_1, \dots, X_n)\), allora:

\[
\sigma(X) = \sigma(X_1, \dots, X_n)
\]

cioè la \(\sigma\)-algebra contiene l'informazione congiunta di tutte le componenti.
\subsection{Informazione nel tempo: filtrazioni}
\label{sec:orgce50298}

Una filtrazione \((\mathcal{F}_t)_{t \geq 0}\) è una famiglia crescente di \(\sigma\)-algebre:

\[
\mathcal{F}_s \subseteq \mathcal{F}_t \quad \text{per ogni } s \leq t
\]

\begin{itemize}
\item \(\mathcal{F}_t\) rappresenta tutta l’informazione disponibile fino al tempo \(t\).
\item Un processo \(X_t\) è adattato se \(X_t\) è \(\mathcal{F}_t\)-misurabile.
\end{itemize}
\subsection{Partizioni e conoscibilità}
\label{sec:org8d4d05f}

Ogni \(\sigma\)-algebra \(\mathcal{G}\) induce una partizione di \(\Omega\) nei suoi eventi ``atomici'', cioè quelli minimali e distinguibili.

\begin{itemize}
\item Ogni osservatore con informazione \(\mathcal{G}\) sa solo a quale blocco della partizione appartiene \(\omega\).
\item Non può distinguere elementi dello stesso blocco.
\end{itemize}
\subsection{Riepilogo}
\label{sec:org2df7f90}

\begin{center}
\begin{tabular}{ll}
Concetto & Interpretazione\\
\hline
\(\sigma\)-algebra \(\mathcal{G}\) & Informazione disponibile\\
Evento \(A \in \mathcal{G}\) & Evento conoscibile\\
\(\mathbb{E}[X \mid \mathcal{G}]\) & Stima di \(X\) con l'informazione\\
\(\sigma(X)\) & Tutta l’informazione contenuta in \(X\)\\
Filtrazione \((\mathcal{F}_t)\) & Informazione che evolve nel tempo\\
Partizione & Eventi distinguibili\\
\end{tabular}
\end{center}
\end{document}
