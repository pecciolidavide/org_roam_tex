% Created 2026-02-07 Sat 19:32
% Intended LaTeX compiler: pdflatex
\documentclass[10pt]{article}
%% CREATO CON ORG - EMACS
\newcommand{\use}[2][]{\usepackage[#1]{#2}}
% PACCHETTI FONDAMENTLAI
\use[utf8]{inputenc}
\use[T1]{fontenc}
\use{graphicx}
\use{longtable}
\use{wrapfig}
\use{rotating}
\use[normalem]{ulem}
\use{amsmath}
\use{amsthm}
\use{amssymb}

\use{eucal} % Cambia mathcal{...}

\use{capt-of}
\use[italian]{babel}
\use[babel]{csquotes}
% bib la TEX lo carica in automatico org-cite
\use{microtype}
\use{lmodern}
\use{subfig} % sottofigure
\use{multicol} % due colonne
\use{lipsum} % lorem ipsum
\use{color} % colori in latex
\use{parskip} % rimuove l'indentazione dei nuovi paragrafi %% Add parbox=false to all new tcolorbox
\use{centernot}
\use[outline]{contour}\contourlength{3pt}
\use{fancyhdr}
\use{layout}
\use[most]{tcolorbox} % Riquadri colorati
\use{ifthen} % IFTHEN
\use{geometry}

% pacchetti matematica
\use{yhmath}
\use{dsfont}
\use{mathrsfs}
\use{cancel} % semplificare
\use{polynom} %divisione tra polinomi
\use{forest} % grafi ad albero
\use{booktabs} % tabelle
\use{commath} %simboli e differenziali
\use{bm} %bold
\use[fulladjust]{marginnote} %to use marginnote for date notes
\use{arrayjobx}%array
\use[intlimits]{empheq} % Riquadri colorati attorno alle equazioni
\use{mathtools}
\use{circuitikz} % Disegnare i circuiti
\use{mathtools}
\use{stmaryrd} % [[ \llbracket ]] \rrbracket
\use{bussproofs} % dimostrazioni

%%%%%%%%%%%%%


%%%% QUIVER
\newcommand{\duepunti}{\,\mathchar\numexpr"6000+`:\relax\,}
% A TikZ style for curved arrows of a fixed height, due to AndréC.
\tikzset{curve/.style={settings={#1},to path={(\tikztostart)
    .. controls ($(\tikztostart)!\pv{pos}!(\tikztotarget)!\pv{height}!270:(\tikztotarget)$)
    and ($(\tikztostart)!1-\pv{pos}!(\tikztotarget)!\pv{height}!270:(\tikztotarget)$)
    .. (\tikztotarget)\tikztonodes}},
    settings/.code={\tikzset{quiver/.cd,#1}
        \def\pv##1{\pgfkeysvalueof{/tikz/quiver/##1}}},
    quiver/.cd,pos/.initial=0.35,height/.initial=0}

% TikZ arrowhead/tail styles.
\tikzset{tail reversed/.code={\pgfsetarrowsstart{tikzcd to}}}
\tikzset{2tail/.code={\pgfsetarrowsstart{Implies[reversed]}}}
\tikzset{2tail reversed/.code={\pgfsetarrowsstart{Implies}}}
% TikZ arrow styles.
\tikzset{no body/.style={/tikz/dash pattern=on 0 off 1mm}}
%%%%%%%%%%


%% DEFINIZIONI COMANDI MATEMATICI
\let\sin\relax %TOGLIE LA DEFINIZIONE SU "\sin"

% cambia la definizione di empty set
% ---
\let\oldemptyset\emptyset
% ---
% \let\emptyset\varnothing
% ---
% \let\emptyset\relax
% \newcommand{\emptyset}{\text{\textnormal{\O}}}
% ---

\DeclareMathOperator{\bounded}{bd}
\DeclareMathOperator{\sin}{sen}
\DeclareMathOperator{\epi}{Epi}
\DeclareMathOperator{\cl}{cl}
\DeclareMathOperator{\graph}{graph}
\DeclareMathOperator{\arcsec}{arcsec}
\DeclareMathOperator{\arccot}{arccot}
\DeclareMathOperator{\arccsc}{arccsc}
\DeclareMathOperator{\spettro}{Spettro}
\DeclareMathOperator{\nulls}{nullspace}
\DeclareMathOperator{\dom}{dom}
\DeclareMathOperator{\ar}{ar}
\DeclareMathOperator{\const}{Const}
\DeclareMathOperator{\fun}{Fun}
\DeclareMathOperator{\rel}{Rel}
\DeclareMathOperator{\altezza}{ht}
\let\det\relax %TOGLIE LA DEFINIZIONE SU "\det"
\DeclareMathOperator{\det}{det}
\DeclareMathOperator{\End}{End}
\DeclareMathOperator{\gl}{GL}
\def\Id{\mathrm{Id}}
\def\id{\mathrm{id}}
\DeclareMathOperator{\I}{\mathds{1}}
\DeclareMathOperator{\II}{II}
\DeclareMathOperator{\rank}{rank}
\DeclareMathOperator{\tr}{tr}
\DeclareMathOperator{\tc}{t.c.}
\DeclareMathOperator{\T}{T}
\DeclareMathOperator{\var}{Var}
\DeclareMathOperator{\cov}{Cov}
\DeclareMathOperator{\st}{st}
\DeclareMathOperator{\mon}{Mon}
\newcommand{\card}[1]{\left\vert #1 \right\vert}
\newcommand{\trasposta}[1]{\prescript{\text{T}}{}{#1}}
\newcommand{\1}{\mathds{1}}
\newcommand{\R}{\mathds{R}}
\newcommand{\diesis}{\#}
\newcommand{\bemolle}{\flat}
\newcommand{\nonstandard}[1]{\prescript{*}{}{#1}}
\newcommand{\starR}{\nonstandard{\R}}
\newcommand{\borel}{\mathscr{B}}
\newcommand{\lebesgue}[1]{\mathscr{L}\left(#1\right)}
\newcommand{\media}{\mathds{E}}
\newcommand{\K}{\mathds{K}}
\newcommand{\A}{\mathds{A}}
\newcommand{\Q}{\mathds{Q}}
\newcommand{\N}{\mathds{N}}
\newcommand{\C}{\mathds{C}}
\newcommand{\Z}{\mathds{Z}}
\newcommand{\qo}{\hspace{1em}\text{q.o.}\,}
\renewcommand{\tilde}[1]{\widetilde{#1}}
\renewcommand{\parallel}{\mathrel{/\mkern-5mu/}}
\newcommand{\parti}[2][]{\wp_{#1}(#2)}
\newcommand{\diff}[1]{\operatorname{d}_{#1}}
\let\oldvec\vec
\renewcommand{\vec}[1]{\overrightarrow{\vphantom{i}#1}}
\newcommand{\floor}[1]{\left\lfloor #1 \right\rfloor}
\newcommand{\cat}[1]{\mathbf{#1}}
\newcommand{\dfreccia}[1]{\xrightarrow{\ #1 \ }}
\newcommand{\sfreccia}[1]{\xleftarrow{\ #1 \ }}
\newcommand{\formalsum}[2]{{\sum_{#1}^{#2}}{\vphantom{\sum}}'}
\newcommand{\minim}[2]{\mu_{#1}\, \left(#2\right)}
\newcommand{\concat}{\null^{\frown}} % concatenazione di stringe
\newcommand{\godelcode}[1]{\langle\!\langle #1 \rangle\!\rangle}
\newcommand{\godeldec}[1]{(\!(#1)\!)}
\newcommand{\termcode}[1]{\ulcorner #1\urcorner}
\newcommand{\partialto}{\dashrightarrow}
\newcommand{\restricted}{\upharpoonright}
\newcommand{\embeds}{\precsim}
\newcommand{\surjects}{\twoheadrightarrow}
\newcommand{\equipotenti}{\asymp}
%% \newcommand{\dotplus}{\mathbin{\dot{+}}} %% A quanto pare esiste già
\newcommand{\bigdot}{\mathbin{\boldsymbol{\cdot}}}
\newcommand{\dotexp}[1]{^{.#1}}
\newcommand{\conv}{\mathbin{*}}
\newcommand{\convolution}[2]{(#1\conv #2)}
\newcommand{\nil}{\mathfrak{N}}
\newcommand{\divisore}{\mathrel{|}}
\newcommand{\simplesso}[1]{\mathrm{e}_{#1}}

\renewcommand{\iff}{\mathrel{\longleftrightarrow}} %% Notazione Logica.
\newcommand{\oldiff}{\mathrel{\Longleftrightarrow}}
\renewcommand{\implies}{\mathrel{\rightarrow}} %% Notazione Logica
\newcommand{\oldimplies}{\mathrel{\Longrightarrow}}
\renewcommand{\impliedby}{\mathrel{\leftarrow}} %% Notazione Logica
\newcommand{\oldimpliedby}{\mathrel{\Longleftarrow}}

\newcommand{\IFF}{\quad\Longleftrightarrow\quad}
\newcommand{\IMPLICA}{\quad\Longrightarrow\quad}


\renewcommand{\descriptionlabel}[1]{\hspace{\labelsep}\normalfont #1} % remove bold from description


%% Definizione di Divergenza di K-L

\DeclarePairedDelimiterX{\infdivx}[2]{(}{)}{%
  #1\;\delimsize\|\;#2%
}
\newcommand{\kldiv}{D_{KL}\infdivx}

%% Definizione di \dotminus

\makeatletter
\newcommand{\dotminus}{\mathbin{\text{\@dotminus}}}

\newcommand{\@dotminus}{%
  \ooalign{\hidewidth\raise1ex\hbox{.}\hidewidth\cr$\m@th-$\cr}%
}
\makeatother

%tramite i prossimi due comandi posso decidere come scrivere i logaritmi naturali in tutti i documenti: ho infatti eliminato qualsiasi differenza tra "ln" e "log": se si vuole qualcosa di diverso bisogna inserire manualmente il tutto
\let\ln\relax
\DeclareMathOperator{\ln}{ln}
\let\log\relax
\DeclareMathOperator{\log}{log}
%%%%%%

%% NUOVI COMANDI
\newcommand{\straniero}[1]{\textit{#1}} %parole straniere
\newcommand{\titolo}[1]{\textsc{#1}} %titoli
\newcommand{\qedd}{\tag*{$\blacksquare$}} %qed per ambienti matemastici
\renewcommand{\qedsymbol}{$\blacksquare$} %modifica colore qed
\newcommand{\ooverline}[1]{\overline{\overline{#1}}}
\newcommand{\circoletto}[1]{\left(#1\right)^{\text{o}}}
%
\newcommand{\qmatrice}[1]{\begin{pmatrix}
#1_{11} & \cdots & #1_{1n}\\
\vdots & \ddots & \vdots \\
#1_{m1} & \cdots & #1_{mn}
\end{pmatrix}}
%
\newcommand{\parentesi}[2]{%
\underset{#1}{\underbrace{#2}}%
}
%
\newcommand{\norma}[1]{% Norma
\left\lVert#1\right\rVert%
}
\newcommand{\scalare}[2]{% Scalare
\left\langle #1, #2\right\rangle
}
%%%%%

%% RESTRIZIONI
\newcommand{\referenze}[2]{
        \phantomsection{}#2\textsuperscript{\textcolor{blue}{\textbf{#1}}}
}

\let\restriction\relax

\def\restriction#1#2{\mathchoice
              {\setbox1\hbox{${\displaystyle #1}_{\scriptstyle #2}$}
              \restrictionaux{#1}{#2}}
              {\setbox1\hbox{${\textstyle #1}_{\scriptstyle #2}$}
              \restrictionaux{#1}{#2}}
              {\setbox1\hbox{${\scriptstyle #1}_{\scriptscriptstyle #2}$}
              \restrictionaux{#1}{#2}}
              {\setbox1\hbox{${\scriptscriptstyle #1}_{\scriptscriptstyle #2}$}
              \restrictionaux{#1}{#2}}}
\def\restrictionaux#1#2{{#1\,\smash{\vrule height .8\ht1 depth .85\dp1}}_{\,#2}}
%%%%%%%%%%%

%%% FORMATTAZIONE FOOTNOTEMARK

\def\footnotemarkformatting#1{[#1]}
\renewcommand{\thefootnote}{\footnotemarkformatting{\arabic{footnote}}}

%% SEZIONE GRAFICA
\use{tikz}
\usetikzlibrary{matrix, patterns, calc, decorations.pathreplacing, hobby, decorations.markings, decorations.pathmorphing, babel}
\use{tikz-3dplot}
\use{mathrsfs} %per geogebra
\use{tikz-cd}
\tikzset
{
  %surface/.style={fill=black!10, shading=ball,fill opacity=0.4},
  plane/.style={black,pattern=north east lines},
  curve/.style={black,line width=0.5mm},
  dritto/.style={decoration={markings,mark=at position 0.5 with {\arrow{Stealth}}}, postaction=decorate},
  rovescio/.style={decoration={markings,mark=at position 0.5 with {\arrow{Stealth[reversed]}}}, postaction=decorate}
}
\use{pgfplots} % stampare le funzioni
        \pgfplotsset{/pgf/number format/use comma,compat=1.15}
        %\pgfplotsset{compat=1.15} %per geogebra
        \usepgfplotslibrary{fillbetween, polar}
%%%%%%

%% CITAZIONI
\use{lineno}

\newcommand{\citazione}[1]{%
  \begin{quotation}
  \begin{linenumbers}
  \modulolinenumbers[5]
  \begingroup
  \setlength{\parindent}{0cm}
  \noindent #1
  \endgroup
  \end{linenumbers}
  \end{quotation}\setcounter{linenumber}{1}
  }
%%%%%%

%%%%%%%%%%%%%%%%%%%%%%%%%%%%%%%%%%%%%%%%%%%%
%%%%%%%%%%%%%%%%%%%%%%%%%%%%%%%%%%%%%%%%%%%%

%% AMS THM

\theoremstyle{definition}% default
\newtheorem{thm}{Teorema}[section]
\newtheorem{lem}[thm]{Lemma}
\newtheorem{prop}[thm]{Proposizione}
\newtheorem{cor}[thm]{Corollario}
\newtheorem{esempio}[thm]{Esempio}
\theoremstyle{plain}
\newtheorem{definizione}[thm]{Definizione}
\theoremstyle{remark}
\newtheorem*{oss}{Osservazione}


%%%%%%%%%%%%%%%%%%%%%%%%%%%%%%%%%%%%%%%%%%%%
%%%%%%%%%%%%%%%%%%%%%%%%%%%%%%%%%%%%%%%%%%%%

\use{hyperref}
\hypersetup{%
        pdfauthor={Davide Peccioli},
        pdfsubject={},
        allcolors=black,
        citecolor=black,
%	colorlinks=true,
        bookmarksopen=true}
\setcounter{secnumdepth}{0} % rimuove i numeri di sezione senza rimuovere le ref
\renewcommand{\href}[2]{\textcolor{blue}{#2}} % disabilita il comando href
\use{enotez} %
\setenotez{%
 mark-format = \footnotemarkformatting % Mette i numeri tra parentesi quadre%
}\let\footnote=\endnote % rende tutte le note a pié pagina come delle note a fine file 


\let\olddocument\document % modifico l'ambiende documenti per non dover stampare \printendnote
\let\oldenddocument\enddocument
\renewenvironment{document}%
{%
  \olddocument
}{%
  \printendnotes\oldenddocument
}
\renewcommand{\thethm}{\arabic{thm}}

\usepackage[hyperref]{biblatex}
\addbibresource{~/Documents/org/roam/bib/master.bib}
\author{Davide Peccioli}
\date{\today}
\title{}
\begin{document}

\newcommand{\U}{\mathcal{U}} % Modello mostro
\newcommand{\acl}{\operatorname{acl}}
\newcommand{\dcl}{\operatorname{dcl}}
\newcommand{\tp}{\operatorname{tp}}
\newcommand{\eq}{\text{eq}}
\newcommand{\Aut}{\operatorname{Aut}}
\newcommand{\equivalentover}[1]{\mathrel{\equiv_{#1}}}
\newcommand{\Shequivalentover}[1]{\mathrel{\overset{\text{Sh}}{\equiv}_{#1}}}
\newcommand{\orbita}{\mathcal{O}}

\begin{prop}
Prove that following are equivalent for every \(A\subseteq\mathcal{U}\)
\begin{enumerate}
\item \(\operatorname{acl}^{\text{eq}} A=\operatorname{dcl}^{\text{eq}}\big(\operatorname{acl} A\big)\)
\item \(\operatorname{Aut}(\mathcal{U}/\operatorname{acl}^{\text{eq}} A)= \operatorname{Aut}(\mathcal{U}/\operatorname{acl} A)\)
\item \(c\mathrel{\equiv_{\operatorname{acl}A}}b\oldiff c\mathrel{\overset{\text{Sh}}{\equiv}_{A}} b\) for all \(c,b \in \U^{\lambda}\).
\end{enumerate}
\end{prop}
\begin{proof}
(\(1.\Rightarrow 2.\)):
Siccome in generale \(\acl A \subseteq \acl^{\eq} A\) allora
\begin{equation*}
\Aut(\U/\acl^\eq A) \subseteq \Aut(\U/\acl A).
\end{equation*}

Viceversa, sia \(f \in \Aut(\U/\acl A)\) e sia \(\mathcal{A} \in \acl^{\eq} A\).
Per \(1.\) allora \(\mathcal{A} \in \dcl^{\eq}\big(\acl A\big)\) e per la caratterizzazione di \(\dcl^{\eq}\) allora esiste \(\psi(x) \in \mathcal{L}(\acl A)\) tale che \(\mathcal{A}=\psi(\U^{x})\).

In particolare esiste \(\psi(x;z) \in \mathcal{L}\), con \(\card{z} = \card{\acl A}\) tale che, detta \(a\) una enumerazione di \(\acl A\):
\begin{equation*}
\psi(x) = \psi(x;a)\quad\oldimplies\quad \mathcal{A}=\psi(\U^{x};a).
\end{equation*}
Pertanto:
\begin{equation*}
f\mathcal{A}=f[\psi(\U^{x};a)] = \psi(\U^{x};fa) = \psi(\U^{x};a) = \mathcal{A}.
\end{equation*}

Per arbitrarietà di \(\mathcal{A}\) si ha \(f\mathcal{A}=\mathcal{A}\) per ogni \(\mathcal{A} \in \acl^{\eq} A\): \(f \in \Aut(\U/\acl^{\eq} A)\).

(\(2.\Rightarrow 3.\)): È noto che per ogni \(c,b \in \U^{\lambda}\)
\begin{equation*}
c \Shequivalentover{A} b\quad\oldiff\quad c\equivalentover{\acl^\eq A} b.
\end{equation*}
È sufficiente quindi dimostrare che 2. implichi
\begin{equation*}
c\equivalentover{\acl A}b
\quad\oldiff\quad
c\equivalentover{\acl^\eq A} b
\end{equation*}
per ogni \(b,c \in\U^{\lambda}\) per ottenere la tesi.

Ma \(c\equivalentover{\acl A}b\) se e solo se\footnote{Il verso \(c\equivalentover{\acl A}b\ \oldimplies\ \orbita (c/\acl A) = \orbita(b/\acl A)\)
è ovvio, dal momento che \(\orbita(b/\acl A) = p(\U^{x})\), dove \(p(x) = \tp(b/\acl A)\).
Viceversa WLOG si supponga per assurdo che vi sia \(\varphi(x) \in \tp(b/\acl A)\) tale che \(\varphi(x) \notin \tp(c/\acl A)\). Allora non vale \(\varphi(c)\) e quindi \(c\notin \orbita(b/\acl A)\). Assurdo poiché \(c\in \orbita(c/\acl A)\).} \(\orbita (c/\acl A) = \orbita(b/\acl A)\).

Per \(2.\) questo è vero se e solo se \(\orbita(c/\acl^{\eq}A) = \orbita(b/\acl^{\eq}A)\) e, come sopra, questo avviene se e solo se \(c\equivalentover{\acl^{\eq} A} b\).

(\(3.\Rightarrow 1.\)):
Vale sempre \(\dcl^{\eq} (\acl A) \subseteq \acl^{\eq} A\). Infatti, sia \(\mathcal{B} \in \dcl^{\eq}(\acl A)\). Allora esiste \(\psi(x;z) \in \mathcal{L}\), \(\card{z}=n\), ed esistono \(a_{1},\dots,a_{n} \in \acl A\) tali che
\begin{equation*}
\mathcal{B} = \psi(\U^{x};a_{1},\dots,a_{n}).
\end{equation*}
Inoltre esistono \(\varphi_{1},\dots,\varphi_{n} \in \mathcal{L}(A)\) tali che \(\varphi_{i}(a_{i})\) e che testimoniano \(a_{i} \in \acl A\). Quindi la formula
\begin{equation*}
\Phi(\mathcal{X}):\quad
\exists z_{1},\dots,z_{n}\
\big(\mathcal{X} = \psi(\U^{x}; z_{1},\dots,z_{n}) \land \bigwedge_{i=1}^{n} \varphi_{i}(z_{i})\big)
\end{equation*}
ha parametri unicamente in \(A\) ed è realizzata da un numero finito di \(\mathcal{X}\) (poiché ciascuna \(\varphi_{i}\) ha un numero finito di realizzazioni), e inoltre \(\Phi(\mathcal{B})\). Pertanto \(\Phi\) testimonia \(\mathcal{B} \in \acl^{\eq} A\).

Resta da dimostrare che \(3.\) implica \(\acl^{\eq} A \subseteq \dcl^{\eq}(\acl A)\).
Sia quindi \(a\in \acl^{\eq} A\).

\begin{itemize}
\item Si noti che
\(\orbita(a/\acl^{\eq}A) = \set{a}\)
poiché \(a \in \acl^{\eq}A\).
\item Inoltre,
\begin{align*}
  \orbita(a/\acl A) &= \set{b \in (\U^{\eq})^{\card{a}}\mid a\equivalentover{\acl A} b}\\
  \orbita(a/\acl^{\eq} A) &= \set{b \in (\U^{\eq})^{\card{a}}\mid a\equivalentover{\acl^{\eq} A} b}.
\end{align*}
\item L'ipotesi \(3.\) implica che
\begin{equation*}
a\equivalentover{\acl A}b
\quad\oldiff\quad
a\equivalentover{\acl^\eq A} b
\end{equation*}
per ogni \(c,b\).
\end{itemize}

Utilizzando i tre fatti di cui sopra, si ottiene che
\begin{equation*}
\orbita(a/\acl A) = \orbita(a/\acl^{\eq}A) = \set{a}
\end{equation*}
ovvero \(a\) invariante su \(\acl A\). Per la caratterizzazione \(a \in \dcl^{\eq}(\acl A)\).
\end{proof}
\end{document}
