% Created 2026-02-07 Sat 19:31
% Intended LaTeX compiler: pdflatex
\documentclass[10pt]{article}
%% CREATO CON ORG - EMACS
\newcommand{\use}[2][]{\usepackage[#1]{#2}}
% PACCHETTI FONDAMENTLAI
\use[utf8]{inputenc}
\use[T1]{fontenc}
\use{graphicx}
\use{longtable}
\use{wrapfig}
\use{rotating}
\use[normalem]{ulem}
\use{amsmath}
\use{amsthm}
\use{amssymb}

\use{eucal} % Cambia mathcal{...}

\use{capt-of}
\use[italian]{babel}
\use[babel]{csquotes}
% bib la TEX lo carica in automatico org-cite
\use{microtype}
\use{lmodern}
\use{subfig} % sottofigure
\use{multicol} % due colonne
\use{lipsum} % lorem ipsum
\use{color} % colori in latex
\use{parskip} % rimuove l'indentazione dei nuovi paragrafi %% Add parbox=false to all new tcolorbox
\use{centernot}
\use[outline]{contour}\contourlength{3pt}
\use{fancyhdr}
\use{layout}
\use[most]{tcolorbox} % Riquadri colorati
\use{ifthen} % IFTHEN
\use{geometry}

% pacchetti matematica
\use{yhmath}
\use{dsfont}
\use{mathrsfs}
\use{cancel} % semplificare
\use{polynom} %divisione tra polinomi
\use{forest} % grafi ad albero
\use{booktabs} % tabelle
\use{commath} %simboli e differenziali
\use{bm} %bold
\use[fulladjust]{marginnote} %to use marginnote for date notes
\use{arrayjobx}%array
\use[intlimits]{empheq} % Riquadri colorati attorno alle equazioni
\use{mathtools}
\use{circuitikz} % Disegnare i circuiti
\use{mathtools}
\use{stmaryrd} % [[ \llbracket ]] \rrbracket
\use{bussproofs} % dimostrazioni

%%%%%%%%%%%%%


%%%% QUIVER
\newcommand{\duepunti}{\,\mathchar\numexpr"6000+`:\relax\,}
% A TikZ style for curved arrows of a fixed height, due to AndréC.
\tikzset{curve/.style={settings={#1},to path={(\tikztostart)
    .. controls ($(\tikztostart)!\pv{pos}!(\tikztotarget)!\pv{height}!270:(\tikztotarget)$)
    and ($(\tikztostart)!1-\pv{pos}!(\tikztotarget)!\pv{height}!270:(\tikztotarget)$)
    .. (\tikztotarget)\tikztonodes}},
    settings/.code={\tikzset{quiver/.cd,#1}
        \def\pv##1{\pgfkeysvalueof{/tikz/quiver/##1}}},
    quiver/.cd,pos/.initial=0.35,height/.initial=0}

% TikZ arrowhead/tail styles.
\tikzset{tail reversed/.code={\pgfsetarrowsstart{tikzcd to}}}
\tikzset{2tail/.code={\pgfsetarrowsstart{Implies[reversed]}}}
\tikzset{2tail reversed/.code={\pgfsetarrowsstart{Implies}}}
% TikZ arrow styles.
\tikzset{no body/.style={/tikz/dash pattern=on 0 off 1mm}}
%%%%%%%%%%


%% DEFINIZIONI COMANDI MATEMATICI
\let\sin\relax %TOGLIE LA DEFINIZIONE SU "\sin"

% cambia la definizione di empty set
% ---
\let\oldemptyset\emptyset
% ---
% \let\emptyset\varnothing
% ---
% \let\emptyset\relax
% \newcommand{\emptyset}{\text{\textnormal{\O}}}
% ---

\DeclareMathOperator{\bounded}{bd}
\DeclareMathOperator{\sin}{sen}
\DeclareMathOperator{\epi}{Epi}
\DeclareMathOperator{\cl}{cl}
\DeclareMathOperator{\graph}{graph}
\DeclareMathOperator{\arcsec}{arcsec}
\DeclareMathOperator{\arccot}{arccot}
\DeclareMathOperator{\arccsc}{arccsc}
\DeclareMathOperator{\spettro}{Spettro}
\DeclareMathOperator{\nulls}{nullspace}
\DeclareMathOperator{\dom}{dom}
\DeclareMathOperator{\ar}{ar}
\DeclareMathOperator{\const}{Const}
\DeclareMathOperator{\fun}{Fun}
\DeclareMathOperator{\rel}{Rel}
\DeclareMathOperator{\altezza}{ht}
\let\det\relax %TOGLIE LA DEFINIZIONE SU "\det"
\DeclareMathOperator{\det}{det}
\DeclareMathOperator{\End}{End}
\DeclareMathOperator{\gl}{GL}
\def\Id{\mathrm{Id}}
\def\id{\mathrm{id}}
\DeclareMathOperator{\I}{\mathds{1}}
\DeclareMathOperator{\II}{II}
\DeclareMathOperator{\rank}{rank}
\DeclareMathOperator{\tr}{tr}
\DeclareMathOperator{\tc}{t.c.}
\DeclareMathOperator{\T}{T}
\DeclareMathOperator{\var}{Var}
\DeclareMathOperator{\cov}{Cov}
\DeclareMathOperator{\st}{st}
\DeclareMathOperator{\mon}{Mon}
\newcommand{\card}[1]{\left\vert #1 \right\vert}
\newcommand{\trasposta}[1]{\prescript{\text{T}}{}{#1}}
\newcommand{\1}{\mathds{1}}
\newcommand{\R}{\mathds{R}}
\newcommand{\diesis}{\#}
\newcommand{\bemolle}{\flat}
\newcommand{\nonstandard}[1]{\prescript{*}{}{#1}}
\newcommand{\starR}{\nonstandard{\R}}
\newcommand{\borel}{\mathscr{B}}
\newcommand{\lebesgue}[1]{\mathscr{L}\left(#1\right)}
\newcommand{\media}{\mathds{E}}
\newcommand{\K}{\mathds{K}}
\newcommand{\A}{\mathds{A}}
\newcommand{\Q}{\mathds{Q}}
\newcommand{\N}{\mathds{N}}
\newcommand{\C}{\mathds{C}}
\newcommand{\Z}{\mathds{Z}}
\newcommand{\qo}{\hspace{1em}\text{q.o.}\,}
\renewcommand{\tilde}[1]{\widetilde{#1}}
\renewcommand{\parallel}{\mathrel{/\mkern-5mu/}}
\newcommand{\parti}[2][]{\wp_{#1}(#2)}
\newcommand{\diff}[1]{\operatorname{d}_{#1}}
\let\oldvec\vec
\renewcommand{\vec}[1]{\overrightarrow{\vphantom{i}#1}}
\newcommand{\floor}[1]{\left\lfloor #1 \right\rfloor}
\newcommand{\cat}[1]{\mathbf{#1}}
\newcommand{\dfreccia}[1]{\xrightarrow{\ #1 \ }}
\newcommand{\sfreccia}[1]{\xleftarrow{\ #1 \ }}
\newcommand{\formalsum}[2]{{\sum_{#1}^{#2}}{\vphantom{\sum}}'}
\newcommand{\minim}[2]{\mu_{#1}\, \left(#2\right)}
\newcommand{\concat}{\null^{\frown}} % concatenazione di stringe
\newcommand{\godelcode}[1]{\langle\!\langle #1 \rangle\!\rangle}
\newcommand{\godeldec}[1]{(\!(#1)\!)}
\newcommand{\termcode}[1]{\ulcorner #1\urcorner}
\newcommand{\partialto}{\dashrightarrow}
\newcommand{\restricted}{\upharpoonright}
\newcommand{\embeds}{\precsim}
\newcommand{\surjects}{\twoheadrightarrow}
\newcommand{\equipotenti}{\asymp}
%% \newcommand{\dotplus}{\mathbin{\dot{+}}} %% A quanto pare esiste già
\newcommand{\bigdot}{\mathbin{\boldsymbol{\cdot}}}
\newcommand{\dotexp}[1]{^{.#1}}
\newcommand{\conv}{\mathbin{*}}
\newcommand{\convolution}[2]{(#1\conv #2)}
\newcommand{\nil}{\mathfrak{N}}
\newcommand{\divisore}{\mathrel{|}}
\newcommand{\simplesso}[1]{\mathrm{e}_{#1}}

\renewcommand{\iff}{\mathrel{\longleftrightarrow}} %% Notazione Logica.
\newcommand{\oldiff}{\mathrel{\Longleftrightarrow}}
\renewcommand{\implies}{\mathrel{\rightarrow}} %% Notazione Logica
\newcommand{\oldimplies}{\mathrel{\Longrightarrow}}
\renewcommand{\impliedby}{\mathrel{\leftarrow}} %% Notazione Logica
\newcommand{\oldimpliedby}{\mathrel{\Longleftarrow}}

\newcommand{\IFF}{\quad\Longleftrightarrow\quad}
\newcommand{\IMPLICA}{\quad\Longrightarrow\quad}


\renewcommand{\descriptionlabel}[1]{\hspace{\labelsep}\normalfont #1} % remove bold from description


%% Definizione di Divergenza di K-L

\DeclarePairedDelimiterX{\infdivx}[2]{(}{)}{%
  #1\;\delimsize\|\;#2%
}
\newcommand{\kldiv}{D_{KL}\infdivx}

%% Definizione di \dotminus

\makeatletter
\newcommand{\dotminus}{\mathbin{\text{\@dotminus}}}

\newcommand{\@dotminus}{%
  \ooalign{\hidewidth\raise1ex\hbox{.}\hidewidth\cr$\m@th-$\cr}%
}
\makeatother

%tramite i prossimi due comandi posso decidere come scrivere i logaritmi naturali in tutti i documenti: ho infatti eliminato qualsiasi differenza tra "ln" e "log": se si vuole qualcosa di diverso bisogna inserire manualmente il tutto
\let\ln\relax
\DeclareMathOperator{\ln}{ln}
\let\log\relax
\DeclareMathOperator{\log}{log}
%%%%%%

%% NUOVI COMANDI
\newcommand{\straniero}[1]{\textit{#1}} %parole straniere
\newcommand{\titolo}[1]{\textsc{#1}} %titoli
\newcommand{\qedd}{\tag*{$\blacksquare$}} %qed per ambienti matemastici
\renewcommand{\qedsymbol}{$\blacksquare$} %modifica colore qed
\newcommand{\ooverline}[1]{\overline{\overline{#1}}}
\newcommand{\circoletto}[1]{\left(#1\right)^{\text{o}}}
%
\newcommand{\qmatrice}[1]{\begin{pmatrix}
#1_{11} & \cdots & #1_{1n}\\
\vdots & \ddots & \vdots \\
#1_{m1} & \cdots & #1_{mn}
\end{pmatrix}}
%
\newcommand{\parentesi}[2]{%
\underset{#1}{\underbrace{#2}}%
}
%
\newcommand{\norma}[1]{% Norma
\left\lVert#1\right\rVert%
}
\newcommand{\scalare}[2]{% Scalare
\left\langle #1, #2\right\rangle
}
%%%%%

%% RESTRIZIONI
\newcommand{\referenze}[2]{
        \phantomsection{}#2\textsuperscript{\textcolor{blue}{\textbf{#1}}}
}

\let\restriction\relax

\def\restriction#1#2{\mathchoice
              {\setbox1\hbox{${\displaystyle #1}_{\scriptstyle #2}$}
              \restrictionaux{#1}{#2}}
              {\setbox1\hbox{${\textstyle #1}_{\scriptstyle #2}$}
              \restrictionaux{#1}{#2}}
              {\setbox1\hbox{${\scriptstyle #1}_{\scriptscriptstyle #2}$}
              \restrictionaux{#1}{#2}}
              {\setbox1\hbox{${\scriptscriptstyle #1}_{\scriptscriptstyle #2}$}
              \restrictionaux{#1}{#2}}}
\def\restrictionaux#1#2{{#1\,\smash{\vrule height .8\ht1 depth .85\dp1}}_{\,#2}}
%%%%%%%%%%%

%%% FORMATTAZIONE FOOTNOTEMARK

\def\footnotemarkformatting#1{[#1]}
\renewcommand{\thefootnote}{\footnotemarkformatting{\arabic{footnote}}}

%% SEZIONE GRAFICA
\use{tikz}
\usetikzlibrary{matrix, patterns, calc, decorations.pathreplacing, hobby, decorations.markings, decorations.pathmorphing, babel}
\use{tikz-3dplot}
\use{mathrsfs} %per geogebra
\use{tikz-cd}
\tikzset
{
  %surface/.style={fill=black!10, shading=ball,fill opacity=0.4},
  plane/.style={black,pattern=north east lines},
  curve/.style={black,line width=0.5mm},
  dritto/.style={decoration={markings,mark=at position 0.5 with {\arrow{Stealth}}}, postaction=decorate},
  rovescio/.style={decoration={markings,mark=at position 0.5 with {\arrow{Stealth[reversed]}}}, postaction=decorate}
}
\use{pgfplots} % stampare le funzioni
        \pgfplotsset{/pgf/number format/use comma,compat=1.15}
        %\pgfplotsset{compat=1.15} %per geogebra
        \usepgfplotslibrary{fillbetween, polar}
%%%%%%

%% CITAZIONI
\use{lineno}

\newcommand{\citazione}[1]{%
  \begin{quotation}
  \begin{linenumbers}
  \modulolinenumbers[5]
  \begingroup
  \setlength{\parindent}{0cm}
  \noindent #1
  \endgroup
  \end{linenumbers}
  \end{quotation}\setcounter{linenumber}{1}
  }
%%%%%%

%%%%%%%%%%%%%%%%%%%%%%%%%%%%%%%%%%%%%%%%%%%%
%%%%%%%%%%%%%%%%%%%%%%%%%%%%%%%%%%%%%%%%%%%%

%% AMS THM

\theoremstyle{definition}% default
\newtheorem{thm}{Teorema}[section]
\newtheorem{lem}[thm]{Lemma}
\newtheorem{prop}[thm]{Proposizione}
\newtheorem{cor}[thm]{Corollario}
\newtheorem{esempio}[thm]{Esempio}
\theoremstyle{plain}
\newtheorem{definizione}[thm]{Definizione}
\theoremstyle{remark}
\newtheorem*{oss}{Osservazione}


%%%%%%%%%%%%%%%%%%%%%%%%%%%%%%%%%%%%%%%%%%%%
%%%%%%%%%%%%%%%%%%%%%%%%%%%%%%%%%%%%%%%%%%%%

\use{hyperref}
\hypersetup{%
        pdfauthor={Davide Peccioli},
        pdfsubject={},
        allcolors=black,
        citecolor=black,
%	colorlinks=true,
        bookmarksopen=true}
\setcounter{secnumdepth}{0} % rimuove i numeri di sezione senza rimuovere le ref
\renewcommand{\href}[2]{\textcolor{blue}{#2}} % disabilita il comando href
\use{enotez} %
\setenotez{%
 mark-format = \footnotemarkformatting % Mette i numeri tra parentesi quadre%
}\let\footnote=\endnote % rende tutte le note a pié pagina come delle note a fine file 


\let\olddocument\document % modifico l'ambiende documenti per non dover stampare \printendnote
\let\oldenddocument\enddocument
\renewenvironment{document}%
{%
  \olddocument
}{%
  \printendnotes\oldenddocument
}
\renewcommand{\thethm}{\arabic{thm}}

\usepackage[hyperref]{biblatex}
\addbibresource{~/Documents/org/roam/bib/master.bib}
\author{Davide Peccioli}
\date{\today}
\title{Istituzioni di Geometria [CORSO]}
\begin{document}

\section{Istituzioni di Geometria (9 cfu)}
\label{sec:org2d38457}

\subsection{Geometria Algebrica (3 cfu)}
\label{sec:orge95b62f}

\begin{itemize}
\item \href{20241219113434-anello_dei_polinomi.org}{Anello-dei-polinomi}
\item \href{20241231112713-campo_algebricamente_chiuso.org}{Campo Algebricamente Chiuso}
\item \href{20241231114009-spazio_affine.org}{Spazio Affine}
\item \href{20241231114256-varieta_algebrica_affine.org}{Varietà Algebrica Affine}
\item \href{20241231115051-spazio_proiettivo.org}{Spazio Proiettivo}
\item \href{20241231121125-polinomi_omogenei.org}{Polinomi Omogenei}
\item \href{20241231123223-varieta_algebrica_proiettiva.org}{Varietà Algebrica Proiettiva}
\item \href{20250102103618-infinitezza_campi_alg_chiusi.org}{Infinitezza Campi alg chiusi}
\item \href{20250102141936-insiemi_finiti_di_punti_dello_spazio_proiettivo.org}{Insiemi finiti di punti dello spazio proiettivo}
\item \href{20250102143739-insiemi_finiti_di_punti_dello_spazio_affine.org}{Insiemi finiti di punti dello spazio affine come zeri di due polinomi}
\item \href{20250102115740-anello_dei_polinomi_ad_ideali_principali.org}{Anello dei polinomi ad ideali principali}
\item \href{20250102115942-anello_noetheriano.org}{Anello Noetheriano}
\item \href{20250102165143-ideali_generati_di_un_anello_noetheriano.org}{Ideali Generati di un Anello Noetheriano}
\item \href{20250102165420-teorema_della_base_di_hilbert.org}{Teorema della Base di Hilbert}
\item \href{20250102175222-campi_sono_anelli_noetheriani.org}{Campi sono anelli Noetheriani}
\item \href{20250102180050-zeri_di_un_ideale_generato_in_uno_spazio_affine.org}{Zeri di un ideale generato in uno spazio affine}
\item \href{20250102182726-ideale_di_polinomi_omogeneo.org}{Ideale di polinomi omogeneo}
\item \href{20250102183523-luogo_di_zeri_di_un_ideale_omogeneo.org}{Luogo di zeri di un ideale omogeneo}
\item \href{20250103100601-varieta_algebriche_luogo_di_zeri_di_finiti_polinomi.org}{Varietà Algebriche luogo di zeri di finiti polinomi}
\item \href{20250103101459-topologia_di_zariski_affine.org}{Topologia di Zariski affine}
\item \href{20250103180214-topologia_di_zariski_proiettiva.org}{Topologia di Zariski proiettiva}
\item \href{20250103144124-ideale_di_un_sottoinsieme.org}{Ideale di un sottoinsieme}
\item \href{20250103145915-zeri_di_un_ideale_di_un_sottoinsieme_affine.org}{Zeri di un ideale di un sottoinsieme affine}
\item \href{20250109115213-zeri_di_un_ideale_di_un_sottoinsieme_proiettivo.org}{Zeri di un ideale di un sottoinsieme proiettivo}
\item \href{20250103152646-spazio_topologico_noetheriano.org}{Spazio topologico Noetheriano}
\item \href{20250103153610-topologia_di_zariski_affine_e_noetheriana.org}{Topologia di Zariski affine è noetheriana}
\item \href{20250103155209-caratterizzazione_topologia_noetheriana.org}{Caratterizzazione topologia noetheriana}
\item \href{20250103163621-spazio_topologico_noetheriano_e_compatto.org}{Spazio topologico Noetheriano è compatto}
\item \href{20250103164917-sottospazio_irriducibile.org}{Sottospazio irriducibile}
\item \href{20250103170236-scomposizione_di_sp_top_noetheriani_in_componenti_irriducibili.org}{Scomposizione di sp top noetheriani in componenti irriducibili}
\item \href{20250103170329-caratterizzazione_di_sottospazi_affini_irriducibili_tramite_ideali.org}{Caratterizzazione di sottospazi affini irriducibili tramite ideali}
\item \href{20250104100907-spazio_affine_e_irriducibile.org}{Spazio affine è irriducibile}
\item \href{20250104110524-morfismo_tra_varieta_algebriche_affini.org}{Morfismo tra varietà algebriche affini}
\item \href{20250104112713-morfismo_tra_varieta_algebriche_affini_e_continuo.org}{Morfismo tra varietà algebriche affini è continuo}
\item \href{20250104114505-morfismo_tra_varieta_algebriche_affini_non_e_chiuso.org}{Morfismo tra varietà algebriche affini non è chiuso}
\item \href{20250104120600-morfismo_tra_varieta_algebriche_proiettive.org}{Morfismo tra varietà algebriche proiettive}
\item \href{20250104121634-curva_razionale_normale.org}{Curva Razionale Normale}
\item \href{20250104121837-mappa_di_veronese.org}{Mappa di Veronese}
\item \href{20250107112123-varieta_algebrica_quasi_proiettiva_qp.org}{Varietà Algebrica Quasi Proiettiva QP}
\item \href{20250107112412-morfismo_tra_varieta_algebriche_qp.org}{Morfismo tra varietà algebriche QP}
\item \href{20250104190620-mappa_di_segre.org}{Mappa di Segre}
\item \href{20250107123816-proiezioni_da_prodotti_di_varieta_proiettive_sono_morfisi.org}{Proiezioni da prodotti di varietà proiettive sono morfisi}
\item \href{20250107143702-copie_dello_spazio_proiettivo_dentro_la_varieta_di_segre_sono_lineari.org}{Copie dello spazio proiettivo dentro la varietà di Segre sono lineari}
\item \href{20250107170928-varieta_proiettiva_dentro_prodotti_di_spazi_proiettivi.org}{Varietà proiettiva dentro prodotti di spazi proiettivi}
\item \href{20250108102640-prodotti_di_varieta_qp.org}{Prodotti di varietà QP}
\item \href{20250108124906-grafico_di_un_morfismo_proiettivo_e_varieta.org}{Grafico di un morfismo proiettivo è varietà}
\item \href{20250108162146-proiezione_su_un_iperpiano_dentro_allo_spazio_proiettivo.org}{Proiezione su un iperpiano dentro allo spazio proiettivo}
\item \href{20250108172559-teorema_proiezione_su_un_iperpiano_di_una_varieta_e_varieta.org}{Teorema proiezione su un iperpiano di una varietà è varietà}
\item \href{20250108173056-fattori_non_costanti_comuni_tra_polinomi.org}{Fattori non costanti comuni tra polinomi}
\item \href{20250109095734-risultante_per_polinomi_in_piu_variabili.org}{Risultante per polinomi in più variabili}
\item \href{20250109120343-cono_in_uno_spazio_proiettivo.org}{Cono in uno spazio proiettivo}
\item \href{20250109141249-morfismo_da_varieta_proiettiva_a_varieta_qp_e_chiuso.org}{Morfismo da varietà proiettiva a varietà qp è chiuso}
\item \href{20250109164824-morfismo_da_varieta_proiettiva_connessa_allo_spazio_affine_unidimensionale_e_costante.org}{Morfismo da varietà proiettiva connessa allo spazio affine unidimensionale è costante}
\item \href{20250109164935-ipersuperfici_dello_spazio_proiettivo_si_intersecano_con_varieta_connesse.org}{Ipersuperficie dello spazio proiettivo interseca ogni varietà connessa diversa da un solo punto}
\item \href{20250110125225-radicale_di_un_ideale.org}{Radicale di un ideale}
\item \href{20250110125621-radicale_di_un_ideale_e_ideale.org}{Radicale di un ideale è ideale}
\item \href{20250110142357-ideale_radicale.org}{Ideale Radicale}
\item \href{20250110182255-ideale_radicale_e_anello_quoziente.org}{Ideale radicale e anello quoziente}
\item \href{20250110144451-ideale_primo_e_radicale.org}{Ideale primo è radicale}
\item \href{20250110143120-nullstellensatz.org}{Nullstellensatz}
\item \href{20250110143644-nullstellensatz_debole.org}{Nullstellensatz debole}
\item \href{20250110145609-nullstellensatz_debole_implica_nullstellensatz.org}{Nullstellensatz debole implica Nullstellensatz}
\item \href{20250110150011-corrispondenza_ideali_radicali_e_chiusi_algebrici_dello_spazio_affine.org}{Corrispondenza ideali radicali e chiusi algebrici dello spazio affine}
\item \href{20250110174110-anello_delle_coordinate.org}{Anello delle coordinate}
\item \href{20250110175552-algebra_su_un_campo.org}{Algebra su un campo}
\item \href{20250110180014-algebra_su_un_campo_finitamente_generata.org}{Algebra su un campo finitamente generata}
\item \href{20250110175521-k_algebre_fin_generate_e_ridotte_come_anelli_delle_coordinate_di_varieta_affini.org}{K-Algebre fin generate e ridotte come anelli delle coordinate di varietà affini}
\end{itemize}
\subsection{Geometria Differenziale}
\label{sec:orgb4fa0bf}
\begin{itemize}
\item \href{20250111092123-varieta_topologica.org}{Varietà Topologica}
\item \href{20250111161343-esempi_fondamentali_di_varieta_topologiche.org}{Esempi fondamentali di varietà topologiche}
\item \href{20250111161808-immagini_tramiti_omeomorfismi_di_varieta_topologiche_sono_varieta_topologiche.org}{Immagini tramiti omeomorfismi di varietà topologiche sono varietà topologiche}
\item \href{20250113095039-atlante_topologico.org}{Atlante topologico}
\item \href{20250113095300-la_sfera_n_dimensionale_e_varieta_topologica.org}{La sfera n-dimensionale è varietà topologica}
\item \href{20250113095404-spazio_proiettivo_e_varieta_topologica.org}{Spazio proiettivo è varietà topologica}
\item \href{20250113095724-spazio_delle_matrici_invertibili_e_varieta_topologica.org}{Spazio delle matrici reali invertibili è varietà topologica}
\item \href{20250113100307-gruppo_lineare_complesso_e_varieta_topologica.org}{Gruppo lineare complesso è varietà topologica}
\item \href{20250113102456-connessione_per_archi_di_varieta_topologiche.org}{Connessione per archi di varietà topologiche}
\item \href{20250113103405-cambio_di_carte_per_un_atlante_topologico.org}{Cambio di carte per un atlante topologico}
\item \href{20250113104517-atlante_astratto.org}{Atlante astratto}
\item \href{20250113105334-compatibilita_tra_atlanti_astratti_e_varieta_topologiche.org}{Compatibilità tra atlanti astratti e varietà topologiche}
\item \href{20250113103136-atlante_topologico_differenziabile.org}{Atlante topologico differenziabile}
\item \href{20250113110035-atlanti_differenziabili_compatibili.org}{Atlanti differenziabili compatibili}
\item \href{20250113115909-struttura_differenziabile.org}{Struttura Differenziabile}
\item \href{20250113120151-esistenza_e_unicita_di_una_struttura_differenziabile_per_ogni_atlante.org}{Esistenza e unicità di una struttura differenziabile per ogni atlante}
\item \href{20250113120835-esempi_fondamentali_di_varieta_differenziabili.org}{Esempi fondamentali di varietà differenziabili}
\item \href{20250113121849-varieta_topologica_omeomorfa_a_varieta_differenziabile_ne_eredita_la_struttura.org}{Varietà Topologica omeomorfa a varietà differenziabile ne eredità la struttura}
\item \href{20250113122627-varieta_topologiche_in_dimensione_minore_uguale_a_tre_hanno_struttura_differenziabile.org}{Varietà topologiche in dimensione minore uguale a tre hanno struttura differenziabile}
\item \href{20250113122808-strutture_differenziabili_sulla_sfera_7_dimensionale.org}{Strutture differenziabili sulla sfera 7 dimensionale}
\item \href{20250113124917-prodotto_di_varieta_differenziabili_e_varieta_differenziabile.org}{Prodotto di varietà differenziabili è varietà differenziabile}
\item \href{20250113125032-toro_ha_struttura_di_varieta_differenziabile.org}{Toro ha struttura di varietà differenziabile}
\item \href{20250113125429-teorema_dell_inversa_locale.org}{Teorema dell'inversa locale}
\item \href{20250113125903-teorema_della_funzione_implicita.org}{Teorema della funzione implicita}
\item \href{20250113141843-differenziale_del_determinante_di_una_matrice_reale.org}{Differenziale del determinante di una matrice reale}
\item \href{20250113142135-gruppo_lineare_speciale_e_varieta_differenziabile.org}{Gruppo lineare speciale è varietà differenziabile}
\item \href{20250113144147-gruppo_ortogonale_e_una_varieta_differenziabile.org}{Gruppo ortogonale è una varietà differenziabile}
\item \href{20250113144514-gruppo_lineare_e_sconnesso_e_compatto.org}{Gruppo ortogonale è sconnesso e compatto}
\item \href{20250113144722-funzioni_cinfinito_tra_varieta_differenziabili.org}{Funzioni Cinfinito tra varietà differenziabili}
\item \href{20250113151718-restrizione_di_funzioni_reali_cinfinito_su_varieta_differenziabili_definite_in_forma_implicita_e_ancora_cinfinito.org}{Restrizione di funzioni reali Cinfinito su varietà differenziabili definite in forma implicita è ancora Cinfinito}
\item \href{20250113152036-lemma_di_incollamento_tra_funzioni_cinfinito_tra_varieta_differenziabili.org}{Lemma di incollamento tra funzioni Cinfinito tra varietà differenziabili}
\item \href{20250113154629-funzione_sulle_carte_indotta_da_funzione_cinfinito_tra_due_varieta_e_cinfinito.org}{Funzione sulle carte indotta da funzione Cinfinito tra due varietà è Cinfinito}
\item \href{20250113172736-composizione_di_funzioni_cinfinito_e_cinfinito.org}{Composizione di funzioni Cinfinito è Cinfinito}
\item \href{20250113172924-diffeomorfismo_tra_varieta_differenziabili.org}{Diffeomorfismo tra varietà differenziabili}
\item \href{20250113173218-diffeomorfismo_tra_le_strutture_differenziali_sui_numeri_reali.org}{Diffeomorfismo tra le strutture differenziali sui numeri reali}
\item \href{20250113173255-gruppo_dei_diffeomorfismi_di_una_varieta_differenziabile.org}{Gruppo dei diffeomorfismi di una varietà differenziabile}
\item \href{20250113175231-rivestimento.org}{Rivestimento}
\item \href{20250114094653-rivestimenti_e_gruppo_fondamentale.org}{Rivestimenti e gruppo fondamentale}
\item \href{20250114095243-teorema_del_rivestimento_universale.org}{Teorema del rivestimento universale}
\item \href{20250114095744-rivestimento_n_1.org}{Rivestimento n:1}
\item \href{20250114095901-teorema_sui_rivestimenti_di_varieta_differenziabili.org}{Teorema sui rivestimenti di varietà differenziabili}
\item \href{20250114100254-germi_di_funzioni.org}{Germi di funzioni}
\item \href{20250114101157-spazio_dei_germi_di_funzioni_e_algebra_reale.org}{Spazio dei germi di funzioni è algebra reale}
\item \href{20250114101437-derivazione_su_una_varieta_differenziabile.org}{Derivazione su una varietà differenziabile}
\item \href{20250114101917-derivazioni_canoniche_su_una_varieta_differenziabile.org}{Derivazioni Canoniche su una varietà differenziabile}
\item \href{20250114102823-spazio_tangente_ad_un_punto_di_una_varieta_differenziabile.org}{Spazio tangente ad un punto di una varietà differenziabile}
\item \href{20250114103339-teorema_sulla_base_dello_spazio_tangente_ad_un_punto_di_una_varieta_differenziabile.org}{Teorema sulla base dello spazio tangente ad un punto di una varietà differenziabile}
\item \href{20250114104914-cambio_di_base_sullo_spazio_tangente_ad_un_punto_di_una_varieta_differenziabile.org}{Cambio di base sullo spazio tangente ad un punto di una varietà differenziabile}
\item \href{20250114110247-curve_su_varieta_e_loro_derivata.org}{Curve su varietà e loro derivata}
\item \href{20250114102823-spazio_tangente_ad_un_punto_di_una_varieta_differenziabile.org}{Vettori dello spazio tangente ad una varietà in un punto come derivate di curve}
\item \href{20250114111331-differenziale_di_una_funzione_tra_varieta_differenziabili.org}{Differenziale di una funzione tra varietà differenziabili}
\item \href{20250114112126-differenziale_di_composizione_di_funzioni_tra_varieta.org}{Differenziale di composizione di funzioni tra varietà}
\item \href{20250114112633-differenziale_di_diffeomorfismo_tra_varieta_e_isomorfismo.org}{Differenziale di diffeomorfismo tra varietà è isomorfismo}
\item \href{20250114112834-differenziale_in_coordinate_locali_su_una_varieta_differenziabile.org}{Differenziale in coordinate locali su una varietà differenziabile}
\item \href{20250114124407-cambiamento_del_differenziale_in_coordinate_locali_tramite_un_cambio_di_coordinate_su_varieta_differenziali.org}{Cambiamento del differenziale in coordinate locali tramite un cambio di coordinate su varietà differenziabii}
\item \href{20250114124504-immersione_di_varieta_differenziabili.org}{Immersione di varietà differenziabili}
\item \href{20250115105535-sommersione_di_varieta_differenziabili.org}{Sommersione di varietà differenziabili}
\item \href{20250114124533-embedding_di_varieta_differenziabili.org}{Embedding di varietà differenziabili}
\item \href{20250115102238-immersione_iniettiva_chiusa_e_embedding.org}{Immersione iniettiva chiusa è embedding}
\item \href{20250114124541-sottovarieta_differenziabile.org}{Sottovarietà differenziabile}
\item \href{20250114130106-esempio_di_sottovarieta_tramite_il_teorema_della_funzione_implicita.org}{Esempio di sottovarietà tramite il teorema della funzione implicita}
\item \href{20250114130223-teorema_delle_slice_delle_sottovarieta.org}{Teorema delle slice delle sottovarietà}
\item \href{20250114131231-teorema_del_rango_per_funzioni_reali.org}{Teorema del rango per funzioni reali}
\item \href{20250114132151-teorema_della_funzione_implicita_tra_varieta_differenziabili.org}{Teorema della funzione implicita tra varietà differenziabili}
\item \href{20250114151933-immersione_da_compatti_e_embedding.org}{Immersione da compatti è embedding}
\item \href{20250114151913-immagine_dello_spazio_tangente_ad_un_punto_di_una_sottovarieta_tramite_il_differenziale_dell_immersione.org}{Immagine dello spazio tangente ad un punto di una sottovarietà tramite il differenziale dell'immersione}
\item \href{20250115095737-restrizione_di_funzioni_a_sottovarieta_e_loro_differenziale.org}{Restrizione di funzioni a sottovarietà e loro differenziale}
\item \href{20250115100507-funzione_propria.org}{Funzione Propria}
\item \href{20250115100745-compattezza_e_compattezza_per_successioni.org}{Compattezza e Compattezza per successioni}
\item \href{20250115100954-funzioni_cinfintio_tra_varieta_e_proprie_sono_chiuse.org}{Funzioni Cinfintio tra varietà e proprie sono chiuse}
\item \href{20250115102602-teoremi_di_whitney.org}{Teoremi di Whitney}
\item \href{20250115103245-fibrato_tangente.org}{Fibrato tangente}
\item \href{20250115104113-atlante_differenziabile_induce_una_topologia.org}{Atlante differenziabile induce una topologia}
\item \href{20250115110259-campo_vettoriale_su_una_varieta_differenziabile.org}{Campo vettoriale su una varietà differenziabile}
\item \href{20250115113256-componenti_locali_di_un_campo_vettoriale.org}{Componenti locali di un campo vettoriale}
\item \href{20250115113948-componenti_locali_di_un_campo_vettoriale_sono_cinfinito.org}{Componenti locali di un campo vettoriale sono Cinfinito}
\item \href{20250115115535-azione_di_un_campo_vettoriale_su_una_funzione.org}{Azione di un campo vettoriale su una funzione}
\item \href{20250113144722-funzioni_cinfinito_tra_varieta_differenziabili.org}{Funzioni da una varietà ai reali Cinfinito}
\item \href{20250115120103-azione_di_un_campo_vettoriale_su_una_funzione_e_cinfinito.org}{Azione di un campo vettoriale su una funzione è Cinfinito}
\item \href{20250115121514-insieme_dei_derivatori_du_una_varieta.org}{Insieme dei derivatori du una varietà}
\item \href{20250115122306-insieme_dei_campi_vettoriali_come_spazio_vettoriale_reale_e_come_modulo.org}{Insieme dei campi vettoriali come spazio vettoriale reale e come modulo}
\item \href{20250115124556-push_forward_di_campi_vettoriali.org}{Push-Forward di campi vettoriali}
\item \href{20250115125404-campi_vettoriali_f_riferiti.org}{Campi vettoriali F-riferiti}
\item \href{20250115144427-differenziale_dell_inclusione_del_fibrato_tangente_ad_una_sottovarieta.org}{Differenziale dell'inclusione del fibrato tangente ad una sottovarietà}
\item \href{20250115144839-restrizione_di_un_campo_vettoriale_ad_una_sottovarieta.org}{Restrizione di un campo vettoriale ad una sottovarietà}
\item \href{20250115145035-varieta_differenziabile_pettinabile.org}{Varietà Differenziabile Pettinabile}
\item \href{20250115145200-pettinabilita_delle_sfere.org}{Pettinabilità delle sfere}
\item \href{20250115150428-varieta_differenziabile_parallelizzabile.org}{Varietà Differenziabile Parallelizzabile}
\item \href{20250115150659-parallelizzabilita_delle_sfere.org}{Parallelizzabilità delle sfere}
\item \href{20250115151027-bracket_di_campi_vettoriali.org}{Bracket di campi vettoriali}
\item \href{20250115151900-scrittura_locale_del_bracket_di_campi_vettoriali.org}{Scrittura locale del bracket di campi vettoriali}
\item \href{20250115162927-bracket_di_campi_vettoriali_f_riferiti.org}{Bracket di campi vettoriali F-riferiti}
\item \href{20250115163338-bracket_di_campi_vettoriali_ristretti_a_sottovarieta.org}{Bracket di campi vettoriali ristretti a sottovarietà}
\end{itemize}
\end{document}
