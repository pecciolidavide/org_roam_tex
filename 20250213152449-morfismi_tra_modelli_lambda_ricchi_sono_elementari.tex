% Created 2026-02-07 Sat 19:34
% Intended LaTeX compiler: pdflatex
\documentclass[10pt]{article}
%% CREATO CON ORG - EMACS
\newcommand{\use}[2][]{\usepackage[#1]{#2}}
% PACCHETTI FONDAMENTLAI
\use[utf8]{inputenc}
\use[T1]{fontenc}
\use{graphicx}
\use{longtable}
\use{wrapfig}
\use{rotating}
\use[normalem]{ulem}
\use{amsmath}
\use{amsthm}
\use{amssymb}

\use{eucal} % Cambia mathcal{...}

\use{capt-of}
\use[italian]{babel}
\use[babel]{csquotes}
% bib la TEX lo carica in automatico org-cite
\use{microtype}
\use{lmodern}
\use{subfig} % sottofigure
\use{multicol} % due colonne
\use{lipsum} % lorem ipsum
\use{color} % colori in latex
\use{parskip} % rimuove l'indentazione dei nuovi paragrafi %% Add parbox=false to all new tcolorbox
\use{centernot}
\use[outline]{contour}\contourlength{3pt}
\use{fancyhdr}
\use{layout}
\use[most]{tcolorbox} % Riquadri colorati
\use{ifthen} % IFTHEN
\use{geometry}

% pacchetti matematica
\use{yhmath}
\use{dsfont}
\use{mathrsfs}
\use{cancel} % semplificare
\use{polynom} %divisione tra polinomi
\use{forest} % grafi ad albero
\use{booktabs} % tabelle
\use{commath} %simboli e differenziali
\use{bm} %bold
\use[fulladjust]{marginnote} %to use marginnote for date notes
\use{arrayjobx}%array
\use[intlimits]{empheq} % Riquadri colorati attorno alle equazioni
\use{mathtools}
\use{circuitikz} % Disegnare i circuiti
\use{mathtools}
\use{stmaryrd} % [[ \llbracket ]] \rrbracket
\use{bussproofs} % dimostrazioni

%%%%%%%%%%%%%


%%%% QUIVER
\newcommand{\duepunti}{\,\mathchar\numexpr"6000+`:\relax\,}
% A TikZ style for curved arrows of a fixed height, due to AndréC.
\tikzset{curve/.style={settings={#1},to path={(\tikztostart)
    .. controls ($(\tikztostart)!\pv{pos}!(\tikztotarget)!\pv{height}!270:(\tikztotarget)$)
    and ($(\tikztostart)!1-\pv{pos}!(\tikztotarget)!\pv{height}!270:(\tikztotarget)$)
    .. (\tikztotarget)\tikztonodes}},
    settings/.code={\tikzset{quiver/.cd,#1}
        \def\pv##1{\pgfkeysvalueof{/tikz/quiver/##1}}},
    quiver/.cd,pos/.initial=0.35,height/.initial=0}

% TikZ arrowhead/tail styles.
\tikzset{tail reversed/.code={\pgfsetarrowsstart{tikzcd to}}}
\tikzset{2tail/.code={\pgfsetarrowsstart{Implies[reversed]}}}
\tikzset{2tail reversed/.code={\pgfsetarrowsstart{Implies}}}
% TikZ arrow styles.
\tikzset{no body/.style={/tikz/dash pattern=on 0 off 1mm}}
%%%%%%%%%%


%% DEFINIZIONI COMANDI MATEMATICI
\let\sin\relax %TOGLIE LA DEFINIZIONE SU "\sin"

% cambia la definizione di empty set
% ---
\let\oldemptyset\emptyset
% ---
% \let\emptyset\varnothing
% ---
% \let\emptyset\relax
% \newcommand{\emptyset}{\text{\textnormal{\O}}}
% ---

\DeclareMathOperator{\bounded}{bd}
\DeclareMathOperator{\sin}{sen}
\DeclareMathOperator{\epi}{Epi}
\DeclareMathOperator{\cl}{cl}
\DeclareMathOperator{\graph}{graph}
\DeclareMathOperator{\arcsec}{arcsec}
\DeclareMathOperator{\arccot}{arccot}
\DeclareMathOperator{\arccsc}{arccsc}
\DeclareMathOperator{\spettro}{Spettro}
\DeclareMathOperator{\nulls}{nullspace}
\DeclareMathOperator{\dom}{dom}
\DeclareMathOperator{\ar}{ar}
\DeclareMathOperator{\const}{Const}
\DeclareMathOperator{\fun}{Fun}
\DeclareMathOperator{\rel}{Rel}
\DeclareMathOperator{\altezza}{ht}
\let\det\relax %TOGLIE LA DEFINIZIONE SU "\det"
\DeclareMathOperator{\det}{det}
\DeclareMathOperator{\End}{End}
\DeclareMathOperator{\gl}{GL}
\def\Id{\mathrm{Id}}
\def\id{\mathrm{id}}
\DeclareMathOperator{\I}{\mathds{1}}
\DeclareMathOperator{\II}{II}
\DeclareMathOperator{\rank}{rank}
\DeclareMathOperator{\tr}{tr}
\DeclareMathOperator{\tc}{t.c.}
\DeclareMathOperator{\T}{T}
\DeclareMathOperator{\var}{Var}
\DeclareMathOperator{\cov}{Cov}
\DeclareMathOperator{\st}{st}
\DeclareMathOperator{\mon}{Mon}
\newcommand{\card}[1]{\left\vert #1 \right\vert}
\newcommand{\trasposta}[1]{\prescript{\text{T}}{}{#1}}
\newcommand{\1}{\mathds{1}}
\newcommand{\R}{\mathds{R}}
\newcommand{\diesis}{\#}
\newcommand{\bemolle}{\flat}
\newcommand{\nonstandard}[1]{\prescript{*}{}{#1}}
\newcommand{\starR}{\nonstandard{\R}}
\newcommand{\borel}{\mathscr{B}}
\newcommand{\lebesgue}[1]{\mathscr{L}\left(#1\right)}
\newcommand{\media}{\mathds{E}}
\newcommand{\K}{\mathds{K}}
\newcommand{\A}{\mathds{A}}
\newcommand{\Q}{\mathds{Q}}
\newcommand{\N}{\mathds{N}}
\newcommand{\C}{\mathds{C}}
\newcommand{\Z}{\mathds{Z}}
\newcommand{\qo}{\hspace{1em}\text{q.o.}\,}
\renewcommand{\tilde}[1]{\widetilde{#1}}
\renewcommand{\parallel}{\mathrel{/\mkern-5mu/}}
\newcommand{\parti}[2][]{\wp_{#1}(#2)}
\newcommand{\diff}[1]{\operatorname{d}_{#1}}
\let\oldvec\vec
\renewcommand{\vec}[1]{\overrightarrow{\vphantom{i}#1}}
\newcommand{\floor}[1]{\left\lfloor #1 \right\rfloor}
\newcommand{\cat}[1]{\mathbf{#1}}
\newcommand{\dfreccia}[1]{\xrightarrow{\ #1 \ }}
\newcommand{\sfreccia}[1]{\xleftarrow{\ #1 \ }}
\newcommand{\formalsum}[2]{{\sum_{#1}^{#2}}{\vphantom{\sum}}'}
\newcommand{\minim}[2]{\mu_{#1}\, \left(#2\right)}
\newcommand{\concat}{\null^{\frown}} % concatenazione di stringe
\newcommand{\godelcode}[1]{\langle\!\langle #1 \rangle\!\rangle}
\newcommand{\godeldec}[1]{(\!(#1)\!)}
\newcommand{\termcode}[1]{\ulcorner #1\urcorner}
\newcommand{\partialto}{\dashrightarrow}
\newcommand{\restricted}{\upharpoonright}
\newcommand{\embeds}{\precsim}
\newcommand{\surjects}{\twoheadrightarrow}
\newcommand{\equipotenti}{\asymp}
%% \newcommand{\dotplus}{\mathbin{\dot{+}}} %% A quanto pare esiste già
\newcommand{\bigdot}{\mathbin{\boldsymbol{\cdot}}}
\newcommand{\dotexp}[1]{^{.#1}}
\newcommand{\conv}{\mathbin{*}}
\newcommand{\convolution}[2]{(#1\conv #2)}
\newcommand{\nil}{\mathfrak{N}}
\newcommand{\divisore}{\mathrel{|}}
\newcommand{\simplesso}[1]{\mathrm{e}_{#1}}

\renewcommand{\iff}{\mathrel{\longleftrightarrow}} %% Notazione Logica.
\newcommand{\oldiff}{\mathrel{\Longleftrightarrow}}
\renewcommand{\implies}{\mathrel{\rightarrow}} %% Notazione Logica
\newcommand{\oldimplies}{\mathrel{\Longrightarrow}}
\renewcommand{\impliedby}{\mathrel{\leftarrow}} %% Notazione Logica
\newcommand{\oldimpliedby}{\mathrel{\Longleftarrow}}

\newcommand{\IFF}{\quad\Longleftrightarrow\quad}
\newcommand{\IMPLICA}{\quad\Longrightarrow\quad}


\renewcommand{\descriptionlabel}[1]{\hspace{\labelsep}\normalfont #1} % remove bold from description


%% Definizione di Divergenza di K-L

\DeclarePairedDelimiterX{\infdivx}[2]{(}{)}{%
  #1\;\delimsize\|\;#2%
}
\newcommand{\kldiv}{D_{KL}\infdivx}

%% Definizione di \dotminus

\makeatletter
\newcommand{\dotminus}{\mathbin{\text{\@dotminus}}}

\newcommand{\@dotminus}{%
  \ooalign{\hidewidth\raise1ex\hbox{.}\hidewidth\cr$\m@th-$\cr}%
}
\makeatother

%tramite i prossimi due comandi posso decidere come scrivere i logaritmi naturali in tutti i documenti: ho infatti eliminato qualsiasi differenza tra "ln" e "log": se si vuole qualcosa di diverso bisogna inserire manualmente il tutto
\let\ln\relax
\DeclareMathOperator{\ln}{ln}
\let\log\relax
\DeclareMathOperator{\log}{log}
%%%%%%

%% NUOVI COMANDI
\newcommand{\straniero}[1]{\textit{#1}} %parole straniere
\newcommand{\titolo}[1]{\textsc{#1}} %titoli
\newcommand{\qedd}{\tag*{$\blacksquare$}} %qed per ambienti matemastici
\renewcommand{\qedsymbol}{$\blacksquare$} %modifica colore qed
\newcommand{\ooverline}[1]{\overline{\overline{#1}}}
\newcommand{\circoletto}[1]{\left(#1\right)^{\text{o}}}
%
\newcommand{\qmatrice}[1]{\begin{pmatrix}
#1_{11} & \cdots & #1_{1n}\\
\vdots & \ddots & \vdots \\
#1_{m1} & \cdots & #1_{mn}
\end{pmatrix}}
%
\newcommand{\parentesi}[2]{%
\underset{#1}{\underbrace{#2}}%
}
%
\newcommand{\norma}[1]{% Norma
\left\lVert#1\right\rVert%
}
\newcommand{\scalare}[2]{% Scalare
\left\langle #1, #2\right\rangle
}
%%%%%

%% RESTRIZIONI
\newcommand{\referenze}[2]{
        \phantomsection{}#2\textsuperscript{\textcolor{blue}{\textbf{#1}}}
}

\let\restriction\relax

\def\restriction#1#2{\mathchoice
              {\setbox1\hbox{${\displaystyle #1}_{\scriptstyle #2}$}
              \restrictionaux{#1}{#2}}
              {\setbox1\hbox{${\textstyle #1}_{\scriptstyle #2}$}
              \restrictionaux{#1}{#2}}
              {\setbox1\hbox{${\scriptstyle #1}_{\scriptscriptstyle #2}$}
              \restrictionaux{#1}{#2}}
              {\setbox1\hbox{${\scriptscriptstyle #1}_{\scriptscriptstyle #2}$}
              \restrictionaux{#1}{#2}}}
\def\restrictionaux#1#2{{#1\,\smash{\vrule height .8\ht1 depth .85\dp1}}_{\,#2}}
%%%%%%%%%%%

%%% FORMATTAZIONE FOOTNOTEMARK

\def\footnotemarkformatting#1{[#1]}
\renewcommand{\thefootnote}{\footnotemarkformatting{\arabic{footnote}}}

%% SEZIONE GRAFICA
\use{tikz}
\usetikzlibrary{matrix, patterns, calc, decorations.pathreplacing, hobby, decorations.markings, decorations.pathmorphing, babel}
\use{tikz-3dplot}
\use{mathrsfs} %per geogebra
\use{tikz-cd}
\tikzset
{
  %surface/.style={fill=black!10, shading=ball,fill opacity=0.4},
  plane/.style={black,pattern=north east lines},
  curve/.style={black,line width=0.5mm},
  dritto/.style={decoration={markings,mark=at position 0.5 with {\arrow{Stealth}}}, postaction=decorate},
  rovescio/.style={decoration={markings,mark=at position 0.5 with {\arrow{Stealth[reversed]}}}, postaction=decorate}
}
\use{pgfplots} % stampare le funzioni
        \pgfplotsset{/pgf/number format/use comma,compat=1.15}
        %\pgfplotsset{compat=1.15} %per geogebra
        \usepgfplotslibrary{fillbetween, polar}
%%%%%%

%% CITAZIONI
\use{lineno}

\newcommand{\citazione}[1]{%
  \begin{quotation}
  \begin{linenumbers}
  \modulolinenumbers[5]
  \begingroup
  \setlength{\parindent}{0cm}
  \noindent #1
  \endgroup
  \end{linenumbers}
  \end{quotation}\setcounter{linenumber}{1}
  }
%%%%%%

%%%%%%%%%%%%%%%%%%%%%%%%%%%%%%%%%%%%%%%%%%%%
%%%%%%%%%%%%%%%%%%%%%%%%%%%%%%%%%%%%%%%%%%%%

%% AMS THM

\theoremstyle{definition}% default
\newtheorem{thm}{Teorema}[section]
\newtheorem{lem}[thm]{Lemma}
\newtheorem{prop}[thm]{Proposizione}
\newtheorem{cor}[thm]{Corollario}
\newtheorem{esempio}[thm]{Esempio}
\theoremstyle{plain}
\newtheorem{definizione}[thm]{Definizione}
\theoremstyle{remark}
\newtheorem*{oss}{Osservazione}


%%%%%%%%%%%%%%%%%%%%%%%%%%%%%%%%%%%%%%%%%%%%
%%%%%%%%%%%%%%%%%%%%%%%%%%%%%%%%%%%%%%%%%%%%

\use{hyperref}
\hypersetup{%
        pdfauthor={Davide Peccioli},
        pdfsubject={},
        allcolors=black,
        citecolor=black,
%	colorlinks=true,
        bookmarksopen=true}
\setcounter{secnumdepth}{0} % rimuove i numeri di sezione senza rimuovere le ref
\renewcommand{\href}[2]{\textcolor{blue}{#2}} % disabilita il comando href
\use{enotez} %
\setenotez{%
 mark-format = \footnotemarkformatting % Mette i numeri tra parentesi quadre%
}\let\footnote=\endnote % rende tutte le note a pié pagina come delle note a fine file 


\let\olddocument\document % modifico l'ambiende documenti per non dover stampare \printendnote
\let\oldenddocument\enddocument
\renewenvironment{document}%
{%
  \olddocument
}{%
  \printendnotes\oldenddocument
}
\renewcommand{\thethm}{\arabic{thm}}

\usepackage[hyperref]{biblatex}
\addbibresource{~/Documents/org/roam/bib/master.bib}
\author{Davide Peccioli}
\date{\today}
\title{Morfismi tra modelli lambda-ricchi sono elementari}
\begin{document}

Si utilizza la \href{20250612143636-notazione_teoria_dei_modelli.org}{Notazione TEORIA DEI MODELLI}.

Sia \(\mathcal{M} = \mathcal{M}_{\text{ob}}\cup \mathcal{M}_{\text{hom}}\) una \href{20241126100904-categoria.org}{categoria} \href{20250213142026-categorie_di_modelli_e_morfismi_parziali.org}{di modelli e morfismi parziali} di \href{20250130162057-linguaggio_del_prim_ordine.org}{linguaggio} \(\mathcal{L}\) tale che
\begin{enumerate}
\item per ogni \(M \in \mathcal{M}_{\text{ob}}\) e per ogni \(A \subseteq M\), \(\id_{A} : M\partialto M\) è \(\id_{A} \in \mathcal{M}_{\text{hom}}\);
\item per ogni \(M,N \in \mathcal{M}_{\text{ob}}\), e \(k:M\partialto N\) \href{20250213105339-funzione_parziale.org}{funzione parziale}, se per ogni \(k' \subseteq k\) \href{20250205120448-classe_finita_e_infinita_mk.org}{finito} si ha \(k' \in \mathcal{M}_{\text{hom}}\), allora \(k \in \mathcal{M}_{\text{hom}}\);
\item per ogni \(k \in \mathcal{M}_{\text{hom}}\), \(k\) è \href{20250111142446-funzione_inversa.org}{invertibile} e la sua \href{20250111142446-funzione_inversa.org}{inversa} \(k^{-1} \in \mathcal{M}_{\text{hom}}\);
\item gli elementi di \(\mathcal{M}_{\text{hom}}\) \href{20250214120959-mappe_tra_strutture_del_prim_ordine.org}{preservano la verità} delle \href{20250131103317-formula_del_prim_ordine.org}{formule atomiche};
\item se \(M \in \mathcal{M}_{\text{ob}}\) e \(N\) è una \href{20250131103035-struttura_del_prim_ordine.org}{struttura} \href{20250131123208-teorie_elementarmente_equivalente.org}{elementarmente equivalente} a \(M\), \(M\equiv N\), allora \(N \in \mathcal{M}_{\text{ob}}\).
\item per ogni \(M, N \in \mathcal{M}_{\text{ob}}\), se \(h:M\partialto N\) è una \href{20250214120959-mappe_tra_strutture_del_prim_ordine.org}{mappa elementare}, allora \(h \in \mathcal{M}_{\text{hom}}\).
\end{enumerate}

Sia \(\lambda\) un \href{20250203161341-cardinali.org}{cardinale} tale che \(\card{\mathcal{L}}\le\lambda\)
\section{Teorema}
\label{sec:org37532db}
Siano \(M,N \in \mathcal{M}_{\text{ob}}\) dei modelli \(\lambda\)-ricchi. Allora, per ogni \(k \in \mathcal{M}_{\text{hom}}(M,N)\), \(k\) è un \href{20250214120959-mappe_tra_strutture_del_prim_ordine.org}{morfismo elementare}.
\subsection{Dimostrazione}
\label{sec:orgdb60714}
Sia \(k \in \mathcal{M}_{\text{hom}}(M,N)\).
\begin{itemize}
\item Se tutte le restrizioni finite di \(k\) sono elementari, allora \(k\) è elementare: si supponga ad esempio che \(M\vDash \varphi[a_{1},\dots,a_{n}]\); poiché \(k\upharpoonright\set{a_{1},\dots,a_{n}}\) è elementare, allora
\begin{equation*}
  N\vDash \varphi\left[k\upharpoonright\set{a_{1},\dots,a_{n}}(a_{1}),\dots,k\upharpoonright\set{a_{1},\dots,a_{n}}(a_{n})\right]
\end{equation*}
ovvero \(N\vDash \varphi[k(a_{1}),\dots,k(a_{n})]\).

Pertanto, WLOG, si dimostra l'enunciato per \(k\) finita.
\item Si costruiscono due \href{20250212102253-sottostruttura_elementare.org}{sottostrutture elementari} (che per 5. sono in \(\mathcal{M}_{\text{ob}}\)): \(\operatorname{dom}k \subseteq M'\preceq M\), \(N'\preceq N\) ed una mappa \(h \in \mathcal{M}_{\text{hom}}(M,N)\) tale che \(k \subseteq h\) e \(h:M'\to N'\) sia biiettiva.

Per le ipotesi 3. e 4. \(h\) è un \href{20250214120959-mappe_tra_strutture_del_prim_ordine.org}{isomorfismo} (e quindi in particolare è una \href{20250214120959-mappe_tra_strutture_del_prim_ordine.org}{mappa elementare}) e quindi, per l'ipotesi 6. \(h \in \mathcal{M}_{\text{hom}}\).
\item Siano ora \(\varphi\) una \(\mathcal{L}\)-formula, e siano \(a_{1},\dots,a_{n} \in \operatorname{dom}(k)\) tali che \(M\vDash \varphi[a_{1},\dots,a_{n}]\). Allora \(a_{1},\dots,a_{n} \in M'\) e pertanto (siccome \(M'\preceq M\))
\begin{equation*}
  M'\vDash \varphi[a_{1},\dots,a_{n}]
\end{equation*}
e poiché \(h\) è un isomorfismo:
\begin{equation*}
  N'\vDash \varphi[ha_{1},\dots,ha_{n}]
\end{equation*}
ma \(N'\preceq N\) e \(h\upharpoonright \operatorname{dom}k = k\) e pertanto
\begin{equation*}
  N\vDash \varphi[ha_{1},\dots,ha_{n}]\quad\leadsto\quad N\vDash \varphi[ka_{1},\dots,ka_{n}].
\end{equation*}
\end{itemize}

Per ricorsione si costruisce una famiglia \(\langle h_{i}: i<\lambda\rangle\) tale che \(\card{h_{i}}<\lambda\) ; \href{20250205181254-order_type_del_prodotto_cartesiano_di_un_cardinale_e_il_cardinale_stesso.org}{si fissi \(\pi:\lambda^{2}\to \lambda\)} una \href{20250104111707-funzione_biunivoca.org}{biiezione} tale che \(j,k\le\pi(j,k)\):
\begin{itemize}
\item si pone \(h_{0}\coloneqq k \in \mathcal{M}_{\text{hom}}\);
\item ai passi \href{20250203161132-ordinale_limite.org}{limite} si considera l'unione; anche questa ha cardinalità \(<\lambda\), e inoltre è in \(\mathcal{M}_{\text{hom}}\) per la proprietà 2.;
\item per i passi \href{20250203161132-ordinale_limite.org}{successori} \(i+1\), si suppongia sia costruita \(h_{j}\) per ogni \(j<i+1\).

\begin{itemize}
\item Siccome \(\operatorname{dom}(h_{j})\) ha cardinalità \(<\lambda\) allora anche \(\card{\mathcal{L}(\operatorname{dom}h_{j})}<\lambda\): sia dunque
\begin{equation*}
	\langle \varphi_{j,k}(x)\mid k<\lambda\rangle
\end{equation*}
una \href{20250203133527-insiemi_ben_ordinati_sono_isomorfi_ad_un_ordinale_unico.org}{enumerazione} delle \href{20250131103317-formula_del_prim_ordine.org}{formule} \href{20250212144403-formula_consistente.org}{consistenti} di \(\mathcal{L}(\operatorname{dom}h_{j})\) in \(M\).

Siano dunque \(j,k<i\) tali che \(\pi(j,k)=i\), e sia \(b \in M\) testimone di \(\varphi_{j,k}\) in \(M\) (ovvero \(M\vDash \varphi_{j,k}[b]\)). Poiché \(\card{h_{i}}<\lambda\), siccome \(N\) è \(\lambda\)-ricco, esiste \(c \in N\) tale che
\begin{equation*}
h_{i}\cup\set{(b,c)} \in \mathcal{M}_{\text{hom}}.
\end{equation*}

Si pone dunque \(h_{i+1/2} \coloneqq h_{i}\cup\set{(b,c)}\). Dunque \(\card{h_{i+1/2}}<\lambda\).

\item Siccome \(\operatorname{rng}(h_{j})\) ha cardinalità \(<\lambda\) allora anche \(\card{\mathcal{L}(\operatorname{rng}h_{j})}<\lambda\): sia dunque
\begin{equation*}
	\langle \psi_{j,k}(x)\mid k<\lambda\rangle
\end{equation*}
una \href{20250203133527-insiemi_ben_ordinati_sono_isomorfi_ad_un_ordinale_unico.org}{enumerazione} delle \href{20250131103317-formula_del_prim_ordine.org}{formule} \href{20250212144403-formula_consistente.org}{consistenti} di \(\mathcal{L}(\operatorname{rng}h_{j})\) in \(N\).

Siano dunque \(j,k < i\) tali che \(\pi(j,k) = i\), e sia \(b \in N\) testimone di \(\psi_{j,k}\) in \(N\) (ovvero \(N\vDash\psi_{j,k}[b]\)). Poiché \(\card{(h_{i+1/2})^{-1}}<\lambda\) e \((h_{i+1/2})^{-1} \in \mathcal{M}_{\text{hom}}\) per l'ipotesi 3., allora siccome \(M\) è \(\lambda\)-ricco, esiste \(c \in N\) tale che
\begin{equation*}
	(h_{i+1/2})^{-1}\cup \set{(b,c)} \in \mathcal{M}_{\text{hom}}.
\end{equation*}

Si pone quindi \(h_{i+1}\coloneqq\left((h_{i+1/2})^{-1}\cup \set{(b,c)}\right)^{-1}\).
\end{itemize}
\end{itemize}

Dunque sia \(h\coloneqq\bigcup_{i<\lambda} h_{i}\) e siano
\begin{equation*}
M' = \operatorname{dom}h,\qquad N': \operatorname{rng}h.
\end{equation*}
\begin{itemize}
\item \(k \subseteq h\), \(h \in \mathcal{M}_{\text{hom}}\) poiché ogni sua restrizione finita lo è (ad ogni passo \(i<\omega\le \lambda\) si ha che \(h_{i} \in \mathcal{M}_{\text{hom}}\)), e inoltre \(h\) è iniettiva, pertanto \(h\upharpoonright M' = h\upharpoonright \operatorname{dom}h\) è una biiezione.
\item \(M'\preceq M\): si utilizza il \href{20250212113245-criterio_di_tarski_vaught.org}{Criterio di Tarski-Vaught}; sia \(\varphi(x)\) una \(\mathcal{L}(M')\)-formula consistente in \(M\). Allora \(\varphi(x)\) è una \(\mathcal{L}(\operatorname{dom}h_{i})\) per qualche \(i<\lambda\) (poiché ogni formula è una stringa finita), e pertanto \(\varphi=\varphi_{i,k}\) per qualche \(k<\lambda\). Quindi, per costruzione, esiste \(b \in \operatorname{dom}h_{\pi(i,k)+1} \subseteq M'\) tale che \(M\vDash \varphi[b]\).
\item \(N'\preceq N\): si utilizza il \href{20250212113245-criterio_di_tarski_vaught.org}{Criterio di Tarski-Vaught}; sia \(\psi(x)\) una \(\mathcal{L}(N')\)-formula consistente in \(N\). Allora \(\varphi(x)\) è una \(\mathcal{L}(\operatorname{rng}h_{i})\) per qualche \(i<\lambda\) (poiché ogni formula è una stringa finita), e pertanto \(\psi=\psi_{i,k}\) per qualche \(k<\lambda\). Quindi, per costruzione, esiste \(b \in \operatorname{rng}h_{\pi(i,k)+1} \subseteq N'\) tale che \(N\vDash \psi[b]\).\qed
\end{itemize}
\end{document}
