% Created 2026-02-07 Sat 19:31
% Intended LaTeX compiler: pdflatex
\documentclass[10pt]{article}
%% CREATO CON ORG - EMACS
\newcommand{\use}[2][]{\usepackage[#1]{#2}}
% PACCHETTI FONDAMENTLAI
\use[utf8]{inputenc}
\use[T1]{fontenc}
\use{graphicx}
\use{longtable}
\use{wrapfig}
\use{rotating}
\use[normalem]{ulem}
\use{amsmath}
\use{amsthm}
\use{amssymb}

\use{eucal} % Cambia mathcal{...}

\use{capt-of}
\use[italian]{babel}
\use[babel]{csquotes}
% bib la TEX lo carica in automatico org-cite
\use{microtype}
\use{lmodern}
\use{subfig} % sottofigure
\use{multicol} % due colonne
\use{lipsum} % lorem ipsum
\use{color} % colori in latex
\use{parskip} % rimuove l'indentazione dei nuovi paragrafi %% Add parbox=false to all new tcolorbox
\use{centernot}
\use[outline]{contour}\contourlength{3pt}
\use{fancyhdr}
\use{layout}
\use[most]{tcolorbox} % Riquadri colorati
\use{ifthen} % IFTHEN
\use{geometry}

% pacchetti matematica
\use{yhmath}
\use{dsfont}
\use{mathrsfs}
\use{cancel} % semplificare
\use{polynom} %divisione tra polinomi
\use{forest} % grafi ad albero
\use{booktabs} % tabelle
\use{commath} %simboli e differenziali
\use{bm} %bold
\use[fulladjust]{marginnote} %to use marginnote for date notes
\use{arrayjobx}%array
\use[intlimits]{empheq} % Riquadri colorati attorno alle equazioni
\use{mathtools}
\use{circuitikz} % Disegnare i circuiti
\use{mathtools}
\use{stmaryrd} % [[ \llbracket ]] \rrbracket
\use{bussproofs} % dimostrazioni

%%%%%%%%%%%%%


%%%% QUIVER
\newcommand{\duepunti}{\,\mathchar\numexpr"6000+`:\relax\,}
% A TikZ style for curved arrows of a fixed height, due to AndréC.
\tikzset{curve/.style={settings={#1},to path={(\tikztostart)
    .. controls ($(\tikztostart)!\pv{pos}!(\tikztotarget)!\pv{height}!270:(\tikztotarget)$)
    and ($(\tikztostart)!1-\pv{pos}!(\tikztotarget)!\pv{height}!270:(\tikztotarget)$)
    .. (\tikztotarget)\tikztonodes}},
    settings/.code={\tikzset{quiver/.cd,#1}
        \def\pv##1{\pgfkeysvalueof{/tikz/quiver/##1}}},
    quiver/.cd,pos/.initial=0.35,height/.initial=0}

% TikZ arrowhead/tail styles.
\tikzset{tail reversed/.code={\pgfsetarrowsstart{tikzcd to}}}
\tikzset{2tail/.code={\pgfsetarrowsstart{Implies[reversed]}}}
\tikzset{2tail reversed/.code={\pgfsetarrowsstart{Implies}}}
% TikZ arrow styles.
\tikzset{no body/.style={/tikz/dash pattern=on 0 off 1mm}}
%%%%%%%%%%


%% DEFINIZIONI COMANDI MATEMATICI
\let\sin\relax %TOGLIE LA DEFINIZIONE SU "\sin"

% cambia la definizione di empty set
% ---
\let\oldemptyset\emptyset
% ---
% \let\emptyset\varnothing
% ---
% \let\emptyset\relax
% \newcommand{\emptyset}{\text{\textnormal{\O}}}
% ---

\DeclareMathOperator{\bounded}{bd}
\DeclareMathOperator{\sin}{sen}
\DeclareMathOperator{\epi}{Epi}
\DeclareMathOperator{\cl}{cl}
\DeclareMathOperator{\graph}{graph}
\DeclareMathOperator{\arcsec}{arcsec}
\DeclareMathOperator{\arccot}{arccot}
\DeclareMathOperator{\arccsc}{arccsc}
\DeclareMathOperator{\spettro}{Spettro}
\DeclareMathOperator{\nulls}{nullspace}
\DeclareMathOperator{\dom}{dom}
\DeclareMathOperator{\ar}{ar}
\DeclareMathOperator{\const}{Const}
\DeclareMathOperator{\fun}{Fun}
\DeclareMathOperator{\rel}{Rel}
\DeclareMathOperator{\altezza}{ht}
\let\det\relax %TOGLIE LA DEFINIZIONE SU "\det"
\DeclareMathOperator{\det}{det}
\DeclareMathOperator{\End}{End}
\DeclareMathOperator{\gl}{GL}
\def\Id{\mathrm{Id}}
\def\id{\mathrm{id}}
\DeclareMathOperator{\I}{\mathds{1}}
\DeclareMathOperator{\II}{II}
\DeclareMathOperator{\rank}{rank}
\DeclareMathOperator{\tr}{tr}
\DeclareMathOperator{\tc}{t.c.}
\DeclareMathOperator{\T}{T}
\DeclareMathOperator{\var}{Var}
\DeclareMathOperator{\cov}{Cov}
\DeclareMathOperator{\st}{st}
\DeclareMathOperator{\mon}{Mon}
\newcommand{\card}[1]{\left\vert #1 \right\vert}
\newcommand{\trasposta}[1]{\prescript{\text{T}}{}{#1}}
\newcommand{\1}{\mathds{1}}
\newcommand{\R}{\mathds{R}}
\newcommand{\diesis}{\#}
\newcommand{\bemolle}{\flat}
\newcommand{\nonstandard}[1]{\prescript{*}{}{#1}}
\newcommand{\starR}{\nonstandard{\R}}
\newcommand{\borel}{\mathscr{B}}
\newcommand{\lebesgue}[1]{\mathscr{L}\left(#1\right)}
\newcommand{\media}{\mathds{E}}
\newcommand{\K}{\mathds{K}}
\newcommand{\A}{\mathds{A}}
\newcommand{\Q}{\mathds{Q}}
\newcommand{\N}{\mathds{N}}
\newcommand{\C}{\mathds{C}}
\newcommand{\Z}{\mathds{Z}}
\newcommand{\qo}{\hspace{1em}\text{q.o.}\,}
\renewcommand{\tilde}[1]{\widetilde{#1}}
\renewcommand{\parallel}{\mathrel{/\mkern-5mu/}}
\newcommand{\parti}[2][]{\wp_{#1}(#2)}
\newcommand{\diff}[1]{\operatorname{d}_{#1}}
\let\oldvec\vec
\renewcommand{\vec}[1]{\overrightarrow{\vphantom{i}#1}}
\newcommand{\floor}[1]{\left\lfloor #1 \right\rfloor}
\newcommand{\cat}[1]{\mathbf{#1}}
\newcommand{\dfreccia}[1]{\xrightarrow{\ #1 \ }}
\newcommand{\sfreccia}[1]{\xleftarrow{\ #1 \ }}
\newcommand{\formalsum}[2]{{\sum_{#1}^{#2}}{\vphantom{\sum}}'}
\newcommand{\minim}[2]{\mu_{#1}\, \left(#2\right)}
\newcommand{\concat}{\null^{\frown}} % concatenazione di stringe
\newcommand{\godelcode}[1]{\langle\!\langle #1 \rangle\!\rangle}
\newcommand{\godeldec}[1]{(\!(#1)\!)}
\newcommand{\termcode}[1]{\ulcorner #1\urcorner}
\newcommand{\partialto}{\dashrightarrow}
\newcommand{\restricted}{\upharpoonright}
\newcommand{\embeds}{\precsim}
\newcommand{\surjects}{\twoheadrightarrow}
\newcommand{\equipotenti}{\asymp}
%% \newcommand{\dotplus}{\mathbin{\dot{+}}} %% A quanto pare esiste già
\newcommand{\bigdot}{\mathbin{\boldsymbol{\cdot}}}
\newcommand{\dotexp}[1]{^{.#1}}
\newcommand{\conv}{\mathbin{*}}
\newcommand{\convolution}[2]{(#1\conv #2)}
\newcommand{\nil}{\mathfrak{N}}
\newcommand{\divisore}{\mathrel{|}}
\newcommand{\simplesso}[1]{\mathrm{e}_{#1}}

\renewcommand{\iff}{\mathrel{\longleftrightarrow}} %% Notazione Logica.
\newcommand{\oldiff}{\mathrel{\Longleftrightarrow}}
\renewcommand{\implies}{\mathrel{\rightarrow}} %% Notazione Logica
\newcommand{\oldimplies}{\mathrel{\Longrightarrow}}
\renewcommand{\impliedby}{\mathrel{\leftarrow}} %% Notazione Logica
\newcommand{\oldimpliedby}{\mathrel{\Longleftarrow}}

\newcommand{\IFF}{\quad\Longleftrightarrow\quad}
\newcommand{\IMPLICA}{\quad\Longrightarrow\quad}


\renewcommand{\descriptionlabel}[1]{\hspace{\labelsep}\normalfont #1} % remove bold from description


%% Definizione di Divergenza di K-L

\DeclarePairedDelimiterX{\infdivx}[2]{(}{)}{%
  #1\;\delimsize\|\;#2%
}
\newcommand{\kldiv}{D_{KL}\infdivx}

%% Definizione di \dotminus

\makeatletter
\newcommand{\dotminus}{\mathbin{\text{\@dotminus}}}

\newcommand{\@dotminus}{%
  \ooalign{\hidewidth\raise1ex\hbox{.}\hidewidth\cr$\m@th-$\cr}%
}
\makeatother

%tramite i prossimi due comandi posso decidere come scrivere i logaritmi naturali in tutti i documenti: ho infatti eliminato qualsiasi differenza tra "ln" e "log": se si vuole qualcosa di diverso bisogna inserire manualmente il tutto
\let\ln\relax
\DeclareMathOperator{\ln}{ln}
\let\log\relax
\DeclareMathOperator{\log}{log}
%%%%%%

%% NUOVI COMANDI
\newcommand{\straniero}[1]{\textit{#1}} %parole straniere
\newcommand{\titolo}[1]{\textsc{#1}} %titoli
\newcommand{\qedd}{\tag*{$\blacksquare$}} %qed per ambienti matemastici
\renewcommand{\qedsymbol}{$\blacksquare$} %modifica colore qed
\newcommand{\ooverline}[1]{\overline{\overline{#1}}}
\newcommand{\circoletto}[1]{\left(#1\right)^{\text{o}}}
%
\newcommand{\qmatrice}[1]{\begin{pmatrix}
#1_{11} & \cdots & #1_{1n}\\
\vdots & \ddots & \vdots \\
#1_{m1} & \cdots & #1_{mn}
\end{pmatrix}}
%
\newcommand{\parentesi}[2]{%
\underset{#1}{\underbrace{#2}}%
}
%
\newcommand{\norma}[1]{% Norma
\left\lVert#1\right\rVert%
}
\newcommand{\scalare}[2]{% Scalare
\left\langle #1, #2\right\rangle
}
%%%%%

%% RESTRIZIONI
\newcommand{\referenze}[2]{
        \phantomsection{}#2\textsuperscript{\textcolor{blue}{\textbf{#1}}}
}

\let\restriction\relax

\def\restriction#1#2{\mathchoice
              {\setbox1\hbox{${\displaystyle #1}_{\scriptstyle #2}$}
              \restrictionaux{#1}{#2}}
              {\setbox1\hbox{${\textstyle #1}_{\scriptstyle #2}$}
              \restrictionaux{#1}{#2}}
              {\setbox1\hbox{${\scriptstyle #1}_{\scriptscriptstyle #2}$}
              \restrictionaux{#1}{#2}}
              {\setbox1\hbox{${\scriptscriptstyle #1}_{\scriptscriptstyle #2}$}
              \restrictionaux{#1}{#2}}}
\def\restrictionaux#1#2{{#1\,\smash{\vrule height .8\ht1 depth .85\dp1}}_{\,#2}}
%%%%%%%%%%%

%%% FORMATTAZIONE FOOTNOTEMARK

\def\footnotemarkformatting#1{[#1]}
\renewcommand{\thefootnote}{\footnotemarkformatting{\arabic{footnote}}}

%% SEZIONE GRAFICA
\use{tikz}
\usetikzlibrary{matrix, patterns, calc, decorations.pathreplacing, hobby, decorations.markings, decorations.pathmorphing, babel}
\use{tikz-3dplot}
\use{mathrsfs} %per geogebra
\use{tikz-cd}
\tikzset
{
  %surface/.style={fill=black!10, shading=ball,fill opacity=0.4},
  plane/.style={black,pattern=north east lines},
  curve/.style={black,line width=0.5mm},
  dritto/.style={decoration={markings,mark=at position 0.5 with {\arrow{Stealth}}}, postaction=decorate},
  rovescio/.style={decoration={markings,mark=at position 0.5 with {\arrow{Stealth[reversed]}}}, postaction=decorate}
}
\use{pgfplots} % stampare le funzioni
        \pgfplotsset{/pgf/number format/use comma,compat=1.15}
        %\pgfplotsset{compat=1.15} %per geogebra
        \usepgfplotslibrary{fillbetween, polar}
%%%%%%

%% CITAZIONI
\use{lineno}

\newcommand{\citazione}[1]{%
  \begin{quotation}
  \begin{linenumbers}
  \modulolinenumbers[5]
  \begingroup
  \setlength{\parindent}{0cm}
  \noindent #1
  \endgroup
  \end{linenumbers}
  \end{quotation}\setcounter{linenumber}{1}
  }
%%%%%%

%%%%%%%%%%%%%%%%%%%%%%%%%%%%%%%%%%%%%%%%%%%%
%%%%%%%%%%%%%%%%%%%%%%%%%%%%%%%%%%%%%%%%%%%%

%% AMS THM

\theoremstyle{definition}% default
\newtheorem{thm}{Teorema}[section]
\newtheorem{lem}[thm]{Lemma}
\newtheorem{prop}[thm]{Proposizione}
\newtheorem{cor}[thm]{Corollario}
\newtheorem{esempio}[thm]{Esempio}
\theoremstyle{plain}
\newtheorem{definizione}[thm]{Definizione}
\theoremstyle{remark}
\newtheorem*{oss}{Osservazione}


%%%%%%%%%%%%%%%%%%%%%%%%%%%%%%%%%%%%%%%%%%%%
%%%%%%%%%%%%%%%%%%%%%%%%%%%%%%%%%%%%%%%%%%%%

\use{hyperref}
\hypersetup{%
        pdfauthor={Davide Peccioli},
        pdfsubject={},
        allcolors=black,
        citecolor=black,
%	colorlinks=true,
        bookmarksopen=true}
\setcounter{secnumdepth}{0} % rimuove i numeri di sezione senza rimuovere le ref
\renewcommand{\href}[2]{\textcolor{blue}{#2}} % disabilita il comando href
\use{enotez} %
\setenotez{%
 mark-format = \footnotemarkformatting % Mette i numeri tra parentesi quadre%
}\let\footnote=\endnote % rende tutte le note a pié pagina come delle note a fine file 


\let\olddocument\document % modifico l'ambiende documenti per non dover stampare \printendnote
\let\oldenddocument\enddocument
\renewenvironment{document}%
{%
  \olddocument
}{%
  \printendnotes\oldenddocument
}
\renewcommand{\thethm}{\arabic{thm}}

\usepackage[hyperref]{biblatex}
\addbibresource{~/Documents/org/roam/bib/master.bib}
\author{Davide Peccioli}
\date{\today}
\title{}
\begin{document}

\section{Teorema di Mayer-Vietoris (in omologia)}
\label{sec:orgd0706b3}
Sia \(R\) un \href{20241219112842-pid.org}{PID}. Sia \(X\) uno \href{20250103145124-topologia.org}{spazio topologico} e siano \(X_{1},X_{2} \subseteq X\) tali che l'\href{20250131155822-operazioni_insiemistiche_tra_classi_mk.org}{unione} delle \href{20250122181431-parte_interna.org}{parti interne} sia tutto lo spazio:
\begin{equation*}
\mathring{X}_{1}\cup \mathring{X}_{2} = X.
\end{equation*}
Si definiscono le seguenti inclusioni:
\begin{align*}
i_{1}: X_{1}\cap X_{2} &\longrightarrow X_{1}\\
i_{2}: X_{1}\cap X_{2} &\longrightarrow X_{2}\\
j_{1}: X_{1} &\longrightarrow X\\
j_{2}: X_{2} &\longrightarrow X.
\end{align*}

\begin{thm}
Si ha la seguente \href{20250120125004-successione_di_r_moduli_esatta.org}{SEL} dei \href{20250122133631-omologia_singolare.org}{moduli di omologia singolare} (detta \textbf{di Mayer-Vietoris})
\begin{equation*}
\scalebox{0.945}{%
\begin{tikzcd}[ampersand replacement=\&]
	{H_{q+1}(X)} \& {H_q(X_1\cap X_2)} \&\& {H_q(X_1)\oplus H_q(X_2)} \&\& {H_q(X)} \& {H_{q-1}(X_1\cap X_2)}
	\arrow[from=1-1, to=1-2]
	\arrow["{(i_1)_{\star}\oplus (i_2)_\star}", from=1-2, to=1-4]
	\arrow["{(j_1)_\star - (j_2)_\star}", from=1-4, to=1-6]
	\arrow[from=1-6, to=1-7]
\end{tikzcd}%
}
\end{equation*}
dove \(\bullet_{\star}\) rappresenta il \href{20250123115927-funtore_di_omologia_singolare.org}{funtore di omologia} e le mappe sono definite come segue:\footnote{Vedi la \href{20241213095808-somma_diretta.org}{somma diretta di morfismi}.}
\begin{align*}
(i_{1})_{\star} \oplus (i_{2})_{\star}: H_{q}(X_{1}\cap X_{2}) & \longrightarrow H_{q}(X_{1})\oplus H_{q}(X_{2})\\
c &\longmapsto \big((i_{1})_{\star}(c),\ (i_{2})_{\star}(c) \big)\\[1em]
(j_{1})_{\star}-(j_{2})_{\star}: H_{q}(X_{1})\oplus H_{q}(X_{2}) & \longrightarrow H_{q}(X)\\
(a,b) &\longmapsto (j_{1})_{\star} (a) - (j_{2})_{\star} (b)
\end{align*}
\end{thm}

\begin{proof}
\href{20241206142802-sottomoduli.org}{Si ha la seguente SEC di moduli}, per ogni \(q\):
\begin{equation*}
\begin{tikzcd}[ampersand replacement=\&,row sep=small]
	0 \& {S_q(X_1)\cap S_q(X_2)} \& {S_q(X_1)\oplus S_q(X_2)} \& {S_q(X_1)+S_q(X_2)} \& 0 \\
	\& c \& {(c,c)} \\
	\&\& {(s_1,s_2)} \& {s_1-s_2}
	\arrow[from=1-1, to=1-2]
	\arrow[from=1-2, to=1-3]
	\arrow[from=1-3, to=1-4]
	\arrow[from=1-4, to=1-5]
	\arrow[maps to, from=2-2, to=2-3]
	\arrow[maps to, from=3-3, to=3-4]
\end{tikzcd}
\end{equation*}
Tutti \href{20250120163759-categoria_complessi_di_catene.org}{questi morfismi commutano con le mappe di bordo}, e pertanto è ben definita la \href{20250120183640-sec_di_complessi_di_catene.org}{SEC di complessi}:\footnote{Vedi:
\begin{itemize}
\item \href{20260203110150-complesso_di_catene_somma.org}{Somma di complessi di catene}
\item \href{20260204100611-somma_diretta_di_complessi_di_catene.org}{Somma diretta di complessi di catene}
\item \href{20260203110150-complesso_di_catene_somma.org}{Intersezione di complessi di catene}
\end{itemize}}
\begin{equation*}
\begin{tikzcd}[ampersand replacement=\&,row sep=small]
	0 \& {\mathcal{S}_{\bullet}(X_1)\cap \mathcal{S}_{\bullet}(X_2)} \& {\mathcal{S}_{\bullet}(X_1)\oplus \mathcal{S}_{\bullet}(X_2)} \& {\mathcal{S}_{\bullet}(X_1)+\mathcal{S}_{\bullet}(X_2)} \& 0
	\arrow[from=1-1, to=1-2]
	\arrow[from=1-2, to=1-3]
	\arrow[from=1-3, to=1-4]
	\arrow[from=1-4, to=1-5]
\end{tikzcd}
\end{equation*}
ed è possibile applicare lo \href{20250120164938-zig_zag_lemma.org}{Zig-Zag Lemma}, ottenendo:
\begin{equation*}
\scalebox{0.86}{%
\begin{tikzcd}[ampersand replacement=\&,row sep=small]
	{H_q\big(\mathcal{S}_{\bullet}(X_1)\cap \mathcal{S}_{\bullet}(X_2)\big)} \& {H_q\big(\mathcal{S}_{\bullet}(X_1)\oplus \mathcal{S}_{\bullet}(X_2)\big)} \& {H_q\big(\mathcal{S}_{\bullet}(X_1)+\mathcal{S}_{\bullet}(X_2)\big)} \& {H_{q-1}\big(\mathcal{S}_{\bullet}(X_1)\cap \mathcal{S}_{\bullet}(X_2)\big)}
	\arrow[from=1-1, to=1-2]
	\arrow[from=1-2, to=1-3]
	\arrow[from=1-3, to=1-4]
\end{tikzcd}%
}
\end{equation*}
dove le mappe sono date dal \href{20250120165029-funtore_tra_chr_e_rmod.org}{funtore di omologia}.

Per ciascuno dei moduli di omologia:
\begin{itemize}
\item Si noti che \(\mathcal{S}_{\bullet}(X_{1})\cap \mathcal{S}_{\bullet}(X_{2}) = \mathcal{S}_{\bullet}^{\cap}(X_{1},X_{2}) = \mathcal{S}_{\bullet}(X_{1}\cap X_{2})\) \href{20250128131221-complesso_di_catene_singolare_somma.org}{complesso di catene singolare intersezione}, e pertanto
\begin{equation*}
  H_q\big(\mathcal{S}_{\bullet}(X_1)\cap \mathcal{S}_{\bullet}(X_2)\big) = H_{q}(X_{1}\cap X_{2}).
\end{equation*}
\item Per il \href{20250126223310-teorema_di_escissione.org}{Teorema di Escissione} (la proposizione sul \href{20250128131221-complesso_di_catene_singolare_somma.org}{complesso di catene singolare somma}),
\begin{equation*}
  H_{q}\big(\mathcal{S}_{\bullet}(X_{1})+ \mathcal{S}_{\bullet}(X_{2})\big) \cong H_{q}(X).
\end{equation*}
con la mappa data dalla semplice inclusione.
\item Siccome \href{20250120164857-modulo_di_omologia_dei_complessi_di_catene.org}{omologia} e \href{20260204100611-somma_diretta_di_complessi_di_catene.org}{somma diretta} \href{20260204120902-omologia_della_somma_diretta_di_complessi_di_catene.org}{commutano},
\begin{equation*}
\begin{tikzcd}[ampersand replacement=\&,row sep=tiny]
        {H_q\big(\mathcal{S}_{\bullet}(X_1)\oplus \mathcal{S}_{\bullet}(X_2)\big)} \&\& {H_q(X_1)\oplus H_q(X_2)} \\
        {\big[(s_1,s_2)\big]} \&\& {\big([s_1],[s_2]\big)}
        \arrow["\cong", from=1-1, to=1-3]
        \arrow[tail reversed, from=2-1, to=2-3]
\end{tikzcd}
\end{equation*}
\end{itemize}
Componendo i morfismi, costruiamo il seguente diagramma commutativo che lega la successione esatta algebrica (riga superiore) con la successione di Mayer-Vietoris desiderata (riga inferiore):

\begin{equation*}
\scalebox{0.85}{%
\begin{tikzcd}[ampersand replacement=\&, row sep=large, column sep=large]
	{H_q(\mathcal{S}_{\bullet}(X_1)\cap \mathcal{S}_{\bullet}(X_2))} \& {H_q(\mathcal{S}_{\bullet}(X_1)\oplus \mathcal{S}_{\bullet}(X_2))} \& {H_q(\mathcal{S}_{\bullet}(X_1)+\mathcal{S}_{\bullet}(X_2))} \\
	{H_q(X_1\cap X_2)} \& {H_q(X_1)\oplus H_q(X_2)} \& {H_q(X)}
	\arrow["{\alpha_*}", from=1-1, to=1-2]
	\arrow["{\beta_*}", from=1-2, to=1-3]
	\arrow["=", from=1-1, to=2-1]
	\arrow["{\Phi}"', "\cong", from=1-2, to=2-2]
	\arrow["{\Psi}"', "\cong", from=1-3, to=2-3]
	\arrow["{(i_1)_\star \oplus (i_2)_\star}"', from=2-1, to=2-2]
	\arrow["{(j_1)_\star - (j_2)_\star}"', from=2-2, to=2-3]
\end{tikzcd}%
}
\end{equation*}

Analizziamo i morfismi verticali e la commutatività dei quadrati:

\begin{enumerate}
\item \textbf{\textbf{Primo quadrato (Mappa diagonale):}}
Ricordiamo che \(\alpha(c) = (c,c)\).
L'isomorfismo \(\Phi\) è l'inverso dell'isomorfismo naturale che porta una coppia di classi nella classe della coppia (vedi punto precedente sulla somma diretta).
Partendo da \(c \in H_q(X_1 \cap X_2)\):
\begin{itemize}
\item Percorso alto: \(c \xrightarrow{\alpha_*} [(c,c)] \xrightarrow{\Phi} ([c], [c]) \in H_q(X_1) \oplus H_q(X_2)\).
\item Percorso basso: \(c \longmapsto ((i_1)_\star(c), (i_2)_\star(c))\).
\end{itemize}
Poiché \(i_1, i_2\) sono inclusioni, le classi coincidono.

\item \textbf{\textbf{Secondo quadrato (Mappa differenza):}}
Ricordiamo che \(\beta(s_1, s_2) = s_1 - s_2\).
L'isomorfismo \(\Psi\) è indotto dall'inclusione \(\iota: S_\bullet(X_1) + S_\bullet(X_2) \hookrightarrow S_\bullet(X)\) (che induce isomorfismo in omologia per il Teorema di Escissione/Barycentric subdivision).
Sia \(([z_1], [z_2]) \in H_q(X_1) \oplus H_q(X_2)\):
\begin{itemize}
\item Percorso basso: \(([z_1], [z_2]) \longmapsto (j_1)_\star([z_1]) - (j_2)_\star([z_2])\).
\item Percorso alto: \(([z_1], [z_2]) \xrightarrow{\Phi^{-1}} [(z_1, z_2)] \xrightarrow{\beta_*} [z_1 - z_2]\) (classe nel complesso somma).
\end{itemize}
Applicando \(\Psi\) (che è indotto dall'inclusione in \(X\)), la classe \([z_1 - z_2]\) diventa la classe in \(H_q(X)\).
Poiché i cicli sono lineari, \([z_1 - z_2] = [z_1] - [z_2]\) in \(H_q(X)\), che coincide esattamente con \((j_1)_\star([z_1]) - (j_2)_\star([z_2])\).
\end{enumerate}

La successione inferiore è quindi esatta poiché isomorfa ad una successione esatta (quella superiore fornita dallo Zig-Zag Lemma). \qedhere
\end{proof}

\begin{oss}
Se \(X\), \(Y\) sono spazi topologici e \(f:X\to Y\) continua tale che, per \(X_{1},X_{2} \subseteq X\), \(Y_{1},Y_{2} \subseteq X\):
\begin{align*}
f &: X \longrightarrow Y\\
\restriction{f}{X_{1}}&: X_{1} \longrightarrow Y_{1}\\
\restriction{f}{X_{2}}&: X_{2} \longrightarrow Y_{2}\\
\restriction{f}{X_{1}\cap X_{2}}&: X_{1}\cap X_{2} \longrightarrow Y_{1}\cap Y_{2}
\end{align*}
sono tutte \href{20250124155008-spazi_topologici_omotopicamente_equivalenti.org}{equivalenze omotopiche}, allora:
\begin{quote}
se \((Y,Y_{1},Y_{2},Y_{1}\cap Y_{2})\) rispetta Mayer-Vietoris, lo fa anche \((X,X_{1},X_{2},X_{1}\cap X_{2})\).
\end{quote}
\end{oss}
\end{document}
