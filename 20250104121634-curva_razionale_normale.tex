% Created 2026-02-07 Sat 19:35
% Intended LaTeX compiler: pdflatex
\documentclass[10pt]{article}
%% CREATO CON ORG - EMACS
\newcommand{\use}[2][]{\usepackage[#1]{#2}}
% PACCHETTI FONDAMENTLAI
\use[utf8]{inputenc}
\use[T1]{fontenc}
\use{graphicx}
\use{longtable}
\use{wrapfig}
\use{rotating}
\use[normalem]{ulem}
\use{amsmath}
\use{amsthm}
\use{amssymb}

\use{eucal} % Cambia mathcal{...}

\use{capt-of}
\use[italian]{babel}
\use[babel]{csquotes}
% bib la TEX lo carica in automatico org-cite
\use{microtype}
\use{lmodern}
\use{subfig} % sottofigure
\use{multicol} % due colonne
\use{lipsum} % lorem ipsum
\use{color} % colori in latex
\use{parskip} % rimuove l'indentazione dei nuovi paragrafi %% Add parbox=false to all new tcolorbox
\use{centernot}
\use[outline]{contour}\contourlength{3pt}
\use{fancyhdr}
\use{layout}
\use[most]{tcolorbox} % Riquadri colorati
\use{ifthen} % IFTHEN
\use{geometry}

% pacchetti matematica
\use{yhmath}
\use{dsfont}
\use{mathrsfs}
\use{cancel} % semplificare
\use{polynom} %divisione tra polinomi
\use{forest} % grafi ad albero
\use{booktabs} % tabelle
\use{commath} %simboli e differenziali
\use{bm} %bold
\use[fulladjust]{marginnote} %to use marginnote for date notes
\use{arrayjobx}%array
\use[intlimits]{empheq} % Riquadri colorati attorno alle equazioni
\use{mathtools}
\use{circuitikz} % Disegnare i circuiti
\use{mathtools}
\use{stmaryrd} % [[ \llbracket ]] \rrbracket
\use{bussproofs} % dimostrazioni

%%%%%%%%%%%%%


%%%% QUIVER
\newcommand{\duepunti}{\,\mathchar\numexpr"6000+`:\relax\,}
% A TikZ style for curved arrows of a fixed height, due to AndréC.
\tikzset{curve/.style={settings={#1},to path={(\tikztostart)
    .. controls ($(\tikztostart)!\pv{pos}!(\tikztotarget)!\pv{height}!270:(\tikztotarget)$)
    and ($(\tikztostart)!1-\pv{pos}!(\tikztotarget)!\pv{height}!270:(\tikztotarget)$)
    .. (\tikztotarget)\tikztonodes}},
    settings/.code={\tikzset{quiver/.cd,#1}
        \def\pv##1{\pgfkeysvalueof{/tikz/quiver/##1}}},
    quiver/.cd,pos/.initial=0.35,height/.initial=0}

% TikZ arrowhead/tail styles.
\tikzset{tail reversed/.code={\pgfsetarrowsstart{tikzcd to}}}
\tikzset{2tail/.code={\pgfsetarrowsstart{Implies[reversed]}}}
\tikzset{2tail reversed/.code={\pgfsetarrowsstart{Implies}}}
% TikZ arrow styles.
\tikzset{no body/.style={/tikz/dash pattern=on 0 off 1mm}}
%%%%%%%%%%


%% DEFINIZIONI COMANDI MATEMATICI
\let\sin\relax %TOGLIE LA DEFINIZIONE SU "\sin"

% cambia la definizione di empty set
% ---
\let\oldemptyset\emptyset
% ---
% \let\emptyset\varnothing
% ---
% \let\emptyset\relax
% \newcommand{\emptyset}{\text{\textnormal{\O}}}
% ---

\DeclareMathOperator{\bounded}{bd}
\DeclareMathOperator{\sin}{sen}
\DeclareMathOperator{\epi}{Epi}
\DeclareMathOperator{\cl}{cl}
\DeclareMathOperator{\graph}{graph}
\DeclareMathOperator{\arcsec}{arcsec}
\DeclareMathOperator{\arccot}{arccot}
\DeclareMathOperator{\arccsc}{arccsc}
\DeclareMathOperator{\spettro}{Spettro}
\DeclareMathOperator{\nulls}{nullspace}
\DeclareMathOperator{\dom}{dom}
\DeclareMathOperator{\ar}{ar}
\DeclareMathOperator{\const}{Const}
\DeclareMathOperator{\fun}{Fun}
\DeclareMathOperator{\rel}{Rel}
\DeclareMathOperator{\altezza}{ht}
\let\det\relax %TOGLIE LA DEFINIZIONE SU "\det"
\DeclareMathOperator{\det}{det}
\DeclareMathOperator{\End}{End}
\DeclareMathOperator{\gl}{GL}
\def\Id{\mathrm{Id}}
\def\id{\mathrm{id}}
\DeclareMathOperator{\I}{\mathds{1}}
\DeclareMathOperator{\II}{II}
\DeclareMathOperator{\rank}{rank}
\DeclareMathOperator{\tr}{tr}
\DeclareMathOperator{\tc}{t.c.}
\DeclareMathOperator{\T}{T}
\DeclareMathOperator{\var}{Var}
\DeclareMathOperator{\cov}{Cov}
\DeclareMathOperator{\st}{st}
\DeclareMathOperator{\mon}{Mon}
\newcommand{\card}[1]{\left\vert #1 \right\vert}
\newcommand{\trasposta}[1]{\prescript{\text{T}}{}{#1}}
\newcommand{\1}{\mathds{1}}
\newcommand{\R}{\mathds{R}}
\newcommand{\diesis}{\#}
\newcommand{\bemolle}{\flat}
\newcommand{\nonstandard}[1]{\prescript{*}{}{#1}}
\newcommand{\starR}{\nonstandard{\R}}
\newcommand{\borel}{\mathscr{B}}
\newcommand{\lebesgue}[1]{\mathscr{L}\left(#1\right)}
\newcommand{\media}{\mathds{E}}
\newcommand{\K}{\mathds{K}}
\newcommand{\A}{\mathds{A}}
\newcommand{\Q}{\mathds{Q}}
\newcommand{\N}{\mathds{N}}
\newcommand{\C}{\mathds{C}}
\newcommand{\Z}{\mathds{Z}}
\newcommand{\qo}{\hspace{1em}\text{q.o.}\,}
\renewcommand{\tilde}[1]{\widetilde{#1}}
\renewcommand{\parallel}{\mathrel{/\mkern-5mu/}}
\newcommand{\parti}[2][]{\wp_{#1}(#2)}
\newcommand{\diff}[1]{\operatorname{d}_{#1}}
\let\oldvec\vec
\renewcommand{\vec}[1]{\overrightarrow{\vphantom{i}#1}}
\newcommand{\floor}[1]{\left\lfloor #1 \right\rfloor}
\newcommand{\cat}[1]{\mathbf{#1}}
\newcommand{\dfreccia}[1]{\xrightarrow{\ #1 \ }}
\newcommand{\sfreccia}[1]{\xleftarrow{\ #1 \ }}
\newcommand{\formalsum}[2]{{\sum_{#1}^{#2}}{\vphantom{\sum}}'}
\newcommand{\minim}[2]{\mu_{#1}\, \left(#2\right)}
\newcommand{\concat}{\null^{\frown}} % concatenazione di stringe
\newcommand{\godelcode}[1]{\langle\!\langle #1 \rangle\!\rangle}
\newcommand{\godeldec}[1]{(\!(#1)\!)}
\newcommand{\termcode}[1]{\ulcorner #1\urcorner}
\newcommand{\partialto}{\dashrightarrow}
\newcommand{\restricted}{\upharpoonright}
\newcommand{\embeds}{\precsim}
\newcommand{\surjects}{\twoheadrightarrow}
\newcommand{\equipotenti}{\asymp}
%% \newcommand{\dotplus}{\mathbin{\dot{+}}} %% A quanto pare esiste già
\newcommand{\bigdot}{\mathbin{\boldsymbol{\cdot}}}
\newcommand{\dotexp}[1]{^{.#1}}
\newcommand{\conv}{\mathbin{*}}
\newcommand{\convolution}[2]{(#1\conv #2)}
\newcommand{\nil}{\mathfrak{N}}
\newcommand{\divisore}{\mathrel{|}}
\newcommand{\simplesso}[1]{\mathrm{e}_{#1}}

\renewcommand{\iff}{\mathrel{\longleftrightarrow}} %% Notazione Logica.
\newcommand{\oldiff}{\mathrel{\Longleftrightarrow}}
\renewcommand{\implies}{\mathrel{\rightarrow}} %% Notazione Logica
\newcommand{\oldimplies}{\mathrel{\Longrightarrow}}
\renewcommand{\impliedby}{\mathrel{\leftarrow}} %% Notazione Logica
\newcommand{\oldimpliedby}{\mathrel{\Longleftarrow}}

\newcommand{\IFF}{\quad\Longleftrightarrow\quad}
\newcommand{\IMPLICA}{\quad\Longrightarrow\quad}


\renewcommand{\descriptionlabel}[1]{\hspace{\labelsep}\normalfont #1} % remove bold from description


%% Definizione di Divergenza di K-L

\DeclarePairedDelimiterX{\infdivx}[2]{(}{)}{%
  #1\;\delimsize\|\;#2%
}
\newcommand{\kldiv}{D_{KL}\infdivx}

%% Definizione di \dotminus

\makeatletter
\newcommand{\dotminus}{\mathbin{\text{\@dotminus}}}

\newcommand{\@dotminus}{%
  \ooalign{\hidewidth\raise1ex\hbox{.}\hidewidth\cr$\m@th-$\cr}%
}
\makeatother

%tramite i prossimi due comandi posso decidere come scrivere i logaritmi naturali in tutti i documenti: ho infatti eliminato qualsiasi differenza tra "ln" e "log": se si vuole qualcosa di diverso bisogna inserire manualmente il tutto
\let\ln\relax
\DeclareMathOperator{\ln}{ln}
\let\log\relax
\DeclareMathOperator{\log}{log}
%%%%%%

%% NUOVI COMANDI
\newcommand{\straniero}[1]{\textit{#1}} %parole straniere
\newcommand{\titolo}[1]{\textsc{#1}} %titoli
\newcommand{\qedd}{\tag*{$\blacksquare$}} %qed per ambienti matemastici
\renewcommand{\qedsymbol}{$\blacksquare$} %modifica colore qed
\newcommand{\ooverline}[1]{\overline{\overline{#1}}}
\newcommand{\circoletto}[1]{\left(#1\right)^{\text{o}}}
%
\newcommand{\qmatrice}[1]{\begin{pmatrix}
#1_{11} & \cdots & #1_{1n}\\
\vdots & \ddots & \vdots \\
#1_{m1} & \cdots & #1_{mn}
\end{pmatrix}}
%
\newcommand{\parentesi}[2]{%
\underset{#1}{\underbrace{#2}}%
}
%
\newcommand{\norma}[1]{% Norma
\left\lVert#1\right\rVert%
}
\newcommand{\scalare}[2]{% Scalare
\left\langle #1, #2\right\rangle
}
%%%%%

%% RESTRIZIONI
\newcommand{\referenze}[2]{
        \phantomsection{}#2\textsuperscript{\textcolor{blue}{\textbf{#1}}}
}

\let\restriction\relax

\def\restriction#1#2{\mathchoice
              {\setbox1\hbox{${\displaystyle #1}_{\scriptstyle #2}$}
              \restrictionaux{#1}{#2}}
              {\setbox1\hbox{${\textstyle #1}_{\scriptstyle #2}$}
              \restrictionaux{#1}{#2}}
              {\setbox1\hbox{${\scriptstyle #1}_{\scriptscriptstyle #2}$}
              \restrictionaux{#1}{#2}}
              {\setbox1\hbox{${\scriptscriptstyle #1}_{\scriptscriptstyle #2}$}
              \restrictionaux{#1}{#2}}}
\def\restrictionaux#1#2{{#1\,\smash{\vrule height .8\ht1 depth .85\dp1}}_{\,#2}}
%%%%%%%%%%%

%%% FORMATTAZIONE FOOTNOTEMARK

\def\footnotemarkformatting#1{[#1]}
\renewcommand{\thefootnote}{\footnotemarkformatting{\arabic{footnote}}}

%% SEZIONE GRAFICA
\use{tikz}
\usetikzlibrary{matrix, patterns, calc, decorations.pathreplacing, hobby, decorations.markings, decorations.pathmorphing, babel}
\use{tikz-3dplot}
\use{mathrsfs} %per geogebra
\use{tikz-cd}
\tikzset
{
  %surface/.style={fill=black!10, shading=ball,fill opacity=0.4},
  plane/.style={black,pattern=north east lines},
  curve/.style={black,line width=0.5mm},
  dritto/.style={decoration={markings,mark=at position 0.5 with {\arrow{Stealth}}}, postaction=decorate},
  rovescio/.style={decoration={markings,mark=at position 0.5 with {\arrow{Stealth[reversed]}}}, postaction=decorate}
}
\use{pgfplots} % stampare le funzioni
        \pgfplotsset{/pgf/number format/use comma,compat=1.15}
        %\pgfplotsset{compat=1.15} %per geogebra
        \usepgfplotslibrary{fillbetween, polar}
%%%%%%

%% CITAZIONI
\use{lineno}

\newcommand{\citazione}[1]{%
  \begin{quotation}
  \begin{linenumbers}
  \modulolinenumbers[5]
  \begingroup
  \setlength{\parindent}{0cm}
  \noindent #1
  \endgroup
  \end{linenumbers}
  \end{quotation}\setcounter{linenumber}{1}
  }
%%%%%%

%%%%%%%%%%%%%%%%%%%%%%%%%%%%%%%%%%%%%%%%%%%%
%%%%%%%%%%%%%%%%%%%%%%%%%%%%%%%%%%%%%%%%%%%%

%% AMS THM

\theoremstyle{definition}% default
\newtheorem{thm}{Teorema}[section]
\newtheorem{lem}[thm]{Lemma}
\newtheorem{prop}[thm]{Proposizione}
\newtheorem{cor}[thm]{Corollario}
\newtheorem{esempio}[thm]{Esempio}
\theoremstyle{plain}
\newtheorem{definizione}[thm]{Definizione}
\theoremstyle{remark}
\newtheorem*{oss}{Osservazione}


%%%%%%%%%%%%%%%%%%%%%%%%%%%%%%%%%%%%%%%%%%%%
%%%%%%%%%%%%%%%%%%%%%%%%%%%%%%%%%%%%%%%%%%%%

\use{hyperref}
\hypersetup{%
        pdfauthor={Davide Peccioli},
        pdfsubject={},
        allcolors=black,
        citecolor=black,
%	colorlinks=true,
        bookmarksopen=true}
\setcounter{secnumdepth}{0} % rimuove i numeri di sezione senza rimuovere le ref
\renewcommand{\href}[2]{\textcolor{blue}{#2}} % disabilita il comando href
\use{enotez} %
\setenotez{%
 mark-format = \footnotemarkformatting % Mette i numeri tra parentesi quadre%
}\let\footnote=\endnote % rende tutte le note a pié pagina come delle note a fine file 


\let\olddocument\document % modifico l'ambiende documenti per non dover stampare \printendnote
\let\oldenddocument\enddocument
\renewenvironment{document}%
{%
  \olddocument
}{%
  \printendnotes\oldenddocument
}
\renewcommand{\thethm}{\arabic{thm}}

\usepackage[hyperref]{biblatex}
\addbibresource{~/Documents/org/roam/bib/master.bib}
\author{Davide Peccioli}
\date{\today}
\title{Curva Razionale Normale}
\begin{document}

Sia \(\K\) un \href{20241231112713-campo_algebricamente_chiuso.org}{campo algebricamente chiuso}. La curva razionale normale è una generalizzazione della \href{20250102104043-cubica_gobba.org}{Cubica Gobba}.

Si definisce
\begin{align*}
\nu_{d}: \mathds{P}^{1} &\longrightarrow \mathds{P}^{d}_{[z_{0}:\dots:z_{d}]}\\
[x_{0}:x_{1}] &\longmapsto [x_{0}^{d}:x_{0}^{d-1}x_{1}:\dots:x_{1}^{d}]
\end{align*}
(vedi \href{20241231115051-spazio_proiettivo.org}{Spazio Proiettivo} e \href{20241231123223-varieta_algebrica_proiettiva.org}{Varietà Algebrica Proiettiva})

Ovviamente \(\nu_{d}\) è un \href{20250104120600-morfismo_tra_varieta_algebriche_proiettive.org}{morfismo}. Sia \(C\coloneqq\nu_{d}(\mathds{P}^{1})\). \(C\) è la curva razionale normale.
\section{\(C\) è una \href{20241231123223-varieta_algebrica_proiettiva.org}{Varietà Algebrica Proiettiva}.}
\label{sec:org928fda3}

Sia \(A\) la \href{20250104111539-spazio_delle_matrici.org}{matrice}:
\begin{equation*}
A=\begin{bmatrix}
z_{0} & z_{1} & \dots & z_{d-1}\\
z_{1} & z_{2} & \dots & z_{d}
\end{bmatrix}
\end{equation*}
e sia \(Y=V(\operatorname{rank} A -1)\) (vedi \href{20241231112823-radici_polinomiali.org}{Luogo di zeri} e \href{20250104170945-rango_di_una_matrice.org}{Rango di una matrice}). Questa è una espressione polinoimale, infatti \(\operatorname{rank}A=1\) se e solo se tutti i \href{20250104171415-minori_di_una_matrice.org}{minori} \(2\times 2\) di \(A\) hannp determinante nullo. Pertanto \(Y\) è il luogo delle soluzioni del seguente sistema
\begin{equation*}
\begin{cases}
z_{0}z_{2}=z_{1}^{2}\\
z_{1}z_{3}=z_{2}^{2}\\
\vdots\\
z_{d-2}z_{d}=z_{d-1}^{2}
\end{cases}
\end{equation*}
Si ha che \(C=Y\).
\subsection{\(C \subseteq Y\)}
\label{sec:org5873274}
Questa inclusione è ovvia: se \(p = [p_{0}:p_{1}] \in \mathds{P}^{1}\) e \(q=[q_{0}:\dots:q_{d}]=\nu_{d}(p)\), allora
\begin{equation*}
q=[p_{0}^{d}:p_{0}^{d-1}p_{1}:\dots:p_{1}^{d}]
\end{equation*}
Facendo i calcoli si ha
\begin{align*}
q_{0}q_{2} &= p_{0}^{d}p_{0}^{d-2}p_{1}^{2} = p_{0}^{2d-2}p_{1}^{2} = (p_{0}^{d-1}p_{1})^{2}=q_{3}^{2}\\
&\vdots\\
q_{i}q_{i+2} &= (p_{0}^{d-i}p_{1}^{i})\ (p_{0}^{d-i-2}p_{1}^{i+2}) = p_{0}^{d-i+d-i-2}p_{1}^{2i+2}\\
&= p_{0}^{2(d-i-1)}p_{1}^{2(i+1)} = (p_{0}^{d-i-1}p_{1}^{i+1})^{2} = q_{i+1}^{2}
\end{align*}

e dunque i punti di \(C\) sono soluzioni del sistema.
\subsection{\(Y \subseteq C\)}
\label{sec:orgfd26597}
Sia \(p \in Y\), \(p=[p_{0}:\dots:p_{d}]\). \(p\) risolve il sistema di equazioni.

Se \(p_{0}=0\) allora \(p_{1}=0\) (applicando la prima equazione).
Se \(p_{1}=0\) allora \(p_{2}=0\) (applicando la seconda equazione).
In generale, se \(p_{i}=0\) per \(i\le d-2\) si ha che \(p_{i+1}=0\) applicando la \(i+1\)-esima equazione.

Dunque, se \(p_{0}=0\) allora \(\forall\, i =1,\dots,d-1\) si ha che \(p_{i}=0\). Ovviamente \(p_{d}=1\), altrimenti \(p\notin \mathds{P}^{d}\).

Allo stesso modo, se \(p_{d}=0\), allora \(\forall\, i = 2,\dots,d-1\) si ha che \(p_{i}=0\). Ovviamente \(p_{0}=1\), altrimenti \(p \notin \mathds{P}^{d}\).

Quindi almeno uno tra \(p_{0}\) e \(p_{d}\) è non nullo.

Suppongo \(p_{0}\neq 0\). Allora posso scrivere
\begin{equation*}
p = [1:t_{1}:\dots:t_{d}]
\end{equation*}
tali che
\begin{equation*}
\begin{cases}
t_{2}=t_{1}^{2}\\
t_{1}\cdot t_{3}=t_{2}^{2}\\
t_{2}\cdot t_{4}=t_{3}^{2}\\
\vdots
\end{cases}\quad \implies\quad \begin{cases}
t_{2}=t_{1}^{2}\\
t_{3}=t_{2}^{2}/t_{1} = t_{1}^{3}\\
t_{4} = t_{3}^{2}/t_{2} = t_{1}^{4}\\
\vdots\\
t_{i} = t_{1}^{i}
\end{cases}
\end{equation*}
e dunque \(p =\nu_{d}\left([1:t_{1}]\right)\), e \(p \in C\).
\section{\(\nu_{d}\) è un isomorfismo}
\label{sec:orgd831818}

Consideriamo la corestrizione
\begin{align*}
\nu_{d}: \mathds{P}^{1} &\longrightarrow C \subseteq \mathds{P}^{d}_{[z_{0}:\dots:z_{d}]}\\
[x_{0}:x_{1}] &\longmapsto [x_{0}^{d}:x_{0}^{d-1}x_{1}:\dots:x_{1}^{d}]
\end{align*}
Questa mappa è un \href{20250104120600-morfismo_tra_varieta_algebriche_proiettive.org}{isomorfismo}. Per dimostrarlo, si scrive esplicitamente l'inversa.

Fissiamo \(p \in C\).

Sia \(H_{i} = V(z_{i})\) per ogni \(i=0,\dots,d\). Per l'argomento visto sopra, \(p\notin H_{0}\cap H_{d}\) e dunque \(p \in U_{0}\cup U_{d}\), dove
\begin{equation*}
U_{i} = \set{z_{i}\neq 0}
\end{equation*}

È possibile dunque definire l'inversa come:
\begin{equation*}
\nu_{d}^{-1}(z) = \begin{cases}
[z_{0}:z_{1}] & z\in C\cap U_{0}\\
[z_{d-1}:z_{d}] & z\in C\cap U_{d}
\end{cases}
\end{equation*}
\subsection{\(\nu_{d}^{-1}\) è veramente l'inversa}
\label{sec:org8d0befb}

Sia \(p = \nu_{d}[x_{0}:x_{1}]\), quindi
\begin{equation*}
p = [x_{0}^{d}:x_{0}^{d-1}x_{1}:\dots:x_{0}x_{1}^{d-1}:x_1^{d}]
\end{equation*}

Se \(p \in U_{0}\), si ha che \(\nu_{d}^{-1}(p)=[x_{0}^{d}:x_{0}^{d-1}x_{1}]=x_{0}^{d-1}[x_{0}:x_{1}] = [x_{0}:x_{1}]\).
Se \(p \in U_{d}\), si ha che \(\nu_{d}^{-1}(p)=[x_{0}x_{1}^{d-1}:x_{1}^{d}]=x_{1}^{d-1}[x_{0}:x_{1}]=[x_{0}:x_{1}]\).
\subsection{\(\nu_{d}^{-1}\) è ben definito.}
\label{sec:org3f47ef6}

Sia ora \(z \in C\cap U_{0}\cap U_{d}\), \(z=\nu_{d}[x_{0}:x_{1}]\)
\begin{equation*}
\frac{z_{1}}{z_{0}} = \frac{x_{0}^{d-1}x_{1}}{x_{0}^{d}}=\frac{x_{1}}{x_{0}}=\frac{x_{1}\ x_{1}^{d-1}}{x_{0}x_{1}^{d-1}} = \frac{z_{d}}{z_{d-1}}
\end{equation*}
e dunque \([z_{0}:z_{1}]=[z_{d-1}:z_{d}]\).
\end{document}
