% Created 2026-02-07 Sat 19:31
% Intended LaTeX compiler: pdflatex
\documentclass[10pt]{article}
%% CREATO CON ORG - EMACS
\newcommand{\use}[2][]{\usepackage[#1]{#2}}
% PACCHETTI FONDAMENTLAI
\use[utf8]{inputenc}
\use[T1]{fontenc}
\use{graphicx}
\use{longtable}
\use{wrapfig}
\use{rotating}
\use[normalem]{ulem}
\use{amsmath}
\use{amsthm}
\use{amssymb}

\use{eucal} % Cambia mathcal{...}

\use{capt-of}
\use[italian]{babel}
\use[babel]{csquotes}
% bib la TEX lo carica in automatico org-cite
\use{microtype}
\use{lmodern}
\use{subfig} % sottofigure
\use{multicol} % due colonne
\use{lipsum} % lorem ipsum
\use{color} % colori in latex
\use{parskip} % rimuove l'indentazione dei nuovi paragrafi %% Add parbox=false to all new tcolorbox
\use{centernot}
\use[outline]{contour}\contourlength{3pt}
\use{fancyhdr}
\use{layout}
\use[most]{tcolorbox} % Riquadri colorati
\use{ifthen} % IFTHEN
\use{geometry}

% pacchetti matematica
\use{yhmath}
\use{dsfont}
\use{mathrsfs}
\use{cancel} % semplificare
\use{polynom} %divisione tra polinomi
\use{forest} % grafi ad albero
\use{booktabs} % tabelle
\use{commath} %simboli e differenziali
\use{bm} %bold
\use[fulladjust]{marginnote} %to use marginnote for date notes
\use{arrayjobx}%array
\use[intlimits]{empheq} % Riquadri colorati attorno alle equazioni
\use{mathtools}
\use{circuitikz} % Disegnare i circuiti
\use{mathtools}
\use{stmaryrd} % [[ \llbracket ]] \rrbracket
\use{bussproofs} % dimostrazioni

%%%%%%%%%%%%%


%%%% QUIVER
\newcommand{\duepunti}{\,\mathchar\numexpr"6000+`:\relax\,}
% A TikZ style for curved arrows of a fixed height, due to AndréC.
\tikzset{curve/.style={settings={#1},to path={(\tikztostart)
    .. controls ($(\tikztostart)!\pv{pos}!(\tikztotarget)!\pv{height}!270:(\tikztotarget)$)
    and ($(\tikztostart)!1-\pv{pos}!(\tikztotarget)!\pv{height}!270:(\tikztotarget)$)
    .. (\tikztotarget)\tikztonodes}},
    settings/.code={\tikzset{quiver/.cd,#1}
        \def\pv##1{\pgfkeysvalueof{/tikz/quiver/##1}}},
    quiver/.cd,pos/.initial=0.35,height/.initial=0}

% TikZ arrowhead/tail styles.
\tikzset{tail reversed/.code={\pgfsetarrowsstart{tikzcd to}}}
\tikzset{2tail/.code={\pgfsetarrowsstart{Implies[reversed]}}}
\tikzset{2tail reversed/.code={\pgfsetarrowsstart{Implies}}}
% TikZ arrow styles.
\tikzset{no body/.style={/tikz/dash pattern=on 0 off 1mm}}
%%%%%%%%%%


%% DEFINIZIONI COMANDI MATEMATICI
\let\sin\relax %TOGLIE LA DEFINIZIONE SU "\sin"

% cambia la definizione di empty set
% ---
\let\oldemptyset\emptyset
% ---
% \let\emptyset\varnothing
% ---
% \let\emptyset\relax
% \newcommand{\emptyset}{\text{\textnormal{\O}}}
% ---

\DeclareMathOperator{\bounded}{bd}
\DeclareMathOperator{\sin}{sen}
\DeclareMathOperator{\epi}{Epi}
\DeclareMathOperator{\cl}{cl}
\DeclareMathOperator{\graph}{graph}
\DeclareMathOperator{\arcsec}{arcsec}
\DeclareMathOperator{\arccot}{arccot}
\DeclareMathOperator{\arccsc}{arccsc}
\DeclareMathOperator{\spettro}{Spettro}
\DeclareMathOperator{\nulls}{nullspace}
\DeclareMathOperator{\dom}{dom}
\DeclareMathOperator{\ar}{ar}
\DeclareMathOperator{\const}{Const}
\DeclareMathOperator{\fun}{Fun}
\DeclareMathOperator{\rel}{Rel}
\DeclareMathOperator{\altezza}{ht}
\let\det\relax %TOGLIE LA DEFINIZIONE SU "\det"
\DeclareMathOperator{\det}{det}
\DeclareMathOperator{\End}{End}
\DeclareMathOperator{\gl}{GL}
\def\Id{\mathrm{Id}}
\def\id{\mathrm{id}}
\DeclareMathOperator{\I}{\mathds{1}}
\DeclareMathOperator{\II}{II}
\DeclareMathOperator{\rank}{rank}
\DeclareMathOperator{\tr}{tr}
\DeclareMathOperator{\tc}{t.c.}
\DeclareMathOperator{\T}{T}
\DeclareMathOperator{\var}{Var}
\DeclareMathOperator{\cov}{Cov}
\DeclareMathOperator{\st}{st}
\DeclareMathOperator{\mon}{Mon}
\newcommand{\card}[1]{\left\vert #1 \right\vert}
\newcommand{\trasposta}[1]{\prescript{\text{T}}{}{#1}}
\newcommand{\1}{\mathds{1}}
\newcommand{\R}{\mathds{R}}
\newcommand{\diesis}{\#}
\newcommand{\bemolle}{\flat}
\newcommand{\nonstandard}[1]{\prescript{*}{}{#1}}
\newcommand{\starR}{\nonstandard{\R}}
\newcommand{\borel}{\mathscr{B}}
\newcommand{\lebesgue}[1]{\mathscr{L}\left(#1\right)}
\newcommand{\media}{\mathds{E}}
\newcommand{\K}{\mathds{K}}
\newcommand{\A}{\mathds{A}}
\newcommand{\Q}{\mathds{Q}}
\newcommand{\N}{\mathds{N}}
\newcommand{\C}{\mathds{C}}
\newcommand{\Z}{\mathds{Z}}
\newcommand{\qo}{\hspace{1em}\text{q.o.}\,}
\renewcommand{\tilde}[1]{\widetilde{#1}}
\renewcommand{\parallel}{\mathrel{/\mkern-5mu/}}
\newcommand{\parti}[2][]{\wp_{#1}(#2)}
\newcommand{\diff}[1]{\operatorname{d}_{#1}}
\let\oldvec\vec
\renewcommand{\vec}[1]{\overrightarrow{\vphantom{i}#1}}
\newcommand{\floor}[1]{\left\lfloor #1 \right\rfloor}
\newcommand{\cat}[1]{\mathbf{#1}}
\newcommand{\dfreccia}[1]{\xrightarrow{\ #1 \ }}
\newcommand{\sfreccia}[1]{\xleftarrow{\ #1 \ }}
\newcommand{\formalsum}[2]{{\sum_{#1}^{#2}}{\vphantom{\sum}}'}
\newcommand{\minim}[2]{\mu_{#1}\, \left(#2\right)}
\newcommand{\concat}{\null^{\frown}} % concatenazione di stringe
\newcommand{\godelcode}[1]{\langle\!\langle #1 \rangle\!\rangle}
\newcommand{\godeldec}[1]{(\!(#1)\!)}
\newcommand{\termcode}[1]{\ulcorner #1\urcorner}
\newcommand{\partialto}{\dashrightarrow}
\newcommand{\restricted}{\upharpoonright}
\newcommand{\embeds}{\precsim}
\newcommand{\surjects}{\twoheadrightarrow}
\newcommand{\equipotenti}{\asymp}
%% \newcommand{\dotplus}{\mathbin{\dot{+}}} %% A quanto pare esiste già
\newcommand{\bigdot}{\mathbin{\boldsymbol{\cdot}}}
\newcommand{\dotexp}[1]{^{.#1}}
\newcommand{\conv}{\mathbin{*}}
\newcommand{\convolution}[2]{(#1\conv #2)}
\newcommand{\nil}{\mathfrak{N}}
\newcommand{\divisore}{\mathrel{|}}
\newcommand{\simplesso}[1]{\mathrm{e}_{#1}}

\renewcommand{\iff}{\mathrel{\longleftrightarrow}} %% Notazione Logica.
\newcommand{\oldiff}{\mathrel{\Longleftrightarrow}}
\renewcommand{\implies}{\mathrel{\rightarrow}} %% Notazione Logica
\newcommand{\oldimplies}{\mathrel{\Longrightarrow}}
\renewcommand{\impliedby}{\mathrel{\leftarrow}} %% Notazione Logica
\newcommand{\oldimpliedby}{\mathrel{\Longleftarrow}}

\newcommand{\IFF}{\quad\Longleftrightarrow\quad}
\newcommand{\IMPLICA}{\quad\Longrightarrow\quad}


\renewcommand{\descriptionlabel}[1]{\hspace{\labelsep}\normalfont #1} % remove bold from description


%% Definizione di Divergenza di K-L

\DeclarePairedDelimiterX{\infdivx}[2]{(}{)}{%
  #1\;\delimsize\|\;#2%
}
\newcommand{\kldiv}{D_{KL}\infdivx}

%% Definizione di \dotminus

\makeatletter
\newcommand{\dotminus}{\mathbin{\text{\@dotminus}}}

\newcommand{\@dotminus}{%
  \ooalign{\hidewidth\raise1ex\hbox{.}\hidewidth\cr$\m@th-$\cr}%
}
\makeatother

%tramite i prossimi due comandi posso decidere come scrivere i logaritmi naturali in tutti i documenti: ho infatti eliminato qualsiasi differenza tra "ln" e "log": se si vuole qualcosa di diverso bisogna inserire manualmente il tutto
\let\ln\relax
\DeclareMathOperator{\ln}{ln}
\let\log\relax
\DeclareMathOperator{\log}{log}
%%%%%%

%% NUOVI COMANDI
\newcommand{\straniero}[1]{\textit{#1}} %parole straniere
\newcommand{\titolo}[1]{\textsc{#1}} %titoli
\newcommand{\qedd}{\tag*{$\blacksquare$}} %qed per ambienti matemastici
\renewcommand{\qedsymbol}{$\blacksquare$} %modifica colore qed
\newcommand{\ooverline}[1]{\overline{\overline{#1}}}
\newcommand{\circoletto}[1]{\left(#1\right)^{\text{o}}}
%
\newcommand{\qmatrice}[1]{\begin{pmatrix}
#1_{11} & \cdots & #1_{1n}\\
\vdots & \ddots & \vdots \\
#1_{m1} & \cdots & #1_{mn}
\end{pmatrix}}
%
\newcommand{\parentesi}[2]{%
\underset{#1}{\underbrace{#2}}%
}
%
\newcommand{\norma}[1]{% Norma
\left\lVert#1\right\rVert%
}
\newcommand{\scalare}[2]{% Scalare
\left\langle #1, #2\right\rangle
}
%%%%%

%% RESTRIZIONI
\newcommand{\referenze}[2]{
        \phantomsection{}#2\textsuperscript{\textcolor{blue}{\textbf{#1}}}
}

\let\restriction\relax

\def\restriction#1#2{\mathchoice
              {\setbox1\hbox{${\displaystyle #1}_{\scriptstyle #2}$}
              \restrictionaux{#1}{#2}}
              {\setbox1\hbox{${\textstyle #1}_{\scriptstyle #2}$}
              \restrictionaux{#1}{#2}}
              {\setbox1\hbox{${\scriptstyle #1}_{\scriptscriptstyle #2}$}
              \restrictionaux{#1}{#2}}
              {\setbox1\hbox{${\scriptscriptstyle #1}_{\scriptscriptstyle #2}$}
              \restrictionaux{#1}{#2}}}
\def\restrictionaux#1#2{{#1\,\smash{\vrule height .8\ht1 depth .85\dp1}}_{\,#2}}
%%%%%%%%%%%

%%% FORMATTAZIONE FOOTNOTEMARK

\def\footnotemarkformatting#1{[#1]}
\renewcommand{\thefootnote}{\footnotemarkformatting{\arabic{footnote}}}

%% SEZIONE GRAFICA
\use{tikz}
\usetikzlibrary{matrix, patterns, calc, decorations.pathreplacing, hobby, decorations.markings, decorations.pathmorphing, babel}
\use{tikz-3dplot}
\use{mathrsfs} %per geogebra
\use{tikz-cd}
\tikzset
{
  %surface/.style={fill=black!10, shading=ball,fill opacity=0.4},
  plane/.style={black,pattern=north east lines},
  curve/.style={black,line width=0.5mm},
  dritto/.style={decoration={markings,mark=at position 0.5 with {\arrow{Stealth}}}, postaction=decorate},
  rovescio/.style={decoration={markings,mark=at position 0.5 with {\arrow{Stealth[reversed]}}}, postaction=decorate}
}
\use{pgfplots} % stampare le funzioni
        \pgfplotsset{/pgf/number format/use comma,compat=1.15}
        %\pgfplotsset{compat=1.15} %per geogebra
        \usepgfplotslibrary{fillbetween, polar}
%%%%%%

%% CITAZIONI
\use{lineno}

\newcommand{\citazione}[1]{%
  \begin{quotation}
  \begin{linenumbers}
  \modulolinenumbers[5]
  \begingroup
  \setlength{\parindent}{0cm}
  \noindent #1
  \endgroup
  \end{linenumbers}
  \end{quotation}\setcounter{linenumber}{1}
  }
%%%%%%

%%%%%%%%%%%%%%%%%%%%%%%%%%%%%%%%%%%%%%%%%%%%
%%%%%%%%%%%%%%%%%%%%%%%%%%%%%%%%%%%%%%%%%%%%

%% AMS THM

\theoremstyle{definition}% default
\newtheorem{thm}{Teorema}[section]
\newtheorem{lem}[thm]{Lemma}
\newtheorem{prop}[thm]{Proposizione}
\newtheorem{cor}[thm]{Corollario}
\newtheorem{esempio}[thm]{Esempio}
\theoremstyle{plain}
\newtheorem{definizione}[thm]{Definizione}
\theoremstyle{remark}
\newtheorem*{oss}{Osservazione}


%%%%%%%%%%%%%%%%%%%%%%%%%%%%%%%%%%%%%%%%%%%%
%%%%%%%%%%%%%%%%%%%%%%%%%%%%%%%%%%%%%%%%%%%%

\use{hyperref}
\hypersetup{%
        pdfauthor={Davide Peccioli},
        pdfsubject={},
        allcolors=black,
        citecolor=black,
%	colorlinks=true,
        bookmarksopen=true}
\setcounter{secnumdepth}{0} % rimuove i numeri di sezione senza rimuovere le ref
\renewcommand{\href}[2]{\textcolor{blue}{#2}} % disabilita il comando href
\use{enotez} %
\setenotez{%
 mark-format = \footnotemarkformatting % Mette i numeri tra parentesi quadre%
}\let\footnote=\endnote % rende tutte le note a pié pagina come delle note a fine file 


\let\olddocument\document % modifico l'ambiende documenti per non dover stampare \printendnote
\let\oldenddocument\enddocument
\renewenvironment{document}%
{%
  \olddocument
}{%
  \printendnotes\oldenddocument
}
\renewcommand{\thethm}{\arabic{thm}}

\usepackage[hyperref]{biblatex}
\addbibresource{~/Documents/org/roam/bib/master.bib}
\author{Davide Peccioli}
\date{\today}
\title{}
\begin{document}

\section{Teorema dell'invarianza per omotopia}
\label{sec:org2957cf2}
Sia \(R\) un \href{20241219112842-pid.org}{PID}.
\begin{thm}
Siano \(X,Y\) \href{20250103145124-topologia.org}{spazi topologici}, e siano \(f,g:X\longrightarrow Y\) \href{20250103103252-funzione_continua.org}{funzioni continue} e \href{20250121094654-omotopia_tra_funzioni_continue.org}{omotope}, \(f\circ g\)
Allora, considerando il \href{20241205131958-funtore.org}{funtore} \((\mathcal{S}_{\bullet}, \diesis): \cat{Top}\longrightarrow \cat{Ch}_{R}\)\footnote{Questo è il \href{20250122154136-funtorialita_dell_omologia_singolare.org}{funtore \(\diesis\)} dalla \href{20241205115600-categoria_top.org}{categoria \(\cat{Top}\)} alla \href{20250120163759-categoria_complessi_di_catene.org}{categoria \(\cat{Ch}_{R}\)}.} si ha che \(f_{\diesis}\) e \(g_{\diesis}\) siano \href{20250121094935-omotopia_tra_morfismi_di_complessi_di_catene.org}{omotopiche}:
\begin{equation*}
f_{\diesis}\sim g_{\diesis}: \mathcal{S}_{\bullet}(X)\longrightarrow \mathcal{S}_{\bullet}(Y).
\end{equation*}
\label{thm:invarianzaperomotopia}
\end{thm}
\begin{cor}
Questo implica che, se \((H_{n},\star)\) è il \href{20250123115927-funtore_di_omologia_singolare.org}{funtore di omologia}, allora \(f_{\star}=g_{\star}\).
\end{cor}
\begin{proof}
Infatti, due \href{20250121094935-omotopia_tra_morfismi_di_complessi_di_catene.org}{funzioni omotope tra complessi di catene} \href{20250121100726-funtore_di_omologia_di_funzioni_omotope.org}{danno luogo alla stessa funzione} \href{20250120164857-modulo_di_omologia_dei_complessi_di_catene.org}{in omologia}.
\end{proof}
\begin{lem}
Se \(j^{0}_{\bullet} \sim j^{1}\bullet: \mathcal{S}_{\bullet}(X) \to \mathcal{S}_{\bullet}(X\times I)\) e \(H_{\bullet}: \mathcal{S}_{\bullet}(X\times I)\to \mathcal{S}_{\bullet}(Y)\) sono morfismi, allora
\begin{equation*}
H_{\bullet} \circ j_{0}_{\bullet} \sim H_{\bullet} \circ j^{1}_{\bullet}.
\end{equation*}
\end{lem}
\begin{proof}
Si è nella situazione di questo diagramma:
\begin{equation*}
\begin{tikzcd}
	{S_{q+1}(X)} && {S_q(X)} && {S_{q-1}(X)} \\
	\\
	{S_{q+1}({X\times I})} && {S_q(X\times I)} && {S_{q-1}({X\times I})} \\
	\\
	{S_{q+1}(Y)} && {S_q(Y)} && {S_{q-1}(Y)}
	\arrow["{\partial^X_{q+1}}", from=1-1, to=1-3]
	\arrow["{j^0_{q+1}}"', shift right, from=1-1, to=3-1]
	\arrow["{j^1_{q+1}}", shift left, from=1-1, to=3-1]
	\arrow["{\partial^X_{q}}", from=1-3, to=1-5]
	\arrow["{s_q}", from=1-3, to=3-1]
	\arrow["{j^0_q}"', shift right, from=1-3, to=3-3]
	\arrow["{j^1_q}", shift left, from=1-3, to=3-3]
	\arrow["{\ell_q}"{description, pos=0.8}, color={rgb,255:red,214;green,92;blue,92}, dashed, from=1-3, to=5-1]
	\arrow["{s_{q-1}}", from=1-5, to=3-3]
	\arrow["{j^0_{q-1}}"', shift right, from=1-5, to=3-5]
	\arrow["{j^1_{q-1}}", shift left, from=1-5, to=3-5]
	\arrow["{\ell_{q-1}}"{description, pos=0.8}, color={rgb,255:red,214;green,92;blue,92}, dashed, from=1-5, to=5-3]
	\arrow["{\partial^{X\times I}_{q+1}}"{pos=0.3}, from=3-1, to=3-3]
	\arrow["{H_{q+1}}"', from=3-1, to=5-1]
	\arrow["{\partial^{X\times I}_{q}}"{pos=0.4}, from=3-3, to=3-5]
	\arrow["{H_q}"', from=3-3, to=5-3]
	\arrow["{H_{q-1}}", from=3-5, to=5-5]
	\arrow["{\partial^Y_{q+1}}", from=5-1, to=5-3]
	\arrow["{\partial^Y_{q}}", from=5-3, to=5-5]
\end{tikzcd}
\end{equation*}
Ponendo \(\ell_{q} \coloneqq H_{q+1}\circ s_{q}\), si ha la tesi:
\begin{align*}
l_{q-1} \circ \partial_q^X + \partial_{q+1}^Y \circ l_q %
&= H_q \circ s_{q-1} \circ \partial_q^X + \partial_{q+1}^Y \circ H_{q+1} \circ s_q \\
&= H_q \circ s_{q-1} \circ \partial_q^X + H_q \circ \partial_{q+1}^{X \times I} \circ s_q \\
&= H_q \left[ s_{q-1} \circ \partial_q^X + \partial_{q+1}^{X \times I} \circ s_q \right] \\
&= H_q \left[ j_q^1 - j_q^0 \right] = H_q \circ j_q^1 - H_q \circ j_q^0 \qedhere
\end{align*}
\end{proof}
\begin{proof}
(idea del Teorema~\ref{thm:invarianzaperomotopia}, da dimostrare per intero).
\begin{equation*}
\begin{tikzcd}
	X &&&&&& {\mathcal{S}_\bullet(X)} \\
	& {X\times I} && Y &&& {} & {\mathcal{S}_\bullet(X\times I)} && {\mathcal{S}_\bullet(Y)} \\
	X &&&&&& {\mathcal{S}_\bullet(X)}
	\arrow["{j^0}"', from=1-1, to=2-2]
	\arrow["f", from=1-1, to=2-4]
	\arrow["{j^0_\diesis}"', from=1-7, to=2-8]
	\arrow["{f_\diesis}", from=1-7, to=2-10]
	\arrow["H"{description}, from=2-2, to=2-4]
	\arrow["{(\mathcal{S}_\bullet, \diesis)}", color={rgb,255:red,214;green,92;blue,92}, squiggly, from=2-4, to=2-7]
	\arrow["{H_\diesis}"{description}, from=2-8, to=2-10]
	\arrow["{j^1}", from=3-1, to=2-2]
	\arrow["g"', from=3-1, to=2-4]
	\arrow["{j^1_\diesis}", from=3-7, to=2-8]
	\arrow["{g_\diesis}"', from=3-7, to=2-10]
\end{tikzcd}
\end{equation*}

Per il lemma è sufficiente mostrare:
\begin{equation*}
j^\#_0 \sim j^\#_1
\end{equation*}

i.e. \(s_n : S_n(X) \to S_{n+1}(X \times I)\) si costruisce \textbf{\textbf{per induzione}}.

\textbf{\textbf{Passo base:}} \(s_0 : S_0(X) \to S_1(X \times I)\)
\begin{align*}
\sigma \in \Sigma_0(X) \longmapsto s_0(\sigma) : \Delta_1 &\longrightarrow X \times I \\
(1-t)e_0 + t e_1 &\longmapsto (\sigma(e_0), t)
\end{align*}
FUNZIONA.

\textbf{\textbf{Ip. induttive:}} \(s_{n-1}, \dots, s_0 \quad \forall X \text{ sp. top.}\)

\begin{enumerate}
\item \(X = \Delta_n\), \(i_n \in \Sigma_n(\Delta_n)\), \(i_n\) identità.
\begin{equation*}
z_n := j^\#_1(i_n) - j^\#_0(i_n) - s_{n-1}(\partial i_n)
\end{equation*}
\(\partial z_n = 0\) + \(H_n(\Delta_n \times I) = 0\) implica che
\begin{equation*}
  \exists \beta_{n+1} \in S_{n+1}(X \times I) \text{ t.c. } \partial \beta_{n+1} = z_n
\end{equation*}

\item \(X\) generico, \(\sigma \in \Sigma_n(X)\)
Si definisce \(s_{n}: S_{n}(X) \to S_{n+1}(X\times I)\):
\begin{equation*}
 	\sigma \in \Sigma_n(X) \longmapsto (\sigma \times \Id)_\# \beta_{n+1}
\end{equation*}

Calcolando \(\partial_{n+1} s_n \sigma\) si ha la tesi ricordando:
\begin{equation*}
\begin{tikzcd}
        {\Delta_n\times I} && {X\times I} &&& {\mathcal{S}_\bullet (\Delta_n\times I)} && {\mathcal{S}_\bullet (X\times I)} \\
        &&& {} & {} \\
        {\Delta_n} && X &&& {\mathcal{S}_\bullet (\Delta_n)} && {\mathcal{S}_\bullet (X)}
        \arrow["{\sigma\times\Id}", from=1-1, to=1-3]
        \arrow["{(\sigma\times\Id)_\diesis}", from=1-6, to=1-8]
        \arrow[squiggly, from=2-4, to=2-5]
        \arrow["{j^0}", from=3-1, to=1-1]
        \arrow["\sigma"', from=3-1, to=3-3]
        \arrow["{j^0}"', from=3-3, to=1-3]
        \arrow["{j^0_\diesis}", from=3-6, to=1-6]
        \arrow["{\sigma_\diesis}"', from=3-6, to=3-8]
        \arrow["{j^0_\diesis}"', from=3-8, to=1-8]
\end{tikzcd}
\end{equation*}\qedhere
\end{enumerate}
\end{proof}
\subsection{Spazi topologici omotopicamente equivalenti hanno moduli di omologia singolare isomorfi}
\label{sec:org7376a0b}
Sia \(R\) un \href{20241219112842-pid.org}{PID} e siano \(X,Y\) spazi topologici.
\begin{cor}
Se \(X,Y\) sono \href{20250124155008-spazi_topologici_omotopicamente_equivalenti.org}{spazi topologici omotopicamente equivalenti}, allora per ogni \(n\) i \href{20241205141053-r_moduli.org}{moduli} di \href{20250122133631-omologia_singolare.org}{omologia singolare} sono \href{20241206115416-morfismi_r_moduli.org}{isomorfi}:
\begin{equation*}
H_{n}(X)\cong H_{n}(Y)
\end{equation*}

In particolare, se \(f,g\) \href{20250124155008-spazi_topologici_omotopicamente_equivalenti.org}{equivalenze omotopiche},
\begin{equation*}
\begin{tikzcd}[ampersand replacement=\&,cramped,column sep=3.15em]
	X \& Y
	\arrow["f", shift left=2, from=1-1, to=1-2]
	\arrow["g", shift left=2, from=1-2, to=1-1]
\end{tikzcd}
\end{equation*}
allora \(f_{\star}, g_{\star}\) sono \href{20241206115416-morfismi_r_moduli.org}{isomorfismi} (applicando il \href{20241205131958-funtore.org}{funtore} \href{20250123115927-funtore_di_omologia_singolare.org}{di omologia singolare} \((H_{n},\star)\))
\end{cor}
\begin{proof}
Siccome \(f,g\) sono equivalenze omotopiche, allora
\begin{equation*}
f \circ g \sim \Id_{X},\qquad g\circ f \sim \Id_{Y}.
\end{equation*}
Applicando il \href{20250123115927-funtore_di_omologia_singolare.org}{funtore di omologia singolare} \hyperref[sec:org2957cf2]{si ottiene}
\begin{equation*}
\Id_{H_{n}(X)} = H_{n}(f\circ g) = H_{n}(f) \circ H_{n}(g)
\end{equation*}
e pertanto \(f_{\star}\) e \(g_{\star}\) sono inverse, \(f_{\star}\) è invertibile e quindi un isomorfismo.
\end{proof}
\end{document}
