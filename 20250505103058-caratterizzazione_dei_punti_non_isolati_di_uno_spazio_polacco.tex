% Created 2026-02-07 Sat 19:33
% Intended LaTeX compiler: pdflatex
\documentclass[10pt]{article}
%% CREATO CON ORG - EMACS
\newcommand{\use}[2][]{\usepackage[#1]{#2}}
% PACCHETTI FONDAMENTLAI
\use[utf8]{inputenc}
\use[T1]{fontenc}
\use{graphicx}
\use{longtable}
\use{wrapfig}
\use{rotating}
\use[normalem]{ulem}
\use{amsmath}
\use{amsthm}
\use{amssymb}

\use{eucal} % Cambia mathcal{...}

\use{capt-of}
\use[italian]{babel}
\use[babel]{csquotes}
% bib la TEX lo carica in automatico org-cite
\use{microtype}
\use{lmodern}
\use{subfig} % sottofigure
\use{multicol} % due colonne
\use{lipsum} % lorem ipsum
\use{color} % colori in latex
\use{parskip} % rimuove l'indentazione dei nuovi paragrafi %% Add parbox=false to all new tcolorbox
\use{centernot}
\use[outline]{contour}\contourlength{3pt}
\use{fancyhdr}
\use{layout}
\use[most]{tcolorbox} % Riquadri colorati
\use{ifthen} % IFTHEN
\use{geometry}

% pacchetti matematica
\use{yhmath}
\use{dsfont}
\use{mathrsfs}
\use{cancel} % semplificare
\use{polynom} %divisione tra polinomi
\use{forest} % grafi ad albero
\use{booktabs} % tabelle
\use{commath} %simboli e differenziali
\use{bm} %bold
\use[fulladjust]{marginnote} %to use marginnote for date notes
\use{arrayjobx}%array
\use[intlimits]{empheq} % Riquadri colorati attorno alle equazioni
\use{mathtools}
\use{circuitikz} % Disegnare i circuiti
\use{mathtools}
\use{stmaryrd} % [[ \llbracket ]] \rrbracket
\use{bussproofs} % dimostrazioni

%%%%%%%%%%%%%


%%%% QUIVER
\newcommand{\duepunti}{\,\mathchar\numexpr"6000+`:\relax\,}
% A TikZ style for curved arrows of a fixed height, due to AndréC.
\tikzset{curve/.style={settings={#1},to path={(\tikztostart)
    .. controls ($(\tikztostart)!\pv{pos}!(\tikztotarget)!\pv{height}!270:(\tikztotarget)$)
    and ($(\tikztostart)!1-\pv{pos}!(\tikztotarget)!\pv{height}!270:(\tikztotarget)$)
    .. (\tikztotarget)\tikztonodes}},
    settings/.code={\tikzset{quiver/.cd,#1}
        \def\pv##1{\pgfkeysvalueof{/tikz/quiver/##1}}},
    quiver/.cd,pos/.initial=0.35,height/.initial=0}

% TikZ arrowhead/tail styles.
\tikzset{tail reversed/.code={\pgfsetarrowsstart{tikzcd to}}}
\tikzset{2tail/.code={\pgfsetarrowsstart{Implies[reversed]}}}
\tikzset{2tail reversed/.code={\pgfsetarrowsstart{Implies}}}
% TikZ arrow styles.
\tikzset{no body/.style={/tikz/dash pattern=on 0 off 1mm}}
%%%%%%%%%%


%% DEFINIZIONI COMANDI MATEMATICI
\let\sin\relax %TOGLIE LA DEFINIZIONE SU "\sin"

% cambia la definizione di empty set
% ---
\let\oldemptyset\emptyset
% ---
% \let\emptyset\varnothing
% ---
% \let\emptyset\relax
% \newcommand{\emptyset}{\text{\textnormal{\O}}}
% ---

\DeclareMathOperator{\bounded}{bd}
\DeclareMathOperator{\sin}{sen}
\DeclareMathOperator{\epi}{Epi}
\DeclareMathOperator{\cl}{cl}
\DeclareMathOperator{\graph}{graph}
\DeclareMathOperator{\arcsec}{arcsec}
\DeclareMathOperator{\arccot}{arccot}
\DeclareMathOperator{\arccsc}{arccsc}
\DeclareMathOperator{\spettro}{Spettro}
\DeclareMathOperator{\nulls}{nullspace}
\DeclareMathOperator{\dom}{dom}
\DeclareMathOperator{\ar}{ar}
\DeclareMathOperator{\const}{Const}
\DeclareMathOperator{\fun}{Fun}
\DeclareMathOperator{\rel}{Rel}
\DeclareMathOperator{\altezza}{ht}
\let\det\relax %TOGLIE LA DEFINIZIONE SU "\det"
\DeclareMathOperator{\det}{det}
\DeclareMathOperator{\End}{End}
\DeclareMathOperator{\gl}{GL}
\def\Id{\mathrm{Id}}
\def\id{\mathrm{id}}
\DeclareMathOperator{\I}{\mathds{1}}
\DeclareMathOperator{\II}{II}
\DeclareMathOperator{\rank}{rank}
\DeclareMathOperator{\tr}{tr}
\DeclareMathOperator{\tc}{t.c.}
\DeclareMathOperator{\T}{T}
\DeclareMathOperator{\var}{Var}
\DeclareMathOperator{\cov}{Cov}
\DeclareMathOperator{\st}{st}
\DeclareMathOperator{\mon}{Mon}
\newcommand{\card}[1]{\left\vert #1 \right\vert}
\newcommand{\trasposta}[1]{\prescript{\text{T}}{}{#1}}
\newcommand{\1}{\mathds{1}}
\newcommand{\R}{\mathds{R}}
\newcommand{\diesis}{\#}
\newcommand{\bemolle}{\flat}
\newcommand{\nonstandard}[1]{\prescript{*}{}{#1}}
\newcommand{\starR}{\nonstandard{\R}}
\newcommand{\borel}{\mathscr{B}}
\newcommand{\lebesgue}[1]{\mathscr{L}\left(#1\right)}
\newcommand{\media}{\mathds{E}}
\newcommand{\K}{\mathds{K}}
\newcommand{\A}{\mathds{A}}
\newcommand{\Q}{\mathds{Q}}
\newcommand{\N}{\mathds{N}}
\newcommand{\C}{\mathds{C}}
\newcommand{\Z}{\mathds{Z}}
\newcommand{\qo}{\hspace{1em}\text{q.o.}\,}
\renewcommand{\tilde}[1]{\widetilde{#1}}
\renewcommand{\parallel}{\mathrel{/\mkern-5mu/}}
\newcommand{\parti}[2][]{\wp_{#1}(#2)}
\newcommand{\diff}[1]{\operatorname{d}_{#1}}
\let\oldvec\vec
\renewcommand{\vec}[1]{\overrightarrow{\vphantom{i}#1}}
\newcommand{\floor}[1]{\left\lfloor #1 \right\rfloor}
\newcommand{\cat}[1]{\mathbf{#1}}
\newcommand{\dfreccia}[1]{\xrightarrow{\ #1 \ }}
\newcommand{\sfreccia}[1]{\xleftarrow{\ #1 \ }}
\newcommand{\formalsum}[2]{{\sum_{#1}^{#2}}{\vphantom{\sum}}'}
\newcommand{\minim}[2]{\mu_{#1}\, \left(#2\right)}
\newcommand{\concat}{\null^{\frown}} % concatenazione di stringe
\newcommand{\godelcode}[1]{\langle\!\langle #1 \rangle\!\rangle}
\newcommand{\godeldec}[1]{(\!(#1)\!)}
\newcommand{\termcode}[1]{\ulcorner #1\urcorner}
\newcommand{\partialto}{\dashrightarrow}
\newcommand{\restricted}{\upharpoonright}
\newcommand{\embeds}{\precsim}
\newcommand{\surjects}{\twoheadrightarrow}
\newcommand{\equipotenti}{\asymp}
%% \newcommand{\dotplus}{\mathbin{\dot{+}}} %% A quanto pare esiste già
\newcommand{\bigdot}{\mathbin{\boldsymbol{\cdot}}}
\newcommand{\dotexp}[1]{^{.#1}}
\newcommand{\conv}{\mathbin{*}}
\newcommand{\convolution}[2]{(#1\conv #2)}
\newcommand{\nil}{\mathfrak{N}}
\newcommand{\divisore}{\mathrel{|}}
\newcommand{\simplesso}[1]{\mathrm{e}_{#1}}

\renewcommand{\iff}{\mathrel{\longleftrightarrow}} %% Notazione Logica.
\newcommand{\oldiff}{\mathrel{\Longleftrightarrow}}
\renewcommand{\implies}{\mathrel{\rightarrow}} %% Notazione Logica
\newcommand{\oldimplies}{\mathrel{\Longrightarrow}}
\renewcommand{\impliedby}{\mathrel{\leftarrow}} %% Notazione Logica
\newcommand{\oldimpliedby}{\mathrel{\Longleftarrow}}

\newcommand{\IFF}{\quad\Longleftrightarrow\quad}
\newcommand{\IMPLICA}{\quad\Longrightarrow\quad}


\renewcommand{\descriptionlabel}[1]{\hspace{\labelsep}\normalfont #1} % remove bold from description


%% Definizione di Divergenza di K-L

\DeclarePairedDelimiterX{\infdivx}[2]{(}{)}{%
  #1\;\delimsize\|\;#2%
}
\newcommand{\kldiv}{D_{KL}\infdivx}

%% Definizione di \dotminus

\makeatletter
\newcommand{\dotminus}{\mathbin{\text{\@dotminus}}}

\newcommand{\@dotminus}{%
  \ooalign{\hidewidth\raise1ex\hbox{.}\hidewidth\cr$\m@th-$\cr}%
}
\makeatother

%tramite i prossimi due comandi posso decidere come scrivere i logaritmi naturali in tutti i documenti: ho infatti eliminato qualsiasi differenza tra "ln" e "log": se si vuole qualcosa di diverso bisogna inserire manualmente il tutto
\let\ln\relax
\DeclareMathOperator{\ln}{ln}
\let\log\relax
\DeclareMathOperator{\log}{log}
%%%%%%

%% NUOVI COMANDI
\newcommand{\straniero}[1]{\textit{#1}} %parole straniere
\newcommand{\titolo}[1]{\textsc{#1}} %titoli
\newcommand{\qedd}{\tag*{$\blacksquare$}} %qed per ambienti matemastici
\renewcommand{\qedsymbol}{$\blacksquare$} %modifica colore qed
\newcommand{\ooverline}[1]{\overline{\overline{#1}}}
\newcommand{\circoletto}[1]{\left(#1\right)^{\text{o}}}
%
\newcommand{\qmatrice}[1]{\begin{pmatrix}
#1_{11} & \cdots & #1_{1n}\\
\vdots & \ddots & \vdots \\
#1_{m1} & \cdots & #1_{mn}
\end{pmatrix}}
%
\newcommand{\parentesi}[2]{%
\underset{#1}{\underbrace{#2}}%
}
%
\newcommand{\norma}[1]{% Norma
\left\lVert#1\right\rVert%
}
\newcommand{\scalare}[2]{% Scalare
\left\langle #1, #2\right\rangle
}
%%%%%

%% RESTRIZIONI
\newcommand{\referenze}[2]{
        \phantomsection{}#2\textsuperscript{\textcolor{blue}{\textbf{#1}}}
}

\let\restriction\relax

\def\restriction#1#2{\mathchoice
              {\setbox1\hbox{${\displaystyle #1}_{\scriptstyle #2}$}
              \restrictionaux{#1}{#2}}
              {\setbox1\hbox{${\textstyle #1}_{\scriptstyle #2}$}
              \restrictionaux{#1}{#2}}
              {\setbox1\hbox{${\scriptstyle #1}_{\scriptscriptstyle #2}$}
              \restrictionaux{#1}{#2}}
              {\setbox1\hbox{${\scriptscriptstyle #1}_{\scriptscriptstyle #2}$}
              \restrictionaux{#1}{#2}}}
\def\restrictionaux#1#2{{#1\,\smash{\vrule height .8\ht1 depth .85\dp1}}_{\,#2}}
%%%%%%%%%%%

%%% FORMATTAZIONE FOOTNOTEMARK

\def\footnotemarkformatting#1{[#1]}
\renewcommand{\thefootnote}{\footnotemarkformatting{\arabic{footnote}}}

%% SEZIONE GRAFICA
\use{tikz}
\usetikzlibrary{matrix, patterns, calc, decorations.pathreplacing, hobby, decorations.markings, decorations.pathmorphing, babel}
\use{tikz-3dplot}
\use{mathrsfs} %per geogebra
\use{tikz-cd}
\tikzset
{
  %surface/.style={fill=black!10, shading=ball,fill opacity=0.4},
  plane/.style={black,pattern=north east lines},
  curve/.style={black,line width=0.5mm},
  dritto/.style={decoration={markings,mark=at position 0.5 with {\arrow{Stealth}}}, postaction=decorate},
  rovescio/.style={decoration={markings,mark=at position 0.5 with {\arrow{Stealth[reversed]}}}, postaction=decorate}
}
\use{pgfplots} % stampare le funzioni
        \pgfplotsset{/pgf/number format/use comma,compat=1.15}
        %\pgfplotsset{compat=1.15} %per geogebra
        \usepgfplotslibrary{fillbetween, polar}
%%%%%%

%% CITAZIONI
\use{lineno}

\newcommand{\citazione}[1]{%
  \begin{quotation}
  \begin{linenumbers}
  \modulolinenumbers[5]
  \begingroup
  \setlength{\parindent}{0cm}
  \noindent #1
  \endgroup
  \end{linenumbers}
  \end{quotation}\setcounter{linenumber}{1}
  }
%%%%%%

%%%%%%%%%%%%%%%%%%%%%%%%%%%%%%%%%%%%%%%%%%%%
%%%%%%%%%%%%%%%%%%%%%%%%%%%%%%%%%%%%%%%%%%%%

%% AMS THM

\theoremstyle{definition}% default
\newtheorem{thm}{Teorema}[section]
\newtheorem{lem}[thm]{Lemma}
\newtheorem{prop}[thm]{Proposizione}
\newtheorem{cor}[thm]{Corollario}
\newtheorem{esempio}[thm]{Esempio}
\theoremstyle{plain}
\newtheorem{definizione}[thm]{Definizione}
\theoremstyle{remark}
\newtheorem*{oss}{Osservazione}


%%%%%%%%%%%%%%%%%%%%%%%%%%%%%%%%%%%%%%%%%%%%
%%%%%%%%%%%%%%%%%%%%%%%%%%%%%%%%%%%%%%%%%%%%

\use{hyperref}
\hypersetup{%
        pdfauthor={Davide Peccioli},
        pdfsubject={},
        allcolors=black,
        citecolor=black,
%	colorlinks=true,
        bookmarksopen=true}
\setcounter{secnumdepth}{0} % rimuove i numeri di sezione senza rimuovere le ref
\renewcommand{\href}[2]{\textcolor{blue}{#2}} % disabilita il comando href
\use{enotez} %
\setenotez{%
 mark-format = \footnotemarkformatting % Mette i numeri tra parentesi quadre%
}\let\footnote=\endnote % rende tutte le note a pié pagina come delle note a fine file 


\let\olddocument\document % modifico l'ambiende documenti per non dover stampare \printendnote
\let\oldenddocument\enddocument
\renewenvironment{document}%
{%
  \olddocument
}{%
  \printendnotes\oldenddocument
}
\renewcommand{\thethm}{\arabic{thm}}

\usepackage[hyperref]{biblatex}
\addbibresource{~/Documents/org/roam/bib/master.bib}
\author{Davide Peccioli}
\date{\today}
\title{Caratterizzazione dei punti non isolati di uno spazio polacco}
\begin{document}

\section{Proposizione}
\label{sec:org0c5df2d}

Prove that for any \href{20250301194013-spazio_polacco.org}{Polish space} \(X\) and \(x \in X\), the singleton \(\{x\}\) is \(\bm{\Pi}^0_1\)-complete if and only if \(x\) is not \href{20250403131856-punto_isolato.org}{isolated} in \(X\). Conclude that the set
\[
  C_1 = \{x \in 2^\omega \mid \exists n \ (x(n) = 0)\}
\]
from Proposition 2.1.31 of the notes is \(\bm{\Sigma}^0_1\)-complete.
\subsection{Dimostrazione}
\label{sec:org26227de}

Siccome \(X\) è uno spazio metrizzabile, allora \(\set{x} \subseteq X\) è chiuso, e pertanto \(\set{x} \in \bm{\Pi_{1}}^{0}(X)\). Bisogna quindi dimostrare che \(\set{x}\) è \(\bm{\Pi}^{0}_{1}\)-hard sse \(x\) è \textbf{non isolato} in \(X\).
\subsubsection{Implicazione ``\(\implies\)''}
\label{sec:orgee70b04}

Sia \(C \in \bm{\Pi}_{1}^{0}(\omega^{\omega})\), e sia \(f: \omega^{\omega}\to X\) continua tale che
\begin{equation*}
f^{-1}(x) = C.
\end{equation*}

Si supponga per assurdo che \(x\) sia isolato. Allora \(\set{x} \subseteq X\) è aperto, e quindi \(C \subseteq \omega^{\omega}\) è aperto (retroimmagine continua di un aperto).

Per l'arbitrarietà di \(C\), questo implica che ogni chiuso di \(\omega^{\omega}\) è un clopen. Inoltre, se \(A \subseteq \omega^{\omega}\) è aperto, allora \(\omega^{\omega}\setminus A\) è chiuso e quindi clopen, e pertanto \(A\) è un chiuso:
\begin{equation*}
\bm{\Sigma}_{1}^{0}(\omega^{\omega}) = \bm{\Delta}_{1}^{0}(\omega^{\omega}) = \bm{\Pi}_{1}^{0}(\omega^{\omega}).
\end{equation*}
Questo contraddice il Theorem 2.1.17 delle note.
\subsubsection{Implicazione ``\(\impliedby\)''\hfill{}\textsc{Modificato}}
\label{sec:org0704083}

Sia \(x \in X\) un punto non isolato, ovvero \(x\) un punto di accumulazione di \(X\), e sia \(B \in \bm{\Pi}_{1}^{0}(\omega^{\omega})\).

\begin{itemize}
\item Si fissi \(d:X\to \R\) una metrica completa su \(X\).
\item Siccome \(x\) è un punto di accumulazione di \(X\), allora esiste una successione \((y_{n})_{n \in \omega} \subseteq X\setminus\set{x}\) tale che \(y_{n}\to x\), ovvero, per ogni intorno \(U\) di \(x\) esiste \(N \in \N\) tale che, per ogni \(j\ge N\), \(y_{j} \in U\).

\item Si costruisce \(\set{U_{n}}_{n \in \omega}\) una famiglia di aperti di \(X\) tali che
\begin{itemize}
\item per ogni \(n \in \omega\): \(U_{n}\setminus \set{x}\neq \emptyset\);
\item l'intersezione \(\bigcap_{n \in \omega} U_{n} = \set{x}\);
\item \(\operatorname{diam}(U_{n})\to 0\);
\item per ogni \(n \in \omega\): \(\operatorname{Cl}(U_{n+1}) \subsetneqq U_{n}\)
\end{itemize}
e una successione \(v_{n} \subseteq X\setminus \set{x}\) tale che \(v_{n} \in U_{n}\setminus \operatorname{Cl}(U_{n+1})\).

Sia \(U_{0} = X\). Si supponga di aver costruito \(U_{n}\), e sia \(\alpha \in U_{n}\setminus\set{x}\). Tale \(\alpha\) esiste, poiché esistono infiniti elementi della successione \((y_{j})_{j \in \omega}\) dentro \(U_{n}\) intorno di \(x\).

Detto \(r\coloneqq\min\set{2^{-n-1}, d(x,\alpha)/2}>0\), sia \(U_{n}'\coloneqq B_{d}(x,r)\). Necessariamente \(\alpha\notin U_{n}'\) e \(U_{n}' \subsetneqq U_{n}\).

È quindi possibile porre \(U_{n+1}\coloneqq B_{d}(x,r/2)\):
\begin{equation*}
  	\operatorname{Cl}(U_{n+1}) = \operatorname{Cl} \left(B_{d}(x,r/2)\right) \subseteq B_{d}^{\text{cl}}(x,r/2) \subseteq B_{d}(x,r) = U_{n}' \subsetneqq U_{n}.
\end{equation*}

Si ponga inoltre \(v_{n} \coloneqq \alpha\), \(v_{n} \in U_{n}\setminus \operatorname{Cl}(U_{n+1})\).

Questa famiglia soddisfa tutte le proprietà elencate.
\end{itemize}

\begin{itemize}
\item Siccome \(B\) è un chiuso di \(\omega^{\omega}\), allora esiste un albero potato \(T \subseteq \omega^{<\omega}\) tale che \(B=[T]\), i.e.
\begin{equation*}
  	B = \set{\alpha \in \omega^{\omega}\mid \forall\,n \in \omega\ (\alpha\upharpoonright n \in T)}
\end{equation*}
\item Si costruisce un \(\omega\)-schema \(\set{B_{s}\mid s \in \omega^{<\omega}}\) su \(X\):
\begin{itemize}
\item se \(s \in T\), allora \(B_{s} \coloneqq U_{\operatorname{lh}(s)}\); in particolare, quindi \(\emptyset \in T\) e \(B_{\emptyset} = U_{0} = X\);
\item se \(s\notin T\), sia \(j_{s}\) il più grande indice tale che \(s\upharpoonright j_{s} \in T\); si pone \(B_{s} \coloneqq \set{v_{j_{s}}}\).
\end{itemize}
\item Questo definisce effettivamente uno schema tale che \(\operatorname{Cl}(B_{s\concat a}) \subseteq B_{s}\) e ciascun \(B_{s}\neq \emptyset\): pertanto è indotta una funzione continua totale (per il Lemma 1.3.6)
\begin{equation*}
  	F:\omega^{\omega}\to X
\end{equation*}
\item Resta da mostrare che \(F^{-1}(x) = B\). Questo per definizione garantisce che \(\set{x}\) sia un \(\bm{\Pi}_{1}^{0}\)-hard.

Per ogni \(\beta \in B\),
\begin{equation*}
  	F(\beta) \in \bigcap_{n \in \omega} B_{\beta\upharpoonright n}
\end{equation*}
dove \(\beta\upharpoonright n \in T\). Quindi \(B_{\beta\upharpoonright n} = U_{n}\). Quindi
\begin{equation*}
  	F(\beta) \in \bigcap_{n \in \omega} U_{n} = \set{x}.
\end{equation*}

Viceversa, se \(\beta \notin B\), allora esiste \(n_{0} \in \omega\) tale che \(\beta\upharpoonright n_{0} \notin T\) e pertanto
\begin{equation*}
  	F(\beta) \in \bigcap_{n \in\omega} B_{\beta\upharpoonright n} \subseteq B_{\beta\upharpoonright n_{0}}
\end{equation*}
e per costruzione \(x\notin B_{\beta\upharpoonright n_{0}}\).
\end{itemize}
\subsubsection{Insieme \(C_{1}\)}
\label{sec:org64dca7c}

Dal momento che \(\bm{\Sigma}_{1}^{0} = \check{\bm{\Pi}}_{1}^{0}\) segue che \(C_{1}\) è \(\bm{\Sigma}_{1}^{0}\)-completo se e solo se \(2^{\omega}\setminus C_1\) è \(\bm{\Pi}_{1}^{0}\)-completo.

Si ha che \(x \in 2^{\omega}\setminus C_{1}\) se e solo se per ogni \(n \in \omega\), \(x(n)\neq 0\), ovvero \(x(n)=1\).

Pertanto \(2^{\omega}\setminus C_{1} =  \set{u}\), dove
\begin{align*}
u: \omega &\longrightarrow 2\\
n &\longmapsto 1
\end{align*}

Per la caratterizzazione di cui sopra, \(C_{1}\) è \(\bm{\Sigma}_{1}^{0}\)-completo se e solo se \(u\) non è un punto isolato di \(2^{\omega}\).

Si consideri ora la successione \((x_{n})_{n \in \omega} \subseteq 2^{\omega}\):
\begin{equation*}
x_{n}(j) = \begin{cases}
1 & j<n\\
0 &j\ge n
\end{cases}
\end{equation*}
Si ha che \(x_{n}\to u\), e pertanto \(u\) non è un punto isolato di \(2^{\omega}\) (per ogni intorno \(I\) di \(u\) esiste \(N \in \omega\) tale che \(x_{N} \in I\setminus\set{u}\)). \qed
\end{document}
