% Created 2026-02-07 Sat 19:35
% Intended LaTeX compiler: pdflatex
\documentclass[10pt]{article}
%% CREATO CON ORG - EMACS
\newcommand{\use}[2][]{\usepackage[#1]{#2}}
% PACCHETTI FONDAMENTLAI
\use[utf8]{inputenc}
\use[T1]{fontenc}
\use{graphicx}
\use{longtable}
\use{wrapfig}
\use{rotating}
\use[normalem]{ulem}
\use{amsmath}
\use{amsthm}
\use{amssymb}

\use{eucal} % Cambia mathcal{...}

\use{capt-of}
\use[italian]{babel}
\use[babel]{csquotes}
% bib la TEX lo carica in automatico org-cite
\use{microtype}
\use{lmodern}
\use{subfig} % sottofigure
\use{multicol} % due colonne
\use{lipsum} % lorem ipsum
\use{color} % colori in latex
\use{parskip} % rimuove l'indentazione dei nuovi paragrafi %% Add parbox=false to all new tcolorbox
\use{centernot}
\use[outline]{contour}\contourlength{3pt}
\use{fancyhdr}
\use{layout}
\use[most]{tcolorbox} % Riquadri colorati
\use{ifthen} % IFTHEN
\use{geometry}

% pacchetti matematica
\use{yhmath}
\use{dsfont}
\use{mathrsfs}
\use{cancel} % semplificare
\use{polynom} %divisione tra polinomi
\use{forest} % grafi ad albero
\use{booktabs} % tabelle
\use{commath} %simboli e differenziali
\use{bm} %bold
\use[fulladjust]{marginnote} %to use marginnote for date notes
\use{arrayjobx}%array
\use[intlimits]{empheq} % Riquadri colorati attorno alle equazioni
\use{mathtools}
\use{circuitikz} % Disegnare i circuiti
\use{mathtools}
\use{stmaryrd} % [[ \llbracket ]] \rrbracket
\use{bussproofs} % dimostrazioni

%%%%%%%%%%%%%


%%%% QUIVER
\newcommand{\duepunti}{\,\mathchar\numexpr"6000+`:\relax\,}
% A TikZ style for curved arrows of a fixed height, due to AndréC.
\tikzset{curve/.style={settings={#1},to path={(\tikztostart)
    .. controls ($(\tikztostart)!\pv{pos}!(\tikztotarget)!\pv{height}!270:(\tikztotarget)$)
    and ($(\tikztostart)!1-\pv{pos}!(\tikztotarget)!\pv{height}!270:(\tikztotarget)$)
    .. (\tikztotarget)\tikztonodes}},
    settings/.code={\tikzset{quiver/.cd,#1}
        \def\pv##1{\pgfkeysvalueof{/tikz/quiver/##1}}},
    quiver/.cd,pos/.initial=0.35,height/.initial=0}

% TikZ arrowhead/tail styles.
\tikzset{tail reversed/.code={\pgfsetarrowsstart{tikzcd to}}}
\tikzset{2tail/.code={\pgfsetarrowsstart{Implies[reversed]}}}
\tikzset{2tail reversed/.code={\pgfsetarrowsstart{Implies}}}
% TikZ arrow styles.
\tikzset{no body/.style={/tikz/dash pattern=on 0 off 1mm}}
%%%%%%%%%%


%% DEFINIZIONI COMANDI MATEMATICI
\let\sin\relax %TOGLIE LA DEFINIZIONE SU "\sin"

% cambia la definizione di empty set
% ---
\let\oldemptyset\emptyset
% ---
% \let\emptyset\varnothing
% ---
% \let\emptyset\relax
% \newcommand{\emptyset}{\text{\textnormal{\O}}}
% ---

\DeclareMathOperator{\bounded}{bd}
\DeclareMathOperator{\sin}{sen}
\DeclareMathOperator{\epi}{Epi}
\DeclareMathOperator{\cl}{cl}
\DeclareMathOperator{\graph}{graph}
\DeclareMathOperator{\arcsec}{arcsec}
\DeclareMathOperator{\arccot}{arccot}
\DeclareMathOperator{\arccsc}{arccsc}
\DeclareMathOperator{\spettro}{Spettro}
\DeclareMathOperator{\nulls}{nullspace}
\DeclareMathOperator{\dom}{dom}
\DeclareMathOperator{\ar}{ar}
\DeclareMathOperator{\const}{Const}
\DeclareMathOperator{\fun}{Fun}
\DeclareMathOperator{\rel}{Rel}
\DeclareMathOperator{\altezza}{ht}
\let\det\relax %TOGLIE LA DEFINIZIONE SU "\det"
\DeclareMathOperator{\det}{det}
\DeclareMathOperator{\End}{End}
\DeclareMathOperator{\gl}{GL}
\def\Id{\mathrm{Id}}
\def\id{\mathrm{id}}
\DeclareMathOperator{\I}{\mathds{1}}
\DeclareMathOperator{\II}{II}
\DeclareMathOperator{\rank}{rank}
\DeclareMathOperator{\tr}{tr}
\DeclareMathOperator{\tc}{t.c.}
\DeclareMathOperator{\T}{T}
\DeclareMathOperator{\var}{Var}
\DeclareMathOperator{\cov}{Cov}
\DeclareMathOperator{\st}{st}
\DeclareMathOperator{\mon}{Mon}
\newcommand{\card}[1]{\left\vert #1 \right\vert}
\newcommand{\trasposta}[1]{\prescript{\text{T}}{}{#1}}
\newcommand{\1}{\mathds{1}}
\newcommand{\R}{\mathds{R}}
\newcommand{\diesis}{\#}
\newcommand{\bemolle}{\flat}
\newcommand{\nonstandard}[1]{\prescript{*}{}{#1}}
\newcommand{\starR}{\nonstandard{\R}}
\newcommand{\borel}{\mathscr{B}}
\newcommand{\lebesgue}[1]{\mathscr{L}\left(#1\right)}
\newcommand{\media}{\mathds{E}}
\newcommand{\K}{\mathds{K}}
\newcommand{\A}{\mathds{A}}
\newcommand{\Q}{\mathds{Q}}
\newcommand{\N}{\mathds{N}}
\newcommand{\C}{\mathds{C}}
\newcommand{\Z}{\mathds{Z}}
\newcommand{\qo}{\hspace{1em}\text{q.o.}\,}
\renewcommand{\tilde}[1]{\widetilde{#1}}
\renewcommand{\parallel}{\mathrel{/\mkern-5mu/}}
\newcommand{\parti}[2][]{\wp_{#1}(#2)}
\newcommand{\diff}[1]{\operatorname{d}_{#1}}
\let\oldvec\vec
\renewcommand{\vec}[1]{\overrightarrow{\vphantom{i}#1}}
\newcommand{\floor}[1]{\left\lfloor #1 \right\rfloor}
\newcommand{\cat}[1]{\mathbf{#1}}
\newcommand{\dfreccia}[1]{\xrightarrow{\ #1 \ }}
\newcommand{\sfreccia}[1]{\xleftarrow{\ #1 \ }}
\newcommand{\formalsum}[2]{{\sum_{#1}^{#2}}{\vphantom{\sum}}'}
\newcommand{\minim}[2]{\mu_{#1}\, \left(#2\right)}
\newcommand{\concat}{\null^{\frown}} % concatenazione di stringe
\newcommand{\godelcode}[1]{\langle\!\langle #1 \rangle\!\rangle}
\newcommand{\godeldec}[1]{(\!(#1)\!)}
\newcommand{\termcode}[1]{\ulcorner #1\urcorner}
\newcommand{\partialto}{\dashrightarrow}
\newcommand{\restricted}{\upharpoonright}
\newcommand{\embeds}{\precsim}
\newcommand{\surjects}{\twoheadrightarrow}
\newcommand{\equipotenti}{\asymp}
%% \newcommand{\dotplus}{\mathbin{\dot{+}}} %% A quanto pare esiste già
\newcommand{\bigdot}{\mathbin{\boldsymbol{\cdot}}}
\newcommand{\dotexp}[1]{^{.#1}}
\newcommand{\conv}{\mathbin{*}}
\newcommand{\convolution}[2]{(#1\conv #2)}
\newcommand{\nil}{\mathfrak{N}}
\newcommand{\divisore}{\mathrel{|}}
\newcommand{\simplesso}[1]{\mathrm{e}_{#1}}

\renewcommand{\iff}{\mathrel{\longleftrightarrow}} %% Notazione Logica.
\newcommand{\oldiff}{\mathrel{\Longleftrightarrow}}
\renewcommand{\implies}{\mathrel{\rightarrow}} %% Notazione Logica
\newcommand{\oldimplies}{\mathrel{\Longrightarrow}}
\renewcommand{\impliedby}{\mathrel{\leftarrow}} %% Notazione Logica
\newcommand{\oldimpliedby}{\mathrel{\Longleftarrow}}

\newcommand{\IFF}{\quad\Longleftrightarrow\quad}
\newcommand{\IMPLICA}{\quad\Longrightarrow\quad}


\renewcommand{\descriptionlabel}[1]{\hspace{\labelsep}\normalfont #1} % remove bold from description


%% Definizione di Divergenza di K-L

\DeclarePairedDelimiterX{\infdivx}[2]{(}{)}{%
  #1\;\delimsize\|\;#2%
}
\newcommand{\kldiv}{D_{KL}\infdivx}

%% Definizione di \dotminus

\makeatletter
\newcommand{\dotminus}{\mathbin{\text{\@dotminus}}}

\newcommand{\@dotminus}{%
  \ooalign{\hidewidth\raise1ex\hbox{.}\hidewidth\cr$\m@th-$\cr}%
}
\makeatother

%tramite i prossimi due comandi posso decidere come scrivere i logaritmi naturali in tutti i documenti: ho infatti eliminato qualsiasi differenza tra "ln" e "log": se si vuole qualcosa di diverso bisogna inserire manualmente il tutto
\let\ln\relax
\DeclareMathOperator{\ln}{ln}
\let\log\relax
\DeclareMathOperator{\log}{log}
%%%%%%

%% NUOVI COMANDI
\newcommand{\straniero}[1]{\textit{#1}} %parole straniere
\newcommand{\titolo}[1]{\textsc{#1}} %titoli
\newcommand{\qedd}{\tag*{$\blacksquare$}} %qed per ambienti matemastici
\renewcommand{\qedsymbol}{$\blacksquare$} %modifica colore qed
\newcommand{\ooverline}[1]{\overline{\overline{#1}}}
\newcommand{\circoletto}[1]{\left(#1\right)^{\text{o}}}
%
\newcommand{\qmatrice}[1]{\begin{pmatrix}
#1_{11} & \cdots & #1_{1n}\\
\vdots & \ddots & \vdots \\
#1_{m1} & \cdots & #1_{mn}
\end{pmatrix}}
%
\newcommand{\parentesi}[2]{%
\underset{#1}{\underbrace{#2}}%
}
%
\newcommand{\norma}[1]{% Norma
\left\lVert#1\right\rVert%
}
\newcommand{\scalare}[2]{% Scalare
\left\langle #1, #2\right\rangle
}
%%%%%

%% RESTRIZIONI
\newcommand{\referenze}[2]{
        \phantomsection{}#2\textsuperscript{\textcolor{blue}{\textbf{#1}}}
}

\let\restriction\relax

\def\restriction#1#2{\mathchoice
              {\setbox1\hbox{${\displaystyle #1}_{\scriptstyle #2}$}
              \restrictionaux{#1}{#2}}
              {\setbox1\hbox{${\textstyle #1}_{\scriptstyle #2}$}
              \restrictionaux{#1}{#2}}
              {\setbox1\hbox{${\scriptstyle #1}_{\scriptscriptstyle #2}$}
              \restrictionaux{#1}{#2}}
              {\setbox1\hbox{${\scriptscriptstyle #1}_{\scriptscriptstyle #2}$}
              \restrictionaux{#1}{#2}}}
\def\restrictionaux#1#2{{#1\,\smash{\vrule height .8\ht1 depth .85\dp1}}_{\,#2}}
%%%%%%%%%%%

%%% FORMATTAZIONE FOOTNOTEMARK

\def\footnotemarkformatting#1{[#1]}
\renewcommand{\thefootnote}{\footnotemarkformatting{\arabic{footnote}}}

%% SEZIONE GRAFICA
\use{tikz}
\usetikzlibrary{matrix, patterns, calc, decorations.pathreplacing, hobby, decorations.markings, decorations.pathmorphing, babel}
\use{tikz-3dplot}
\use{mathrsfs} %per geogebra
\use{tikz-cd}
\tikzset
{
  %surface/.style={fill=black!10, shading=ball,fill opacity=0.4},
  plane/.style={black,pattern=north east lines},
  curve/.style={black,line width=0.5mm},
  dritto/.style={decoration={markings,mark=at position 0.5 with {\arrow{Stealth}}}, postaction=decorate},
  rovescio/.style={decoration={markings,mark=at position 0.5 with {\arrow{Stealth[reversed]}}}, postaction=decorate}
}
\use{pgfplots} % stampare le funzioni
        \pgfplotsset{/pgf/number format/use comma,compat=1.15}
        %\pgfplotsset{compat=1.15} %per geogebra
        \usepgfplotslibrary{fillbetween, polar}
%%%%%%

%% CITAZIONI
\use{lineno}

\newcommand{\citazione}[1]{%
  \begin{quotation}
  \begin{linenumbers}
  \modulolinenumbers[5]
  \begingroup
  \setlength{\parindent}{0cm}
  \noindent #1
  \endgroup
  \end{linenumbers}
  \end{quotation}\setcounter{linenumber}{1}
  }
%%%%%%

%%%%%%%%%%%%%%%%%%%%%%%%%%%%%%%%%%%%%%%%%%%%
%%%%%%%%%%%%%%%%%%%%%%%%%%%%%%%%%%%%%%%%%%%%

%% AMS THM

\theoremstyle{definition}% default
\newtheorem{thm}{Teorema}[section]
\newtheorem{lem}[thm]{Lemma}
\newtheorem{prop}[thm]{Proposizione}
\newtheorem{cor}[thm]{Corollario}
\newtheorem{esempio}[thm]{Esempio}
\theoremstyle{plain}
\newtheorem{definizione}[thm]{Definizione}
\theoremstyle{remark}
\newtheorem*{oss}{Osservazione}


%%%%%%%%%%%%%%%%%%%%%%%%%%%%%%%%%%%%%%%%%%%%
%%%%%%%%%%%%%%%%%%%%%%%%%%%%%%%%%%%%%%%%%%%%

\use{hyperref}
\hypersetup{%
        pdfauthor={Davide Peccioli},
        pdfsubject={},
        allcolors=black,
        citecolor=black,
%	colorlinks=true,
        bookmarksopen=true}
\setcounter{secnumdepth}{0} % rimuove i numeri di sezione senza rimuovere le ref
\renewcommand{\href}[2]{\textcolor{blue}{#2}} % disabilita il comando href
\use{enotez} %
\setenotez{%
 mark-format = \footnotemarkformatting % Mette i numeri tra parentesi quadre%
}\let\footnote=\endnote % rende tutte le note a pié pagina come delle note a fine file 


\let\olddocument\document % modifico l'ambiende documenti per non dover stampare \printendnote
\let\oldenddocument\enddocument
\renewenvironment{document}%
{%
  \olddocument
}{%
  \printendnotes\oldenddocument
}
\renewcommand{\thethm}{\arabic{thm}}

\usepackage[hyperref]{biblatex}
\addbibresource{~/Documents/org/roam/bib/master.bib}
\author{Davide Peccioli}
\date{\today}
\title{Coomologia delle sfere}
\begin{document}

\section{Coomologia delle sfere}
\label{sec:orgf3ed79f}
\begin{thm}
Sia \(\mathds{S}^{n}\) la \href{20250115150754-sfera_n_dimensionale.org}{sfera \(n\)-dimensionale}. Allora la sua \href{20251115172442-gruppo_di_coomologia_di_de_rham.org}{coomologia di De Rham} è:
\begin{equation*}
H^{k}(\mathds{S}^{n}) = %
\begin{cases}
\R & k=0,n\\
0 & k\neq 0,n
\end{cases}
\end{equation*}
\end{thm}
\begin{proof}
Vedi Esempio~5.3.7 di \autocite{abateGeometriaDifferenziale2011}.

Si considerino i poli \(\textbf{N}\in \R^{n+1}\) e \(\textbf{S} \in \R^{n+1}\) di \(\mathds{S}^{n}\), e siano
\begin{equation*}
U_{0} \coloneqq \mathds{S}^{n}\setminus\set{\textbf{N}}, \qquad U_{1}\coloneqq \mathds{S}^{n}\setminus\set{\textbf{S}}.
\end{equation*}
\begin{itemize}
\item Siccome \(U_{0}\cong U_{1} \cong \R^{n}\) \href{20250113172924-diffeomorfismo_tra_varieta_differenziabili.org}{diffeomorfi}, \href{20251115173810-varieta_differenziabili_diffeomorfe_hanno_stessa_coomologia_di_de_rham.org}{allora}\footnote{Vedi anche ``\href{20251115173611-coomologia_di_de_rham_di_r.org}{Coomologia di De Rham di R}''}
\begin{equation*}
  \forall k \in \N: \qquad H^{k}(U_{0}) \cong H^{k}(U_{1}) \cong H^{k}(\R) = \begin{cases}
  	\R & k = 0\\
  	0 &\text{altimenti}.
      \end{cases}
\end{equation*}
\item \(U_{0}\cap U_{1} = \mathds{S}^{n}\setminus\set{\textbf{N},\textbf{S}} \cong \mathds{S}^{n-1} \times \R\), e \href{20251223145108-riduzione_della_coomologia_di_un_prodotto_con_la_retta_reale.org}{pertanto}
\begin{equation*}
  \forall k \in \N:\qquad H^{k}(U_{0}\cap U_{1}) \cong H^{k}(\mathds{S}^{n-1}).
\end{equation*}
\end{itemize}

Si dimostra il teorema per induzione su \(n\ge 1\).
\begin{itemize}
\item \uline{Passo base \(n=1\)}.

\href{20251115184223-coomologia_della_circonferenza.org}{Si è già dimostrato} che
\begin{equation*}
H^{k}(\mathds{S}^{1}) = %
\begin{cases}
\R & k = 0,1\\
0 & k \ge 2
\end{cases}
\end{equation*}

\item \uline{Passo induttivo}. Sia \(n > 1\) e si suppona il teorema vero per \(\mathds{S}^{n-1}\). Si dimostra per \(\mathds{S}^{n}\)

\begin{itemize}
\item \uline{Se \(k=0\)}: Siccome \(\mathds{S}^{n}\) è \href{20250103165325-spazio_topologico_connesso.org}{connesso}, \href{20251115174538-0_gruppo_di_coomologia_di_de_rham_di_una_varieta_connessa.org}{allora} \(H^{0}(\mathds{S}^{n}) \cong \R\).

\item \uline{Se \(k=1\)}: Si consideri \href{20251115183635-teorema_di_mayer_vietoris_in_coomologia.org}{Mayer-Vietoris}:
\begin{equation*}
\begin{tikzcd}[sep=tiny]
        0 & {H^0(\mathds{S}^n)} & {H^0(U_0)\oplus H^0(U_1)} & {H^0(U_0\cap U_1)} & {H^1(\mathds{S}^n)} & {H^1(U_0)\oplus H^1(U_1)}
        \arrow[from=1-1, to=1-2]
        \arrow[from=1-2, to=1-3]
        \arrow[from=1-3, to=1-4]
        \arrow[from=1-4, to=1-5]
        \arrow[from=1-5, to=1-6]
\end{tikzcd}
\end{equation*}
per le considerazioni di cui sopra si ha:
\begin{itemize}
\item \(H^{0}(\mathds{S}^{n}) \cong \R\)
\item \(H^{0}(U_{0}\cap U_{1}) \cong \R\) \href{20251115174538-0_gruppo_di_coomologia_di_de_rham_di_una_varieta_connessa.org}{in quanto} connesso
\item \(H^{0}(U_{0}) \oplus H^{0}(U_{1}) \cong \R\oplus \R\)\footnote{Vedi ``\href{20241213095808-somma_diretta.org}{Somma Diretta}''}
\item \(H^{1}(U_{0}) \oplus H^{1}(U_{1}) \cong 0\)
\end{itemize}
e pertanto si ottiene:
\begin{equation*}
\begin{tikzcd}[sep=small]
        0 & \R & {\R\oplus \R} & \R & {H^1(\mathds{S}^n)} & 0
        \arrow[from=1-1, to=1-2]
        \arrow[from=1-2, to=1-3]
        \arrow[from=1-3, to=1-4]
        \arrow["{\partial^*}", from=1-4, to=1-5]
        \arrow[from=1-5, to=1-6]
\end{tikzcd}
\end{equation*}
Siccome \(\partial^{*}\) è \href{20241213105600-funzione_suriettiva.org}{suriettiva}\footnote{Vedi ``\href{20250120130155-caratterizzazione_di_alcune_successioni_esatte_di_r_moduli.org}{Caratterizzazione SEC}''}, allora \(\dim H^{1}(\mathds{S}^{n})< \infty\) e quindi, usando la \href{20251115182133-successione_di_spazi_vettoriali_esatta.org}{somma alterna delle dimensioni}, si ottiene
\begin{equation*}
      H^{1}(\mathds{S}^{n}) = 0
\end{equation*}

\item Se \(1<k\le n\): Si consideri Mayer-Vietoris:
\begin{equation*}
\begin{tikzcd}
        {H^{k-1}(U_0)\oplus H^{k-1}(U_1)} && {H^{k-1}(U_0\cap U_1)} \\
        \\
        {H^k(\mathds{S}^n)} && {H^k(U_0)\oplus H^k(U_1)}
        \arrow[from=1-1, to=1-3]
        \arrow["{{\partial^*}}"', from=1-3, to=3-1]
        \arrow[from=3-1, to=3-3]
\end{tikzcd}
\end{equation*}
Per le considerazioni di cui sopra:
\begin{itemize}
\item \(k-1 > 0\), e pertanto \(H^{k-1}(U_{0})\oplus H^{k-1}(U_{1}) \cong 0\);
\item \(k > 0\), e pertanto \(H^{k}(U_{0})\oplus H^{k}(U_{1}) \cong 0\);
\item \(H^{k-{1}}(U_{0}\cap U_{1}) \cong H^{k-1}(\mathds{S}^{n-1})\).
\end{itemize}
La successione diventa:
\begin{equation*}
\begin{tikzcd}
        0 & {H^{k-1}(\mathds{S}^{n-1})} && {H^k(\mathds{S}^n)} & 0
        \arrow[from=1-1, to=1-2]
        \arrow["{\partial^*}"', from=1-2, to=1-4]
        \arrow[from=1-4, to=1-5]
\end{tikzcd}
\end{equation*}
e \href{20250120130155-caratterizzazione_di_alcune_successioni_esatte_di_r_moduli.org}{pertanto}:
\begin{equation*}
H^{k}(\mathds{S}^{n}) = H^{k-1}(\mathds{S}^{n-1}).
\end{equation*}

Dunque, per ipotesi induttiva:
\begin{align*}
&& H^{n}(\mathds{S}^{n}) &= H^{n-1}(\mathds{S}^{n-1}) \cong \R\\
&1<k<n & H^{k}(\mathds{S}^{n}) &= H^{k-1}(\mathds{S}^{n-1}) \cong 0.
\end{align*}

\item Se \(k>n\): per motivi di dimensione, \(H^{k}(\mathds{S}^{n}) = 0\).
\qedhere
\end{itemize}
\end{itemize}
\end{proof}

\begin{prop}
Un \href{20250102163502-base_di_uno_spazio_vettoriale.org}{generatore} di \(H^{n}(\mathds{S}^{n})\) è \([\nu]\), dove \(\nu\) è una \href{20251115184544-forma_volume_su_una_varieta_differenziabile.org}{forma volume} su \(\mathds{S}^{n}\).
\end{prop}
\begin{proof}
La dimostrazione si articola in alcune fasi
\begin{enumerate}
\item \uline{Esistenza di una forma volume}

Questo \href{20251115185324-caratterizzazione_varieta_differenziabile_orientabile_tramite_forma_voluma.org}{segue} dal fatto che \href{20251223161934-orientabilita_delle_sfere.org}{\(\mathds{S}^{n}\) è orientabile}.

\item \uline{La forma volume non è esatta}

Se per assurdo \(\nu\) fosse esatta, allora \(\nu = \dif \omega\). Poiché \(\mathds{S}^{n}\) è \href{20250103163701-spazio_topologico_compatto.org}{compatta}, allora per il \href{20251115190058-teorema_di_stokes.org}{Teorema di Stokes}:
\begin{equation*}
 \int_{M} \nu = \int_{M} \dif \omega = 0
\end{equation*}
ma questo è assurdo.
\end{enumerate}

Siccome la forma volume è una \(n\)-forma, allora è \href{20251115172517-forma_differenziale_chiusa.org}{chiusa}, e pertanto \([\nu] \in H^{n}(\mathds{S}^{n})\), e siccome non è esatta allora \([\nu] \neq 0\). Quindi \([\nu]\) genera \(H^{n}(\mathds{S}^{n}) \cong \R\).
\end{proof}

\begin{prop}
Si definisca la mappa data dall'\href{20251115185654-integrazione_di_forme_su_varieta_differenziabile_orientata.org}{integrale}:
\begin{align*}
\int: H^{n}(\mathds{S}^{n}) &\longrightarrow \R\\
[\omega] &\longmapsto \int_{\mathds{S}^{n}}\omega
\end{align*}

Questo è un \href{20250113125833-isomorfismo_tra_spazi_vettoriali.org}{isomorfismo}.
\end{prop}
\begin{proof}
La diumostrazione si sviluppa in diversi passi.
\begin{enumerate}
\item \uline{La mappa è ben definita in coomologia}

Questo vale per il teorema di Stokes.

\item \uline{La mappa è lineare}
\item \uline{La mappa è non nulla}

Infatti \(\int_{\mathds{S}^{n}} \nu > 0\) per \(\nu\) \href{20251115184544-forma_volume_su_una_varieta_differenziabile.org}{forma volume}. (Questo lo si può fare in quanto \(\mathds{S}^{n}\) è \href{20250103163701-spazio_topologico_compatto.org}{compatta} e \href{20251223152054-varieta_differenziabile_orientabile.org}{orientabile}.)
\end{enumerate}

Da questo segue che sia un isomorfismo:
\begin{itemize}
\item è \href{20241219101956-funzione_iniettiva.org}{iniettiva} in quanto il suo \href{20251121143525-kernel_di_una_funzione_tra_spazi_vettoriali.org}{kernel} (siccome \(H^{n}(\mathds{S}^{n})\) ha dimensione 1) può avere solo dimensione \(0\) (non ha dimensione 1 per il punto 3.)
\item è \href{20241213105600-funzione_suriettiva.org}{suriettiva} per il punto 3 (siccome raggiunge un elemento non nullo, siccome \(\R\) ha dimensione 1, per linearità raggiunge tutti gli elementi).
\end{itemize}
\end{proof}
\end{document}
