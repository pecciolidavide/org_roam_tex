% Intended LaTeX compiler: pdflatex
\documentclass[../main]{subfiles}


\begin{document}

\section{Campo vettoriale su una varietà differenziabile}
\label{sec:org9dcc87a}
Sia \(M\) una \href{20250113115909-struttura_differenziabile.org}{varietà differenziabile}, e sia \(U \subseteq M\) \href{20250103145124-topologia.org}{aperto}.

\begin{definizione}
Un \textbf{campo vettoriale} su \(U\) è una \href{20250113144722-funzioni_cinfinito_tra_varieta_differenziabili.org}{funzione \(C^{\infty}\)}(infatti \href{20250113152503-aperti_di_una_varieta_differenziabile_sono_varieta_differenziabili.org}{Aperti di una varietà differenziabile sono varietà differenziabili})
\begin{align*}
X: U &\longrightarrow \operatorname{T}M\\
p &\longmapsto X_{p}
\end{align*}
tale che per ogni \(p \in {U}\), \(X_{p} \in \operatorname{T}_{p}M\) (vedi \href{20250114102823-spazio_tangente_ad_un_punto_di_una_varieta_differenziabile.org}{Spazio tangente ad un punto di una varietà differenziabile} e \href{20250115103245-fibrato_tangente.org}{Fibrato tangente})

Si denota con
\begin{equation*}
\Gamma(U)\coloneqq \set{X:U\longrightarrow \operatorname{T}M: X\text{ campo vettoriale}}
\end{equation*}
\end{definizione}
\end{document}
