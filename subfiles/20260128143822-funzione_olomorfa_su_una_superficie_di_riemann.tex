% Intended LaTeX compiler: pdflatex
\documentclass[../main]{subfiles}


\begin{document}

\section{Funzione olomorfa su una superficie di Riemann a valori complessi}
\label{sec:org21917af}
Sia \(X\) una \href{20260127112828-superficie_di_riemann.org}{superficie di Riemann}, \(U \subseteq X\) aperto, e \(f:U \to \C\).

\begin{definizione}
La funzione \(f\) si dice \textbf{olomorfa in \(p \in U\)} se esiste una \href{20260127112715-atlante_complesso.org}{carta locale}
\begin{equation*}
\varphi: U_{0} \to V \subseteq \C
\end{equation*}
con \(p \in U_{0}\) tale che \(f \circ \varphi^{-1}\) sia \href{20260126110551-funzione_olomorfa.org}{olomorfa} in \(\varphi(p)\).
\end{definizione}

\begin{oss}
Questa definizione non dipende dalla scelta della carta locale.
\end{oss}

\begin{definizione}
\(f\) si dice \textbf{olomorfa in \(U\)} se è olomorfa in \(p\) per ogni \(p \in U\).
\end{definizione}

\begin{prop}
Se \(f\) è olomorfa su \(U\), allora è anche \href{20250103103252-funzione_continua.org}{continua} e \href{20250113144722-funzioni_cinfinito_tra_varieta_differenziabili.org}{\(\mathcal{C}^{\infty}\)} vedendo \(X\) e \(\C\) come \href{20250113115909-struttura_differenziabile.org}{varietà differenziabili} reale.
\end{prop}

\begin{oss}
L'insieme delle funzioni olomorfe in \(U\):
\begin{equation*}
\set{f:U\subseteq X \to \C \mid f\text{ olomorfa}}
\end{equation*}
è una \href{20250110175552-algebra_su_un_campo.org}{\(\C\)-algebra}
\end{oss}
\section{Funzione olomorfa tra superfici di Riemann}
\label{sec:orgdc92f2c}
Siano \(X,Y\) due \href{20260127112828-superficie_di_riemann.org}{superfici di Riemann}, \(F:X\to Y\) una \href{20250202170607-classe_relazione_binaria.org}{funzione}.

\begin{definizione}
\(F\) si dice \textbf{olomorfa in \(p \in X\)} se esistono
\begin{itemize}
\item una \href{20260127112715-atlante_complesso.org}{carta locale} per \(X\): \(\varphi_{1} : U_{1}\to V_{1} \subseteq \C\), con \(p \in U_{1}\);
\item una \href{20260127112715-atlante_complesso.org}{carta locale} per \(Y\): \(\varphi_2 : U_2\to V_{2} \subseteq \C\), con \(F(p) \in U_{2}\);
\end{itemize}
tali che \(\varphi_{2}\circ F \circ \varphi_{1}^{-1}\) sia definita e olomorfa in un intorno di \(\varphi_{1}(p) \in V_{1} \subseteq \C\).
\end{definizione}

\begin{definizione}
\(F\) si dice \textbf{olomorfa} se è olomorfa in \(p\) per ogni \(p \in X\).
\end{definizione}

\begin{prop}
\begin{enumerate}
\item La funzione identità \(\Id:X\to X\) e tutte le funzioni costanti sono olomorfe.
\item La condizione non dipende dalla scelta delle carte locali.
\item Se \(Y=\C\), si ottiene la stessa definizione di \hyperref[sec:org21917af]{funzioni olomorfe} \(F:X\to \C\).
\item Se \(F\) è olomorfa, allora \(F\) è \href{20250103103252-funzione_continua.org}{continua}, \href{20250113144722-funzioni_cinfinito_tra_varieta_differenziabili.org}{\(\mathcal{C}^{\infty}\)} e preserva \href{20251223152054-varieta_differenziabile_orientabile.org}{l'orientazione} come mappa tra \href{20250113115909-struttura_differenziabile.org}{varietà differenziabili}.
\item Composizione di mappe olomorfe è olomorfa.
\end{enumerate}
\end{prop}
\end{document}
