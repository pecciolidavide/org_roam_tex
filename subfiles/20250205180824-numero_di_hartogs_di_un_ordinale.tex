% Intended LaTeX compiler: pdflatex
\documentclass[../main]{subfiles}


\begin{document}

Contesto: \href{20250130104245-morse_kelly_set_theory.org}{Morse Kelly Set Theory}
\section{Definizione}
\label{sec:org0fa000a}
Sia \(\alpha\) un \href{20250203111003-ordinali.org}{ordinale}. Si denota il \href{20250205152531-numeri_di_hartogs.org}{numero di Hartogs} di \(\alpha\) come
\begin{equation*}
\alpha^{+}\coloneqq \operatorname{Hrtg}(\alpha)
\end{equation*}

Dunque \(\alpha^{+}\) è il \href{20250203111003-ordinali.org}{più piccolo} cardinale strettamente maggiore di \(\alpha\), e se \(\alpha\le \omega\) (vedi \href{20250203161110-numeri_naturali_sono_ordinali.org}{Ordinale omega}) allora
\begin{equation*}
\alpha^{+} = \bigcup \set{\beta\,|\,|\beta|=|\alpha|} = \set{\beta\,|\,|\beta|\le|\alpha|}
\end{equation*}
(vedi \href{20250131155822-operazioni_insiemistiche_tra_classi_mk.org}{Classe Unione Generalizzata} e \href{20241213101756-cardinalita.org}{Cardinalità})
\subsection{Ordinale omega1}
\label{sec:org29b231e}
Si denota \(\omega^{+}\) con \(\omega_{1}\).
\subsection{Osservazione}
\label{sec:orgb975f71}
\href{20250205181254-order_type_del_prodotto_cartesiano_di_un_cardinale_e_il_cardinale_stesso.org}{Siccome} \(\omega\times\omega \asymp \omega\) (vedi \href{20250131183735-prodotto_cartesiano_di_classi_mk.org}{Prodotto cartesiano di classi MK} e \href{20250619101109-classi_equipotenti.org}{Classi equipotenti MK}), si ha che \href{20250205152531-numeri_di_hartogs.org}{esiste} una \href{20241213105600-funzione_suriettiva.org}{funzione suriettiva}
\begin{equation*}
\parti{\omega}\to \omega^{+}
\end{equation*}
Notiamo inoltre che \(\R\asymp \parti{\omega}\)
\end{document}
