% Intended LaTeX compiler: pdflatex
\documentclass[../main]{subfiles}


\begin{document}

In questa sezione si identificano i \href{20250131155822-operazioni_insiemistiche_tra_classi_mk.org}{sottinsiemi} di \(\N^{k}\) (vedi \href{20250202130045-insieme_dei_numeri_naturali_mk.org}{Insieme dei numeri naturali MK}) con i \href{20250131103317-formula_del_prim_ordine.org}{predicati} \(k\)-ari (ovvero con \(k\) \href{20250131103429-variabile_libera_di_una_formula.org}{variabili libere}), per mezzo degli \href{20250131122913-soddisfazione_di_una_formula.org}{insiemi di verità}.

Scriveremo indifferentemente \((x_{1},\dots,x_{k}) \in P\) oppure \(P(x_{1},\dots,x_{k})\) per dire che \(\N\vDash P(x_{1},\dots,x_{k})\).
\section{Proposizione}
\label{sec:org8142750}
La collezione degli \href{20250520113238-insieme_semiricorsivo.org}{insiemi semiricorsivi} è chiusa per:
\begin{enumerate}
\item sostituzioni \href{20250215151458-funzioni_ricorsive.org}{ricorsive} (mediante \href{20250213105339-funzione_parziale.org}{funzioni totali});
\item proiezioni (ovvero quantificatori esistenziali);
\item \href{20250131155822-operazioni_insiemistiche_tra_classi_mk.org}{intersezioni} (ovvero conginuzioni);
\item \href{20250131155822-operazioni_insiemistiche_tra_classi_mk.org}{unioni} (ovvero disgiunzioni);
\item quantificazioni limitate.
\end{enumerate}
\subsection{Dimostrazione}
\label{sec:org508b5af}
\begin{enumerate}
\item Sia \(P \subseteq \N^{k}\) un insieme semiricorsivo, e siano, per \(1\le i\le k\): \(f_{i}: \N^{\ell}\to \N\) funzioni ricorsive totali. L'insieme \(R \subseteq \N^{\ell}\) definito da
\begin{equation*}
 R(\oldvec{x}^{\ell}) \quad \iff \quad P\left(f_{1}(\oldvec{x}^{\ell}),\dots,f_{k}(\oldvec{x}^{\ell})\right)
\end{equation*}
è semiricorsivo. Infatti, siccome \(P\) è semiricorsivo, allora esiste \(S \subseteq \N^{k+1}\) ricorsivo tale che
\begin{equation*}
 P(\oldvec{x}^{k}) \quad \iff\quad \exists\,y\ S(\oldvec{x}^{k},y).
\end{equation*}
Dunque, ponendo \(\oldvec{x}^{k} \coloneqq \left(f_{1}(\oldvec{x}^{\ell}),\dots,f_{k}(\oldvec{x}^{\ell})\right)\) si ha:
\begin{align*}
 R(\oldvec{x}^{\ell}) \quad &\iff \quad P\left(f_{1}(\oldvec{x}^{\ell}),\dots,f_{k}(\oldvec{x}^{\ell})\right)\\
 &\iff\quad P(\oldvec{x}^{k})\\
 &\iff\quad \exists\,y\ S(\oldvec{x}^{k},y)\\
 &\iff\quad \exists\,y \ S\left(f_{1}(\oldvec{x}^{\ell}),\dots,f_{k}(\oldvec{x}^{\ell}),y \right)
\end{align*}
ma \(S\left(f_{1}(\oldvec{x}^{\ell}),\dots,f_{k}(\oldvec{x}^{\ell}),y \right)\) definisce un insieme ricorsivo, quindi la tesi.
\item Sia \(P \subseteq \N^{k+1}\) semiricorsivo, e sia \(R \subseteq \N^{k}\) definito da
\begin{equation*}
 R(\oldvec{x}) \quad \iff \quad \exists\, y\ P(\oldvec{x},y).
\end{equation*}
\(R\) è semiricorsivo. Infatti, siccome \(P\) è semiricorsivo, allora esiste \(S \subseteq \N^{k+2}\) ricorsivo tale che
\begin{equation*}
 P(\oldvec{x},y)\quad \iff\quad \exists\,z\ S(\oldvec{x},y,z).
\end{equation*}
Ma allora
\begin{align*}
 R(\oldvec{x}) \quad &\iff\quad \exists\,y\ P(\oldvec{x},y)\\
 &\iff\quad \exists\, y\exists\,z S(\oldvec{x},y,z)\\
 &\iff\quad \exists\,t\ \left(\exists\, y\le t\ \exists\,z\le t\  S(\oldvec{x},y,z)\right)
\end{align*}
e l'insieme definito da \(\left(\exists\, y\le t\ \exists\,z\le t\  S(\oldvec{x},y,z)\right)\) \href{20250519112917-proprieta_di_chiusura_degli_insiemi_ricorsivi.org}{è ricorsivo}.
\item Come sopra
\item Come sopra
\item Come sopra
\end{enumerate}
\end{document}
