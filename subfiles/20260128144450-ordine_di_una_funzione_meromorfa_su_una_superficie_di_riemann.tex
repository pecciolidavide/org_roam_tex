% Intended LaTeX compiler: pdflatex
\documentclass[../main]{subfiles}


\begin{document}

\section{Ordine di una funzione meromorfa su una superficie di Riemann}
\label{sec:org25ff9ad}
\begin{definizione}
Sia \(X\) una \href{20260127112828-superficie_di_riemann.org}{superficie di Riemann}, \(U \subseteq X\) \href{20250103145124-topologia.org}{aperto} e \(p_{0} \in U\).

Sia \(f\) \href{20260128144105-funzione_meromorfa_su_una_superficie_di_riemann.org}{meromorfa} su \(U\). Se \(\varphi: U_{p}\to V\) è una \href{20260127112715-atlante_complesso.org}{carta locale} per \(p \in U_{p}\) intorno aperto, allora si pone\footnote{Vedi ``\href{20260128144433-ordine_di_una_funzione_meromorfa.org}{Ordine di una funzione meromorfa}''}
\begin{equation*}
\mathrm{ord}_{p_{0}} f \coloneqq \mathrm{ord}_{\varphi(p_{0})} (f\circ \varphi^{-1}).
\end{equation*}
\end{definizione}
\end{document}
