% Intended LaTeX compiler: pdflatex
\documentclass[../main]{subfiles}


\begin{document}

\section{Moduli}
\label{sec:orgb871e08}

\begin{definizione}
Sia \(R\) un \href{20241205141119-anello.org}{anello} commutativo con unità, e sia \(M\) un \href{20241205141053-r_moduli.org}{\(R\)-modulo}. \(E \subseteq M\) si dice \textbf{linearmente indipendente} se per ogni \(e_1,\dots,e_n \in E\) e per ogni \(a_1,\dots,a_n \in R\)
\begin{equation*}
\sum_{i=1}^n a_i\ e_i = 0 \IMPLICA a_1=\dots=a_n=0.
\end{equation*}
\end{definizione}

\begin{oss}
Osserviamo che non tutti i moduli hanno insiemi linearmente indipendenti. Consideriamo \(\Z_2\)\footnote{Vedi ``\href{20241213093511-gruppi_zp.org}{Gruppi-Zp}''} come \(\Z\)-modulo (siccome è un \href{20250127093245-gruppo_abeliano.org}{gruppo abeliano}):
\begin{equation*}
\Z_2 = \set{\overline{0},\overline{1}}
\end{equation*}

Entrambi gli insiemi \(\set{\overline{0}},\set{\overline{1}}\) sono linearmente dipendenti:
\begin{itemize}
\item \(\set{\overline{0}}\): per ogni \(z \in \Z\): \(z \cdot \overline{0} = \overline{0}\)
\item \(\set{\overline{1}}\): \(2 \cdot \overline{1} = \overline{0}\), ma \(2\) non è lo zero di \(\Z\).
\end{itemize}
Dunque \(\Z_2\) come \(\Z\)-modulo non ammette insiemi linearmente indipendenti.

\textbf{Tuttavia}, considerando \(\Z_2\) come \(\Z_2\)-modulo, questo è addirittura uno \href{20241205142027-spazio_vettoriale.org}{spazio vettoriale} (poiché \(\Z_{2}\) è campo).
\end{oss}
\section{Vettori}
\label{sec:org2254c8f}
\end{document}
