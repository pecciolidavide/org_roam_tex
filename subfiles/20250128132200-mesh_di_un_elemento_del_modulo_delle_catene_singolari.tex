% Intended LaTeX compiler: pdflatex
\documentclass[../main]{subfiles}

\use{upgreek}
\def\tau{\uptau}


\begin{document}

\section{Mesh di una catena singolare}
\label{sec:orgbc1987c}
Sia \(R\) un \href{20241219112842-pid.org}{PID}. Si consideri \(\Delta_{q}\) il \href{20250121122324-simplesso_standard.org}{simplesso standard}.
\begin{itemize}
\item Si consideri \(\Sigma_{q}(\Delta_{q})\) l'insieme dei \href{20250122133435-simplesso_singolare.org}{simplessi singolari},
\item Si consideri \(S_{q}(\Delta_{q}) = R^{(\Sigma_{q}(\Delta_{q}))}\)\footnote{Questa è la \href{20241213095808-somma_diretta.org}{somma diretta}.} il \href{20241205141053-r_moduli.org}{modulo} delle \href{20250122133435-simplesso_singolare.org}{catene singolari}
\end{itemize}
Per ogni \(c \in S_{q}(\Delta_{q})\), siano\footnote{Queste esistono poiché \(\Sigma_{q}(\Delta_{q})\) è \href{20241213094625-modulo_libero.org}{base} di \(S_{q}(\Delta_{q})\).} \(\tau_{1},\dots,\tau_{n} \in \Sigma_{q}(\Delta_{q})\) tali che
\begin{equation*}
c=\sum_{i}a_{i}\cdot \tau_{i},\qquad a_{i } \in R
\end{equation*}

\begin{definizione}
Si definisce la mesh di \(c\):\footnote{Vedi ``\href{20250202190147-immagine_punto_a_punto_di_due_classi.org}{Immagine e retroimmagine tramite una funzione}'' e ``\href{20250203102516-massimo_e_minimo.org}{Elemento Massimo}''.}
\begin{equation*}
\operatorname{mesh}(c) \coloneqq \max_{i}\set{\operatorname{diam}\big(\operatorname{Im}(\tau_{i})\big)}_{i}
\end{equation*}
dove \(\operatorname{diam}\) è la massima distanza tra due punti di un insieme di \(\R^{k}\).
\end{definizione}

\begin{oss}
Se \(\tau:\Delta_{q}\to \Delta_{q}\) \href{20250129094132-trasformazione_affine.org}{trasformazione affine}, con \(P_{i} \coloneqq \tau(e_{i})\)\footnote{Con \(e_{i}\) si indicano \href{20250121122324-simplesso_standard.org}{i punti base di \(\Delta_{q}\)}}, allora
\begin{equation*}
\operatorname{diam}\big(\operatorname{Im}(\tau_{i})\big) = %
\operatorname{diam} [P_{0},\dots,P_{q}] = \max_{i,j} \norma{P_{i}-P_{j}}.
\end{equation*}
\end{oss}
\subsection{Legame con la mappa di suddivisione}
\label{sec:org4d4d0a7}

Si consideri \(\operatorname{sd}_{q} :S_{q}(\Delta_{q})\to S_{q}(\Delta_{q})\) \href{20250128132040-mappa_di_suddivisione_tra_complessi_di_catene_singolari.org}{mappa di suddivisione}.

\begin{lem}
Se \(\tau:\Delta_{q}\to \Delta_{q}\) \href{20250129094132-trasformazione_affine.org}{trasformazione affine}, \(\tau \in S_{q}(\Delta_{q})\), allora
\begin{equation*}
\operatorname{mesh}(\operatorname{sd}_{q}\tau) \le \frac{q}{q+1} \operatorname{mesh}(\tau).
\end{equation*}
\end{lem}

\begin{cor}
Se \(\iota_{q}: \Delta_{q} \longrightarrow\Delta_{q}\) è l'identità, allora per ogni \(\varepsilon>0\) esiste \(m\) tale che\footnote{Con \(\operatorname{sd}_{q}^{m}\) si indicano \(m\) \href{20250202170607-classe_relazione_binaria.org}{composizioni} della mappa \(\operatorname{sd}_{q}\).}
\begin{equation*}
\operatorname{mesh}\big(
\operatorname{sd}_{q}^{m}(\iota_{q})
\big)<\varepsilon
\end{equation*}
\end{cor}
\begin{proof}
\(\displaystyle \operatorname{mesh}(\operatorname{sd}^{m}(\iota)) \le \left(\frac{q}{q+1}\right)^{m} \operatorname{mesh}\iota\).
\end{proof}
\end{document}
