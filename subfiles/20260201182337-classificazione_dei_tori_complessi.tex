% Intended LaTeX compiler: pdflatex
\documentclass[../main]{subfiles}

\begin{document}

\section{Classificazione dei tori complessi}
\label{sec:orgd8123db}
Si consideri il semipiano superiore di \(\C\):
\begin{equation*}
\mathds{H} \coloneqq \set{z \in \C \mid \operatorname{Im} z > 0}.
\end{equation*}
Per ogni \(\uptau \in \mathds{H}\), sia \(L_{\uptau}\) il reticolo generato da \(1\) e \(\uptau\):
\begin{equation*}
L_{\uptau} \coloneqq \set{z_{1} + z_{2}\,\uptau \mid z_{1},z_{2} \in \Z}
\end{equation*}
e sia \(X_{\uptau} \coloneqq \C/L_{\uptau}\) il \href{20260127113001-toro_complesso.org}{toro complesso} dato dal \href{20250127093819-quoziente_di_gruppo_e_sottogruppo.org}{quoziente} con \(L_{\uptau}\).

\begin{prop}
Sia \(Y\) un toro complesso qualsiasi. Allora esiste \(\uptau \in \mathds{H}\) tale che
\begin{equation*}
Y = X_{\uptau}.
\end{equation*}
\end{prop}

\begin{proof}
Sia \(Y=\C/M\), con \(M\), e siano \(w_{1},w_{2} \in \C\) tale che
\begin{equation*}
M \coloneqq \set{z_{1}\,w_{1}+z_{2}\,w_{2} \mid z_{1},z_{2} \in \Z}
\end{equation*}
Allora \(w_{1},w_{2}\) sono \(\R\)-\href{20241212142019-insiemi_linearmente_indipendenti.org}{linearmente indipendenti}, e \(w_{1} \neq 0\). Sia \(\gamma \coloneqq \frac{1}{w_{1}}\).

Allora \(\gamma\cdot M\) è il reticolo generato da \(1,\frac{\omega_{2}}{\omega_{1}}\), linearmente indipendenti su \(\R\). Pertanto, \(\frac{w_{2}}{w_{1}}\notin \R\).

\begin{itemize}
\item Se \(\operatorname{Im}\frac{w_{2}}{w_{1}} > 0\), allora \(\uptau\coloneqq \frac{w_{2}}{w_{1}}\) e
\begin{equation*}
  \gamma \cdot M = L_{\uptau}
\end{equation*}
e \href{20260201182313-isomorfismo_tra_tori_complessi.org}{quindi} \(Y \cong X_{\uptau}\).

\item Se \(\operatorname{Im}\frac{w_{2}}{w_{1}} < 0\), allora \(\uptau\coloneqq - \frac{w_{2}}{w_{1}}\) e\footnote{Ovviamente \(\left(1,\frac{w_{2}}{w_{1}}\right)\) e \(\left(1,-\frac{w_{2}}{w_{1}}\right)\) generano lo stesso reticolo.}
\begin{equation*}
  \gamma \cdot M = L_{\uptau}
\end{equation*}
e \href{20260201182313-isomorfismo_tra_tori_complessi.org}{quindi} \(Y \cong X_{\uptau}\).
\qedhere
\end{itemize}
\end{proof}

La situazione quindi è la seguente:\footnote{A meno di \href{20260128144717-isomorfismo_tra_superfici_di_riemann.org}{isomorfismo} tra \href{20260127112828-superficie_di_riemann.org}{superfici di Riemann}.}
\begin{equation*}
\begin{tikzcd}[ampersand replacement=\&]
	{\set{\text{tori complessi}}/\text{iso}} \&\&\& {\set{X_{\uptau}}_{\uptau \in \mathds{H}}/\text{iso}}
	\arrow["{1\duepunti 1}", tail reversed, from=1-1, to=1-4]
\end{tikzcd}
\end{equation*}

\begin{prop}
Siano \(\uptau,\uptau' \in \mathds{H}\). Sono fatti equivalenti:
\begin{enumerate}
\item sono isomorfi: \(X_{\uptau} \cong X_{\uptau'}\);
\item esiste \(\left(\begin{smallmatrix} a & b\\ c & d\end{smallmatrix}\right) \in \operatorname{SL}(2;\Z)\)\footnote{\(\operatorname{SL}(n, \K)\) è il \href{20250113142317-gruppo_lineare_speciale.org}{gruppo lineare speciale}} tale che
\begin{equation*}
 \uptau = \frac{a\uptau' + b}{c\uptau' + d}.
\end{equation*}
\end{enumerate}
\end{prop}

\begin{proof}
Per la \href{20260201182313-isomorfismo_tra_tori_complessi.org}{caratterizzazione dei tori isomorfi}, \(X_{\uptau}\cong X_{\uptau'}\) se e solo se esiste \(\gamma \in \C\setminus \set{0}\) tale che \(\gamma\cdot L_{\uptau} = L_{\uptau'}\), dove
\begin{align*}
L_{\uptau} &= \set{z_{1}+\uptau \, z_{2} \mid z_{1},z_{2} \in \Z}\\
L_{\uptau'} &= \set{z_{1}+\uptau' \, z_{2} \mid z_{1},z_{2} \in \Z}\\
\gamma \cdot L_{\uptau} &= \set{\gamma\cdot(z_{1}+\uptau \, z_{2}) \mid z_{1},z_{2} \in \Z} = \set{z_{1}\,\gamma + z_{2}\, (\gamma\uptau) \mid z_{1},z_{2} \in \Z}
\end{align*}
ovvero se e solo se esiste \(\gamma \in \C\setminus\set{0}\) tale che \(\gamma,\gamma\uptau \in L_{\uptau'}\), e lo generano come gruppo additivo.

(\(\Rightarrow\)): Siccome \(\gamma, \gamma\uptau \in L_{\uptau'}\) allora esistono \(a,b,c,d \in \Z\) tali che
\begin{align*}
0 \neq \gamma &= c\uptau' + d\\
\gamma\uptau &= a\uptau' + b
\end{align*}
Allora \(\uptau = \frac{a\uptau'+b}{c\uptau' + d}\), e \(\left(\begin{smallmatrix} a & b\\ c & d \end{smallmatrix}\right) \in \Z^{2\times 2}\):
\begin{equation*}
\begin{pmatrix}
\gamma\uptau\\
\gamma
\end{pmatrix} =
\begin{pmatrix}
a & b\\
c & d
\end{pmatrix}\
\begin{pmatrix}
\uptau'\\
1
\end{pmatrix}
\end{equation*}

Siccome \(1, \uptau' \in \gamma L_{\uptau}\) allora esistono \(a',b',c',d' \in \Z\) tali che
\begin{align*}
1 &= \gamma(c'\uptau + d')\\
\uptau' &= \gamma(a'\uptau + b')
\end{align*}
Allora \(\left(\begin{smallmatrix} a' & b'\\ c' & d' \end{smallmatrix}\right) \in \Z^{2\times 2}\) è tale che :
\begin{equation*}
\begin{pmatrix}
\uptau'\\
1
\end{pmatrix} =
\begin{pmatrix}
a' & b'\\
c' & d'
\end{pmatrix}\
\begin{pmatrix}
\gamma \uptau\\
\gamma
\end{pmatrix}
\end{equation*}

Segue che \(\left(\begin{smallmatrix} a & b\\ c & d \end{smallmatrix}\right)\) è invertibile, e con inversa in \(\Z^{2\times 2}\), pertanto
\begin{equation*}
\det \begin{pmatrix} a' & b'\\ c' & d' \end{pmatrix} \cdot \det \begin{pmatrix} a & b\\ c & d \end{pmatrix} = 1
\end{equation*}
e devono entrambi essere numeri interi. Segue che
\begin{equation*}
\det \begin{pmatrix} a & b\\ c & d \end{pmatrix} = \pm 1
\end{equation*}

Inoltre si ha che\footnote{\(\overline{\uptau'}\) è il \href{20260201232630-complesso_coniugato.org}{complesso coniugato}}
\begin{align*}
\operatorname{Im} \uptau &= \operatorname{Im}\left( %
\frac{a\uptau'+ b}{c\uptau' + d}
\right) = \operatorname{Im}\left( %
\frac{(a\uptau'+ b)(c\overline{\uptau'}+d)}{|c\uptau' + d|^{2}}
\right)\\[2ex]
&= \operatorname{Im}\left( %
\frac{%
	ac \, |\uptau'|^{2} + ad \,\uptau' + bc\, \overline{\uptau'} + bc%
}{|c\uptau' + d|^{2}}
\right)\\
&= \frac{1}{|c\uptau' + d|^{2}} \cdot \operatorname{Im}(ac \, |\uptau'|^{2} + ad \,\uptau' + bc\, \overline{\uptau'} + bc)\\
&= \frac{1}{|c\uptau' + d|^{2}} \cdot \operatorname{Im}(ad \,\uptau' + bc\, \overline{\uptau'})\\
&= \frac{1}{|c\uptau' + d|^{2}} \cdot \bigg[ad\,\operatorname{Im}(\uptau') + bc\, \operatorname{Im}(\overline{\uptau'})\bigg]\\
&= \frac{1}{|c\uptau' + d|^{2}} \cdot \bigg[ad\,\operatorname{Im}(\uptau') - bc\, \operatorname{Im}({\uptau'})\bigg]\\
&= \frac{1}{|c\uptau' + d|^{2}} \cdot (ad-bc)\operatorname{Im}(\uptau')\\
&= \frac{1}{|c\uptau' + d|^{2}} \cdot \det\begin{pmatrix}
a & b\\
c & d
\end{pmatrix}\cdot \operatorname{Im}(\uptau')
\end{align*}
e quindi, per concordanza dei segni (\(\operatorname{Im}\uptau > 0\), \(\operatorname{Im}\uptau' >0\)), si deve avere \(\det\left(\begin{smallmatrix} a & b\\c & d \end{smallmatrix}\right) > 0'\).

(\(\Leftarrow\)): Supponiamo quindi che \(\uptau = \frac{a\uptau'+b}{c\uptau' + d}\), con
\begin{equation*}
\begin{pmatrix}
a & b\\
c & d
\end{pmatrix} \in \operatorname{SL}(2;\Z).
\end{equation*}

Sia \(\gamma \coloneqq c\uptau' + d\).
\begin{itemize}
\item \(\gamma \neq 0\), poiché \((c,d) \neq (0,0)\) visto che la matrice è invertibile, e \(\uptau'\notin \R\).
\item Ovviamente \(\gamma, \gamma\,\uptau \in L_{\uptau'}\), in quanto
\begin{equation*}
  \gamma\,\uptau = \gamma\, \frac{a\uptau'+b}{c\uptau' + d} = a\uptau' + b.
\end{equation*}
\item Per dimostrare che \(\gamma,\gamma\,\uptau\) generino \(L_{\uptau'}\) come gruppo additivo, è sufficiente scrivere \(1,\uptau'\) come \(\Z\)-combinazione lineare di \(\gamma,\gamma\,\uptau\). In particolare, sia \(A' \in \Z^{2\times 2}\) la matrice inversa di \(\left(\begin{smallmatrix} a & b\\ c & d\end{smallmatrix}\right)\). Allora
\begin{equation*}
  A'\cdot \begin{pmatrix}
  	\gamma\,\uptau\\
  	\gamma
  \end{pmatrix} = A' \cdot \begin{pmatrix}
  	a & b\\
  	c & d
  \end{pmatrix} \cdot \begin{pmatrix}
  	\uptau'\\
  	1
  \end{pmatrix} = \begin{pmatrix}
  	\uptau'\\
  	1
  \end{pmatrix}.
  \qedhere
\end{equation*}
\end{itemize}
\end{proof}

\begin{prop}
Il gruppo lineare speciale \(\operatorname{SL}(2;\Z)\) \href{20260201185059-azione_di_gruppo.org}{agisce} su \(\mathds{H}\) tramite
\begin{equation*}
\begin{pmatrix}
a & b\\
c & d
\end{pmatrix} \in \operatorname{SL}(2;\Z),\qquad \begin{aligned}
\mathds{H} & \longrightarrow \mathds{H}\\
\uptau &\longmapsto \frac{a\uptau + b}{c\uptau + d}.
\end{aligned}
\end{equation*}
\end{prop}

\begin{oss}
In particolare, quindi, \(X_{\uptau} \cong X_{\uptau'}\) se e solo se \(\uptau,\uptau'\) sono nella stessa \href{20260201185059-azione_di_gruppo.org}{orbita} per questa azione. In definitiva, si ha
\begin{equation*}
\begin{tikzcd}[ampersand replacement=\&]
	{\set{\text{tori complessi}}/\text{iso}} \&\& {\set{X_{\uptau}}_{\uptau \in \mathds{H}}/\text{iso}} \&\& {\mathds{H}/\operatorname{SL}(2;\Z)}
	\arrow["{{1\duepunti 1}}", tail reversed, from=1-1, to=1-3]
	\arrow["{1\duepunti 1}", tail reversed, from=1-3, to=1-5]
\end{tikzcd}
\end{equation*}
\end{oss}
\end{document}
