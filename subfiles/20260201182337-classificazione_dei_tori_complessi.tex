% Intended LaTeX compiler: pdflatex
\documentclass[../main]{subfiles}

\use{upgreek}
\def\tau{\uptau}


\begin{document}

\section{Classificazione dei tori complessi}
\label{sec:orgd8123db}
Si consideri il semipiano superiore di \(\C\):
\begin{equation*}
\mathds{H} \coloneqq \set{z \in \C \mid \operatorname{Im} z > 0}.
\end{equation*}
Per ogni \(\tau \in \mathds{H}\), sia \(L_{\tau}\) il reticolo generato da \(1\) e \(\tau\):
\begin{equation*}
L_{\tau} \coloneqq \set{z_{1} + z_{2}\,\tau \mid z_{1},z_{2} \in \Z}
\end{equation*}
e sia \(X_{\tau} \coloneqq \C/L_{\tau}\) il \href{20260127113001-toro_complesso.org}{toro complesso} dato dal \href{20250127093819-quoziente_di_gruppo_e_sottogruppo.org}{quoziente} con \(L_{\tau}\).

\begin{prop}
Sia \(Y\) un toro complesso qualsiasi. Allora esiste \(\tau \in \mathds{H}\) tale che
\begin{equation*}
Y = X_{\tau}.
\end{equation*}
\end{prop}

\begin{proof}
Sia \(Y=\C/M\), con \(M\), e siano \(w_{1},w_{2} \in \C\) tale che
\begin{equation*}
M \coloneqq \set{z_{1}\,w_{1}+z_{2}\,w_{2} \mid z_{1},z_{2} \in \Z}
\end{equation*}
Allora \(w_{1},w_{2}\) sono \(\R\)-\href{20241212142019-insiemi_linearmente_indipendenti.org}{linearmente indipendenti}, e \(w_{1} \neq 0\). Sia \(\gamma \coloneqq \frac{1}{w_{1}}\).

Allora \(\gamma\cdot M\) è il reticolo generato da \(1,\frac{\omega_{2}}{\omega_{1}}\), linearmente indipendenti su \(\R\). Pertanto, \(\frac{w_{2}}{w_{1}}\notin \R\).

\begin{itemize}
\item Se \(\operatorname{Im}\frac{w_{2}}{w_{1}} > 0\), allora \(\tau\coloneqq \frac{w_{2}}{w_{1}}\) e
\begin{equation*}
  \gamma \cdot M = L_{\tau}
\end{equation*}
e \href{20260201182313-isomorfismo_tra_tori_complessi.org}{quindi} \(Y \cong X_{\tau}\).

\item Se \(\operatorname{Im}\frac{w_{2}}{w_{1}} < 0\), allora \(\tau\coloneqq - \frac{w_{2}}{w_{1}}\) e\footnote{Ovviamente \(\left(1,\frac{w_{2}}{w_{1}}\right)\) e \(\left(1,-\frac{w_{2}}{w_{1}}\right)\) generano lo stesso reticolo.}
\begin{equation*}
  \gamma \cdot M = L_{\tau}
\end{equation*}
e \href{20260201182313-isomorfismo_tra_tori_complessi.org}{quindi} \(Y \cong X_{\tau}\).
\qedhere
\end{itemize}
\end{proof}

La situazione quindi è la seguente:\footnote{A meno di \href{20260128144717-isomorfismo_tra_superfici_di_riemann.org}{isomorfismo} tra \href{20260127112828-superficie_di_riemann.org}{superfici di Riemann}.}
\begin{equation*}
\begin{tikzcd}[ampersand replacement=\&]
	{\set{\text{tori complessi}}/\text{iso}} \&\&\& {\set{X_{\tau}}_{\tau \in \mathds{H}}/\text{iso}}
	\arrow["{1\duepunti 1}", tail reversed, from=1-1, to=1-4]
\end{tikzcd}
\end{equation*}

\begin{prop}
Siano \(\tau,\tau' \in \mathds{H}\). Sono fatti equivalenti:
\begin{enumerate}
\item sono isomorfi: \(X_{\tau} \cong X_{\tau'}\);
\item esiste \(\left(\begin{smallmatrix} a & b\\ c & d\end{smallmatrix}\right) \in \operatorname{SL}(2;\Z)\)\footnote{\(\operatorname{SL}(n, \K)\) è il \href{20250113142317-gruppo_lineare_speciale.org}{gruppo lineare speciale}} tale che
\begin{equation*}
 \tau = \frac{a\tau' + b}{c\tau' + d}.
\end{equation*}
\end{enumerate}
\end{prop}

\begin{proof}
Per la \href{20260201182313-isomorfismo_tra_tori_complessi.org}{caratterizzazione dei tori isomorfi}, \(X_{\tau}\cong X_{\tau'}\) se e solo se esiste \(\gamma \in \C\setminus \set{0}\) tale che \(\gamma\cdot L_{\tau} = L_{\tau'}\), dove
\begin{align*}
L_{\tau} &= \set{z_{1}+\tau \, z_{2} \mid z_{1},z_{2} \in \Z}\\
L_{\tau'} &= \set{z_{1}+\tau' \, z_{2} \mid z_{1},z_{2} \in \Z}\\
\gamma \cdot L_{\tau} &= \set{\gamma\cdot(z_{1}+\tau \, z_{2}) \mid z_{1},z_{2} \in \Z} = \set{z_{1}\,\gamma + z_{2}\, (\gamma\tau) \mid z_{1},z_{2} \in \Z}
\end{align*}
ovvero se e solo se esiste \(\gamma \in \C\setminus\set{0}\) tale che \(\gamma,\gamma\tau \in L_{\tau'}\), e lo generano come gruppo additivo.

(\(\Rightarrow\)): Siccome \(\gamma, \gamma\tau \in L_{\tau'}\) allora esistono \(a,b,c,d \in \Z\) tali che
\begin{align*}
0 \neq \gamma &= c\tau' + d\\
\gamma\tau &= a\tau' + b
\end{align*}
Allora \(\tau = \frac{a\tau'+b}{c\tau' + d}\), e \(\left(\begin{smallmatrix} a & b\\ c & d \end{smallmatrix}\right) \in \Z^{2\times 2}\):
\begin{equation*}
\begin{pmatrix}
\gamma\tau\\
\gamma
\end{pmatrix} =
\begin{pmatrix}
a & b\\
c & d
\end{pmatrix}\
\begin{pmatrix}
\tau'\\
1
\end{pmatrix}
\end{equation*}

Siccome \(1, \tau' \in \gamma L_{\tau}\) allora esistono \(a',b',c',d' \in \Z\) tali che
\begin{align*}
1 &= \gamma(c'\tau + d')\\
\tau' &= \gamma(a'\tau + b')
\end{align*}
Allora \(\left(\begin{smallmatrix} a' & b'\\ c' & d' \end{smallmatrix}\right) \in \Z^{2\times 2}\) è tale che :
\begin{equation*}
\begin{pmatrix}
\tau'\\
1
\end{pmatrix} =
\begin{pmatrix}
a' & b'\\
c' & d'
\end{pmatrix}\
\begin{pmatrix}
\gamma \tau\\
\gamma
\end{pmatrix}
\end{equation*}

Segue che \(\left(\begin{smallmatrix} a & b\\ c & d \end{smallmatrix}\right)\) è invertibile, e con inversa in \(\Z^{2\times 2}\), pertanto
\begin{equation*}
\det \begin{pmatrix} a' & b'\\ c' & d' \end{pmatrix} \cdot \det \begin{pmatrix} a & b\\ c & d \end{pmatrix} = 1
\end{equation*}
e devono entrambi essere numeri interi. Segue che
\begin{equation*}
\det \begin{pmatrix} a & b\\ c & d \end{pmatrix} = \pm 1
\end{equation*}

Inoltre si ha che\footnote{\(\overline{\tau'}\) è il \href{20260201232630-complesso_coniugato.org}{complesso coniugato}}
\begin{align*}
\operatorname{Im} \tau &= \operatorname{Im}\left( %
\frac{a\tau'+ b}{c\tau' + d}
\right) = \operatorname{Im}\left( %
\frac{(a\tau'+ b)(c\overline{\tau'}+d)}{|c\tau' + d|^{2}}
\right)\\[2ex]
&= \operatorname{Im}\left( %
\frac{%
	ac \, |\tau'|^{2} + ad \,\tau' + bc\, \overline{\tau'} + bc%
}{|c\tau' + d|^{2}}
\right)\\
&= \frac{1}{|c\tau' + d|^{2}} \cdot \operatorname{Im}(ac \, |\tau'|^{2} + ad \,\tau' + bc\, \overline{\tau'} + bc)\\
&= \frac{1}{|c\tau' + d|^{2}} \cdot \operatorname{Im}(ad \,\tau' + bc\, \overline{\tau'})\\
&= \frac{1}{|c\tau' + d|^{2}} \cdot \bigg[ad\,\operatorname{Im}(\tau') + bc\, \operatorname{Im}(\overline{\tau'})\bigg]\\
&= \frac{1}{|c\tau' + d|^{2}} \cdot \bigg[ad\,\operatorname{Im}(\tau') - bc\, \operatorname{Im}({\tau'})\bigg]\\
&= \frac{1}{|c\tau' + d|^{2}} \cdot (ad-bc)\operatorname{Im}(\tau')\\
&= \frac{1}{|c\tau' + d|^{2}} \cdot \det\begin{pmatrix}
a & b\\
c & d
\end{pmatrix}\cdot \operatorname{Im}(\tau')
\end{align*}
e quindi, per concordanza dei segni (\(\operatorname{Im}\tau > 0\), \(\operatorname{Im}\tau' >0\)), si deve avere \(\det\left(\begin{smallmatrix} a & b\\c & d \end{smallmatrix}\right) > 0'\).

(\(\Leftarrow\)): Supponiamo quindi che \(\tau = \frac{a\tau'+b}{c\tau' + d}\), con
\begin{equation*}
\begin{pmatrix}
a & b\\
c & d
\end{pmatrix} \in \operatorname{SL}(2;\Z).
\end{equation*}

Sia \(\gamma \coloneqq c\tau' + d\).
\begin{itemize}
\item \(\gamma \neq 0\), poiché \((c,d) \neq (0,0)\) visto che la matrice è invertibile, e \(\tau'\notin \R\).
\item Ovviamente \(\gamma, \gamma\,\tau \in L_{\tau'}\), in quanto
\begin{equation*}
  \gamma\,\tau = \gamma\, \frac{a\tau'+b}{c\tau' + d} = a\tau' + b.
\end{equation*}
\item Per dimostrare che \(\gamma,\gamma\,\tau\) generino \(L_{\tau'}\) come gruppo additivo, è sufficiente scrivere \(1,\tau'\) come \(\Z\)-combinazione lineare di \(\gamma,\gamma\,\tau\). In particolare, sia \(A' \in \Z^{2\times 2}\) la matrice inversa di \(\left(\begin{smallmatrix} a & b\\ c & d\end{smallmatrix}\right)\). Allora
\begin{equation*}
  A'\cdot \begin{pmatrix}
  	\gamma\,\tau\\
  	\gamma
  \end{pmatrix} = A' \cdot \begin{pmatrix}
  	a & b\\
  	c & d
  \end{pmatrix} \cdot \begin{pmatrix}
  	\tau'\\
  	1
  \end{pmatrix} = \begin{pmatrix}
  	\tau'\\
  	1
  \end{pmatrix}.
  \qedhere
\end{equation*}
\end{itemize}
\end{proof}

\begin{prop}
Il gruppo lineare speciale \(\operatorname{SL}(2;\Z)\) \href{20260201185059-azione_di_gruppo.org}{agisce} su \(\mathds{H}\) tramite
\begin{equation*}
\begin{pmatrix}
a & b\\
c & d
\end{pmatrix} \in \operatorname{SL}(2;\Z),\qquad \begin{aligned}
\mathds{H} & \longrightarrow \mathds{H}\\
\tau &\longmapsto \frac{a\tau + b}{c\tau + d}.
\end{aligned}
\end{equation*}
\end{prop}

\begin{oss}
In particolare, quindi, \(X_{\tau} \cong X_{\tau'}\) se e solo se \(\tau,\tau'\) sono nella stessa \href{20260201185059-azione_di_gruppo.org}{orbita} per questa azione. In definitiva, si ha
\begin{equation*}
\begin{tikzcd}[ampersand replacement=\&]
	{\set{\text{tori complessi}}/\text{iso}} \&\& {\set{X_{\tau}}_{\tau \in \mathds{H}}/\text{iso}} \&\& {\mathds{H}/\operatorname{SL}(2;\Z)}
	\arrow["{{1\duepunti 1}}", tail reversed, from=1-1, to=1-3]
	\arrow["{1\duepunti 1}", tail reversed, from=1-3, to=1-5]
\end{tikzcd}
\end{equation*}
\end{oss}
\end{document}
