% Intended LaTeX compiler: pdflatex
\documentclass[../main]{subfiles}


\begin{document}

Sia \(X\) uno \href{20250103145124-topologia.org}{spazio topologico}.
\begin{definizione}
Un \href{20250103163814-sottospazio_topologico.org}{sottospazio} \(A \subseteq X\), con l'inclusione
\(i_{A}: A\hookrightarrow X\)
si dice \textbf{retratto di deformazione di \(X\)} se esiste una \href{20250122155714-retratto_di_uno_spazio_topologico.org}{retrazione}
\(r: X\longrightarrow A\)
tale che \(i_{A}\circ r \sim \operatorname{Id}_{X}\)\footnote{Qui \(\sim\) indica \href{20250121094654-omotopia_tra_funzioni_continue.org}{due funzioni omotope}.} ed inoltre la funzione
\begin{equation*}
H:X \times [0,1]\longrightarrow X
\end{equation*}
omotopia tra \(i_{A}\circ r\) e \(\operatorname{Id}_{X}\) è tale che
\begin{equation*}
\forall\, x \in X:\qquad H(x,0) = i_{A}\circ r (x), \quad H(x,1) = x
\end{equation*}
ed inoltre \(\restriction{H(\cdot, t)}{A} = \operatorname{Id}_{A}\), ovvero:
\begin{equation*}
\forall\,a \in A, \forall\, t \in [0,1],\qquad H(a,t) = a
\end{equation*}
\end{definizione}
\end{document}
