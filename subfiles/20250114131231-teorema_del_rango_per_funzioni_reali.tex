% Intended LaTeX compiler: pdflatex
\documentclass[../main]{subfiles}

\usepackage[hyperref]{biblatex}
\date{}
\title{}
\begin{document}

\section{Teorema del rango per funzioni reali}
\label{sec:orgfbafa71}
\subsection{Teorema}
\label{sec:org3a82d40}
Sia \(\Omega \subseteq \R^{n}\) un aperto, e sia \(F:\Omega \longrightarrow \R^{m}\) una \href{20250113125602-classe_c_di_una_funzione.org}{funzione \(C^{\infty}\) (come funzione reale)} tale che per ogni \(p \in \Omega\)
\begin{equation*}
\operatorname{rank}\operatorname{J}_{F}(p)=k
\end{equation*}
(vedi \href{20250104170945-rango_di_una_matrice.org}{Rango di una matrice}, \href{20250114123754-matrice_jacobiana.org}{Matrice Jacobiana}, \href{20250113125641-differenziale_di_una_funzione_reale.org}{Differenziale di una funzione reale}).

Allora per ogni \(p \in \Omega\) esiste una \href{20250111092123-varieta_topologica.org}{carta} \((U,\varphi)\) di \(\R^{n}\) con \(p \in U\) ed esiste una \href{20250111092123-varieta_topologica.org}{carta} \((V,\psi)\) di \(\R^{m}\), con \(F(U) \subseteq V\) tali che
\begin{equation*}
\psi\circ F\circ\varphi^{-1}(x^{1},\dots,x^{k},x^{k+1},\dots,x^{n}) = (x^{1},\dots,x^{k},0,\dots,0)
\end{equation*}
\subsubsection{{\bfseries\sffamily TODO} Osservazione\hfill{}\textsc{matematica\_lm:geo\_diff}}
\label{sec:orgeca1612}
Questo teorema implica il \href{20250113125429-teorema_dell_inversa_locale.org}{TIV}.
\subsubsection{{\bfseries\sffamily TODO} Dimostazione\hfill{}\textsc{matematica\_lm:geo\_diff}}
\label{sec:org6f8b741}
\end{document}
