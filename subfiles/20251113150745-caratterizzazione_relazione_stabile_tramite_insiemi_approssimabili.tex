% Intended LaTeX compiler: pdflatex
\documentclass[../main]{subfiles}


\begin{document}

\section{Caratterizzazione relazione stabile tramite insiemi approssimabili}
\label{sec:org252ffb4}
\def\U{\mathcal{U}}
\def\L{\mathcal{L}}
\def\X{\mathcal{X}}
\def\Z{\mathcal{Z}}
\def\D{\mathcal{D}}
%
\def\eq{{\rm eq}}
\def\Ueq{\U^\eq}
%
\def\orbita{\mathcal{O}}
\def\Aut{\operatorname{Aut}}
%
\def\tc{\mid}
\def\tp{\operatorname{tp}}
\def\EMtp{\operatorname{EM}\text{-}\operatorname{tp}}
\def\<{\langle}
\def\>{\rangle}
%
\def\restricted#1{\,\mathord{\upharpoonright}{{\scriptstyle #1}}}
\def\equivalentover#1{\mathrel{\equiv_{ #1 }}}

%% NON FORKING
\def\nonforkSymbol{%
\mathbin{\raise1.8ex%
\rlap{\kern0.6ex\rule{0.6ex}{0.1ex}}%
\rlap{\kern1.1ex\rule{0.1ex}{1.9ex}}\raise-0.3ex\hbox{$\smile$}}}
\def\defaultnonforkmodel{M}
\def\nonfork{\nonforkSymbol}
\renewcommand{\nonfork}[1][\defaultnonforkmodel]{%
\mathrel{\nonforkSymbol_{#1}}}

Siano \(\X, \Z\) due \href{20250130104331-insieme_mk.org}{insiemi} arbitrari, e sia \(\pi \subseteq \X\times \Z\) una \href{20250202170607-classe_relazione_binaria.org}{relazione} binaria. Si scriverà indifferentemente
\begin{equation*}
(x,z) \in \pi,\qquad \pi(x,z).
\end{equation*}

\begin{thm}
Se \(\pi \subseteq \X \times \Z\) è \href{20251113150354-relazione_stabile.org}{stabile} e \(\D \subseteq \Z\) è \href{20251113150652-insieme_approssimabile_da_una_relazione.org}{approssimabile} da \(\pi\), allora esiste una \href{20250206170922-sequenze_e_stringhe.org}{sequenza} (una \href{20250104111539-spazio_delle_matrici.org}{matrice})
\(\< a_{ij} \mid i,j < m\>\) in \(\X\) tali che
\begin{equation*}
\D = \bigcup_{i=1}^{m} \bigcap_{j=1}^{m} \pi(a_{ij}, \Z)
\end{equation*}
dove con \(\pi(a, \Z) = \set{b \in \Z \mid \pi(a,b)}\).
\end{thm}
\section{Caratterizzazione relazione stabile in un modello mostro tramite insiemi approssimabili}
\label{sec:org7c40eed}
\def\U{\mathcal{U}}
\def\L{\mathcal{L}}
\def\X{\mathcal{X}}
\def\Z{\mathcal{Z}}
\def\D{\mathcal{D}}
%
\def\eq{{\rm eq}}
\def\Ueq{\U^\eq}
%
\def\orbita{\mathcal{O}}
\def\Aut{\operatorname{Aut}}
%
\def\tc{\mid}
\def\tp{\operatorname{tp}}
\def\EMtp{\operatorname{EM}\text{-}\operatorname{tp}}
\def\<{\langle}
\def\>{\rangle}
%
\def\restricted#1{\,\mathord{\upharpoonright}{{\scriptstyle #1}}}
\def\equivalentover#1{\mathrel{\equiv_{ #1 }}}

%% NON FORKING
\def\nonforkSymbol{%
\mathbin{\raise1.8ex%
\rlap{\kern0.6ex\rule{0.6ex}{0.1ex}}%
\rlap{\kern1.1ex\rule{0.1ex}{1.9ex}}\raise-0.3ex\hbox{$\smile$}}}
\def\defaultnonforkmodel{M}
\def\nonfork{\nonforkSymbol}
\renewcommand{\nonfork}[1][\defaultnonforkmodel]{%
\mathrel{\nonforkSymbol_{#1}}}

Si utilizza la \href{20250612143636-notazione_teoria_dei_modelli.org}{Notazione della TEORIA DEI MODELLI}

Sia \(\L\) un \href{20250130162057-linguaggio_del_prim_ordine.org}{linguaggio}, \(T\) una \href{20250130114950-teoria_del_prim_ordine.org}{teoria} \href{20250131123151-teoria_completa.org}{completa} senza \href{20250131122945-modello_di_un_insieme_di_formule.org}{modelli} finiti e \(\U\) un \href{20250617095548-modello_lambda_saturo.org}{modello saturo} di \href{20241213101756-cardinalita.org}{cardinalità} \href{20250211123155-cardinale_limite_forte.org}{inaccessibile} \(\kappa>\card{\L}+ \omega\). \(\U\) è un \href{20250617102733-modello_mostro.org}{modello mostro}.

Una qualsiasi \href{20250131103317-formula_del_prim_ordine.org}{formula} \(\psi(x;z) \in \L(\U)\) si indende come relazione, vedendola come
\begin{equation*}
\psi(\U^{x};\U^{z}) \subseteq \U^{x} \times \U^{z}.
\end{equation*}
\begin{thm}
Se \(\varphi(x;z) \in \L(\U)\) è \href{20251113150436-relazione_stabile_in_un_modello_mostro.org}{stabile} e \(\D\) è \href{20251113150815-insieme_approssimabile_da_una_relazione_in_un_modello_mostro.org}{approssimabile da \(\varphi(x;z)\)} allora esiste \(\langle a_{ij} \mid i,j<m\rangle\) in \(\U^{x}\) tale che\footnote{con \(\varphi(a; \U^{z})\) si intende l'\href{20250131122913-soddisfazione_di_una_formula.org}{insieme definito da \(\varphi(a;z)\)}.}
\begin{equation*}
\D = \bigcup_{i=1}^{m}\bigcap_{j=1}^{m} \varphi(a_{ij}, \U^{z}).
\end{equation*}
\label{teorema_obiettivo_sjdancvlkjfbhsdlkvsj}
\end{thm}
\begin{lem}
Se \(\varphi(x;z) \in \L(\U)\) è stabile, \(\D\) \href{20251123155658-insieme_approssimabile_dal_basso_da_una_relazione.org}{approssimato da \(\varphi(x;z)\) dal basso}, allora esiste
\(\< a_{i} \mid i<m \>\)
tale che
\begin{equation*}
\D = \bigcup_{i=1}^{m} \varphi(a_{i}, \U^{z}).
\end{equation*}
\label{lem:lemma1_ouiahbkjlihugyfjcghvjkguioyftucjghvkguoyiftucgjhvk}
\end{lem}
\begin{lem}
Se \(\varphi(x;z)\) è stabile e \(\D\) è approssimata da \(\varphi(x;z)\) allora \(\D\) è approssimata dal basso dalla formula
\begin{equation*}
\psi(x_{1},\dots,x_{m}; z) = \bigwedge_{i=1}^{m} \varphi(x_{i};z).
\end{equation*}
\label{lem:lemma2_onsaihkjdhbmhjkbhjvbkjh}
\end{lem}
\begin{proof}
(del Teorema~\ref{teorema_obiettivo_sjdancvlkjfbhsdlkvsj}).
Per il Lemma~\ref{lem:lemma2_onsaihkjdhbmhjkbhjvbkjh} \(\D\) è approssimata dal basso da
\begin{equation}
\psi(x_{1},\dots,x_{m}; z) = \bigwedge_{i=1}^{m} \varphi(x_{i};z) %
\label{eq:thethomare}
\end{equation}
e per il Lemma~\ref{lem:lemma1_ouiahbkjlihugyfjcghvjkguioyftucjghvkguoyiftucgjhvk} esiste
\(\langle a_{i{1}},\dots,a_{im} \mid i <m \rangle\)
tale che
\begin{equation*}
\D = \bigcup_{i=1}^{m} \psi(a_{i{1}},\dots,a_{im}; \U^{z}).
\end{equation*}
Per la~\eqref{eq:thethomare} si ha
\begin{equation*}
\psi(a_{i{1}},\dots,a_{im}; \U^{z}) = \bigcap_{j=1}^{m} \varphi(a_{ij}; \U^{z})
\end{equation*}
e pertanto si ha la tesi:
\begin{equation*}
\D = \bigcup_{i=1}^{m} \bigcap_{j=1}^{m} \varphi(a_{ij}; \U^{z}).%
\qedhere
\end{equation*}
\end{proof}
\end{document}
