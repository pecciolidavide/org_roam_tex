% Intended LaTeX compiler: pdflatex
\documentclass[../main]{subfiles}

\def\HdRham{\operatorname{H}_{\text{dR}}}


\begin{document}

\section{Varietà differenziabili diffeomorfe hanno stessa Coomologia di de Rham}
\label{sec:orgc36b310}
Siano \(M, N\) due \href{20250113115909-struttura_differenziabile.org}{varietà differenziabili} di dimensione, rispettivamente, \(m\) ed \(n\). Si indichi con \(\HdRham\) la \href{20251115172442-gruppo_di_coomologia_di_de_rham.org}{coomologia di De Rham}.
\begin{thm}
Se \(F:M\to N\) è un \href{20250113172924-diffeomorfismo_tra_varieta_differenziabili.org}{diffeomorfismo} tra \href{20250113115909-struttura_differenziabile.org}{varietà differenziabili}, allora per ogni \(k \in \N\)  si ha l'\href{20250113125833-isomorfismo_tra_spazi_vettoriali.org}{isomorfismo tra spazi vettoriali}:
\begin{equation*}
\HdRham^{k}(M) \cong \HdRham^{k}(N)
\end{equation*}
\end{thm}

\begin{proof}
Fissato \(k \in \N\), si consideri il \href{20251115174001-pullback_di_una_funzione_tra_varieta_differenziabili.org}{pullback} sulle \href{20251115155511-forma_differenziale_in_un_punto.org}{forme}:
\begin{equation*}
F^{*}:\Omega^{k}(N) \to \Omega^{k}(M)
\end{equation*}
\href{20251115174001-pullback_di_una_funzione_tra_varieta_differenziabili.org}{che soddisfa} \(\dif F^{*}\omega = F^{*} \dif \omega\) (dove \(\dif\) è la \href{20251115160537-differenziale_di_una_forma.org}{derivata esterna}).
\begin{itemize}
\item Se \(\omega \in \Omega^{k}(N)\) è \href{20251115172517-forma_differenziale_chiusa.org}{chiusa}, allora \(\dif \omega = 0\)
\begin{equation*}
  \dif F^{*}\omega = F^{*}\dif \omega = 0
\end{equation*}
e pertanto \(F^{*}\omega\) è chiusa.
\item Se \(\omega \in \Omega^{k}(N)\) è \href{20251115172517-forma_differenziale_chiusa.org}{esatta}, allora esiste \(\eta\) tale che \(\omega = \dif\eta\):
\begin{equation*}
  F^{*}\omega = F^{*}\dif\eta = \dif F^{*}\eta
\end{equation*}
e quindi \(F^{*}\omega\) è esatta.
\end{itemize}

Ragionando allo stesso modo con \(F^{-1}\) e \((F^{-1})^{*}\) si ottiene la tesi.
\end{proof}
\begin{thm}
Se \(F:M\to N\) è un \href{20250113172924-diffeomorfismo_tra_varieta_differenziabili.org}{diffeomorfismo} tra \href{20250113115909-struttura_differenziabile.org}{varietà differenziabili}, allora si ha l'\href{20251201160758-isomorfismo_tra_algebre_graduate.org}{isomorfismo tra algebre graduate}:
\begin{equation*}
\HdRham^{\bullet}(M) \cong \HdRham^{\bullet}(N)
\end{equation*}
\end{thm}
\end{document}
