% Intended LaTeX compiler: pdflatex
\documentclass[../main]{subfiles}


\begin{document}

\textbf{\textbf{\uline{NOTA}: quando si parla di \uline{classi}, se ne parla nell'ambito della \href{20250130104245-morse_kelly_set_theory.org}{Morse Kelly Set Theory}; quando si parla di insiemi, il discorso ha validità più generale.}}

Sia \(X\) un insieme (o classe) e sia \(R\) una relazione binaria.
\section{Relazione binaria riflessiva}
\label{sec:org037d5e2}
\(R\) si dice \uline{riflessiva} se \(\forall\,x \in X\ (x\mathrel{R}x)\).
\section{Relazione binaria irriflessiva}
\label{sec:org7df000a}
\(R\) si dice \uline{irriflessiva} se \(\forall\,x \in X\ (x\not\mathrel{R}x)\).
\section{Relazione binaria simmetrica}
\label{sec:org61671ab}
\(R\) si dice \uline{simmetrica} se \(\forall\,x, y \in X (x\mathrel{R}y\implies y\mathrel{R}x)\).
\section{Relazione binaria antisimmetrica}
\label{sec:org805d48b}
\(R\) si dice \uline{antisimmetrica} se \(\forall\,x,y \in X \left((x\mathrel{R}y \,\land\, y\mathrel{R}x)\implies x=y\right)\).
\section{Relazione binaria connessa}
\label{sec:orge681132}
\(R\) si dice \uline{connessa} se \(\forall\,x,y \in X\ (x=y \,\lor\,x\mathrel{R}y \,\lor\,y\mathrel{R}x)\).
\section{Relazione binaria transitiva}
\label{sec:orgc338299}
\(R\) si dice \uline{transitiva} se \(\forall\,x,y,z \in X\ ((x\mathrel{R}y \,\land\, y\mathrel{R}z)\implies x\mathrel{R}z )\).
\end{document}
