% Intended LaTeX compiler: pdflatex
\documentclass[../main]{subfiles}


\begin{document}

\section{Topologia debole di un complesso simpliciale}
\label{sec:org2ae9d05}
\begin{definizione}
Sia \(K\) un \href{20250121124147-complesso_simpliciale.org}{complesso simpliciale} in \(\R^{n+1}\), e sia \(|K|\) il suo \href{20250121124230-supporto_di_un_complesso_simpliciale.org}{supporto}.

Si definsice una \href{20250103145124-topologia.org}{topologia} su \(|K|\), detta topologia debole:
\(C \subseteq |K|\) è \href{20250103145124-topologia.org}{chiuso} se e solo se, per ogni \(\sigma \in K\), \(C\cap \sigma \subseteq \R^{n+1}\) è \href{20250103145124-topologia.org}{chiuso}.
\end{definizione}
\begin{prop}
Se \(K\) è un insieme finito, allora la topologia debole di \(|K|\) è uguale alla \href{20250103163814-sottospazio_topologico.org}{topologia indotta} \(|K| \subseteq \R^{n+1}\).
\end{prop}

\begin{esempio}
In \(\R^{2}\), sia \(K\) formato dai seguenti \href{20250121121923-simplessi.org}{simplessi}: detto \(P_{\theta} = (\cos\theta,\sin\theta) \in \R^{2}\):
\begin{equation*}
K = \set{[0]}\cup\set{[P_{\theta}], [0,P_{\theta}]}_{\theta \in [0,2\pi]}.
\end{equation*}

Si ha che \(|K| = B_{1}(0)\) (vedi \href{20250121130442-palla_n_dimensionale.org}{Palla n-dimensionale}).

Consideiramo l'insieme
\begin{equation*}
C=\set{(x,y) \in |K|: x>0}\cup\set{0}
\end{equation*}
C non è un chiuso nella topologia indotta, ma lo è nella topologia debole, poiché
\begin{equation*}
C=\bigcup_{-\pi/2<\theta<\pi/2} [0,P_{\theta}]
\end{equation*}
\end{esempio}
\end{document}
