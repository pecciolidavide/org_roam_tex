% Intended LaTeX compiler: pdflatex
\documentclass[../main]{subfiles}


\begin{document}

\section{Classe propria e transitiva sottoclasse di Ord è Ord}
\label{sec:org3e2a908}
Contesto: \href{20250130104245-morse_kelly_set_theory.org}{Morse Kelly Set Theory}
\subsection{Proposizione}
\label{sec:org90aff3f}
Sia \(\operatorname{Ord}\) la \href{20250130104320-classe_mk.org}{classe} degli \href{20250203111003-ordinali.org}{ordinali}, e sia \(C \subseteq \operatorname{Ord}\) una \href{20250131155822-operazioni_insiemistiche_tra_classi_mk.org}{sottoclasse}.
Se \(C\) è una \href{20250130104320-classe_mk.org}{classe propria} e \href{20250203110714-classe_transitiva.org}{transitiva} allora \(C=\operatorname{Ord}\).
\subsubsection{Dimostrazione\hfill{}\textsc{IdL:matematica\_lm}}
\label{sec:org99205ea}
Per assurdo, sia \(\alpha \in \operatorname{Ord}\) tale che \(\alpha\notin C\).

Per ogni \(\beta \in C \subseteq \operatorname{Ord}\) \href{20250203111003-ordinali.org}{si ha che} una delle tre è verificata:
\begin{equation*}
\alpha \in \beta;\quad \alpha=\beta; \quad \beta \in \alpha
\end{equation*}
\begin{itemize}
\item Se \(\alpha \in \beta \in C\), siccome \(C\) è transitiva, allora \(\alpha \in C\). Assurdo.
\item Se \(\alpha = \beta \in C\) allora \(\alpha \in C\). Assurdo
\end{itemize}

Dunque \(\beta \in \alpha\), e pertanto \(C \subseteq \alpha\). \href{20250131160822-ogni_sottoclasse_di_un_insieme_e_un_insieme_mk.org}{Quindi} \(C\) è un \href{20250130104331-insieme_mk.org}{insieme}. Assurdo.
\end{document}
