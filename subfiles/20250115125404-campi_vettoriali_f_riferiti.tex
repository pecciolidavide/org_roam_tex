% Intended LaTeX compiler: pdflatex
\documentclass[../main]{subfiles}


\begin{document}

Siano \(M,N\) \href{20250113115909-struttura_differenziabile.org}{varietà differenziabili}, e sia \(F:M\longrightarrow N\) una \href{20250113144722-funzioni_cinfinito_tra_varieta_differenziabili.org}{funzione \(C^{\infty}\)} con \href{20250114111331-differenziale_di_una_funzione_tra_varieta_differenziabili.org}{differenziale}
\begin{equation*}
\restriction{F_{\star}}{p}: \operatorname{T}_{p} M \longrightarrow \operatorname{T}_{F(p)}N
\end{equation*}

Siano \(X \in\Gamma(M)\), \(Y \in \Gamma(N)\) \href{20250115110259-campo_vettoriale_su_una_varieta_differenziabile.org}{campi vettoriali}.
\section{Definizione}
\label{sec:org6d51dca}
\(X\) e \(Y\) si dicono \(F\)-riferiti se per ogni \(p \in M\)
\begin{equation*}
\restriction{F_{\star}}{p} (X_{p}) = Y_{F(p)}
\end{equation*}
\section{Teorema di caratterizzazione}
\label{sec:org6e9e2d6}
\(X,Y\) sono \(F\)-riferiti se e solo se, per ogni \(g \in \operatorname{C}^{\infty}(N)\) (vedi \href{20250113144722-funzioni_cinfinito_tra_varieta_differenziabili.org}{Anello delle funzioni da una varietà ai reali Cinfinito}) si ha che
\begin{equation*}
X(g\circ F) = Y(g) \circ F
\end{equation*}
(vedi \href{20250115115535-azione_di_un_campo_vettoriale_su_una_funzione.org}{Azione di un campo vettoriale su una funzione})
\end{document}
