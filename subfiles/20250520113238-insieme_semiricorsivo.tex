% Intended LaTeX compiler: pdflatex
\documentclass[../main]{subfiles}

\usepackage[hyperref]{biblatex}
\date{}
\title{}
\begin{document}

\section{Insieme semiricorsivo}
\label{sec:orgf742e31}
In questa sezione si identificano i \href{20250131155822-operazioni_insiemistiche_tra_classi_mk.org}{sottinsiemi} di \(\N^{k}\) (vedi \href{20250202130045-insieme_dei_numeri_naturali_mk.org}{Insieme dei numeri naturali MK}) con i \href{20250131103317-formula_del_prim_ordine.org}{predicati} \(k\)-ari (ovvero con \(k\) \href{20250131103429-variabile_libera_di_una_formula.org}{variabili libere}), per mezzo degli \href{20250131122913-soddisfazione_di_una_formula.org}{insiemi di verità}.

Scriveremo indifferentemente \((x_{1},\dots,x_{k}) \in P\) oppure \(P(x_{1},\dots,x_{k})\) per dire che \(\N\vDash P(x_{1},\dots,x_{k})\).
\subsection{Definizione}
\label{sec:orgfa1175e}

Un insieme/predicato \(P \subseteq \N^{k}\) è \uline{semiricorsivo} se è la proiezione di un \href{20250216173925-insieme_ricorsivo.org}{sottoinsieme ricorsivo} di \(\N^{k+1}\), ovvero se esiste \(R \subseteq \N^{k+1}\) ricorsivo tale che
\begin{equation*}
P(\oldvec{x}) \quad\iff\quad \exists\, y \ R(\oldvec{x},y).
\end{equation*}
\subsection{Insiemi ricorsivi sono semiricorsivi}
\label{sec:org942f1cb}
Ogni insieme ricorsivo è \uline{semiricorsivo}, ma \href{20250520113349-teorema_di_post.org}{non vale il viceversa}.
\end{document}
