% Intended LaTeX compiler: pdflatex
\documentclass[../main]{subfiles}


\begin{document}

\section{Isomorfismo tra tori complessi}
\label{sec:org4111ce7}
\begin{prop}
Siano \(X = \C/L\), \(Y=\C/M\) \href{20260127113001-toro_complesso.org}{tori complessi}, \href{20250127093819-quoziente_di_gruppo_e_sottogruppo.org}{quozienti} con \(L, M \subseteq \C\) reticoli. Sono fatti equivalenti
\begin{enumerate}
\item \(X\cong Y\) sono \href{20260128144717-isomorfismo_tra_superfici_di_riemann.org}{isomorfi};
\item esiste \(\gamma \in \C\setminus\set{0}\) tale che \(\gamma\cdot L = M\).
\end{enumerate}
\end{prop}

\begin{proof}
(\(2.\Rightarrow 1.\)): la funzione \(f:\C\to \C: z\mapsto \gamma z\) \href{20260201172613-mappe_tra_tori_complessi.org}{induce} un biolomorfismo \(F:X\to Y\).

(\(1.\Rightarrow 2.\)): Sia \(F:X\to Y\) biolomorfismo. A meno di comporre per una traslazione in \(Y\), possiamo supporre che \(F(0) = 0\). \href{20260201172613-mappe_tra_tori_complessi.org}{Quindi} \(F\) è indotto da una \(f:\C\to \C\) lineare, con \(f(L) \subseteq M\). Siccome \(F\) è biunivoca, allora \(f(L) = M\).
\end{proof}
\end{document}
