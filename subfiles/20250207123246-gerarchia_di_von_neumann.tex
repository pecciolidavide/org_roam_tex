% Intended LaTeX compiler: pdflatex
\documentclass[../main]{subfiles}


\begin{document}

\section{Gerarchia di Von Neumann}
\label{sec:org9f0d4af}
Contesto: \href{20250130104245-morse_kelly_set_theory.org}{Morse Kelly Set Theory}
\begin{definizione}
Sia \(\alpha\) un \href{20250203111003-ordinali.org}{ordinale}. Si definisce \(V_{\alpha}\) come
\begin{equation*}
V_{\alpha} \coloneqq \set{x\mid \operatorname{rank}(x) <\alpha}
\end{equation*}
dove \(\rank\) è il \href{20250207123105-rango_di_un_insieme.org}{rango di \(x\)}. La \uline{gerarchia di Von Neumann} è la \href{20250202170607-classe_relazione_binaria.org}{classe funzione} da \(\operatorname{Ord}\) a \(\operatorname{V}\)\footnote{\(\operatorname{V}\) è la \href{20250203104513-classe_totale.org}{classe totale}.} che ad \(\alpha\) associa \(V_{\alpha}\).
\end{definizione}

Si noti che \(V_{\alpha}\cap \operatorname{Ord} = \alpha\).
\begin{thm}
Per ogni ordinale \(\alpha\), \(V_{\alpha}\) è
\begin{enumerate}
\item un insieme
\item transitivo
\item e vale\footnote{Vedi:
\begin{itemize}
\item \href{20250131155822-operazioni_insiemistiche_tra_classi_mk.org}{Unione generalizzata}
\item \href{20250130104245-morse_kelly_set_theory.org}{Insieme delle parti per MK}
\end{itemize}}
\begin{equation*}
 V_{\alpha} = \bigcup_{\beta<\alpha} \parti{V_{\alpha}}.
\end{equation*}
\end{enumerate}
\end{thm}
\begin{cor}
Si ha che
\begin{enumerate}
\item \(V_{0} = \emptyset\);
\item Se \(\alpha<\beta\) allora \(V_{\alpha} \in V_{\beta}\) e \(V_{\alpha} \subseteq V_{\beta}\);
\item \(V_{\operatorname{S}(\alpha)} = \parti{V_{\alpha}}\)\footnote{\(\operatorname{S}\) è il \href{20250202124648-successore_di_un_insieme_mk.org}{successore}.}
\item se \(\lambda\) è \href{20250203161132-ordinale_limite.org}{limite}, allora
\begin{equation*}
 V_{\lambda}=\bigcup_{\alpha<\lambda} V_{\alpha};
\end{equation*}
\item \(\operatorname{V} = \bigcup_{\alpha \in \operatorname{Ord}} V_{\alpha}\).
\end{enumerate}
\end{cor}
\begin{oss}
Siccome si ha che\footnote{\(\card{\cdot}\) indica la \href{20241213101756-cardinalita.org}{cardinalità}} \(\card{V_{n+1}} = 2^{n}\) per ogni \(n \in \omega\)\footnote{Vedi ``\href{20250203161110-numeri_naturali_sono_ordinali.org}{Ordinale Omega}''}, si ha che \(V_{\omega} \asymp \omega\)\footnote{\(\asymp\) indica due \href{20250619101109-classi_equipotenti.org}{insiemi in biiezione}.} e \(V_{\omega+1}\asymp \parti{\omega}\asymp \R\).
\end{oss}
\begin{prop}
Se \(x,y \in V_{\alpha}\) allora
\begin{align*}
\set{x,y} &\in \parti{V_{\alpha}} = V_{\alpha+1}\\
(x,y) = \set{\set{x},\set{x,y}} &\in\parti{\parti{V_{\alpha}}} = V_{\alpha+1}.
\end{align*}
\end{prop}
\begin{oss}
Segue che \(\N\times\N \subseteq V_{\omega}\), poiché ogni \(n \in \N\) è in \(V_{n+1}\) e \href{20250203161110-numeri_naturali_sono_ordinali.org}{\(\omega\) è limite}.
\end{oss}
\end{document}
