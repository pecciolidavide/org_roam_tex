% Intended LaTeX compiler: pdflatex
\documentclass[../main]{subfiles}


\begin{document}

\section{Funzione discriminatoria (Machine Learning)}
\label{sec:orged09d2f}
\begin{definizione}
Sia \(\mathcal{M}\) la famiglia delle \href{20250625104200-misura_di_baire.org}{misure di Baire} sul cubo \(I^{n} \coloneqq [0,1]^{n} \subseteq \R\), \href{20250625110016-misura_finita.org}{finite}, \href{20250625110024-misura_con_segno.org}{con segno} e \href{20250625110032-misura_regolare.org}{regolari}.

Una funzione \(f: \R\to \R\) si dice \uline{discriminatoria} se per ogni \(\mu \in \mathcal{M}\):
\begin{equation*}
\left(\forall \bm{w} \in \R^{n},\, \forall \theta \in \R\quad \int_{I^{n}} f(\bm{w}\cdot\bm{x}) \dif \mu(\bm{x}) = 0\right)\implies \mu=0
\end{equation*}
\end{definizione}
\begin{definizione}
Una funzione \(\sigma:\R\to \R\) si dice \uline{discriminatoria in senso \(L^{2}\)} se:
\begin{enumerate}
\item \(\forall x \in \R\ 0\le \sigma(x)\le 1\);
\item se \(g \in L^{2}(I_{n})\)\footnote{Con \(I_{n}\) si indica il cubo
\begin{equation*}
I_{n} = [0,1]\times \dots \times [0,1] = [0,1]^{n}.
\end{equation*}
Si vedano ``\href{20250624162220-spazi_lp.org}{Spazi Lp}''\label{org12c1c03}} e
\begin{equation*}
 \forall \bm{w} \in \R^{n}\ \forall \theta \in \R:\qquad\int_{I_{n}} \sigma(\bm{w}\cdot\bm{x} + \theta)g(\bm{x})\dif\bm{x} = 0
\end{equation*}
allora \(g=0\)
\end{enumerate}
\label{def9.3.10}
\end{definizione}
\begin{esempio}
La funzione logistica \(\displaystyle\sigma(x) = \frac{1}{1+e^{-x}}\) è discriminatoria in senso \(L^{2}\).
\label{ex9.3.15}
\end{esempio}
\begin{definizione}
Una \href{20250202170607-classe_relazione_binaria.org}{funzione} \(\sigma:\R\to \R\) si dice \uline{discriminatoria in senso \(L^{1}\)} se:
\begin{enumerate}
\item \(\sigma\) è \href{20250704104947-funzione_misurabile.org}{misurabile} e \href{20250704145518-funzione_limitata.org}{limitata};
\item \(\sigma\) è \href{20250625110110-funzione_sigmoidale.org}{sigmoidale};
\item se \(g \in L^{\infty}(I_{n})\)\textsuperscript{\ref{org12c1c03}} e
\begin{equation*}
 \forall \bm{w} \in \R^{n}\ \forall \theta \in \R:\qquad\int_{I_{n}} \sigma(\bm{w}\cdot\bm{x} + \theta)g(\bm{x})\dif\bm{x} = 0
\end{equation*}
allora \(g=0\)
\end{enumerate}
\label{def9.3.16}
\end{definizione}
\end{document}
