% Intended LaTeX compiler: pdflatex
\documentclass[../main]{subfiles}


\begin{document}

Sia \(R\) un \href{20241205141119-anello.org}{anello} commutativo con unità, e siano \(\mathcal{C}_{\bullet},\mathcal{D}_{\bullet}\) due \href{20250120163114-complesso_di_catene.org}{complessi di catene} di \href{20241205141053-r_moduli.org}{\(R\)-moduli}:
\begin{equation*}
\mathcal{C}_{\bullet}=\set{(C_{n},\partial_{n}^{C})}_{n \in \Z},\qquad \mathcal{D}_{\bullet}=\set{(D_{n},\partial_{n}^{D})}_{n \in \Z}
\end{equation*}
e siano \(f_{\bullet},g_{\bullet}: \mathcal{C}_{\bullet}\to \mathcal{D}_{\bullet}\) due \href{20250120163759-categoria_complessi_di_catene.org}{morfismi},
\begin{equation*}
f_{\bullet} = \set{f_{n}:C_{n}\to D_{n}}_{n \in \Z},\qquad g_{\bullet} = \set{g_{n}: C_{n}\to D_{n}}_{n \in \Z}
\end{equation*}
\begin{definizione}
Le mappe \(f_{\bullet}, g_{\bullet}\) si dicono omotopiche, e si scrive \(f_{\bullet} \mathrel{\sim} g_{\bullet}\) se per ogni \(n \in \Z\) esistono \href{20241206115416-morfismi_r_moduli.org}{morfismi tra \(R\)-moduli}
\begin{equation*}
s_{n} : C_{n}\longrightarrow D_{n+1}
\end{equation*}
tali che
\begin{equation*}
f_{k}-g_{k} = s_{k-1}\circ \partial_{k}^{C} + \partial_{k+1}^{D}\circ s_{k}
\end{equation*}
come raffigurato in Figura~\ref{fig:omotopiamorfismi}
\end{definizione}
\begin{figure}
\begin{equation*}
\begin{tikzcd}[ampersand replacement=\&,sep=huge]
	\cdots \& {C_{n+1}} \& {C_n} \& {C_{n-1}} \& \cdots \\
	\cdots \& {D_{n+1}} \& {D_n} \& {D_{n-1}} \& \cdots
	\arrow["{\partial^C_{n+2}}", from=1-1, to=1-2]
	\arrow["{\partial^C_{n+1}}", from=1-2, to=1-3]
	\arrow["{s_{n+1}}"'{pos=0.2}, color={rgb,255:red,214;green,92;blue,92}, from=1-2, to=2-1]
	\arrow["{f_{n+1}}"', shift right=2, from=1-2, to=2-2]
	\arrow["{g_{n+1}}", shift left=2, from=1-2, to=2-2]
	\arrow["{\partial^C_{n}}", from=1-3, to=1-4]
	\arrow["{s_{n}}"'{pos=0.2}, color={rgb,255:red,214;green,92;blue,92}, from=1-3, to=2-2]
	\arrow["{f_n}"', shift right=2, from=1-3, to=2-3]
	\arrow["{g_n}", shift left=2, from=1-3, to=2-3]
	\arrow["{\partial^C_{n-1}}", from=1-4, to=1-5]
	\arrow["{s_{n-1}}"'{pos=0.2}, color={rgb,255:red,214;green,92;blue,92}, from=1-4, to=2-3]
	\arrow["{f_{n-1}}"', shift right=2, from=1-4, to=2-4]
	\arrow["{g_{n-1}}", shift left=2, from=1-4, to=2-4]
	\arrow["{s_{n-2}}"'{pos=0.2}, color={rgb,255:red,214;green,92;blue,92}, from=1-5, to=2-4]
	\arrow["{\partial^D_{n+2}}"', from=2-1, to=2-2]
	\arrow["{\partial^D_{n+1}}"', from=2-2, to=2-3]
	\arrow["{\partial^D_{n}}"', from=2-3, to=2-4]
	\arrow["{\partial^D_{n-1}}"', from=2-4, to=2-5]
\end{tikzcd}
\end{equation*}
\caption{\label{fig:omotopiamorfismi}\(f_{\bullet},g_{\bullet}\) omotopi.}
\end{figure}
\end{document}
