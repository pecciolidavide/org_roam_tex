% Intended LaTeX compiler: pdflatex
\documentclass[../main]{subfiles}


\begin{document}

\section{Rivestimenti e gruppo fondamentale}
\label{sec:org96b9555}
Sia \(X\) uno \href{20250103145124-topologia.org}{spazio topologico} \href{20250109155715-spazio_topologico_di_hausdorff.org}{T2}, \href{20250111142303-spazio_topologico_a_base_numerabile.org}{a base numerabile}, \href{20250103165325-spazio_topologico_connesso.org}{connesso} e \href{20250113102837-spazio_topologico_localmente_connesso_per_archi.org}{localmente cpa}. Sia \((\tilde{X},\pi)\) un suo \href{20250113175231-rivestimento.org}{rivestimento}, con \(\tilde{X}\) \href{20250109155715-spazio_topologico_di_hausdorff.org}{T2}, \href{20250111142303-spazio_topologico_a_base_numerabile.org}{a base numerabile}, \href{20250103165325-spazio_topologico_connesso.org}{connesso} e \href{20250113102837-spazio_topologico_localmente_connesso_per_archi.org}{localmente cpa}.
Siano \(x_{0} \in X, \tilde{x}_{0} \in \pi^{-1}(x_{0})\). Allora
\begin{equation*}
\pi_{\star}: \Pi_{1}(\tilde{X}, \tilde{x}_{0})\longrightarrow \Pi_{1}(X,x_{0})
\end{equation*}
è \href{20241219101956-funzione_iniettiva.org}{iniettiva}. (Vedi \href{20241204223712-gruppo_fondamentale.org}{Gruppo fondamentale} e \href{20250114095222-funtore_del_gruppo_fondamentale.org}{Funtore del Gruppo Fondamentale})
\end{document}
