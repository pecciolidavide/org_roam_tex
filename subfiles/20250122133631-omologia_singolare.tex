% Intended LaTeX compiler: pdflatex
\documentclass[../main]{subfiles}

\usepackage[hyperref]{biblatex}
\date{}
\title{}
\begin{document}

\section{Omologia Singolare}
\label{sec:org58434dd}
Sia \(R\) un \href{20241219112842-pid.org}{PID}.

\begin{definizione}
Sia \(X\) uno \href{20250103145124-topologia.org}{spazio topologico}, e sia \(\mathcal{S}_{\bullet}(X)\) il suo \href{20250122133614-mappa_di_bordo_tra_moduli_di_catene_singolari.org}{complesso di catene singolare}. Si definisce, per ogni \(n\), l'\textbf{\(n\)-esimo \href{20241205141053-r_moduli.org}{modulo} di omologia singolare di \(X\)} come l'\(n\)-esimo \href{20250120164857-modulo_di_omologia_dei_complessi_di_catene.org}{modulo di omologia} del \href{20250120163114-complesso_di_catene.org}{complesso di catene} \(\mathcal{S}_{\bullet}(X)\):
\begin{equation*}
H_{n}(X)\coloneqq H_{n}\left(\mathcal{S}_{\bullet}(X)\right).
\end{equation*}
\end{definizione}
\end{document}
