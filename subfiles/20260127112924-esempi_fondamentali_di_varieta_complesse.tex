% Intended LaTeX compiler: pdflatex
\documentclass[../main]{subfiles}


\begin{document}

\section{Esempi fondamentali di varietà complesse}
\label{sec:org1b8f906}
\begin{esempio}
\(\C^{n}\) è una \href{20260127112752-varieta_complessa.org}{varietà complessa} di dimensione \(n\).
\end{esempio}
\subsection{Spazio proiettivo complesso è una varietà complessa}
\label{sec:org29093b8}
Lo \href{20241231115051-spazio_proiettivo.org}{spazio proiettivo} complesso \(\mathds{P}^{n}(\C)\), di coordinate omogenee \((z_{0}:\dots:z_{n})\) è \href{20260127112752-varieta_complessa.org}{varietà complessa} con l'\href{20260127112715-atlante_complesso.org}{atlante}: \(U_{i} \coloneqq \set{z_{i} \neq 0}\);
\begin{align*}
\varphi_{i}: U_{i} &\longrightarrow \C^{n}\\
(z_{0}:\dots:z_{n}) &\longmapsto \left(\frac{z_{0}}{z_{i}}; \dots; \check{\frac{z_{i}}{z_{i}}};\dots;\frac{z_{n}}{z_{i}}\right).
\end{align*}

I cambiamenti di coordinate sono \href{20260127132618-funzione_biolomorfa.org}{biolomorfismi}.

\begin{esempio}
Lo spazio \(\mathds{P}^{1}_{\C}\) è una \href{20260127112828-superficie_di_riemann.org}{superficie di Riemann} \href{20250103163701-spazio_topologico_compatto.org}{compatta} di {[}BROKEN LINK: 40333d07-ddb5-4df6-bbfe-85e21ab20dc3] zero, con  le due carte
\begin{align*}
\varphi_{0}: U_{0} &\longrightarrow \C & \varphi_{1}: U_{1} &\longrightarrow \C\\
(z_{0},z_{1}) &\longmapsto \frac{z_{1}}{z_{0}} & (z_{0},z_{1}) &\longmapsto \frac{z_{0}}{z_{1}}
\end{align*}

Il cambiamento di coordinate è dato da
\begin{align*}
\C\setminus\set{0} &\longrightarrow \C\setminus\set{0}\\
z &\longmapsto \frac{1}{z}
\end{align*}
\end{esempio}
\subsection{Aperti connessi di una varietà complessa sono varietà complessa}
\label{sec:org4c521c7}
\begin{prop}
Ogni \href{20250103145124-topologia.org}{aperto} \href{20250103165325-spazio_topologico_connesso.org}{connesso} di una \href{20260127112752-varieta_complessa.org}{varietà complessa} è una varietà complessa della stessa dimensione.
\end{prop}
\end{document}
