% Intended LaTeX compiler: pdflatex
\documentclass[../main]{subfiles}

%% \def\LT{\mathrm{LT}}
%% \def\LM{\mathrm{LM}}
%% \def\LC{\mathrm{LC}}
%% \def\tail{\mathrm{coda}}
%% \use{algorithm}
%% \use{algpseudocode}


\begin{document}

\section{Prova}

\lipsum[1]

%% \section{Lezione 30 - \textit{<2025-12-02 Tue>}}
%% \label{sec:org06f9eeb}

%% \subsection{Criterio di Buchberger}
%% \label{sec:org8360a0d}

%% Fissiamo un \emph{term order}. Sia \(G = \{g_1, \dots, g_n\} \subseteq P\) e sia \(I = (G)\) l'ideale generato da \(G\).

%% \begin{thm}
%% (Criterio di Buchberger).
%% \(G\) è una Base di Gröbner (GB) di \(I\) se e solo se:
%% \begin{equation*}
%% \forall i < j : \overline{S(g_i, g_j)}^G = 0
%% \end{equation*}
%% Ovvero, il resto della divisione degli S-polinomi rispetto a \(G\) è zero.
%% \label{crit:Buchberger}
%% \end{thm}

%% \begin{proof}
%% (\(\Rightarrow\)):
%% Se \(G\) è una GB di \(I\), sappiamo che per definizione gli S-polinomi appartengono all'ideale:
%% \begin{equation*}
%% S(g_i, g_j) = u g_i - v g_j \in I = (G)
%% \end{equation*}
%% Poiché \(G\) è una GB, il resto della divisione di qualsiasi elemento di \(I\) per \(G\) è unico e deve essere 0. Pertanto:
%% \begin{equation*}
%% S(g_i, g_j) \in I \implies \overline{S(g_i, g_j)}^G = 0. \qedhere
%% \end{equation*}
%% \end{proof}

%% Per dimostrare l'altra impliaczione è necessario un lemma:
%% \begin{lem}
%% Fissiamo un term order. Sia \(\{f_1, \dots, f_l\}\) un insieme di polinomi con \(f_i \in P\) tale che \(\LT(f_i) = t \in T^{(n)}\) per ogni \(i\).
%% Sia \(f = \sum_{i=1}^{l} c_i f_i\) con \(c_i \in K\) tale che \(\LT(f) < t\).
%% Allora \(f\) è una combinazione lineare degli S-polinomi:
%% \begin{equation*}
%% f = \sum c_{jk} S(f_j, f_k) \quad \text{con } \LT(S(f_j, f_k)) < t
%% \end{equation*}
%% \end{lem}
%% \begin{proof}
%% Consideriamo l'S-polinomio tra \(f_j\) e \(f_k\). Poiché \(\LT(f_j) = \LT(f_k) = t\), il minimo comune multiplo è \(t\).
%% \begin{equation*}
%% S(f_j, f_k) = \frac{t}{\LC(f_j)\LT(f_j)} f_j - \frac{t}{\LC(f_k)\LT(f_k)} f_k = \frac{1}{\LC(f_j)} f_j - \frac{1}{\LC(f_k)} f_k
%% \end{equation*}
%% Scriviamo \(f_i = d_i \cdot t + \tail(f_i)\), dove \(d_i = \LC(f_i)\).
%% Per ipotesi, \(\LT(\sum c_i f_i) < t\). Questo implica che il termine di grado \(t\) si cancella nella somma, ovvero:
%% \begin{equation*}
%% \sum_{i=1}^{l} c_i d_i = 0
%% \end{equation*}
%% Definiamo \(p_i = \frac{f_i}{d_i}\). Notiamo che \(S(f_j, f_k) = p_j - p_k\).
%% Riscriviamo la combinazione lineare usando una somma telescopica:
%% \begin{align*}
%% \sum_{i=1}^{l} c_i f_i &= \sum_{i=1}^{l} c_i d_i p_i \\
%% &= c_1 d_1 p_1 + c_2 d_2 p_2 + \dots + c_l d_l p_l \\
%% &= c_1 d_1 (p_1 - p_2) + (c_1 d_1 + c_2 d_2)(p_2 - p_3) + \dots + \left(\sum_{i=1}^{l} c_i d_i\right) p_l
%% \end{align*}
%% L'ultimo termine è nullo perché \(\sum c_i d_i = 0\).
%% Quindi abbiamo riscritto \(f\) come combinazione lineare di termini del tipo \(p_j - p_{j+1}\), che sono multipli scalari di \(S(f_j, f_{j+1})\). Inoltre, poiché i termini di testa si cancellano in \(p_j - p_k\), vale \(\LT(S(f_j, f_k)) < t\).
%% \end{proof}

%% \begin{proof}
%% (del Teorema\nbps{}\ref{crit:Buchberger}, \(\Leftarrow\)).

%% \emph{Ipotesi:} \(G = \{g_1, \dots, g_n\} \subseteq P\) e \(I=(G)\). Per ogni \(i < j\), \(\overline{S(g_i, g_j)}^G = 0\).

%% \emph{Tesi:} \(G\) è una GB di \(I\), ovvero \(\forall f \in I, \LT(f) \in (\LT(g_1), \dots, \LT(g_n))\).

%% Sia \(f \in I\). Poiché \(G\) genera \(I\), possiamo scrivere:
%% \begin{equation*}
%% f = \sum_{i=1}^{N} h_i g_i
%% \end{equation*}
%% Consideriamo il termine massimo in questa rappresentazione: \(\delta = \max_i \{ \LT(h_i g_i) \}\).
%% Certamente \(\LT(f) \le \delta\).

%% \begin{enumerate}
%% \item Caso 1: \(\LT(f) = \delta\).

%% Allora esiste un indice \(k\) tale che \(\LT(f) = \LT(h_k g_k) = \LT(h_k)\LT(g_k)\). Questo implica che \(\LT(f)\) è divisibile per \(\LT(g_k)\), quindi \(\LT(f) \in (\LT(G))\), e la condizione di GB è soddisfatta.

%% \item Caso 2: \(\LT(f) < \delta\).

%% Questo significa che i termini di testa nella somma \(\sum h_i g_i\) si cancellano.
%% Scegliamo una rappresentazione di \(f = \sum h_i g_i\) tale che \(\delta = \max \{ \LT(h_i g_i) \}\) sia \textbf{minimale}.
%% Chiamiamo questo massimo \(t\). Per assurdo, supponiamo \(\LT(f) < t\).

%% Isoliamo i termini che hanno grado \(t\):
%% \begin{equation*}
%% f = \sum_{\LT(h_i g_i) = t} h_i g_i + \sum_{\LT(h_i g_i) < t} h_i g_i
%% \end{equation*}
%% Consideriamo la prima parte. Possiamo scrivere \(h_i = \LC(h_i) \cdot \LM(h_i) + \tail(h_i)\).
%% La somma dei termini di testa deve annullarsi (perché \(\LT(f) < t\)):
%% \begin{equation*}
%% \sum_{\LT(h_i g_i)=t} \LC(h_i) \LM(h_i) g_i
%% \end{equation*}
%% Questa è una combinazione lineare di termini \(M_i g_i\) (dove \(M_i\) sono monomi) che hanno tutti lo stesso \(\LT = t\).
%% Per il \textbf{Lemma} dimostrato sopra, questa somma può essere riscritta come combinazione lineare degli S-polinomi di questi termini \(\{M_i g_i\}\):
%% \begin{equation*}
%% \sum c_{jk} S(M_j g_j, M_k g_k)
%% \end{equation*}
%% Nota: \(S(M_j g_j, M_k g_k) = \frac{t}{w_{jk}} S(g_j, g_k)\) dove \(w_{jk} = \text{mcm}(\LT(g_j), \LT(g_k))\).

%% Per ipotesi, \(\overline{S(g_j, g_k)}^G = 0\), il che significa che l'algoritmo di divisione ci dà:
%% \begin{equation*}
%% S(g_j, g_k) = \sum q_{\nu} g_{\nu} \quad \text{con } \LT(q_{\nu} g_{\nu}) \le \LT(S(g_j, g_k)) < w_{jk}
%% \end{equation*}
%% Sostituendo queste espressioni nella somma originale, otteniamo una nuova scrittura per \(f\):
%% \begin{equation*}
%% f = \sum \tilde{h}_i g_i
%% \end{equation*}
%% dove tutti i termini hanno grado strettamente minore di \(t\).
%% Questo contraddice la \textbf{minimalità} di \(t\) che avevamo assunto.
%% Quindi il caso \(\LT(f) < \max \LT(h_i g_i)\) non può verificarsi con la scelta minimale. Deve valere \(\LT(f) = \max \LT(h_i g_i)\), il che implica che \(G\) è una Base di Gröbner.
%% \end{enumerate}
%% \end{proof}
%% \subsection{Esempi}
%% \label{sec:org6b47623}

%% \begin{esempio}
%% Sia \(G = \{y - x^2, z - x^3\} \subseteq K[x,y,z]\) con term order Lex \(z > y > x\).

%% \begin{itemize}
%% \item \(g_1 = y - x^2\), \(\LT(g_1) = y\)
%% \item \(g_2 = z - x^3\), \(\LT(g_2) = z\)
%% \end{itemize}

%% Calcoliamo l'S-polinomio:
%% \begin{equation*}
%% S(g_1, g_2) = \frac{yz}{y}(y-x^2) - \frac{yz}{z}(z-x^3) = z(y-x^2) - y(z-x^3) = zy - zx^2 - yz + yx^3 = yx^3 - zx^2
%% \end{equation*}
%% Ordiniamo il risultato: \(yx^3 - zx^2\). Dividiamo per \(G\):

%% \begin{itemize}
%% \item \(yx^3\) è divisibile per \(\LT(g_1)=y\)? Sì. \(yx^3 = x^3 \cdot y\).
%% \item Sottraiamo \(x^3 g_1 = x^3(y-x^2) = x^3y - x^5\).
%% \item Resto parziale: \((yx^3 - zx^2) - (yx^3 - x^5) = -zx^2 + x^5\).
%% \item \(-zx^2\) è divisibile per \(\LT(g_2)=z\)? Sì.
%% \item Sottraiamo \(-x^2 g_2 = -x^2(z-x^3) = -x^2z + x^5\).
%% \item Resto: \((-zx^2 + x^5) - (-zx^2 + x^5) = 0\).
%% \end{itemize}

%% Poiché \(\overline{S(g_1, g_2)}^G = 0\), per il criterio di Buchberger, \(G\) è una GB.
%% \end{esempio}

%% \begin{esempio}
%% Siano \(f_1 = x^3 - 2xy\), \(f_2 = x^2y - 2y^2 + x\) in \(K[x,y]\) con ordine DegLex \(x > y\).
%% \(G = \{f_1, f_2\}\). È una GB?
%% \begin{equation*}
%% S(f_1, f_2) = \frac{x^3y}{x^3}f_1 - \frac{x^3y}{x^2y}f_2 = y f_1 - x f_2 = y(x^3 - 2xy) - x(x^2y - 2y^2 + x) = -x^2
%% \end{equation*}
%% Il resto della divisione di \(-x^2\) per \(f_1, f_2\) è \(-x^2\) (nessun LT divide \(x^2\)).
%% Poiché il resto è non nullo, aggiungiamo \(f_3 = -x^2\) (o meglio \(x^2\)) a \(G\).
%% Nuovo insieme \(G' = \{f_1, f_2, x^2\}\).
%% Bisogna iterare calcolando \(S(f_1, f_3)\) e \(S(f_2, f_3)\) finché tutti gli S-polinomi riducono a zero.
%% \end{esempio}
%% \subsection{Algoritmo di Buchberger}
%% \label{sec:orgfed931f}

%% \begin{algorithm}
%% \caption{Algoritmo di Buchberger}
%% \begin{algorithmic}[1]
%% \State \textbf{Input:} \(F = \{f_1, \dots, f_s\} \subseteq P\), un term order.
%% \State \textbf{Output:} \(G = \{g_1, \dots, g_l\}\) tale che \((G)=(F)\) e \(G\) è GB.
%% \State \(G \gets F\)
%% \State \textbf{Repeat}
%%     \State \(G' \gets G\)
%%     \For{ogni coppia \(\{p, q\} \subseteq G', p \neq q\)}
%%         \State \(S \gets \overline{S(p, q)}^{G'}\) \Comment{Resto della divisione}
%%         \If{\(S \neq 0\)}
%%             \State \(G \gets G \cup \{S\}\)
%%         \EndIf
%%     \EndFor
%% \State \textbf{Until} \(G = G'\)
%% \State \Return \(G\)
%% \end{algorithmic}
%% \end{algorithm}

%% L'algoritmo termina sempre perché \(K[x_1, \dots, x_n]\) è noetheriano (Teorema della base di Hilbert), quindi la catena ascendente degli ideali generati dai termini di testa deve stabilizzarsi (\(LT(G_0) \subseteq LT(G_1) \subseteq \dots\)).
\end{document}
