% Intended LaTeX compiler: pdflatex
\documentclass[../main]{subfiles}


\begin{document}

\section{Gioco di Banach-Mazur}
\label{sec:org484e85d}
Sia \(X\) uno \href{20250103145124-topologia.org}{spazio topologico} non vuoto, e sia \(A \subseteq X\).

Il \uline{gioco di Banach-Mazur} di \(A\), denotato con \(G^{**}(A)\) oppure con \(G^{**}(A,X)\) è un \href{20250513155732-logic_game.org}{gioco logico} \href{20250513171520-giochi_di_gale_stewart.org}{di Gale-Stewart} codificato come segue: i giocatori I e II si alternano scegliendo sottoinsiemi aperti non vuoti di \(X\)
\begin{equation*}
\begin{tikzcd}[ampersand replacement=\&,cramped,sep=tiny]
	{\text{I}} \& {U_0} \&\& {U_1} \&\& {U_2} \&\& \cdots \\
	{\text{II}} \&\& {V_0} \&\& {V_1} \&\& \cdots
\end{tikzcd}
\end{equation*}
\href{20250513171520-giochi_di_gale_stewart.org}{tali che} \(U_{0}\supseteq V_{0}\supseteq U_{1}\supseteq V_{1}\supseteq \dots\)

Il giocatore II vince se
\begin{equation*}
\bigcap_{n \in \omega} U_{n} = \bigcap_{n \in \omega} V_{n} \subseteq A.
\end{equation*}
\section{Gioco di Banach-Mazur unfolded}
\label{sec:org36b5c72}
Sia \(X\) uno \href{20250301194013-spazio_polacco.org}{spazio polacco} non vuoto con una \href{20250301193511-spazio_metrico.org}{metrica} fissata e sia \(\mathcal{W}\) una \href{20250525113346-base_debole_di_uno_spazio_topologico.org}{base debole} \href{20250111143651-insieme_numerabile.org}{numerabile} di \(X\).

Dato \(F \subseteq X\times \omega^{\omega}\), il \uline{gioco di Banach-Mazur unfolded} \(G^{**}_{\text{u}}(F)\) è il \href{20250513155732-logic_game.org}{gioco logico} \href{20250513171520-giochi_di_gale_stewart.org}{di Gale-Stewart} codificato come segue:
\begin{equation*}
\begin{tikzcd}[ampersand replacement=\&,cramped,sep=tiny]
	{\text{I}} \& {U_0} \&\& {U_1} \&\& \dots \\
	{\text{II}} \&\& {y_0,V_0} \&\& {y_1, V_{1}} \& \dots
\end{tikzcd}
\end{equation*}
tali che:
\begin{itemize}
\item per ogni \(i \in \omega\): \(U_{i}, V_{i} \in \mathcal{W}\), \(y_{n} \in \omega\);
\item \(\operatorname{diam}(U_{n}), \operatorname{diam}(V_{n}) < 2^{-n}\);
\item \(U_{0}\supseteq V_{0}\supseteq U_{1}\supseteq V_{1}\supseteq \dots\)
\end{itemize}

posto
\begin{equation*}
\set{x}\coloneqq\bigcap_{i \in \omega} \operatorname{Cl}_{X}(U_{n}) = \bigcap_{i \in \omega} \operatorname{Cl}_{X}(V_{n})
\end{equation*}
e \(y\coloneqq (y_{i})_{i \in \omega} \in \omega^{\omega}\), il \uline{giocatore II vince} sse
\begin{equation*}
(x,y) \in F \subseteq X\times \omega^{\omega}.
\end{equation*}
\subsection{Applicazione del Teorema di Gale-Stewart}
\label{sec:org7d9a038}

Se \(F\) è aperto o chiuso di \(X\times\omega^{\omega}\), allora \(G^{**}_{\text{u}}(F)\) è determinato.
\subsubsection{Dimostrazione}
\label{sec:orgd7e2685}

\begin{itemize}
\item Si costruisce un \(\mathcal{A}\)-schema su \(X\). Per ogni \(\langle (A_{0},a_{0}),\dots,(A_{k},a_{k})\rangle \in \mathcal{A}^{<\omega}\), si definisce
\begin{equation*}
  B_{\langle (A_{0},a_{0}),\dots,(A_{k},a_{k})\rangle} \coloneqq \begin{cases}
  	\bigcap_{i\le k} \operatorname{Cl}_{X}(A_{i}) & \forall\,i\le k: \ \operatorname{diam}(A_{i})<2^{-i}\\
  	& A_{0} \supseteq A_{1}\supseteq \dots \supseteq A_{k}\\[1em]
  	\emptyset &\text{altrimenti}
      \end{cases}
\end{equation*}

Ovviamente, per ogni \(s \in \mathcal{A}^{<\omega}\) e per ogni \(a \in \mathcal{A}\):
\begin{equation*}
  	B_{s\concat a} \subseteq B_{s}
\end{equation*}
ed inoltre per ogni \(x \in \mathcal{A}^{\omega}\): \(\operatorname{diam}(B_{x\upharpoonright n})\to 0\).

Inoltre ciascun \(B_{s}\) è chiuso (poiché intersezione finita di chiusi oppure il vuoto) e pertanto, per il Lemma 1.3.6, questo schema induce una funzione continua
\begin{equation*}
  	f:[T]\to X,\quad T\coloneqq \set{s \in \mathcal{A}^{<\omega}\mid B_{s}\neq \emptyset}.
\end{equation*}

\item Si osserva che \(T=\set{\langle (A_{i},a_{i})\rangle_{i\le k}: A_{0}\supseteq A_{1}\supseteq \dots A_{k} \,\land\, \forall\, i\le k:\ \operatorname{diam}A_{i}<2^{-i}}\), ovvero è esattamente l'albero delle posizioni ammissibili di \(G^{**}_{\text{u}}(F)\), ed inoltre \(T\) è un albero potato non vuoto.

\item La funzione
\begin{align*}
g: \mathcal{A}^{\omega} &\longrightarrow \omega^{\omega}\\
\left((A_{i},a_{i})\right)_{i \in\omega}&\longmapsto (a_{2i+1})_{i \in\omega}
\end{align*}
è continua.

\item Si ottiene quindi una funzione continua
\begin{align*}
\psi: [T] &\longrightarrow X\times\omega^{\omega}\\
s &\longmapsto \left(f(s),g(s)\right)
\end{align*}
\end{itemize}

Sia ora dunque \(F \subseteq X\times\omega^{\omega}\) aperto o chiuso, sia \(F'\coloneqq [T]\setminus\psi^{-1}(F)\) aperto o chiuso, e si consideri il \href{20250513171520-giochi_di_gale_stewart.org}{Gioco di Gale-Stewart con posizioni ammissibili} \(G(T,F')\). Per il \href{20250514144736-teorema_di_gale_stewart.org}{Teorema di Gale-Stewart}, questo è determinato, poiché \(F'\) è aperto o chiuso; ovvero esattamente uno tra i giocatori I e II ha una strategia vincente.

\uline{Caso 1}. Sia \(\sigma'\) una strategia vincente per il giocatore I nel gioco \(G(T,F')\), \(\sigma' \subseteq T \subseteq \mathcal{A}^{<\omega}\) con \([\sigma'] \subseteq F'\), ovvero \([\sigma'] \cap \psi^{-1}(F)=\emptyset\).

Si costruisce una strategia \(\sigma\) per il giocatore I nel gioco \(G^{**}_{\text{u}}(F)\):
\begin{equation*}
\sigma \coloneqq \set{
\begin{gathered}
\langle A_{0}, (A_{1},a_{1}),\dots,(A_{2k-1}, a_{2k-1}), A_{2k}\rangle,\\
\langle A_{0}, (A_{1},a_{1}),\dots,(A_{2k+1}, a_{2k+1})\rangle
\end{gathered}
\mid \langle (A_{0},a_{0}), (A_{1},a_{1}),\dots,(A_{k}, a_{k})\rangle \in \sigma'
}.
\end{equation*}

Sia quindi \(\left( U_{i}, (V_{i},y_{i}) \right)_{i \in \omega}\) una giocata per I seguendo la strategia \(\sigma\), e siano
\begin{equation*}
\set{x} \coloneqq \bigcap_{i \in \omega} U_{i} = \bigcap_{i \in \omega} V_{i},\quad y\coloneqq(y_{i})_{i \in\omega}.
\end{equation*}
Allora esiste \(s \in [\sigma']\) tale che \((x,y) = \psi(s)\) per costruzione. Siccome \([\sigma'] \cap \psi^{-1}(F) = \emptyset\) segue che \((x,y)\notin F\).

\uline{Caso 2}. Sia \(\sigma'\) una strategia vincente per il giocatore II nel gioco \(G(T,F')\), \(\sigma' \subseteq T \subseteq \mathcal{A}^{<\omega}\) con \([\sigma'] \cap  F' = \emptyset\), ovvero \([\sigma'] \subseteq \psi^{-1}(F)\)

Si costruisce una strategia \(\sigma\) per il giocatore II nel gioco \(G^{**}_{\text{u}}(F)\):
\begin{equation*}
\sigma \coloneqq \set{
\begin{gathered}
\langle A_{0}, (A_{1},a_{1}),\dots,(A_{2k-1}, a_{2k-1}), A_{2k}\rangle,\\
\langle A_{0}, (A_{1},a_{1}),\dots,(A_{2k+1}, a_{2k+1})\rangle
\end{gathered}
\mid \langle (A_{0},a_{0}), (A_{1},a_{1}),\dots,(A_{k}, a_{k})\rangle \in \sigma'
}.
\end{equation*}

Sia quindi \(\left( U_{i}, (V_{i},y_{i}) \right)_{i \in \omega}\) una giocata per I seguendo la strategia \(\sigma\), e siano
\begin{equation*}
\set{x} \coloneqq \bigcap_{i \in \omega} U_{i} = \bigcap_{i \in \omega} V_{i},\quad y\coloneqq(y_{i})_{i \in\omega}.
\end{equation*}
Allora esiste \(s \in [\sigma'] \subseteq \psi^{-1}(F)\) tale che \((x,y) = \psi(s)\) per costruzione, e pertanto \((x,y) \in F\).\qed
\end{document}
