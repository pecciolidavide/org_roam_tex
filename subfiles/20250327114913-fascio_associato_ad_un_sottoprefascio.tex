% Intended LaTeX compiler: pdflatex
\documentclass[../main]{subfiles}


\begin{document}

\section{Fascio associato ad un sottoprefascio}
\label{sec:orgcba2673}
\begin{oss}
Sia \(\mathcal{F}\) un \href{20250324174728-fascio.org}{fascio}, e sia \(\mathcal{G} \subseteq \mathcal{F}\) un \href{20250325150647-sottoprefascio.org}{sotto-prefascio}.

Allora \(\mathcal{G}\) soddisfa l'\href{20250324174728-fascio.org}{assioma dell'unicità}, e il \href{20250327114851-fascio_associato_ad_un_prefascio.org}{fascificato} \(\mathcal{G}^{+} \subseteq \mathcal{F}\) è un \href{20250325150647-sottoprefascio.org}{sottofascio} di \(\mathcal{F}\):
\begin{equation*}
\mathcal{G} \subseteq \mathcal{G}^{+} \subseteq \mathcal{F}
\end{equation*}
con
\begin{equation*}
\mathcal{G}^{+}(U) = \set{s \in \mathcal{F}(U) \mid \forall  p \in U\ \exists W \subseteq U \text{ intorno aperto di \(p\)}:\ \restriction{s}{W} \in \mathcal{G}(W)}.
\end{equation*}
\end{oss}

\begin{proof}
È necessario dimostrare che \(\mathcal{G}^{+}(U)\) sia effettivamente (isomorfo a) l'insieme definito sopra. Si ricorda che per definizione
\begin{equation*}
\mathcal{G}^{+}(U) =  \set{s: U \to \coprod_{p \in U} \mathcal{G}_{p} \mid
s(p) \in \mathcal{G}_{p},\
\forall  p \in U\ \exists V \subseteq U \ \exists f \in \mathcal{G}(V):\
\forall q \in V\ s(q)=[f]_{q}
}
\end{equation*}
dove \(\mathcal{G}_{p}\) è la \href{20250325183434-spiga_di_un_prefascio.org}{spiga di \(\mathcal{G}\) in \(p\)}.

(\(\supseteq\)): Sia \(s \in \mathcal{F}(U)\) che soddisfi le condizioni di cui sopra. Si definisce
\begin{align*}
\tilde{s}: U &\longrightarrow \coprod_{p \in U}\mathcal{G}_{p}\\
p &\longmapsto [\restriction{s}{W}]_{p}
\end{align*}
per \(p \in W \subseteq U\) tale che \(\restriction{s}{W} \in \mathcal{G}(W)\).

È possibile scegliere i \(W\) in maniera tale che per ogni \(q \in W\), \(\tilde{s}(q) = [\restriction{s}{W}]_{q}\).

Pertanto:
\begin{itemize}
\item \(\tilde{s}(p) = [\restriction{s}{W}]_{p} \in \mathcal{G}_{p}\) in quanto \(\restriction{s}{W} \in \mathcal{G}(W)\);
\item per ogni \(p \in U\) esiste \(W \subseteq U\) ed esiste \(f=\restriction{s}{W} \in \mathcal{G}(W)\) tale che
\begin{equation*}
  \forall q \in W:\qquad \tilde{s}(q) = [\restriction{s}{W}]_{q}.
\end{equation*}
\end{itemize}

Quindi \(\tilde{s} \in \mathcal{G}^{+}(U)\).

(\(\subseteq\)): Sia \(s:U\to \coprod_{p \in U} \mathcal{G}_{p}\) in  \(\mathcal{G}^{+}(U)\).

Per definizione, per ogni \(p \in U\) esiste \(V_{p} \subseteq U\) ed esiste \(f_{p} \in \mathcal{G}(V_{p}) \subseteq \mathcal{F}(V_{p})\) tali che \(s(q) = [f_{p}]_{q}\) per ogni \(q \in V_{p}\).

\begin{itemize}
\item Siano \(p_{1},p_{2} \in U\) tali che \(V_{p_{1}}\cap V_{p_{2}} \neq \emptyset\). Per ogni punto \(q \in V_{p_{1}}\cap V_{p_{2}}\), si ha che
\begin{equation*}
  [f_{p_{1}}]_{q} = s(q) = [f_{p_{2}}]_{q}
\end{equation*}
e pertanto si ha l'uguaglianza tra \href{20250325183434-spiga_di_un_prefascio.org}{germi}: \([f_{p_{1}}]_{q} = [f_{p_{2}}]_{q}\).

Pertanto esiste \(W_{q} \subseteq V_{p_{1}}\cap V_{p_{2}}\) tale che
\begin{equation*}
  \restriction{(f_{p_{1}})}{W_{q}} = \restriction{(f_{p_{2}})}{W_{q}}
\end{equation*}
ovvero \(\restriction{(f_{p_{1}}-f_{p_{2}})}{W_{q}} = \restriction{\left(\restriction{(f_{p_{1}}-f_{p_{2}})}{V_{p_{1}}\cap V_{p_{2}}}\right)}{W_{q}}=0\)\footnote{Infatti i morfismi di restrizione sono \href{20241206115531-morfismo_di_gruppi.org}{morfismi di gruppo}.}.

Siccome \(\set{W_{q}}_{q \in V_{p_{1}}\cap V_{p_{2}}}\) è un ricoprimento aperto di \(V_{p_{1}}\cap V_{p_{2}}\), per l'\href{20250324174728-fascio.org}{assioma di unicità}
\begin{equation*}
  \restriction{(f_{p_{1}}-f_{p_{2}})}{V_{p_{1}}\cap V_{p_{2}}} = 0
\end{equation*}
ovvero
\begin{equation*}
  \restriction{(f_{p_{1}})}{V_{p_{1}}\cap V_{p_{2}}} = \restriction{(f_{p_{2}})}{V_{p_{1}}\cap V_{p_{2}}}.
\end{equation*}

\item Siccome \(\set{V_{p}\cap U}_{p \in U}\) è un ricoprimento aperto di \(U\), allora per l'assioma di fascio esiste \(\sigma \in \mathcal{F}(U)\) tale che
\begin{equation*}
  \forall  p \in U:\qquad \restriction{\sigma}{V_{p}} = f_{p}.
\end{equation*}
\end{itemize}

Questo \(\sigma\) soddisfa effettivamente le condizioni richieste: per ogni \(p \in U\), esiste \(V_{p} \subseteq U\) intorno aperto di \(p\) tale che
\begin{equation*}
\restriction{\sigma}{V_{p}} = f_{p} \in \mathcal{G}(V_{p}).%
\qedhere
\end{equation*}
\end{proof}
\end{document}
