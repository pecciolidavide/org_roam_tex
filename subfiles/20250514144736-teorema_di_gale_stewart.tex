% Intended LaTeX compiler: pdflatex
\documentclass[../main]{subfiles}


\begin{document}

\section{Teorema di Gale-Stewart}
\label{sec:org209f802}
Sia \(A\) uno \href{20250103145124-topologia.org}{spazio topologico} \href{20250317165247-topologia_discreta.org}{discreto} e sia \(A^{\omega}\) dotato della \href{20250109154723-topologia_prodotto.org}{topologia prodotto}.
\subsection{Teorema}
\label{sec:org753258a}
Sia \(T\) un \href{20250514142154-albero_teoria_descrittiva_degli_insiemi.org}{albero} \href{20250514142208-albero_potato.org}{potato} non vuoto su \(A\). Se \(C \subseteq [T]\) è \href{20250103145124-topologia.org}{aperto} o \href{20250103145124-topologia.org}{chiuso} in \([T]\) \footnote{Vedi \href{20250514142251-corpo_di_un_albero.org}{Corpo di un albero} e \href{20250103163814-sottospazio_topologico.org}{Sottospazio topologico}}, allora \href{20250513171520-giochi_di_gale_stewart.org}{il gioco} \(G(T,C)\) è \href{20250513155732-logic_game.org}{determinato}.
\end{document}
