% Intended LaTeX compiler: pdflatex
\documentclass[../main]{subfiles}


\begin{document}

\section{Superficie di Riemann}
\label{sec:orgb37b530}
\begin{definizione}
Una \textbf{superficie di Riemann} è una \href{20260127112752-varieta_complessa.org}{varietà complessa} di dimensione 1.
\end{definizione}
\subsection{Genere topologico di una Superficie di Riemann}
\label{sec:orgff11fbc}
\begin{oss}
Una superficie di Riemann è sempre una \href{20251230172241-superficie_topologica.org}{superficie topologica} \href{20260130155129-superficie_topologica_orientabile.org}{orientabile}, e quindi, se è \href{20250103163701-spazio_topologico_compatto.org}{compatta}, \href{20260130155105-teorema_di_classificazione_delle_superfici_topologiche.org}{è omeomorfo ad una sfera con \(g\) manici}.
\end{oss}

\begin{definizione}
Se \(X\) è una superficie di Riemann \href{20250103163701-spazio_topologico_compatto.org}{compatta}, \(X\mathrel{\overset{omeo}{\approx}} T_{g}\).\footnote{Vedi:
\begin{itemize}
\item \href{20250111142332-omeomorfismo.org}{Omeomorfismo}
\item \href{20260130155514-sfera_con_g_buchi.org}{Toro con g buchi}
\end{itemize}} Allora \(g\) è il \textbf{genere topologico} di \(X\), e si indica con \(g(X)\).
\end{definizione}
\end{document}
