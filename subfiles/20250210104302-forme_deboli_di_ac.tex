% Intended LaTeX compiler: pdflatex
\documentclass[../main]{subfiles}

\usepackage[hyperref]{biblatex}
\date{}
\title{}
\begin{document}

\section{Forme deboli di AC}
\label{sec:org86e1fde}
Contesto: \href{20250130104245-morse_kelly_set_theory.org}{Morse Kelly Set Theory}

Si elencano delle versioni deboli di \href{20250206171508-axiom_of_choiche.org}{AC}.
\subsection{Axiom of Choice - Indice}
\label{sec:orgc41589a}
\begin{definizione}
Sia \(I\) un insieme.
\begin{description}
\item[{\(\mathrm{AC}_{I}\):}] data una \href{20250206170922-sequenze_e_stringhe.org}{sequenza} \(\langle A_{i}\ |\ i \in I\rangle\) tale che per ogni \(i \in I\) si ha \(A_{i}\neq \emptyset\) non \href{20250131161811-insieme_vuoto_mk.org}{vuoto}, esiste una sequenza
\begin{equation*}
  \langle a_{i}\ |\ i \in I \rangle
\end{equation*}
tale che per ogni \(i \in I\) si ha \(a_{i} \in A_{i}\).
\end{description}
\end{definizione}
\subsubsection{Axiom of countable Choice}
\label{sec:orgdf96d13}
\begin{definizione}
L'Axiom of Countable Choice ACC è \(\mathrm{AC}_{\omega}\)
\begin{description}
\item[{\(\mathrm{ACC}\):}] data una \href{20250206170922-sequenze_e_stringhe.org}{sequenza} \(\langle A_{i}\ |\ i \in \omega\rangle\) tale che per ogni \(i \in \omega\) si ha \(A_{i}\neq \emptyset\) non \href{20250131161811-insieme_vuoto_mk.org}{vuoto}, esiste una sequenza
\begin{equation*}
  \langle a_{i}\ |\ i \in \omega \rangle
\end{equation*}
tale che per ogni \(i \in \omega\) si ha \(a_{i} \in A_{i}\).
\end{description}
\end{definizione}
\subsection{Axiom of Choice - Insieme}
\label{sec:orgdb59bb3}
\begin{definizione}
Sia \(X\) un \href{20250130104331-insieme_mk.org}{insieme}.
\begin{description}
\item[{\(\mathrm{AC}(X)\):}] se \(X\) è un \href{20250130104331-insieme_mk.org}{insieme} non \href{20250131161811-insieme_vuoto_mk.org}{vuoto} allora esiste una \href{20250203105434-funzione_di_scelta.org}{funzione di scelta} su \(X\).
\end{description}
\end{definizione}
\subsection{Axiom of Chioce - Insieme e Indice}
\label{sec:org59543fb}
\begin{definizione}
Siano \(I, X\) due \href{20250130104331-insieme_mk.org}{insiemi}.
\begin{description}
\item[{\(\mathrm{AC}_{I}(X)\):}] Se \(X\) è un \href{20250130104331-insieme_mk.org}{insieme} non \href{20250131161811-insieme_vuoto_mk.org}{vuoto} e vi è una \href{20250202170607-classe_relazione_binaria.org}{funzione}\footnote{Vedi ``\href{20250130104245-morse_kelly_set_theory.org}{Insieme delle parti per MK}''}
\begin{align*}
  I &\longrightarrow \parti{X}\setminus\set{\emptyset}\\
  i &\longmapsto A_{i}
\end{align*}
allora esiste una \href{20250202170607-classe_relazione_binaria.org}{funzione}
\begin{align*}
  I &\longrightarrow X\\
  i &\longmapsto a_{i} \in A_{i}
\end{align*}
\end{description}
\end{definizione}
\subsection{Axiom of Dependent Choices}
\label{sec:orgf7887e0}
\subsection{Implicazioni tra le forme deboli di AC}
\label{sec:orga194d57}
\begin{prop}
È facile verificare che
\begin{align*}
\operatorname{AC}_{I}\,&\iff\, \forall\, X\ \operatorname{AC}_{I}(X);\\
\operatorname{AC}(X)\,&\iff\, \forall\, I \operatorname{AC}_{I} (X);\\
\operatorname{AC}\,&\iff\, \forall\, I \forall\,X\ \operatorname{AC}_{I}(X)
\end{align*}

Inoltre, se esiste \(f:X\to Y\) \href{20241213105600-funzione_suriettiva.org}{suriettiva} e \(J\embeds I\)\footnote{Con ``\(\embeds\)'' si denota l'esistenza di una \href{20241219101956-funzione_iniettiva.org}{funzione iniettiva}.}, si ha che
\begin{equation*}
\operatorname{AC}_{I}(X)\,\implies\, \operatorname{AC}_{J}(Y)
\end{equation*}
\end{prop}
\end{document}
