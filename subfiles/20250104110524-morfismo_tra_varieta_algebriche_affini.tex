% Intended LaTeX compiler: pdflatex
\documentclass[../main]{subfiles}


\begin{document}

\section{Morfismo tra varietà algebriche affini}
\label{sec:orgaa9ad6a}
Sia \(\K\) un \href{20241231112713-campo_algebricamente_chiuso.org}{campo algebricamente chiuso} e siano \(X \subseteq \A^{n}\), \(Y \subseteq \A^{m}\) \href{20241231114256-varieta_algebrica_affine.org}{varietà algebriche affini}. (Vedi \href{20241231114009-spazio_affine.org}{Spazio Affine})
\subsection{Definizione}
\label{sec:orgc3530f1}
Una funzione \(f: X \longrightarrow Y\) è un \textbf{morfismo tra varietà affini} se esiste una mappa polinomiale
\begin{equation*}
\overline{f}: \A^{n} \longrightarrow \A^{m}
\end{equation*}
tale che, per ogni \(x \in X\), \(f(x)=\overline{f}(x)\).
\subsubsection{Precisazione}
\label{sec:orge9da42b}
Per mappa polinomiale si intende
\begin{equation*}
\overline{f}(x_{1},\dots,x_{n}) = \left(f_{1}(x_{1},\dots,x_{n}),\dots,f_{m}(x_{1},\dots,x_{n})\right)
\end{equation*}
con \(f_{i} \in \K[x_{1},\dots,x_{n}]\) (vedi \href{20241219113434-anello_dei_polinomi.org}{Anello-dei-polinomi}).
\subsubsection{Osservazione}
\label{sec:orgfcf8b6e}
La definizione richiede che \(\overline{f}(X) \subseteq Y\), pertanto non tutti i polinomi di \(\K[x_{1},\dots,x_{n}]\) danno luogo a morfismi da \(X\longrightarrow Y\).
\subsection{Isomorfismo tra varietà algebriche affini}
\label{sec:org29f5eed}
\subsubsection{Definizione}
\label{sec:org32384a4}
Un morfismo \(F: X \longrightarrow Y\) è un \href{20241128162125-isomorfismo.org}{isomorfismo} se
\begin{itemize}
\item \(F\) è \href{20250104111707-funzione_biunivoca.org}{biunivoca} (e quindi esiste \(F^{-1}\));
\item \(F^{-1}\) è un morfismo.
\end{itemize}
\subsection{Esempi di morfismi}
\label{sec:orgd5f6f4f}

\subsubsection{Esempio 1}
\label{sec:org360d66b}

Sia \(A \in \K^{n,n}\) (vedi \href{20250104111539-spazio_delle_matrici.org}{Spazio delle matrici}) e sia \(B \in \K^{n}\).
\begin{align*}
F: \A^{n} &\longrightarrow \A^{n}\\
x &\longmapsto Ax+B
\end{align*}
è un morfismo, poiché è polinomiale (data da polinomi di primo grado).

Inoltre, \(F\) è invertibile se e solo se \(A\) è \href{20250104111735-matrice_invertibile.org}{invertibile} (ovver sse \(\det A \neq 0\), vedi \href{20250104111751-determinante_di_una_matrice.org}{Determinante di una matrice}).
\subsubsection{Esempio 2}
\label{sec:org15a4f67}
Sia \(n>m\), e consideriamo
\begin{align*}
\pi: \A^{n} &\longrightarrow \A^{m}\\
(x_{1},\dots,x_{m},x_{m+1},\dots,x_{n}) &\longmapsto (x_{1},\dots,x_{m})
\end{align*}
la proiezione sulle prime \(m\) coordinate.

Evidentemente è un morfismo, poiché le mappe sono polinomiali. Allo stesso tempo è evidente che non sia un isomorfismo.
\subsubsection{Esempio 3}
\label{sec:org46ce77a}
Sia \(f \in \K[x]\), e sia \(g(x,y) = y-f(x)\), \(g \in \K[x,y]\)
Consideriamo \(X=V(g) \subseteq \A^{2}\) (vedi \href{20241231112823-radici_polinomiali.org}{Luogo di zeri}). \(X\) è il \href{20250104112443-grafico_di_una_funzione.org}{grafico} di \(f\).

Si hanno le mappe
\begin{align*}
F: \A^{1} &\longrightarrow X\\
t &\longmapsto \left(t,f(t)\right)
\end{align*}
che è un morfismo poiché polinomiale, e la sua inversa
\begin{align*}
F^{-1}: X &\longrightarrow \A^{1}\\
(x,y) &\longmapsto x
\end{align*}
che è morfismo per l'esempio precedente, poiché proiezione sul primo fattore.

Dunque \(F\) è un \hyperref[sec:org29f5eed]{isomorfismo}.
\end{document}
