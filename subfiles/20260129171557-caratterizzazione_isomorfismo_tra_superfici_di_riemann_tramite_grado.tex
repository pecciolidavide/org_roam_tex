% Intended LaTeX compiler: pdflatex
\documentclass[../main]{subfiles}

\def\mult#1{\mathrm{mult}_{#1}\,}


\begin{document}

\section{Caratterizzazione isomorfismo tra superfici di Riemann tramite grado}
\label{sec:org7daffdd}
Siano \(X,Y\) \href{20260127112828-superficie_di_riemann.org}{superfici di Riemann} \href{20250103163701-spazio_topologico_compatto.org}{compatte}, e sia \(F:X\to Y\) \href{20260128143822-funzione_olomorfa_su_una_superficie_di_riemann.org}{olomorfismo} non costante.

\begin{prop}
La mappa \(F\) è un \href{20260128144717-isomorfismo_tra_superfici_di_riemann.org}{isomorfismo} se è solo se ha \href{20260129171444-teorema_del_grado_per_olomorfismi_tra_superfici_di_riemannn.org}{grado} \(\deg F = 1\).
\end{prop}
\begin{proof}
(\(\Rightarrow\)): Se \(F\) è isomorfismo allora è \href{20250104111707-funzione_biunivoca.org}{biunivoca}, e pertanto:
\begin{itemize}
\item ogni punto di \(Y\) ha un solo elemento nella \href{20260128184014-fibra_di_una_funzione.org}{fibra};
\item ogni punti di \(X\) ha \href{20260129104215-forma_normale_locale_per_superfici_di_riemann.org}{molteplicità} 1
\end{itemize}
e quindi \(\deg F = 1\).

(\(\Leftarrow\)): Poiché \(\deg F = 1\),
\begin{itemize}
\item per ogni \(x \in X\), \(\mult{x} F = 1\), e quindi esiste \(U_{x}\) intorno di \(x\) e \(V_{x}\) intorno di \(F(x)\) tale che \(F\) sia un \href{20260128144717-isomorfismo_tra_superfici_di_riemann.org}{biolomorfismo};
\item allora per ogni \(y \in Y\) esiste un unico \(x \in X\) tale che \(F(x) = y\)
\end{itemize}
Questo prova la biiettività.

Sia quindi \(G\coloneqq F^{-1} : Y\to X\). Dimostrando che \(G\) sia olomorfa si ha la tesi.

Sia \(y \in Y\) e \(x = G(y)\). Allora \(F\) è un biolomorfismo locale tra \(U_{x}\) e \(V_{x}\), e pertanto
\begin{equation*}
G: V_{x}\to U_{x}
\end{equation*}
è olomorfa. Quindi \(G\) è olomorfa.
\end{proof}
\end{document}
