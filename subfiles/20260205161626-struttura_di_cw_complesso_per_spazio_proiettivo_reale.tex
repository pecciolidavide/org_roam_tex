% Intended LaTeX compiler: pdflatex
\documentclass[../main]{subfiles}


\begin{document}

\section{Struttura di CW-complesso per spazio proiettivo reale}
\label{sec:orga108019}
\begin{lem}
Si consideri lo \href{20241231115051-spazio_proiettivo.org}{spazio proiettivo} reale \(\R\mathds{P}^{n}\) come \href{20250129155316-spazio_topologico_quoziente.org}{quoziente} sulla \href{20250115150754-sfera_n_dimensionale.org}{sfera}, e la mappa:
\begin{align*}
h: \mathds{S}^{n-1} &\longrightarrow\R\mathds{P}^{n-1} \\
p &\longmapsto [p]
\end{align*}
Allora si ha il seguente \href{20250111142332-omeomorfismo.org}{omeomorfismo} con l'\href{20250215122759-export.org_archive}{attaccamento} del \href{20250127170831-disco_n_dimensionale.org}{disco \(\mathds{D}^{n}\)}:
\begin{equation*}
\R\mathds{P}^{n} \cong \R\mathds{P}^{n-1} \mathrel{\cup_{h}} \mathds{D}^{n}
\end{equation*}
\end{lem}
\begin{proof}
Definiamo la funzione
\begin{equation*}
f: \R\mathds{P}^{n-1} \amalg \mathds{D}^{n} \longrightarrow \R\mathds{P}^{n}
\end{equation*}
come segue:
\begin{itemize}
\item per \([x_{0},\dots,x_{n-1}] \in \R\mathds{P}^{n-1}\): \(f[x_{0},\dots,x_{n-1}]= [x_{0},\dots,x_{n-1},0]\);
\item per \(\bm{x} = (x_{0},\dots,x_{n-1}) \in \mathds{D}^{n}\): \(f(\bm{x}) = \left[x_{0},\dots,x_{n-1}, \sqrt{1-\norma{\bm{x}}^{2}}\right]\), dove \(\norma{\cdot}\) è la \href{20250625123506-spazio_normato.org}{norma euclidea di \(\R^{n}\)}.
\end{itemize}

Si ha che \textbf{\(f\) è suriettiva}. Infatti, sia \([x_{0},\dots,x_{n}] \in \R\mathds{P}^{n}\).
\begin{itemize}
\item Se \(x_{n} = 0\), allora \([x_{0},\dots,x_{n}] = f\parentesi{\null\in \R\mathds{P}^{n-1}}{[x_{0},\dots,x_{n-1}]}\).
\item Se \(x_{n} \neq 0\), allora, siccome \(\R\mathds{P}^{n}\) è quoziente sulla sfera:
\begin{equation*}
  \sum_{i = 0}^{n-1}x_{i}^{2} + x_{n}^{2} = 1%
  \IMPLICA%
  x_{n}^{2} = 1- \sum x_{i}^{2}.
\end{equation*}
\begin{itemize}
\item Se \(x_{n} > 0\), allora \([x_{0},\dots,x_{n}] = f(\bm{x})\), con
\begin{equation*}
\bm{x} = (x_{0},\dots,x_{n-1}) \in \mathds{D}^{n}
\end{equation*}
\item Se \(x_{n} < 0\), allora \([x_{0},\dots,x_{n}] = [-x_{0},\dots,-x_{n}]\) e si ricade nel caso precedente.
\end{itemize}
\end{itemize}

Inoltre \textbf{\(f\) è chiusa}: infatti, sia \(C \subseteq \R\mathds{P}^{n-1}\amalg \mathds{D}^{n}\) un chiuso. Siccome \(\R\mathds{P}^{n-1}\amalg \mathds{D}^{n}\) è \href{20250103163701-spazio_topologico_compatto.org}{compatto} (in quanto \href{20260207173708-unione_finita_di_compatti_e_compatta.org}{unione di compatti}), \href{20250401125136-chiuso_in_un_compatto_e_compatto.org}{allora} \(C\) è \href{20250103163701-spazio_topologico_compatto.org}{compatto}. \href{20251229125103-immagine_continua_di_spazio_compatto_e_compatto.org}{Pertanto}, l'\href{20250202190147-immagine_punto_a_punto_di_due_classi.org}{immagine} \(f(C) \subseteq \R\mathds{P}^{n}\) è compatto, con \(\R\mathds{P}^{n}\) di \href{20250109155715-spazio_topologico_di_hausdorff.org}{Hausdorff}. Ma un \href{20250331174140-compatto_in_un_haussdorf_e_chiuso.org}{compatto in un Hausdorff è chiuso}, e quindi \(f(C)\) è chiuso.

Si consideri ora la \href{20250215122759-export.org_archive}{proiezione al quoziente \(\pi: \R\mathds{P}^{n-1} \amalg \mathds{D}^{n} \longrightarrow \R\mathds{P}^{n-1}\mathrel{\cup_{h}} \mathds{D}^{n}\)}. Si dimostra che \(f\) \href{20250129155316-spazio_topologico_quoziente.org}{fattorizza} con \(\pi\). Siano quindi \(p,q \in \R\mathds{P}^{n-1} \amalg \mathds{D}^{n}\) tali che \(\pi(p) = \pi(q)\).
\begin{itemize}
\item Se \(p \in \partial \mathds{D}^{n}\) e \(q \in \R\mathds{P}^{n-1}\) sono tali che \(h(p) = q\)
\begin{align*}
  p &= (x_{0},\dots,x_{n-1}) & &\leadsto & f(p)&= [x_{0},\dots,x_{n-1},0]\\
  q&= [y_{0},\dots,y_{n-1}] & &\leadsto & f(q)&= [y_{0},\dots,y_{n-1}, 0]
\end{align*}
e si ha che
\begin{equation*}
[x_{0},\dots,x_{n-1}] = [y_{0},\dots,y_{n-1}].
\end{equation*}
Quindi \(f(p)=f(q)\).
\item se \(p,q \in \partial\mathds{D}^{n}\) tali che \(h(p) = h(q)\):
\begin{align*}
  p &= (x_{0},\dots,x_{n-1}) & &\leadsto & f(p)&= [x_{0},\dots,x_{n-1},0]\\
  q&= (y_{0},\dots,y_{n-1}) & &\leadsto & f(q)&= [y_{0},\dots,y_{n-1}, 0]
\end{align*}
e si ha che
\begin{equation*}
[x_{0},\dots,x_{n-1}] = [y_{0},\dots,y_{n-1}]
\end{equation*}
Quindi \(f(p)=f(q)\).
\end{itemize}

Per la \href{20250129155316-spazio_topologico_quoziente.org}{proprietà universale della topologia quoziente}, esiste \(\bar{f}\) tale che
\begin{equation*}
\begin{tikzcd}[ampersand replacement=\&]
	{\R\mathds{P}^{n-1}\amalg \mathds{D}^n} \&\& {\R\mathds{P}^n} \\
	\\
	{\R\mathds{P}^{n-1} \mathrel{\cup_h} \mathds{D}^n}
	\arrow["f", from=1-1, to=1-3]
	\arrow[from=1-1, to=3-1]
	\arrow["{\bar{f}}"', from=3-1, to=1-3]
\end{tikzcd}
\end{equation*}
commuti, e in particolare \(\bar{f}\) è un \href{20250111142332-omeomorfismo.org}{omeomorfismo}.
\end{proof}

\begin{cor}
\(\R\mathds{P}^{n}\) ha una struttura di \href{20260205161432-cw_complesso.org}{CW-complesso}: al passo \(k\) si aggiunge una \href{20250215122759-export.org_archive}{cella} di dimensione \(k\), per ogni \(k \in \set{0,\dots,n}\):
\begin{equation*}
\R\mathds{P}^{n} = \bigg(\big((\mathds{D}^{0} \mathrel{\cup_{h}} \mathds{D}^{1}) \mathrel{\cup_{h}} \mathds{D}^{2}\big) \mathrel{\cup_{h}} \dots \bigg) \mathrel{\cup_{h}} \mathds{D}^{n}
\end{equation*}
\end{cor}
\section{Struttura di CW-complesso per spazio proiettivo complesso}
\label{sec:orgc5003b2}
\begin{lem}
Si consideri lo \href{20241231115051-spazio_proiettivo.org}{spazio proiettivo} complesso \(\C\mathds{P}^{n}\) come \href{20250129155316-spazio_topologico_quoziente.org}{quoziente} sulla \href{20250115150754-sfera_n_dimensionale.org}{sfera}, e la mappa:
\begin{align*}
h: \mathds{S}^{2n-1} &\longrightarrow\R\mathds{P}^{n-1} \\
p &\longmapsto [p]
\end{align*}
Allora si ha il seguente \href{20250111142332-omeomorfismo.org}{omeomorfismo} con l'\href{20250215122759-export.org_archive}{attaccamento} del \href{20250127170831-disco_n_dimensionale.org}{disco \(\mathds{D}^{n}\)}:
\begin{equation*}
\C\mathds{P}^{n} \cong \C\mathds{P}^{n-1} \mathrel{\cup_{h}} \mathds{D}^{2n}
\end{equation*}
\end{lem}

\begin{cor}
\(\C\mathds{P}^{n}\) ha una struttura di \href{20260205161432-cw_complesso.org}{CW-complesso}: al passo \(k\) si aggiunge una \href{20250215122759-export.org_archive}{cella} di dimensione \(2k\), per ogni \(k \in \set{0,\dots,n}\).
\end{cor}
\end{document}
