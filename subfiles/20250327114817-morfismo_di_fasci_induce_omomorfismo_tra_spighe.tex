% Intended LaTeX compiler: pdflatex
\documentclass[../main]{subfiles}


\begin{document}

\section{Morfismo di fasci induce omomorfismo tra spighe}
\label{sec:org4042b19}
Sia \(X\) uno \href{20250103145124-topologia.org}{spazio topologico}, \(\mathcal{F}, \mathcal{G}\) due (\href{20250324165349-prefascio.org}{pre})\href{20250324174728-fascio.org}{fasci} di \href{20241205141146-gruppo_abeliano.org}{gruppi} su \(X\) e \(f:\mathcal{F}\to\mathcal{G}\) un morfismo di (\href{20250325180613-morfismo_di_prefasci.org}{pre})\href{20250325180613-morfismo_di_prefasci.org}{fasci}.

Allora, per ogni \(p \in X\), \(f\) induce un \href{20241206115531-morfismo_di_gruppi.org}{morfismo di gruppi} tra le \href{20250325183434-spiga_di_un_prefascio.org}{spighe} corrispondenti\footnote{Infatti \href{20250325183434-spiga_di_un_prefascio.org}{le spighe hanno struttura di gruppo}}
\begin{align*}
f_{p}: \mathcal{F}_{p} &\longrightarrow \mathcal{G}_{p}\\
s &\longmapsto [f_{U}(\psi)]_{p}
\end{align*}
dove \(U\) e \(\psi\) sono come segue: per \(s \in \mathcal{F}_{p}\), esiste \(U \ni p\) \href{20250111142313-intorno.org}{intorno} \href{20250103145124-topologia.org}{aperto} ed esiste \(\psi \in \mathcal{F}(U)\) tale che
\begin{equation*}
s = [\psi]_{p}, \qquad f_{U}(\psi) \in \mathcal{G}(U).
\end{equation*}

\uline{Esercizio}: mostrare che sia ben definito e omomorfismo di gruppi.
\end{document}
