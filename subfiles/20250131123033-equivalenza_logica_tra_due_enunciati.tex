% Intended LaTeX compiler: pdflatex
\documentclass[../main]{subfiles}


\begin{document}

\section{Equivalenza logica tra due enunciati}
\label{sec:orge0fc7c8}
Due \(\mathcal{L}\)-\href{20250131103446-enunciato_del_prim_ordine.org}{enunciati} si dicono \uline{logicamente equivalente} se il loro \href{20250131122913-soddisfazione_di_una_formula.org}{valore di verità} è lo stesso in ogni \href{20250131103035-struttura_del_prim_ordine.org}{struttura} del \href{20250130162057-linguaggio_del_prim_ordine.org}{linguaggio}.
\section{Equivalenza logica tra due formule}
\label{sec:org20f93a4}
Due \(\mathcal{L}\)-formule si dicono \uline{logicamente equivalenti} se i loro \href{20250131122913-soddisfazione_di_una_formula.org}{insiemi di verità} coincidono in ogni struttura del linguaggio.
\subsection{Equivalenza tra due formule in un modello}
\label{sec:org304b4e1}

Fissata una \(\mathcal{L}\)-struttura \(M\), due formule \(\varphi,\psi\) si dicono equivalenti in \(M\) se
\begin{equation*}
M\vDash \varphi(\bm{x})\,\iff\, \psi(\bm{x}).
\end{equation*}
\end{document}
