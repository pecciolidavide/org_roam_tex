% Intended LaTeX compiler: pdflatex
\documentclass[../main]{subfiles}


\begin{document}

\section{Proprietà di Baire}
\label{sec:org9eaedf4}
Sia \(X\) uno \href{20250103145124-topologia.org}{spazio topologico}.
\subsection{Insiemi uguali modulo un magro}
\label{sec:org556143d}
Dati due \href{20250131155822-operazioni_insiemistiche_tra_classi_mk.org}{sottoinsiemi} \(A, B \subseteq X\) si dirà che
\begin{equation*}
A \mathrel{=^{*}} B
\end{equation*}
se la \href{20250131155822-operazioni_insiemistiche_tra_classi_mk.org}{differenza simmetrica} \(A\mathrel{\triangle} B \coloneqq (A\setminus B)\cup (B\setminus A)\) è un \href{20250419122752-insieme_magro.org}{insieme magro}. \footnote{Vedi ``\href{20250131155822-operazioni_insiemistiche_tra_classi_mk.org}{Sottrazione di classi MK}'' e ``\href{20250131155822-operazioni_insiemistiche_tra_classi_mk.org}{Unione di classi MK}''}
\subsection{Definizione}
\label{sec:orge6ded10}

Un \href{20250131155822-operazioni_insiemistiche_tra_classi_mk.org}{sottoinsieme} \(A \subseteq X\) ha la \uline{proprietà di Baire} (\emph{Baire Property} o BP) se esiste un \uline{\href{20250103145124-topologia.org}{aperto}} \(U \subseteq X\) tale che
\begin{equation*}
A\mathrel{=^{*}}B.
\end{equation*}

Si definisce \(\operatorname{BP}(X)\) come la collezione di tutti i sottoinsiemi di \(X\) con BP.
\end{document}
