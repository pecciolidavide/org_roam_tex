% Intended LaTeX compiler: pdflatex
\documentclass[../main]{subfiles}


\begin{document}

\def\U{\mathcal{U}}
\def\L{\mathcal{L}}
\def\X{\mathcal{X}}
\def\Z{\mathcal{Z}}
%
\def\eq{{\rm eq}}
\def\Ueq{\U^\eq}
%
\def\orbita{\mathcal{O}}
\def\Aut{\operatorname{Aut}}
%
\def\tc{\mid}
\def\tp{\operatorname{tp}}
\def\EMtp{\operatorname{EM}\text{-}\operatorname{tp}}
\def\<{\langle}
\def\>{\rangle}
%
\def\restricted#1{\,\mathord{\upharpoonright}{{\scriptstyle #1}}}
\def\equivalentover#1{\mathrel{\equiv_{ #1 }}}

%% NON FORKING
\def\nonforkSymbol{%
\mathbin{\raise1.8ex%
\rlap{\kern0.6ex\rule{0.6ex}{0.1ex}}%
\rlap{\kern1.1ex\rule{0.1ex}{1.9ex}}\raise-0.3ex\hbox{$\smile$}}}
\def\defaultnonforkmodel{M}
\def\nonfork{\nonforkSymbol}
\renewcommand{\nonfork}[1][\defaultnonforkmodel]{%
\mathrel{\nonforkSymbol_{#1}}}
\section{Relazione stabile}
\label{sec:org7932244}
Siano \(\X, \Z\) due \href{20250130104331-insieme_mk.org}{insiemi} arbitrari, e sia \(\pi \subseteq \X\times \Z\) una \href{20250202170607-classe_relazione_binaria.org}{relazione} binaria. Si scriverà indifferentemente
\begin{equation*}
(x,z) \in \pi,\qquad \pi(x,z).
\end{equation*}

\begin{definizione}
\(\pi\) è \uline{instabile} (o ha la \emph{order property}) se per ogni \(m<\omega\) esiste una \href{20250206170922-sequenze_e_stringhe.org}{sequenza}
\(\< (a_{i},\, b_{i}) : i<m \>\), \((a_{i},b_{i}) \in \X\times\Z\)
tale che
\begin{equation*}
i<j<m \IMPLICA \pi(a_{i}, b_{j}) \land \lnot\pi(a_{j}, b_{i}).
\end{equation*}
Questa sequenza prende il nome di \uline{scaletta di lunghezza \(m\) per \(\pi\)} (o \emph{ladder sequence}).
\end{definizione}

\begin{prop}
LSASE:
\begin{enumerate}
\item \(\pi\) è instabile;
\item per ogni \(m<\omega\) esiste \(B \subseteq \Z\) e
\(\< a_{i} \mid i < m \>\) (con  \(a_{i} \in \X\))
tale che
\begin{equation*}
 \pi(a_{0}, B) \subsetneq \pi(a_{1}, B) \subsetneq \dots \subsetneq \pi(a_{m-1},B).
\end{equation*}
dove \(\pi(a, B) \coloneqq \set{b \in B \mid \pi(a,b)}\).
\end{enumerate}
\end{prop}
\end{document}
