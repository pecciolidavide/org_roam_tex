% Intended LaTeX compiler: pdflatex
\documentclass[../main]{subfiles}


\begin{document}

\section{Definizione}
\label{sec:org36a50e5}
Sia \(M\) una \href{20250113115909-struttura_differenziabile.org}{varietà differenziabile}, e sia \(p \in M\). Lo spazio tangente ad \(M\) in \(p\) è
\begin{equation*}
\operatorname{T}_{p}M\coloneqq \set{v: C^{\infty}_{p}\longrightarrow \R: v\text{ derivazione}}
\end{equation*}
(vedi \href{20250114101437-derivazione_su_una_varieta_differenziabile.org}{Derivazione su una varietà differenziabile} e \href{20250114100254-germi_di_funzioni.org}{Germi di funzioni})
\section{Osservazione}
\label{sec:org139191a}
\(\operatorname{T}_{p} M\) è uno \href{20241205142027-spazio_vettoriale.org}{spazio vettoriale} reale. Inoltre \(\operatorname{T}_{p} M \subseteq \mathscr{L}(C^{\infty}_{p}; \R)\) è un \href{20250114103118-sottospazio_vettoriale.org}{sottospazio vettoriale}. (Vedi \href{20250114103136-spazio_delle_funzioni_lineare.org}{Spazio delle funzioni lineare})
\section{Descrizione alternativa}
\label{sec:org75136b6}
Si ha che:
\begin{equation*}
\set{\dot{\alpha}(0):\ \alpha:(-\varepsilon,\varepsilon)\longrightarrow M\text{ funzione }C^{\infty}} = \operatorname{T}_{p}M
\end{equation*}
(vedi \href{20250114110247-curve_su_varieta_e_loro_derivata.org}{Derivata di curve su varietà come derivazioni})
\subsection{Vettori dello spazio tangente ad una varietà in un punto come derivate di curve}
\label{sec:org26f4fb8}
\begin{prop}
Sia \(M\) una \href{20250113115909-struttura_differenziabile.org}{varietà differenziabile} e sia \(p \in M\). Sia \(\bm{v} \in \operatorname{T}_{p}M\) (vedi \href{20250114102823-spazio_tangente_ad_un_punto_di_una_varieta_differenziabile.org}{Spazio tangente ad un punto di una varietà differenziabile}).

Allora esiste \(\alpha:(-\varepsilon,\varepsilon)\longrightarrow M\) \href{20250113144722-funzioni_cinfinito_tra_varieta_differenziabili.org}{funzione \(C^{\infty}\)} tale che (vedi \href{20250114110247-curve_su_varieta_e_loro_derivata.org}{Curve su varietà e loro derivata})
\begin{align*}
\alpha(0) &=p \\
\dot{\alpha}(0) &=\bm{v}
\end{align*}
\end{prop}
\end{document}
