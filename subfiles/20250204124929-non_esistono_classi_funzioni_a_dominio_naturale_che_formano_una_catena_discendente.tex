% Intended LaTeX compiler: pdflatex
\documentclass[../main]{subfiles}


\begin{document}

Contesto: \href{20250130104245-morse_kelly_set_theory.org}{Morse Kelly Set Theory}
\section{Teorema}
\label{sec:org297c38a}
Non esiste nessuna classe-funzione \(F\) tale che \(\operatorname{dom}F=\N\) (vedi \href{20250202173528-dominio_range_e_campo_di_una_classe_relazione.org}{Dominio, Range e Campo di una Classe Relazione}) e
\begin{equation*}
\forall\, n \in \N:\qquad F\left(\operatorname{S}(n)\right) \in F(n)
\end{equation*}
(vedi \href{20250202130045-insieme_dei_numeri_naturali_mk.org}{Insieme dei numeri naturali MK} e \href{20250202124648-successore_di_un_insieme_mk.org}{Successore di un insieme MK})
\subsection{Dimostrazione}
\label{sec:orge139b49}
Supponiamo tale \(F\) esista. Sicuramente \(\operatorname{ran}F\neq \emptyset\), e pertanto, per l'Axiom of Foundation esiste \(y \in \operatorname{ran}F\) tale che
\begin{equation*}
y \cap \operatorname{ran} F = \emptyset
\end{equation*}

Sia ora \(n \in \N\) tale che \(y = F(n)\). Allora \(F\left(\operatorname{S}(n)\right) \in F(n)\) per ipotesi, e \(F\left(\operatorname{S}(n)\right) \in \operatorname{ran}F\) per definizione, e dunque
\begin{equation*}
F\left(\operatorname{S}(n)\right) \in F(n)\cap \operatorname{ran}F = y \cap \operatorname{ran}F = \emptyset
\end{equation*}
\end{document}
