% Intended LaTeX compiler: pdflatex
\documentclass[../main]{subfiles}


\begin{document}

\section{Algebra Graduata}
\label{sec:org1b5abae}
\begin{definizione}
Sia \(R\) un anello e sia \(V\) una \href{20250110175552-algebra_su_un_campo.org}{\(R\)-algebra} (dotata di un prodotto \(g:V\times V\to V\)). \(V\) si dice \uline{algebra graduata} se esistono \(V_{i}\) per \(i \in \N\) delle \(R\)-algebre tali che la \href{20241213095808-somma_diretta.org}{somma diretta (come \(R\)-moduli)}:
\begin{equation*}
V = \bigoplus_{i \in \N} V_{i}
\end{equation*}
e tali che, per ogni \(i,j \in \N\) le \href{20250202173528-dominio_range_e_campo_di_una_classe_relazione.org}{immagini}:
\begin{equation*}
g[V_{i}\times V_{j}] \subseteq V_{i+j}.
\end{equation*}
\end{definizione}
\section{Morfismo tra algebre graduate}
\label{sec:org30ff7cd}
\begin{definizione}
Siano \(V=\bigoplus_{i \in \N} V_{i}\), \(W = \bigoplus_{j \in \N} W_{j}\) due \href{20250110175552-algebra_su_un_campo.org}{algebre} \hyperref[sec:org1b5abae]{graduate}.
\begin{itemize}
\item Un \uline{morfismo di algebre graduate} è \(F:V\to W\) \href{20250110175552-algebra_su_un_campo.org}{morfismo tra algebre} tale che, per ogni \(i \in \N\), l'\href{20250202173528-dominio_range_e_campo_di_una_classe_relazione.org}{immagine}:
\begin{equation*}
  F[V_{i}] \subseteq W_{i}.
\end{equation*}
\item Un \uline{isomorimorfismo tra algebre graduate} è un morfismo \href{20250104111707-funzione_biunivoca.org}{biiettivo} tale che la sua \href{20250111142446-funzione_inversa.org}{inversa} sia ancora un morfismo.
\item Un \uline{morfismo graduato} di grado \(d\) è \(F:V\to W\) \href{20250110175552-algebra_su_un_campo.org}{morfismo tra algebre} tale che, per ogni \(i \in \N\), l'immagine:
\begin{equation*}
  F[V_{i}] \subseteq W_{i+d}.
\end{equation*}
\end{itemize}
\end{definizione}
\end{document}
