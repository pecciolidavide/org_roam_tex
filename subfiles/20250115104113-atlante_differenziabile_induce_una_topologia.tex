% Intended LaTeX compiler: pdflatex
\documentclass[../main]{subfiles}


\begin{document}

\section{Atlante differenziabile induce una topologia}
\label{sec:org0e015ee}
\subsection{Teorema.}
\label{sec:org5a2cae9}
Sia \(X\) un insieme, e sia \(\set{(U_{\alpha},\varphi_{\alpha})}\) una famiglia in cui \(U_{\alpha} \subseteq X\) e \(\varphi_{\alpha}: U_{\alpha}\longrightarrow \R^{N}\) \href{20241219101956-funzione_iniettiva.org}{iniettiva} tale che
\begin{enumerate}
\item per ogni \(\alpha, \beta\) tali che \(U_{\alpha}\cap U_{\beta} \neq \emptyset\)
\begin{equation*}
 \varphi_{\alpha}\left(U_{\alpha}\cap U_{\beta}\right)
\end{equation*}
è \href{20250103145124-topologia.org}{aperto} di \(\R^{N}\) e la funzione
\begin{equation*}
 \varphi_{\beta}\circ\varphi_{\alpha}^{-1}:\varphi_{\alpha}(U_{\alpha}\cap U_{\beta})\longrightarrow \varphi_{\beta}(U_{\alpha}\cap U_{\beta})
\end{equation*}
è una \href{20250113125602-classe_c_di_una_funzione.org}{funzione \(C^{\infty}\) tra aperti di \(\R^{N}\)};

\item \(\set{U_{\alpha}}\) è un \href{20250103164252-ricoprimento.org}{ricoprimento} di \(X\) e ammette un sottoricoprimento \href{20250111143651-insieme_numerabile.org}{numerabile};
\item per ogni \(p,q \in X\) vale una delle seguenti:
\begin{itemize}
\item esiste \({\alpha}\) tale che \(p,q \in U_{\alpha}\);
\item esistono \(\alpha,\beta\) tali che \(U_{\alpha}\cap U_{\beta} = \emptyset\) e \(p \in U_{\alpha}\), \(q \in U_{\beta}\).
\end{itemize}
\end{enumerate}

Allora esiste una \href{20250103145124-topologia.org}{topologia} di \(X\) \href{20250109155715-spazio_topologico_di_hausdorff.org}{T2} e \href{20250111142303-spazio_topologico_a_base_numerabile.org}{a base numerabile} rispetto alla quale \(\set{(U_{\alpha},\varphi_{\alpha})}\) è un \href{20250113103136-atlante_topologico_differenziabile.org}{atlante differenziabile}.
\end{document}
