% Intended LaTeX compiler: pdflatex
\documentclass[../main]{subfiles}


\begin{document}

\section{0-gruppo di coomologia di de Rham di una varietà connessa}
\label{sec:orged2161d}
Sia \(M\) una varietà differenziabile, e sia \(\operatorname{H}^{0}(M)\) lo \(0\)-\href{20251115172442-gruppo_di_coomologia_di_de_rham.org}{gruppo di coomologia di De Rham} di \(M\).
\begin{itemize}
\item Se \(M\) è connesso, allora \(\operatorname{H}^{0}(M) = \R\).
\item Se \(M = \coprod_{\alpha \in A} M_{\alpha}\), con \(M_{\alpha}\) \href{20250325160128-componente_connessa_di_uno_spazio_topologico.org}{componenti connesse}, allora \(\operatorname{H}^{0}(M) = \R^{A}\), dove con \(\R^{A}\) si intende la \href{20241213095808-somma_diretta.org}{somma diretta} di \(A\) copie di \(\R\).
\end{itemize}

Questo segue dal fatto che le funzioni a valori in \(\R\) con le derivate nulle sono necessariamente costanti sulle componenti connesse (vedi ``\href{20250325154046-funzione_localmente_costante_sse_costante_sulle_componenti_connesse.org}{Funzione localmente costante sse costante sulle componenti connesse}'' e ``\href{20251222142956-teorema_della_derivata_nulla.org}{Teorema della derivata nulla}'').
\end{document}
