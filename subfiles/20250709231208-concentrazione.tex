% Intended LaTeX compiler: pdflatex
\documentclass[../main]{subfiles}

\usepackage[hyperref]{biblatex}
\date{}
\title{}
\begin{document}

\section{Concentrazione}
\label{sec:org00ea864}
\subsection{Metodi per concentrarsi (e mantenere la concentrazione) \textit{<2025-07-09 Mer>}}
\label{sec:org3dd7067}

\begin{itemize}
\item Le sessioni di concentrazione non dovrebbero essere troppo lunghe: concentrarsi per più di 90 minuti richiede un grande allenamento. Dividere tutte le sessioni di studio e concentrazione in slot orari da 60 minuti.
\item Affinché si riesca a trovare la forza di volontà per concentrarsi, è necessario avere degli \uline{obiettivi}:
\begin{itemize}
\item quelli a lungo termine (come ad esempio passare un esame)
\item quelli a breve termine: suddividere il nostro obiettivo in tanti step intermedi aiuta a non essere sopraffatti dal carico di lavoro, e \uline{raggiungere un obiettivo} ci garantisce della soddisfazione che riesce ad alimentare la nostra volontà di fare.
\end{itemize}
\item Per aiutare la nostra mente a non vagare, tenere di fianco alla postazione di studio un quadernino, in cui \uline{svuotare la mente}: ogni idea, cosa da fare, pensiero che rischia di distrarci, deve finire dentro al quadernino. In questo modo, non lasceremo vagare la nostra mente, e ci scaricheremo il carico mentale di tenere a mente ``cose inutili''. Finita la sessione, potremo occuparci del nostro quaderno.
\end{itemize}
\end{document}
