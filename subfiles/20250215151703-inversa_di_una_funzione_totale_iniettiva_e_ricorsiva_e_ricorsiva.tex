% Intended LaTeX compiler: pdflatex
\documentclass[../main]{subfiles}


\begin{document}

Vedi: \href{20250202130045-insieme_dei_numeri_naturali_mk.org}{Insieme dei numeri naturali MK}
\begin{prop}
Sia \(f: \N\to \N\) una \href{20250213105339-funzione_parziale.org}{funzione totale} \href{20241219101956-funzione_iniettiva.org}{iniettiva}. Se \(f\) è una \href{20250207104855-funzione_ricorsiva.org}{funzione ricorsiva}, allora la \href{20250213105339-funzione_parziale.org}{funzione parziale} \(f^{-1}: \N\to \N\) è \href{20250207104855-funzione_ricorsiva.org}{ricorsiva}.
\end{prop}
\begin{proof}
Sia
\begin{align*}
h: \N^{2} &\longrightarrow \N\\
(x,y) &\longmapsto |f(y)-x|
\end{align*}
che è una funzione ricorsiva poiché la \href{20250215141024-funzioni_primitive_ricordive.org}{distanza} è una \href{20250215141024-funzioni_primitive_ricordive.org}{funzione ricorsiva primitiva}, a cui è applicato lo \href{20250215151458-funzioni_ricorsive.org}{schema di composizione}.

Allora\footnote{Vedi \href{20250215151458-funzioni_ricorsive.org}{Schema di minimizzazione}}
\begin{equation*}
f^{-1}(x) = \minim{y}{h(x,y)=0}
\end{equation*}
\end{proof}
\end{document}
