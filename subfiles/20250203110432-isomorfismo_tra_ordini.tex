% Intended LaTeX compiler: pdflatex
\documentclass[../main]{subfiles}


\begin{document}

\textbf{\textbf{\uline{NOTA}: quando si parla di \uline{classi}, se ne parla nell'ambito della \href{20250130104245-morse_kelly_set_theory.org}{Morse Kelly Set Theory}; quando si parla di insiemi, il discorso ha validità più generale.}}
\section{Definizione}
\label{sec:orgbc03894}

Siano \(X,Y\) due \href{20250130104331-insieme_mk.org}{insiemi} (o \href{20250130104320-classe_mk.org}{classi}) \href{20250203101604-ordine.org}{ordinati} da \(R_{1},R_{2}\) e sia \(f:X\to Y\) una \href{20250202170607-classe_relazione_binaria.org}{funzione} (o \href{20250202170607-classe_relazione_binaria.org}{classe funzione}).

\(f\) è un \uline{isomorfismo} se è \href{20250104111707-funzione_biunivoca.org}{biiettiva} e
\begin{equation*}
\forall\,x,y \in X\ x\mathrel{R_{1}}y\iff f(x)\mathrel{R_{2}}f(y)
\end{equation*}
\end{document}
