% Intended LaTeX compiler: pdflatex
\documentclass[../main]{subfiles}


\begin{document}

\section{Omologia Singolare Ridotta}
\label{sec:orgdf89b2f}
Sia \(R\) un \href{20241219112842-pid.org}{PID} e sia \(X\) uno spazio topologico.

\begin{definizione}
Se \(\mathcal{\tilde{S}}_{\bullet}(X)\) è il \href{20250120163114-complesso_di_catene.org}{complesso di catene} \href{20250122154349-complesso_di_catene_singolare_augmentato.org}{singolare augmentato}, si definisce l'\textbf{omologia singolare ridotta} di \(X\) come l'\href{20250120164857-modulo_di_omologia_dei_complessi_di_catene.org}{omologia} di \(\mathcal{\tilde{S}}_{\bullet}(X)\):
\begin{equation*}
\tilde{H}_{q}(X)\coloneqq H_{q}\left(\mathcal{\tilde{S}}_{\bullet}(X)\right)
\end{equation*}
\end{definizione}

\begin{prop}
\href{20250122103014-omologia_ridotta_di_un_complesso_simpliciale.org}{Analogamente} all'\href{20250121160827-omologia_simpliciale.org}{omologia simpliciale}, indicando con \(H_{q}(X)\) \href{20250122133631-omologia_singolare.org}{l'omologia singolare di \(X\)}, si ha che\footnote{Vedi ``\href{20241213095808-somma_diretta.org}{Somma Diretta}''}
\begin{equation*}
q>0:\ \tilde{H}_{q}(X) = H_{q}(X) \qquad\qquad
H_{0}(X) = \tilde{H}_{0}(X)\oplus R.
\end{equation*}
\end{prop}
\end{document}
