% Intended LaTeX compiler: pdflatex
\documentclass[../main]{subfiles}


\begin{document}

\section{Grafico di funzioni olomorfe è superficie di Riemann}
\label{sec:org10369c1}
Sia \(f:U \subseteq \C\to \C\) una \href{20260126110551-funzione_olomorfa.org}{funzione olomorfa}, con \(U\) \href{20250103145124-topologia.org}{aperto}. Sia
\begin{equation*}
\Gamma_{f} \coloneqq \set{(z,f(z)) \in \C^{2} \mid z \in U}
\end{equation*}
il suo \href{20250104112443-grafico_di_una_funzione.org}{grafico}.

\begin{prop}
\(\Gamma_{f}\) è una \href{20260127112828-superficie_di_riemann.org}{superficie di Riemann} con \href{20260127112715-atlante_complesso.org}{carta locale} data dalla proiezione sul primo fattore
\begin{align*}
\pi: \Gamma_{f} &\longrightarrow \C\\
(z,f(z)) &\longmapsto z
\end{align*}
\end{prop}

\begin{oss}
In generale, date \(f_{1},\dots,f_{n}: U \to \C\) olomorfe, il grafico:
\begin{equation*}
\Gamma_{f} \coloneqq \set{\big(z,f_{1}(z),\dots,f_{n}(z)\big) \in \C^{n+1} \mid z \in U}
\end{equation*}
è una \href{20260127112828-superficie_di_riemann.org}{superficie di Riemann} con \href{20260127112715-atlante_complesso.org}{carta locale} data dalla proiezione sul primo fattore
\begin{align*}
\pi: \Gamma_{f} &\longrightarrow \C\\
\big(z,f_{1}(z),\dots,f_{n}(z)\big) &\longmapsto z
\end{align*}
\end{oss}
\end{document}
