% Intended LaTeX compiler: pdflatex
\documentclass[../main]{subfiles}

\usepackage[hyperref]{biblatex}
\date{}
\title{}
\begin{document}

\section{Fattori non costanti comuni tra polinomi}
\label{sec:orge543b2d}
Sia \(A\) un \href{20250108173116-ufd.org}{UFD}. (\href{20250108173116-ufd.org}{Allora} \(A[X]\) è un UFD)

Siano \(f,g \in A[x]\),
\begin{align*}
f(x)&=a_{m}\ x^{m}+a_{m-1}\ x^{m-1} + \dots + a_{0}\\
g(x)&= b_{n}\ x^{n} + b_{n-1}\ x^{n-1}+\dots+b_{0}
\end{align*}
\subsection{Lemma 1}
\label{sec:org81174f3}
\(f\) e \(g\) hanno \href{20250108173845-fattore_comune.org}{fattori comuni} non costanti se e solo se esiste \(h \in A[x]\) tale che
\begin{itemize}
\item \(\deg h < n+m=\deg fg\) (vedi \href{20241231124742-grado_polinomi.org}{Grado-Polinomi})
\item \(f\) e \(g\) \href{20250108174027-divisione.org}{dividono} \(h\).
\end{itemize}
\subsubsection{Dimostrazione}
\label{sec:org451e4c7}

\paragraph{``\(\implies\)''}
\label{sec:orgd3f2949}
Supponiamo che \(f\) e \(g\) abbiamo come fattore comune non costante \(k \in A[x]\):
\begin{equation*}
f=uk,\qquad g=vk
\end{equation*}
con \(u,v \in A[x]\), \(\deg u<\deg f\), \(\deg v<\deg g\).

Allora si ponga \(h=uvk\).
\begin{equation*}
\deg h = \deg u + \deg vk < \deg f + \deg g = n+m
\end{equation*}
\paragraph{``\(\impliedby\)''}
\label{sec:orge6ab75c}
Sia \(h=uf=vg\). Allora \(\deg(u)=\deg h - \deg f<n = \deg g\)
Dunque i fattori di \(g\) non possono essere dentro ad \(u\), e pertanto ci sono fattori di \(g\) dentro ad \(f\).
\subsection{Matrice di Sylvester}
\label{sec:org044dbd9}
La matrice di Sylvester tra \(f\) e \(g\) è una matrice \((n+m)\ (n+m)\)
\begin{equation*}
S_{f,g} =\left[\begin{aligned}
            \hphantom{\,n\lbrace}&\overset{m+1}{\overbrace{\hphantom{\begin{matrix}
                a_0 & a_1 & \dots & a_{m}\\[1ex]
                0 & a_0 & a_1 & \dots\\[1ex]
                \vdots & & \ddots & \ddots \\[1ex]
                0 & \dots & 0 & a_0
            \end{matrix}}}} & & \overset{n-1}{\!\overbrace{\hphantom{\begin{matrix}
                0 & 0 & \dots & 0\\[1ex]
                a_{m} & 0 & \dots & 0\\[1ex]
                \\[1ex]
                \dots & \dots & \dots & a_{m}
            \end{matrix}}}}\\[-2ex]
            \hphantom{\,n\lbrace}&\begin{matrix}
                a_0 & a_1 & \dots & a_{m}\\[1ex]
                0 & a_0 & a_1 & \dots\\[1ex]
                \vdots & & \ddots & \ddots \\[1ex]
                0 & \dots & 0 & a_0
            \end{matrix} & &\!\left.\begin{matrix}
                0 & 0 & \dots & 0\\[1ex]
                a_{m} & 0 & \dots & 0\\[1ex]
                \\[1ex]
                \dots & \dots & \dots & a_{m}
            \end{matrix}\right\rbrace\,n\\\hline
            \hphantom{\,n\lbrace}& \begin{matrix}
                b_0 & b_1 & \dots & b_{n}\\[1ex]
                0 & b_0 & b_1 & \dots\\[1ex]
                \vdots & & \ddots & \ddots\\[1ex]
                0 & \dots & 0 & b_0
            \end{matrix} & &\!\left.\begin{matrix}
                0 & 0 & \dots & 0\\[1ex]
                b_{n} & 0 & \dots & 0\\[1ex]
                \\[1ex]
                \dots & \dots & \dots & b_{n}
            \end{matrix}\right\rbrace\,m\\[-3.6ex]
            \hphantom{\,n\lbrace}&\parentesi{n+1}{\hphantom{\begin{matrix}
                b_0 & b_1 & \dots & b_{n}\\[1ex]
                0 & b_0 & b_1 & \dots\\[1ex]
                \vdots\\[1ex]
                0 & \dots & 0 & b_0
            \end{matrix}}} & &\parentesi{m-1}{\hphantom{\!
                \begin{matrix}
                    0 & 0 & \dots & 0\\[1ex]
                    b_{n} & 0 & \dots & 0\\[1ex]
                    \\[1ex]
                    \dots & \dots & \dots & b_{n}
                \end{matrix}
            }}
        \end{aligned}\right]
\end{equation*}
Il risultante è
\begin{equation*}
\operatorname{Res}(f,g)\coloneqq \det S_{f,g} \in A
\end{equation*}
(vedi \href{20250104111751-determinante_di_una_matrice.org}{Determinante di una matrice})
\subsection{Lemma 2}
\label{sec:org4111c9b}
\(f\) e \(g\) hanno un fattore non costante in comune se e solo se \(\operatorname{Res}(f,g)=0\).
\subsubsection{Dimostrazione}
\label{sec:org995dc6e}
Dal lemma 1, \(f,g\) hanno un fattore non costante in comune se e solo se esiste \(h \in A[x]\) di grado minore di \(m+n\) divisibile per \(f\) e per \(g\).
Ma tale \(h\) esiste se e solo se i polinomi seguenti sono \href{20241212142019-insiemi_linearmente_indipendenti.org}{linearmente dipendenti}:
\begin{equation*}
E=\set{f,xf,x^{2}f,\dots,x^{n-1}f, g, xg, x^{2}g, \dots, x^{m-1}g}
\end{equation*}

\begin{itemize}
\item Se tali polinomi sono linearmente dipendenti allora esiste \(\set{\alpha_{i}}_{i=0}^{n+m-1}\) non tutti nulli tale che
\begin{equation*}
  \alpha_{0}f + \alpha_{1}xf + \dots + \alpha_{n-1}x^{n-1}f + \alpha_{n}g + \alpha_{n+1}xg + \dots + \alpha_{n+m-1}x^{m-1}g = 0
\end{equation*}
Pertanto
\begin{equation*}
  \parentesi{u}{(\alpha_{0} + \alpha_{1}x + \dots +\alpha_{n-1}x^{n-1})}\ f = \parentesi{v}{-(\alpha_{n} + \alpha_{n+1}x + \dots + \alpha_{n + m-1}x^{m-1})}\ g
\end{equation*}
Siccome \(\deg u < \deg g\), ci devono necessariamente essere dei fattori di \(g\) dentro ad \(f\), e pertanto esiste \(h\) per il lemma precedente.

\item Viceversa, se \(h\) esiste allora \(h=uf=vg\) con
\begin{equation*}
  p=\deg u < \deg g=n,\qquad q= \deg v<\deg f=m
\end{equation*}
e pertanto \(uf-vg\) è \href{20250108181703-combinazione_lineare.org}{combinazione lineare} di elementi di \(E\); infatti
\begin{equation*}
  uf = (c_{p}x^{p} +\dots +c_{0} ) f  = c_{p} (x^{p}f) + \dots + c_{1}(xf) c_{0} f,\qquad p<n
\end{equation*}
Siccome \(uf-vg=0\) a è combinazione lineare coefficienti non nulli, allora \(E\) è linearmente dipendente.
\end{itemize}


Siccome la Matrice di Sylvester ha per righe le componenti dei polinomi di \(E\) rispetto alla base \(\set{1,x,x^{2},\dots,x^{m+n-1}}\), \(E\) è linearmente dipendente se e solo se \(\det S_{f,g}=0\) (vedi \href{20250104111751-determinante_di_una_matrice.org}{Determinante di una matrice}).
\end{document}
