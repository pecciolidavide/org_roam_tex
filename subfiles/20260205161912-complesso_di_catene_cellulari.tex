% Intended LaTeX compiler: pdflatex
\documentclass[../main]{subfiles}

\usepackage[hyperref]{biblatex}
\date{}
\title{}
\begin{document}

\section{Complesso di catene cellulari}
\label{sec:orgb107e71}
Sia \(X\) un \href{20260205161432-cw_complesso.org}{CW-complesso}, \(R\) \href{20241219112842-pid.org}{PID} fissato.

\begin{definizione}
Il \textbf{complesso di catene cellulari} \(\mathcal{S}_{\bullet}^{\text{cw}}(X)\) è il complesso di catene di \(R\)-moduli,
\begin{equation*}
\mathcal{S}_{\bullet}^{\text{cw}}(X) \coloneqq \set{(S_{q}^{\text{cw}}, \operatorname{d}_{q})}_{q \in \Z}
\end{equation*}
tale che
\begin{enumerate}
\item \(S_{q}^{\text{cw}} = H_{q}(X_{q}, X_{q-1}) \cong \oplus_{\alpha} R\cdot [e_{\alpha}^{q}]\) è l'\href{20250122154903-omologia_singolare_relativa.org}{omologia singolare relativa} \href{20260205161836-omologia_singolare_di_cw_complessi.org}{del CW-complesso}.
\item \(\operatorname{d}_{q}: S_{q}^{\text{cw}}(X) \longrightarrow S_{q-1}^{\text{cw}}(X)\) è definita come \(\operatorname{d}_{q} \coloneqq \pi_{q-1}^{(q-1,q-2)}\circ \partial_q^{(q,q-1)\) nel diagramma commutativo di Figura~\ref{fig_SELdoppia}, dove le righe sono le \href{20250122154927-successione_esatta_di_una_coppia_topologica.org}{SEL dell'omologia relativa}.
\end{enumerate}
\end{definizione}

\begin{figure}
\begin{equation*}
\scalebox{0.92}{%
\begin{tikzcd}[ampersand replacement=\&]
	{H_q(X_q)} \& \textcolor{rgb,255:red,214;green,92;blue,92}{{{H_q(X_q,X_{q-1})}}} \&\& {H_{q-1}(X_{q-1})} \&\& {H_{q-1}(X_q)} \\
	\\
	\& {H_{q-1}(X_{q-2})} \&\& {H_{q-1}(X_{q-1})} \&\& \textcolor{rgb,255:red,214;green,92;blue,92}{{{H_{q-1}(X_{q-1},X_{q-2})}}} \& {H_{q-2}(X_{q-2})}
	\arrow[from=1-1, to=1-2]
	\arrow["{{\partial_q^{(q,q-1)}}}", from=1-2, to=1-4]
	\arrow[from=1-4, to=1-6]
	\arrow[equals, from=1-4, to=3-4]
	\arrow[from=3-2, to=3-4]
	\arrow["{{\pi_{q-1}^{(q-1,q-2)}}}"', from=3-4, to=3-6]
	\arrow[from=3-6, to=3-7]
\end{tikzcd}%
}
\end{equation*}
\caption{\label{fig_SELdoppia}Le due righe della successione esatta lunga per l'omologia relativa}
\end{figure}

\begin{prop}
La composizione \(\operatorname{d}_{q-1}\circ \operatorname{d}_{q} = 0\).
\end{prop}
\begin{proof}
È sufficiente considera il seguente diagramma commutativo
\begin{equation*}
\scalebox{0.6}{%
\begin{tikzcd}[ampersand replacement=\&]
	{H_q(X_q)} \& \textcolor{rgb,255:red,214;green,92;blue,92}{{{H_q(X_q,X_{q-1})}}} \&\& {H_{q-1}(X_{q-1})} \&\& {H_{q-1}(X_q)} \\
	\\
	\& {H_{q-1}(X_{q-2})} \&\& {H_{q-1}(X_{q-1})} \&\& \textcolor{rgb,255:red,214;green,92;blue,92}{{{H_{q-1}(X_{q-1},X_{q-2})}}} \&\& {H_{q-2}(X_{q-2})} \\
	\\
	\&\&\&\&\& {H_{q-2}(X_{q-3})} \&\& {H_{q-2}(X_{q-2})} \&\& \textcolor{rgb,255:red,214;green,92;blue,92}{{{H_{q-2}(X_{q-2},X_{q-3})}}} \& {H_{q-2}(X_{q-2})}
	\arrow[from=1-1, to=1-2]
	\arrow["{{\partial_q^{(q,q-1)}}}", from=1-2, to=1-4]
	\arrow["{\operatorname{d}_q}"'{pos=0.3}, color={rgb,255:red,214;green,92;blue,92}, from=1-2, to=3-6]
	\arrow[from=1-4, to=1-6]
	\arrow[equals, from=1-4, to=3-4]
	\arrow[from=3-2, to=3-4]
	\arrow["{{\pi_{q-1}^{(q-1,q-2)}}}"', from=3-4, to=3-6]
	\arrow["{\partial_{q-1}^{(q-1,q-2)}}", from=3-6, to=3-8]
	\arrow["{\operatorname{d}_{q-1}}"'{pos=0.3}, color={rgb,255:red,214;green,92;blue,92}, from=3-6, to=5-10]
	\arrow[equals, from=3-8, to=5-8]
	\arrow[from=5-6, to=5-8]
	\arrow["{{\pi_{q-2}^{(q-2,q-3)}}}"', from=5-8, to=5-10]
	\arrow[from=5-10, to=5-11]
\end{tikzcd}%
}
\end{equation*}
e notare che
\begin{equation*}
\operatorname{d}_{{q-1}}\circ \operatorname{d}_{{q}} = \bigg({\pi_{q-2}^{(q-2,q-3)}} \circ \partial_{q-1}^{(q-1,q-2)}\bigg)\circ \bigg({\pi_{q-1}^{(q-1,q-2)}}\circ {\partial_q^{(q,q-1)}}\bigg)
\end{equation*}
con \(\partial_{q-1}^{(q-1,q-2)} \circ \pi_{q-1}^{(q-1,q-2)}=0\) in quanto mappe successive di una successione esatta.
\end{proof}
\subsection{Calcolo esplicito della mappa di bordo}
\label{sec:orga891f09}

TODO
\section{Omologia Cellulare}
\label{sec:orga1fa8ea}
\begin{definizione}
L'\textbf{omologia cellulare} di \(X\) è l'\href{20250120164857-modulo_di_omologia_dei_complessi_di_catene.org}{omologia del complesso} di \hyperref[sec:orgb107e71]{catene cellulari}:
\begin{equation*}
H_{q}^{\text{cw}} (X) \coloneqq %
H_{q}\bigg(\mathcal{S}_{\bullet}^{\text{cw}}(X)\bigg).
\end{equation*}
\end{definizione}
\begin{prop}
Se \(X\) è un CW-complesso, allora per ogni \(q\) l'\href{20250122133631-omologia_singolare.org}{omologia singolare} \href{20241206115416-morfismi_r_moduli.org}{coincide} con l'\hyperref[sec:orga1fa8ea]{omologia cellulare}:
\begin{equation*}
H_{q}(X) \cong H_{q}^{\text{cw}}(X).
\end{equation*}
\end{prop}
\begin{proof}
Per definizione, dato questo diagramma:
\begin{equation*}
\scalebox{0.8}{%
\begin{tikzcd}[ampersand replacement=\&]
	\textcolor{rgb,255:red,214;green,92;blue,92}{{{{H_{q+1}(X_{q+1},X_q)}}}} \&\& {H_q(X_q)} \&\& {H_q(X_{q+1})} \&\& {H_q(X_{q+1},X_q)} \\
	\\
	{H_q(X_{q-1})} \&\& {H_q(X_q)} \&\& \textcolor{rgb,255:red,214;green,92;blue,92}{{{{{H_q(X_q,X_{q-1})}}}}} \&\& {H_{q-1}(X_{q-1})} \&\& {H_{q-1}(X_q)} \\
	\\
	\&\&\&\& {H_{q-1}(X_{q-2})} \&\& {H_{q-1}(X_{q-1})} \&\& \textcolor{rgb,255:red,214;green,92;blue,92}{{{{{H_{q-1}(X_{q-1},X_{q-2})}}}}}
	\arrow["{{{\partial_{q+1}^{(q+1,q)}}}}", from=1-1, to=1-3]
	\arrow["{{{\operatorname{d}_{q+1}}}}"'{pos=0.2}, color={rgb,255:red,214;green,92;blue,92}, from=1-1, to=3-5]
	\arrow[from=1-3, to=1-5]
	\arrow[from=1-5, to=1-7]
	\arrow[from=3-1, to=3-3]
	\arrow[equals, from=3-3, to=1-3]
	\arrow["{{{\pi_q^{(q,q-1)}}}}", from=3-3, to=3-5]
	\arrow["{{{{\partial_q^{(q,q-1)}}}}}", from=3-5, to=3-7]
	\arrow["{{{\operatorname{d}_q}}}"'{pos=0.3}, color={rgb,255:red,214;green,92;blue,92}, from=3-5, to=5-9]
	\arrow[from=3-7, to=3-9]
	\arrow[equals, from=3-7, to=5-7]
	\arrow[from=5-5, to=5-7]
	\arrow["{{{{\pi_{q-1}^{(q-1,q-2)}}}}}"', from=5-7, to=5-9]
\end{tikzcd}%
}
\end{equation*}
si ha che \(H_{q}^{\text{cw}}(X) = \frac{\ker \operatorname{d}_{q}}{\operatorname{Im}\operatorname{d}_{q+1}}\)\footnote{Vedi ``\href{20241213105201-kernel.org}{Kernel}'' e ``\href{20250202190147-immagine_punto_a_punto_di_due_classi.org}{Immagine e retroimmagine tramite una funzione}'' e ``\href{20241206142802-sottomoduli.org}{Quoziente di Moduli}''}.

Sostituendo le informazioni sull'\href{20260205161836-omologia_singolare_di_cw_complessi.org}{omologia singolare} e \href{20260205161836-omologia_singolare_di_cw_complessi.org}{relativa degli scheletri}:
\begin{equation*}
\scalebox{0.8}{%
\begin{tikzcd}[ampersand replacement=\&]
	\textcolor{rgb,255:red,214;green,92;blue,92}{{{{H_{q+1}(X_{q+1},X_q)}}}} \&\& {H_q(X_q)} \&\& \textcolor{rgb,255:red,255;green,51;blue,85}{{H_q(X)}} \& \textcolor{rgb,255:red,255;green,51;blue,85}{0} \\
	\\
	\textcolor{rgb,255:red,255;green,51;blue,85}{0} \&\& {H_q(X_q)} \&\& \textcolor{rgb,255:red,214;green,92;blue,92}{{{{{H_q(X_q,X_{q-1})}}}}} \&\& {H_{q-1}(X_{q-1})} \&\& {H_{q-1}(X_q)} \\
	\\
	\&\&\&\& \textcolor{rgb,255:red,255;green,51;blue,85}{0} \&\& {H_{q-1}(X_{q-1})} \&\& \textcolor{rgb,255:red,214;green,92;blue,92}{{{{{H_{q-1}(X_{q-1},X_{q-2})}}}}}
	\arrow["{{{\partial_{q+1}^{(q+1,q)}}}}", from=1-1, to=1-3]
	\arrow["{{{\operatorname{d}_{q+1}}}}"'{pos=0.2}, color={rgb,255:red,214;green,92;blue,92}, from=1-1, to=3-5]
	\arrow[from=1-3, to=1-5]
	\arrow[from=1-5, to=1-6]
	\arrow[from=3-1, to=3-3]
	\arrow[equals, from=3-3, to=1-3]
	\arrow["{{{\pi_q^{(q,q-1)}}}}", from=3-3, to=3-5]
	\arrow["{{{{\partial_q^{(q,q-1)}}}}}", from=3-5, to=3-7]
	\arrow["{{{\operatorname{d}_q}}}"'{pos=0.3}, color={rgb,255:red,214;green,92;blue,92}, from=3-5, to=5-9]
	\arrow[from=3-7, to=3-9]
	\arrow[equals, from=3-7, to=5-7]
	\arrow[from=5-5, to=5-7]
	\arrow["{{{{\pi_{q-1}^{(q-1,q-2)}}}}}"', from=5-7, to=5-9]
\end{tikzcd}%
}
\end{equation*}
e \href{20250120130155-caratterizzazione_di_alcune_successioni_esatte_di_r_moduli.org}{quindi}, dall'alto verso il basso::
\begin{enumerate}
\item \(H_{q}(X) \cong \frac{H_{q}(X_{q})}{\operatorname{Im}\partial_{q+1}^{(q+1,q)}}\);
\item \(\pi_{q}^{(q,q-1)} : H_{q}(X_{q}) \longrightarrow H_{q}(X_{q}, X_{q-1})\) è \href{20241219101956-funzione_iniettiva.org}{iniettiva};
\item \(\pi_{q-1}^{(q-1,q-2)} : H_{q-1}(X_{q-1})\longrightarrow H_{q-1}(X_{q-1}, X_{q-2})\) è iniettiva.
\end{enumerate}

Siccome \(\pi_{q}^{(q,q-1)}\) è iniettiva, \href{20241206142802-sottomoduli.org}{allora} da 1. posso arrivare a
\begin{equation*}
H_{q}(X) \cong \frac{H_{q}(X_{q})}{\operatorname{Im}\partial_{q+1}^{(q+1,q)}} %
\cong \frac{\operatorname{Im} \pi_{q}^{(q,q-1)}}{\operatorname{Im} \bigg(\pi_{q}^{(q,q-1)}\circ \partial_{q+1}^{q+1,q}\bigg)} \cong %
\frac{\ker \partial_{q}^{(q,q-1)}}{\operatorname{Im}\operatorname{d}_{q+1}}
\end{equation*}
dove l'ultimo \href{20241206115416-morfismi_r_moduli.org}{isomorfismo} è dato dal fatto che le righe siano \href{20250120125004-successione_di_r_moduli_esatta.org}{successioni esatte} e dalla definizione di \(\operatorname{d}_{q+1}\).

Siccome \(\pi_{q-1}^{(q-1,q-2)}\) è iniettiva, \href{20241213105201-kernel.org}{allora}
\begin{equation*}
\ker \partial_{q}^{(q,q-1)} = \ker \pi_{q-1}^{(q-1,q-2)}\circ \partial_{q}^{(q,q-1)} = \ker \operatorname{d}_{q}
\end{equation*}
e dunque
\begin{equation*}
H_{q}(X) \cong \frac{\ker \partial_{q}^{(q,q-1)}}{\operatorname{Im}\operatorname{d}_{q+1}} %
\cong \frac{\ker \operatorname{d}_{q}}{\operatorname{Im}\operatorname{d}_{q+1}} = H_{q}^{\text{cw}}(X).\qedhere
\end{equation*}
\end{proof}
\end{document}
