% Intended LaTeX compiler: pdflatex
\documentclass[../main]{subfiles}

\usepackage[hyperref]{biblatex}
\date{}
\title{}
\begin{document}

\section{Estensione finita di una teoria decidibile è decidibile}
\label{sec:orgce2d367}
Si utilizza la notazione di ``\href{20250609104647-buona_codifica_di_un_linguaggio.org}{Aritmetizzazione della sintassi}''.

Sia \(L\) un linguaggio del prim'ordine numerabile. Sia \(\#\) una \href{20250609104647-buona_codifica_di_un_linguaggio.org}{buona codifica per \(L\)}.
\subsection{Teorema}
\label{sec:org1f41f19}

Se \(T\) è una \(L\)-\href{20250130114950-teoria_del_prim_ordine.org}{teoria} \href{20250610134232-teoria_decidibile.org}{decidibile}, allora per ogni \(L\)-\href{20250131103446-enunciato_del_prim_ordine.org}{enunciato} \(\sigma\), \(T\cup\set{\sigma}\) è \href{20250610134232-teoria_decidibile.org}{decidibile}.
\subsubsection{Idea di dimostrazione}
\label{sec:org14f02b3}

Utilizza la funzione
\begin{equation*}
f:\N\to \N:n\mapsto \godelcode{\#(\implies), \termcode{\sigma}, n}
\end{equation*}
tale che \(f(\termcode{\varphi}) = \termcode{\sigma\implies \varphi}\), dove \(\godelcode{}\) è la \href{20250531110737-codifica_delle_sequenze_finite_tramite_beta_di_godel.org}{codifica di Godel}.

Nota
\begin{equation*}
T\cup\set{\sigma}\vdash \varphi\quad\iff\quad T\vdash\sigma\implies\varphi
\end{equation*}
\subsection{Corollario}
\label{sec:orgec1021d}

Se una \href{20250130114950-teoria_del_prim_ordine.org}{estensione finita} di una \href{20250130114950-teoria_del_prim_ordine.org}{teoria} \(T\) è \href{20250610134232-teoria_decidibile.org}{indecidibile}, allora anche \(T\) è \href{20250610134232-teoria_decidibile.org}{indecidibile}.
\end{document}
