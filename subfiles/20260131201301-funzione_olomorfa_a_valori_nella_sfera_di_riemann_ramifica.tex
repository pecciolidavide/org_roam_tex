% Intended LaTeX compiler: pdflatex
\documentclass[../main]{subfiles}

\def\mult#1{\mathrm{mult}_{#1}\,}


\begin{document}

\section{Funzione olomorfa a valori nella sfera di Riemann ramifica}
\label{sec:org6837c27}
Sia \(X\) una \href{20260127112828-superficie_di_riemann.org}{superficie di Riemann} \href{20250103163701-spazio_topologico_compatto.org}{compatta}, \(\C_{\infty}\) la \href{20260127112905-sfera_di_riemann.org}{sfera di Riemann}, e \(F:X\to Y\) una \href{20260128143822-funzione_olomorfa_su_una_superficie_di_riemann.org}{funzione olomorfa} non costante.

\begin{prop}
Se \(F\) non è un \href{20260128144717-isomorfismo_tra_superfici_di_riemann.org}{isomorfismo}, allora esistono dei \href{20260129104340-punto_di_ramificazione_per_una_funzione_tra_superfici_di_riemann.org}{punti di ramificazione} (ovvero \(F\) ramifica).
\end{prop}
\begin{proof}
Per la \href{20260129171832-formula_di_hurewicz_superfici_di_riemann.org}{Formula di Hurwitz} si ha che, detto \(Y\coloneqq\C_{\infty}\)
\begin{equation*}
2g(X)-2 = (\deg F)\big(2g(Y)-2\big) + \sum_{p \in X} (\mult{p}F-1).
\end{equation*}
dove \(g\) indica il \href{20260127112828-superficie_di_riemann.org}{genere topologico} e \(\mult{p}F\) indica la \href{20260129104215-forma_normale_locale_per_superfici_di_riemann.org}{molteplicità}.

Vale che:
\begin{itemize}
\item il \href{20260127112828-superficie_di_riemann.org}{genere di \(\C_{\infty}\) è \(0\)}.
\item \href{20260130162420-genere_topologico_di_una_superficie_di_riemann_non_cresce_per_olomorfismi.org}{necessariamente} \(g(X) = 0\),
\item siccome \(F\) non è isomorfismo, \href{20260129171557-caratterizzazione_isomorfismo_tra_superfici_di_riemann_tramite_grado.org}{allora} \(\deg F \ge 2\).
\end{itemize}

Quindi
\begin{equation*}
-2 = -2 (\deg F) + \sum_{p \in X} (\mult{p}F-1) %
\IMPLICA %
2\parentesi{\ge 1}{(\deg F -1)} = \sum_{p \in X} (\mult{p}F-1)
\end{equation*}
Quindi \(\sum_{p \in X} (\mult{p}F-1) > 0\), e pertanto \(F\) ha almeno un punto di ramificazione.
\end{proof}
\end{document}
