% Intended LaTeX compiler: pdflatex
\documentclass[../main]{subfiles}


\begin{document}

\def\U{\mathcal{U}}
\def\L{\mathcal{L}}
\def\X{\mathcal{X}}
\def\Z{\mathcal{Z}}
\def\D{\mathcal{D}}
%
\def\eq{{\rm eq}}
\def\Ueq{\U^\eq}
%
\def\orbita{\mathcal{O}}
\def\Aut{\operatorname{Aut}}
%
\def\tc{\mid}
\def\tp{\operatorname{tp}}
\def\EMtp{\operatorname{EM}\text{-}\operatorname{tp}}
\def\<{\langle}
\def\>{\rangle}
%
\def\restricted#1{\,\mathord{\upharpoonright}{{\scriptstyle #1}}}
\def\equivalentover#1{\mathrel{\equiv_{ #1 }}}

%% NON FORKING
\def\nonforkSymbol{%
\mathbin{\raise1.8ex%
\rlap{\kern0.6ex\rule{0.6ex}{0.1ex}}%
\rlap{\kern1.1ex\rule{0.1ex}{1.9ex}}\raise-0.3ex\hbox{$\smile$}}}
\def\defaultnonforkmodel{M}
\def\nonfork{\nonforkSymbol}
\renewcommand{\nonfork}[1][\defaultnonforkmodel]{%
\mathrel{\nonforkSymbol_{#1}}}
\section{Insieme esternamente definibile in un modello mostro}
\label{sec:org614f8ce}
Si utilizza la \href{20250612143636-notazione_teoria_dei_modelli.org}{Notazione della TEORIA DEI MODELLI}

Sia \(\L\) un \href{20250130162057-linguaggio_del_prim_ordine.org}{linguaggio}, \(T\) una \href{20250130114950-teoria_del_prim_ordine.org}{teoria} \href{20250131123151-teoria_completa.org}{completa} senza \href{20250131122945-modello_di_un_insieme_di_formule.org}{modelli} finiti e \(\U\) un \href{20250617095548-modello_lambda_saturo.org}{modello saturo} di \href{20241213101756-cardinalita.org}{cardinalità} \href{20250211123155-cardinale_limite_forte.org}{inaccessibile} \(\kappa>\card{\L}+ \omega\). \(\U\) è un \href{20250617102733-modello_mostro.org}{modello mostro}.

\begin{definizione}
Un insieme \(\D \subseteq \U^{z}\) si dice \uline{esternamente definibile} da \(\varphi(x;z) \in \L\) se esiste \(p(x) \in S_{\varphi}(\U)\)\footnote{Con \(S_{\varphi}(\U)\) si intende l'\href{20251026161720-phi_formula_su_un_insieme.org}{insieme dei \(\varphi\)-tipi globali}} tale che
\begin{equation*}
\D = \mathscr{D}_{p,\varphi} \coloneqq \set{a \in \U^{z} \mid \varphi(x;a) \in p(x)}.
\end{equation*}
\end{definizione}

\begin{prop}
Sia \(\D \subseteq \U^{z}\). LSASE:
\begin{enumerate}
\item \(\D\) è esternamente definibile da \(\varphi(x;z)\);
\item esiste una \href{20250212102253-sottostruttura_elementare.org}{estensione elementare} \(\nonstandard{\U} \succ \U\), esiste \(\nonstandard{a} \in \nonstandard{\U}\) tale che
\begin{equation*}
\D = \varphi(\nonstandard{a}, \nonstandard{\U}^{z}) \cap \U^{z}
\end{equation*}
\end{enumerate}
\end{prop}
\end{document}
