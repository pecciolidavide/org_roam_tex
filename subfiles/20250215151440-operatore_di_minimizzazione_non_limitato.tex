% Intended LaTeX compiler: pdflatex
\documentclass[../main]{subfiles}


\begin{document}

\section{Operatore di minimizzazione non limitato}
\label{sec:orgff8f735}
Vedi: \href{20250202130045-insieme_dei_numeri_naturali_mk.org}{Insieme dei numeri naturali MK}
\begin{definizione}
L'operatore \(\mu\) di \uline{minimizzazione non limitato} porta una \href{20250202170607-classe_relazione_binaria.org}{funzione}, anche \href{20250213105339-funzione_parziale.org}{parziale}, \(h:\N^{k+1}\to \N\) nella \href{20250213105339-funzione_parziale.org}{funzione parziale} \(f: \N^{k}\to \N\):
\begin{equation*}
f(x_{1},\dots,x_{k}) = \minim{z}{h(x_{1},\dots,x_{k},z) = 0}
\end{equation*}
che ha per \href{20250202173528-dominio_range_e_campo_di_una_classe_relazione.org}{dominio} tutti quei punti \((x_{1},\dots,x_{k}) \in \N^{k}\) per cui esiste \(z \in \N\) tale che
\begin{enumerate}
\item per ogni \(u\le z\), \((x_{1},\dots,x_{k}, u) \in \dom h\);
\item \(h(x_{1},\dots,x_{k},z) = 0\).
\end{enumerate}

Per ogni punto del dominio \((x_{1},\dots,x_{n}) \in \dom f\),
\begin{equation*}
f(x_{1},\dots,x_{n}) = \min\set{z\ |\ h(x_{1},\dots,x_{k},z) = 0}
\end{equation*}
(vedi \href{20250203102516-massimo_e_minimo.org}{Elemento Minimo}).
\end{definizione}
\end{document}
