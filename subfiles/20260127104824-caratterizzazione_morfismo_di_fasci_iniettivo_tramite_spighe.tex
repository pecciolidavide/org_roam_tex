% Intended LaTeX compiler: pdflatex
\documentclass[../main]{subfiles}


\begin{document}

\section{Caratterizzazione morfismo di fasci iniettivo tramite spighe}
\label{sec:orgba334a1}
Sia \(X\) uno \href{20250103145124-topologia.org}{spazio topologico}, e siano \(\mathcal{F},\mathcal{G}\) due \href{20250324174728-fascio.org}{fasci} di \href{20241205141146-gruppo_abeliano.org}{gruppi} su \(X\).

\begin{prop}
Se \(\varphi:\mathcal{F}\to\mathcal{G}\) è un \href{20250325180613-morfismo_di_prefasci.org}{morfismo di fasci}, LSASE:
\begin{enumerate}
\item \(\varphi\) è un \href{20250327115206-morfismo_di_fasci_iniettivo.org}{morfismo di fasci iniettivo};
\item per ogni \(p \in X\): \(\varphi_{p}:\mathcal{F}_{p}\to \mathcal{G}_{p}\)\footnote{\(\varphi_{p}\) è il \href{20250327114817-morfismo_di_fasci_induce_omomorfismo_tra_spighe.org}{morfismo indotto} sulle \href{20250325183434-spiga_di_un_prefascio.org}{spighe} \(\mathcal{F}_{p},\mathcal{G}_{p}\)\label{org685cd6e}} è un \href{20241206115531-morfismo_di_gruppi.org}{morfismo di gruppi} \href{20241219101956-funzione_iniettiva.org}{iniettivo};
\item per ogni \(U \subseteq X\) aperto: \(\varphi_{U} :\mathcal{F}(U) \to \mathcal{G}(U)\) è un \href{20241206115531-morfismo_di_gruppi.org}{morfismo di gruppi} \href{20241219101956-funzione_iniettiva.org}{iniettivo}.
\end{enumerate}
\end{prop}

\begin{proof}
La dimostrazione è ovvia considerati:
\begin{itemize}
\item \href{20260127101434-caratterizzazione_di_successione_esatta_di_fasci_tramite_spighe.org}{Caratterizzazione di successione esatta di fasci tramite spighe}
\item \href{20250327150529-caratterizzazione_di_successione_esatta_di_fasci_tramite_sezioni.org}{Caratterizzazione di successione esatta di fasci tramite sezioni}
\item \href{20250327150404-successione_di_fasci_esatta.org}{Caratterizzazione morfismo di fasci iniettivo tramite successione esatta}
\qedhere
\end{itemize}
\end{proof}
\section{Caratterizzazione morfismo di fasci suriettivo tramite spighe}
\label{sec:orgfd4f350}

Sia \(X\) uno \href{20250103145124-topologia.org}{spazio topologico}, e siano \(\mathcal{F},\mathcal{G}\) due \href{20250324174728-fascio.org}{fasci} di \href{20241205141146-gruppo_abeliano.org}{gruppi} su \(X\).

\begin{prop}
Se \(\varphi:\mathcal{F}\to\mathcal{G}\) è un \href{20250325180613-morfismo_di_prefasci.org}{morfismo di fasci}, LSASE:
\begin{enumerate}
\item \(\varphi\) è un \href{20250327115214-morfismo_di_fasci_suriettivo.org}{morfismo di fasci suriettivo};
\item per ogni \(p \in X\): \(\varphi_{p}:\mathcal{F}_{p}\to \mathcal{G}_{p}\)\textsuperscript{\ref{org685cd6e}} è un \href{20241206115531-morfismo_di_gruppi.org}{morfismo di gruppi} \href{20241213105600-funzione_suriettiva.org}{suriettivo}.
\end{enumerate}
\end{prop}

\begin{proof}
La dimostrazione è ovvia considerati:
\begin{itemize}
\item \href{20260127101434-caratterizzazione_di_successione_esatta_di_fasci_tramite_spighe.org}{Caratterizzazione di successione esatta di fasci tramite spighe}
\item \href{20250327150404-successione_di_fasci_esatta.org}{Caratterizzazione morfismo di fasci suriettivo tramite successione esatta}
\qedhere
\end{itemize}
\end{proof}
\end{document}
