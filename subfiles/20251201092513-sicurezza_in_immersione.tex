% Intended LaTeX compiler: pdflatex
\documentclass[../main]{subfiles}

\usepackage[hyperref]{biblatex}
\date{}
\title{}
\begin{document}

\section{Immersione Subacquea ricreativa}
\label{sec:orga5d0187}
\subsection{Pianificazione dell'immersione}
\label{sec:org5c7746f}
\href{20250515141706-da_finire.org}{DA FINIRE}
\subsection{Briefing}
\label{sec:orga965dbc}
\subsubsection{Briefing}
\label{sec:orgd8f2419}

Descrizione immersione:
\begin{itemize}
\item Descrizione del punto e del percorso d'immersione:
\begin{itemize}
\item Schizzo del fondale
\item Descrizione del percorso in immersione
\item Ambiente sottomarino, flora e fauna
\item Rispetto dell'ambiente marino
\end{itemize}
\item Descrizione puntuale degli esercizi
\item Organizzazione del gruppo
\item Definizione delle coppie e ordine di immersione
\item Profondità massima e tempo massimo (verifica autonomia/pressione bombole)
\end{itemize}

Ripasso delle procedure:
\begin{itemize}
\item Verifica dei segnali
\item Procedura di risalita di emergenza (compagno perso)
\item Comportamento di coppia e controlli di coppia:
\begin{itemize}
\item 5 cinghiaggi: pinne e coltello, zavorra, jacket e cinghia bombola, strumenti, maschera
\item 5 aria: primo erogatore, secondo erogatore, manometro, carico jacket, scarichi jacket
\item Non lasciare mai il proprio compagno
\end{itemize}
\item Mantenere la propria posizione nel gruppo
\item Non superare la profondità della guida
\item Mantenere assetto neutro
\end{itemize}

Aspetti tecnici immersione:
\begin{itemize}
\item Modalità di discesa e risalita (problemi di compensazione e altri inconvenienti)
\item Modalità per tappa di sicurezza e sosta profonda
\item Modalità d'ingresso e uscita dall'acqua e punto di ritrovo
\end{itemize}
\paragraph{Migliorie da implementare}
\label{sec:orgdd6c16e}

\uline{Immersione 30 novembre 2025} (Ce.F.Is):
\begin{itemize}
\item Fare un briefing in ordine cronologico
\item Durante il briefing \uline{dire quello che si farà} davvero, ed in particolare, in immersione, \uline{fare quello che si è detto}.
\end{itemize}
\subsubsection{Debriefing}
\label{sec:orgd2f08d7}
\begin{itemize}
\item Analisi critica dell'andamento/errori di tutti i partecipanti;
\item Descrizione dell'ambiente sottomarino: flora e fauna avvistate.
\end{itemize}
\paragraph{Migliorie da implementare}
\label{sec:org98ab9b8}

\uline{Immersione 30 novembre 2025} (Ce.F.Is):
\begin{itemize}
\item Il debriefing deve essere cronologico.
\item Renderlo coinvolgente.
\item Deve essere approfondito.
\end{itemize}
\subsection{Gestione dell'immersione da istruttore}
\label{sec:org89b8196}
Alcuni suggerimenti da tenere a mente, di validità generale:
\subsubsection{Comfort degli allievi}
\label{sec:org39428b7}

\begin{itemize}
\item In Acqua, fare quanto dichiarato nel briefing.
\end{itemize}
\subsubsection{Sicurezza}
\label{sec:orgbb38940}

\begin{itemize}
\item Accertarsi del \emph{\href{20251201094746-decompressione_gas_in_immersione.org}{No Deco Time}} di ciascun partecipante
\item Fare attenzione alla profondità massima: farci solo una ``toccata e fuga''.
\begin{itemize}
\item Non chiedere aria quando c'è il rischio di sforare la profondità massima.
\end{itemize}
\end{itemize}
\subsubsection{Percorso}
\label{sec:org4f4404c}

\begin{itemize}
\item È importante, però non concentrartici troppo: è meglio ``sbagliare strada'' ma andare pianino e tenere il gruppo compatto piuttosto che correre per cercare la strada e perdersi gli allievi (o far sgranare il gruppo).
\end{itemize}
\subsubsection{Altri consigli}
\label{sec:org1c580d1}

\begin{itemize}
\item Quando ci si ferma per far compattare il gruppo, mostrare qualcosa affinché le persone non si annoino
\end{itemize}
\end{document}
