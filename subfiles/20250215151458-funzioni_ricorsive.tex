% Intended LaTeX compiler: pdflatex
\documentclass[../main]{subfiles}

\usepackage[hyperref]{biblatex}
\date{}
\title{}
\begin{document}

\section{Funzioni ricorsive}
\label{sec:org6862ec8}
Vedi \href{20250202130045-insieme_dei_numeri_naturali_mk.org}{Insieme dei numeri naturali MK}.

Tutte le funzioni sono da considerarsi potenzialmente \href{20250213105339-funzione_parziale.org}{parziali}.
\subsection{Definizione}
\label{sec:orge74bced}
Questa definizione può essere \href{20250216162850-funzioni_ricorsive_in_piu_dimensioni.org}{ampliata} a funzioni in più dimensioni.

L'\href{20250130104331-insieme_mk.org}{insieme} \(\mathcal{R}\) delle \textbf{\href{20250202170607-classe_relazione_binaria.org}{funzioni} ricorsive} è la più piccola \href{20250130104320-classe_mk.org}{classe} contenente:
\begin{itemize}
\item la funzione costante nulla,
\begin{equation*}
  c_{0}:\N\to \N: x\mapsto 0
\end{equation*}
\item la \href{20250202124648-successore_di_un_insieme_mk.org}{funzione successore}:
\begin{equation*}
  S:\N\to \N: x\mapsto x+1
\end{equation*}
\item le funzioni proiezione \(U_{i}^{k}\), per ogni \(k \in \N^{+}\) e per ogni \(0<i \le k\):
\begin{align*}
  U_{i}^{k}: \N^{k} &\longrightarrow \N\\
  (x_{1},\dots,x_{k})&\longmapsto x_{i}
\end{align*}
In particolare, la funzione \(U_{1}^{1}:\N\to \N\) è la funzione identità;
\end{itemize}

e chiusa per:
\subsubsection{Schema di Composizione di funzioni ricorsive}
\label{sec:orgcbf1686}
Siano \(k,\ell \in \N^{+}\). Se \(h: \N^{k}\to \N\) e per ogni \(1\le i\le k\): \(g_{i}: \N^{\ell}\to \N\) sono funzioni ricorsive, allora la funzione \(f:\N^{\ell}\to \N\):
\begin{equation*}
f(x_{1},\dots,x_{\ell}) \coloneqq h\left(g_{1}(x_{1},\dots,x_{\ell}),\dots,g_{k}(x_{1},\dots,x_{\ell})\right)
\end{equation*}
è ricorsiva. In particolare, il \href{20250202173528-dominio_range_e_campo_di_una_classe_relazione.org}{dominio} di \(f\) è composto di tutti e soli gli \((x_{1},\dots,x_{\ell}) \in \N^{\ell}\) tali che:
\begin{enumerate}
\item \(\forall\, i\), \((x_{1},\dots,x_{\ell}) \in \dom g_{i}\);
\item \(\left(g_{1}(x_{1},\dots,x_{\ell}), \dots, g_{k}(x_{1},\dots,x_{\ell})\right) \in \dom h\).
\end{enumerate}
\subsubsection{Schema di Ricorsione di funzioni ricorsive}
\label{sec:org511067c}
Sia \(k \in \N^{+}\). Se \(h:\N^{n +2}\to \N\) e \(g:\N^{ k}\to \N\) sono funzioni ricorsive allora la funzione \(f:\N^{k+1}\to \N\) definita dalle condizioni
\begin{equation*}
\begin{cases}
f(x_{1},\dots,x_{k},0) = g(x_{1},\dots,x_{k})\\
f(x_{1},\dots,x_{n},y+1) = h\left(x_{1},\dots,x_{k}, y, f(x_{1},\dots,x_{n},y)\right)
\end{cases}
\end{equation*}
è ricorsiva.

Notiamo che questa funzione esiste per il \href{20250207121906-teorema_di_ricorsione.org}{Teorema di Ricorsione}.

In particolare, il \href{20250202173528-dominio_range_e_campo_di_una_classe_relazione.org}{dominio} di \(f\) è composto di tutti e soli gli \((x_{1},\dots,x_{k},y) \in \N^{k+1}\) tali che
\begin{enumerate}
\item \((x_{1},\dots,x_{k}, 0) \in \dom g\);
\item \(\forall\, z< y\) si ha che \(\left(x_{1},\dots,x_{k}, z, f(x_{1},\dots,x_{k}, z)\right) \in\dom h\)
\end{enumerate}

Vedi la \href{20250601161456-generalizzazione_schema_di_ricorsione.org}{generalizzazione}.
\subsubsection{Schema di minimizzazione}
\label{sec:org3d54410}
Se \(h:\N^{k+1}\to \N\) è una funzione ricorsiva allora applicando l'\href{20250215151440-operatore_di_minimizzazione_non_limitato.org}{operatore} \(\mu\) si ottiene una funzione ricorsiva \(f:\N^{k}\to \N\):
\begin{equation*}
f(x_{1},\dots,x_{k}) = \minim{z}{h(x_{1},\dots,x_{k}, z) = 0}
\end{equation*}
che ha per \href{20250202173528-dominio_range_e_campo_di_una_classe_relazione.org}{dominio} tutti quei punti \((x_{1},\dots,x_{k}) \in \N^{k}\) per cui esiste \(z \in \N\) tale che
\begin{enumerate}
\item per ogni \(u\le z\), \((x_{1},\dots,x_{k}, u) \in \dom h\);
\item \(h(x_{1},\dots,x_{k},z) = 0\).
\end{enumerate}
\begin{oss}
Notiamo che se esiste un unico \(\tilde{z}\) tale che \(h(x_{1},\dots,x_{k}, \tilde{z}) = 0\), allora
\begin{equation*}
\minim{z}{h(x_{1},\dots,x_{k},z)=0} = \tilde{z}
\end{equation*}
\end{oss}
\begin{oss}
Ogni \href{20250215141024-funzioni_primitive_ricordive.org}{funzione ricorsiva primitiva} è ricorsiva: \(\mathcal{P} \subsetneqq \mathcal{R}\).
\end{oss}
\subsection{Definizione equivalente di funzioni ricorsive}
\label{sec:org2e4078d}
Sia \(\mathcal{F}\) la più piccola famiglia di funzioni \(f:\N^{k}\to \N\) contenente le funzioni \(U_{i}^{k}, +,\cdot,\chi_{\le}\) e chiusa per composizione e applicazioni dell'\href{20250215151440-operatore_di_minimizzazione_non_limitato.org}{operatore di minimizzazione} a funzioni \href{20250213105339-funzione_parziale.org}{totali}.
Allora
\begin{equation*}
\mathcal{F} = \mathcal{R}.
\end{equation*}

La dimostrazione si articola tramite una serie di lemmi.

In questa sezione si identificano i \href{20250131155822-operazioni_insiemistiche_tra_classi_mk.org}{sottinsiemi} di \(\N^{k}\) (vedi \href{20250202130045-insieme_dei_numeri_naturali_mk.org}{Insieme dei numeri naturali MK}) con i \href{20250131103317-formula_del_prim_ordine.org}{predicati} \(k\)-ari (ovvero con \(k\) \href{20250131103429-variabile_libera_di_una_formula.org}{variabili libere}), per mezzo degli \href{20250131122913-soddisfazione_di_una_formula.org}{insiemi di verità}.

Scriveremo indifferentemente \((x_{1},\dots,x_{k}) \in P\) oppure \(P(x_{1},\dots,x_{k})\) per dire che \(\N\vDash P(x_{1},\dots,x_{k})\).
\subsubsection{Lemma 1}
\label{sec:org2a7abe3}
Le funzioni \(\operatorname{sgn}\) e \(\overline{\operatorname{sgn}}\) appartengono ad \(\mathcal{F}\). (Vedi ``\href{20250215141024-funzioni_primitive_ricordive.org}{Esempi di funzioni primitive ricorsive}'' per la definizione di queste funzioni).
\subsubsection{Lemma 2}
\label{sec:orgf90540f}
La collezione dei predicati le cui funzioni caratteristiche sono in \(\mathcal{F}\) (ovvero gli \(\mathcal{F}\)-predicati) è chiusa per intersezioni, unioni e complementi. In particolare \(\le,\ge,=,\neq,<,>\) sono \(\mathcal{F}\)-predicati.
\subsubsection{Lemma 3}
\label{sec:org0b1b7c8}
Le funzioni costanti \(c_{n}\) e la funzione successore \(S\) appartengono ad \(\mathcal{F}\).
\subsubsection{Lemma 4}
\label{sec:org0fc9e86}
Gli \(\mathcal{F}\)-predicati sono chusi per quantificazioni limitate. Se \(P \subseteq \N^{k+1}\) è un \(\mathcal{F}\)-predicato allora lo sono anche \(Q_{1},Q_{1} \subseteq \N^{k+1}\) definiti da:
\begin{align*}
Q_{1}(\bm{x},y)\quad &\iff\quad \exists\,z\le y\ P(\bm{x},y)\\
Q_{2}(\bm{x},y)\quad &\iff\quad \forall\,z\le y\ P(\bm{x},y).
\end{align*}
\subsubsection{Lemma 5}
\label{sec:orgfed9f64}
Le seguenti funzioni appartengono ad \(\mathcal{F}\):
\begin{itemize}
\item \(\bm{J}, (\cdot)_{0}, (\cdot)_{1}\) (vedi ``\href{20250215151413-biiezione_canonica_tra_n_e_n2.org}{Biiezione canonica tra N e prodotti cartesiani di N}'');
\item \(\operatorname{Res}\) (vedi ``\href{20250215171731-quoziente_e_resto_sono_funzioni_ricorsive_primitive.org}{Quoziente, resto, MCD e mcm sono funzioni ricorsive primitive}'' per una definizione);
\item \(\beta,\ell,\godeldec{\cdot,\cdot}\) (vedi ``\href{20250531110737-codifica_delle_sequenze_finite_tramite_beta_di_godel.org}{Codifica delle sequenze finite tramite beta di Godel}'')
\end{itemize}
\subsubsection{Lemma 6}
\label{sec:orga40a2b4}
Le \href{20250215141024-funzioni_primitive_ricordive.org}{funzioni primitive ricorsive} sono in \(\mathcal{F}\): \(\mathcal{P} \subseteq \mathcal{F}\). In particolare, i \href{20250216174510-insieme_ricorsivo_primitivo.org}{predicati primitivi ricorsivi} sono \(\mathcal{F}\)-predicati.
\end{document}
