% Intended LaTeX compiler: pdflatex
\documentclass[../main]{subfiles}


\begin{document}

\begin{lem}
Sia \(g \in L^{1}(I_{n})\)\footnote{Con \(I_{n}\) si indica il cubo
\begin{equation*}
I_{n} = [0,1]\times \dots \times [0,1] = [0,1]^{n}.
\end{equation*}
Si vedano ``\href{20250624162220-spazi_lp.org}{Spazi Lp}''}. Se per ogni \(\bm{w} \in \R^{n}\) e per ogni \(\theta \in \R\), detto \(\mathcal{H}_{\bm{w},\theta} \coloneqq \set{x \in I_{n}\mid \bm{w}\cdot\bm{x}+\theta>0}\) il semispazio, si ha che
\begin{equation*}
\int_{\mathcal{H}_{\bm{w},\theta}} g(\bm{x})\dif\bm{x} = 0
\end{equation*}
allora \(g=0\) \href{20250630122745-proprieta_vera_quasi_ovunque.org}{q.o.}
\label{lem9.3.12}
\end{lem}
\end{document}
