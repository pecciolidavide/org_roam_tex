% Intended LaTeX compiler: pdflatex
\documentclass[../main]{subfiles}


\begin{document}

\def\L{\mathcal{L}} % Il linguaggio
\def\U{\mathcal{U}} % Modello mostro
\section{Formula invariante su un insieme}
\label{sec:org5f4971a}
Si utilizza la \href{20250612143636-notazione_teoria_dei_modelli.org}{Notazione della TEORIA DEI MODELLI}

Sia \(\L\) un \href{20250130162057-linguaggio_del_prim_ordine.org}{linguaggio}, \(T\) una \href{20250130114950-teoria_del_prim_ordine.org}{teoria} \href{20250131123151-teoria_completa.org}{completa} senza \href{20250131122945-modello_di_un_insieme_di_formule.org}{modelli} finiti e \(\U\) un \href{20250617095548-modello_lambda_saturo.org}{modello saturo} di \href{20241213101756-cardinalita.org}{cardinalità} \href{20250211123155-cardinale_limite_forte.org}{inaccessibile} \(\kappa>\card{\L}+ \omega\). \(\U\) è un \href{20250617102733-modello_mostro.org}{modello mostro}.

Sia \(A \subseteq \U\) un insieme piccolo

\begin{definizione}
Una \href{20250131103317-formula_del_prim_ordine.org}{formula} \(\varphi(x) \in \L(\U)\) è \uline{\(A\)-invariante} se \(\varphi(\U^{x})\)\footnote{Questo è l'\href{20250131122913-soddisfazione_di_una_formula.org}{insieme definito da \(\varphi\) in \(\U\)}} è un \href{20251020150308-insieme_invariante_su_un_insieme_in_un_modello_mostro.org}{insieme \(A\)-invariante}.
\end{definizione}
\end{document}
