% Intended LaTeX compiler: pdflatex
\documentclass[../main]{subfiles}


\begin{document}

\section{Cardinalità dei reali}
\label{sec:org69cd6b7}
Contesto: \href{20250130104245-morse_kelly_set_theory.org}{Morse Kelly Set Theory}
\begin{oss}
Si ha che\footnote{Vedi
\begin{itemize}
\item \href{20250130104245-morse_kelly_set_theory.org}{Insieme delle parti per MK}
\item \href{20250619101109-classi_equipotenti.org}{Classi equipotenti MK}
\end{itemize}}
\begin{equation*}
\R\asymp \parti{\N}
\end{equation*}
per il \href{20250205150457-teorema_di_cantor_bernstein_schroder.org}{Teorema di Cantor-Bernstein-Schröder}.

\uline{Assumento \href{20250206171508-axiom_of_choiche.org}{AC}} \href{20250210104427-assiomi_equivalenti_ad_ac.org}{allora} \(\R\) è \href{20250203161431-classe_ben_ordinabile_mk.org}{ben ordinabile} e ha \href{20241213101756-cardinalita.org}{cardinalità} \(\card{\R} = \card{\parti{\N}}\): \(\parti{\N}\equipotenti 2^{\N}\) e quindi\footnote{Vedi
\begin{itemize}
\item \href{20250612151505-aritmetica_dei_cardinali.org}{Cardinal exponentiation}
\item \href{20250202192030-classe_delle_classi_funzioni.org}{Insieme delle funzioni}
\end{itemize}}
\begin{equation*}
\card{\R} = 2^{\aleph_{0}}.
\end{equation*}
\end{oss}
\end{document}
