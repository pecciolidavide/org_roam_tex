% Intended LaTeX compiler: pdflatex
\documentclass[../main]{subfiles}

\usepackage[hyperref]{biblatex}
\date{}
\title{}
\begin{document}

\section{Gruppo dei diffeomorfismi di una varietà differenziabile}
\label{sec:orged6c64e}
Sia \(M\) una \href{20250113115909-struttura_differenziabile.org}{varietà differenziabile}. Si definisce il gruppo dei \href{20250113172924-diffeomorfismo_tra_varieta_differenziabili.org}{diffeomorfismi} di \(M\) come
\begin{equation*}
\operatorname{Diff}(M) \coloneqq \set{ F: M\longrightarrow M: F\text{ diffeomorfismo}}
\end{equation*}

Questo è un \href{20241205141146-gruppo_abeliano.org}{gruppo} con la composizione di funzioni, ed è un invariante, ovvero se per due varietà i gruppi dei diffeomorfismi sono diversi, allora queste non sono \href{20250113172924-diffeomorfismo_tra_varieta_differenziabili.org}{diffeomorfe}.
\end{document}
