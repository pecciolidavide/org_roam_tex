% Intended LaTeX compiler: pdflatex
\documentclass[../main]{subfiles}


\begin{document}

Sia \(R\) un \href{20241219112842-pid.org}{PID} e sia \(X\) uno \href{20250103145124-topologia.org}{spazio topologico}.
\begin{prop}
Sia \(A \subseteq X\) un \href{20250103163814-sottospazio_topologico.org}{sottospazio}, e sia
\begin{equation*}
i_{A}: A\hookrightarrow X
\end{equation*}
l'inclusione. Consideriamo il \href{20241205131958-funtore.org}{funtore} \href{20250123115927-funtore_di_omologia_singolare.org}{di omologia singolare} \((H_{n},\star)\)

\begin{enumerate}
\item Se \(A\) è un \href{20250122155714-retratto_di_uno_spazio_topologico.org}{retratto} di \(X\), allora
\begin{equation*}
 (i_{A})_{\star}: H_{n}(A)\hookrightarrow H_{n}(X)
\end{equation*}
è \href{20241219101956-funzione_iniettiva.org}{iniettiva}
\item Se \(A\) è un \href{20250122155727-retratto_di_deformazione_di_uno_spazio_topologico.org}{retratto di deformazione} di \(X\), allora
 \begin{equation*}
 (i_{A})_{\star}: H_{n}(A)\hookrightarrow H_{n}(X)
\end{equation*}
è un \href{20241206115416-morfismi_r_moduli.org}{isomorfismo}.
\end{enumerate}
\end{prop}
\end{document}
