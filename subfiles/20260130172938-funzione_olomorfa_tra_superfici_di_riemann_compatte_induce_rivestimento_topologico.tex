% Intended LaTeX compiler: pdflatex
\documentclass[../main]{subfiles}


\begin{document}

\section{Funzione olomorfa tra superfici di Riemann compatte induce rivestimento topologico}
\label{sec:org546c457}
Siano \(X,Y\) \href{20260127112828-superficie_di_riemann.org}{superfici di Riemann} \href{20250103163701-spazio_topologico_compatto.org}{compatte}, e sia \(F:X\to Y\) \href{20260128143822-funzione_olomorfa_su_una_superficie_di_riemann.org}{olomorfismo} non costante.

\begin{prop}
Sia \(D \subseteq Y\) l'insieme (finito) dei \href{20260129104340-punto_di_ramificazione_per_una_funzione_tra_superfici_di_riemann.org}{punti di diramazione di \(F\)}. Allora
\begin{equation*}
\restriction{F}{X\setminus F^{-1}(D)} : X\setminus F^{-1}(D) \to Y\setminus S
\end{equation*}
è un \href{20250113175231-rivestimento.org}{rivestimento topologico} di grado \(d = \deg F\)\footnote{\(\deg F\) è il \href{20260129171444-teorema_del_grado_per_olomorfismi_tra_superfici_di_riemannn.org}{grado di \(F\)}.}.
\end{prop}

\begin{proof}
Questo segue dalla \href{20260129171444-teorema_del_grado_per_olomorfismi_tra_superfici_di_riemannn.org}{dimostrazione del Teorema del Grado}.
\end{proof}

\begin{definizione}
Si dice che \(F\) è un \uline{rivestimento ramificato}.
\end{definizione}
\end{document}
