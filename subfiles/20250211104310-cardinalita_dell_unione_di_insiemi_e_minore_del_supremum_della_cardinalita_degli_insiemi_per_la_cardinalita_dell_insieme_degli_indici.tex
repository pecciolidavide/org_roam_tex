% Intended LaTeX compiler: pdflatex
\documentclass[../main]{subfiles}

\usepackage[hyperref]{biblatex}
\date{}
\title{}
\begin{document}

\section{Cardinalità dell'unione di insiemi è minore del supremum della cardinalità degli insiemi per la cardinalità dell'insieme degli indici}
\label{sec:org9747a94}
Contesto: \href{20250130104245-morse_kelly_set_theory.org}{Morse Kelly Set Theory}+\href{20250206171508-axiom_of_choiche.org}{AC}
\subsection{Teorema}
\label{sec:orgf263a2b}
Se \(I\) e \(\set{X_{i}\ |\ i \in I}\) sono insiemi (vedi \href{20250206170922-sequenze_e_stringhe.org}{Sequenza}), allora
\begin{equation*}
\card{\bigcup_{i \in I}X_{i}}\le \card{I}\cdot \sup_{i \in I}\card{X_{i}}
\end{equation*}
(vedi \href{20241213101756-cardinalita.org}{Cardinalità}, \href{20250131155822-operazioni_insiemistiche_tra_classi_mk.org}{Classe Unione Generalizzata}, \href{20250612151505-aritmetica_dei_cardinali.org}{Prodotto di cardinali}, \href{20250203111003-ordinali.org}{Relazione d'ordine sugli ordinali}, \href{20250203102516-massimo_e_minimo.org}{Infimum e supremum} e \href{20250203161341-cardinali.org}{Supremum di un insieme di cardinali è un cardinale})
\subsubsection{Dimostrazione}
\label{sec:orgf50d4a4}

Per ogni \(i \in I\) sia \(f_{i}: X_{i}\to \card{X_{i}}\) una \href{20250104111707-funzione_biunivoca.org}{biiezione}.

Per ogni \(x \in \bigcup_{i \in I} X_{i}\) sia \(i(x) \in I\) tale che \(x \in X_{i(x)}\) (è possibile sceglierlo grazie ad \href{20250206171508-axiom_of_choiche.org}{AC}).

La funzione
\begin{align*}
\bigcup_{i \in I} X_{i} &\longrightarrow \bigcup_{i \in I}(\set{i}\times \card{X_{i}})\\
x &\longmapsto \left(i(x),f_{i(x)}(x)\right)
\end{align*}
è \href{20241219101956-funzione_iniettiva.org}{iniettiva} e \href{20241213101756-cardinalita.org}{pertanto}
\begin{equation*}
\card{\bigcup_{i \in I} X_{i}}\le \sum_{i \in I} \card{X_{i}}
\end{equation*}
(vedi \href{20250612151505-aritmetica_dei_cardinali.org}{Somma di cardinali})

Applicando \href{20250211104227-somma_generalizzata_di_cardinali_e_minore_del_supremum_dei_cardinali.org}{Somma generalizzata di cardinali è minore del supremum dei cardinali} si ottiene
\begin{equation*}
\sum_{i \in I}\card{X_{i}} \le \card{I}\cdot \sup_{i \in I}\card{X_{i}}
\end{equation*}
da cui la tesi.
\end{document}
