% Intended LaTeX compiler: pdflatex
\documentclass[../main]{subfiles}

\usepackage[hyperref]{biblatex}
\date{}
\title{}
\begin{document}

\section{Radicale di un ideale è ideale}
\label{sec:org3abae24}
\subsection{Proposizione}
\label{sec:org0ae064f}
Sia \(S\) un \href{20241205141119-anello.org}{anello commutativo}, \(J \subseteq S\) un \href{20241219112955-ideale.org}{ideale} e \(\sqrt{J}\) il \href{20250110125225-radicale_di_un_ideale.org}{radicale} di \(J\). Allora \(\sqrt{J}\) è un ideale.
\subsubsection{Dimostrazione}
\label{sec:org374be67}

\paragraph{Somma}
\label{sec:org3cc364c}
Siano \(f,g \in \sqrt{J}\). Allora \(f^{r}, g^{s} \in J\) per qualche \(r,s \in \N\).

Allora \((f+g)^{r+s} \in J\), poiché
\begin{equation*}
(f+g)^{r+s} = \sum_{i=0}^{r+s} \binom{r+s}{i} f^{i}g^{r+s-i}
\end{equation*}
per il \href{20250110142129-teorema_del_coefficiente_binomiale.org}{teorema del coefficiente binomiale}: quindi ciascun termine della somma contiene una potenza di \(f\) maggiore o uguale ad \(r\) oppure una potenza di \(g\) maggiore o uguale a \(s\).
Quindi \(f+g \in \sqrt{J}\).
\paragraph{Prodotto}
\label{sec:org13f180e}
Siano \(f,g \in \sqrt{J}\). Allora \(f^{r}, g^{s} \in J\) per qualche \(r,s \in \N\). Senza perdita di generalità si supponga \(r>s\).

Allora \((fg)^{r} \in J\), poiché
\begin{equation*}
(fg)^{r} = f^{r} \cdot g^{s}\cdot g^{r-s}
\end{equation*}
con \(g^{r-s} \in S\).

Quindi \(fg \in \sqrt{J}\).
\end{document}
