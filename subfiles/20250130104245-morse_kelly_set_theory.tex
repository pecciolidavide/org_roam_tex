% Intended LaTeX compiler: pdflatex
\documentclass[../main]{subfiles}


\begin{document}

\section{Morse Kelly Set Theory}
\label{sec:org42248e4}
La teoria MK è una \(\mathcal{L}_{\in}\)-\href{20250130114950-teoria_del_prim_ordine.org}{teoria della logica del prim'ordine}, dove il \href{20250130162057-linguaggio_del_prim_ordine.org}{linguaggio} è
\begin{equation*}
\mathcal{L}_{\in} = \set{\in}
\end{equation*}
con \(\in\) \href{20250130162057-linguaggio_del_prim_ordine.org}{simbolo di relazione} binaria
\subsection{Definizioni di base}
\label{sec:org7b332d2}

\begin{itemize}
\item \href{20250130104320-classe_mk.org}{Classe MK}
\item \href{20250130104331-insieme_mk.org}{Insieme MK}
\item Operazioni insiemistiche:
\begin{itemize}
\item \href{20250131155822-operazioni_insiemistiche_tra_classi_mk.org}{Unione}
\item \href{20250131155822-operazioni_insiemistiche_tra_classi_mk.org}{Intersezione}
\item \href{20250131155822-operazioni_insiemistiche_tra_classi_mk.org}{Sottrazione}
\item \href{20250131155822-operazioni_insiemistiche_tra_classi_mk.org}{Differenza simmetrica}
\end{itemize}
\item \hyperref[sec:org527b953]{Insieme delle parti}
\item \href{20250131161811-insieme_vuoto_mk.org}{Insieme vuoto MK}
\item \href{20250131162451-coppia_ordinata_mk.org}{Coppia ordinata}
\item Generalizzazione delle operazioni insiemistiche
\begin{itemize}
\item \href{20250131155822-operazioni_insiemistiche_tra_classi_mk.org}{Classe Unione Generalizzata MK}
\item \href{20250131155822-operazioni_insiemistiche_tra_classi_mk.org}{Classe Intersezione Generalizzata}
\item \href{20250131183735-prodotto_cartesiano_di_classi_mk.org}{Prodotto cartesiano}
\end{itemize}
\item \href{20250202190147-immagine_punto_a_punto_di_due_classi.org}{Immagine punto a punto di due classi MK}
\item \href{20250202190147-immagine_punto_a_punto_di_due_classi.org}{Preimmagine di una classe MK}
\item \href{20250205170515-restrizione_di_una_classe.org}{Restrizione di una classe MK}
\item \href{20250202192030-classe_delle_classi_funzioni.org}{Classe delle Classi-Funzioni}
\item \href{20241219101956-funzione_iniettiva.org}{Classe si inietta}
\item \href{20241213105600-funzione_suriettiva.org}{Classe si surietta}
\item \href{20250104111707-funzione_biunivoca.org}{Classe-Funzione Biiettiva}
\item {[}BROKEN LINK: a9dda44b-5a98-4c24-9382-15083956fa9c]
\item \href{20250619101109-classi_equipotenti.org}{Classi equipotenti MK}
\end{itemize}
\subsection{Assiomi di MK}
\label{sec:orgc2bb5b8}
(vedi \href{20250131123109-insieme_di_assiomi_per_una_teoria.org}{Insieme di assiomi per una teoria} e \href{20250131103446-enunciato_del_prim_ordine.org}{Enunciato del prim'ordine})
\subsubsection{Axiom of Extensionality}
\label{sec:org4826263}
\begin{equation*}
\forall\, A\ \forall\, B\ \left(\left( \forall\, x\ (x \in A \iff x \in B)\right) \implies A = B\right)
\end{equation*}
\subsubsection{Axiom of Comprehension}
\label{sec:orgd75b255}
Sia \(\varphi(x,y_{1},\dots,y_{n})\) una \href{20250131103317-formula_del_prim_ordine.org}{formula} in cui la variabile \(x\) occorra \href{20250131103429-variabile_libera_di_una_formula.org}{libera}, e sia \(A\) una variabile diversa da \(x,y_{1},\dots,y_{n}\).
\begin{equation*}
\forall\,y_{1}\cdots \forall\,y_{n}\ \exists\, A\ \forall\,x \left(x \in A\,\iff\, \left(\operatorname{Set}(x) \,\land\, \varphi(x,y_{1},\dots,y_{n})\right)\right)
\end{equation*}
\paragraph{Osservazioni}
\label{sec:orgec7562a}
La classe \(A\) è unica per l'assioma di Extensionality, e si denota
\begin{equation*}
A = \set{x\,|\, \varphi(x,y_{1},\dots,y_{n})}
\end{equation*}
Questo assioma ci dice che la scrittura \(\set{x : \varphi(x)}\) è una \href{20250130104320-classe_mk.org}{classe} (definibile anche con dei parametri), ma che i suoi elementi devono essere necessariamente \href{20250130104331-insieme_mk.org}{insiemi}.
Notiamo inoltre che questo assioma ci consente di scrivere \(\set{x_{1},\dots,x_{n}}\) per \(x_{1},\dots,x_{n}\) \textbf{\textbf{insiemi}}, in quanto questo è
\begin{equation*}
\set{x_{1},\dots,x_{n}}\coloneqq \set{x\,|\, x=x_{1} \,\lor\, x=x_{2} \,\lor\,\dots \,\lor\, x=x_{n}}
\end{equation*}
\subsubsection{Axiom of Set-existence}
\label{sec:orgc710714}
\begin{equation*}
\exists\, x\ \operatorname{Set}(x)
\end{equation*}
(vedi \href{20250130104331-insieme_mk.org}{Insieme MK})
\subsubsection{Axiom of Power-Set}
\label{sec:org527b953}
\begin{equation*}
\forall\,A\ \left(\operatorname{Set}(A)\,\implies\,\exists\, P\ \left(\operatorname{Set}(P) \,\land\, \left(\forall\, B\ (B \subseteq A\ \iff\ B \in P)\right)\right)\right)
\end{equation*}
(vedi \href{20250130104331-insieme_mk.org}{Insieme MK} e \href{20250131155822-operazioni_insiemistiche_tra_classi_mk.org}{Sottoclasse MK})

\(P\) è l'insieme delle parti di \(A\), e si indica con \(\parti{A}\).
\subsubsection{Axiom of Pairing}
\label{sec:org96cf968}
\begin{equation*}
\forall\, x\ \forall\, y\ \left(\operatorname{Set}(x) \,\land\, \operatorname{Set}(y)\,\implies\, \operatorname{Set}\left(\set{x,y}\right)\right)
\end{equation*}
(vedi \href{20250130104331-insieme_mk.org}{Insieme MK})
\subsubsection{Axiom of Foundation}
\label{sec:orgfb2b182}
\begin{equation*}
\forall\, A\ \left(A\neq \emptyset\,\implies\, \left(\exists B (B \in A \,\land\, A\cap B=\emptyset)\right)\right)
\end{equation*}

Questo assioma ci garantisce che, per ogni classe, ci siano degli elementi al suo interno che non contengano elementi della classe più grande.
\subsubsection{Axiom of Union}
\label{sec:org154c2cc}
\begin{equation*}
\forall\, A\ \left(\operatorname{Set}(A)\,\implies\, \operatorname{Set}\left(\bigcup A\right)\right)
\end{equation*}
(vedi \href{20250130104331-insieme_mk.org}{Insieme MK} e \href{20250131155822-operazioni_insiemistiche_tra_classi_mk.org}{Classe Unione}).
\subsubsection{Axiom of Infinity}
\label{sec:orga7a3c0a}
Esiste un \href{20250130104331-insieme_mk.org}{insieme} \href{20250202125627-insieme_induttivo_mk.org}{induttivo}, ovvero:
\begin{equation*}
\exists\, x\ \left(\operatorname{Set}(x) \,\land\, \emptyset \in x \,\land\,  \forall\, y\ \left(y \in x \,\implies\, \operatorname{S}(y) \in x\right)\right)
\end{equation*}
(vedi \href{20250130104331-insieme_mk.org}{Insieme MK} e \href{20250202124648-successore_di_un_insieme_mk.org}{Successore di un insieme MK}).
\subsubsection{Axiom of Replacement}
\label{sec:org8e4e88f}
Se \(F\) è una \href{20250202170607-classe_relazione_binaria.org}{classe-funzione} e \(A\) è un \href{20250130104331-insieme_mk.org}{insieme}, allora \(F[A]\) è un \href{20250130104331-insieme_mk.org}{insieme}. (vedi \href{20250202190147-immagine_punto_a_punto_di_due_classi.org}{Immagine punto a punto di due classi MK})
\subsection{Risultati di base}
\label{sec:org97b9a44}
\href{20250131160822-ogni_sottoclasse_di_un_insieme_e_un_insieme_mk.org}{Ogni sottoclasse di un insieme è un insieme MK}
\href{20250131160843-ogni_sottoclasse_di_una_classe_propria_e_una_classe_propria_mk.org}{Ogni sovraclasse di una classe propria è una classe propria MK}
\href{20250131180704-nessun_insieme_appartiene_a_se_stesso.org}{Nessuna classe appartiene a se stessa}
\href{20250202124855-esistono_infiniti_insiemi_mk.org}{Esistono infiniti insiemi MK}
\href{20250202180416-unione_di_relazioni_funzionali_mk.org}{Unione di funzioni MK}
\href{20250204124929-non_esistono_classi_funzioni_a_dominio_naturale_che_formano_una_catena_discendente.org}{Non esistono classi-funzioni a dominio naturale che formano una catena discendente}
\end{document}
