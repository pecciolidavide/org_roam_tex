% Intended LaTeX compiler: pdflatex
\documentclass[../main]{subfiles}

\usepackage[hyperref]{biblatex}
\date{}
\title{}
\begin{document}

\section{Complesso Simpliciale generato da un simplesso}
\label{sec:org8f5b1d4}
Siano \(p_{0},\dots,p_{q} \in \R^{n+1}\), e sia \(\sigma=[p_{0},\dots,p_{q}] \subseteq \R^{n+1}\) un \(q\)-\href{20250121121923-simplessi.org}{simplesso}.

\begin{definizione}
Con \(K(\sigma)\) si indica il più piccolo \href{20250121124147-complesso_simpliciale.org}{complesso simpliciale} tale che \(\sigma \in K(\sigma)\).

Con \(K(\sigma)^*\) si indica
\begin{equation*}
K(\sigma)^{*} \coloneqq K(\sigma)\setminus\set{\sigma}
\end{equation*}
\end{definizione}
\end{document}
