% Intended LaTeX compiler: pdflatex
\documentclass[../main]{subfiles}


\begin{document}

Sia \(\mathcal{L}\) un \href{20250130162057-linguaggio_del_prim_ordine.org}{linguaggio del prim'ordine}, e sia \(\kappa\) un \href{20250203161341-cardinali.org}{cardinale} tale che \(\card{\mathcal{L}} + \omega < \kappa\). Sia \(\mathcal{U}\) una \(\mathcal{L}\)-\href{20250131103035-struttura_del_prim_ordine.org}{struttura} di \href{20241213101756-cardinalita.org}{cardinalità} \(\kappa\).

\begin{prop}
Sono fatti equivalenti:
\begin{enumerate}
\item \(\mathcal{U}\) è saturo;
\item \(\mathcal{U}\) è omogeneo e debolmente saturo.
\end{enumerate}
\end{prop}

\begin{oss}
Questo teorema è l'equivalente de ``\href{20250213152410-modello_e_ricco_sse_omogeneo_e_universale.org}{Modello è ricco sse omogeneo e universale}'', in quanto essere debolmente saturo significa essere universale nella categoria di modelli e morfismi ricchi.
\end{oss}
\end{document}
