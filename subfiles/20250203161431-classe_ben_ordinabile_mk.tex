% Intended LaTeX compiler: pdflatex
\documentclass[../main]{subfiles}


\begin{document}

Contesto: \href{20250130104245-morse_kelly_set_theory.org}{Morse Kelly Set Theory}
\section{Definizione}
\label{sec:org864a5ac}
Una classe \(X\) si dice \textbf{ben ordinabile} se esiste un \href{20250203104134-buon_ordine_mk.org}{buon ordine} su \(X\) o, \href{20250203133527-insiemi_ben_ordinati_sono_isomorfi_ad_un_ordinale_unico.org}{equivalentemente}, se \(X\) è in \href{20250104111707-funzione_biunivoca.org}{biiezione} con qualche \(\Omega \le \operatorname{Ord}\) (vedi \href{20250203111003-ordinali.org}{Ordinali})
\subsection{Osservazione}
\label{sec:org74f6a30}
Se \(X\) è ben ordinabile e \(Y\) è in \href{20250104111707-funzione_biunivoca.org}{biiezione} con \(X\) (o semplicemente se è \href{20250202190147-immagine_punto_a_punto_di_due_classi.org}{immagine} \href{20241213105600-funzione_suriettiva.org}{suriettiva} di \(X\)), allora \(Y\) è ben ordinabile.

Viceversa, se \(Y\) è ben ordinabile e \(X\preceq Y\) (vedi \href{20241219101956-funzione_iniettiva.org}{Classe si inietta MK}), allora \(X\) è ben ordinabile.
\end{document}
