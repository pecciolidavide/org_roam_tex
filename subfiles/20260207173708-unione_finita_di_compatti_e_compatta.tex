% Intended LaTeX compiler: pdflatex
\documentclass[../main]{subfiles}


\begin{document}

\section{Unione finita di compatti è compatta}
\label{sec:org8b1e827}
Sia \(X\) uno \href{20250103145124-topologia.org}{spazio topologico}.

\begin{thm}
Siano \(K_{1}, \dots, K_{n} \subseteq X\) un numero finito di \href{20250103163814-sottospazio_topologico.org}{sottinsiemi} \href{20250103163701-spazio_topologico_compatto.org}{compatti}.
Allora la loro \href{20250131155822-operazioni_insiemistiche_tra_classi_mk.org}{unione} è compatta:
\begin{equation*}
K = \bigcup_{i=1}^{n} K_{i} \subseteq X \quad \text{è compatto}.
\end{equation*}
\end{thm}
\subsection{Compattezza dell'unione disgiunta}
\label{sec:org3c541d1}

Sia \(\{X_{i}\}_{i\in I}\) una famiglia di \href{20250103145124-topologia.org}{spazi topologici} e sia
\begin{equation*}
X = \coprod_{i \in I} X_{i}
\end{equation*}
la loro \href{20250113175700-unione_disgiunta.org}{unione disgiunta}.

\begin{thm}
Se ogni spazio \(X_{i}\) è \href{20250103163701-spazio_topologico_compatto.org}{compatto} e l'insieme \(I\) degli indici è finito, allora \(X\) è compatto.
\end{thm}
\end{document}
