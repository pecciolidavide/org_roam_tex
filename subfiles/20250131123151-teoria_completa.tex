% Intended LaTeX compiler: pdflatex
\documentclass[../main]{subfiles}

\usepackage[hyperref]{biblatex}
\date{}
\title{}
\begin{document}

\section{Teoria completa}
\label{sec:org5c21170}
\begin{definizione}
Una \href{20250130114950-teoria_del_prim_ordine.org}{teoria del prim'ordine} \(T\) si dice \uline{completa} se per ogni \href{20250131103446-enunciato_del_prim_ordine.org}{enunciato} \(\sigma\) nel suo \href{20250130162057-linguaggio_del_prim_ordine.org}{linguaggio} si ha che \(T\) \href{20250131123011-conseguenza_logica.org}{dimostri}
\begin{equation*}
T\vdash \sigma\quad\text{oppure}\quad T\vdash\lnot\sigma.
\end{equation*}

Se ciò non accade, \(T\) si dice \uline{incompleta}. Gli enunciati \(\sigma\) tali che \(T\not\vdash \sigma\) e \(T\not\vdash\lnot\sigma\) si dicono \uline{indipendenti} da \(T\).
\end{definizione}
\end{document}
