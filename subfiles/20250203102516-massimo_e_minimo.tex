% Intended LaTeX compiler: pdflatex
\documentclass[../main]{subfiles}


\begin{document}

\section{Massimo e minimo}
\label{sec:org2420e39}
\textbf{\textbf{\uline{NOTA}: quando si parla di \uline{classi}, se ne parla nell'ambito della \href{20250130104245-morse_kelly_set_theory.org}{Morse Kelly Set Theory}; quando si parla di insiemi, il discorso ha validità più generale.}}

Sia \(X\) un \href{20250130104331-insieme_mk.org}{insieme} (o una \href{20250130104320-classe_mk.org}{classe}) e sia \(R\) una \href{20250202170607-classe_relazione_binaria.org}{relazione binaria} su \(X\); sia \(Y \subseteq X\) un \href{20250131155822-operazioni_insiemistiche_tra_classi_mk.org}{sottoinsieme} (o \href{20250131155822-operazioni_insiemistiche_tra_classi_mk.org}{sottoclasse}) non vuoto, e sia \(a \in P\).
\subsection{Elemento Massimo}
\label{sec:orgb65afd4}
\(a\) si dice \uline{massimo} di \(Y\) se
\begin{equation*}
a \in Y\quad \land \quad \forall\, x \in Y\ (x\neq a\implies x\mathrel{R}a);
\end{equation*}
\subsection{Elemento Massimale}
\label{sec:org6b9fe84}
\(a\) si dice \uline{elemento \(R\)-massimale} di \(Y\) se
\begin{equation*}
a \in Y\quad \land \quad \forall\, x \in Y\ (x\neq a\implies a\not\mathrel{R}x);
\end{equation*}
\subsection{Elemento Minimo}
\label{sec:org8c0accd}
\(a\) si dice \uline{minimo} di \(Y\) se
\begin{equation*}
a \in Y\quad \land \quad \forall\, x \in Y\ (x\neq a\implies a\mathrel{R}x);
\end{equation*}
\subsection{Elemento Minimale}
\label{sec:orga073a3f}
\(a\) si dice \uline{elemento \(R\)-minimale} di \(Y\) se
\begin{equation*}
a \in Y\quad \land \quad \forall\, x \in Y\ (x\neq a\implies x\not\mathrel{R}a);
\end{equation*}
\subsection{Elemento Maggiorante}
\label{sec:orged5a409}
\(a\) si dice \uline{maggiorante} di \(Y\) se
\begin{equation*}
\forall\, x \in Y\ (x\neq a\implies x\mathrel{R}a);
\end{equation*}
\subsection{Elemento Minorante}
\label{sec:org281731c}
\(a\) si dice \uline{minorante} di \(Y\) se
\begin{equation*}
\forall\, x \in Y\ (x\neq a\implies a\mathrel{R}x);
\end{equation*}
\subsection{Supremum}
\label{sec:org8be996d}
\(a\) è il \uline{supremum} (o \uline{estremo superiore}) di \(Y\) se è il minimo dei maggioranti di \(Y\).
\subsection{Infimum}
\label{sec:orgcd90d0b}
\(a\) è l'\uline{infimum} (o \uline{estremo inferiore}) di \(Y\) se è il massimo dei minoranti.
\end{document}
