% Intended LaTeX compiler: pdflatex
\documentclass[../main]{subfiles}


\begin{document}

\section{Topologia Algebrica (6 cfu)}
\label{sec:org2095457}

\subsection{{\bfseries\sffamily DONE} Lezione 1}
\label{sec:org1d03b88}

\begin{itemize}
\item \href{20241126100904-categoria.org}{Categoria} e \href{20241205113007-categorie_esempi.org}{Categorie-Esempi}
\item \href{20241128141258-inverso_categoriale.org}{Inverso-Categoriale}
\item \href{20241128162125-isomorfismo.org}{Isomorfismo}
\item \href{20241128162518-unicita_inverso.org}{Unicità-inverso}
\item \href{20241204222455-funtore_covariante.org}{Funtore-Covariante}
\item \href{20241204223502-funtore_controvariante.org}{Funtore-Controvariante}
\item \href{20241205131908-funtori_e_isomorfismi.org}{Funtori-e-isomorfismi}
\item \href{20241205132705-trasformazioni_naturali.org}{Trasformazioni-Naturali}
\end{itemize}
\subsection{{\bfseries\sffamily DONE} Lezione 2}
\label{sec:orgbe79cc4}

\begin{itemize}
\item \href{20241205141053-r_moduli.org}{R-Moduli}
\item \href{20241206115416-morfismi_r_moduli.org}{Morfismi-R-Moduli}
\item \href{20250120155457-morfismo_iniettivo_di_r_moduli_induce_isomorfismo.org}{Primo Teorema di Isomorfismo (Moduli)}
\item \href{20241206115740-categoria_degli_r_moduli.org}{Categoria-degli-R-Moduli}
\item \href{20241206142802-sottomoduli.org}{Sottomoduli}
\item \href{20241212141101-generatori_modulo.org}{Generatori-Modulo}
\item \href{20241212142019-insiemi_linearmente_indipendenti.org}{Insiemi-linearmente-indipendenti}
\item \href{20241213094625-modulo_libero.org}{Modulo-Libero}
\item \href{20241213100845-modulo_finitamente_generato.org}{Modulo-Finitamente-Generato}
\item \href{20250120121333-quoziente_di_modulo_fg_e_fg.org}{Quoziente di modulo FG è FG}
\item \href{20241213101621-teorema_della_base.org}{Teorema-della-Base}
\item \href{20241213102346-moduli_quoziente_mod_liberi.org}{Moduli-Quoziente-Mod-Liberi}
\item \href{20241213110005-caratterizzazione_moduli_fg.org}{Caratterizzazione-Moduli-FG}
\item \href{20241217101345-scomposizione_modulo_quozienti_liberi.org}{Scomposizione-Modulo-Quozienti-Liberi}
\item \href{20241219112842-pid.org}{PID}
\item \href{20241219192830-pid_sottomoduli_sono_liberi.org}{PID-Sottomoduli-sono-liberi}
\item \href{20241220000402-torsione_moduli.org}{Torsione-Moduli}
\end{itemize}
\subsection{{\bfseries\sffamily DONE} Lezione 3}
\label{sec:orgad0fa9e}

\begin{itemize}
\item \href{20250120102645-torsione_di_moduli_su_un_dominio_di_integrita.org}{Torsione di moduli su un dominio di integrità}
\item \href{20250120103005-modulo_libero_da_torsione.org}{Modulo libero da torsione}
\item \href{20250120103129-modulo_di_torsione.org}{Modulo di torsione}
\item \href{20250120103205-modulo_di_torsione_finitamente_generato_e_libero.org}{Modulo libero da torsione finitamente generato è libero}
\item \href{20260109173733-rango_di_un_modulo.org}{Rango di un modulo FG}
\item \href{20250120122601-teorema_fondamentale_dei_moduli_finitamente_generati_su_un_pid.org}{Teorema fondamentale dei moduli finitamente generati su un PID}
\item \href{20250120125644-successione_di_r_moduli.org}{Successione di R-Moduli}
\item \href{20250120125004-successione_di_r_moduli_esatta.org}{Successione di R-Moduli ESATTA}
\item \href{20250120130155-caratterizzazione_di_alcune_successioni_esatte_di_r_moduli.org}{Caratterizzazione di alcune successioni esatte di R-Moduli}
\item \href{20250120131527-sec.org}{SEC}
\item \href{20250120131729-teorema_di_spezzamento_sec.org}{Teorema di Spezzamento SEC}
\item \href{20250120131729-teorema_di_spezzamento_sec.org}{SEC con modulo finale libero}
\item \href{20260108192915-funtore_da_rmod_a_rmod_torsione.org}{Funtore da RMod a Rmod - Torsione}
\end{itemize}
\subsection{{\bfseries\sffamily DONE} Lezione 4 [100\%]}
\label{sec:org26b0aa4}

\begin{itemize}
\item[{$\boxtimes$}] \href{20250120155801-lemma_del_cinque.org}{Lemma del cinque}
\item[{$\boxtimes$}] \href{20250120163114-complesso_di_catene.org}{Complesso di catene}
\item[{$\boxtimes$}] \href{20250120163759-categoria_complessi_di_catene.org}{Categoria Complessi di Catene}
\item[{$\boxtimes$}] \href{20250120164857-modulo_di_omologia_dei_complessi_di_catene.org}{Modulo di omologia dei complessi di catene}
\item[{$\boxtimes$}] \href{20250120164930-morfismo_tra_complessi_di_catene_induce_morfismo_tra_moduli_di_omologia.org}{Morfismo tra complessi di catene induce morfismo tra moduli di omologia}
\item[{$\boxtimes$}] \href{20250120165029-funtore_tra_chr_e_rmod.org}{Funtore tra ChR e RMod}
\item[{$\boxtimes$}] \href{20250120164938-zig_zag_lemma.org}{Zig-Zag Lemma} (studiare dim)
\end{itemize}
\subsection{{\bfseries\sffamily DONE} Lezione 5 [100\%]}
\label{sec:orgb4b9f2e}

\begin{itemize}
\item[{$\boxtimes$}] \href{20250121094654-omotopia_tra_funzioni_continue.org}{Omotopia tra funzioni continue}
\item[{$\boxtimes$}] \href{20250121094935-omotopia_tra_morfismi_di_complessi_di_catene.org}{Omotopia tra morfismi di complessi di catene}
\item[{$\boxtimes$}] \href{20250121100726-funtore_di_omologia_di_funzioni_omotope.org}{Funtore di omologia di funzioni omotope}
\item[{$\boxtimes$}] \href{20250121102813-categoria_omotopica_dei_complessi_di_catene.org}{Categoria omotopica dei complessi di catene}
\item[{$\boxtimes$}] \href{20250121103544-complessi_di_catene_omotopicamente_equivalenti.org}{Complessi di catene omotopicamente equivalenti}
\item[{$\boxtimes$}] \href{20250121104306-complesso_di_catene_aciclico.org}{Complesso di catene aciclico}
\item[{$\boxtimes$}] \href{20250121104453-complesso_di_catene_contraibile.org}{Complesso di catene contraibile}
\item[{$\boxtimes$}] \href{20250121105137-complesso_di_catene_contraibile_e_aciclico.org}{Complesso di catene contraibile è aciclico} (studiare dim)
\item[{$\boxtimes$}] \href{20250121110644-complesso_di_catene_aciclico_libero_e_contraibile.org}{Complesso di catene aciclico libero è contraibile} (studiare dim)
\item[{$\boxtimes$}] \href{20250121121613-indipendenza_affine.org}{Indipendenza Affine}
\item[{$\boxtimes$}] \href{20250121121923-simplessi.org}{Simplessi}
\item[{$\boxtimes$}] \href{20250121122324-simplesso_standard.org}{Simplesso Standard}
\item[{$\boxtimes$}] \href{20250121122621-faccia_di_un_simplesso.org}{Faccia di un simplesso}
\item[{$\boxtimes$}] \href{20250121124147-complesso_simpliciale.org}{Complesso Simpliciale}
\item[{$\boxtimes$}] \href{20250121124213-sottocomplesso_simpliciale.org}{Sottocomplesso Simpliciale}
\item[{$\boxtimes$}] \href{20250121124230-supporto_di_un_complesso_simpliciale.org}{Supporto di un complesso simpliciale}
\item[{$\boxtimes$}] \href{20250121125446-complesso_simpliciale_generato_da_un_simplesso.org}{Complesso Simpliciale generato da un simplesso}
\item[{$\boxtimes$}] \href{20250121124249-topologia_debole_di_un_complesso_simpliciale.org}{Topologia debole di un complesso simpliciale}
\item[{$\boxtimes$}] \href{20250121124310-scheletro_di_un_complesso_simpliciale.org}{Scheletro di un complesso simpliciale}
\item[{$\boxtimes$}] \href{20250121131005-complesso_simpliciale_totalmente_ordinato.org}{Complesso Simpliciale totalmente ordinato}
\end{itemize}
\subsection{{\bfseries\sffamily DONE} Lezione 6 [100\%]}
\label{sec:orgc840cd9}

\begin{itemize}
\item[{$\boxtimes$}] \href{20250121124410-modulo_delle_catene_simpliciali.org}{Modulo delle catene simpliciali}
\item[{$\boxtimes$}] \href{20250121132856-mappe_di_bordo_tra_moduli_di_catene_simpliciali.org}{Mappe di bordo tra moduli di catene simpliciali}
\item[{$\boxtimes$}] \href{20250121160600-complesso_di_catene_simpliciali.org}{Complesso di catene simpliciali}
\item[{$\boxtimes$}] \href{20250121160827-omologia_simpliciale.org}{Omologia Simpliciale}
\item[{$\boxtimes$}] \href{20250122100541-notazione_per_i_simplessi.org}{Notazione per i simplessi}
\item[{$\boxtimes$}] \href{20250121162333-calcolo_dell_omologia_simpliciale_per_il_complesso_di_catene_generato_da_un_2_simplesso.org}{Calcolo dell'omologia simpliciale per il complesso di catene generato da un 2-simplesso}
\item[{$\boxtimes$}] \href{20250122100507-omologia_simpliciale_per_il_complesso_di_catene_simpliciali_di_un_complesso_simpliciale_generato_da_un_generico_simplesso.org}{Omologia Simpliciale per il complesso di catene simpliciali di un complesso simpliciale generato da un generico simplesso}
\item[{$\boxtimes$}] \href{20250122101105-complesso_augmentato_di_un_complesso_simpliciale.org}{Complesso Augmentato di un complesso simpliciale}
\item[{$\boxtimes$}] \href{20250122103014-omologia_ridotta_di_un_complesso_simpliciale.org}{Omologia Ridotta di un complesso simpliciale}
\item[{$\boxtimes$}] \href{20250122111953-complesso_di_catene_relative.org}{Complesso di catene relative}
\item[{$\boxtimes$}] \href{20250122112115-omologia_simpliciale_relativa.org}{Omologia Simpliciale Relativa}
\item[{$\boxtimes$}] \href{20250122112244-successione_esatta_di_una_coppia_di_complesso_e_sottocomplesso_in_omologia.org}{Successione esatta di una coppia di complesso e sottocomplesso in omologia}
\end{itemize}
\subsection{{\bfseries\sffamily DONE} Lezione 7 [100\%]}
\label{sec:org4476207}

\begin{itemize}
\item[{$\boxtimes$}] \href{20250122133125-mappa_simpliciale.org}{Mappa simpliciale}
\item[{$\boxtimes$}] \href{20250122133147-categoria_di_complessi_e_mappe_simpliciali.org}{Categoria di complessi e mappe simpliciali}
\item[{$\boxtimes$}] \href{20250122133308-funtore_diesis_da_complessi_simpliciali_a_complessi_di_catene.org}{Funtore diesis da complessi simpliciali a complessi di catene}
\item[{$\boxtimes$}] \href{20250122175051-funtore_hn_da_categoria_pl_a_categoria_r_mod.org}{Funtore di omologia da Categoria-Pl a categoria R-Mod}
\item[{$\boxtimes$}] \href{20250122133348-funtore_da_complessi_simpliciali_a_spazi_topologici.org}{Funtore da complessi simpliciali a spazi topologici}
\item[{$\boxtimes$}] \href{20250122133435-simplesso_singolare.org}{Simplesso singolare}
\item[{$\boxtimes$}] \href{20250122133535-operatori_di_facciata_del_simplesso_standard.org}{Operatori di facciata del simplesso standard}
\item[{$\boxtimes$}] \href{20250122133614-mappa_di_bordo_tra_moduli_di_catene_singolari.org}{Mappa di bordo tra moduli di catene singolari}
\item[{$\boxtimes$}] \href{20250122133631-omologia_singolare.org}{Omologia Singolare}
\end{itemize}
\subsection{{\bfseries\sffamily DONE} Lezione 8 [100\%]}
\label{sec:org74bc8b3}

\begin{itemize}
\item[{$\boxtimes$}] \href{20250122154136-funtorialita_dell_omologia_singolare.org}{Funtore da Top a ChR}
\item[{$\boxtimes$}] \href{20250123115927-funtore_di_omologia_singolare.org}{Funtore di omologia singolare}
\item[{$\boxtimes$}] \href{20250123124404-riassunto_funtori_di_omologia.org}{Riassunto funtori di omologia}
\item[{$\boxtimes$}] \href{20250122154153-calcolo_dell_omologia_del_punto.org}{Calcolo dell'omologia del punto}
\item[{$\boxtimes$}] \href{20250122154543-significato_geometrico_del_modulo_di_omologia_singolare_0.org}{Significato geometrico del modulo di omologia singolare 0}
\item[{$\boxtimes$}] \href{20250122154349-complesso_di_catene_singolare_augmentato.org}{Complesso di catene singolare Augmentato}
\item[{$\boxtimes$}] \href{20250122154406-omologia_singolare_ridotta.org}{Omologia Singolare Ridotta}
\item[{$\boxtimes$}] \href{20250122154451-spazio_topologico_aciclico.org}{Spazio topologico aciclico}
\item[{$\boxtimes$}] \href{20250122154613-insieme_stellato.org}{Insieme stellato}
\item[{$\boxtimes$}] \href{20250122154711-estensione_di_un_simplesso_singolare_in_uno_spazio_stellato.org}{Estensione di un simplesso singolare in uno spazio stellato}
\item[{$\boxtimes$}] \href{20250122154637-insieme_stellato_e_spazio_topologico_aciclico.org}{Insieme stellato è spazio topologico aciclico}
\item[{$\boxtimes$}] \href{20250122154728-coppia_topologica.org}{Coppia topologica}
\item[{$\boxtimes$}] \href{20250122154809-complesso_di_catene_singolari_relative.org}{Complesso di catene singolari relative}
\item[{$\boxtimes$}] \href{20250122154903-omologia_singolare_relativa.org}{Omologia Singolare Relativa}
\item[{$\boxtimes$}] \href{20250122154927-successione_esatta_di_una_coppia_topologica.org}{Successione esatta di una coppia topologica}
\end{itemize}
\subsection{{\bfseries\sffamily DONE} Lezione 9 [100\%]}
\label{sec:org8b51775}

\begin{itemize}
\item[{$\boxtimes$}] \href{20250122155528-teorema_dell_invarianza_per_omotopia.org}{Teorema dell'invarianza per omotopia}
\item[{$\boxtimes$}] \href{20250122155528-teorema_dell_invarianza_per_omotopia.org}{Spazi topologici omotopicamente equivalenti hanno moduli di omologia singolare isomorfi}
\item[{$\boxtimes$}] \href{20250122155640-spazio_topologico_contraibile.org}{Spazio topologico contraibile}
\item[{$\boxtimes$}] \href{20250122155700-spazio_topologico_contraibile_e_aciclico.org}{Spazio topologico contraibile è aciclico}
\item[{$\boxtimes$}] \href{20250122155714-retratto_di_uno_spazio_topologico.org}{Retratto di uno spazio topologico}
\item[{$\boxtimes$}] \href{20250122155727-retratto_di_deformazione_di_uno_spazio_topologico.org}{Retratto di deformazione di uno spazio topologico}
\item[{$\boxtimes$}] \href{20250129112659-retratto_di_deformazione_forte_di_uno_spazio_topologico.org}{Retratto di deformazione forte di uno spazio topologico}
\item[{$\boxtimes$}] \href{20250122160057-inclusione_di_un_retratto_induce_iniezione_in_omologia_singolare.org}{Inclusione di un retratto induce iniezione in omologia singolare}
\item[{$\boxtimes$}] \href{20250122160145-categoria_topp.org}{Categoria-TopP}
\item[{$\boxtimes$}] \href{20250124163705-funtore_da_topp_a_chr_diesis.org}{Funtore da TopP a ChR - diesis}
\item[{$\boxtimes$}] \href{20250126191208-funtore_da_topp_a_rmod_di_omologia.org}{Funtore da TopP a RMod - di omologia}
\item[{$\boxtimes$}] \href{20250124163710-diagramma_commutativo_dell_omologia_di_due_coppie_topologiche.org}{Diagramma commutativo dell'omologia di due coppie topologiche}
\end{itemize}
\subsection{{\bfseries\sffamily DONE} Lezione 10 [100\%]}
\label{sec:org5240ef2}

\begin{itemize}
\item[{$\boxtimes$}] \href{20250126190440-equivalenze_omotopiche_tra_coppie_topologiche_induce_isomorfismo_tra_omologia_singolare_relativa.org}{Equivalenze omotopiche tra coppie topologiche induce isomorfismo tra omologia singolare relativa}
\item[{$\boxtimes$}] \href{20250127093652-gruppo_abelianizzato.org}{Gruppo abelianizzato}
\item[{$\boxtimes$}] \href{20250127093602-ogni_funzione_da_un_gruppo_ad_un_gruppo_abeliano_fattorizza_tramite_il_gruppo_abelianizzato.org}{Ogni funzione da un gruppo ad un gruppo abeliano fattorizza tramite il gruppo abelianizzato}
\item[{$\boxtimes$}] \href{20250126215428-funtore_da_grp_a_ab.org}{Funtore da Grp a Ab}
\item[{$\boxtimes$}] \href{20250126215534-teorema_di_hurewicz.org}{Teorema di Hurewicz} (studiare dimostrazione)
\end{itemize}
\subsection{Lezione 11 [91\%]}
\label{sec:org86daee6}

\begin{itemize}
\item[{$\boxtimes$}] \href{20250126215534-teorema_di_hurewicz.org}{Mappa naturale tra il funtore del gruppo di omotopia e il primo gruppo di omologia}
\item[{$\boxtimes$}] \href{20250126223310-teorema_di_escissione.org}{Teorema di Escissione}
\item[{$\boxtimes$}] \href{20250128151459-sottocomplesso_di_catene.org}{Sottocomplesso di catene}
\item[{$\boxtimes$}] \href{20260203110150-complesso_di_catene_somma.org}{Complesso di catene Somma}
\item[{$\boxtimes$}] \href{20250128151511-quoziente_di_complesso_di_catene_e_sottocomplesso.org}{Quoziente di complesso di catene e sottocomplesso}
\item[{$\boxtimes$}] \href{20260203110110-complesso_di_catene_immagine.org}{Complesso di catene Immagine}
\item[{$\boxtimes$}] \href{20250120163759-categoria_complessi_di_catene.org}{Morfismo tra complessi di catene}
\item[{$\boxtimes$}] \href{20250120183640-sec_di_complessi_di_catene.org}{SEC di Complessi di Catene}
\item[{$\boxtimes$}] \href{20250120155457-morfismo_iniettivo_di_r_moduli_induce_isomorfismo.org}{Teoremi di isomorfismo}
\item[{$\boxtimes$}] \href{20250128131221-complesso_di_catene_singolare_somma.org}{Complesso di catene singolare somma}
\item[{$\boxtimes$}] \href{20250128131221-complesso_di_catene_singolare_somma.org}{Complesso di catene singolare intersezione}
\item[{$\square$}] \href{20250127162702-calcolo_dell_omologia_singolare_della_sfera_e_dell_omologia_singolare_relativa_del_disco_rispetto_alla_sfera.org}{Calcolo dell'omologia singolare della sfera e dell'omologia singolare relativa del disco rispetto alla sfera} TODO
\end{itemize}
\subsection{Lezione 12 [60\%]}
\label{sec:orgf8b8a8d}

\begin{itemize}
\item[{$\boxtimes$}] \href{20250128131908-suddivisione_baricentrica.org}{Suddivisione baricentrica}
\item[{$\square$}] \href{20250128132040-mappa_di_suddivisione_tra_complessi_di_catene_singolari.org}{Mappa di suddivisione tra complessi di catene singolari} TODO
\item[{$\boxtimes$}] \href{20250128132104-mappa_di_suddivisione_e_omotopa_a_identita.org}{Mappa di suddivisione è omotopa a identità}
\item[{$\square$}] \href{20250128132200-mesh_di_un_elemento_del_modulo_delle_catene_singolari.org}{Mesh di un elemento del modulo delle catene singolari} TODO
\item[{$\boxtimes$}] \href{20250126223310-teorema_di_escissione.org}{Teorema di Escissione}
\end{itemize}
\subsection{Lezione 13 [88\%]}
\label{sec:org40b18be}

\begin{itemize}
\item[{$\boxtimes$}] \href{20260204100611-somma_diretta_di_complessi_di_catene.org}{Somma diretta di complessi di catene}
\item[{$\boxtimes$}] \href{20260204120902-omologia_della_somma_diretta_di_complessi_di_catene.org}{Omologia della somma diretta di complessi di catene}
\item[{$\boxtimes$}] \href{20250128132648-teorema_di_mayer_vietoris.org}{Teorema di Mayer-Vietoris}
\item[{$\boxtimes$}] \href{20250128132743-prodotto_wedge_di_spazi_topologici_puntati.org}{Prodotto wedge di spazi topologici puntati}
\item[{$\square$}] \href{20250128132815-calcolo_dell_omologia_singolare_del_punto.org}{Calcolo dell'omologia singolare del toro} TODO
\item[{$\boxtimes$}] \href{20250128132830-teorema_del_punto_fisso_di_brower.org}{Teorema del punto fisso di Brower}
\item[{$\boxtimes$}] \href{20250128132917-omologia_locale.org}{Omologia Locale}
\item[{$\boxtimes$}] \href{20250128132928-omologia_locale_di_una_varieta_topologica.org}{Omologia Locale di una varietà topologica}
\item[{$\boxtimes$}] \href{20250128133005-teorema_di_invarianza_della_dimensione.org}{Teorema di invarianza della dimensione}
\end{itemize}
\subsection{Lezione 14 [88\%]}
\label{sec:orgbd00760}

\begin{itemize}
\item[{$\boxtimes$}] \href{20250129170910-grado_di_un_endomorfismo_della_sfera.org}{Grado di un endomorfismo della sfera}
\item[{$\boxtimes$}] \href{20250120164938-zig_zag_lemma.org}{Zig-Zag Lemma per due SEC con funzioni finali opposte}
\item[{$\boxtimes$}] \href{20250129174802-grado_di_una_matrice_ortogonale_come_endomorfismo_della_sfera.org}{Grado di una matrice ortogonale come endomorfismo della sfera}
\item[{$\square$}] \href{20250129175123-gruppo_ortogonale_speciale_e_connesso_per_archi.org}{Gruppo ortogonale speciale è connesso per archi} TODO?
\item[{$\boxtimes$}] \href{20250129180509-endomorfismo_di_una_sfera_senza_punti_fissi_e_omotopa_alla_mappa_antipodale.org}{Endomorfismo di una sfera senza punti fissi è omotopa alla mappa antipodale}
\item[{$\boxtimes$}] \href{20250102154204-teorema_fondamentale_dell_algebra.org}{Teorema Fondamentale dell'Algebra}
\item[{$\boxtimes$}] \href{20250129183256-coppia_topologica_buona.org}{Coppia topologica buona}
\item[{$\boxtimes$}] \href{20250129183516-quoziente_di_una_coppia_topologica_buona.org}{Contrazione di un sottospazio topologico ad un punto}
\item[{$\boxtimes$}] \href{20260205123011-omologia_singolare_relativa_di_una_coppia_topologica_buona.org}{Omologia Singolare Relativa di una coppia topologica buona}
\end{itemize}
\subsection{{\bfseries\sffamily DONE} Lezione 15 [100\%]}
\label{sec:org95c717a}

\begin{itemize}
\item[{$\boxtimes$}] \href{20250215122759-export.org_archive}{Attaccamento di una k-cella a uno spazio topologico}
\item[{$\boxtimes$}] \href{20260205160413-omologia_per_attaccamento_di_k_cella.org}{Omologia per attaccamento di k-cella}
\item[{$\boxtimes$}] \href{20260205161432-cw_complesso.org}{CW-complesso}
\end{itemize}
\subsection{{\bfseries\sffamily DONE} Lezione 16 [100\%]}
\label{sec:org3fe77da}

\begin{itemize}
\item[{$\boxtimes$}] \href{20260205161432-cw_complesso.org}{CW-complesso}
\item[{$\boxtimes$}] \href{20260205161626-struttura_di_cw_complesso_per_spazio_proiettivo_reale.org}{Struttura di CW-complesso per spazio proiettivo reale}
\item[{$\boxtimes$}] \href{20260207173708-unione_finita_di_compatti_e_compatta.org}{Unione finita di compatti è compatta}
\item[{$\boxtimes$}] \href{20260205161626-struttura_di_cw_complesso_per_spazio_proiettivo_reale.org}{Struttura di CW-complesso per spazio proiettivo complesso}
\item[{$\boxtimes$}] \href{20260205161836-omologia_singolare_di_cw_complessi.org}{Omologia singolare di CW-complessi}
\item[{$\boxtimes$}] \href{20260205161912-complesso_di_catene_cellulari.org}{Complesso di catene cellulari}
\item[{$\boxtimes$}] \href{20260205161912-complesso_di_catene_cellulari.org}{Omologia Cellulare}
\end{itemize}
\subsection{Lezione 17 [\%]}
\label{sec:org6460283}
\end{document}
