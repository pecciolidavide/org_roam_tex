% Intended LaTeX compiler: pdflatex
\documentclass[../main]{subfiles}


\begin{document}

\section{Proposizione}
\label{sec:orgc547599}
Sia \(M\) una \href{20250113115909-struttura_differenziabile.org}{varietà differenziabile} e sia \(N \subseteq M\) una \href{20250114124541-sottovarieta_differenziabile.org}{sottovarietà}. Se \(i:N\hookrightarrow M\) è l'inclusione, allora l'immagine del suo \href{20250114111331-differenziale_di_una_funzione_tra_varieta_differenziabili.org}{differenziale}
\begin{equation*}
\restriction{i_{\star}}{p}(\operatorname{T}_{p}N) \subseteq \operatorname{T}_{p}M
\end{equation*}
(vedi \href{20250114102823-spazio_tangente_ad_un_punto_di_una_varieta_differenziabile.org}{Spazio tangente ad un punto di una varietà differenziabile})
In particolare
\begin{equation*}
\restriction{i_{\star}}{p}(\operatorname{T}_{p}N)= \set{\bm{v} \in \operatorname{T}_{p}M: \forall\, f \in \mathscr{F}_{p}: \restriction{f}{N}\equiv 0,\  \bm{v}(f_{p}) = 0 }
\end{equation*}
(vedi \href{20250114101437-derivazione_su_una_varieta_differenziabile.org}{Derivazione su una varietà differenziabile}, \href{20250114100254-germi_di_funzioni.org}{Germi di funzioni})
\subsection{{\bfseries\sffamily TODO} Dimostrazione\hfill{}\textsc{matematica\_lm:geo\_diff}}
\label{sec:org99ad456}
\end{document}
