% Intended LaTeX compiler: pdflatex
\documentclass[../main]{subfiles}

\usepackage[hyperref]{biblatex}
\date{}
\title{}
\begin{document}

\section{Topologia indotta da una distanza}
\label{sec:orge7f49ce}
\href{https://it.wikiversity.org/wiki/Metriche\_e\_Topologie\_indotte}{Wikiversity}
\subsection{Definizione}
\label{sec:orgcb6a33f}
Sia \((X,d)\) uno \href{20250301193511-spazio_metrico.org}{spazio metrico} in questione.

Si denoti con \(B_{d}(x,r) \coloneqq \set{y \in X\mid d(x,y)<r}\). La \uline{\href{20250103145124-topologia.org}{topologia} indotta dalla metrica} \(\tau\) di \(X\) è quella che per \href{20250111142837-base_di_una_topologia.org}{base}:
\begin{equation*}
\mathcal{B} = \set{B_{d}(x,y)\mid x \in X \,\land\, r > 0}
\end{equation*}
\subsubsection{Proposizione}
\label{sec:org7b342ed}
Per ogni \(\varepsilon >0\), la seguente è una base di \(\tau\):
\begin{equation*}
\mathcal{B}_{\varepsilon} = \set{B_{d}(x,r)\mid x \in X \,\land\, r \in \Q^{+}\cap (0,\varepsilon)}
\end{equation*}
\end{document}
