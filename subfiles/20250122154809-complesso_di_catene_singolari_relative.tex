% Intended LaTeX compiler: pdflatex
\documentclass[../main]{subfiles}


\begin{document}

Sia \(R\) un \href{20241219112842-pid.org}{PID} e sia \((A,X)\) una \href{20250122154728-coppia_topologica.org}{coppia topologica}
\begin{definizione}
Si definisce il \href{20241205141053-r_moduli.org}{modulo} delle \(q\)-catene singolari relative come il \href{20241206142802-sottomoduli.org}{quoziente}\footnote{\(S_{q}(X)\) è il \href{20250122133435-simplesso_singolare.org}{modulo delle \(q\)-catene singolari}: siccome \(A \subseteq X\), allora \(\Sigma_{q}(A) \subseteq \Sigma_{q}(X)\), e quindi
\begin{equation*}
R^{(\Sigma_{q}(A))}= S_{q}(A) \subseteq S_{q}(X) = R^{(\Sigma_{q}(X))}
\end{equation*}
sono \href{20241206142802-sottomoduli.org}{sottomoduli}.}
\begin{equation*}
S_{q}(X,A) = \frac{S_{q}(X)}{S_{q}(A)}
\end{equation*}
\end{definizione}
\begin{definizione}
Si definisce la mappa di bordo
\begin{align*}
\overline{\partial}_{q}: S_{q}(X,A) &\longrightarrow S_{q-1}(X,A)\\
\end{align*}
che per ogni \(\sigma \in \Sigma_{q}(X)\)\footnote{\(\Sigma_{q}(X)\) è l'\href{20250122133435-simplesso_singolare.org}{insieme dei simplessi singolari su \(X\)}, ed è una \href{20241213094625-modulo_libero.org}{base} della \href{20241213095808-somma_diretta.org}{somma diretta} \(S_{q}(X) \coloneqq R^{(\Sigma_{q})}\).}, associa
\begin{equation*}
\overline{\partial}_{q}(\sigma+S_{q}(A)) \coloneqq \partial_{q}\sigma + S_{q-1}(A)
\end{equation*}
\end{definizione}
\begin{oss}
Questo è il \href{20250128151511-quoziente_di_complesso_di_catene_e_sottocomplesso.org}{quoziente} \(\mathcal{S}_{\bullet}(X)/\mathcal{S}_{\bullet}(A)\).
\end{oss}
\begin{definizione}
Il \textbf{complesso di catene singolari relative} di \((X,A)\) è il \href{20250120163114-complesso_di_catene.org}{complesso di catene} di \(R\)-moduli:
\begin{equation*}
\mathcal{S}_{\bullet}(X,A) = \set{\left(S_{q}(X,A),\overline{\partial}_{q}\right)}_{q}
\end{equation*}
\end{definizione}
\end{document}
