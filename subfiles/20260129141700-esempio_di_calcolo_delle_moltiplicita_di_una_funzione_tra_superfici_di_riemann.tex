% Intended LaTeX compiler: pdflatex
\documentclass[../main]{subfiles}


\begin{document}

\section{Esempio di calcolo delle moltiplicità di una funzione tra superfici di Riemann}
\label{sec:orgedb02f2}
\begin{esempio}
Si consideri la seguente \href{20260128143822-funzione_olomorfa_su_una_superficie_di_riemann.org}{funzione olomorfa} tra le \href{20260127112905-sfera_di_riemann.org}{sfere di Riemann}:
\begin{align*}
F: \C_{\infty} &\longrightarrow \C_{\infty}\\
z &\longmapsto z^{2}\\
\infty &\longmapsto \infty
\end{align*}
Se ne vogliono calcolare le \href{20260129104215-forma_normale_locale_per_superfici_di_riemann.org}{molteplicità}. Allora:
\begin{itemize}
\item per \(0 \in \C \subseteq \C_{\infty}\), si ha che \(\mathrm{mult}_{0}\, F = 2\), in quanto la funzione è localmente iniettiva;
\item per \(z_{0} \in \C\setminus \set{0} \subseteq \C_{\infty}\), si ha \(\mathrm{mult}_{z_{0}}\, F = 1\);
\item in \(\infty\), consideriamo la scrittura tramite la carta locale \(w=1/z\): allora
\begin{equation*}
  w\mapsto w^{2}
\end{equation*}
è l'espressione di \(F\) nell'intorno di \(\infty\), e quindi \(\mathrm{mult}_{\infty}\, F = 2\).
\end{itemize}
\end{esempio}
\end{document}
