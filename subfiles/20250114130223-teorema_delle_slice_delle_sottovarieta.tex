% Intended LaTeX compiler: pdflatex
\documentclass[../main]{subfiles}


\begin{document}

\section{Teorema delle slice delle sottovarietà}
\label{sec:org0a76d57}
\subsection{Teorema}
\label{sec:orgbfdbb64}
Sia \(M\) una \href{20250113115909-struttura_differenziabile.org}{varietà differenziabile} di dimensione \(n\), e sia \(N \subseteq M\) una \href{20250114124541-sottovarieta_differenziabile.org}{sottovarietà} di dimensione \(k\). Allora per ogni \(p \in N\), esiste una \href{20250111092123-varieta_topologica.org}{carta} \((U,\psi)\) di \(M\) con \(p \in U\) tale che
\begin{equation*}
\psi(U\cap N) = \psi(U)\cap \left(\R^{k}\times\set{\bm{0}}\right)
\end{equation*}
dove \(\bm{0} \in \R^{n-k}\).

\((U,\psi)\) si dice \textbf{slice} di \(N\).
\subsubsection{{\bfseries\sffamily TODO} Dimostrazione\hfill{}\textsc{matematica\_lm:geo\_diff}}
\label{sec:org61b343d}
Vedi Lezione 7 Ist. Geo Diff
\subsubsection{Viceversa}
\label{sec:orgd80c692}
Se \(N \subseteq M\) è un sottinsieme tale che per ogni \(p \in N\) esista una \href{20250111092123-varieta_topologica.org}{carta} \((U,\psi)\) di \(M\) con \(p \in U\) tale che
\begin{equation*}
\psi(U\cap N) = \psi(U)\cap \left(\R^{k}\times\set{\bm{0}}\right)
\end{equation*}
dove \(\bm{0} \in \R^{n-k}\), allora \(N\) ha una naturale struttura di \href{20250114124541-sottovarieta_differenziabile.org}{sottovarietà} di \(M\).
\paragraph{{\bfseries\sffamily TODO} Dimostrazione\hfill{}\textsc{matematica\_lm:geo\_diff}}
\label{sec:org268edcf}
Vedi Lezione 7 Ist. Geo Diff
\end{document}
