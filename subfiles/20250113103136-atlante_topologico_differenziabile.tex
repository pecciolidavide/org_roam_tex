% Intended LaTeX compiler: pdflatex
\documentclass[../main]{subfiles}


\begin{document}

\section{Atlante topologico differenziabile}
\label{sec:org93103dd}
\begin{definizione}
Sia \(M\) una \href{20250111092123-varieta_topologica.org}{varietà topologica}, e sia \(\mathcal{U} = \set{(U_{\alpha},\varphi_{\alpha})}\) un \href{20250113095039-atlante_topologico.org}{atlante topologico}.
\(\mathcal{U}\) si dice \textbf{atlante topologico differenziabile} o \textbf{atlante differnziabile} se tutti i \href{20250113103405-cambio_di_carte_per_un_atlante_topologico.org}{cambi di carte} sono dei \href{20250113103415-diffeomorfismo.org}{diffeomorfismi} tra \href{20250103145124-topologia.org}{aperti} di \(\R^{n}\).
\end{definizione}

\begin{oss}
Se \(M\) è una varietà topologica con un'unica carta, \((M,\varphi)\), allora \(\set{(M,\varphi)}\) è un atlante differenziabile, infatti l'unico cambio di carte è \(\operatorname{Id}=\varphi\circ\varphi^{-1}\)
\end{oss}
\end{document}
