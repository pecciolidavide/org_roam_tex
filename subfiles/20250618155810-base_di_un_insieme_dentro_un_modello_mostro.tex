% Intended LaTeX compiler: pdflatex
\documentclass[../main]{subfiles}


\begin{document}

Si utilizza la \href{20250612143636-notazione_teoria_dei_modelli.org}{notazione della Teoria dei Modelli}.

Sia \(\mathcal{L}\) un \href{20250130162057-linguaggio_del_prim_ordine.org}{linguaggio del prim'ordine} fissato, \(T\) una \(\mathcal{L}\)-\href{20250130114950-teoria_del_prim_ordine.org}{teoria} \href{20250131123151-teoria_completa.org}{completa} senza \href{20250131122945-modello_di_un_insieme_di_formule.org}{modelli} finiti, ed un modello \(\mathcal{U}\) di \href{20241213101756-cardinalita.org}{cardinalità} \(\kappa>\card{\mathcal{L}}\), con \(\kappa\) \href{20250211123155-cardinale_limite_forte.org}{inaccessibile}. Si lavora nell'ambito di un \href{20250617102733-modello_mostro.org}{MODELLO MOSTRO}.
\section{Definizione}
\label{sec:org71b70d5}

Sia \(C \subseteq \mathcal{U}\). Un insieme \(B \subseteq C\) si dice \uline{base di \(C\)} se
\begin{itemize}
\item \(B\) è un insieme \href{20250618095344-elementi_algebrici_e_definibili_teoria_dei_modelli.org}{algebricamente indipendente};
\item \href{20250618095344-elementi_algebrici_e_definibili_teoria_dei_modelli.org}{\(C \subseteq \operatorname{acl}(B)\)}.
\end{itemize}
\section{Caratterizzazione}
\label{sec:org690407e}

Se \(T\) è \href{20250618153446-struttura_minimale.org}{fortemente minimale} allora per ogni \(B \subseteq C \subseteq \mathcal{U}\) sono fatti equivalenti:
\begin{enumerate}
\item \(B\) è una base di \(C\):
\item \(B\) è un sottoinsieme indipendente \href{20250203102516-massimo_e_minimo.org}{massimale} (rispetto all'inclusione) di \(C\).
\end{enumerate}
\end{document}
