% Intended LaTeX compiler: pdflatex
\documentclass[../main]{subfiles}


\begin{document}

\textbf{\textbf{\uline{NOTA}: quando si parla di \uline{classi}, se ne parla nell'ambito della \href{20250130104245-morse_kelly_set_theory.org}{Morse Kelly Set Theory}; quando si parla di insiemi, il discorso ha validità più generale.}}
\begin{definizione}
Sia \(X\) un \href{20250130104331-insieme_mk.org}{insieme} (o una \href{20250130104320-classe_mk.org}{classe}) e sia \(R\) una \href{20250202170607-classe_relazione_binaria.org}{relazione binaria} su \(X\), che sia un \href{20250619163724-preordine.org}{preordine}.

\begin{itemize}
\item \(X\) si dice \uline{diretto superiormente} se
\begin{equation*}
  \forall\,x,y \in X\ \exists\,z \in X\ (x\mathrel{R}z \,\land\, y\mathrel{R}z).
\end{equation*}
\item \(X\) si dice \uline{diretto inferiormente} se
\begin{equation*}
  \forall\,x,y \in X\ \exists\,z \in X\ (z\mathrel{R}x \,\land\, z\mathrel{R}y).
\end{equation*}
\end{itemize}
\end{definizione}
\end{document}
