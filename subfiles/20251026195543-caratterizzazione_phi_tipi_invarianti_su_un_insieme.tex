% Intended LaTeX compiler: pdflatex
\documentclass[../main]{subfiles}


\begin{document}

\def\L{\mathcal{L}} % Il linguaggio
\def\U{\mathcal{U}} % Modello mostro
\def\equivalentover#1{\mathrel{\equiv_{#1}}}
\def\restricted{\upharpoonright}
\section{Caratterizzazione phi-tipi invarianti su un insieme}
\label{sec:org071c676}
Si utilizza la \href{20250612143636-notazione_teoria_dei_modelli.org}{Notazione della TEORIA DEI MODELLI}

Sia \(\L\) un \href{20250130162057-linguaggio_del_prim_ordine.org}{linguaggio}, \(T\) una \href{20250130114950-teoria_del_prim_ordine.org}{teoria} \href{20250131123151-teoria_completa.org}{completa} senza \href{20250131122945-modello_di_un_insieme_di_formule.org}{modelli} finiti e \(\U\) un \href{20250617095548-modello_lambda_saturo.org}{modello saturo} di \href{20241213101756-cardinalita.org}{cardinalità} \href{20250211123155-cardinale_limite_forte.org}{inaccessibile} \(\kappa>\card{\L}+ \omega\). \(\U\) è un \href{20250617102733-modello_mostro.org}{modello mostro}.

Sia \(\varphi(x;z) \in \L\) una \href{20250131103317-formula_del_prim_ordine.org}{formula}, e si denoti con \(S_{\varphi}(\U)\) l'insieme di tutti i \href{20251026161720-phi_formula_su_un_insieme.org}{\(\varphi\)-tipi globali}.

Se \(p(x) \in S_{\varphi}(\U)\) allora a meno di equivalenza si può considerare
\begin{equation*}
p(x) \subseteq \set{\varphi(x;b),\lnot\varphi(x;b)\mid b \in \U^{z}}.
\end{equation*}

??? \href{20250515141706-da_finire.org}{DA FINIRE}

\begin{equation*}
p(x) \subseteq \set{\land,\lor}\! \set{\varphi(x;b),\lnot\varphi(x;b)\mid b \in \U^{z}}
\end{equation*}


\uline{Notazione}: definisco
\begin{equation*}
\mathscr{D}_{p,\varphi} \coloneqq \set{b \in \U^{z}\mid \varphi(x;b) \in p(x)}.
\end{equation*}

\(\mathscr{D}_{p,\varphi}\) è un insieme \textbf{esternamente definibile} (da \(\varphi(x;z)\) o da \(p\)).

Sia \(\nonstandard{\U}\succeq \U\) saturo di cardinalità maggiore, tale che
\begin{equation*}
\nonstandard{\U},\nonstandard{a}\vDash p(x)
\end{equation*}
allora
\begin{equation*}
\mathscr{D}_{p,\varphi} = \U^{z}\cap \varphi(\nonstandard{\!a},\nonstandard{\U}^{z}).
\end{equation*}

\begin{prop}
\(p(x)\) è \(A\)-invariante sse \(\mathscr{D}_{p,\varphi}\) è \(A\)-invariante.
\end{prop}

\uline{Fatto.}
Sia \(p(x) \subseteq \L_{\varphi}(\U)\). \(p(x)\) è \(A\)-invariante sse per ogni \(\psi(x,\overline{b}) \in \L_{\varphi}(\U)\) e per ogni \(a \equivalentover{A} b\) si ha:
\begin{equation*}
	p(x)\vdash \psi(x;a) \oldiff p(x)\vdash \psi(x;b)
\end{equation*}

\uline{Fatto.}
Sia \(p(x) \in S_{\varphi}(\U)\). \(p(x)\) è \(A\)-invariante sse per ogni \(a \equivalentover{A} b\) si ha:
\begin{equation*}
p(x)\vdash \varphi(x;a)\iff\varphi(x;b).
\end{equation*}

\uline{Fatto.}
Sia \(p(x) \in S_{\varphi}(\U)\). \(p(x)\) è \(A\)-invariante sse per ogni \(a \equivalentover{A} b\) e per ogni \(c\vdash p(x) \restricted A\cup\set{a,b}\)\footnote{Ovvero tenendo soltanto le formule con parametri in \(A\cup\set{a,b}\).} si ha:
\begin{equation*}
\varphi(c;a)\iff\varphi(c;b).
\end{equation*}

\uline{Fatto.}
Sia \(p(x) \in S_{\varphi}(\U)\). \(p(x)\) è \(A\)-invariante sse per ogni \(a \equivalentover{A} b\) e per ogni \(c\vdash p(x) \restricted A\) si ha:
\begin{equation*}
a \equivalentover{A\cup\set{c}} b.
\end{equation*}
\end{document}
