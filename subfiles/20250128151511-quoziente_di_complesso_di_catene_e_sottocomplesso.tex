% Intended LaTeX compiler: pdflatex
\documentclass[../main]{subfiles}


\begin{document}

\section{Quoziente di complessi di catene}
\label{sec:orgfc5c6fb}
Sia \(R\) un \href{20241205141119-anello.org}{anello} commutativo con unità, e sia \(\mathcal{C}_{\bullet} = \set{(C_{n}, \partial_{n})}_{n \in \Z}\) un \href{20250120163114-complesso_di_catene.org}{complesso di catene di \(R\)-moduli},
\(\mathcal{D}_{\bullet} = \set{(D_{n}, \partial_{n}^{D})}_{n \in \Z}\)
un \href{20250128151459-sottocomplesso_di_catene.org}{sottocomplesso} di \(\mathcal{C}_{\bullet}\).

\begin{definizione}
Il \textbf{quoziente} è il complesso di catene:
\begin{equation*}
\mathcal{C}_{\bullet}/\mathcal{D}_{\bullet} \coloneqq \set{\left(\frac{C_{n}}{D_{n}}, \partial_{n}^{*}\right)}_{n \in \Z}
\end{equation*}
dove \(C_{n}/D_{n}\) è il \href{20241206142802-sottomoduli.org}{quoziente di \(R\)-moduli} e
\begin{align*}
\partial_{n}^{*}: \frac{C_{n}}{D_{n}} &\longrightarrow \frac{C_{n-1}}{D_{n-1}} \\
x + D_{n} &\longmapsto \partial_{n} x + D_{n-1}
\end{align*}
\end{definizione}

La definizione è ben posta poiché \(\partial_{n}[D_{n}] \subseteq D_{n-1}\)
\end{document}
