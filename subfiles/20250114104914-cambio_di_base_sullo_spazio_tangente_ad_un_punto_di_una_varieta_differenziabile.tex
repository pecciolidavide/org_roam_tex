% Intended LaTeX compiler: pdflatex
\documentclass[../main]{subfiles}

\usepackage[hyperref]{biblatex}
\date{}
\title{}
\begin{document}

\section{Cambio di base sullo spazio tangente ad un punto di una varietà differenziabile}
\label{sec:org2fb1979}
Sia \(M\) una \href{20250113115909-struttura_differenziabile.org}{varietà differenziabile} di dimensione \(n\), sia \(p \in M\), e siano \((U,\varphi_{\alpha})\), \((V,\varphi_{\beta})\) due \href{20250111092123-varieta_topologica.org}{carte} di \(M\), con \(p \in U\cap V\).
Siano
\begin{align*}
\varphi_{\alpha} &= (x^{1}_{\alpha},\dots,x^{n}_{\alpha})\\
\varphi_{\beta} &= (x^{1}_{\beta},\dots,x^{n}_{\beta})
\end{align*}

Ciascuna delle \(x_{\gamma}^{i}: U\cap V\longrightarrow \R\), e pertanto \(\left(x^{i}_{\gamma}\right)_{p} \in \mathscr{F}_{p}\) (Vedi \href{20250114100254-germi_di_funzioni.org}{Germi di funzioni}).

\(\varphi_{\alpha}\) induce la \href{20250102163502-base_di_uno_spazio_vettoriale.org}{base} di \(\operatorname{T}_{p}M\) (vedi \href{20250114102823-spazio_tangente_ad_un_punto_di_una_varieta_differenziabile.org}{Spazio tangente ad un punto di una varietà differenziabile} e \href{20250114103339-teorema_sulla_base_dello_spazio_tangente_ad_un_punto_di_una_varieta_differenziabile.org}{Teorema sulla base dello spazio tangente ad un punto di una varietà differenziabile}):
\begin{equation*}
\set{\restriction{\dpd{}{{x^{1}_{\alpha}}}}{p}, \dots, \restriction{\dpd{}{{x^{n}_{\alpha}}}}{p}}
\end{equation*}

\(\varphi_{\beta}\) induce la \href{20250102163502-base_di_uno_spazio_vettoriale.org}{base} di \(\operatorname{T}_{p}M\) (vedi \href{20250114102823-spazio_tangente_ad_un_punto_di_una_varieta_differenziabile.org}{Spazio tangente ad un punto di una varietà differenziabile} e \href{20250114103339-teorema_sulla_base_dello_spazio_tangente_ad_un_punto_di_una_varieta_differenziabile.org}{Teorema sulla base dello spazio tangente ad un punto di una varietà differenziabile}):
\begin{equation*}
\set{\restriction{\dpd{}{{x^{1}_{\beta}}}}{p}, \dots, \restriction{\dpd{}{{x^{n}_{\beta}}}}{p}}
\end{equation*}
\subsection{Proposizione}
\label{sec:org172d905}
La \href{20250104111539-spazio_delle_matrici.org}{matrice} del \href{20250114105929-matrice_del_cambiamento_di_base.org}{cambiamento di base} \(A^{k}_{r}\) è
\begin{equation*}
A^{k}_{r} = \restriction{\dpd{}{{x^{r}_{\beta}}}}{p}\left(x_{\alpha}^{i}\right)_{p}
\end{equation*}
dove, con la \href{20250114105957-notazione_di_einstein.org}{notazione di Einstein}
\begin{equation*}
\restriction{\dpd{}{{x^{r}_{\beta}}}}{p} = A^{k}_{r}\restriction{\dpd{}{{x^{k}_{\alpha}}}}{p}
\end{equation*}
\subsubsection{{\bfseries\sffamily TODO} Dimostrazione\hfill{}\textsc{matematica\_lm:geo\_diff}}
\label{sec:org76a29ec}

\subsubsection{Notazione}
\label{sec:orgb64c1c6}
La matrice \(A^{k}_{r}\) si indica anche come segue:
\begin{equation*}
A^{k}_{r} = \restriction{\dpd{{x^k_{\alpha}}}{{x^{r}_{\beta}}}}{p} = \dpd{{x^k_{\alpha}}}{{x^{r}_{\beta}}}(p)
\end{equation*}
\end{document}
