% Intended LaTeX compiler: pdflatex
\documentclass[../main]{subfiles}

\usepackage[hyperref]{biblatex}
\date{}
\title{}
\begin{document}

\section{Teoremi di Whitney}
\label{sec:org130fa23}
\subsection{Teorema I}
\label{sec:orge9336d6}
Sia \(M\) una \href{20250113115909-struttura_differenziabile.org}{varietà differenziabile} di dimensione \(n\). Allora esistono
\begin{equation*}
i: M\hookrightarrow \R^{2n+1}
\end{equation*}
un \href{20250114124533-embedding_di_varieta_differenziabili.org}{embedding} \href{20250115100507-funzione_propria.org}{proprio} e
\begin{equation*}
j:M\longrightarrow \R^{2n}
\end{equation*}
una \href{20250114124504-immersione_di_varieta_differenziabili.org}{immersione}.
\subsection{Teorema II}
\label{sec:org262b3d5}
Sia \(M\) una \href{20250113115909-struttura_differenziabile.org}{varietà differenziabile} di dimensione \(n\). Allora esiste
\begin{equation*}
i:M\hookrightarrow \R^{2n}
\end{equation*}
un \href{20250114124533-embedding_di_varieta_differenziabili.org}{embedding} e, se \(n\ge 2\), esiste
\begin{equation*}
j:M\longrightarrow \R^{2n-1}
\end{equation*}
\href{20250114124504-immersione_di_varieta_differenziabili.org}{immersione}.
\end{document}
