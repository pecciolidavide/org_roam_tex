% Intended LaTeX compiler: pdflatex
\documentclass[../main]{subfiles}


\begin{document}

\section{Coomologia a supporto compatto}
\label{sec:org616f407}
\begin{oss}
Si considerino \(\R\) e \(\R^{n}\), con \(n\neq 1\). \href{20251229100321-rn_e_contraibile.org}{Allora}
\begin{equation*}
\R \approx \set{p} \approx \R^{n}
\end{equation*}
dove ``\(\approx\)'' indica che sono \href{20250124155008-spazi_topologici_omotopicamente_equivalenti.org}{omotopi}.

Considerando invece la coomologia a supporto compatto \href{20251115192308-coomologia_a_supporto_compatto_di_r.org}{di \(\R\)} e \href{20251115192344-coomologia_a_supporto_compatto_di_rn.org}{di \(\R^{n}\)}:
\begin{equation*}
\forall  k:\qquad H^{k}_{\text{c}}(\R) \neq H^{k}_{\text{c}}(\R^{n})
\end{equation*}
e pertanto, possiamo affermare che \(H^{k}_{\text{c}}(M)\) non è invariante per omotopia.
\end{oss}
\begin{oss}
La \hyperref[sec:org616f407]{coomologia a supporto compatto} è uguale alla \href{20251115182537-coomologia_di_un_complesso_di_cocatene.org}{coomologia di complesso di cocatene} di \href{20251229110021-forma_differenziale_a_supporto_compatto.org}{\(A^{\bullet}_{\text{c}}\)}.
\end{oss}
\section{Coomologia a supporto compatto 0-dimensionale}
\label{sec:org40a1dd5}
Si confronti con ``\href{20251115174538-0_gruppo_di_coomologia_di_de_rham_di_una_varieta_connessa.org}{Coomologia di De Rham 0-dimensionale}''

Se \(M\) è una \href{20250113115909-struttura_differenziabile.org}{varietà differenziabile}:
\begin{itemize}
\item Se \(M\) è \href{20250103165325-spazio_topologico_connesso.org}{connessa}, allora:
\begin{itemize}
\item se \(M\) è \href{20250103163701-spazio_topologico_compatto.org}{compatta}, allora \(H^{0}_{\text{c}}(M) \cong \R\);
\item se \(M\) non è \href{20250103163701-spazio_topologico_compatto.org}{compatta}, allora \(H^{0}_{\text{c}}(M) = 0\).
\end{itemize}
\end{itemize}
\end{document}
