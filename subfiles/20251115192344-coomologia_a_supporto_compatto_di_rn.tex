% Intended LaTeX compiler: pdflatex
\documentclass[../main]{subfiles}


\begin{document}

\section{Coomologia a supporto compatto di Rn}
\label{sec:org59eaa2b}
\begin{thm}
(Lemma di Poincaré a supporto compatto).
Se \(H^{k}_{\text{c}}\) indica la \href{20251115192241-coomologia_a_supporto_compatto.org}{coomologia a supporto compatto}, allora
\begin{equation*}
H^{k}_{\text{c}}(\R^{n}) = %
\begin{cases}
\R & k = n\\
0 & \text{altrimenti}.
\end{cases}
\end{equation*}
\end{thm}

\begin{oss}
Consideriamo \(e : \R \to \R^{+}\) \href{20250113125602-classe_c_di_una_funzione.org}{funzione \(C^{\infty}\)} a \href{20250701115005-supporto_di_una_funzione.org}{supporto} \href{20250103163701-spazio_topologico_compatto.org}{compatto}, tale che
\begin{equation*}
\int_{\R} e \dif t = 1.
\end{equation*}
Sia \(\varepsilon \in A^{n}_{\text{c}}(\R^{n})\) una \href{20251115155511-forma_differenziale_in_un_punto.org}{\(n\)-forma differenziale} a \href{20251223153341-supporto_di_una_forma_differenziale.org}{supporto} \href{20250103163701-spazio_topologico_compatto.org}{compatto}:
\begin{equation*}
\varepsilon \coloneqq %
	e(x^{1}) \cdot \dots \cdot e(x^{n}) %
	\dif x^{1} \wedge \dots \wedge \dif x^{n},
\qquad [\varepsilon] \in H^{n}_{\text{c}}(\R^{n}).
\end{equation*}
Allora l'\href{20251115185654-integrazione_di_forme_su_varieta_differenziabile_orientata.org}{integrale} \(\int_{\R^{n}} \varepsilon = 1\) e pertanto \([\varepsilon] \in H^{n}_{\text{c}}(\R^{n})\) lo genera.
\end{oss}
\end{document}
