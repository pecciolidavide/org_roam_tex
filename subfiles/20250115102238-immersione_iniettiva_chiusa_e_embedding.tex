% Intended LaTeX compiler: pdflatex
\documentclass[../main]{subfiles}

\usepackage[hyperref]{biblatex}
\date{}
\title{}
\begin{document}

\section{Immersione iniettiva chiusa è embedding}
\label{sec:orgbad2696}
Siano \(M, N\) \href{20250113115909-struttura_differenziabile.org}{varietà differenziabili}, e sia \(F: M\longrightarrow N\) una \href{20250114124504-immersione_di_varieta_differenziabili.org}{immersione}.
\subsection{Proposizione}
\label{sec:org6edcbbd}
Se \(F\) è \href{20241219101956-funzione_iniettiva.org}{iniettiva} e \href{20250104114559-funzione_chiusa.org}{chiusa}, allora \(F\) è un \href{20250114124533-embedding_di_varieta_differenziabili.org}{embedding}.
\subsubsection{Dimostrazione}
\label{sec:orgfa794a9}
Se \(F\) è \href{20241219101956-funzione_iniettiva.org}{iniettiva}, allora \(F\) è \href{20250104111707-funzione_biunivoca.org}{biunivoca} tra \(M\) e \(F(M)\). Inoltre \(F\) è \href{20250104114559-funzione_chiusa.org}{chiusa} implica che \(F^{-1}\) sia \href{20250103103252-funzione_continua.org}{continua}, dunque
\begin{equation*}
F:M\longrightarrow F(M)
\end{equation*}
è \href{20250111142332-omeomorfismo.org}{omeomorfismo}.
\end{document}
