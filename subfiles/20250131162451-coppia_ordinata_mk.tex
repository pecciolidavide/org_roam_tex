% Intended LaTeX compiler: pdflatex
\documentclass[../main]{subfiles}


\begin{document}

\section{Coppia ordinata}
\label{sec:orge70412c}
\subsection{Formalizzazione in MK}
\label{sec:org7b26482}

Contesto: \href{20250130104245-morse_kelly_set_theory.org}{Morse Kelly Set Theory}
\subsubsection{Coppia ordinata di insiemi (Kuratowski)}
\label{sec:org3ea3edc}

Se \(x,y\) sono \href{20250130104331-insieme_mk.org}{insiemi}, allora si definisce la coppia ordinata
\begin{equation*}
(x,y) \coloneqq \set{\set{x},\set{x,y}}
\end{equation*}
(che è ancora un insieme per l'assioma of Pairing).

Ovviamente \((x,y)\neq (y,x)\).
\subsubsection{Proposizione}
\label{sec:orgb83844a}
Per ogni insiemi \(x,y,z,w\) si ha
\begin{equation*}
(x,y) = (z,w)\,\iff\, x=z \,\land\, y=w
\end{equation*}
\paragraph{Dimostrazione}
\label{sec:org09506e5}

Si dimostra solo (\(\implies\)) perché l'altra implicazione è ovvia
\begin{enumerate}
\item Caso 1:
\label{sec:orgad0c33d}

Se \(x=y\), allora
\begin{equation*}
\set{\set{x}}= (x,y) = (z,w) = \set{\set{z},\set{z,w}}
\end{equation*}
e dunque \(\set{x}=\set{z}=\set{z,w}\).

Questo implica che \(x=z\) e \(z=w\).
Quindi \(x=y=z=w\), che implica la tesi.

Allo stesso modo \(y=w\) implica la tesi.
\item Caso 2:
\label{sec:org6d11e0f}

Supponiamo ora \(x\neq y\) e \(z\neq w\).

Siccome \(\set{x} \in (x,y)=(z,w) = \set{\set{z},\set{z,w}}\) allora necessariamente o \(x=z\) oppure \(x=z=w\). Ma per ipotesi \(z\neq w\), e quindi \(x=z\).

Siccome \(\set{x,y} \in (x,y) = (z,w) = (x,w)\) allora \(\set{x,y} = \set{x}\) oppure \(\set{x,y} = \set{x,w}\). Per ipotesi \(x\neq y\), e quindi
\begin{equation*}
\set{x,y} = \set{x,w}
\end{equation*}
e dunque \(y \in \set{x,w}\), e dunque \(x=y\) oppure \(y=w\). Per ipotesi \(x\neq y\), dunque \(y=w\).
\end{enumerate}
\subsubsection{Coppia ordinata di classi}
\label{sec:orgab4dfa9}

Se \(A, B\) sono \href{20250130104320-classe_mk.org}{classi}, e almeno una delle due è una \href{20250130104320-classe_mk.org}{classe propria}, allora la coppia \((A,B)\) è definita come l'\href{20250113175700-unione_disgiunta.org}{unione disgiunta}.
\begin{equation*}
(A,B) \coloneqq A\uplus B
\end{equation*}
\end{document}
