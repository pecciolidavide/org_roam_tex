% Intended LaTeX compiler: pdflatex
\documentclass[../main]{subfiles}

\usepackage[hyperref]{biblatex}
\date{}
\title{}
\begin{document}

\section{Unicità dell'inverso}
\label{sec:orgb84f830}
\subsection{Teoria delle Categorie}
\label{sec:org45130da}

\begin{prop}
Sia \(\mathcal{C}\) una categoria, siano \(X,Y \in \operatorname{Ob}(\mathcal{C})\) e sia \(f \in \operatorname{Hom}_{\mathcal{C}}(X,Y)\) tale che i morfismi
\begin{equation*}
   g_{s},g_{d} \in \operatorname{Hom}_{\mathcal{C}}(Y,X)
\end{equation*}
siano \(g_s\) \href{20241128141258-inverso_categoriale.org}{l'inverso \textbf{destro}} di \(f\), e \(g_d\) \href{20241128141258-inverso_categoriale.org}{l'inverso \textbf{sinistro}} di \(g\).

Allora \(g_d=g_s\) e l'inverso è unico.
\end{prop}

\begin{proof}
Consideriamo il diagramma
\begin{equation*}
\begin{tikzcd}[ampersand replacement=\&]
	Y \& X \& Y \& X
	\arrow["{g_d}", from=1-1, to=1-2]
	\arrow["{\1_Y}", bend right=-50, from=1-1, to=1-3]
	\arrow["f", from=1-2, to=1-3]
	\arrow["{\1_X}"', bend right=50, from=1-2, to=1-4]
	\arrow["{g_s}", from=1-3, to=1-4]
\end{tikzcd}
\end{equation*}
È noto che \(g_s\circ f =\1_X\) e \(f\circ g_d = \1_Y\) poiché inverse.
Per associativita
\begin{equation*}
    g_s\circ(f\circ g_d) = (g_s\circ f )\circ g_d
\end{equation*}
ovvero
\begin{align*}
	g_s\circ \1_Y &= g_s\circ(f\circ g_d)\\
	&= (g_s\circ f)\circ g_d = \1_X \circ g_d
\end{align*}
e dunque \(g_s=g_d\).

Inoltre, se \(g_s'\) fosse un altro inverso sinistro, allora
\begin{equation*}
    g_s'=g_d=g_s.\qedhere
\end{equation*}
\end{proof}
\end{document}
