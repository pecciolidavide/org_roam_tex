% Intended LaTeX compiler: pdflatex
\documentclass[../main]{subfiles}


\begin{document}

\section{Funzioni ricorsive definite per casi}
\label{sec:org3d375e8}
In questa sezione si identificano i \href{20250131155822-operazioni_insiemistiche_tra_classi_mk.org}{sottinsiemi} di \(\N^{k}\) (vedi \href{20250202130045-insieme_dei_numeri_naturali_mk.org}{Insieme dei numeri naturali MK}) con i \href{20250131103317-formula_del_prim_ordine.org}{predicati} \(k\)-ari (ovvero con \(k\) \href{20250131103429-variabile_libera_di_una_formula.org}{variabili libere}), per mezzo degli \href{20250131122913-soddisfazione_di_una_formula.org}{insiemi di verità}.

Scriveremo indifferentemente \((x_{1},\dots,x_{k}) \in P\) oppure \(P(x_{1},\dots,x_{k})\) per dire che \(\N\vDash P(x_{1},\dots,x_{k})\).
\subsection{Proposizione}
\label{sec:org3022938}

Se \(P_{1},\dots,P_{k} \subseteq \N^{\ell}\) sono \href{20250216173925-insieme_ricorsivo.org}{insiemi ricorsivi} (\href{20250216174510-insieme_ricorsivo_primitivo.org}{primitivi}) a due a due \href{20250520110008-insiemi_disgiunti.org}{disgiunti}, e per ogni \(1\le i \le k\) le
\begin{equation*}
f_{i} : \N^{\ell}\to \N
\end{equation*}
sono \href{20250215151458-funzioni_ricorsive.org}{funzioni ricorsive} (\href{20250215141024-funzioni_primitive_ricordive.org}{primitive}) \uline{\href{20250213105339-funzione_parziale.org}{totali}}, allora la funzione \(f: \N^{\ell}\to \N\) definita come
\begin{equation*}
f(\oldvec{x}) = \begin{cases}
f_{1}(\oldvec{x}) & P_{1}(\oldvec{x})\\
f_{2}(\oldvec{x}) & P_{2}(\oldvec{x})\\
\vdots\\
f_{k}(\oldvec{x}) & P_{k}(\oldvec{x})\\
0 &\text{altrimenti}
\end{cases}
\end{equation*}
è ricorsiva (primitiva).
\subsubsection{Dimostrazione}
\label{sec:org6d7b783}
\begin{equation*}
f(\oldvec{x}) = \sum_{i=1}^{k} f_{i}(\oldvec{x})\cdot \chi_{P_{i}}(\oldvec{x}).
\end{equation*}
(\href{20250519112500-proprieta_di_chiusura_delle_funzioni_primitive_ricorsive.org}{Proprietà di chiusura delle funzioni primitive ricorsive} e \href{20250215151458-funzioni_ricorsive.org}{Funzioni ricorsive})
\end{document}
