% Intended LaTeX compiler: pdflatex
\documentclass[../main]{subfiles}


\begin{document}

\section{Modello lambda ricco}
\label{sec:org79f86f5}
Sia \(\lambda\) un \href{20250203161341-cardinali.org}{cardinale}.

Sia \(\mathcal{M} = \mathcal{M}_{\text{ob}}\cup \mathcal{M}_{\text{hom}}\) una \href{20241126100904-categoria.org}{categoria} \href{20250213142026-categorie_di_modelli_e_morfismi_parziali.org}{di modelli e morfismi parziali} di \href{20250130162057-linguaggio_del_prim_ordine.org}{linguaggio} \(\mathcal{L}\) tale che
\begin{enumerate}
\item per ogni \(M \in \mathcal{M}_{\text{ob}}\) e per ogni \(A \subseteq M\), \(\id_{A} : M\partialto M\) è \(\id_{A} \in \mathcal{M}_{\text{hom}}\);
\item per ogni \(M,N \in \mathcal{M}_{\text{ob}}\), e \(k:M\partialto N\) \href{20250213105339-funzione_parziale.org}{funzione parziale}, se per ogni \(k' \subseteq k\) \href{20250205120448-classe_finita_e_infinita_mk.org}{finito} si ha \(k' \in \mathcal{M}_{\text{hom}}\), allora \(k \in \mathcal{M}_{\text{hom}}\);
\item per ogni \(k \in \mathcal{M}_{\text{hom}}\), \(k\) è \href{20250111142446-funzione_inversa.org}{invertibile} e la sua \href{20250111142446-funzione_inversa.org}{inversa} \(k^{-1} \in \mathcal{M}_{\text{hom}}\).
\end{enumerate}
\subsection{Definizione}
\label{sec:orga9f54b2}
Un modello \(N \in \mathcal{M}_{\text{ob}}\) si dice \uline{\(\lambda\)-ricco} se: per ogni \(M \in \mathcal{M}_{\text{ob}}\), per ogni \(b \in M\) e per ogni \(k \in \mathcal{M}_{\text{hom}}(M,N)\) di \href{20241213101756-cardinalita.org}{cardinalità} \(\card{\kappa}<\lambda\) esiste \(c \in N\) tale che
\begin{equation*}
k\cup\set{(b,c)} \in \mathcal{M}_{\text{hom}}(M,N)
\end{equation*}
\subsubsection{Modello ricco}
\label{sec:org67c2f80}
Diciamo che \(N\) è un \uline{modello ricco} se è \(\card{N}\)-ricco.
\end{document}
