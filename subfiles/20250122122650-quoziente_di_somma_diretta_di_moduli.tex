% Intended LaTeX compiler: pdflatex
\documentclass[../main]{subfiles}


\begin{document}

\section{Quoziente di somma diretta di moduli}
\label{sec:orgc06dd7f}
Sia \(R\) un \href{20241205141119-anello.org}{anello} commutativo con unità, e sia \(\set{M_{i}}_{i \in I}\) una famiglia di \href{20241205141053-r_moduli.org}{\(R\)-moduli}.
\begin{prop}
Se per ogni \(i \in I\), \(N_{i} \mathrel{\subseteq_{R}} M_{i}\) è un \href{20241206142802-sottomoduli.org}{sottomodulo}, allora il \href{20241206142802-sottomoduli.org}{quoziente} della somma diretta è isomorfo alla somma diretta dei quozienti:
\begin{equation*}
\frac{\bigoplus_{i \in I} M_{i}}{\bigoplus_{i \in I} N_{i}} \cong \bigoplus_{i \in I} M_{i}/N_{i}
\end{equation*}
\end{prop}
\begin{cor}
Se \(M\coloneqq\bigoplus_{i \in I} M_{i}\) è la \href{20241213095808-somma_diretta.org}{somma diretta} e, per \(J \subseteq I\) si definisce \(N\coloneqq\bigoplus_{j \in J} M_{j}\), allora il \href{20241206142802-sottomoduli.org}{quoziente}:
\begin{equation*}
M/N \cong \bigoplus_{i \in I\setminus J} M_{i}
\end{equation*}
\end{cor}
\end{document}
