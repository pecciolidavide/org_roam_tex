% Intended LaTeX compiler: pdflatex
\documentclass[../main]{subfiles}

\usepackage[hyperref]{biblatex}
\date{}
\title{}
\begin{document}

\section{Insieme dei campi vettoriali come spazio vettoriale reale e come modulo}
\label{sec:orga3810be}
Sia \(M\) una \href{20250113115909-struttura_differenziabile.org}{varietà differenziabile}, e sia \(\Gamma(M)\) l'\href{20250115110259-campo_vettoriale_su_una_varieta_differenziabile.org}{insieme dei campi vettoriali} su \(M\).
\subsection{Proposizione 1}
\label{sec:org02601f6}
\(\Gamma(M)\) ha una struttura di \href{20241205142027-spazio_vettoriale.org}{spazio vettoriale} reale.
\subsection{Proposizione 2}
\label{sec:org924daf5}
\(\Gamma(M)\) ha una struttura di \href{20241205141053-r_moduli.org}{modulo} su \(\operatorname{C}^{\infty}(M)\) (vedi \href{20250113144722-funzioni_cinfinito_tra_varieta_differenziabili.org}{Anello delle funzioni da una varietà ai reali Cinfinito}).
\subsubsection{{\bfseries\sffamily {[}?]} Dimostrazione\hfill{}\textsc{matematica\_lm:geo\_diff}}
\label{sec:org0810c00}
Se \(f,g \in \operatorname{C}^{\infty}(M)\) e \(X, Y \in \Gamma(M)\) definisco \(fX+gY \in \Gamma(M)\); lo definisco per ogni \(p \in M\) come \href{20250114101437-derivazione_su_una_varieta_differenziabile.org}{derivazione}:
\begin{equation*}
(fX+gY)_{p} h_{p} \coloneqq f(p)\ X_{p}(h_{p}) + g(p)\ Y_{p}(h_{p})
\end{equation*}
\end{document}
