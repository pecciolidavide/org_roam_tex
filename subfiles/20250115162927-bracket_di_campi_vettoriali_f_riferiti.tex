% Intended LaTeX compiler: pdflatex
\documentclass[../main]{subfiles}


\begin{document}

Siano \(M, N\) \href{20250113115909-struttura_differenziabile.org}{varietà differenziabili}, \(X_{1},X_{2} \in \Gamma(M)\), \(Y_{1},Y_{2} \in \Gamma(N)\) \href{20250115110259-campo_vettoriale_su_una_varieta_differenziabile.org}{campi vettoriali}.
Sia \(F: M\longrightarrow N\) una \href{20250113144722-funzioni_cinfinito_tra_varieta_differenziabili.org}{funzione \(C^{\infty}\)} tale che \(X_{i}\) è \href{20250115125404-campi_vettoriali_f_riferiti.org}{\(F\)-riferito} ad \(Y_{i}\) (per \(i=1,2\)).
\section{Proposizione}
\label{sec:orge81736f}
Si ha che il \href{20250115151027-bracket_di_campi_vettoriali.org}{bracket} \([X_{1},X_{2}]\) è \href{20250115125404-campi_vettoriali_f_riferiti.org}{\(F\)-riferito} al \href{20250115151027-bracket_di_campi_vettoriali.org}{bracket} \([Y_{1},Y_{2}]\).
In particolare, se \(F:M\longrightarrow N\) è un \href{20250113172924-diffeomorfismo_tra_varieta_differenziabili.org}{diffeomorfismo}, allora
\begin{equation*}
F_{\star}\left([X_{1},X_{2}]\right) = \left[F_{\star}(X_{1}),F_{\star}X_{2}\right]
\end{equation*}
(vedi \href{20250115124556-push_forward_di_campi_vettoriali.org}{Push-Forward di campi vettoriali})
\subsection{{\bfseries\sffamily TODO} Dimostrazione\hfill{}\textsc{matematica\_lm:geo\_diff}}
\label{sec:org0446ba8}
\end{document}
