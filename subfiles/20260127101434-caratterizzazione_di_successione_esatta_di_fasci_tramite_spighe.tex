% Intended LaTeX compiler: pdflatex
\documentclass[../main]{subfiles}


\begin{document}

\section{Caratterizzazione di successione esatta di fasci tramite spighe}
\label{sec:org7796d92}
Sia \(X\) uno \href{20250103145124-topologia.org}{spazio topologico}, e sia
\begin{equation*}
\begin{tikzcd}[ampersand replacement=\&]
	{\mathcal{F}} \& {\mathcal{G}} \& {\mathcal{H}}
	\arrow["\varphi", from=1-1, to=1-2]
	\arrow["\psi", from=1-2, to=1-3]
\end{tikzcd}
\end{equation*}
una \href{20250327122142-successione_di_una_categoria.org}{successione} di \href{20250325180613-morfismo_di_prefasci.org}{morfismi} di \href{20250325180613-morfismo_di_prefasci.org}{fasci} di \href{20241205141146-gruppo_abeliano.org}{gruppi}.

\begin{prop}
La successione è \href{20250327150404-successione_di_fasci_esatta.org}{esatta in \(\mathcal{G}\)} se e solo se, per ogni \(p \in X\):
\begin{equation*}
\begin{tikzcd}[ampersand replacement=\&]
	{\mathcal{F}_p} \& {\mathcal{G}_p} \& {\mathcal{H}_p}
	\arrow["{\varphi_p}", from=1-1, to=1-2]
	\arrow["{\psi_p}", from=1-2, to=1-3]
\end{tikzcd}
\end{equation*}
la successione delle \href{20250325183434-spiga_di_un_prefascio.org}{spighe}\footnote{\(\varphi_{p}\) e \(\psi_{p}\) sono \href{20250327114817-morfismo_di_fasci_induce_omomorfismo_tra_spighe.org}{i morfismi indotti}} è una \href{20250327122142-successione_di_una_categoria.org}{successione di gruppi}\footnote{Si ha che \href{20250325183434-spiga_di_un_prefascio.org}{ciascuna spiga è un gruppo}.} \href{20260127101935-successione_di_morfismi_esatta.org}{esatta} in \(\mathcal{G}_{p}\).
\end{prop}

\begin{proof}
(\(\Rightarrow\)): Lasciata per esercizio.

(\(\Leftarrow\)): Si utilizza la \href{20250327150529-caratterizzazione_di_successione_esatta_di_fasci_tramite_sezioni.org}{caratterizzazione dell'esattezza sulle sezioni}. Sia \(U \subseteq X\) aperto:
\begin{equation*}
\begin{tikzcd}[ampersand replacement=\&]
	{\mathcal{F}(U)} \& {\mathcal{G}(U)} \& {\mathcal{H}(U)}
	\arrow["{\varphi_U}", from=1-1, to=1-2]
	\arrow["{\psi_U}", from=1-2, to=1-3]
\end{tikzcd}
\end{equation*}
\begin{enumerate}
\item \uline{Mostriamo che \(\psi_{U}\circ\varphi_{U} \equiv 0\)}.

Sia \(f \in \mathcal{F}(U)\). Allora
\begin{equation*}
 \psi_{U}\varphi_{U} (f) \in \mathcal{H}(U)
\end{equation*}
e pertanto, per ogni \(p \in U\):
\begin{equation*}
 \mathcal{H}_{p}\ni\big[\psi_{U}\varphi_{U} (f)\big]_{p} = \psi_{p} \varphi_{p} [f]_{p} = 0
\end{equation*}
dove l'ultima uguaglianza è per ipotesi.

\href{20250325183434-spiga_di_un_prefascio.org}{Allora}, siccome \(\mathcal{H}\) è un \href{20250324174728-fascio.org}{fascio}, si ha
\begin{equation*}
 \psi_{U}\varphi_{U}(f)=0.
\end{equation*}

\item Sia \(g \in \ker\psi_{U} \subseteq \mathcal{G}(U)\), e sia \(p \in U\):
\begin{equation*}
 [g]_{p} \in \mathcal{G}_{p}.
\end{equation*}
Allora \(\psi_{p}[g]_{p} = [\psi_{U}(g)]_{p} = 0\) \href{20250325183434-spiga_di_un_prefascio.org}{in quanto \(\psi_{U}(g) = 0\)}.

Pertanto \([g]_{p} \in \ker \psi_{p} = \operatorname{Im} \varphi_{p}\), e quindi esiste \(s \in \mathcal{F}_{p}\) tale che
\begin{equation*}
 [g]_{p} = \psi_{p}(s)
\end{equation*}
ovvero esiste \(W \subseteq U\) intorno aperto di \(p\), esiste \(f \in \mathcal{F}(W)\) tale che
\begin{equation*}
 s=[f]_{p} \IMPLICA [g]_{p} = \varphi_{p}[f]_{p} = [\varphi_{W}(f)]_{p}
\end{equation*}
Quindi esiste \(W' \subseteq W\) intorno aperto di \(p\) tale che
\begin{equation*}
 \restriction{g}{W'} = \restriction{\varphi_{W}(f)}{W'} = \varphi_{W'}(f)
\end{equation*}
e quindi \(\restriction{g}{W'} \in \operatorname{Im} \varphi_{W'}\).\qedhere
\end{enumerate}
\end{proof}
\end{document}
