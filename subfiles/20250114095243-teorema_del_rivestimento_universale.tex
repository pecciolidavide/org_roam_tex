% Intended LaTeX compiler: pdflatex
\documentclass[../main]{subfiles}

\usepackage[hyperref]{biblatex}
\date{}
\title{}
\begin{document}

\section{Teorema del rivestimento universale}
\label{sec:org77d6e50}
\subsection{Teorema}
\label{sec:orgfad2d75}
Sia \(X\) uno \href{20250103145124-topologia.org}{spazio topologico} \href{20250109155715-spazio_topologico_di_hausdorff.org}{T2}, \href{20250111142303-spazio_topologico_a_base_numerabile.org}{a base numerabile}, \href{20250103165325-spazio_topologico_connesso.org}{connesso} e \href{20250113102837-spazio_topologico_localmente_connesso_per_archi.org}{localmente cpa}.

Allora esiste un unico (a meno di \href{20241128162125-isomorfismo.org}{isomorfismi}) \href{20250113175231-rivestimento.org}{rivestimento} \((\tilde{X}, \pi)\) di \(X\), \href{20250103165325-spazio_topologico_connesso.org}{connesso}, tale che \(\Pi_{1}(\tilde{X}) = \set{0}\) (vedi \href{20241204223712-gruppo_fondamentale.org}{Gruppo-Fondamentale}). Questo si dice \textbf{rivestimento universale} di \(X\).
Inoltre, se \(\Pi_{1}(X)\) è finito, allora \(\lvert \Pi_{1}(X)\rvert\) è la \href{20241213101756-cardinalita.org}{cardinalità} delle fibre di \(X\).
\end{document}
