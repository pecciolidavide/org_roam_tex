% Intended LaTeX compiler: pdflatex
\documentclass[../main]{subfiles}


\begin{document}

\section{Morfismo tra varietà algebriche QP}
\label{sec:orga7053f1}
Sia \(\K\) un \href{20241231112713-campo_algebricamente_chiuso.org}{campo algebricamente chiuso} e siano \(X,Y\) \href{20250107112123-varieta_algebrica_quasi_proiettiva_qp.org}{varietà QP}.
\subsection{Definizione}
\label{sec:orgc15f4f6}
La funzione \(F: X\longrightarrow Y\) è un morfismo di varietà qp se, \(\forall\, p \in X\):
\begin{itemize}
\item esiste \(U \subseteq X\) \href{20250103145124-topologia.org}{aperto} con \(p \in U\);
\item esistono \(F_{0},\dots,F_{m}\) \href{20241231121125-polinomi_omogenei.org}{polinomi omogenei} dello stesso \href{20241231124742-grado_polinomi.org}{grado} senza zeri comuni su \(U\)
\end{itemize}
tali che \(\restriction{F}{U}=\left(F_{0}(x),\dots,F_{m}(x)\right)\).
\subsubsection{Osservazione}
\label{sec:org203ae1b}
È possibile dimostrare che quando le varietà QP coinvolte sono \href{20241231114256-varieta_algebrica_affine.org}{varietà affini}, allora la definizione di morfismo QP è equivalente a quella di \href{20250104110524-morfismo_tra_varieta_algebriche_affini.org}{morfismo tra varietà affini}.
\subsection{Isomorfismo tra varietà algebriche QP}
\label{sec:org84036e3}
\subsubsection{Definizione}
\label{sec:orgce9f790}
Un morfismo \(F: X \longrightarrow Y\) è un \href{20241128162125-isomorfismo.org}{isomorfismo} se
\begin{itemize}
\item \(F\) è \href{20250104111707-funzione_biunivoca.org}{biunivoca} (e quindi esiste \(F^{-1}\));
\item \(F^{-1}\) è un morfismo.
\end{itemize}
\end{document}
