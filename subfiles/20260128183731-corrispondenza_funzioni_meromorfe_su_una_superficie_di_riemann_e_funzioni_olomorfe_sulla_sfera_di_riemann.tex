% Intended LaTeX compiler: pdflatex
\documentclass[../main]{subfiles}


\begin{document}

\section{Corrispondenza funzioni meromorfe su una superficie di Riemann e funzioni olomorfe sulla Sfera di Riemann}
\label{sec:orgcc983d8}
Sia \(X\) una \href{20260127112828-superficie_di_riemann.org}{superficie di Riemann}, sia \(\C_{\infty}\) la \href{20260127112905-sfera_di_riemann.org}{Sfera di Riemann}, e sia \(\mathds{P}^{1}_{\C}\) il \href{20241231115051-spazio_proiettivo.org}{piano proiettivo} \href{20260127112924-esempi_fondamentali_di_varieta_complesse.org}{complesso}, \href{20260128144717-isomorfismo_tra_superfici_di_riemann.org}{biolomorfo} a \(\C_{\infty}\)\footnote{Vedi ``\href{20260128181135-sfera_di_riemann_biolomorfa_al_piano_proiettivo_complesso.org}{Sfera di Riemann biolomorfa al piano proiettivo complesso}''}.

\begin{prop}
C'è una corrispondenza biunivoca tra:
\begin{itemize}
\item le \href{20260128144105-funzione_meromorfa_su_una_superficie_di_riemann.org}{funzioni meromorfe su \(X\) a valori in \(\C\)}: \(\mathcal{M}_{X}(X)\)
\item le \href{20260128143822-funzione_olomorfa_su_una_superficie_di_riemann.org}{funzioni olomorfe} \(F:X\to \C_{\infty} \cong \mathds{P}^{1}_{\C}\) non costanti in \(\infty\).
\end{itemize}
\begin{equation*}
\set{\parbox{10em}{\centering%
	funzioni meromorfe \(f\) su \(X\) a valori in \(\C\)%
}} \xleftrightarrow{\hspace{1em} 1:1 \hspace{1em}} %
\set{\parbox{10em}{\centering%
	mappe olomorfe \(F:X\to \C_{\infty}\cong \mathds{P}^{1}_{\C}\) non costanti in \(\infty\)
}}.
\end{equation*}
\end{prop}
\begin{proof}
(\(\longrightarrow\)): Sia \(f \in \mathcal{M}_{X}(X)\): si definisce
\begin{align*}
F: X &\longrightarrow \C_{\infty}\\
x &\longmapsto \begin{cases}%
f(x) \in \C & \text{\(x\) non è un polo}\\%
\infty & \text{\(x\) è un polo}.%
\end{cases}
\end{align*}
o, equivalentemente,
\begin{align*}
F: X &\longrightarrow \mathds{P}^{1}_{\C}\\
x &\longmapsto \begin{cases}%
(f(x):1) \in \C & \text{\(x\) non è un polo}\\%
(1:0) & \text{\(x\) è un polo}.%
\end{cases}
\end{align*}

Verifichiamo che \(F\) è \href{20260128143822-funzione_olomorfa_su_una_superficie_di_riemann.org}{olomorfa}:
\begin{itemize}
\item Se \(x_{0} \in X\) non è un polo per \(f\), allora \(F(x_{0}) \in U_{1} = \set{z_{1}\neq 0} \in \mathds{P}^{1}_{\C}\), quindi componendo le carte:
\begin{equation*}
\begin{tikzcd}[ampersand replacement=\&,row sep=scriptsize]
        X \&\& {U_1 \subseteq \mathds{P}_{\C}^1} \&\& \C \\
        {x_0} \&\& {F(x_0) = (f(x)\duepunti 1)} \&\& {f(x)}
        \arrow["F", from=1-1, to=1-3]
        \arrow["{(x\duepunti y)\mapsto \frac{x}{y}}", from=1-3, to=1-5]
        \arrow[maps to, from=2-1, to=2-3]
        \arrow[maps to, from=2-3, to=2-5]
\end{tikzcd}
\end{equation*}
e si ottiene un mappa olomorfa per ipotesi.
\item Se \(x_{0} \in X\) è un polo per \(f\), allora in un intorno \(U_{x_{0}} \subseteq X\) di \(x_{0}\)\footnote{Siccome \(f\) è \href{20260128144105-funzione_meromorfa_su_una_superficie_di_riemann.org}{meromorfa}, allora l'insieme delle singolarità e degli zeri di \(f\) è \href{20260128123515-sottoinsieme_discreto.org}{discreto}, e pertanto è possibile richiedere che dentro ad \(U_{x_{0}}\) non ci siano altri zeri di \(f\).} si ha che
\begin{equation*}
  \frac{1}{f(x)} \eqqcolon g(x)
\end{equation*}
è olomorfa e nulla in \(x_{0}\)\footnote{Questo segue dalle proprietà dell'\href{20260128144433-ordine_di_una_funzione_meromorfa.org}{ordine delle funzioni meromorfe}.}. Quindi, per ogni \(x \in U_{x_{0}} \setminus\set{x_{0}}\):
\begin{equation*}
  F(x) = (f(x):1) = \left(\frac{1}{g(x)} : 1\right) = (1:g(x))
\end{equation*}
mentre per \(x=x_{0}\):
\begin{equation*}
  F(x) = (1:0) = (1:g(x_{0}))
\end{equation*}

Quindi, per ogni \(x \in U_{x_{0}}\), \(F(x) \in U_{0} \coloneqq \set{z_{0}\neq 0} \subseteq \mathds{P}^{1}_{\C}\), e si ottiene:
\begin{equation*}
\begin{tikzcd}[ampersand replacement=\&,row sep=scriptsize]
        {U_{x_0}\subseteq X} \&\& {U_0 \subseteq \mathds{P}_{\C}^1} \&\& \C \\
        x \&\& {F(x) = \big(1\duepunti g(x)\big)} \&\& {g(x)}
        \arrow["F", from=1-1, to=1-3]
        \arrow["{(x\duepunti y)\mapsto \frac{y}{x}}", from=1-3, to=1-5]
        \arrow[maps to, from=2-1, to=2-3]
        \arrow[maps to, from=2-3, to=2-5]
\end{tikzcd}
\end{equation*}
che è una funzione olomorfa.
\end{itemize}

Quindi \(F\) è olomorfa, ed è la funzione associata ad \(f \in \mathcal{M}_{X}(X)\).

(\(\longleftarrow\)): Sia \(F: X\to \mathds{P}^{1}_{\C}\) una funzione olomorfa non costante in \((1:0)\).

Sia \(S\coloneqq F^{-1}\left((1:0)\right)\): per il \href{20260128144016-principio_di_identita_per_funzioni_olomorfe_su_superfici_di_riemann.org}{principio di identità}, questo è un sottoinsieme discreto.

Inoltre, l'\href{20250202190147-immagine_punto_a_punto_di_due_classi.org}{immagine} \(F(X\setminus S) \subseteq U_{1} \coloneqq \set{z_{1}\neq 0} \subseteq \mathds{P}^{1}_{\C}\): quindi si definisce \(f: X\setminus S\to \C\) come segue:
\begin{equation*}
\begin{tikzcd}[ampersand replacement=\&,row sep=scriptsize]
	{X\setminus S} \&\& {U_1 \subseteq \mathds{P}_{\C}^1} \&\& \C
	\arrow["F", from=1-1, to=1-3]
	\arrow["f"', bend right=24pt, from=1-1, to=1-5]
	\arrow["{(z_0\duepunti z_1)\mapsto \frac{z_0}{z_1}}", from=1-3, to=1-5]
\end{tikzcd}
\end{equation*}

In particolare si ha che
\begin{equation*}
\restriction{F}{X\setminus S} = \big(f(x):1\big).
\end{equation*}

Verificare che:
\begin{enumerate}
\item \(f \in \mathcal{O}(X\setminus S)\);
\item \(f \in \mathcal{M}_{X}(X)\).
\end{enumerate}
\end{proof}

\begin{oss}
La biiezione si restringe a
\begin{equation*}
\set{\parbox{10em}{\centering%
	funzioni meromorfe \(f\) su \(X\) a valori in \(\C\) non costanti
}} \xleftrightarrow{\hspace{1em} 1:1 \hspace{1em}} %
\set{\parbox{10em}{\centering%
	mappe olomorfe \(F:X\to \C_{\infty}\cong \mathds{P}^{1}_{\C}\) non costanti
}}.
\end{equation*}
\end{oss}
\subsection{Esistenza funzioni meromorfe a valori complessi per superfici di Riemann compatte}
\label{sec:org01fc4c1}
\begin{prop}
Se \(X\) è una \href{20260127112828-superficie_di_riemann.org}{superficie di Riemann} \href{20250103163701-spazio_topologico_compatto.org}{compatta}, sono fatti equivalenti:
\begin{enumerate}
\item esiste \(f \in\mathcal{M}_{X}(M)\) \href{20260128144105-funzione_meromorfa_su_una_superficie_di_riemann.org}{funzione meromorfa} \textbf{non costante};
\item esiste \(F: X \to \mathds{P}^{1}_{\C}\) \href{20260128143822-funzione_olomorfa_su_una_superficie_di_riemann.org}{funzione olomorfa} e \href{20241213105600-funzione_suriettiva.org}{suriettiva}.
\end{enumerate}
\end{prop}

\begin{proof}
Questo segue banalmente da:
\begin{itemize}
\item \hyperref[sec:orgcc983d8]{Corrispondenza funzioni meromorfe su una superficie di Riemann e funzioni olomorfe sulla Sfera di Riemann}
\item \href{20260128183856-funzione_olomorfa_tra_superfici_di_riemann_condominio_compatto.org}{Proprietà funzione olomorfa non costante tra superfici di Riemann con dominio compatto}
\end{itemize}
\end{proof}
\end{document}
