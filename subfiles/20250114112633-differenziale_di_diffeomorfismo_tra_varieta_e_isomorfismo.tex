% Intended LaTeX compiler: pdflatex
\documentclass[../main]{subfiles}


\begin{document}

\section{Proposizione}
\label{sec:org56f1dfc}
Siano \(M,N\) due \href{20250113115909-struttura_differenziabile.org}{varietà differenziabili}, e sia \(F:M\longrightarrow N\) un \href{20250113172924-diffeomorfismo_tra_varieta_differenziabili.org}{diffeomorfismo}.
Allora per ogni \(p \in M\) il \href{20250114111331-differenziale_di_una_funzione_tra_varieta_differenziabili.org}{differenziale} (vedi \href{20250114102823-spazio_tangente_ad_un_punto_di_una_varieta_differenziabile.org}{Spazio tangente ad un punto di una varietà differenziabile})
\begin{equation*}
\restriction{F_{\star}}{p}: \operatorname{T}_{p}M\longrightarrow T_{F(p)}N
\end{equation*}
è un \href{20250113125833-isomorfismo_tra_spazi_vettoriali.org}{isomorfismo di spazi vettoriali}.
\subsection{{\bfseries\sffamily TODO} Dimostrazione\hfill{}\textsc{matematica\_lm:geo\_diff}}
\label{sec:orgec355bd}
\end{document}
