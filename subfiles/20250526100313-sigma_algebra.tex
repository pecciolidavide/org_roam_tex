% Intended LaTeX compiler: pdflatex
\documentclass[../main]{subfiles}


\begin{document}

\section{sigma-algebra}
\label{sec:org2b9eeba}
\begin{definizione}
Sia \(\Omega\) un \href{20250130104331-insieme_mk.org}{insieme}. Una \textbf{\textbf{\(\sigma\)-algebra}} \(\mathcal{F}\) su \(\Omega\) è una collezione di \href{20250131155822-operazioni_insiemistiche_tra_classi_mk.org}{sottoinsiemi} di \(\Omega\) che soddisfa le seguenti proprietà:
\begin{enumerate}
\item \(\Omega \in \mathcal{F}\)
\item Se \(A \in \mathcal{F}\), allora anche \(\Omega\setminus A \in \mathcal{F}\)\footnote{Vedi ``\href{20250131155822-operazioni_insiemistiche_tra_classi_mk.org}{Sottrazione insiemistica}''}
\item Se \(\{A_n\}_{n \in \mathbb{N}} \subseteq \mathcal{F}\), allora anche
\end{enumerate}
\begin{equation*}
\bigcup_{n=1}^{\infty} A_n \in \mathcal{F}
\end{equation*}

Dunque, una \(\sigma\)-algebra è una famiglia di insiemi chiusa per complementazione e per unioni numerabili, e che contiene l’insieme base \(\Omega\).
\end{definizione}

\begin{prop}
Sia \(\mathcal{F}\) una \(\sigma\)-algebra su uno spazio \(\Omega\). Allora valgono le seguenti proprietà fondamentali:

\begin{enumerate}
\item \(\emptyset \in \mathcal{F}\)

\emph{Dimostrazione:} segue da \(\Omega \in \mathcal{F}\) e dalla chiusura per complementi, poiché
\begin{equation*}
\emptyset = \Omega\setminus\Omega \in \mathcal{F}
\end{equation*}

\item \(\mathcal{F}\) è chiusa per intersezioni numerabili:
se \(A_n \in \mathcal{F}\) per ogni \(n \in \mathbb{N}\), allora
\begin{equation*}
\bigcap_{n=1}^{\infty} A_n \in \mathcal{F}
\end{equation*}

\emph{Dimostrazione:} segue dalla chiusura per complementi e per unioni numerabili, usando \href{20250714161943-leggi_di_de_morgan.org}{De Morgan}.

\item \(\mathcal{F}\) è chiusa per differenza:
se \(A, B \in \mathcal{F}\), allora
\begin{equation*}
A \setminus B = A \cap (\Omega\setminus B) \in \mathcal{F}
\end{equation*}

\item \(\mathcal{F}\) è chiusa per unioni finite e intersezioni finite:
se \(A, B \in \mathcal{F}\), allora \(A \cup B \in \mathcal{F}\) e \(A \cap B \in \mathcal{F}\)
\end{enumerate}
\end{prop}
\end{document}
