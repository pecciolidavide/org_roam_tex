% Intended LaTeX compiler: pdflatex
\documentclass[../main]{subfiles}


\begin{document}

\section{SEC di Complessi di Catene}
\label{sec:org032b4dd}
Sia \(R\) un \href{20241205141119-anello.org}{anello} commutativo con unità. Siano \(\mathcal{C}_{\bullet}, \mathcal{C}_{\bullet}', \mathcal{C}_{\bullet}''\) dei \href{20250120163114-complesso_di_catene.org}{complessi di catene} e \(f_{\bullet}: \mathcal{C}_{\bullet}'\longrightarrow \mathcal{C}_{\bullet}\), \(g_{\bullet}: \mathcal{C}_{\bullet}\longrightarrow \mathcal{C}_{\bullet}''\)  \href{20250120163759-categoria_complessi_di_catene.org}{morfismi}.

Una \textbf{successione esatta corta} di complessi di catene è
\begin{equation*}
\begin{tikzcd}[ampersand replacement=\&,column sep=large,row sep=tiny]
	0 \& {\mathcal{C}_{\bullet}'} \& {\mathcal{C}_{\bullet}} \& {\mathcal{C}_{\bullet}''} \& 0
	\arrow[from=1-1, to=1-2]
	\arrow["{f_{\bullet}}", from=1-2, to=1-3]
	\arrow["{g_{\bullet}}", from=1-3, to=1-4]
	\arrow[from=1-4, to=1-5]
\end{tikzcd}
\end{equation*}
tale che, se
\begin{align*}
\mathcal{C}_{\bullet} &= \set{(C_{n},\partial)}_{z \in \Z}\\
\mathcal{C}_{\bullet}' &= \set{(C_{n}',\partial')}_{z \in \Z}\\
\mathcal{C}_{\bullet}'' &= \set{(C_{n}'',\partial'')}_{z \in \Z}
\end{align*}
e i morfismi sono
\begin{align*}
f_{\bullet} &= \set{f_{n}:C_{n}'\longrightarrow C_{n}}_{z \in \Z}\\
g_{\bullet} &= \set{g_{n}:C_{n}\longrightarrow C_{n}''}_{z \in \Z}
\end{align*}
allora per ogni \(n \in \Z\) la seguente è una \href{20250120131527-sec.org}{SEC} di \(R\)-moduli:
\begin{equation*}
\begin{tikzcd}[ampersand replacement=\&,column sep=large,row sep=tiny]
	0 \& {C_n'} \& {C_n} \& {C_n''} \& 0
	\arrow[from=1-1, to=1-2]
	\arrow["{f_n}", from=1-2, to=1-3]
	\arrow["{g_n}", from=1-3, to=1-4]
	\arrow[from=1-4, to=1-5]
\end{tikzcd}
\end{equation*}
ovvero se il \href{https://q.uiver.app/\#q=WzAsMjEsWzAsMiwiMCJdLFsxLDIsIkNfbiciXSxbMiwyLCJDX24iXSxbMywyLCJDX24nJyJdLFs0LDIsIjAiXSxbMSwxLCJDX3tuKzF9JyJdLFsyLDEsIkNfe24rMX0iXSxbMywxLCJDX3tuKzF9JyciXSxbMCwxLCIwIl0sWzQsMSwiMCJdLFsxLDMsIkNfe24tMX0nIl0sWzIsMywiQ197bi0xfSJdLFszLDMsIkNfe24tMX0nJyJdLFs0LDMsIjAiXSxbMCwzLCIwIl0sWzEsMCwiXFxjZG90cyJdLFsyLDAsIlxcY2RvdHMiXSxbMywwLCJcXGNkb3RzIl0sWzEsNCwiXFxjZG90cyJdLFsyLDQsIlxcY2RvdHMiXSxbMyw0LCJcXGNkb3RzIl0sWzAsMV0sWzEsMiwiZl9uIl0sWzIsMywiZ19uIl0sWzMsNF0sWzE0LDEwXSxbMTAsMTEsImZfe24tMX0iXSxbMTEsMTIsImdfe24tMX0iXSxbMTIsMTNdLFs4LDVdLFs1LDYsImZfe24rMX0iXSxbNiw3LCJnX3tuKzF9Il0sWzcsOV0sWzE1LDVdLFsxNyw3XSxbNSwxLCJcXHBhcnRpYWxfe24rMX0nIiwyXSxbNiwyLCJcXHBhcnRpYWxfe24rMX0iLDJdLFs3LDMsIlxccGFydGlhbF97bisxfScnIl0sWzEsMTAsIlxccGFydGlhbF9uJyIsMl0sWzIsMTEsIlxccGFydGlhbF9uIiwyXSxbMywxMiwiXFxwYXJ0aWFsX24nJyJdLFsxMCwxOCwiXFxwYXJ0aWFsX3tuLTF9IiwyXSxbMTEsMTksIlxccGFydGlhbF97bi0xfSIsMl0sWzEyLDIwLCJcXHBhcnRpYWxfe24tMX0nJyJdLFsxNiw2XV0=}{seguente} diagramma commutativo è a righe \href{20250120125004-successione_di_r_moduli_esatta.org}{esatte}:
\begin{equation*}
\begin{tikzcd}[ampersand replacement=\&,sep=large]
	\& \cdots \& \cdots \& \cdots \\
	0 \& {C_{n+1}'} \& {C_{n+1}} \& {C_{n+1}''} \& 0 \\
	0 \& {C_n'} \& {C_n} \& {C_n''} \& 0 \\
	0 \& {C_{n-1}'} \& {C_{n-1}} \& {C_{n-1}''} \& 0 \\
	\& \cdots \& \cdots \& \cdots
	\arrow[from=1-2, to=2-2]
	\arrow[from=1-3, to=2-3]
	\arrow[from=1-4, to=2-4]
	\arrow[from=2-1, to=2-2]
	\arrow["{f_{n+1}}", from=2-2, to=2-3]
	\arrow["{\partial_{n+1}'}"', from=2-2, to=3-2]
	\arrow["{g_{n+1}}", from=2-3, to=2-4]
	\arrow["{\partial_{n+1}}"', from=2-3, to=3-3]
	\arrow[from=2-4, to=2-5]
	\arrow["{\partial_{n+1}''}", from=2-4, to=3-4]
	\arrow[from=3-1, to=3-2]
	\arrow["{f_n}", from=3-2, to=3-3]
	\arrow["{\partial_n'}"', from=3-2, to=4-2]
	\arrow["{g_n}", from=3-3, to=3-4]
	\arrow["{\partial_n}"', from=3-3, to=4-3]
	\arrow[from=3-4, to=3-5]
	\arrow["{\partial_n''}", from=3-4, to=4-4]
	\arrow[from=4-1, to=4-2]
	\arrow["{f_{n-1}}", from=4-2, to=4-3]
	\arrow["{\partial_{n-1}'}"', from=4-2, to=5-2]
	\arrow["{g_{n-1}}", from=4-3, to=4-4]
	\arrow["{\partial_{n-1}}"', from=4-3, to=5-3]
	\arrow[from=4-4, to=4-5]
	\arrow["{\partial_{n-1}''}", from=4-4, to=5-4]
\end{tikzcd}
\end{equation*}

Generalizzando:

\begin{definizione}
Una \href{20250327122142-successione_di_una_categoria.org}{successione di complessi di catene e morfismi di catene}
\begin{equation*}
\dots \longrightarrow \mathcal{C}_{\bullet}^{i-1} \xrightarrow{f^{i-1}_{\bullet}} \mathcal{C}_{\bullet}^{i} \xrightarrow{f^{i}_{\bullet}} \mathcal{C}_{\bullet}^{i+1} \longrightarrow \dots
\end{equation*}
si dice \textbf{\textbf{esatta}} se per ogni \(n \in \Z\) la successione indotta sui moduli componenti è \href{20250120125004-successione_di_r_moduli_esatta.org}{esatta}, ovvero se:\footnote{Vedi ``\href{20250202190147-immagine_punto_a_punto_di_due_classi.org}{Immagine e retroimmagine tramite una funzione}'' e ``\href{20241213105201-kernel.org}{Kernel}''}
\begin{equation*}
\operatorname{Im}(f^{i-1}_{n}) = \ker(f^{i}_{n}) \quad \forall n \in \Z
\end{equation*}
Equivalentemente, la successione è esatta se \(\operatorname{Im}(f^{i-1}_{\bullet}) = \ker(f^{i}_{\bullet})\) come \href{20250128151459-sottocomplesso_di_catene.org}{sottocomplessi}.
\end{definizione}
\end{document}
