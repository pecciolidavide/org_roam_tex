% Intended LaTeX compiler: pdflatex
\documentclass[../main]{subfiles}


\begin{document}

\section{Teorema}
\label{sec:org009e45d}
Sia \(A\) un \href{20250102115942-anello_noetheriano.org}{Anello Noetheriano}. Allora l'\href{20241219113434-anello_dei_polinomi.org}{Anello-dei-polinomi} ad una indeterminata \(A[X]\) è un \href{20250102115942-anello_noetheriano.org}{Anello Noetheriano}.
\subsection{dim.}
\label{sec:orge091e1b}
Per assurdo, sia \(I \subseteq A[X]\) un ideale non finitamente generato.
Sia \(f_{1}\in I\) il polinomio di grado \textbf{minimo} tale che \((f_{1}) \neq I\). Si denoti con \(d_{1}=\deg f_{1}\).
Sia \(f_{2}\in I\setminus(f_{1})\) il polinomio di grado minimo tale che \((f_{1},f_{2})\neq I\). Si denoti con \(d_{2}=\deg f_{2}\).


Si costruisce in questo modo la sequenza \(f_{1},f_{2},\dots\), ed in particolare si ha che
\begin{equation*}
d_{1}\le d_{2}\le d_{3}\le \dots
\end{equation*}
Inoltre, \(\forall\, i\) si ha che \(f_{i}\notin(f_{1},\dots,f_{i-1})\).


Sia ora \(a_{i}\) il \href{20241231112750-polinomio.org}{coefficiente direttore} di \(f_{i}\), e si consideri l'ideale
\begin{equation*}
J=(a_{1},a_{2},a_{3},\dots) \subseteq A
\end{equation*}
Poiché \(A\) è un \href{20250102115942-anello_noetheriano.org}{anello noetheriano} e la seguente è una \href{20250102120836-catena.org}{Catena} ascendente
\begin{equation*}
(a_{1}) \subseteq (a_{1},a_{2}) \subseteq (a_{1},a_{2},a_{3}) \subseteq \dots
\end{equation*}
allora esiste \(N\) tale che
\begin{equation*}
J=(a_{1},a_{2},\dots,a_{N})
\end{equation*}


Siccome \(a_{N+1}\in J\), si ha che
\begin{equation*}
a_{N+1}=\sum_{i=1}^{N}\lambda_{i}a_{i}
\end{equation*}
per alcuni \(\lambda_{i}\in A\).


Sia ora
\begin{align*}
g&=\sum_{i=1}^{N} \lambda_{i}f_{i} \cdot x^{d_{N+1}-d_{i}}\\
&= \sum_{i=1}^{N} \lambda_{i}(a_{i}x_{i}^{d_{i}} + \dots) \cdot x^{d_{N+1}-d_{i}}\\
&= \sum_{i=1}^{n} \lambda_{i} a_{i} x^{d_{N+1}} + \dots\\
&= a_{N+1}x^{d_{N+1}} + \dots
\end{align*}
dunque \(g\) è un polinomio di grado \(d_{N+1}\), e, per costruzione, \(g \in (f_{1},\dots,f_{N})\). Inoltre, per costruzione, \$ f\textsubscript{N+1}\(\notin\) (f\textsubscript{1},\dots{},f\textsubscript{N}) \$ ma coefficiente direttore \$ a\textsubscript{N+1} \$, come \$ g \$.

Il polinomio \$ f\textsubscript{N+1}-g \$ ha grado strettamente minore di \$ d\textsubscript{N+1} \$ e non appartiene a \$ (f\textsubscript{1},\dots{},f\textsubscript{N}) \$.\footnote{Se per assurdo \$ f\textsubscript{N+1}-g \(\in\) (f\textsubscript{1},\dots{},f\textsubscript{N}) \$, allora
\begin{equation*}
f_{N+1} = (f_{N+1}-g) + g
\end{equation*}
e, siccome gli ideali sono chiusi rispetto alla somma, si ha che \$ f\textsubscript{N+1} \(\in\) (f\textsubscript{1},\dots{},f\textsubscript{N}) \$. Assurdo.}
Pertanto il polinomio scelto \$ f\textsubscript{N+1} \$ non è quello con grado minimo a soddisfare le richieste. Assurdo.
\section{Corollario}
\label{sec:org8ce04f8}
Sia \(A\) un \href{20250102115942-anello_noetheriano.org}{Anello Noetheriano}. Allora l'\href{20241219113434-anello_dei_polinomi.org}{anello dei polinomi} \(A[x_{1},\dots,x_{n}]\) è noetheriano.
\end{document}
