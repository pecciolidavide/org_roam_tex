% Intended LaTeX compiler: pdflatex
\documentclass[../main]{subfiles}


\begin{document}

\section{Teorema dell'inversa locale}
\label{sec:org82d14f7}
\subsection{Teorema dell'inversa locale}
\label{sec:org6c34e47}

Sia \(\Omega \subseteq \R^{n}\) un \href{20250103145124-topologia.org}{aperto}, e sia \(F:\Omega \longrightarrow \R^{n}\) una funzione di \href{20250113125602-classe_c_di_una_funzione.org}{classe \(C^{\infty}\)}.
Sia \(p \in \Omega\) tale che il \href{20250113125641-differenziale_di_una_funzione_reale.org}{differenziale} di \(F\) in \(p\)
\begin{equation*}
\restriction{F_{\star}}{p}: \R^{n}\longrightarrow \R^{n}
\end{equation*}
sia un \href{20250113125833-isomorfismo_tra_spazi_vettoriali.org}{isomorfismo}.

Allora esistono \(U \subseteq\Omega\) \href{20250111142313-intorno.org}{intorno} \href{20250103145124-topologia.org}{aperto} di \(p\), \(V \subseteq \R^{n}\) intorno aperto di \(F(p)\) tali che
\begin{equation*}
\restriction{F}{U}: U \longrightarrow V
\end{equation*}
sia un \href{20250113103415-diffeomorfismo.org}{diffeomorfismo}.
\end{document}
