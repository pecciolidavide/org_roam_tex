% Intended LaTeX compiler: pdflatex
\documentclass[../main]{subfiles}

\usepackage[hyperref]{biblatex}
\date{}
\title{}
\begin{document}

\section{Esempi di sottoinsiemi pi03 completi}
\label{sec:orge69bca6}
\subsection{Esercizio 3}
\label{sec:org0aac54e}

Prove that the sets
\begin{align*}
C_0 &= c_0 \cap [0,1]^\omega = \left\{(x_n)_{n \in \omega} \in [0,1]^\omega \,\middle|\, x_n \to 0 \right\}\\
C &= \left\{(x_n)_{n \in \omega} \in [0,1]^\omega \,\middle|\, (x_n)_{n \in \omega} \text{ converges} \right\}
\end{align*}
are both \(\bm{\Pi}^0_3\)-complete.

\emph{Hint.} For the hardness part, compare these sets with the \(\bm{\Pi}^0_3\)-complete set \(C_3\) from Exercise 2.1.27 in the notes.
\subsubsection{Soluzione}
\label{sec:org6bf8ee0}

\paragraph{\(C_{0}\) e \(C\) sono degli insiemi \(\mathbf{\Pi}_{0}^{3}\).}
\label{sec:org53520e2}

\begin{enumerate}
\item Insieme \(C_{0}\).
\label{sec:org9a7202a}

Si ha che \((x_{j})_{j \in \omega} \in C_{0}\) se e solo se \((x_{j})_{j \in \omega} \in [0,1]^{\omega}\) e:
\begin{equation*}
	\forall\, \varepsilon \in \Q^{+}\ \exists\,N \in \N \ \forall\, n > N\ \left( |x_{n}|\le\varepsilon\right)
\end{equation*}
ovvero, se \(U_{n, \varepsilon} \coloneqq \set{(x_{j})_{j \in \omega} \in [0,1]^{\omega}: |x_{n}|\le\varepsilon}\), allora
\begin{equation*}
	C_{0} = \bigcap_{\varepsilon \in \Q^{+}} \bigcup_{N \in \N} \bigcap_{n>N} U_{n,\varepsilon}.
\end{equation*}

Quindi, dette \(\pi_{m} : [0,1]^{\omega}\to [0,1]\) le \(m\)-esime proiezioni (continue per definizione di topologia prodotto):
\begin{equation*}
	U_{n,\varepsilon}= \pi_{n}^{-1}\left([-\varepsilon,\varepsilon]\right)
\end{equation*}
e pertanto \(U_{n,\varepsilon}\) è chiuso. Per il Lemma 2.1.5:
\begin{align*}
	\bigcap_{n > N} U_{n,\varepsilon} &\in \bm{\Pi}_{1}^{0}\\
	\bigcup_{N \in \N}\bigcap_{n >N} U_{n,\varepsilon} &\in \bm{\Sigma}_{2}^{0}\\
	C_{0} = \bigcap_{\varepsilon \in \Q^{+}} \bigcup_{N \in \N} \bigcap_{n>N} U_{n,\varepsilon} &\in \bm{\Pi}_{3}^{0}.
\end{align*}
e si ottiene che \(C_{0} \in \bm{\Pi}_{3}^{0}\left([0,1]^{\omega}\right)\).
\item Insieme \(C\).
\label{sec:orgf98d777}

Si ha che \((x_{j})_{j \in\omega} \in C\) se e solo se \((x_{j})_{j \in\omega} \in [0,1]^{\omega}\) e
\begin{equation*}
\forall\, \varepsilon \in \Q^{+}\ \exists\, N \in \N \ \forall\,n,m> N\ (|x_{n}-x_{m}|\le\varepsilon)
\end{equation*}
ovvero, se \(V_{m,n}^{\varepsilon} \coloneqq \set{(x_{j})_{j \in \omega} \in [0,1]^{\omega}: |x_{n}-x_{m}|\le\varepsilon}}\), allora
\begin{equation*}
C = \bigcap_{\varepsilon \in \Q^{+}}\bigcup_{N \in \N}\bigcap_{n,m>N} V_{n,m}^{\varepsilon}.
\end{equation*}

Poiché la funzione \((\pi_{n}-\pi_{m}):[0,1]^{\omega}\to \R\) è continua, allora
\begin{equation*}
V_{n,m}^{\varepsilon} \coloneqq (\pi_{n}-\pi_{m})^{-1}\left([-\varepsilon,\varepsilon]\right)
\end{equation*}
e quindi \(V_{n,m}^{\varepsilon}\) è chiuso. Per il Lemma 2.1.5:
\begin{align*}
\bigcap_{n,m > N} V_{n,m}^{\varepsilon} &\in \bm{\Pi}_{1}^{0}\\
\bigcup_{N \in \N}\bigcap_{n,m > N} V_{n,m}^{\varepsilon} &\in \bm{\Sigma}_{2}^{0}\\
C = \bigcap_{\varepsilon \in \Q^{+}}\bigcup_{N \in \N}\bigcap_{n,m>N} V_{n,m}^{\varepsilon} &\in \bm{\Pi}_{3}^{0}
\end{align*}
e si ottiene che \(C \in \bm{\Pi}_{3}^{0}\left([0,1]^{\omega}\right)\).
\end{enumerate}
\paragraph{Hardness}
\label{sec:org2da9842}

È noto (Esercizio 2.1.27) che l'insieme \(C_{3} \coloneqq \set{x \in \omega^{\omega}\mid \lim_{n\to\infty}x(n) = \infty}\) sia \(\bm{\Pi}_{3}^{0}\)-hard. Pertanto si cercano delle funzioni continue
\begin{equation*}
\begin{tikzcd}[ampersand replacement=\&,cramped]
	{\omega^\omega} \& {[0,1]^\omega} \& {[0,1]^\omega}
	\arrow["F", from=1-1, to=1-2]
	\arrow["G", from=1-2, to=1-3]
\end{tikzcd}
\end{equation*}
tali che
\begin{equation*}
F^{-1}(C_{0}) = C_{3},\qquad G^{-1}(C) = C_{0}.
\end{equation*}
Questo, per mezzo del Lemma 2.1.23, garantisce che \(C_{0},C\) siano insiemi \(\bm{\Pi}_{3}^{0}\)-hard (e quindi, per il punto precedente, completi).

Le due funzioni si definiscono come segue:
\begin{align*}
&\begin{aligned}
F: \omega^{\omega} &\longrightarrow[0,1]^{\omega}\\
(x_{j})_{j \in \omega} &\longmapsto \left(\phi(x_{j})\right)_{j \in \omega}
\end{aligned} & &\text{dove} & &\begin{aligned}
\phi: \N &\longrightarrow [0,1]\\
m &\longmapsto \begin{cases}
1/m & m\neq 0\\
1 & m=0.
\end{cases}
\end{aligned}\\[1.5em]
&\begin{aligned}
G: [0,1]^{\omega} &\longrightarrow [0,1]^{\omega}\\
(x_{j})_{j \in \omega} &\longmapsto (y_{j})_{j \in \omega}
\end{aligned} & &\text{dove} &
&y_{j} \coloneqq \begin{cases}
0 & j\text{ dispari}\\
x_{j/2} & j\text{ pari}.
\end{cases}
\end{align*}
\begin{enumerate}
\item \underline{\(F\) è continua.}
\label{sec:org97823a4}

La funzione \(F\) è continua poiché lo è su ciascuna componente (in quanto \(\N\) ha la topologia discreta).
\item \underline{\(G\) è continua.}
\label{sec:org4f0e3c5}

La funzione \(F\) è continua poiché lo è su ciascuna componente:
\begin{itemize}
\item la componente \(j\)-esima di \(G\), con \(j\) dispari, è data dalla funzione costante nulla, continua;
\item la componente \(j\)-esima di \(G\), con \(j\) pari, è data dalla funzione proiezione \(\pi_{j/2}: [0,1]^{\omega}\to [0,1]\), continua per definizione di topologia prodotto.
\end{itemize}
\item \underline{\(F^{-1}(C_{0})=C_{3}\).}
\label{sec:org0d884e2}

Si dimostra che \(\alpha \in C_{3}\) sse \(F(\alpha) \in C_{0}\).

\begin{itemize}
\item Se \(\alpha = (x_{j})_{j \in \omega}\in C_{3}\) allora esiste \(N \in \N\) tale che, per ogni \(j>N\) si ha\(x_{j}\neq 1\).

Pertanto, per ogni \(j>N\), \(\phi(x_{j}) = 1/x_{j}\) e, siccome \(x_{j}\to \infty\), \(\phi(x_{j})\to 0\). Quindi \(F(\alpha) \in C_{0}\).
\item Viceversa, sia \(\alpha = (x_{j})_{j \in \omega} \notin C_{3}\). Si supponga per assurdo che \((y_{j})_{j \in \omega} = F(\alpha) \in C_{0}\).

Allora, definitivamente, \(y_{j} = 1/x_{j}\) (e in particolare \(x_{j}\neq 0\neq y_{j}\)), poiché altrimenti non si avrebbe convergenza a \(0\). In particolare, \(x_{j} = 1/y_{j}\), definitivamente:
\begin{equation*}
  \lim_{j\to \infty} x_{j} = \lim_{j\to\infty}\frac{1}{y_{j}} = \infty
\end{equation*}
poiché \(y_{j}\to 0\). Quindi \((x_{j})_{j \in \omega} \in C_{3}\). Assurdo.

Si ottiene perciò che \(F(\alpha) \notin C_{0}\).
\end{itemize}
\item \underline{\(G^{-1}(C)=C_{0}\).}
\label{sec:org74e1729}

Si dimostra che \(\alpha \in C_{0}\) sse \(G(\alpha) \in C\).

\begin{itemize}
\item Se \(\alpha = (x_{j})_{j \in \omega} \in C_{0}\) allora la successione \(\beta= (y_{j})_{j \in \omega} \coloneqq G(\alpha)\) converge a \(0\), e pertanto converge: \(G(\alpha) \in C\).
\item Viceversa, se \(\alpha = (x_{j})_{j \in \omega}\notin C_{0}\) , allora la successione \(\beta= (y_{j})_{j \in \omega} \coloneqq G(\alpha)\) non converge, in quanto presenta due sottosuccessioni (\((y_{2j+1})_{j \in \omega}\) e \((y_{2j})_{j \in \omega}\)) con caratteri diversi: \(G(\alpha)\notin C\).\qed
\end{itemize}
\end{enumerate}
\end{document}
