% Intended LaTeX compiler: pdflatex
\documentclass[../main]{subfiles}


\begin{document}

Contesto: \href{20250130104245-morse_kelly_set_theory.org}{Morse Kelly Set Theory}.
\section{Proposizione}
\label{sec:org72785d8}
La classe di tutti gli insiemi contiene infiniti elementi.
\subsection{Dimostrazione}
\label{sec:orgc83e9c4}
Si parta dall'\href{20250131161811-insieme_vuoto_mk.org}{insieme vuoto} \(\emptyset\), e se ne consideri il \href{20250202124648-successore_di_un_insieme_mk.org}{successore}:
\begin{equation*}
\operatorname{S}(\emptyset) = \set{\emptyset}
\end{equation*}
Si può considerare quindi la seguente lista di insiemi distinti:
\begin{align*}
\set{\emptyset} &= \operatorname{S}(\emptyset)\\
\set{\emptyset,\set{\emptyset}} &= \operatorname{S}\left(\set{\emptyset}\right)\\
\set{{\emptyset,\set{\emptyset}},\set{\emptyset,\set{\emptyset}}} &= \operatorname{S}\left(\set{\emptyset,\set{\emptyset}}\right)
\end{align*}
Pertanto, vi sono infiniti insiemi.
\end{document}
