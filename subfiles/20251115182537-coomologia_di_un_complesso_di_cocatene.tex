% Intended LaTeX compiler: pdflatex
\documentclass[../main]{subfiles}


\begin{document}

\section{Coomologia di un complesso di cocatene}
\label{sec:orgcd49c21}
\begin{definizione}
Sia \(A^{\bullet} \coloneqq \left\langle (A^{k}, \operatorname{d}^{k} )\right\rangle_{k \in \Z}\) un \href{20251115182320-complesso_di_cocatene.org}{complesso di cocatene}.

Si definisce il \(k\)-esimo gruppo di coomologia del complesso il \href{20251121143644-quoziente_di_spazi_vettoriali.org}{quoziente}\footnote{Vedi:
\begin{itemize}
\item \href{20250202173528-dominio_range_e_campo_di_una_classe_relazione.org}{Range di una funzione}
\item \href{20251121143525-kernel_di_una_funzione_tra_spazi_vettoriali.org}{Kernel di una funzione tra spazi vettoriali}
\end{itemize}}:
\begin{equation*}
\operatorname{H}^{k}(A^{\bullet}) \coloneqq \frac{\ker \operatorname{d}^{k}}{\operatorname{Im} \mathrm{d}^{k-1}}.
\end{equation*}
Spesso si indica con:
\begin{itemize}
\item \(Z^{k} \coloneqq \ker \mathrm{d}^{k}\) l'\uline{insieme dei \(k\)-coclicli};
\item \(B^{k} \coloneqq \operatorname{Im} \mathrm{d}^{k-1}\) l'\uline{insieme dei \(k\)-cobordi}.
\end{itemize}
\end{definizione}

\begin{oss}
Il quoziente di cui sopra è ben definito in quanto le seguenti inclusioni sono tutti \href{20250114103118-sottospazio_vettoriale.org}{sottospazi vettoriali}:
\begin{equation*}
\operatorname{Im} \mathrm{d}^{k-1} \subseteq \ker \mathrm{d}^{k} \subseteq A^{k}
\end{equation*}
\end{oss}
\begin{oss}
Per ciascun \(k \in \N\), \(\operatorname{H}^{k}(A^{\bullet})\) è uno \href{20241205142027-spazio_vettoriale.org}{spazio vettoriale}.
\end{oss}
\end{document}
