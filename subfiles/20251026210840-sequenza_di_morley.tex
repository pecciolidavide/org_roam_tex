% Intended LaTeX compiler: pdflatex
\documentclass[../main]{subfiles}


\begin{document}

\def\D{\mathcal{D}}
\def\DD{\bm{\D}}
\def\U{\mathcal{U}}
\def\eq{{\rm eq}}
\def\Ueq{\U^\eq}
\def\L{\mathcal{L}}
\def\orbita{\mathcal{O}}
\def\Aut{\operatorname{Aut}}
\def\tc{\mid}
\def\tp{\operatorname{tp}}
\def\<{\langle}
\def\>{\rangle}
\def\b{\bm{b}}

\def\restricted#1{\,\mathord{\upharpoonright}{{\scriptstyle #1}}}
\def\equivalentover#1{\mathrel{\equiv_{ #1 }}}

%% NON FORKING
\def\nonforkSymbol{%
\mathbin{\raise1.8ex%
\rlap{\kern0.6ex\rule{0.6ex}{0.1ex}}%
\rlap{\kern1.1ex\rule{0.1ex}{1.9ex}}\raise-0.3ex\hbox{$\smile$}}}
\def\defaultnonforkmodel{M}
\def\nonfork{\nonforkSymbol}
\renewcommand{\nonfork}[1][\defaultnonforkmodel]{%
\mathrel{\nonforkSymbol_{#1}}}
\section{Sequenza di Morley}
\label{sec:orgd485542}
Si utilizza la \href{20250612143636-notazione_teoria_dei_modelli.org}{Notazione della TEORIA DEI MODELLI}

Sia \(\L\) un \href{20250130162057-linguaggio_del_prim_ordine.org}{linguaggio}, \(T\) una \href{20250130114950-teoria_del_prim_ordine.org}{teoria} \href{20250131123151-teoria_completa.org}{completa} senza \href{20250131122945-modello_di_un_insieme_di_formule.org}{modelli} finiti e \(\U\) un \href{20250617095548-modello_lambda_saturo.org}{modello saturo} di \href{20241213101756-cardinalita.org}{cardinalità} \href{20250211123155-cardinale_limite_forte.org}{inaccessibile} \(\kappa>\card{\L}+ \omega\). \(\U\) è un \href{20250617102733-modello_mostro.org}{modello mostro}.

\begin{definizione}
Sia \(A \subseteq \U\) e sia \(p(x) \in S(\U)\) un \href{20250212164424-tipo_teoria_dei_modelli.org}{tipo} \href{20250617102733-modello_mostro.org}{globale} \href{20251026161851-phi_tipo_invariante_su_un_insieme.org}{invariante su \(A\)}.

Una \uline{\href{20250206170922-sequenze_e_stringhe.org}{sequenza} di Morley} di \(p(x)\) su \(A\) è
\(\overline{c} = \langle c_{i}: i<\alpha\rangle\), con \(c_{i} \in \U^{x}\),
tale che
\begin{equation*}
\forall  i < \alpha: \quad c_{i}\vDash p(x)\restricted{A\cup{\overline{c}\restricted{i}}}.
\end{equation*}
\end{definizione}
\end{document}
