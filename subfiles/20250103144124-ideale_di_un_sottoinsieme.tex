% Intended LaTeX compiler: pdflatex
\documentclass[../main]{subfiles}


\begin{document}

\section{Ideale di un sottoinsieme}
\label{sec:org89dbcb8}
Sia \(\K\) un \href{20241231112713-campo_algebricamente_chiuso.org}{campo algebricamente chiuso}.
\subsection{\href{20241231114256-varieta_algebrica_affine.org}{Varietà affini}.}
\label{sec:org14e9cd8}

Sia \(Y \subseteq \A^{n}\) un sottoinsieme (vedi \href{20241231114009-spazio_affine.org}{Spazio Affine}). Si definisce l'\textbf{ideale di \(Y\)}
\begin{equation*}
I(Y)\coloneqq\set{f \in \K[x_{1},\dots,x_{n}]: f(p)=0,\ \forall\, p \in Y}
\end{equation*}
\subsubsection{Proposizione}
\label{sec:org8655f2d}
\(I(Y)\) è un \href{20241219112955-ideale.org}{Ideale} di \(\K[x_{1},\dots,x_{n}]\) (vedi \href{20241219113434-anello_dei_polinomi.org}{Anello-dei-polinomi}).
\subsubsection{Osservazione:}
\label{sec:org700f09f}
Valgono le seguenti proprietà, per \(Y \subseteq \A^{n}\) e \(J \subseteq \K[x_{1},\dots,x_{n}]\).
1.\(Y \subseteq V\left(I(Y)\right)\) (vedi \href{20241231112823-radici_polinomiali.org}{Luogo di zeri} ), poiché
\begin{equation*}
I(Y)=\set{f \in \K[x_{1},\dots,x_{n}]: f(p)=0\ \forall\, p \in Y}
\end{equation*}
e dunque, se \(y \in Y\), allora per ogni \(f \in I(Y)\) si ha che \(f(y)=0\)
\begin{enumerate}
\item \(J \subseteq I\left(V(J)\right)\) dal momento che se \(f \in J\) allora per ogni \(p \in V(J)\) si ha che \(f(p)=0\) e pertanto \(f \in I\left(V(J)\right)\).
\item Se \(A \subseteq B \subseteq \A^{n}\) allora \(I(B) \subseteq I(A)\). Sia \(f \in I(B)\). Allora per ogni \(b \in B\) si ha che \(f(b)=0\). In particolare, per ogni \(a \in A \subseteq B\) si ha che \(f(a)=0\), e pertanto \(f \in I(A)\).
\end{enumerate}
\subsection{\href{20241231123223-varieta_algebrica_proiettiva.org}{Varietà proiettive}}
\label{sec:orgd17a267}
Sia \(X \subseteq \mathds{P}^{n}\) un sottoinsieme. (Vedi \href{20241231115051-spazio_proiettivo.org}{Spazio Proiettivo}). Si definisce l'ideale di \(X\) l'ideale generato di \(\K[x_{0},\dots,x_{n}]\): (vedi \href{20241231121125-polinomi_omogenei.org}{Polinomi Omogenei})
\begin{equation*}
I(X)\coloneqq \langle \set{F \in \K[x_{0},\dots,x_{n}]^{h}: \forall\, p \in X, \ F(p) = 0}\rangle
\end{equation*}
\subsubsection{Osservazione}
\label{sec:org1728112}
In accordo con quanto fatto in ``\href{20250102183523-luogo_di_zeri_di_un_ideale_omogeneo.org}{Luogo di zeri di un ideale omogeneo}'', con
\begin{equation*}
I(X)^{h} \subseteq I(X)
\end{equation*}
si intende l'insieme degli \href{20241231121125-polinomi_omogenei.org}{polinomi omogenei} di \(I(X)\).

Si noti inoltre che
\begin{equation*}
V\left(I(X)\right) = V\left(I(X)^{h}\right)
\end{equation*}
per definizione.
\end{document}
