% Intended LaTeX compiler: pdflatex
\documentclass[../main]{subfiles}


\begin{document}

Una varietà algebrica \(X\) (\href{20241231123223-varieta_algebrica_proiettiva.org}{proiettiva}, \href{20241231114256-varieta_algebrica_affine.org}{affine} o \href{20250107112123-varieta_algebrica_quasi_proiettiva_qp.org}{qp}) si dice \textbf{ipersuperficie} se è definita da una singola equazione, ovvero se
\begin{equation*}
X=V(F)
\end{equation*}
per un appropriato \href{20241231112750-polinomio.org}{polinomio} \(F\) (vedi \href{20241231112823-radici_polinomiali.org}{Luogo di zeri}).
\end{document}
