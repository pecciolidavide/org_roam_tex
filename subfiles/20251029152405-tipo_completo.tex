% Intended LaTeX compiler: pdflatex
\documentclass[../main]{subfiles}

\def\L{\mathcal{L}}


\begin{document}

\section{Tipo completo}
\label{sec:org52d9613}
Sia \(\L\) un \href{20250130162057-linguaggio_del_prim_ordine.org}{linguaggio del prim'ordine}, \(M\) una \(\L\)-\href{20250131103035-struttura_del_prim_ordine.org}{struttura} e \(A \subseteq M\).

\begin{definizione}
Un \href{20250212164424-tipo_teoria_dei_modelli.org}{tipo} \(p(x)\) si dice \uline{completo su \(A\)} se per ogni \href{20250131103317-formula_del_prim_ordine.org}{formula} \href{20250212102927-enunciato_con_parametri.org}{con parametri} \(\varphi(x) \in \L(A)\) si ha
\begin{equation*}
\varphi(x) \in p(x) \quad \lor \quad \lnot\varphi(x) \in p(x).
\end{equation*}
\begin{itemize}
\item L'insieme dei tipi completi su \(A\) \href{20250212164424-tipo_teoria_dei_modelli.org}{finitamente soddisfacibili} si indica con \(S(A)\).
\item L'insieme dei tipi \emph{di variabile libera \(x\)} completi su \(A\) finitamente soddisfacibili si indica con \(S_{x}(A)\).
\end{itemize}
\end{definizione}
\end{document}
