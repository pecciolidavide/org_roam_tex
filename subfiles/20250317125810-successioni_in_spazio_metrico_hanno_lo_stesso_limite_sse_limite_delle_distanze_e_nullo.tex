% Intended LaTeX compiler: pdflatex
\documentclass[../main]{subfiles}


\begin{document}

\section{Successione delle distanze tende a zero implica successioni hanno lo stesso limite}
\label{sec:orgad27ffb}
\subsection{Teorema}
\label{sec:org6949d7e}
Sia \((X,d)\) uno \href{20250301193511-spazio_metrico.org}{spazio metrico}, e siano \((z_{n})_{n \in \N}\), \((w_{n})_{n \in \N}\) due \href{20250115100904-successione.org}{successioni}. Se
\begin{itemize}
\item \((w_{n})_{n \in \N}\) è \href{20250115100930-convergenza_per_una_successione.org}{convergente};
\item \(\lim_{n\to \infty}d(z_{n}, w_{n}) = 0\)
\end{itemize}
allora \((z_{n})_{n \in \N}\) \href{20250115100930-convergenza_per_una_successione.org}{converge} e
\begin{equation*}
\lim_{n \to \infty} z_{n} = \lim_{n \to \infty} w_{n}
\end{equation*}
\subsubsection{Dimostrazione}
\label{sec:org2358c1a}

Sia \(k \coloneqq \lim_{n\to \infty} w_{n}\).

Allora, per ogni \(\varepsilon>0\), esiste
\begin{itemize}
\item \(N \in \N\) tale che, \(\forall\, n>N\): \(d(z_{n}, w_{n})<\frac{\varepsilon}{2}\);
\item \(M \in \N\) tale che, \(\forall\, n > M\): \(d(w_{n}, k) < \frac{\varepsilon}{2}\).
\end{itemize}

Quindi, per ogni \(n>\max\set{N,M}\),
\begin{equation*}
d(z_{n}, k) \le d(z_{n}, w_{n}) + d(w_{n}, k) < \varepsilon
\end{equation*}
grazie alla \href{20250306115949-disuguaglianza_triangolare.org}{disuguaglianza triangolare}.

Siccome \(X\) è \href{20250109155715-spazio_topologico_di_hausdorff.org}{T2} (vedi \href{20250317130124-spazi_metrici_sono_t2.org}{Spazi metrici sono T2}), allora per \href{20250304162602-unicita_del_limite.org}{unicità del limite}:
\begin{equation*}
\lim_{n\to \infty} z_{n} = k
\end{equation*}
\paragraph{Limiti uguali \%\% Dimostrazione alternativa}
\label{sec:org71406bb}
Siano quindi
\begin{equation*}
\ell \coloneqq \lim_{n\to\infty} z_{n},\quad k \coloneqq\lim_{n\to\infty}w_{n}
\end{equation*}
se per assurdo \(\ell\neq k\) allora, poiché \(X\) è \href{20250109155715-spazio_topologico_di_hausdorff.org}{Hausdorff} (vedi \href{20250317130124-spazi_metrici_sono_t2.org}{Spazi metrici sono T2}), esiste \(\varepsilon\) tale che, posto
\begin{equation*}
B_{d}(y, \varepsilon) \coloneqq \set{x \in Y\ |\ d(x,y)<\varepsilon}
\end{equation*}
si ha che
\begin{equation*}
B_{d}(\ell,\varepsilon)\cap B_{d}(k,\varepsilon) = \emptyset
\end{equation*}
Inoltre, per definizione di \href{20250115100930-convergenza_per_una_successione.org}{convergenza}, esiste \(M \in \N\) tale per cui, se \(n>M\), allora
\begin{equation*}
z_{n} \in B_{d}\left(\ell,\frac{\varepsilon}{4}\right) \subseteq B_{d}(\ell,\varepsilon), \quad w_{n} \in B_{d}\left(k,\frac{\varepsilon}{4}\right)
\end{equation*}

Allora \(d\left(z_{n}, w_{n}\right)\ge\frac{\varepsilon}{2}\). Questo contraddice l'ipotesi che
\begin{equation*}
\lim_{n\to \infty} d(z_{n},w_{n}) =0.
\end{equation*}


Infatti, se per assurdo
\begin{equation*}
d\left(z_{n}, w_{n}\right)<\frac{\varepsilon}{2}
\end{equation*}
allora per la disuguaglianza triangolare
\begin{equation*}
d\left(z_{n}, k\right) \le d\left(z_{n}, w_{n}\right) + d\left(w_{n}, k\right) < \frac{\varepsilon}{2} + \frac{\varepsilon}{4}<\varepsilon
\end{equation*}
e quindi \(z_{n} \in B_{d}(k,\varepsilon)\). Assurdo.
\end{document}
