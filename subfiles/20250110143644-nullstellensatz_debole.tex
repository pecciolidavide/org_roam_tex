% Intended LaTeX compiler: pdflatex
\documentclass[../main]{subfiles}

\usepackage[hyperref]{biblatex}
\date{}
\title{}
\begin{document}

\section{Nullstellensatz debole}
\label{sec:orgda7f3f3}
\subsection{Nullstellensatz debole}
\label{sec:org1b5da3e}
Sia \(\K\) un \href{20241231112713-campo_algebricamente_chiuso.org}{campo algebricamente chiuso}, e sia \(J \subseteq \K[x_{1},\dots,x_{n}]\) un \href{20241219112955-ideale.org}{ideale}. Sia  \(V(J) \subseteq \A^{n}\) (vedi \href{20241231114009-spazio_affine.org}{Spazio Affine} e \href{20241231112823-radici_polinomiali.org}{Luogo di zeri}).

Se \(J\) è un ideale proprio, ovvero \(J\neq \K[x_{1},\dots,x_{n}]\), allora \(V(J)\neq \emptyset\).
\end{document}
