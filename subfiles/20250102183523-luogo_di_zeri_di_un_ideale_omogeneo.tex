% Intended LaTeX compiler: pdflatex
\documentclass[../main]{subfiles}


\begin{document}

\section{Luogo di zeri di un ideale omogeneo}
\label{sec:orgc5a5cb2}
Sia \(\K\) un \href{20241231112713-campo_algebricamente_chiuso.org}{campo algebricamente chiuso} e sia \(S=\K[x_{0},\dots,x_{n}]\) l'\href{20241219113434-anello_dei_polinomi.org}{anello dei polinomi}.
\subsection{Definizione}
\label{sec:org0bb2447}
Se \(I\) è un \href{20250102182726-ideale_di_polinomi_omogeneo.org}{ideale omogeneo} allora si definisce
\begin{equation*}
V(I)\coloneqq V(I^{h}) \subseteq \mathds{P}^{n}
\end{equation*}
dove con \(I^{h}\) si intende l'insieme dei \href{20241231121125-polinomi_omogenei.org}{polinomi omogenei} di \(I\) e \(V(I^{h})\) è definito come \href{20241231112823-radici_polinomiali.org}{Luogo di zeri} (vedi anche \href{20241231123223-varieta_algebrica_proiettiva.org}{Varietà Algebrica Proiettiva}).
\subsubsection{Osservazione}
\label{sec:org507c130}
La definizione precedente è necessaria, poiché \(I\) omogeneo non contiene soltanto polinomi omogenei, e dunque non ha intrinsecamente senso chiedersi quali siano, in ambito \href{20241231115051-spazio_proiettivo.org}{proiettivo}, gli \href{20241231112823-radici_polinomiali.org}{zeri} di tutti i suoi polinomi.
\end{document}
