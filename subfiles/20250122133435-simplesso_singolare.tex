% Intended LaTeX compiler: pdflatex
\documentclass[../main]{subfiles}


\begin{document}

\section{Simplesso singolare}
\label{sec:orgb1047ea}
Sia \(X\) uno \href{20250103145124-topologia.org}{spazio topologico} e sia \(\Delta_{q}\) il \(q\)-\href{20250121122324-simplesso_standard.org}{simplesso standard} dotato della topologia euclidea.

\begin{definizione}
Un \(q\)-simplesso singolare è una mappa \href{20250103103252-funzione_continua.org}{continua}
\begin{equation*}
\sigma:\Delta_{q}\longrightarrow X
\end{equation*}
\end{definizione}

Sia quindi \(\Sigma_{q}(X)\) l'insieme di tutti i \(q\)-simplessi singolari,
\begin{equation*}
\Sigma_{q}(X) \coloneqq \set{\sigma:\Delta_{q}\longrightarrow X: \sigma\text{ continua}}
\end{equation*}
\section{Modulo delle catene singolari su uno spazio topologico.}
\label{sec:orgafac6f4}
Sia \(R\) un \href{20241219112842-pid.org}{PID}. Si definisce il \href{20241205141053-r_moduli.org}{modulo} delle \textbf{\(q\)-catene singolari} come la \href{20241213095808-somma_diretta.org}{somma diretta}:
\begin{equation*}
S_{q}(X) \coloneqq R^{(\Sigma_{q})} = \set{\sum r_{i}\ \sigma_{i}: r_{i} \in R}
\end{equation*}
dove le somme sono formali. Gli elementi vengono dette \textbf{catene singolari}.
\end{document}
