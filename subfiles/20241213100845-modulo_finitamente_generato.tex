% Intended LaTeX compiler: pdflatex
\documentclass[../main]{subfiles}


\begin{document}

\section{Modulo Finitamente Generato}
\label{sec:org97d5feb}
Sia \(R\) un \href{20241205141119-anello.org}{anello commutativo con unità}.

\begin{definizione}
Un \href{20241205141053-r_moduli.org}{\(R\)-modulo} \(M\) si dice \textbf{finitamente generato} se esiste \(S \subseteq M\) \textbf{finito} tale che \(S\) sia un \href{20241212141101-generatori_modulo.org}{insieme di generatori}:
\begin{equation*}
M = (S).
\end{equation*}
\end{definizione}

\begin{prop}
Un \href{20241205141053-r_moduli.org}{\(R\)-modulo} \href{20241213094625-modulo_libero.org}{libero} e finitamente generato ammette una \href{20241213094625-modulo_libero.org}{base} finita.
\end{prop}

\begin{oss}
Immagine di un modulo finitamente generato tramite \href{20241206115416-morfismi_r_moduli.org}{morfismo} è ancora un modulo finitamente generato.
\end{oss}
\end{document}
