% Intended LaTeX compiler: pdflatex
\documentclass[../main]{subfiles}

\usepackage[hyperref]{biblatex}
\date{}
\title{}
\begin{document}

\section{Omologia Simpliciale}
\label{sec:org596993f}
Sia \(R\) un \href{20241219112842-pid.org}{PID} e sia \(K\) un \href{20250121124147-complesso_simpliciale.org}{complesso simpliciale} \href{20250121131005-complesso_simpliciale_totalmente_ordinato.org}{totalmente ordinato.}

\begin{definizione}
Per ogni \(q \in\N\) si definisce l'\textbf{omologia simpliciale di \(K\)}, \(H_{q}(K)\), come il \href{20250120164857-modulo_di_omologia_dei_complessi_di_catene.org}{\(q\)-esimo modulo di omologia} del \href{20250121160600-complesso_di_catene_simpliciali.org}{complesso di catene simpliciali}:
\begin{equation*}
H_{q}(K) \coloneqq H_{q}\left(\mathcal{C}_{\bullet}(K)\right)
\end{equation*}
\end{definizione}
\end{document}
