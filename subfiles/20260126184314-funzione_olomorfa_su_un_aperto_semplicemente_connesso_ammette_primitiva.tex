% Intended LaTeX compiler: pdflatex
\documentclass[../main]{subfiles}


\begin{document}

\section{Funzione olomorfa su un aperto semplicemente connesso ammette primitiva}
\label{sec:org948f5fa}
\begin{prop}
Se \(U \subseteq \C\) è un \href{20250103145124-topologia.org}{aperto} \href{20250113100451-spazio_topologico_semplicemente_connesso.org}{semplicemente connesso} e \(f:U \to \C\) è \href{20260126110551-funzione_olomorfa.org}{olomorfa}, allora \(f\) ammette una primitiva su \(U\), ovvero esiste \(g:U\to \C\) olomorfa tale che\footnote{Con \(g'\) si intende la \href{20250114110703-derivata.org}{derivata} di \(g\).}
\begin{equation*}
f=g'.
\end{equation*}
\end{prop}
\end{document}
