% Intended LaTeX compiler: pdflatex
\documentclass[../main]{subfiles}


\begin{document}

\section{Funzione olomorfa su un aperto semplicemente connesso ammette radice m-esima}
\label{sec:org32e8053}
\begin{prop}
Se \(U \subseteq \C\) è un \href{20250103145124-topologia.org}{aperto} \href{20250113100451-spazio_topologico_semplicemente_connesso.org}{semplicemente connesso} e \(f:U\to \C\) tale che \(f(x) \neq 0\), allora per ogni \(m \in \N\) esiste \(g:U\to \C\) tale che \(g(x)\neq 0\) tale che
\begin{equation*}
[g(z)]^{m} = f(z).
\end{equation*}
\end{prop}
\begin{proof}
Se \(h(z)\) è \href{20260126184347-logaritmo_complesso_e_olomorfo_su_un_aperto_semplicemente_connesso.org}{logaritmo per \(f(z)\)}, è sufficiente porre
\begin{equation*}
g(z)\coloneqq e^{\frac{1}{m}h(z)}.\qedhere
\end{equation*}
\end{proof}
\end{document}
