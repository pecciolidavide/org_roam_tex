% Intended LaTeX compiler: pdflatex
\documentclass[../main]{subfiles}


\begin{document}

Sia \(\K\) un \href{20241231112713-campo_algebricamente_chiuso.org}{campo algebricamente chiuso} e siano \(X \subseteq \mathds{P}^{n}\), \(Y \subseteq \mathds{P}^{m}\) \href{20241231123223-varieta_algebrica_proiettiva.org}{varietà algebriche proiettive}. (Vedi \href{20241231115051-spazio_proiettivo.org}{spazio proiettivo})
\section{Definizione}
\label{sec:orgc637b47}
La funzione \(F: X\longrightarrow Y\) è un morfismo di varietà proiettive se, \(\forall\, p \in X\):
\begin{itemize}
\item esiste \(U \subseteq X\) \href{20250103145124-topologia.org}{aperto} con \(p \in U\);
\item esistono \(F_{0},\dots,F_{m}\) \href{20241231121125-polinomi_omogenei.org}{polinomi omogenei} dello stesso \href{20241231124742-grado_polinomi.org}{grado} senza zeri comuni su \(U\)
\end{itemize}
tali che \(\restriction{F}{U}=\left[F_{0}(x):\dots:F_{m}(x)\right]\).
\section{Isomorfismo tra varietà algebriche proiettive}
\label{sec:org21e5a7c}
\subsection{Definizione}
\label{sec:org9228a4c}
Un morfismo \(F: X \longrightarrow Y\) è un \href{20241128162125-isomorfismo.org}{isomorfismo} se
\begin{itemize}
\item \(F\) è \href{20250104111707-funzione_biunivoca.org}{biunivoca} (e quindi esiste \(F^{-1}\));
\item \(F^{-1}\) è un morfismo.
\end{itemize}
\end{document}
