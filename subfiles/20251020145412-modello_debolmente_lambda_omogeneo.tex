% Intended LaTeX compiler: pdflatex
\documentclass[../main]{subfiles}

\usepackage[hyperref]{biblatex}
\date{}
\title{}
\begin{document}

\section{Modello debolmente lambda omogeneo}
\label{sec:org16bc6a8}
Sia \(\mathcal{L}\) un \href{20250130162057-linguaggio_del_prim_ordine.org}{linguaggio del prim'ordine}, e sia \(\lambda\) un \href{20250203161341-cardinali.org}{cardinale} tale che \(\card{\mathcal{L}} + \omega < \lambda\).

Sia \(M\) una \(\mathcal{L}\)-\href{20250131103035-struttura_del_prim_ordine.org}{struttura} infinita.

\begin{definizione}
\(M\) è \uline{debolmente \(\lambda\)-omogeneo} se per ogni \(f:M\to M\) \href{20250214120959-mappe_tra_strutture_del_prim_ordine.org}{mappa elementare} di \href{20241213101756-cardinalita.org}{cardinalità} \(\card{f}<\lambda\), per ogni \(b \in M\) esiste \(c \in M\) tale che \(f\cup\set{\langle b,c\rangle }: M\to M\) elementare.

\(M\) si dice anche \uline{back and forth \(\lambda\)-omogeneo}.

\(M\) si dice \uline{debolmente omogeneo} se è \uline{debolmente \(\card{M}\)-omogeneo}.
\end{definizione}
\end{document}
