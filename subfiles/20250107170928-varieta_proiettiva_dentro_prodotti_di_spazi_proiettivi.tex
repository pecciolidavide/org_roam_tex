% Intended LaTeX compiler: pdflatex
\documentclass[../main]{subfiles}


\begin{document}

Sia \(\K\) un \href{20241231112713-campo_algebricamente_chiuso.org}{campo algebricamente chiuso}, e sia \(\Sigma_{n,m} \subseteq \mathds{P}^{N=(n+1)(m+1)-1}_{z=z_{ij}}\) la \href{20250104190620-mappa_di_segre.org}{varietà di Segre} (vedi \href{20241231115051-spazio_proiettivo.org}{Spazio Proiettivo})
Preso \(F\) un \href{20241231121125-polinomi_omogenei.org}{polinomio omogeneo} in \(\K[z_{ij}]\) (vedi \href{20241219113434-anello_dei_polinomi.org}{Anello-dei-polinomi}) di \href{20241231124742-grado_polinomi.org}{grado} \(d\), si ha che
\begin{equation*}
Y\coloneqq V(F)\cap \Sigma_{n,m}
\end{equation*}
(vedi \href{20241231112823-radici_polinomiali.org}{Luogo di zeri})
è una \href{20241231123223-varieta_algebrica_proiettiva.org}{varietà algebrica di }\(\mathds{P}^{N}\) ed è \(\subseteq \Sigma_{n,m}\). Pertanto, per \href{20250104190620-mappa_di_segre.org}{definizione}, \(\sigma^{-1}(Y)\), dove \(\sigma\) è la \href{20250104190620-mappa_di_segre.org}{mappa di Segre} della dimensione giusta, è una varietà algebrica di \(\mathds{P}^{n}\times \mathds{P}^{m}\).

In \(Y\), attraverso la mappa di Segre, posso scrivere \(z_{ij}=x_{i}y_{j}\). Posso pertanto ottenere delle equazioni polinomiali in \(\mathds{P}^{n}\times \mathds{P}^{m}\) \href{20250107172231-polinomi_biomogenei.org}{biomogenee} di \href{20250107172249-bigrado_di_un_polinomio.org}{bigrado} \((d,d)\).
\section{Proposizione}
\label{sec:org77e5785}
Sia \(F \in \K[x_{0},\dots,x_{n}; y_{0},\dots,y_{m}]\), \(F\) \href{20250107172231-polinomi_biomogenei.org}{biomogeneo} di \href{20250107172249-bigrado_di_un_polinomio.org}{bigrado} \((d,e)\). È ben definito
\begin{equation*}
X=V(F) \subseteq \mathds{P}^{n}\times \mathds{P}^{m}
\end{equation*}
(vedi \href{20241231112823-radici_polinomiali.org}{Luogo di zeri di polinomi}).
\(X\) è una \href{20241231123223-varieta_algebrica_proiettiva.org}{varietà algebrica proiettiva}.

Equivalentemente, esistono \(F_{1},\dots,F_{k} \in \K[z_{ij}]\) polinomi omogenei tali che
\begin{equation*}
\sigma\left(V(F)\right)=\Sigma_{n,m}\cap V(F_{1},\dots,F_{k})
\end{equation*}
\subsection{Dimostrazione}
\label{sec:orgf3e50ef}

\subsubsection{Caso \(e=d\)}
\label{sec:orga6555a1}
\(F\) è un polinomio biomogeneo di bigrado \((d,d)\). Pertanto, siccome in \(\Sigma_{n,m}\) si ha che \(z_{ij}=x_{i}y_{j}\), si ottiene un polinomio omogeneo \(F_{1} \in \K[z_{ij}]\) sostituendo in \(F\) a ciascun prodotto \(x_{i}y_{j}\) la variabile \(z_{ij}\).
\subsubsection{Caso \(d<e\)}
\label{sec:org1f940df}

Osserviamo che \(F=0\) se e solo se
\begin{equation*}
x_{0}F=x_{1}F=\dots=x_{n}F=0
\end{equation*}

Iterando questo processo, si ottiene che \(F=0\) se e solo se
\begin{equation*}
x_{0}^{e-d}F=x_{1}^{e-d}F=\dots=x_{n}^{e-d}F=0
\end{equation*}

Tutti i polinomi \(x_{i}^{e-d}F\) sono biomogenei di bigrado \((e,e)\) e pertanto, applicando il punto precedente, si ha la tesi. Inoltre \(k\le n+1\).
\end{document}
