% Intended LaTeX compiler: pdflatex
\documentclass[../main]{subfiles}


\begin{document}

\section{Teorema di Hales-Jewett}
\label{sec:org0f30861}
\def\D{\mathcal{D}}
\def\DD{\bm{\D}}
\def\U{\mathcal{U}}
\def\eq{{\rm eq}}
\def\Ueq{\U^\eq}
\def\L{\mathcal{L}}
\def\orbita{\mathcal{O}}
\def\Aut{\operatorname{Aut}}
\def\tc{\mid}
\def\tp{\operatorname{tp}}
\def\<{\langle}
\def\>{\rangle}
\def\b{\bm{b}}

\def\restricted#1{\,\mathord{\upharpoonright}{{\scriptstyle #1}}}
\def\equivalentover#1{\mathrel{\equiv_{ #1 }}}

%% NON FORKING
\def\nonforkSymbol{%
\mathbin{\raise1.8ex%
\rlap{\kern0.6ex\rule{0.6ex}{0.1ex}}%
\rlap{\kern1.1ex\rule{0.1ex}{1.9ex}}\raise-0.3ex\hbox{$\smile$}}}
\def\defaultnonforkmodel{M}
\def\nonfork{\nonforkSymbol}
\renewcommand{\nonfork}[1][\defaultnonforkmodel]{%
\mathrel{\nonforkSymbol_{#1}}}
\begin{thm}
Sia \(S\) un \href{20251104163738-semigruppo.org}{semigruppo} e sia \(C \subseteq S\) un sotto-semigruppo tale che per ogni \(a\cdot b \in C\) si ha \(a \in C\) e \(b \in C\)\footnote{Diremo che il sottosemigruppo è \emph{nice}.}.

Dato \(k \in \omega\) sia\footnote{Vedi ``\href{20251104163738-semigruppo.org}{Omomorfismo di semigruppi}''}
\begin{equation*}
\Sigma\coloneqq \set{%
\sigma_{i}: S\to C \text{ omomorfismo}%
\mid \sigma_{i} \restricted{C} = \Id_{C},\, i<k}
\end{equation*}
Per ogni \href{20251111150142-colorazione.org}{colorazione} finita di \(C\) esiste \(a \in S\setminus C\) tale che
\(\set{\sigma_{i}(a) \mid i < k}\)
è \href{20251111150142-colorazione.org}{monocromatico}.
\end{thm}

Una dimostrazione di questo può essere prodotta utilizzando gli strumenti della \href{20250612143636-notazione_teoria_dei_modelli.org}{Teoria dei Modelli}, sfruttando il \href{20251104154427-estensione_di_un_semigruppo_ad_un_modello_mostro.org}{Semigruppo di Ellis}. Se ne lascia qui una traccia:

\def\C{\mathcal{C}}
\begin{proof}
Sia \(\L=\set{\cdot}\cup\parti{S}\cup \Sigma\). Sia \(\U\succeq S\) \href{20250617102733-modello_mostro.org}{modello mostro}, e sia \(\C = \phi(\U^{x})\), dove
\begin{equation*}
\phi(x):\qquad x \in C.
\end{equation*}
\begin{itemize}
\item Dal momento che la proprietà \emph{nice} è esprimibile al prim'ordine, anche \(\mathcal{C}\) è \emph{nice}. Pertanto \(\U\setminus\C\) è ideale sinistro.
\item Sia \(M \subseteq \U\setminus\C\) ideale sinistro \href{20250203102516-massimo_e_minimo.org}{minimale} \href{20250212164424-tipo_teoria_dei_modelli.org}{tipo-definibile}, e sia \(N \subseteq \C\) ideale sinistro di \(\C\) minimale tipo-definibile.
\item Sia \(v \in N\) idempotente, e sia \(u \in v*M*v\) idempotente.
\item Per ogni \(\sigma \in \Sigma\)
\begin{equation}
\sigma(u) \in \sigma(v)*\sigma(M)*\sigma(v) = v*\sigma(M)*v %
\label{eq:vsigmaMv}
\end{equation}
con \(\sigma(u)\) idempotente.
\item \(\sigma(M)\) è ideale sinistro di \(\C\), in quanto \(\sigma\) è l'identità su \(\C\):
\begin{equation*}
\C*\sigma(M) = \sigma(\C*M) \subseteq \sigma(M).
\end{equation*}
\item Si ha che \(\sigma(M)*v \subseteq N\) e per minimalità di \(N\):
\begin{equation*}
\sigma(M)*v = N.
\end{equation*}
\item Applicando alla~\eqref{eq:vsigmaMv}: \(\sigma(u) \in v*N\), ma per un lemma già visto \(v*N\) è un gruppo con identità \(v\), e quindi \(\sigma(u) \equivalentover{S} v\) per ogni \(\sigma \in \Sigma\).
\end{itemize}

Dunque
\begin{equation*}
\U\vDash \exists x \in (\U\setminus\C)\ %
\bigg[ %
\bigwedge_{\sigma \in \Sigma}%
\text{``}\sigma(x)\text{ ha colore 1''}%
\bigg].
\end{equation*}
e pertanto vale anche per \(S\).
\end{proof}
\end{document}
