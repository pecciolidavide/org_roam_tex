% Intended LaTeX compiler: pdflatex
\documentclass[../main]{subfiles}


\begin{document}

\section{Successione Esatta di Complessi di Cocatene}
\label{sec:org361c744}
Siano \({C^{\bullet}}'\), \({C^{\bullet}}\) e \({C^{\bullet}}''\) dei complessi di cocatene, e siano \(F^{\bullet}:{C^{\bullet}}'\to {C^{\bullet}}\) e \(G^{\bullet}:{C^{\bullet}}\to {C^{\bullet}}''\) dei morfismi. La successione
\begin{equation*}
{C^{\bullet}}' \xrightarrow{F^{\bullet}} {C^{\bullet}} \xrightarrow{G^{\bullet}}  {C^{\bullet}}''
\end{equation*}
si dice \uline{successione esatta} se:
\begin{equation*}
\begin{tikzcd}[sep=large]
	& \dots & \dots & \dots \\
	0 & {(C^{k-1})'} & {C^{k-1}} & {(C^{k-1})''} & 0 \\
	0 & {(C^k)'} & {C^{k}} & {(C^{k})''} & 0 \\
	0 & {(C^{k+1})'} & {C^{k+1}} & {(C^{k+1})''} & 0 \\
	& \dots & \dots & \dots
	\arrow["{(\mathrm{d}^{k-2})'}"', from=1-2, to=2-2]
	\arrow["{\mathrm{d}^{k-2}}", from=1-3, to=2-3]
	\arrow["{(\mathrm{d}^{k-2})''}", from=1-4, to=2-4]
	\arrow[from=2-1, to=2-2]
	\arrow["{F^{k-1}}", from=2-2, to=2-3]
	\arrow["{(\mathrm{d}^{k-1})'}"', from=2-2, to=3-2]
	\arrow["{G^{k-1}}", from=2-3, to=2-4]
	\arrow["{\mathrm{d}^{k-1}}", from=2-3, to=3-3]
	\arrow[from=2-4, to=2-5]
	\arrow["{(\mathrm{d}^{k-1})''}", from=2-4, to=3-4]
	\arrow[from=3-1, to=3-2]
	\arrow["{F^{k}}", from=3-2, to=3-3]
	\arrow["{(\mathrm{d}^{k})'}"', from=3-2, to=4-2]
	\arrow["{G^{k}}", from=3-3, to=3-4]
	\arrow["{\mathrm{d}^{k}}", from=3-3, to=4-3]
	\arrow[from=3-4, to=3-5]
	\arrow["{(\mathrm{d}^{k})''}", from=3-4, to=4-4]
	\arrow[from=4-1, to=4-2]
	\arrow["{F^{k+1}}", from=4-2, to=4-3]
	\arrow["{(\mathrm{d}^{k+1})'}"', from=4-2, to=5-2]
	\arrow["{G^{k+1}}", from=4-3, to=4-4]
	\arrow["{\mathrm{d}^{k+1}}", from=4-3, to=5-3]
	\arrow[from=4-4, to=4-5]
	\arrow["{(\mathrm{d}^{k+1})''}", from=4-4, to=5-4]
\end{tikzcd}
\end{equation*}
\begin{itemize}
\item il diagramma commuta;
\item tutte le righe sono \href{20251115182133-successione_di_spazi_vettoriali_esatta.org}{successioni esatte corte di spazi vettoriali}.
\end{itemize}
\end{document}
