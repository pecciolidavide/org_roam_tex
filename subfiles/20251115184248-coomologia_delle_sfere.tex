% Intended LaTeX compiler: pdflatex
\documentclass[../main]{subfiles}


\begin{document}

\section{Coomologia delle sfere}
\label{sec:org1dd6b7b}
\begin{thm}
Sia \(\mathds{S}^{n}\) la \href{20250115150754-sfera_n_dimensionale.org}{sfera \(n\)-dimensionale}. Allora la sua \href{20251115172442-gruppo_di_coomologia_di_de_rham.org}{coomologia di De Rham} è:
\begin{equation*}
H^{k}(\mathds{S}^{n}) = %
\begin{cases}
\R & k=0,n\\
0 & k\neq 0,n
\end{cases}
\end{equation*}
\end{thm}
\begin{proof}
Vedi Esempio~5.3.7 di .

Si considerino i poli \(\textbf{N}\in \R^{n+1}\) e \(\textbf{S} \in \R^{n+1}\) di \(\mathds{S}^{n}\), e siano
\begin{equation*}
U_{0} \coloneqq \mathds{S}^{n}\setminus\set{\textbf{N}}, \qquad U_{1}\coloneqq \mathds{S}^{n}\setminus\set{\textbf{S}}.
\end{equation*}
\begin{itemize}
\item Siccome \(U_{0}\cong U_{1} \cong \R^{n}\) \href{20250113172924-diffeomorfismo_tra_varieta_differenziabili.org}{diffeomorfi}, \href{20251115173810-varieta_differenziabili_diffeomorfe_hanno_stessa_coomologia_di_de_rham.org}{allora}\footnote{Vedi anche ``\href{20251115173611-coomologia_di_de_rham_di_r.org}{Coomologia di De Rham di R}''}
\begin{equation*}
  \forall k \in \N: \qquad H^{k}(U_{0}) \cong H^{k}(U_{1}) \cong H^{k}(\R) = \begin{cases}
  	\R & k = 0\\
  	0 &\text{altimenti}.
      \end{cases}
\end{equation*}
\item \(U_{0}\cap U_{1} = \mathds{S}^{n}\setminus\set{\textbf{N},\textbf{S}} \cong \mathds{S}^{n-1} \times \R\), e \href{20251223145108-riduzione_della_coomologia_di_un_prodotto_con_la_retta_reale.org}{pertanto}
\begin{equation*}
  \forall k \in \N:\qquad H^{k}(U_{0}\cap U_{1}) \cong H^{k}(\mathds{S}^{n-1}).
\end{equation*}
\end{itemize}

Si dimostra il teorema per induzione su \(n\ge 1\).
\begin{itemize}
\item \uline{Passo base \(n=1\)}.

\href{20251115184223-coomologia_della_circonferenza.org}{Si è già dimostrato} che
\begin{equation*}
H^{k}(\mathds{S}^{1}) = %
\begin{cases}
\R & k = 0,1\\
0 & k \ge 2
\end{cases}
\end{equation*}

\item \uline{Passo induttivo}. Sia \(n > 1\) e si suppona il teorema vero per \(\mathds{S}^{n-1}\). Si dimostra per \(\mathds{S}^{n}\)

\begin{itemize}
\item \uline{Se \(k=0\)}: Siccome \(\mathds{S}^{n}\) è \href{20250103165325-spazio_topologico_connesso.org}{connesso}, \href{20251115174538-0_gruppo_di_coomologia_di_de_rham_di_una_varieta_connessa.org}{allora} \(H^{0}(\mathds{S}^{n}) \cong \R\).

\item \uline{Se \(k=1\)}: Si consideri \href{20251115183635-teorema_di_mayer_vietoris_in_coomologia.org}{Mayer-Vietoris}:
\begin{equation*}
\begin{tikzcd}[sep=tiny]
        0 & {H^0(\mathds{S}^n)} & {H^0(U_0)\oplus H^0(U_1)} & {H^0(U_0\cap U_1)} & {H^1(\mathds{S}^n)} & {H^1(U_0)\oplus H^1(U_1)}
        \arrow[from=1-1, to=1-2]
        \arrow[from=1-2, to=1-3]
        \arrow[from=1-3, to=1-4]
        \arrow[from=1-4, to=1-5]
        \arrow[from=1-5, to=1-6]
\end{tikzcd}
\end{equation*}
per le considerazioni di cui sopra si ha:
\begin{itemize}
\item \(H^{0}(\mathds{S}^{n}) \cong \R\)
\item \(H^{0}(U_{0}\cap U_{1}) \cong \R\) \href{20251115174538-0_gruppo_di_coomologia_di_de_rham_di_una_varieta_connessa.org}{in quanto} connesso
\item \(H^{0}(U_{0}) \oplus H^{0}(U_{1}) \cong \R\oplus \R\)\footnote{Vedi ``\href{20241213095808-somma_diretta.org}{Somma Diretta}''}
\item \(H^{1}(U_{0}) \oplus H^{1}(U_{1}) \cong 0\)
\end{itemize}
e pertanto si ottiene:
\begin{equation*}
\begin{tikzcd}[sep=small]
        0 & \R & {\R\oplus \R} & \R & {H^1(\mathds{S}^n)} & 0
        \arrow[from=1-1, to=1-2]
        \arrow[from=1-2, to=1-3]
        \arrow[from=1-3, to=1-4]
        \arrow["{\partial^*}", from=1-4, to=1-5]
        \arrow[from=1-5, to=1-6]
\end{tikzcd}
\end{equation*}
Siccome \(\partial^{*}\) è \href{20241213105600-funzione_suriettiva.org}{suriettiva}\footnote{Vedi ``\href{20250120130155-caratterizzazione_di_alcune_successioni_esatte_di_r_moduli.org}{Caratterizzazione SEC}''}, allora \(\dim H^{1}(\mathds{S}^{n})< \infty\) e quindi, usando la \href{20251115182133-successione_di_spazi_vettoriali_esatta.org}{somma alterna delle dimensioni}, si ottiene
\begin{equation*}
      H^{1}(\mathds{S}^{n}) = 0
\end{equation*}

\item Se \(1<k\le n\): Si consideri Mayer-Vietoris:
\begin{equation*}
\begin{tikzcd}
        {H^{k-1}(U_0)\oplus H^{k-1}(U_1)} && {H^{k-1}(U_0\cap U_1)} \\
        \\
        {H^k(\mathds{S}^n)} && {H^k(U_0)\oplus H^k(U_1)}
        \arrow[from=1-1, to=1-3]
        \arrow["{{\partial^*}}"', from=1-3, to=3-1]
        \arrow[from=3-1, to=3-3]
\end{tikzcd}
\end{equation*}
Per le considerazioni di cui sopra:
\begin{itemize}
\item \(k-1 > 0\), e pertanto \(H^{k-1}(U_{0})\oplus H^{k-1}(U_{1}) \cong 0\);
\item \(k > 0\), e pertanto \(H^{k}(U_{0})\oplus H^{k}(U_{1}) \cong 0\);
\item \(H^{k-{1}}(U_{0}\cap U_{1}) \cong H^{k-1}(\mathds{S}^{n-1})\).
\end{itemize}
La successione diventa:
\begin{equation*}
\begin{tikzcd}
        0 & {H^{k-1}(\mathds{S}^{n-1})} && {H^k(\mathds{S}^n)} & 0
        \arrow[from=1-1, to=1-2]
        \arrow["{\partial^*}"', from=1-2, to=1-4]
        \arrow[from=1-4, to=1-5]
\end{tikzcd}
\end{equation*}
e \href{20250120130155-caratterizzazione_di_alcune_successioni_esatte_di_r_moduli.org}{pertanto}:
\begin{equation*}
H^{k}(\mathds{S}^{n}) = H^{k-1}(\mathds{S}^{n-1}).
\end{equation*}

Dunque, per ipotesi induttiva:
\begin{align*}
&& H^{n}(\mathds{S}^{n}) &= H^{n-1}(\mathds{S}^{n-1}) \cong \R\\
&1<k<n & H^{k}(\mathds{S}^{n}) &= H^{k-1}(\mathds{S}^{n-1}) \cong 0.
\end{align*}

\item Se \(k>n\): per motivi di dimensione, \(H^{k}(\mathds{S}^{n}) = 0\).
\qedhere
\end{itemize}
\end{itemize}
\end{proof}

\begin{prop}
Un \href{20250102163502-base_di_uno_spazio_vettoriale.org}{generatore} di \(H^{n}(\mathds{S}^{n})\) è \([\nu]\), dove \(\nu\) è una \href{20251115184544-forma_volume_su_una_varieta_differenziabile.org}{forma volume} su \(\mathds{S}^{n}\).
\end{prop}
\begin{proof}
La dimostrazione si articola in alcune fasi
\begin{enumerate}
\item \uline{Esistenza di una forma volume}

Questo \href{20251115185324-caratterizzazione_varieta_differenziabile_orientabile_tramite_forma_voluma.org}{segue} dal fatto che \href{20251223161934-orientabilita_delle_sfere.org}{\(\mathds{S}^{n}\) è orientabile}.

\item \uline{La forma volume non è esatta}

Se per assurdo \(\nu\) fosse esatta, allora \(\nu = \dif \omega\). Poiché \(\mathds{S}^{n}\) è \href{20250103163701-spazio_topologico_compatto.org}{compatta}, allora per il \href{20251115190058-teorema_di_stokes.org}{Teorema di Stokes}:
\begin{equation*}
 \int_{M} \nu = \int_{M} \dif \omega = 0
\end{equation*}
ma questo è assurdo.
\end{enumerate}

Siccome la forma volume è una \(n\)-forma, allora è \href{20251115172517-forma_differenziale_chiusa.org}{chiusa}, e pertanto \([\nu] \in H^{n}(\mathds{S}^{n})\), e siccome non è esatta allora \([\nu] \neq 0\). Quindi \([\nu]\) genera \(H^{n}(\mathds{S}^{n}) \cong \R\).
\end{proof}

\begin{prop}
Si definisca la mappa data dall'\href{20251115185654-integrazione_di_forme_su_varieta_differenziabile_orientata.org}{integrale}:
\begin{align*}
\int: H^{n}(\mathds{S}^{n}) &\longrightarrow \R\\
[\omega] &\longmapsto \int_{\mathds{S}^{n}}\omega
\end{align*}

Questo è un \href{20250113125833-isomorfismo_tra_spazi_vettoriali.org}{isomorfismo}.
\end{prop}
\begin{proof}
La diumostrazione si sviluppa in diversi passi.
\begin{enumerate}
\item \uline{La mappa è ben definita in coomologia}

Questo vale per il teorema di Stokes.

\item \uline{La mappa è lineare}
\item \uline{La mappa è non nulla}

Infatti \(\int_{\mathds{S}^{n}} \nu > 0\) per \(\nu\) \href{20251115184544-forma_volume_su_una_varieta_differenziabile.org}{forma volume}. (Questo lo si può fare in quanto \(\mathds{S}^{n}\) è \href{20250103163701-spazio_topologico_compatto.org}{compatta} e \href{20251223152054-varieta_differenziabile_orientabile.org}{orientabile}.)
\end{enumerate}

Da questo segue che sia un isomorfismo:
\begin{itemize}
\item è \href{20241219101956-funzione_iniettiva.org}{iniettiva} in quanto il suo \href{20251121143525-kernel_di_una_funzione_tra_spazi_vettoriali.org}{kernel} (siccome \(H^{n}(\mathds{S}^{n})\) ha dimensione 1) può avere solo dimensione \(0\) (non ha dimensione 1 per il punto 3.)
\item è \href{20241213105600-funzione_suriettiva.org}{suriettiva} per il punto 3 (siccome raggiunge un elemento non nullo, siccome \(\R\) ha dimensione 1, per linearità raggiunge tutti gli elementi).
\end{itemize}
\end{proof}
\end{document}
