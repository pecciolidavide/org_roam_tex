% Intended LaTeX compiler: pdflatex
\documentclass[../main]{subfiles}


\begin{document}

\section{Principio di induzione}
\label{sec:orge56be82}
\section{Costruzione in MK}
\label{sec:org00ba3cb}

Contesto: \href{20250130104245-morse_kelly_set_theory.org}{Morse Kelly Set Theory}.
\subsection{Definizione\hfill{}\textsc{IdL:matematica\_lm}}
\label{sec:org6c36f24}

Sia \(\mathcal{J}\) la \href{20250130104320-classe_mk.org}{classe} di tutti gli \href{20250202125627-insieme_induttivo_mk.org}{insiemi induttivi} (che esiste per l'Axiom of Comprehension). Si definisce \(\N\) come l'insieme \href{20250131155822-operazioni_insiemistiche_tra_classi_mk.org}{intersezione}:
\begin{equation*}
\N\coloneqq \bigcap \mathcal{J}
\end{equation*}
Pertanto \(\N\) è il più piccolo insieme contenente \(\emptyset\) (vedi \href{20250131161811-insieme_vuoto_mk.org}{Insieme vuoto MK}) e chiuso per l'operazione di \href{20250202124648-successore_di_un_insieme_mk.org}{successore}.
Si definiscono:
\begin{align*}
0 &\coloneqq \emptyset\\
1 &\coloneqq \operatorname{S}(0)\\
2 &\coloneqq \operatorname{S}(1)\\
\vdots
\end{align*}
\subsection{Proposizione}
\label{sec:orgde8ec1d}
\(\N\) è un insieme induttivo e se \(n \in\N\) allora \(n = 0\) oppure \(n=\operatorname{S}(m)\) per qualche \(m \in \N\).
\subsubsection{Dimostrazione}
\label{sec:org5506a99}

\paragraph{Insieme induttivo}
\label{sec:orge2e8de6}

\(\N\) è sicuramente un insieme \href{20250131155822-operazioni_insiemistiche_tra_classi_mk.org}{poiché è una intersezione}. Inoltre \(\emptyset \in \N\) poiché \(\forall\, J \in \mathcal{J}\ (\emptyset \in J)\).
Inoltre, se \(n \in \N\) allora \(\forall\,J \in \mathcal{J}\ (n \in J)\), e siccome \(J\) induttivo allora \(\operatorname{S}(n) \in J\). Dunque \(\forall\, J \in \mathcal{J}\ \left(\operatorname{S}(n) \in J\right)\) e dunque \(\operatorname{S}(n) \in \N\).
\paragraph{Elementi di \(\N\)}
\label{sec:orgb09a99e}

Supponiamo per assurdo che \(n \in \N\setminus\set{\emptyset}\) (vedi \href{20250131155822-operazioni_insiemistiche_tra_classi_mk.org}{Sottrazione di classi MK}) sia tale che \(\operatorname{S}(m)\neq n\) per ogni \(m \in \N\).
Allora \(J \coloneqq \N\setminus \set{n}\) è un insieme induttivo, e dunque \(J \in \mathcal{J}\) e quindi \(\N \subseteq J\). Ma \(J \subset \N\), contraddizione.
\subsection{Induzione per i naturali}
\label{sec:org519e2f2}

Sia \(I \subseteq \N\) (vedi \href{20250131155822-operazioni_insiemistiche_tra_classi_mk.org}{Sottoclasse MK}) tale che
\begin{enumerate}
\item \(\emptyset \in I\);
\item \(\forall\, n \ \left(n \in I \,\implies\, \operatorname{S}(n) \in I\right)\)
\end{enumerate}
Allora \(I=\N\).
\subsubsection{Dimostrazione}
\label{sec:orgeed81f2}

Basta mostrare che \(N \subseteq I\), ma \(I\) è un insieme induttivo per definizione.
\end{document}
