% Intended LaTeX compiler: pdflatex
\documentclass[../main]{subfiles}


\begin{document}

In questa sezione si identificano i \href{20250131155822-operazioni_insiemistiche_tra_classi_mk.org}{sottinsiemi} di \(\N^{k}\) (vedi \href{20250202130045-insieme_dei_numeri_naturali_mk.org}{Insieme dei numeri naturali MK}) con i \href{20250131103317-formula_del_prim_ordine.org}{predicati} \(k\)-ari (ovvero con \(k\) \href{20250131103429-variabile_libera_di_una_formula.org}{variabili libere}), per mezzo degli \href{20250131122913-soddisfazione_di_una_formula.org}{insiemi di verità}.
Scriveremo indifferentemente \((x_{1},\dots,x_{k}) \in P\) oppure \(P(x_{1},\dots,x_{k})\) per dire che \(\N\vDash P(x_{1},\dots,x_{k})\).
\begin{definizione}
Un insieme \(P \subseteq \N^{k}\) è \uline{ricorsivo primitivo} se la sua \href{20250215160218-funzione_caratteristica.org}{funzione caratteristica} \(\chi_{P} : \N^{k}\to \N\) è una \href{20250215141024-funzioni_primitive_ricordive.org}{funzione ricorsiva primitiva}.
\end{definizione}
\begin{oss}
Un insieme \(P \subseteq \N^{k}\) è ricorsivo primitivo se e solo se \({\bm{J}}^{k}[P]\) è ricorsivo primitivo (vedi \href{20250215151413-biiezione_canonica_tra_n_e_n2.org}{Biiezione canonica tra N e prodotti cartesiani di N} e \href{20250202190147-immagine_punto_a_punto_di_due_classi.org}{Immagine punto a punto di due classi MK})
\end{oss}
\end{document}
