% Intended LaTeX compiler: pdflatex
\documentclass[../main]{subfiles}


\begin{document}

\section{Catena}
\label{sec:org629b2ac}
\textbf{\textbf{\uline{NOTA}: quando si parla di \uline{classi}, se ne parla nell'ambito della \href{20250130104245-morse_kelly_set_theory.org}{Morse Kelly Set Theory}; quando si parla di insiemi, il discorso ha validità più generale.}}
\begin{definizione}
Sia \(X\) un insieme (o una classe) e sia \(R\) un ordine (stretto). Un sottoinsieme (o sottoclasse) \(C \subseteq X\) è una \uline{catena} se
\begin{equation*}
\forall\,x,y \in C\ (x\neq y\implies(x\mathrel{R} y \,\lor\, y\mathrel{R}x))
\end{equation*}
\end{definizione}
\begin{definizione}
Una catena \(C \subseteq X\) si dice \uline{massimale} se per ogni \(C' \subseteq X\)
\begin{equation*}
\text{se }C \subsetneqq C'\text{ allora }C'\text{ non è una catena.}
\end{equation*}
\end{definizione}
\end{document}
