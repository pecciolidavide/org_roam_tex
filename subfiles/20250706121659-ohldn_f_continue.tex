% Intended LaTeX compiler: pdflatex
\documentclass[../main]{subfiles}


\begin{document}

\section{One Hidden Layer Discriminatory Network impara funzioni continue sul cubo}
\label{sec:orgc0b86d1}
\begin{lem}
Sia \(I_{n}\coloneqq[0,1]^{n}\), e si consideri \(C(I_{n})\)\footnote{Vedi ``\href{20250113125602-classe_c_di_una_funzione.org}{Classe C di una funzione}''\label{org43f1c9b}} dotato della \href{20250103145124-topologia.org}{topologia} \href{20250301193530-topologia_indotta_da_una_distanza.org}{indotta} dalla \href{20250301193511-spazio_metrico.org}{metrica}
\begin{equation*}
d(f,g) \coloneqq \sup_{x \in I_{n}} |f(x)-g(x)| = \max_{x \in I_{n}}|f(x)-g(x)|.
\end{equation*}
Sia \(\mathcal{M}\) la famiglia delle \href{20250625104200-misura_di_baire.org}{misure di Baire} per \(\R^{n}\) sul cubo \(I^{n}\R\), \href{20250625110016-misura_finita.org}{finite}, \href{20250625110024-misura_con_segno.org}{con segno} e \href{20250625110032-misura_regolare.org}{regolari}.

Sia \(\mathcal{U} \subseteq C(I_{n})\) un \href{20250114103118-sottospazio_vettoriale.org}{sottospazio vettoriale} non \href{20250301193045-sottoinsieme_denso.org}{denso}. Allora esiste una \href{20250704163455-misura.org}{misura} \(\mu \in \mathcal{M}\) tale che
\begin{equation*}
\forall  h \in \mathcal{U}:\qquad \int_{I_{n}} h(x)\dif\mu = 0
\end{equation*}
\label{lem:9.3.3}
\end{lem}
\begin{prop}
Sia \(\sigma\) una \href{20250625105528-funzione_discriminatoria_per_una_misura_di_baire_sul_cubo_unitario.org}{funzione discriminatoria} e \href{20250103103252-funzione_continua.org}{continua}. Allora l'insieme
\begin{equation*}
\set{G(x) = \sum_{i =1}^{N}\alpha_{j}\sigma(\bm{w}_{j}\cdot \bm{x} + \theta_{j})\mid N \in \N, \bm{w}_{j} \in \R^{n}, \alpha_{j}, \theta_{j} \in \R}
\end{equation*}
è denso in \(C(I_{n})\)\textsuperscript{\ref{org43f1c9b}} dotato della \href{20250103145124-topologia.org}{topologia} \href{20250301193530-topologia_indotta_da_una_distanza.org}{indotta} dalla \href{20250301193511-spazio_metrico.org}{metrica}
\begin{equation*}
d(f,g) \coloneqq \sup_{x \in I_{n}} |f(x)-g(x)| = \max_{x \in I_{n}}|f(x)-g(x)|.
\end{equation*}
\label{prop9.3.5}
\end{prop}
\begin{thm}
Sia \(\sigma:\R\to [0,1]\) una funzione \href{20250625110110-funzione_sigmoidale.org}{sigmoidale}. Allora l'insieme
\begin{equation*}
\set{G(x) = \sum_{i =1}^{N}\alpha_{j}\sigma(\bm{w}_{j}\cdot \bm{x} + \theta_{j})\mid N \in \N, \bm{w}_{j} \in \R^{n}, \alpha_{j}, \theta_{j} \in \R}
\end{equation*}
è denso in \(C(I_{n})\)\textsuperscript{\ref{org43f1c9b}} dotato della \href{20250103145124-topologia.org}{topologia} \href{20250301193530-topologia_indotta_da_una_distanza.org}{indotta} dalla \href{20250301193511-spazio_metrico.org}{metrica}
\begin{equation*}
d(f,g) \coloneqq \sup_{x \in I_{n}} |f(x)-g(x)| = \max_{x \in I_{n}}|f(x)-g(x)|.
\end{equation*}
\label{teo9.3.6}
\end{thm}

Il teorema continua a valere anche in questa forma:
\begin{thm}
Sia \(\sigma:\R\to [0,1]\) una funzione \href{20250625110110-funzione_sigmoidale.org}{sigmoidale}. Allora l'insieme
\begin{equation*}
\set{G(x) = \sum_{i =1}^{N}\alpha_{j}\sigma(\bm{w}_{j}\cdot \bm{x} + \theta_{j})\mid N \in \N, \bm{w}_{j} \in \R^{n}, \alpha_{j}, \theta_{j} \in \R}
\end{equation*}
è denso in \(C(K)\) dotato della \href{20250103145124-topologia.org}{topologia} \href{20250301193530-topologia_indotta_da_una_distanza.org}{indotta} dalla \href{20250301193511-spazio_metrico.org}{metrica}
\begin{equation*}
d(f,g) \coloneqq \sup_{x \in K} |f(x)-g(x)| = \max_{x \in K}|f(x)-g(x)|,
\end{equation*}
per ogni \(K\) \href{20250103163701-spazio_topologico_compatto.org}{compatto} di \(\R^{n}\).
\end{thm}
\end{document}
