% Intended LaTeX compiler: pdflatex
\documentclass[../main]{subfiles}


\begin{document}

Sia \(S=\K[x_{1},\dots,x_{n}]\) \href{20241219113434-anello_dei_polinomi.org}{l'anello dei polinomi}
\section{Definizione}
\label{sec:org410d430}
Un polinomio \(F\in S\) si dice \textbf{omogeneo} di grado \(d\) se per ogni \(\lambda \in \K\) si ha che
\begin{equation*}
F(\lambda x_{0},\lambda x_{1},\dots,\lambda x_{n}) = \lambda^{d}F(x_{0},x_{1},\dots,x_{n})
\end{equation*}
\section{Insiemi di polinomi omogeneri}
\label{sec:org807696b}
Definiamo
\begin{align*}
S^{h}&=\set{\text{polinomi omogenei}} \subseteq S\\
S_{d}&=\set{\text{polinomi omogeneri di grado }d} \subseteq S
\end{align*}

Si ha che \(S^{h}\) è \textbf{solo} un sottoinsieme di \(S\), mentre \(S_{d}\) è un \href{20241206143051-sottogruppo.org}{sottogruppo} di \(S\), con la convenzione che il polinomio nullo sia di grado qualsiasi, e per di più:
\begin{equation*}
S=\bigoplus_{d=0}^{+ \infty}S_{d}
\end{equation*}
(vedi \href{20241213095808-somma_diretta.org}{Somma-Diretta}).
\subsection{Dimensione di \(S_{d}\) come \href{20241205142027-spazio_vettoriale.org}{spazio vettoriale}}
\label{sec:org5a64174}
Consideriamo \(I=(i_{0},\dots,i_{n})\) un \href{20250105122522-multi_indice.org}{multi-indice}; definiamo \(x^{I} \in S\) come
\begin{equation*}
x^{I}\coloneqq x_{0}^{i_{0}}\cdot \dots\cdot x_{n}^{i_{n}}
\end{equation*}
Allora l'insieme \(\set{x^{I}}_{I \in D}\), con \(D=\set{(i_{0},\dots,i_{n}): \sum i_{j} = d, i_{j}\ge 0}\) è una base di \(S_{d}\).

Da qui segue che
\begin{equation*}
\dim S_{d}=\binom{n+d}{d}
\end{equation*}
\end{document}
