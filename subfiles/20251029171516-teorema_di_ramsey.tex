% Intended LaTeX compiler: pdflatex
\documentclass[../main]{subfiles}


\begin{document}

\section{Teorema di Ramsey}
\label{sec:org93a3e80}
Si veda la Sezione~15.1 di 

\def\M{\mathcal{M}}
\def\U{\mathcal{U}}
\def\L{\mathcal{L}}
\def\equivalentover#1{\mathrel{\equiv_{#1}}}
\def\restricted#1{\,\mathord{\upharpoonright}{\scriptstyle #1}}
%% Alcuni simboli
\def\nonforkSymbol{\mathbin{\raise1.8ex\rlap{\kern0.6ex\rule{0.6ex}{0.1ex}}\rlap{\kern1.1ex\rule{0.1ex}{1.9ex}}\raise-0.3ex\hbox{$\smile$} } }
\renewcommand{\nonfork}[1][M]{\mathrel{\nonforkSymbol_{#1}}}

\uline{Notazione}:
\begin{itemize}
\item per \(k \in \N\), \((k] = \set{1,\dots,k}\)
\item per \(I\) insieme e \(\lambda\) \href{20250203161341-cardinali.org}{cardinale}, si indica con
\begin{equation*}
  \binom{I}{\lambda} = I^{(\lambda)} =  \set{J \subseteq I \mid \card{J} = \lambda}.
\end{equation*}
l'insieme dei \href{20250131155822-operazioni_insiemistiche_tra_classi_mk.org}{sottoinsiemi} di \(I\) di \href{20241213101756-cardinalita.org}{cardinalità} fissata.
\end{itemize}

\begin{definizione}
Una \uline{colorazione} per un insieme \(M^{(n)}\) è una funzione
\begin{equation*}
f: M^{(n)} \to (k]
\end{equation*}
per qualche \(n,k \in \omega\) fissato.
\end{definizione}

\begin{definizione}
Dato un insieme \(M\) ed una colorazione \(f:M^{(n)}\to (k]\), \(H \subseteq M\) è \uline{monocromatico} se la \href{20250205170515-restrizione_di_una_classe.org}{restrizione} \(f \upharpoonright H^{(n)}\) è costante.
\end{definizione}

\begin{thm}
Per ogni \(M\) insieme infinito e per ogni colorazione
\begin{equation*}
f: \binom{M}{n}\to (k]
\end{equation*}
esiste \(H \subseteq M\) infinito e monocromatico.
\end{thm}

\begin{proof}
Sia \(\L\) il \href{20250130162057-linguaggio_del_prim_ordine.org}{linguaggio del prim'ordine} contenente \(k\) relazioni \(n\)-arie:
\begin{equation*}
\L = \set{r_{i} \mid i \in (k]}
\end{equation*}
e sia \(\M\) la \(\L\)-struttura di dominio \(M\), in cui
\begin{equation*}
r_{i}^{\M} \coloneqq \set{(a_{1},\dots,a_{n}) \mid f\big(\set{a_{1},\dots,a_{n}}\big) = i}.
\end{equation*}

Sia \(\U \succeq \mathcal{M}\)\footnote{con ``\(\preceq\)'' si intende che \(\mathcal{M}\) è \href{20250212102253-sottostruttura_elementare.org}{Sottostruttura elementare} di \(\U\).} un \href{20250617095548-modello_lambda_saturo.org}{modello saturo} \href{20250617102733-modello_mostro.org}{mostro}. Si noti che esiste una \href{20250131103317-formula_del_prim_ordine.org}{formula del prim'ordine} che afferma che \(r_{i}(x_{1},\dots,x_{n})\) rappresentano una colorazione di \(M^{(n)}\) (ovvero che le relazioni sono simmetriche e sono un ``ricoprimento''), e pertanto, per elementarietà, questo vale anche in \(\U\).

Sia \(x\) una variabile di lunghezza 1 e sia \(q(x)\) il tipo \(x\notin M\). Questo è \href{20250212164424-tipo_teoria_dei_modelli.org}{finitamente soddisfacibile}, e \href{20251026195832-estensione_tipo_finitamente_soddisfacibile_in_un_insieme_a_tipo_massimale_e_completo.org}{pertanto} esiste un \href{20250617102733-modello_mostro.org}{tipo globale} \(p(x) \supseteq q(x)\), \(p(x) \in S(\U)\) finitamente soddisfacibile in \(M\). \href{20251026195746-tipo_finitamente_soddisfacibile_e_invariante.org}{Inoltre} \(p(x)\) è invariante su \(M\).

Sia \(\overline{c} = \langle c_{i} \mid i<\omega \rangle\) una \href{20251026211057-sequenza_di_coeredi.org}{sequenza di coeredi di \(p(x)\) su \(M\)}.

In particolare, siccome \(\overline{c}\) è un \href{20251026210840-sequenza_di_morley.org}{sequenza di Morley} su \(p(x)\) invariante su \(M\), \href{20251026211005-sequenza_di_morely_inviariante_e_indiscernibile.org}{allora} è una \href{20251026210908-sequenza_di_indiscernibili.org}{sequenza di indiscernibili} su \(M\). Pertanto per ogni \(I,J \in \omega^{(n)}\)
\begin{equation*}
c \restricted{I} \equivalentover{M} c \restricted{J}.
\end{equation*}
e dunque, per ogni \(i=1,\dots,k\)
\begin{equation*}
r_{i}(c\restricted{I}) \IFF r_{i}(c \restricted{J})
\end{equation*}
Segue che \(\set{c_{i} \mid i <\omega}\) è monocromatico (WLOG di colore 1). Si osservi però che questa sequenza è in \(\U\), non in \(M\).

Si vuole costruire per induzione una sequenza \(\overline{a} = \langle a_{i} : i <\omega \rangle\) in \(M\) monocromatica di colore 1.

Per induzione su \(i\): si supponga per ipotesi induttiva che tutte le sottosequenze di lunghezza \(n\) di
\begin{equation*}
(a\restricted{i}) \concat (c\restricted{n}),
\end{equation*}
dove ``\(\concat\)'' rappresenta la concatenazione tra \href{20250206170922-sequenze_e_stringhe.org}{stringhe}, abbiano lo stesso colore 1. Si cerca quindi \(a_{i} \in M\) in maniera che
\begin{equation*}
(a\restricted{i}) \textcolor{red}{\concat a_{i}} \concat (c\restricted{n}) = (a\restricted{i+1}) \concat (c\restricted{n})
\end{equation*}
abbia la stessa proprietà.

Siccome \(\overline{c}\) è sequenza di indiscernibili su \(M\)\footnote{Ovvero per ogni \(I,J \in \omega^{(n)}\)
\begin{equation*}
c \restricted{I} \equivalentover{M} c \restricted{J}.
\end{equation*}} in particolare
\begin{equation*}
(a\restricted{i}) \concat (c\restricted{n}) \concat c_{n}
\end{equation*}
ha tutte le sottostringhe di lunghezza \(n\) di colore 1 (poiché \(a\restricted{i} \subseteq M\)).

In particolare esiste una formula \(\varphi \in \L\) tale che \(\varphi(a\restricted{i};\, c \restricted{n};\, c_{n})\) affermi che tutte le sottosequenze di lunghezza \(n\) della stringa abbiano colore 1.
Questa formula per sua natura è simmetrica nelle variabili libere.

Per la \href{20251026211057-sequenza_di_coeredi.org}{caratterizzazione delle sequenze di coeredi} si ha che
\begin{equation*}
c_{n} \nonfork c \restricted{n}
\end{equation*}
e dunque per \href{20251029154447-tuple_indipendenti_rispetto_ad_un_modello_monster_model.org}{definizione} esiste \(\bm{a} \in M\) tale che
\begin{equation*}
\varphi(a\restricted{i};\, c \restricted{n};\, \textcolor{red}{\bm{a}})
\end{equation*}
e, per l'osservazione di cui sopra,
\begin{equation*}
\varphi(a\restricted{i};\, \bm{a};\, c \restricted{n}).
\end{equation*}

Quindi la stringa
\begin{equation*}
(a \restricted{i}) \concat \bm{a} \concat (c\restricted{n})
\end{equation*}
è monocromatica di colore 1. Ponendo \(a_{i} \coloneqq \bm{a}\) si ha la tesi.
\end{proof}
\end{document}
