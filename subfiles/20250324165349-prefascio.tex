% Intended LaTeX compiler: pdflatex
\documentclass[../main]{subfiles}


\begin{document}

\section{Prefascio}
\label{sec:org2278b8f}
Sia \(X\) uno \href{20250103145124-topologia.org}{spazio topologico}.

\begin{definizione}
Sia \(\mathcal{C}\) una \href{20241126100904-categoria.org}{categoria}. Un \uline{prefascio} \(\mathcal{F}\) su \(X\) a valori in \(\mathcal{C}\) è il dato di:
\begin{enumerate}
\item per ogni \(U \subseteq X\) \href{20250103145124-topologia.org}{aperto}, un oggetto \(\mathcal{F}(U) \in \mathcal{C}\) (a volte si indica anche con \(\Gamma(\mathcal{F},U)\));
\item per ogni \(V \subseteq U \subseteq X\) aperti, un \href{20241126100904-categoria.org}{morfismo} \(\rho_{V}^{U} \in \mathcal{C}\)
\begin{equation*}
 \rho_{V}^{U}: \mathcal{F}(U)\to \mathcal{F}(V)
\end{equation*}
tale che:
\begin{enumerate}
\item \(\mathcal{F}(\emptyset)\) è l'\href{20250324174714-oggetto_terminale_di_una_categoria.org}{oggetto terminale} di \(\mathcal{C}\);
\item per ogni \(U \subseteq X\) aperto: \(\rho_{U}^{U} = \Id_{\mathcal{F}(U)}\);
\item per ogni \(W \subseteq V \subseteq U \subseteq X\) aperti:
  \begin{equation*}
\rho_{W}^{U} = \rho_{W}^{V}\circ \rho_{V}^{U}
  \end{equation*}
\end{enumerate}
\end{enumerate}
\end{definizione}
\uline{Nomenclatura}:
\begin{itemize}
\item Gli elementi di \(\mathcal{F}(U)\) si dicono \uline{sezioni di \(\mathcal{F}\) su \(U\)}.
\item Gli elementi di \(\mathcal{F}(X)\) si dicono \uline{sezioni globali} di \(\mathcal{F}\).
\item I morfismi \(\rho_{V}^{U}\) si dicono \uline{morfismi di restrizione} e per ogni \(s \in \mathcal{F}(U)\) si denota:
\begin{equation*}
  \rho_{V}^{U}(s) \eqqcolon \restriction{s}{V}
\end{equation*}
\end{itemize}
\subsection{Prefascio di gruppi}
\label{sec:org4ce0819}
\begin{definizione}
Un \uline{prefascio} \(\mathcal{F}\) di \href{20241205141146-gruppo_abeliano.org}{gruppi} su \(X\) è il dato di:
\begin{enumerate}
\item per ogni \(U \subseteq X\) \href{20250103145124-topologia.org}{aperto}, un \uline{gruppo} denotato con \(\mathcal{F}(U)\);
\item per ogni \(V \subseteq U \subseteq X\) aperti, un \href{20241206115531-morfismo_di_gruppi.org}{morfismo di gruppi}:
\begin{equation*}
 \rho_{V}^{U}: \mathcal{F}(U)\to \mathcal{F}(V)
\end{equation*}
tale che:
\begin{enumerate}
\item \(\mathcal{F}(\emptyset)=\set{0}\) è il \href{20250324165859-gruppo_banale.org}{gruppo banale};
\item per ogni \(U \subseteq X\) aperto: \(\rho_{U}^{U} = \Id_{\mathcal{F}(U)}\)\footnote{\(\Id\) è la \href{20250310111151-funzione_identita.org}{funzione identità}};
\item per ogni \(W \subseteq V \subseteq U \subseteq X\) aperti:\footnote{Vedi anche ``\href{20250310111151-funzione_identita.org}{Funzione identità}''}
  \begin{equation*}
\rho_{W}^{U} = \rho_{W}^{V}\circ \rho_{V}^{U}.
  \end{equation*}
\end{enumerate}
\end{enumerate}
\end{definizione}
\end{document}
