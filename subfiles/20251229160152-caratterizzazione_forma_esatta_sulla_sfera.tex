% Intended LaTeX compiler: pdflatex
\documentclass[../main]{subfiles}


\begin{document}

\section{Caratterizzazione forma esatta sulla sfera}
\label{sec:org823cd07}
\end{document}
