% Intended LaTeX compiler: pdflatex
\documentclass[../main]{subfiles}


\begin{document}

\section{Operazione su una classe MK}
\label{sec:org84ce61d}
Una \uline{operazione finitaria} (o \uline{funzione finitaria}) su una \href{20250130104320-classe_mk.org}{classe} \(X\) è una \href{20250202170607-classe_relazione_binaria.org}{Classe-Funzione}\footnote{Vedi ``\href{20250131183735-prodotto_cartesiano_di_classi_mk.org}{Potenza di una classe}''}
\begin{equation*}
f:X^{n}\to X.
\end{equation*}

\(n \in N\) viene denotato con \(\operatorname{ar}(f)\) ed è l'\uline{arietà} di \(f\).
\end{document}
