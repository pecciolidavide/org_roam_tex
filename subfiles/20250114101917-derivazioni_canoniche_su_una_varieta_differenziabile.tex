% Intended LaTeX compiler: pdflatex
\documentclass[../main]{subfiles}


\begin{document}

\section{Derivazioni Canoniche su una varietà differenziabile}
\label{sec:orgfde1073}
Sia \(M\) una \href{20250113115909-struttura_differenziabile.org}{varietà differenziabile} di dimensione \(n\), e sia \(p \in M\) fissato. Sia \((U,\varphi)\) una \href{20250111092123-varieta_topologica.org}{carta} su \(M\), con \(p \in U\).

\begin{definizione}
Per ogni \(f_{p} \in C^{\infty}_{p}\) (vedi \href{20250114100254-germi_di_funzioni.org}{Germi di funzioni}), di cui \(f\) è un \href{20250113110148-relazione_di_equivalenza.org}{rappresentante} in un \href{20250111142313-intorno.org}{intorno} \(V\) di \(p\), e per ogni \(k=1,\dots,n\) si definisce la \href{20250114101437-derivazione_su_una_varieta_differenziabile.org}{derivazione} \(v_{k}\):
\begin{equation*}
v_{k}(f_{p}) \coloneqq \restriction{\dpd{(f\circ \varphi^{-1})}{{x^{k}}}}{\varphi(p)}
\end{equation*}
Le \(v_{k}\) sono dette derivazioni canoniche.
\end{definizione}

Questa è una \href{20250114103236-derivata_parziale.org}{derivata parziale} su \(\R^{n}\):
\begin{equation*}
\begin{tikzcd}[ampersand replacement=\&]
	\& \R \\
	{M\supseteq U\cap V} \\
	\& {\R^n}
	\arrow["f", from=2-1, to=1-2]
	\arrow["\varphi"', from=2-1, to=3-2]
	\arrow["{f\circ\varphi^{-1}}"', from=3-2, to=1-2]
\end{tikzcd}
\end{equation*}

Inoltre \(v_{k}(f_{p})\) è ben definita; infatti due rappresentanti di \(f_{p}\), \(f,g\), coincidono in un intorno di \(p\), e quindi \(f\circ\varphi^{-1},g\circ\varphi^{-1}\) coincidono in un intorno di \(\varphi(p)\), e quindi hanno la stessa derivata.

\uline{Notazione}: Se \(\varphi = (x^{1},\dots,x^{n})\), allora si scrive
\begin{equation*}
v_{k} \coloneqq \restriction{\dpd{}{{x^{k}}}}{p}
\end{equation*}
\end{document}
