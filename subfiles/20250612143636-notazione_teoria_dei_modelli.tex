% Intended LaTeX compiler: pdflatex
\documentclass[../main]{subfiles}


\begin{document}

\section{Notazione della TEORIA DEI MODELLI}
\label{sec:org2fce1bf}
\subsection{Variabili}
\label{sec:orgd2b5fc8}

Le variabili saranno indicate spesso con un'unica lettera, anche se sono una \href{20250206170922-sequenze_e_stringhe.org}{stringa}: \(x = \langle x_{i}:i <\alpha\rangle\) dove \(\alpha\) è un \href{20250203111003-ordinali.org}{ordinale} qualsiasi.

Se \(M\) è una struttura qualsiasi, si indicherà con \(M^{x}\) il \href{20250131183735-prodotto_cartesiano_di_classi_mk.org}{prodotto cartesiano} di tante copie di \(M\) quanto la \href{20241213101756-cardinalita.org}{cardinalità} di \(\alpha\)
\begin{equation*}
M^{x}\coloneqq M^{\card{\alpha}}
\end{equation*}
\subsection{Funzioni}
\label{sec:org244fb1f}

Se non diversamente specificato, le funzioni \uline{sono parziali}.

Se \(f\) è una funzione in una variabile e \(\alpha=\langle \alpha_{i}: i<\lambda\rangle\) è una tupla, con \(f(\alpha)\) si intende la tupla:
\begin{equation*}
f(\alpha) \coloneqq \langle f(\alpha_{i}):i <\lambda\rangle
\end{equation*}

Allo stesso modo se \(b=\langle b_{i}\mid i<\card{x}\rangle\), \(c=\langle c_{i}\mid i<\card{x}\rangle\)
\begin{equation*}
\set{\langle b,c\rangle}\coloneqq\set{\langle b_{i},c_{i}\rangle\mid i<\card{x}}.
\end{equation*}
\subsection{Insiemi definiti da una formula o da un tipo}
\label{sec:org0f87681}

Se \(\varphi(x)\) è una \href{20250131103317-formula_del_prim_ordine.org}{formula} e \(p(x)\) è un \href{20250212164424-tipo_teoria_dei_modelli.org}{tipo}, allora, per ogni modello \(M\), si indicano nel seguente modo gli \href{20250131122913-soddisfazione_di_una_formula.org}{insiemi di verità}:
\begin{align*}
\varphi(M^{x}) &\coloneqq \set{a \in M^{x}\mid M\vDash \varphi[a]};
p(M^{x}) &\coloneqq \set{a \in M^{x}\mid M\vDash p[a]}.
\end{align*}
(vedi \href{20250131122913-soddisfazione_di_una_formula.org}{Soddisfazione di una formula} e \href{20250212164424-tipo_teoria_dei_modelli.org}{Soddisfazione di un tipo})
\subsection{Insieme delle formule}
\label{sec:org6c95d2e}

Se \(\mathcal{L}\) è un \href{20250130162057-linguaggio_del_prim_ordine.org}{linguaggio del prim'ordine}, allora si denota con \(\mathcal{L}\) \uline{anche} l'insieme delle \(\mathcal{L}\) \href{20250131103317-formula_del_prim_ordine.org}{formule}.

In particolare, se \(A\) è un insieme di parametri, si indica con \(\mathcal{L}(A)\) l'insieme di tutte le \(\mathcal{L}\)-formule a \href{20250212102927-enunciato_con_parametri.org}{parametri in \(A\)}.
\subsection{Cardinalità di un linguaggio del prim'ordine}
\label{sec:orga5a8517}
Sia \(\mathcal{L}\) un \href{20250130162057-linguaggio_del_prim_ordine.org}{linguaggio del prim'ordine}. Si denota con \(\card{\mathcal{L}}\) la \href{20241213101756-cardinalita.org}{cardinalità} di
\begin{equation*}
\mathcal{L}_{\text{fun}}\cup\mathcal{L}_{\text{rel}\cup \mathcal{L}_{\text{const}}}\cup \omega
\end{equation*}
(vedi \href{20250203161110-numeri_naturali_sono_ordinali.org}{Ordinale omega} e \href{20250131155822-operazioni_insiemistiche_tra_classi_mk.org}{Unione di classi MK})

Quindi, utilizzando questa notazione, non c'è alcuna ambiguità nello scrivere: \(\card{L}\); in virtù del \href{20251026160523-insieme_delle_formula_del_prim_ordine.org}{lemma}, questa è la cardinalità dell'insieme delle formule \uline{e} anche di
\begin{equation*}
\mathcal{L}_{\text{fun}}\cup\mathcal{L}_{\text{rel}}\cup \omega
\end{equation*}

QUESTA ROBA NON LA VOGLIAMO
\end{document}
