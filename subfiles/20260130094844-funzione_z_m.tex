% Intended LaTeX compiler: pdflatex
\documentclass[../main]{subfiles}


\begin{document}

\section{Mappa di potenza complessa}
\label{sec:orgf28a579}
Consideriamo, per \(m \in \N\), \(m\ge {1}\):
\begin{align*}
\phi : \C &\longrightarrow \C\\
z &\longmapsto z^{m}
\end{align*}
\begin{prop}
Per ogni \(z_{0}\in \C\setminus \set{0}\), esiste \(U\) \href{20250111142313-intorno.org}{intorno} \href{20250103145124-topologia.org}{aperto} di \(z_{0}\) tale che la \href{20250205170515-restrizione_di_una_classe.org}{restrizione} \(\restriction{\phi}{U}\) sia \href{20241219101956-funzione_iniettiva.org}{iniettiva}.
\end{prop}
\begin{proof}
È noto che:\footnote{Vedi ``\href{20260130100707-differenza_di_potenze.org}{Scomposizione della differenza di potenze}''} per ogni \(w,z \in \C\):
\begin{equation*}
w^{m}-z^{m} = (w-z)\left(\sum_{i=1}^{m} z^{m-i}\cdot w^{i-1}\right).
\end{equation*}
Si consideri quindi
\begin{equation*}
P(w,z) \coloneqq \sum_{i=1}^{m} z^{m-i}\cdot w^{i-1}
\end{equation*}
\begin{itemize}
\item Questo è un \href{20241231112750-polinomio.org}{Polinomio}, \href{20260130101500-polinomi_sono_continui.org}{quindi} è \href{20250103103252-funzione_continua.org}{continuo}, ed in particolare \href{20250306140014-funzione_continua_in_un_punto.org}{continuo in ogni punto}.
\item \(P(z_{0},z_{0})\) è
\begin{equation*}
  \sum_{i=1}^{m} z_{0}^{m-i}\cdot z_{0}^{i-1} = \sum_{i=1}^{m} z_{0}^{m-1} = m\, z_{0}^{m-1}
\end{equation*}
e siccome \(m \neq 0\), \(P(z_{0},z_{0}) \neq 0\).
\end{itemize}

Quindi esiste \(V \subseteq \C^{2}\) tale che per ogni \((w,z) \in V\): \(P(w,z) \neq 0\), ed esiste\footnote{\href{20250301193341-sistema_fondamentale_di_intorni.org}{Sistema fondamentale di intorni} e \href{20250109154723-topologia_prodotto.org}{Topologia prodotto}} \(U \subseteq \C\) intorno di \(z_{0}\) tale che \(V\supseteq U\times U\).

Siano quindi ora \(z_{1},z_{2} \in U\) tali che \(\phi(z_{1})=\phi(z_{2})\): si ha che \((z_{1},z_{2}) \in V\) e
\begin{equation*}
0 = (z_{1}^{m}-z_{2}^{m}) = (z_{1}-z_{2}) \cdot \parentesi{\neq 0}{P(z_{1},z_{2})}
\end{equation*}
e quindi \(z_{1}=z_{2}\).
\end{proof}
\end{document}
