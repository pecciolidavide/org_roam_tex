% Intended LaTeX compiler: pdflatex
\documentclass[../main]{subfiles}

\usepackage[hyperref]{biblatex}
\date{}
\title{}
\begin{document}

\section{Zeri di un ideale generato in uno spazio affine}
\label{sec:org9623552}
\subsection{Proposizione}
\label{sec:org18e822a}
Sia (\K) un \href{20241231112713-campo_algebricamente_chiuso.org}{campo algebricamente chiuso}, e sia \(T \subseteq \K[x_{1},\dots,x_{n}]\) (vedi \href{20241219113434-anello_dei_polinomi.org}{Anello-dei-polinomi}).
Allora
\begin{equation*}
V(T) =V\left(\(\langle\) T\(\rangle\) \right) \subseteq \A\textsuperscript{n}
\end{equation*}[fn:1]
\subsubsection{Dimostrazione}
\label{sec:org16d11c9}
Infatti:
\begin{enumerate}
\item \(T \subseteq \langle T\rangle\) e dunque \(V(T)\supseteq V\left(\langle T\rangle\right)\) (poiché meno equazioni significano più soluzsSiioni);
\item Sia \(x \in V(T)\), allora per ogni \(t \in T\) si ha \(t(x)=0\). Sia ora \(p \in \langle T \rangle\), allora \(p=\sum a_{i}t_{i}\) per \(a_{i} \in \K[x_{1},\dots,x_{n}]\) e \(t_{i} \in T\), e dunque \(p(x)=0\). Pertanto \(x \in V\left(\langle T\rangle\right)\). Dunque \(V(T) \subseteq V\left(\langle T\rangle\right)\).
\end{enumerate}
\subsection{Corollario}
\label{sec:org185cc97}
Sia \(X\) una \href{20241231114256-varieta_algebrica_affine.org}{Varietà Algebrica Affine}. Allora \(X=V(T_{0})\), dove \(T_{0} \subseteq \K[x_{1},\dots,x_{n}]\) è un insieme \textbf{finito}
\subsubsection{Dimostrazione}
\label{sec:org2d9bca6}

Sia \(T \subseteq \K[x_{1},\dots,x_{n}]\) tale che \(X=V(T)\). Per la proposizione, \(X=V\left(\langle T \rangle \right)\). Inoltre, per il \href{20250102165420-teorema_della_base_di_hilbert.org}{Teorema della Base di Hilbert}, \(\K[x_{1},\dots,x_{n}]\) è un \href{20250102115942-anello_noetheriano.org}{Anello Noetheriano} e, pertanto, per il corollario \href{20250102165143-ideali_generati_di_un_anello_noetheriano.org}{Ideali Generati di un Anello Noetheriano}, \(\langle T\rangle \subseteq \K[x_{1},\dots,x_{n}]\) è \href{20241219113154-ideale_generato.org}{finitamente generato}.
\begin{equation*}
\langle T \rangle = (t_{1},\dots,t_{k})
\end{equation*}
Applicando nuovamente la proposizione, \(V\left(\langle T\rangle\right) = V(t_{1},\dots,t_{k})\). Ponendo \(T_{0}=\set{t_{1},\dots,t_{k}}\) si ha la tesi.
\end{document}
