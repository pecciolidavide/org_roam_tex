% Intended LaTeX compiler: pdflatex
\documentclass[../main]{subfiles}

\usepackage[hyperref]{biblatex}
\date{}
\title{}
\begin{document}

\section{Teoria dei grafi}
\label{sec:orge4a0c86}
Sia \(\mathcal{L}=\set{r}\) un \href{20250130162057-linguaggio_del_prim_ordine.org}{linguaggio del prim'ordine} con un \href{20250130162057-linguaggio_del_prim_ordine.org}{simbolo di relazione} \href{20250130162057-linguaggio_del_prim_ordine.org}{binaria}.

La \uline{teoria dei grafi} \(T_{\text{gph}}\) è la \href{20250130114950-teoria_del_prim_ordine.org}{teoria del prim'ordine} dotata dei seguenti \href{20250131123109-insieme_di_assiomi_per_una_teoria.org}{assiomi}:
\begin{itemize}
\item \uline{irriflessiva}: \(\forall\,x\ \lnot\,r(x,x)\);
\item \uline{simmetrica}: \(\forall\,x \forall\, y\ \left[r(x,y)\,\implies\, r(y,x)\right]\).
\end{itemize}
\subsection{Grafo}
\label{sec:org648fbd6}
Un \uline{grafo} è un \href{20250131122945-modello_di_un_insieme_di_formule.org}{modello} \(M\) di \(T_{\text{gph}}\).
\subsubsection{Vertice di un grafo}
\label{sec:orgc00073c}
Gli elementi di \(M\) sono detti \uline{vertici} o \uline{nodi del grafo}
\subsubsection{Lato di un grafo}
\label{sec:org47727e1}
Un \uline{lato} di \(M\) è una coppia non ordinata \(\set{a,b} \subseteq M\) tale che
\begin{equation*}
M\vDash r(a,b)
\end{equation*}
(vedi \href{20250131122913-soddisfazione_di_una_formula.org}{Soddisfazione di una formula})
\subsection{Teoria dei grafi aleatori}
\label{sec:org32d4d65}
La \uline{teoria dei grafi aleatori} è la \(\mathcal{L}\)-\href{20250130114950-teoria_del_prim_ordine.org}{teoria}: \(T_{\text{rg}} \coloneqq T_{\text{gph}}\cup\set{\operatorname{nt}}\cup \set{ \operatorname{r}_{n}\ |\ n \in \N\setminus\set{0}}\), dove
\begin{align*}
\operatorname{nt}&: & &\exists\,x\exists\, y\ (x\neq y)\\
\operatorname{r}_{n} &: & &\left(\bigwedge_{i,j = 1}^{n} x_{i}\neq y_{j}\right)\,\implies\, \exists\,z \ \left[
\bigwedge_{i = 1}^{n} \left[
r(x_{i},z) \,\land\, \lnot\,r(z,y_{i}) \,\land\, z\neq y_{i}
\right]
\right]
\end{align*}

L'assioma \(r_{n}\) garantisce che per ogni coppia di insiemi di vertici \(A\), \(B\), disgiunti e di cardinalità minore di \(n\), esista un elemento \(z\) collegato a tutti i vertici di \(A\) e a nessun vertice di \(B\); inoltre, si richiede che \(z\) non sia in \(B\) (il fatto che \(z\) non sia in \(A\) segue dal fatto che \(z\) sia collegato a tutti i suoi elementi).
\subsubsection{Grafo aleatorio}
\label{sec:org4f5c8cf}
Un \uline{grafo aleatorio} è un \href{20250131122945-modello_di_un_insieme_di_formule.org}{modello} di \(T_{\text{rg}}\).
\end{document}
