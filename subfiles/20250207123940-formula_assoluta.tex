% Intended LaTeX compiler: pdflatex
\documentclass[../main]{subfiles}


\begin{document}

\section{Assolutezza di una formula}
\label{sec:org68b86cf}
Siano \(\mathcal{M},\mathcal{N}\) due \(\mathcal{L}\)-\href{20250131103035-struttura_del_prim_ordine.org}{strutture} tali che \(\mathcal{M} \subseteq \mathcal{N}\) sia una \href{20250131103212-sottostruttura_del_prim_ordine.org}{sottostruttura}.

\begin{definizione}
Una \(\mathcal{L}\)-\href{20250131103317-formula_del_prim_ordine.org}{formula} \(\varphi(x_{1},\dots,x_{n})\) si dice:
\begin{itemize}
\item \uline{assoluta verso l'alto} tra \(\mathcal{M}\) e \(\mathcal{N}\) se per ogni \(a_{1},\dots,a_{n} \in \mathcal{M}\) se\footnote{Con ``\(\vDash\)'' si intende la \href{20250131123011-conseguenza_logica.org}{Conseguenza logica}}
\begin{equation*}
  \mathcal{M}\vDash \varphi[a_{1},\dots,a_{n}]
\end{equation*}
allora \(\mathcal{N}\vDash\varphi[a_{1},\dots,a_{n}]\).
\item \uline{assoluta verso il basso} tra \(\mathcal{M}\) e \(\mathcal{N}\) se per ogni \(a_{1},\dots,a_{n} \in \mathcal{M}\) se
\begin{equation*}
  \mathcal{N}\vDash \varphi[a_{1},\dots,a_{n}]
\end{equation*}
allora \(\mathcal{M}\vDash\varphi[a_{1},\dots,a_{n}]\).
\item \uline{assoluta} tra \(\mathcal{M}\) e \(\mathcal{N}\) se per ogni \(a_{1},\dots,a_{n} \in \mathcal{M}\)
\begin{equation*}
  \mathcal{M}\vDash\varphi[a_{1},\dots,a_{n}]\quad\iff\quad\mathcal{N}\vDash \varphi[a_{1},\dots,a_{n}]
\end{equation*}
\end{itemize}
\end{definizione}
\subsection{Nella teoria degli insiemi}
\label{sec:orge177e9d}

Nell'ambito della teoria degli insiemi, utilizzando assiomatizzazioni che consentano di parlare di classi (come \href{20250130104245-morse_kelly_set_theory.org}{MK}), le \href{20250130104320-classe_mk.org}{classi proprie} non possono essere \href{20250131103035-struttura_del_prim_ordine.org}{l'universo} di una \href{20250131103035-struttura_del_prim_ordine.org}{struttura del prim'ordine}. Ciò nonostante, questi concetti possono ancora essere sfruttati: si consideri infatti il linguaggio del prim'ordine \(\mathcal{L}_{\in} = \set{\in}\). Un qualsiasi \href{20250130104331-insieme_mk.org}{insieme} \href{20250131161811-insieme_vuoto_mk.org}{non vuoto} \(M\) può essere visto, in un certo senso, come una ``\href{20250131103212-sottostruttura_del_prim_ordine.org}{sottostruttura}'' della \href{20250203104513-classe_totale.org}{classe totale \(\operatorname{V}\)} (infatti \(M \subseteq \operatorname{V}\), e inoltre il significato di \(\in\) è lo stesso).

Ha quindi senso dare questa definizione.
\begin{definizione}
Una \(\mathcal{L}_{\in}\)-\href{20250131103317-formula_del_prim_ordine.org}{formula} \(\varphi(x_{1},\dots,x_{n})\) si dice:
\begin{itemize}
\item \uline{assoluta verso l'alto tra \(M\) e \(V\)} se
\begin{equation*}
  \forall a_{1},\dots,a_{n} \in M\ \bigg(\big(\langle M,\in\rangle\vDash \varphi[a_{1},\dots,a_{n}]\big)\implies \varphi(a_{1},\dots,a_{n})\bigg)
\end{equation*}

\item \uline{assoluta verso il basso tra \(M\) e \(V\)} se
\begin{equation*}
  \forall a_{1},\dots,a_{n} \in M\ \bigg(\varphi(a_{1},\dots,a_{n})\implies\big(\langle M,\in\rangle\vDash \varphi[a_{1},\dots,a_{n}]\big)\bigg)
\end{equation*}
\item \uline{assoluta tra \(M\) e \(V\)} se
\begin{equation*}
  \forall a_{1},\dots,a_{n} \in M\ \bigg(\big(\langle M,\in\rangle\vDash \varphi[a_{1},\dots,a_{n}]\big)\iff\varphi(a_{1},\dots,a_{n})\bigg)
\end{equation*}
\end{itemize}
\end{definizione}
\end{document}
