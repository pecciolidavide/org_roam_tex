% Intended LaTeX compiler: pdflatex
\documentclass[../main]{subfiles}


\begin{document}

\section{Fascio nucleo}
\label{sec:org5b47db9}
Siano \(\mathcal{F},\mathcal{G}\) due \href{20250324174728-fascio.org}{fasci} sullo \href{20250103145124-topologia.org}{spazio topologico} \(X\), e sia \(\varphi: \mathcal{F}\to \mathcal{G}\) un \href{20250325180613-morfismo_di_prefasci.org}{morfismo di fasci}.

\begin{definizione}
Il \textbf{fascio nucleo} di \(\varphi\), indicato con \(\ker\varphi\), è definito ponendo per ogni \(U \subseteq X\) aperto:\footnote{Vedi ``\href{20241213105201-kernel.org}{Kernel}''}
\begin{equation*}
(\ker \varphi) (U) \coloneqq \ker \varphi_{U} \subseteq \mathcal{F}(U).
\end{equation*}
\end{definizione}

\begin{prop}
\(\ker \varphi\) è \href{20250325150647-sottoprefascio.org}{sottofascio} di \(\mathcal{F}\).
\end{prop}

\begin{proof}
Bisogna dimostrare le seguenti cose.
\begin{enumerate}
\item \((\ker \varphi)(U) \subseteq \mathcal{F}(U)\) è \href{20241206143051-sottogruppo.org}{sottogruppo}.
\item Dobbiamo dimostrare che \(\ker\varphi\) \href{20250324165349-prefascio.org}{definisca effettivamente un prefascio}.

Siano quindi \(W \subseteq U \subseteq X\) aperti. Se \(f \in (\ker \varphi)(U) = \ker \varphi_{U}\), devo dimostrare che \(\restriction{f}{W} \in (\ker \varphi)(W) = \ker \varphi_{W}\).

Ma \(\varphi_{U} f = 0\), quindi
\begin{equation*}
 0=\restriction{\varphi_{U} f}{W} = \varphi_{W} \restriction{f}{W}
\end{equation*}
e pertanto \(\restriction{f}{W} \in \ker\varphi_{W} = (\ker\varphi)(W)\).

\item Si dimostra che \(\ker\varphi\) soddisfi l'\href{20250324174728-fascio.org}{assioma di esistenza di fascio}. Non è necessaria l'\href{20250324174728-fascio.org}{unicità}, in quanto i sottoprefasci la soddisfano sempre.

Sia \(U \subseteq X\) aperto, \(\set{U_{i}}\) ricoprimento aperto di \(U\). Se \(f \in \mathcal{F}(U)\) è tale che
\begin{equation*}
 \forall  i:\qquad \restriction{f}{U_{i}} \in (\ker\varphi)(U_{i})
\end{equation*}
vogliamo dimostrare che \(f \in (\ker\varphi)(U)\).

Si consideri quindi \(\varphi_{U}(f) \in \mathcal{G}({U})\):
\begin{equation*}
 \forall i: \qquad \restriction{\varphi_{U}(f)}{U_{i}} = \varphi_{U_{i}}(\restriction{f}{U_{i}}) = 0
\end{equation*}
e pertanto, siccome \(\mathcal{G}\) è un fascio, allora \(\varphi_{U}(f) = 0\).
\qedhere
\end{enumerate}
\end{proof}
\end{document}
