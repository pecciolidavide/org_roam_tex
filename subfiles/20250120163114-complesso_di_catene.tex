% Intended LaTeX compiler: pdflatex
\documentclass[../main]{subfiles}


\begin{document}

\section{Complesso di catene}
\label{sec:orgf383c77}
\subsection{Complesso di catene di \(R\)-moduli}
\label{sec:org384e207}

Sia \(R\) un \href{20241205141119-anello.org}{anello} commutativo con unità.

\begin{definizione}
Un complesso di catene è una \href{20250206170922-sequenze_e_stringhe.org}{successione} di coppie
\begin{equation*}
\mathcal{C}_{\bullet} = \set{(C_{n}, \partial_{n})}_{n \in \Z}
\end{equation*}
tali che
\begin{itemize}
\item \(C_{n}\) sia un \href{20241205141053-r_moduli.org}{\(R\)-modulo};
\item \(\partial_{n}:C_{n}\longrightarrow C_{n-1}\) è un \href{20241206115416-morfismi_r_moduli.org}{morfismo di moduli tale} che
\begin{equation*}
  \partial_{n}\circ\partial_{n-1} \equiv 0
\end{equation*}
(se e solo se \(\operatorname{Im}\partial_{n} \subseteq \operatorname{ker}\partial_{n-1}\)).
\end{itemize}

Gli elementi di \(C_{n}\) si dicono \(n\)-catene.
\end{definizione}
\end{document}
