% Intended LaTeX compiler: pdflatex
\documentclass[../main]{subfiles}


\begin{document}

\section{Teorema della mappa aperta (superfici di Riemann)}
\label{sec:org9267143}
Siano \(X, Y\) due \href{20260127112828-superficie_di_riemann.org}{superfici di Riemann}.

\begin{thm}
Se \(F:X\to Y\) è una \href{20260128143822-funzione_olomorfa_su_una_superficie_di_riemann.org}{funzione olomorfa} non costante, allora \(F\) è \href{20250104114559-funzione_chiusa.org}{aperta}.
\end{thm}

\begin{proof}
Sia \(U \subseteq X\) aperto, e sia \(x_{0} \in U\). Mostriamo che \(F(x_{0})\) è \href{20250122181431-parte_interna.org}{interno} a \(F(U)\)\footnote{Vedi ``\href{20250202190147-immagine_punto_a_punto_di_due_classi.org}{Immagine e retroimmagine tramite una funzione}'' .}.

Si scelgano due carte locali:
\begin{itemize}
\item \(U_{1} \subseteq U\) connesso, \(x_{0} \in U_{1}\):
\begin{equation*}
  \varphi_{1}:U_{1}\to V_{1}.
\end{equation*}
\item \(\varphi_{2}: U_{2} \to V_{2}\), con \(F(x_{0}) \in U_{2}\).
\end{itemize}

Si definisce \(f\coloneqq \varphi_{2}\circ F\circ \varphi_{1}^{-1} : V_{1}\to \C\), \href{20260126110551-funzione_olomorfa.org}{olomorfa} per definizione, con \(V_{1}\) \href{20250103165325-spazio_topologico_connesso.org}{connesso} perché \href{20250111142332-omeomorfismo.org}{omeomorfo} ad \(U_{1}\).

Quindi, per il \href{20260128182932-teorema_della_mappa_aperta_analisi_complessa.org}{teorema della mappa aperta}, \(f\) è costante oppure \(f\) è aperta.
\begin{itemize}
\item Se \(f\) è costante, allora\footnote{Perché \(\varphi_{2}\) è una biiezione.} \(F\) è costante in \(U_{1}\), e quindi per il \href{20260128144016-principio_di_identita_per_funzioni_olomorfe_su_superfici_di_riemann.org}{Principio di identità} \(F\) è costante in \(X\). Assurdo.
\end{itemize}

Quindi \(f\) è aperta, e pertanto \(f[V_{1}]\) aperto in \(\C\), con \(f(\varphi_{1}(x_{0})) \in f[V_{1}]\). Quindi
\begin{equation*}
U_{2}\supseteq \varphi_{2}^{-1}[f[V_{1}]] = F[\varphi_{1}^{-1}[V_{1}]] = F[U_{1}]
\end{equation*}
è aperto in \(U_{2}\) e contiene \(F(x_{0})\).
\end{proof}
\end{document}
