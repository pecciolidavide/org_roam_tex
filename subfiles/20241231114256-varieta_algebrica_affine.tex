% Intended LaTeX compiler: pdflatex
\documentclass[../main]{subfiles}

\usepackage[hyperref]{biblatex}
\date{}
\title{}
\begin{document}

\section{Varietà Algebrica Affine}
\label{sec:org9c67d64}
\subsection{Definizione di base}
\label{sec:org3cdd339}
Sia \(\K\) un \href{20241231112713-campo_algebricamente_chiuso.org}{Campo Algebricamente Chiuso} e sia \(T \subseteq \K[x_{1},\dots,x_{n}]\) (Vedi \href{20241219113434-anello_dei_polinomi.org}{Anello-dei-polinomi}). Una \textbf{varietà algebrica affine} è il \href{20241231112823-radici_polinomiali.org}{luogo degli zeri} di \(T\):  \(X=V(T)\), dove
\begin{equation*}
V(T)=\set{p \in \A^{n}: f(p)=0\ \forall\,f \in T}
\end{equation*}
(vedi \href{20241231114009-spazio_affine.org}{Spazio Affine})

Alla luce delle proprietà degli \href{20250102180050-zeri_di_un_ideale_generato_in_uno_spazio_affine.org}{Zeri di un ideale generato in uno spazio affine}, si può dare la seguente definizione:
\subsection{Definizione più complessa}
\label{sec:org21c8980}
Sia \(\K\) un \href{20241231112713-campo_algebricamente_chiuso.org}{campo algebricamente chiuso}. \(Y \subseteq \A^{n}\) si dice \textbf{varietà algebrica affine} se esiste un ideale \(I \subseteq \K[x_{1},\dots,x_{n}]\) tale che
\begin{equation*}
Y=V(I)
\end{equation*}
(vedi \href{20241231112823-radici_polinomiali.org}{Luogo di zeri})
\subsection{{\bfseries\sffamily TODO} Notazione\hfill{}\textsc{matematica\_lm:geo\_alg}}
\label{sec:orgccd3c29}
Spesso, quando si parla di varietà affine, si fa riferimento ad una \href{20250107112123-varieta_algebrica_quasi_proiettiva_qp.org}{varietà QP} \href{20250107112412-morfismo_tra_varieta_algebriche_qp.org}{isomorfa come varietà QP} ad un \href{20250103101459-topologia_di_zariski_affine.org}{chiuso algebrico} di \(\A^{n}\).
\subsection{{\bfseries\sffamily {[}?]} Osservazione\hfill{}\textsc{geo\_alg:matematica\_lm}}
\label{sec:org804614e}
Una varietà algebrica affine può incontrare una retta in due modi possibili:
\begin{itemize}
\item la varietà contiene la retta;
\item la varietà e la retta hanno un numero \textbf{finito} di intersezioni.
\end{itemize}

Infatti,
\end{document}
