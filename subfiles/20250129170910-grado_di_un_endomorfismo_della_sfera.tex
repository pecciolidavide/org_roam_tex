% Intended LaTeX compiler: pdflatex
\documentclass[../main]{subfiles}


\begin{document}

\section{Grado di un endomorfismo della sfera}
\label{sec:org8f45209}
Sia \(R=\Z\) un \href{20241219112842-pid.org}{PID} e sia \(\mathds{S}^{n}\) la \href{20250115150754-sfera_n_dimensionale.org}{sfera \(n\)-dimensionale}
\begin{definizione}
Sia \(f: \mathds{S}^{n} \longrightarrow \mathds{S}^{n}\) una \href{20250103103252-funzione_continua.org}{funzione continua}. Allora applicando il \href{20241205131958-funtore.org}{funtore} \href{20250123115927-funtore_di_omologia_singolare.org}{di omologia singolare}, questa induce un \href{20241206115416-morfismi_r_moduli.org}{morfismo}
\begin{equation*}
f_{\star}: H_{n}(\mathds{S}^{n}) \longrightarrow H_{n}(\mathds{S}^{n})
\end{equation*}
\href{20250127162702-calcolo_dell_omologia_singolare_della_sfera_e_dell_omologia_singolare_relativa_del_disco_rispetto_alla_sfera.org}{ovvero} \(f_{\star} : \Z \longrightarrow \Z\) \href{20241206115531-morfismo_di_gruppi.org}{morfismo di gruppi}, tale che \(f_{\star}(x) = m\cdot x\) per qualche \(m \in \Z\).
Si definisce il grado di \(f\) come
\begin{equation*}
\operatorname{deg}f\coloneqq m
\end{equation*}
\end{definizione}
\begin{lem}
Siano \(f,g: \mathds{S}^{n}\longrightarrow\mathds{S}^{n}\) \href{20250103103252-funzione_continua.org}{funzioni continue}.
\begin{enumerate}
\item Se \(f\sim g\) \href{20250121094654-omotopia_tra_funzioni_continue.org}{funzioni omotope}, allora \(\operatorname{deg} f= \operatorname{deg}g\);
\item Se \(f \sim \Id\), allora \(\operatorname{deg}f = 1\);
\item Se \(f\) è omotopa alla funzione costante, allora \(\operatorname{deg}f = 0\);
\item \(\operatorname{deg}(f\circ g) = \operatorname{deg}f\cdot \operatorname{deg}g\)
\end{enumerate}
\end{lem}

\begin{proof}
\begin{enumerate}
\item Se due funzioni sono omotope, allora \href{20250122155528-teorema_dell_invarianza_per_omotopia.org}{inducono la stessa mappa in omologia}.
\item Per funtorialità, \((\Id_{\mathcal{S}^{n}})_{\star} = \Id_{\Z}\) e pertanto \(\deg \Id = 1\).
\item Se \(g\) è costante, allora \(g\) fattorizza come segue:
\begin{equation*}
\begin{tikzcd}[ampersand replacement=\&]
        \& {\set{p}} \\
        {\mathds{S}^n} \&\& {\mathds{S}^n}
        \arrow[hook, from=1-2, to=2-3]
        \arrow["g", from=2-1, to=1-2]
        \arrow["g"', from=2-1, to=2-3]
\end{tikzcd}
\end{equation*}
e per funtorialità
\begin{equation*}
\begin{tikzcd}[ampersand replacement=\&]
        \& {H_n(\set{p}) = 0} \\
        {H_n(\mathds{S}^n)} \&\& {H_n(\mathds{S}^n)}
        \arrow[hook, from=1-2, to=2-3]
        \arrow["0", from=2-1, to=1-2]
        \arrow["g"', from=2-1, to=2-3]
\end{tikzcd}
\end{equation*}
ottenendo che \(g_{\star} = 0\), \(\deg g = 0\).
\item Applicando il \href{20241205131958-funtore.org}{funtore} \href{20250123115927-funtore_di_omologia_singolare.org}{di omologia singolare}:
\begin{equation*}
\begin{tikzcd}[ampersand replacement=\&,row sep=tiny]
        {\mathds{S}^n} \& {\mathds{S}^n} \& {\mathds{S}^n} \\
        \\
        \\
        \\
        {H_n(\mathds{S}^n)} \& {H_n(\mathds{S}^n)} \& {H_n(\mathds{S}^n)} \\
        x \& {\deg g\cdot x} \\
        \& y \& {\deg f \cdot y} \\
        x \&\& {(\deg g)\cdot(\deg f)\cdot x}
        \arrow["g", from=1-1, to=1-2]
        \arrow["f", from=1-2, to=1-3]
        \arrow[Rightarrow, from=1-2, to=5-2]
        \arrow["{g_\star}", from=5-1, to=5-2]
        \arrow["{f_\star}", from=5-2, to=5-3]
        \arrow[maps to, from=6-1, to=6-2]
        \arrow[maps to, from=7-2, to=7-3]
        \arrow["{(f\circ g)_{\star}}", maps to, from=8-1, to=8-3]
\end{tikzcd}
\end{equation*}
e pertanto la tesi.\qedhere
\end{enumerate}
\end{proof}
\subsection{Grado dell'endomorfismo di riflessione sulla sfera}
\label{sec:orgc635bfc}
\begin{prop}
Sia
\begin{align*}
r: \mathds{S}^{n} &\longrightarrow \mathds{S}^{n}\\
(x_{1},\dots,x_{n+1}) &\longmapsto (x_{1},\dots,x_{n},-x_{n+1})
\end{align*}
Allora \(\deg r = -1\)
\end{prop}

\begin{proof}
Consideriamo i due insiemi
\begin{align*}
A^{+} &= \set{(x_{0},\dots,x_{n}) \in \mathds{S}^{n} \mid x_{n}\ge 0 }\\
A^{-} &= \set{(x_{0},\dots,x_{n}) \in \mathds{S}^{n} \mid x_{n}\le 0 }\\
E &= A^{+} \cap A^{-}
\end{align*}
Allora \(E\) è \href{20250111142332-omeomorfismo.org}{omeomorfo} alla \href{20250115150754-sfera_n_dimensionale.org}{sfera \(\mathds{S}^{n-1}\)} e \(\restriction{r}{E} = \Id_{E}\).

Se consideriamo ora, per \(\varepsilon>0\):
\begin{align*}
A^{+}_{\varepsilon} &= \set{(x_{0},\dots,x_{n}) \in \mathds{S}^{n} \mid x_{n} > - \varepsilon }\\
A^{-} &= \set{(x_{0},\dots,x_{n}) \in \mathds{S}^{n} \mid x_{n}< \varepsilon }\\
B &= A^{+}_{\varepsilon} \cap A^{-}_{\varepsilon}
\end{align*}
allora \((\mathds{S}^{n}, A^{+}_{\varepsilon}, A^{-}_{\varepsilon}, B)\) soddisfa \href{20250128132648-teorema_di_mayer_vietoris.org}{Mayer-Vietoris}, e inoltre la mappa
\begin{equation*}
\Id: (\mathds{S}^{n}, A^{+}, A^{-}, E)\longrightarrow (\mathds{S}^{n}, A^{ +}_{\varepsilon}, A^{-}_{\varepsilon}, B)
\end{equation*}
dà origine a tutti \href{20250122155727-retratto_di_deformazione_di_uno_spazio_topologico.org}{retratti di deformazione}.

\textbf{Si ripercorre la \href{20250128132648-teorema_di_mayer_vietoris.org}{dimostrazione del Teorema di Mayer-Vietoris}}.
Ottengo le due \href{20250120183640-sec_di_complessi_di_catene.org}{SEC}:\footnote{Vedi \href{20260203110150-complesso_di_catene_somma.org}{somma} e \href{20260204100611-somma_diretta_di_complessi_di_catene.org}{somma diretta} di \href{20250120163114-complesso_di_catene.org}{complessi di catene}.}
\begin{equation*}
\begin{tikzcd}[ampersand replacement=\&]
	0 \& {\mathcal{S}_{\bullet}(E)} \& {\mathcal{S}_{\bullet}(A^+)\oplus \mathcal{S}_{\bullet}(A^-)} \&\& {\mathcal{S}_{\bullet}(\mathds{S}^n)} \& 0 \\
	\\
	0 \& {\mathcal{S}_{\bullet}(E)} \& {\mathcal{S}_{\bullet}(A^-)\oplus \mathcal{S}_{\bullet}(A^+)} \&\& {\mathcal{S}_{\bullet}(\mathds{S}^n)} \& 0
	\arrow[from=1-1, to=1-2]
	\arrow[from=1-2, to=1-3]
	\arrow["{(j_1)_{\diesis} - (j_2)_{\diesis}}", from=1-3, to=1-5]
	\arrow[from=1-5, to=1-6]
	\arrow[from=3-1, to=3-2]
	\arrow[from=3-2, to=3-3]
	\arrow["{(j_2)_{\diesis} - (j_1)_{\diesis}}"', from=3-3, to=3-5]
	\arrow[from=3-5, to=3-6]
\end{tikzcd}
\end{equation*}
Inoltre \(r\) \href{20250122154136-funtorialita_dell_omologia_singolare.org}{induce} tutte le mappe\footnote{Vedi la \href{20260204100611-somma_diretta_di_complessi_di_catene.org}{somma diretta di morfismi}} che rendono il diagramma commutativo:
\begin{equation*}
\begin{tikzcd}[ampersand replacement=\&]
	0 \& {\mathcal{S}_{\bullet}(E)} \& {\mathcal{S}_{\bullet}(A^+)\oplus \mathcal{S}_{\bullet}(A^-)} \&\& {\mathcal{S}_{\bullet}(\mathds{S}^n)} \& 0 \\
	\\
	0 \& {\mathcal{S}_{\bullet}(E)} \& {\mathcal{S}_{\bullet}(A^-)\oplus \mathcal{S}_{\bullet}(A^+)} \&\& {\mathcal{S}_{\bullet}(\mathds{S}^n)} \& 0
	\arrow[from=1-1, to=1-2]
	\arrow[from=1-2, to=1-3]
	\arrow["{(\restriction{r}{E})_{\diesis}}"', from=1-2, to=3-2]
	\arrow["{(j_1)_{\diesis} - (j_2)_{\diesis}}", from=1-3, to=1-5]
	\arrow["{(\restriction{r}{A^+})_{\diesis}\oplus (\restriction{r}{A^-})_{\diesis}}", from=1-3, to=3-3]
	\arrow[from=1-5, to=1-6]
	\arrow["{(r)_{\diesis}}", from=1-5, to=3-5]
	\arrow[from=3-1, to=3-2]
	\arrow[from=3-2, to=3-3]
	\arrow["{(j_2)_{\diesis} - (j_1)_{\diesis}}"', from=3-3, to=3-5]
	\arrow[from=3-5, to=3-6]
\end{tikzcd}
\end{equation*}
e per \href{20250120165029-funtore_tra_chr_e_rmod.org}{funtorialità} + \href{20250120164938-zig_zag_lemma.org}{Zig-Zag Lemma}
\begin{equation*}
\begin{tikzcd}[ampersand replacement=\&,sep=scriptsize]
	{H_n(A^+)\oplus H_n (A^-)} \&\& {H_n(\mathds{S}^n)} \&\& {H_{n-1}(E)} \&\& {H_{n-1}(A^+)\oplus H_{n-1} (A^-)} \\
	\\
	{H_n(A^-)\oplus H_n (A^+)} \&\& {H_n(\mathds{S}^n)} \&\& {H_{n-1}(E)} \&\& {H_{n-1}(A^-)\oplus H_{n-1} (A^+)}
	\arrow["{j_1-j_2}", from=1-1, to=1-3]
	\arrow["{r_\star}"', from=1-1, to=3-1]
	\arrow["{\partial_\star}", from=1-3, to=1-5]
	\arrow["{r_\star}"', from=1-3, to=3-3]
	\arrow[from=1-5, to=1-7]
	\arrow["\Id", from=1-5, to=3-5]
	\arrow["{r_\star}"', from=1-7, to=3-7]
	\arrow["{j_2-j_1}"', from=3-1, to=3-3]
	\arrow["{\partial_\star'}"', from=3-3, to=3-5]
	\arrow[from=3-5, to=3-7]
\end{tikzcd}
\end{equation*}
Per il \href{20250120164938-zig_zag_lemma.org}{lemma}, \(\partial_{\star}' = -\partial_{\star}\).

Siccome \(A^{+}, A^{-}\) sono \href{20250122155640-spazio_topologico_contraibile.org}{contraibili}, \href{20250122155700-spazio_topologico_contraibile_e_aciclico.org}{allora}
\begin{equation*}
H_{n}(A^{ +}) = H_{n}(A^{-}) = 0 = H_{n-1}(A^{ +}) = H_{{n-1}}(A^{-})
\end{equation*}
e \href{20250120130155-caratterizzazione_di_alcune_successioni_esatte_di_r_moduli.org}{quindi} \(\partial_{\star}\) e \(\partial_{\star}'\) sono isomorfismi:
\begin{align*}
r_{\star} &= (\partial_{\star}')^{-1} \circ \Id \circ \partial_{\star} = (\partial_{\star}')^{-1} \circ \Id \circ (- \partial_{\star}') \\
&= - (\partial_{\star}')^{-1}\circ (\partial_{\star}') = - \Id.\qedhere
\end{align*}
\end{proof}
\subsubsection{Grado della mappa antipodale sulla sfera}
\label{sec:org1bcf003}
\begin{cor}
La mappa antipodale
\begin{align*}
a: \mathds{S}^{n} &\longrightarrow \mathds{S}^{n}\\
x &\longmapsto -x
\end{align*}
ha grado \(\deg a = (-1)^{n+1}\).
\end{cor}
\end{document}
