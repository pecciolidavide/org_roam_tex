% Intended LaTeX compiler: pdflatex
\documentclass[../main]{subfiles}


\begin{document}

\section{Anello Noetheriano}
\label{sec:org61bcd60}
\subsection{Definizioni}
\label{sec:orge567cc0}
Un \href{20241205141119-anello.org}{anello commutativo con unità} \(A\) si dice \uline{Noetheriano} (vedi \href{20250102120132-emmy_noether.org}{Emmy Noether}) se vale una delle seguenti proprietà equivalenti:
\begin{enumerate}
\item Ogni suo \href{20241219112955-ideale.org}{Ideale} \(I\) è \href{20241219113154-ideale_generato.org}{finitamente generato}
\item Ogni \uline{catena ascendente di ideali}
\begin{equation*}
   I_{1} \subseteq I_{2} \subseteq \dots \subseteq I_{j} \subseteq
\end{equation*}
è \href{20250102120655-catena_stazionaria.org}{\uline{stazionaria}}, ovvero esiste \(n\) tale per cui per ogni \(k>0\)
\begin{equation*}
 I_{n+k}=I_{n}
\end{equation*}
(vedi \href{20250102120722-acc.org}{ACC})
\end{enumerate}
\subsection{Dimostrazione}
\label{sec:org53275e0}
Le due condizioni di cui sopra sono equivalenti
\subsubsection{\(1\implies 2\)}
\label{sec:org1268d5e}

Sia \(I_{1} \subseteq I_{2} \subseteq \dots\subseteq I_{j} \subseteq \dots\) una catena ascendente di ideali. Sia
\begin{equation*}
I \coloneqq \bigcup_{j}I_{j}
\end{equation*}
\(I\) è ancora un ideale, e per 1. si ha che \(I=(r_{1},\dots,r_{n})\) per \(r_{i} \in I\), poiché \(I\) è finitamente generato. Per ogni \(k=1,\dots,n\) sia \(j_{k}\) tale che
\begin{equation*}
r_{j_{k}} \in I_{j_{k}}
\end{equation*}
Sia dunque \(N= \max \set{j_{k}}_{k}\). Si ha che \(r_{1},\dots,r_{n}\in I_{N}\), e pertanto \((r_{1},\dots,r_{n})\subseteq I_{N}\) e \(I_{N} \subseteq I\).

Dunque \(I_{N}=I\) e, per ogni \(k\ge 0\) si ha
\begin{equation*}
I_{N+m}\supseteq I_{N}=I,\qquad I_{N+m} \subseteq I
\end{equation*}
e pertanto \(I_{N+m}=I_{N}\).
\subsubsection{\(2\implies 1\)}
\label{sec:orgb5fc71e}

Osservazione: se \(I\) è un ideale non finitamente generato, e \(r_{1},\dots,r_{n} \in I\), allora anche \(I\setminus (r_{1},\dots,r_{n})\) è non finitamente generato. Infatti, se per assurdo \(I\setminus(r_{1},\dots,r_{n})\) fosse generato da \(t_{1},\dots,t_{k}\), allora \(I\) sarebbe generato da \(t_{1},\dots,t_{k},r_{1},\dots,r_{n}\).

Sia \(I\) un ideale \textbf{non} finitamente generato, e sia \(a_{1}\in I\) tale che \((a_{1}) \subsetneqq I\). Allora \(I \setminus (a_{1})\) è non finitamente generato.
Sia \(a_{2} \in I \setminus (a_{1})\) tale che \((a_{1},a_{2})\subsetneqq I\). Allora \(I\setminus(a_{1},a_{2})\) è non finitamente generato.

Continuando in questo modo si crea una catena di ideali
\begin{equation*}
(a_{1})\subsetneqq (a_{1},a_{2}) \subsetneqq (a_{1},a_{2},a_{3}) \subsetneqq ...
\end{equation*}
non stazionaria.
\end{document}
