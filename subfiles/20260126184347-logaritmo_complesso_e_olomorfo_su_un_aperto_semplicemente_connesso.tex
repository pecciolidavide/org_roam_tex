% Intended LaTeX compiler: pdflatex
\documentclass[../main]{subfiles}


\begin{document}

\section{Logaritmo complesso è olomorfo su un aperto semplicemente connesso}
\label{sec:orga00a977}
\begin{prop}
Se \(U \subseteq \C\) è un \href{20250103145124-topologia.org}{aperto} \href{20250113100451-spazio_topologico_semplicemente_connesso.org}{semplicemente connesso} tale che \(0 \notin U\), allora
\begin{equation*}
\log : U\to \C
\end{equation*}
è una funzione \href{20260126110551-funzione_olomorfa.org}{olomorfa}.
\end{prop}

\begin{proof}
Consideriamo \(f(z) = 1/z\). Questa è una funzione olomorfa su \(U\), \href{20260126184314-funzione_olomorfa_su_un_aperto_semplicemente_connesso_ammette_primitiva.org}{e quindi esiste \(g\) una primitiva di \(f\)}, \(g:U\to \C\) olomorfa. Facendo la \href{20250114110703-derivata.org}{derivata}:
\begin{equation*}
\dod{}{z}\big(\frac{1}{z} e^{g(z)}\big) = \frac{1}{z} e^{g(z)}g'(z) - \frac{1}{z^{2}} e^{g}(z) = 0.
\end{equation*}
\href{20260127100741-spazio_topologico_semplicemente_connesso_e_connesso.org}{Siccome} \(U\) è \href{20250103165325-spazio_topologico_connesso.org}{connesso}, \href{20251222142956-teorema_della_derivata_nulla.org}{allora} \(\frac{1}{z} e^{g(z)} = \lambda > 0\) costante, ovvero
\begin{equation*}
e^{g(z)} = \lambda z.
\end{equation*}
Segue che \(e^{-\log\lambda\, g(z)} = z\), \(-\log\lambda\,g(z)\) \href{20260126110551-funzione_olomorfa.org}{olomorfa}.
\end{proof}
\section{Funzione olomorfa su aperto semplicemente connesso ammette un logaritmo}
\label{sec:org69d3a1a}
\begin{prop}
Se \(U\) è un \href{20250103145124-topologia.org}{aperto} \href{20250113100451-spazio_topologico_semplicemente_connesso.org}{semplicemente connesso} e \(f:U\to \C\) è olomorfa e tale che \(f(x) \neq 0\) per ogni \(x\), allora esiste \(g:U\to \C\) olomorfa tale che
\begin{equation*}
e^{g(z)} = f(z).
\end{equation*}
\end{prop}
\begin{proof}
È sufficiente prendere una \href{20260126184314-funzione_olomorfa_su_un_aperto_semplicemente_connesso_ammette_primitiva.org}{primitiva} per \(f'(z)/f(z)\).
\end{proof}
\end{document}
