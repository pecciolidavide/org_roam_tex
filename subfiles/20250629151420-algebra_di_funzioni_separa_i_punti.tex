% Intended LaTeX compiler: pdflatex
\documentclass[../main]{subfiles}


\begin{document}

\begin{definizione}
Sia \(\mathcal{A}\) una \(\R\)-\href{20250110175552-algebra_su_un_campo.org}{algebra} di funzioni di dominio \(K \subseteq \R^{n}\) a valori in \(\R\).

Si dice che \(\mathcal{A}\) \uline{separa i punti} se per ogni \(x,y \in K\) esistono \(f,g \in \mathcal{A}\) tali che
\begin{equation*}
f(x) \neq f(y).
\end{equation*}
\end{definizione}
\begin{esempio}
Sia \(\mathcal{A}\) l'insieme dei polinomi definiti su \([a,b]\):
\begin{equation*}
\mathcal{A} \coloneqq \set{\restriction{f}{[a,b]}\mid f(x) = \sum_{k=0}^{n} c_{k}x^{k}\mid c_{k} \in \R, n \in \N}.
\end{equation*}
Ovviamente \(\mathcal{A}\) è una \(\R\)-\href{20250629165520-algebra_di_funzioni_reali.org}{algebra di funzioni}, e inoltre separa i punti, poiché \(f(x) = \Id_{[a,b]} \in \mathcal{A}\)\footnote{Vedi ``\href{20250310111151-funzione_identita.org}{Funzione identità}''}.

Inoltre \(1 \in \mathcal{A}\), e pertanto \(\mathcal{A}\) contiene tutte le funzioni costanti.
\end{esempio}
\end{document}
