% Intended LaTeX compiler: pdflatex
\documentclass[../main]{subfiles}


\begin{document}

\begin{definizione}
Siano \(\mathcal{C}, \mathcal{D}\) due \href{20241126100904-categoria.org}{categorie} e siano
\(F,G: \mathcal{C}\longrightarrow \mathcal{D}\)
due \textbf{\href{20241204222455-funtore_covariante.org}{funtori covarianti}}. Una \textbf{trasformazione naturale} da \(F\) a \(G\)
\begin{equation*}
T: F \longrightarrow G
\end{equation*}
è l'insieme dei morfismi \(T=(T_{X})_{X \in \operatorname{Ob}(\mathcal{C})}\)
\begin{equation*}
T_X:F(X) \longrightarrow G(X)
\end{equation*}
per ogni \(X\) oggetto in \(\operatorname{\mathcal{C}}\) tali che, per ogni \(X \dfreccia{f}Y\) in \(\mathcal{C}\) il diagramma seguente commuti
\begin{equation*}
\begin{tikzcd}[ampersand replacement=\&,sep=large]
	{F(X)} \& {F(Y)} \\
	{G(X)} \& {G(Y)}
	\arrow["{F(f)}", from=1-1, to=1-2]
	\arrow["{T_X}"', from=1-1, to=2-1]
	\arrow["{T_Y}", from=1-2, to=2-2]
	\arrow["{G(f)}", from=2-1, to=2-2]
\end{tikzcd}
\end{equation*}
ovvero
\begin{equation*}
T_Y \circ F(f) = G(f) \circ T_X.
\end{equation*}
\end{definizione}
\end{document}
