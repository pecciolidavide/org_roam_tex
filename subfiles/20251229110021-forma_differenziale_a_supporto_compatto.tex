% Intended LaTeX compiler: pdflatex
\documentclass[../main]{subfiles}


\begin{document}

\section{Forma differenziale a supporto compatto}
\label{sec:orgb911c3e}
Sia \(M\) una \href{20250113115909-struttura_differenziabile.org}{varietà differenziabile}.

\begin{definizione}
Si denota con \(A^{k}_{\text{c}}(M)\) l'insieme delle \href{20251115155511-forma_differenziale_in_un_punto.org}{\(k\)-forme differenziali} a \href{20251223153341-supporto_di_una_forma_differenziale.org}{supporto} \href{20250103163701-spazio_topologico_compatto.org}{compatto}:
\begin{equation*}
A^{k}_{\text{c}}(M) \subseteq A^{k} (M).
\end{equation*}
\end{definizione}

Queste formano uno \href{20251201163733-gruppo_abeliano_graduato.org}{spazio vettoriale graduato}: \(A^{\bullet}_{\text{c}}(M)\):\footnote{Vedi ``\href{20241213095808-somma_diretta.org}{Somma Diretta}''}
\begin{equation*}
A^{\bullet}_{\text{c}}(M) \coloneqq \bigoplus_{k \in \N} A^{k}_{\text{c}}(M)
\end{equation*}
nonché un \href{20251115182320-complesso_di_cocatene.org}{complesso di cocatene} con il \href{20251115160537-differenziale_di_una_forma.org}{differenziale}. Infatti, se \(\omega \in A^{k}_{\text{c}}(M)\), allora \(\dif \omega \in A^{k+1}_{\text{c}}(M)\), in quanto
\begin{equation*}
\operatorname{supp} \dif \omega \subseteq \operatorname{supp}\omega.
\end{equation*}
\end{document}
