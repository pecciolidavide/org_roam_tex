% Intended LaTeX compiler: pdflatex
\documentclass[../main]{subfiles}


\begin{document}

\section{Contrazione di un sottospazio topologico ad un punto}
\label{sec:org23c4374}
Sia \((X, \tau)\) uno \href{20250103145124-topologia.org}{spazio topologico} e sia \(A \subseteq X\) un \href{20250103163814-sottospazio_topologico.org}{sottospazio}.

\begin{definizione}
Si definisce la \href{20250113110148-relazione_di_equivalenza.org}{relazione di equivalenza} \(\sim\) su \(X\) che identifica tutti i punti di \(A\) in un unico punto, lasciando inalterati i punti di \(X \setminus A\):
\begin{equation*}
x \sim y \IFF (x = y) \lor (x \in A \land y \in A)
\end{equation*}

Lo spazio quoziente rispetto a questa relazione si denota con \(X/A\), dotat della \href{20250129155316-spazio_topologico_quoziente.org}{topologia quoziente} indotta dalla proiezione canonica \(\pi: X \to X/A: x\mapsto [x]_{\sim}\).
\end{definizione}

Gli elementi di \(X/A\) sono:
\begin{itemize}
\item I singoletti \(\{x\}\) per ogni \(x \in X \setminus A\);
\item Il punto speciale \(p_{A}\), (la classe di equivalenza che contiene ogni punto di \(A\)).
\end{itemize}

\begin{oss}
Un sottoinsieme \(U \subseteq X/A\) è aperto se e solo se \(\pi^{-1}(U)\) è aperto in \(X\).
Esplicitamente, gli aperti di \(X/A\) sono di due tipi:
\begin{enumerate}
\item Sottoinsiemi aperti di \(X \setminus A\) (che non contengono il punto \([A]\));
\item Insiemi della forma \(V \cup \{[A]\}\), dove \(V \subseteq X \setminus A\) è tale che \(V \cup A\) sia un aperto in \(X\) (ovvero intorni aperti del sottospazio \(A\) in \(X\)).
\end{enumerate}
\end{oss}
\end{document}
