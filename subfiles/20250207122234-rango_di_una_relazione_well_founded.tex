% Intended LaTeX compiler: pdflatex
\documentclass[../main]{subfiles}

\usepackage[hyperref]{biblatex}
\date{}
\title{}
\begin{document}

\section{Rango di una relazione}
\label{sec:org92848c8}
Contesto: \href{20250130104245-morse_kelly_set_theory.org}{Morse Kelly Set Theory}.
Sia \(X\) una \href{20250130104320-classe_mk.org}{classe}, e sia \(R\) una \href{20250202170607-classe_relazione_binaria.org}{relazione} su \(X\).

\begin{definizione}
Se \(R\) è \href{20250619161501-caratteristiche_delle_relazioni_binarie.org}{irriflessiva}, \href{20250203095749-relazione_left_narrow_mk.org}{regolare} e \href{20250203100901-relazione_well_founded_mk.org}{ben fondata}, allora si definisce la \href{20250202170607-classe_relazione_binaria.org}{classe funzione} \uline{rango} \(\bm{\varrho}_{R,X}\) che a \(x \in X\) associa \footnote{Vedi:
\begin{itemize}
\item \href{20250202124648-successore_di_un_insieme_mk.org}{Successore di un insieme}
\item \href{20250131155822-operazioni_insiemistiche_tra_classi_mk.org}{Unione}
\end{itemize}}
\begin{equation*}
\bm{\varrho}_{R,X}(x) = \bigcup \set{\operatorname{S}\left(\bm{\varrho}_{R,X}(y)\right) \ |\ y\mathrel{R}x}
\end{equation*}
\end{definizione}

Questa è ben definita per il \href{20250207121906-teorema_di_ricorsione.org}{Teorema di Ricorsione}.
\begin{prop}
\begin{enumerate}
\item La classe \(\operatorname{ran}\left(\bm{\varrho}_{R,X}\right)\)\footnote{Vedi: ``\href{20250202173528-dominio_range_e_campo_di_una_classe_relazione.org}{Range di una funzione}''} è un \href{20250206120526-segmento_iniziale_per_un_ordine.org}{segmento iniziale} di \(\operatorname{Ord}\)\footnote{Vedi: ``\href{20250203111003-ordinali.org}{Classe degli ordinali}''}, ovvero vale una delle seguenti.
\begin{itemize}
\item \(\operatorname{ran}\left(\bm{\varrho}_{R,X}\right) \in \operatorname{Ord}\);
\item \(\operatorname{ran}\left(\bm{\varrho}_{R,X}\right) = \operatorname{Ord}\);
\end{itemize}
in entrambi i casi, per ogni \(x \in X\), si ha che \(\bm{\varrho}_{R,X}(x) \in \operatorname{Ord}\).
\item Inoltre \(x\mathrel{R}y\) implica \(\bm{\varrho}_{R,X}(x) < \bm{\varrho}_{R,X}(y)\) e vale\footnote{Vedi: ``\href{20250203102516-massimo_e_minimo.org}{Infimum}''}
\begin{equation*}
\bm{\varrho}_{R,X}(x) = \inf\set{
\alpha \in \operatorname{Ord}\ |\ \forall\, y\ \left(y\mathrel{R}x\,\implies\, \bm{\varrho}_{R,X}(y)<\alpha\right)
}
\end{equation*}
\end{enumerate}
\end{prop}
Nella dimostrazione si mostrerà, in particolare, che se \(X\) è una classe propria, allora \(\operatorname{ran}(\bm{\varrho}_{R,X}) = \operatorname{Ord}\).
\begin{oss}
Si ha che \(\bm{\varrho}_{R,X}(x) = 0\) se e solo se \(x\) è \(R\)-\href{20250203102516-massimo_e_minimo.org}{minimale} in \(X\), e inoltre \(\bm{\varrho}_{R,X}(x) = \alpha\) se e solo se \(x\) è \(R\)-minimale rispetto a
\begin{equation*}
X\setminus\set{
y \in X\ |\ \bm{\varrho}_{R,X}(y)< \alpha
}
\end{equation*}
\end{oss}
\begin{esempio}
Si consideri la relazione \(R\) indicata in figura~\ref{fig:rangorelazione} sull'insieme \(X=\set{a,b,c,d,e,f,g}\), dove \(x\to y\) indica che \(x\mathrel{R}y\). Per semplicità, si scrive \(\bm{\varrho}(x)\) in luogo di \(\bm{\varrho}_{R,X}(x)\).
\begin{align*}
\bm{\varrho}(a) &= \bigcup\set{\operatorname{S}(\bm{\varrho}(y))\mid y\mathrel{R}a} = \bigcup \emptyset= \emptyset=0.\\
\bm{\varrho}(c) &= 0.\\
\bm{\varrho}(f) &= 0.\\
\bm{\varrho}(b) &= \bigcup\set{\operatorname{S}(\bm{\varrho}(y)) \mid y\mathrel{R}b}= \bigcup\set{\operatorname{S}(\bm{\varrho}(c))}=\\
&= \bigcup\set{\operatorname{S}(\emptyset)} = \bigcup\set{\set{\emptyset}} =  \set{\emptyset} = 1.\\
\bm{\varrho}(e) &= 1\\
\bm{\varrho}(d) &= \bigcup\set{
\operatorname{S}(\bm{\varrho}(e))
} = \bigcup\set{2} = 2.\\
\bm{\varrho}(g) &=\bigcup\set{\operatorname{S}(\bm{\varrho}(a)),\operatorname{S}(\bm{\varrho}(b)),\operatorname{S}(\bm{\varrho}(d))} = \bigcup\set{1,2,3}=3.
\end{align*}
Si ricorda inoltre la definizione di \href{20250202130045-insieme_dei_numeri_naturali_mk.org}{numeri naturali}:
\begin{equation*}
0\coloneqq\emptyset,\qquad 0\neq n\coloneqq\set{0,1,\dots,n-1}
\end{equation*}
\end{esempio}

\begin{figure}
\begin{equation*}
\begin{tikzcd}[ampersand replacement=\&,cramped]
	\& g \\
	a \& b \& d \\
	\& c \&\& e \\
	\&\&\& f
	\arrow[from=2-1, to=1-2]
	\arrow[from=2-2, to=1-2]
	\arrow[from=2-3, to=1-2]
	\arrow[from=3-2, to=2-2]
	\arrow[from=3-4, to=2-3]
	\arrow[from=4-4, to=3-4]
\end{tikzcd}
\end{equation*}
\caption{\label{fig:rangorelazione}Esempio di relazione sull'insieme \(X=\set{a,b,c,d,e,f,g}\)}
\end{figure}
\end{document}
