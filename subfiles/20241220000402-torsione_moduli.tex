% Intended LaTeX compiler: pdflatex
\documentclass[../main]{subfiles}


\begin{document}

Sia \(R\) un \href{20241219112842-pid.org}{PID}, e sia \(M\) un \href{20241205141053-r_moduli.org}{\(R\)-modulo}. 

\begin{definizione}
Per ogni \(m \in M\) si definisce \textbf{annullatore di \(m\)}:
\begin{equation*}
\operatorname{ann}_R (m) \coloneqq \set{a\in R: am=0} \subseteq R
\end{equation*}
\end{definizione}

Questo è un \href{20241219112955-ideale.org}{ideale} di \(R\). Siccome \(R\) è un \href{20241219112842-pid.org}{PID}, allora
\begin{equation*}
\operatorname{ann}_R (m) = (a)
\end{equation*}
per qualche \(a \in R\).

\begin{definizione}
La \textbf{torsione di \(M\)} è
\begin{equation*}
\operatorname{Tor}_R (M)=\set{m \in M: \operatorname{ann}_R(m)\neq \set{0}}.
\end{equation*}
\end{definizione}

La torsione, quindi, è l'insieme degli elementi del modulo che \textbf{hanno} annullatori non banali.
\end{document}
