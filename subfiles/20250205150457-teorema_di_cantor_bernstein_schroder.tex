% Intended LaTeX compiler: pdflatex
\documentclass[../main]{subfiles}


\begin{document}

\section{Teorema di Cantor-Bernstein-Schröder}
\label{sec:org25c2c85}
\textbf{\textbf{\uline{NOTA}: quando si parla di \uline{classi}, se ne parla nell'ambito della \href{20250130104245-morse_kelly_set_theory.org}{Morse Kelly Set Theory}; quando si parla di insiemi, il discorso ha validità più generale.}}
\subsection{Teorema}
\label{sec:orgae6ac52}
Se \(A,B\) sono due \href{20250130104331-insieme_mk.org}{insiemi} (o due \href{20250130104320-classe_mk.org}{classi}) e
\begin{equation*}
f: A\longrightarrow B,\qquad g:B\longrightarrow A
\end{equation*}
sono due \href{20241219101956-funzione_iniettiva.org}{funzioni iniettive} (o \href{20250202170607-classe_relazione_binaria.org}{classi funzioni} \href{20241219101956-funzione_iniettiva.org}{iniettiva}), allora esiste
\begin{equation*}
h:A\longrightarrow B
\end{equation*}
\href{20250104111707-funzione_biunivoca.org}{funzione biiettiva} (o classe funzione \href{20250104111707-funzione_biunivoca.org}{biiettiva}).
\end{document}
