% Intended LaTeX compiler: pdflatex
\documentclass[../main]{subfiles}


\begin{document}

\section{Funzioni ricorsive in più dimensioni}
\label{sec:org6ae4385}
Vedi \href{20250202130045-insieme_dei_numeri_naturali_mk.org}{Insieme dei numeri naturali MK}.

Le funzioni \uline{possono essere} \href{20250213105339-funzione_parziale.org}{PARZIALI}
\begin{definizione}
È possibile espandere la definizione di \href{20250202170607-classe_relazione_binaria.org}{funzione} (\href{20250215141024-funzioni_primitive_ricordive.org}{primitiva}) \href{20250207104855-funzione_ricorsiva.org}{ricorsiva} alle funzioni
\begin{equation*}
f: \N^{k}\to \N^{\ell}: x\mapsto \left(f_{1}(x),\dots,f_{\ell}(x)\right)
\end{equation*}
con \(\ell>1\): infatti, diremo che tale funzione è (primitiva) ricorsiva sse lo sono tutte le \(f_{i}=f\circ U_{i}\) per \(i=1,\dots,\ell\).
\end{definizione}
\begin{oss}
Si noti che:
\begin{enumerate}
\item Considerando la \href{20250215151413-biiezione_canonica_tra_n_e_n2.org}{biiezione canonica} \(\bm{J}^{\ell}: \N^{\ell}\to \N\) poiché le sue \href{20250111142446-funzione_inversa.org}{inverse} \((\cdot)^{\ell}_{i}\) per \(i=1,\dots,\ell\) sono \uline{ricorsive primitive}, si ha che
\begin{equation*}
 (\bm{J}^{\ell})^{-1}: \N\to \N^{\ell}: x\mapsto \left((x)^{\ell}_{1},\dots,(x)^{\ell}_{\ell}\right)
\end{equation*}
è una funzione ricorsiva primitiva.
\item Equivalentemente, la funzione
\begin{equation*}
 f: \N^{k}\to \N^{\ell}: x\mapsto \left(f_{1}(x),\dots,f_{\ell}(x)\right)
\end{equation*}
è ricorsiva primitiva se e solo se lo è la funzione
\begin{equation*}
 \bm{J}^{\ell}\circ f: \N^{k}\to \N.
\end{equation*}
\end{enumerate}
\end{oss}
\end{document}
