% Intended LaTeX compiler: pdflatex
\documentclass[../main]{subfiles}

\usepackage[hyperref]{biblatex}
\date{}
\title{}
\begin{document}

\section{Logical Game}
\label{sec:org5054d60}
\subsection{Definizione}
\label{sec:org0decaed}
Un \uline{gioco logico} è una quadrupla \(\mathcal{G} \coloneqq (\Omega, f, W_{\text{I}}, W_{\text{II}})\) dove:
\begin{itemize}
\item \(\Omega\) è un \href{20250130104331-insieme_mk.org}{insieme}, chiamato il \uline{dominio del gioco};
\item \(f:\Omega^{<\omega}\to \set{\text{I},\text{II}}\) è una \href{20250202170607-classe_relazione_binaria.org}{funzione}, chiamata \uline{funzione di turno} o \uline{funzione del giocatore};
\item \(W_{\text{I}},W_{\text{II}} \subseteq \Omega^{<\omega}\cup \Omega^{\omega}\) \footnote{\href{20250202192030-classe_delle_classi_funzioni.org}{Classe delle Classi-Funzioni} e \href{20250203161110-numeri_naturali_sono_ordinali.org}{Ordinale omega}} sono tali che
\begin{enumerate}
\item \(W_{\text{I}}\cap W_{\text{II}} = \emptyset\);
\item per ogni \(\bm{a} \in W_{\bullet}\) e per ogni \(\bm{b} \in\Omega^{<\omega}\cup \Omega^{\omega}\):
\end{enumerate}
\begin{equation*}
  \bm{a} \subseteq \bm{b}\quad\implies\quad \bm{b} \in W_{\bullet}
\end{equation*}
\end{itemize}

Gli elementi di \(\Omega^{<\omega}\) sono chiamati \uline{posizioni del gioco} \(\mathcal{G}\), mentre un elemento di \(\Omega^{\omega}\) è detto \uline{giocata} di \(\mathcal{G}\).

I giocatori I e II giocano scegliendo a turno elementi di \(\Omega\). La funzione di turno \(f\) associa a ciascuna posizione uno dei due giocatori: se
\begin{equation*}
f(a_{0},a_{1},\dots,a_{n}) = \text{I}
\end{equation*}
allora l'elemento \(a_{n+1}\) sarà scelto dal giocatore I.

Si dirà che il giocatore I \uline{vince la giocata \(\bm{a}\)} se \(\bm{a} \in W_{\text{I}}\); si dirà che il giocatore II \uline{vince la giocata \(\bm{b}\)} se \(\bm{b} \in W_{\text{II}}\).
\subsubsection{Gioco Logico totale}
\label{sec:org7b2a258}
Un gioco è detto \uline{totale} se \(\Omega^{\omega} \subseteq W_{\text{I}}\cup W_{\text{II}}\).
\subsubsection{Strategia per un gioco logico}
\label{sec:org5c2ab6d}
Si definiscano i seguenti insiemi:
\begin{align*}
\Omega^{<\omega}_{\text{I}} &\coloneqq \set{s \in \Omega^{<\omega}\mid f(s) = \text{I}}\\
\Omega^{<\omega}_{\text{II}} &\coloneqq \set{s \in \Omega^{<\omega}\mid f(s) = \text{II}}
\end{align*}

A strategy for a player is a set of rules that describe exactly how that player should choose, depending on how the two players have chosen at earlier moves.

Formalmente, una strategia per il giocatore \(j\) (con \(j=\text{I},\text{II}\)) è una funzione
\begin{equation*}
\varphi: \Omega_{j}^{<\omega} \to \Omega
\end{equation*}

A strategy for a player is said to be \uline{winning} if that player wins every play in which he or she uses the strategy, regardless of what the other player does.

Un gioco si dice \uline{determinato} se esiste una strategia vincente per I o per II.
\subsection{Bibliography}
\label{sec:org1527e1d}
\begin{itemize}
\item Hodges, Wilfrid and Jouko Väänänen, ``Logic and Games'', The Stanford Encyclopedia of Philosophy (Winter 2024 Edition), Edward N. Zalta \& Uri Nodelman (eds.), URL = \url{https://plato.stanford.edu/archives/win2024/entries/logic-games/}.
\end{itemize}
\end{document}
