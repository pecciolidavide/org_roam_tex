% Intended LaTeX compiler: pdflatex
\documentclass[../main]{subfiles}


\begin{document}

\section{Gruppo delle funzioni in a valore intero}
\label{sec:orgedf7c7c}
Sia \(X\) un \href{20250130104331-insieme_mk.org}{insieme} qualsiasi. Allora l'\href{20250202192030-classe_delle_classi_funzioni.org}{insieme delle funzioni}
\begin{equation*}
\Z^{X}\coloneqq \set{f: X\to \Z}
\end{equation*}
ha una \href{20250127093245-gruppo_abeliano.org}{struttura di gruppo abeliano} rispetto alla somma puntuale: per ogni \(f,g \in \Z^{X}\), si definisce \((f+g)\) come:
\begin{align*}
(f+g): X &\longrightarrow \Z\\
x &\longmapsto f(x)+g(x).
\end{align*}
\subsection{Supporto di una funzione a valore intero}
\label{sec:org723f898}
\begin{definizione}
Sia \(D \in \Z^{X}\). Si definisce il \textbf{supporto di \(D\)} come:
\begin{equation*}
\operatorname{supp} D \coloneqq \set{p \in X \mid D(p) \neq 0}.
\end{equation*}
\end{definizione}
\end{document}
