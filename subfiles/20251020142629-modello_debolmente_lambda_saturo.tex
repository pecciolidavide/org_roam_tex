% Intended LaTeX compiler: pdflatex
\documentclass[../main]{subfiles}


\begin{document}

Sia \(\mathcal{L}\) un \href{20250130162057-linguaggio_del_prim_ordine.org}{linguaggio del prim'ordine}, e sia \(\lambda\) un \href{20250203161341-cardinali.org}{cardinale} tale che \(\card{\mathcal{L}} + \omega < \lambda\).
\section{Modello debolmente lambda saturo}
\label{sec:org9b310e9}
Sia \(M\) una \(\mathcal{L}\)-\href{20250131103035-struttura_del_prim_ordine.org}{struttura} infinita.

\begin{definizione}
\(M\) è \uline{debolmente \(\lambda\)-saturo} se realizza ogni \href{20250212164424-tipo_teoria_dei_modelli.org}{tipo} \(p(x) \subseteq \mathcal{L}(\emptyset)\), \href{20250212164424-tipo_teoria_dei_modelli.org}{finitamente consistente} in \(M\), con \(\card{x}\le \lambda\).

\(M\) si dice \uline{debolmente saturo} se è debolmente \(\card{M}\)-saturo.
\end{definizione}
\section{Modello debolmente lambda saturo sse lambda universale}
\label{sec:org0d290c1}

Sia \(M\) una \(\mathcal{L}\)-\href{20250131103035-struttura_del_prim_ordine.org}{struttura} infinita.

\begin{prop}
Sono fatti equivalenti:
\begin{enumerate}
\item \(M\) è debolmente \(\lambda\)-saturo;
\item \(M\) è \href{20250213151951-modello_lambda_universale.org}{\(\lambda\)-universale}.
\end{enumerate}
\end{prop}

Questo teorema vale per \href{20250212172708-teorema_di_lowenheim_skolem_all_insu.org}{Teorema di Löwenheim-Skolem all'insù} e \href{20250212115524-teorema_di_lowenheim_skolem_all_ingiu.org}{Teorema di Löwenheim-Skolem all'ingiù}
\end{document}
