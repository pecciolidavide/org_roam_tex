% Intended LaTeX compiler: pdflatex
\documentclass[../main]{subfiles}

\usepackage[hyperref]{biblatex}
\date{}
\title{}
\begin{document}

\section{Dominio di una funzione}
\label{sec:orgf2892ca}
\subsection{Generalizzazione in MK}
\label{sec:org99408cf}

Contesto: \href{20250130104245-morse_kelly_set_theory.org}{Morse Kelly Set Theory}

Sia \(R\) una \href{20250202170607-classe_relazione_binaria.org}{classe relazione}.
\subsubsection{Dominio di una Classe-Relazione}
\label{sec:orgf329602}
Si definiscono il dominio, il range e il campo di \(R\) come segue:
\begin{align*}
\operatorname{dom}(R) &\coloneqq \set{x\,|\, \exists\, y\ (x,y) \in R}\\
\operatorname{ran}(R) &\coloneqq \set{y\,|\,\exists\, x\ (x,y) \in R}\\
\operatorname{fld}(R) &\coloneqq \operatorname{dom}(R)\cup \operatorname{ran}(R)
\end{align*}
(vedi \href{20250131155822-operazioni_insiemistiche_tra_classi_mk.org}{Unione di classi MK}), che esistono per l'Axiom of comprehension.
\begin{prop}
Se \(R\) è un \href{20250130104331-insieme_mk.org}{insieme}, allora anche \(\operatorname{dom}(R)\), \(\operatorname{ran}(R)\), \(\operatorname{fld}(R)\) sono insiemi.
\end{prop}
\end{document}
