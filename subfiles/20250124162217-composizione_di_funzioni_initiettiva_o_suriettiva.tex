% Intended LaTeX compiler: pdflatex
\documentclass[../main]{subfiles}


\begin{document}

\section{Composizione di funzioni initiettiva o suriettiva o biiettiva}
\label{sec:org96c4182}
Siano \(X,Y,Z\) tre insiemi e siano \(f,g\) due funzioni:
\begin{equation*}
\begin{tikzcd}[ampersand replacement=\&,cramped,column sep=3.15em]
	X \& Y \& Z
	\arrow["f", from=1-1, to=1-2]
	\arrow["g", from=1-2, to=1-3]
\end{tikzcd}
\end{equation*}

\begin{itemize}
\item Se \(f,g\) sono \href{20241219101956-funzione_iniettiva.org}{funzioni iniettive}, allora \(g\circ f\) è \href{20241219101956-funzione_iniettiva.org}{iniettiva};
\item Se \(f,g\) sono \href{20241213105600-funzione_suriettiva.org}{funzioni suriettive}, allora \(g\circ f\) è \href{20241213105600-funzione_suriettiva.org}{suriettiva};
\item Se \(f,g\) sono \href{20250104111707-funzione_biunivoca.org}{funzioni biiettive}, allora \(g\circ f\) è \href{20250104111707-funzione_biunivoca.org}{biiettiva};
\end{itemize}

Inoltre
\begin{itemize}
\item Se \(g\circ f\) è \href{20241213105600-funzione_suriettiva.org}{suriettiva}, allora \(g\) è \href{20241213105600-funzione_suriettiva.org}{suriettiva};
\item Se \(g\circ f\) è \href{20241219101956-funzione_iniettiva.org}{iniettiva}, allora \(f\) è \href{20241219101956-funzione_iniettiva.org}{iniettiva}.
\end{itemize}
\end{document}
