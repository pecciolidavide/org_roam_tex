% Intended LaTeX compiler: pdflatex
\documentclass[../main]{subfiles}


\begin{document}

\section{Teorema di invarianza omotopica per la coomologia di De Rham}
\label{sec:org689e17f}
\begin{thm}
Siano \(M,N\) \href{20250113115909-struttura_differenziabile.org}{varietà differenziabili}. Se sono \href{20250124155008-spazi_topologici_omotopicamente_equivalenti.org}{omotopicamente equivalenti}, allora hanno \href{20251115172442-gruppo_di_coomologia_di_de_rham.org}{coomologia di De Rham} \href{20251201160758-isomorfismo_tra_algebre_graduate.org}{isomorfa} (come \href{20251201160758-isomorfismo_tra_algebre_graduate.org}{algebra graduata}\footnote{Vedi anche ``\href{20251115175943-prodotto_wedge_in_coomologia_di_de_rham.org}{Coomologia di De Rham come algebra graduata}''}):
\begin{equation*}
\operatorname{H}^{\bullet}_{\text{dR}}(M) \cong \operatorname{H}^{\bullet}_{\text{dR}}(N).
\end{equation*}
\end{thm}
\end{document}
