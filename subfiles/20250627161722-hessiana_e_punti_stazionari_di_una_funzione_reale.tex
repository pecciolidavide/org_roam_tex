% Intended LaTeX compiler: pdflatex
\documentclass[../main]{subfiles}


\begin{document}

\begin{prop}
Sia \(A \subseteq \R^{n}\) aperto e sia \(f \in C^{2}(A)\)\footnote{Vedi ``\href{20250113125602-classe_c_di_una_funzione.org}{Funzione di classe Ck}''}. Sia \(c\) un \href{20250702101346-punto_critico_di_una_funzione_reale.org}{punto critico} per \(f\). Sia \(H_{f}(c)\) l'\href{20250627161731-matrice_hessiana.org}{Hessiana} di \(f\) calcolata nel punto \(c\).
\begin{itemize}
\item Se \(H_{f}(c)\) è \href{20250702102213-matrice_definita_positiva.org}{definita positiva}, allora \(c\) è un punto di \href{20250627153543-massimo_e_minimo_di_una_funzione_reale.org}{minimo locale forte}.
\item Se \(H_{f}(c)\) è \href{20250702102213-matrice_definita_positiva.org}{definita negativa}, allora \(c\) è un punto di \href{20250627153543-massimo_e_minimo_di_una_funzione_reale.org}{massimo locale forte}.
\item Se \(H_{f}(c)\) è \href{20250702102213-matrice_definita_positiva.org}{indefinita}, allora \(c\) è un punto di \href{20250702102107-punto_di_sella.org}{sella}.
\end{itemize}
\end{prop}

\begin{proof}
Si dimostra che se \(H_{f}(c)\) è definita positiva, allora \(c\) è un punto di minimo locale.

Sia \(\gamma:[-\varepsilon,\varepsilon]\to A\) una curva di classe \(C^{2}\) tale che \(\gamma(0) = c\) e \(\norma{\gamma'(0)} = 1\). Sia \(\bm{v} \coloneqq \gamma'(0)\). Sia \(g(t) \coloneqq f\circ\gamma(t)\).
\begin{itemize}
\item \(g'(0) = 0\). Infatti, applicando la \href{20250702114407-chain_rule.org}{chain rule}:\footnote{Vedi ``\href{20250624171244-gradiente_di_una_funzione.org}{Gradiente di una funzione}''}
\begin{align*}
  g'(0) &= \nabla f (\gamma(0)) \cdot \gamma'(0)\\
  &= \nabla f(c) \cdot \bm{v} = 0\cdot \bm{v}=0.
\end{align*}
\item \(g''(0) >0\). Infatti, si noti che\footnote{Vedi ``\href{20250702114642-derivata_direzionale.org}{Derivata direzionale}''}
\begin{equation*}
  g''(0) = D_{\bm{v}}\left(D_{\bm{v}} f\right)(c)
\end{equation*}
Inoltre, siccome \(f\) è differenziabile, si ha che \(D_{\bm{v}} f(x) = \nabla f (x)\cdot \bm{v}\).

Anche \(D_{\bm{v}}f\) è differenziabile, e pertanto <
\begin{equation*}
  D_{\bm{v}}\left(D_{\bm{v}} f\right)(x) = \nabla (D_{\bm{w}}f) (x)\cdot \bm{v}
\end{equation*}
Ma
\begin{equation*}
  \nabla(\nabla f (x)\cdot \bm{v}) = H_{f}(x) \bm{v}
\end{equation*}
e pertanto \(g''(0) = H_{f}(c)\bm{v}\cdot\bm{v}>0\).\qedhere
\end{itemize}
\end{proof}
\end{document}
