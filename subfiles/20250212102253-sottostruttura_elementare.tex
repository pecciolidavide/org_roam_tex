% Intended LaTeX compiler: pdflatex
\documentclass[../main]{subfiles}

\usepackage[hyperref]{biblatex}
\date{}
\title{}
\begin{document}

\section{Sottostruttura elementare}
\label{sec:org8ff0c94}
Siano \(\mathcal{N} \subseteq \mathcal{M}\) due \(\mathcal{L}\)-\href{20250131103035-struttura_del_prim_ordine.org}{strutture}.

\begin{definizione}
Diciamo che \(\mathcal{N}\) è una \uline{sottostruttura elementare} di \(\mathcal{M}\) (oppure che \(\mathcal{M}\) è una \uline{estensione elementare} di \(\mathcal{N}\)) se per ogni \(\mathcal{L}\)-formula \(\varphi\left(x_1, \ldots, x_n\right)\) ed ogni \(a_1, \ldots, a_n \in N\)
\begin{equation*}
\mathcal{M} \models \varphi\left[a_1, \ldots, a_n\right] \quad \text { se e solo se } \quad \mathcal{N} \models \varphi\left[a_1, \ldots, a_n\right].
\end{equation*}
(vedi \href{20250131123011-conseguenza_logica.org}{Conseguenza logica})
\end{definizione}

Si osservi che se \(\mathcal{N}\) è una sottostruttura elementare di \(\mathcal{M}\), allora in particolare:
\begin{itemize}
\item è una \href{20250131103212-sottostruttura_del_prim_ordine.org}{sottostruttura} \(\mathcal{N} \subseteq \mathcal{M}\);
\item \(\mathcal{M}\) ed \(\mathcal{N}\) sono \href{20250131123208-teorie_elementarmente_equivalente.org}{elementarmente equivalenti}: \(\mathcal{N} \equiv \mathcal{M}\).
\item l'iniezione \(\iota:\mathcal{N}\to \mathcal{M}\) è una \href{20250214120959-mappe_tra_strutture_del_prim_ordine.org}{immersione elementare}
\end{itemize}
\end{document}
