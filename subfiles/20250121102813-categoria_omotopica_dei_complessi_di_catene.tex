% Intended LaTeX compiler: pdflatex
\documentclass[../main]{subfiles}


\begin{document}

Sia \(R\) un \href{20241205141119-anello.org}{anello} commutativo con unità.

Sia \(\sim\) è la \href{20250113110148-relazione_di_equivalenza.org}{relazione di equivalenza} di \href{20250121094935-omotopia_tra_morfismi_di_complessi_di_catene.org}{omotopia} tra \href{20250120163759-categoria_complessi_di_catene.org}{morfismi} di \href{20250120163114-complesso_di_catene.org}{complessi di catene}.
\begin{definizione}
Si definisce la \href{20241126100904-categoria.org}{categoria} \(\cat{h\text{-}Ch}_{R}\) data da:
\begin{enumerate}
\item \(\operatorname{Ob}(\cat{h\text{-}Ch}_{R}) = \operatorname{Ob}(\cat{Ch}_{R})\)\footnote{\(\cat{Ch}_{R}\) è la \href{20250120163759-categoria_complessi_di_catene.org}{categoria dei complessi di catene di \(R\)-moduli}};
\item Per ogni \(\mathcal{C}_{\bullet}, \mathcal{D}_{\bullet} \in \operatorname{Ob}(\cat{h\text{-}Ch}_{R})\) si definisce il \href{20250114100810-quoziente_rispetto_a_relazione_di_equivalenza.org}{quoziente}:
\begin{equation*}
 \operatorname{Hom}_{\cat{h\text{-}Ch}_{R}} (\mathcal{C}_{\bullet},\mathcal{D}_{\bullet}) = \operatorname{Hom}_{\cat{Ch}_{R}}(\mathcal{C}_{\bullet},\mathcal{D}_{\bullet})/\sim.
\end{equation*}
\end{enumerate}
\end{definizione}
\section{Funtore da h-ChR a RMod - di omologia}
\label{sec:org7ccec66}
Inoltre, il \href{20241205131958-funtore.org}{funtore} \href{20250120165029-funtore_tra_chr_e_rmod.org}{di omologia}:
\begin{equation*}
H_{n} : \cat{Ch}_{R}\to \cat{RMod}
\end{equation*}
induce un funtore
\begin{equation*}
H_{n}: \cat{h\text{-}Ch}_{R}\to \cat{RMod}
\end{equation*}
nella \href{20241206115740-categoria_degli_r_moduli.org}{categoria degli \(R\)-moduli}.
\end{document}
