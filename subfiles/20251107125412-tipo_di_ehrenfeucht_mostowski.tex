% Intended LaTeX compiler: pdflatex
\documentclass[../main]{subfiles}


\begin{document}

\def\a{\overline{a}}
\def\x{\overline{x}}
\def\D{\mathcal{D}}
\def\DD{\bm{\D}}
\def\U{\mathcal{U}}
\def\eq{{\rm eq}}
\def\Ueq{\U^\eq}
\def\L{\mathcal{L}}
\def\orbita{\mathcal{O}}
\def\Aut{\operatorname{Aut}}
\def\tc{\mid}
\def\tp{\operatorname{tp}}
\def\<{\langle}
\def\>{\rangle}
\def\b{\bm{b}}
\def\EMtp{\operatorname{EM}\text{-}\operatorname{tp}}
\def\ot{\operatorname{ot}}

\def\restricted#1{\,\mathord{\upharpoonright}{{\scriptstyle #1}}}
\def\equivalentover#1{\mathrel{\equiv_{ #1 }}}

%% NON FORKING
\def\nonforkSymbol{%
\mathbin{\raise1.8ex%
\rlap{\kern0.6ex\rule{0.6ex}{0.1ex}}%
\rlap{\kern1.1ex\rule{0.1ex}{1.9ex}}\raise-0.3ex\hbox{$\smile$}}}
\def\defaultnonforkmodel{M}
\def\nonfork{\nonforkSymbol}
\renewcommand{\nonfork}[1][\defaultnonforkmodel]{%
\mathrel{\nonforkSymbol_{#1}}}
\section{Tipo di Ehrenfeucht-Mostowski}
\label{sec:orgc31a33f}
Si utilizza la \href{20250612143636-notazione_teoria_dei_modelli.org}{Notazione della TEORIA DEI MODELLI}

Sia \(\L\) un \href{20250130162057-linguaggio_del_prim_ordine.org}{linguaggio}, \(T\) una \href{20250130114950-teoria_del_prim_ordine.org}{teoria} \href{20250131123151-teoria_completa.org}{completa} senza \href{20250131122945-modello_di_un_insieme_di_formule.org}{modelli} finiti e \(\U\) un \href{20250617095548-modello_lambda_saturo.org}{modello saturo} di \href{20241213101756-cardinalita.org}{cardinalità} \href{20250211123155-cardinale_limite_forte.org}{inaccessibile} \(\kappa>\card{\L}+ \omega\). \(\U\) è un \href{20250617102733-modello_mostro.org}{modello mostro}.

Sia \(A \subseteq \U\) e sia \((I,<)\) un \href{20250130104331-insieme_mk.org}{insieme} \href{20250203101604-ordine.org}{parzialmente ordinato} senza \href{20250203102516-massimo_e_minimo.org}{elementi massimi}

\begin{definizione}
Sia \(\a \coloneqq \langle a_{i}: i \in I \rangle\), con \(a_{i} \in \U^{z}\).

Sia \(\x = \langle x_{i} : i < \omega \rangle\).
Si definisce il \uline{\href{20250212164424-tipo_teoria_dei_modelli.org}{tipo} di Ehrenfeucht-Mostowski} come
\begin{equation*}
\EMtp(\overline{a}/A) \coloneqq
\set{%
\varphi(\overline{x}) \in \L(A) \mid %
\text{per ogni }J \in I^{(\omega)}:\quad\vDash\varphi(a \restricted{J}) %
}.
\end{equation*}
dove \(I^{(\omega)} = \set{X \in \parti{I} : \ot({X}) = \omega}\).
\end{definizione}
\begin{oss}
Siccome \(\ot({J}) = \omega\), allora
\(a\restricted{J} = \< a_{j_{i}} \mid i <\omega \>\),
con \(\< j_{i} \mid i < \omega \>\) successione ordinata in \(J\), e pertanto ha senso la scrittura
\(\varphi(a \restricted{J})\).
\end{oss}
\begin{oss}
Se \(\overline{a}\) è una \href{20251026210908-sequenza_di_indiscernibili.org}{sequenza di indiscernibili} su \(A\), allora \(\EMtp(\overline{a}/A)\) è \href{20251029152405-tipo_completo.org}{completo}.
\end{oss}
\end{document}
