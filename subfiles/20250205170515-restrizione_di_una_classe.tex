% Intended LaTeX compiler: pdflatex
\documentclass[../main]{subfiles}


\begin{document}

\section{Restrizione di una funzione}
\label{sec:org4cb5920}
\subsection{Generalizzazione in MK}
\label{sec:orga0bdcdc}

Contesto: \href{20250130104245-morse_kelly_set_theory.org}{Morse Kelly Set Theory}
\subsubsection{Restrizione di una classe MK}
\label{sec:org3c4bd36}
Siano \(F,A\) due \href{20250130104320-classe_mk.org}{classi} qualsiasi. La \uline{restrizione di \(F\) ad \(A\)}: \(F\upharpoonright A\) è
\begin{equation*}
F\upharpoonright A \coloneqq \set{(x,y) \in F \mid x \in A}.
\end{equation*}
(vedi \href{20250131162451-coppia_ordinata_mk.org}{Coppia ordinata})

Se \(R\) è una relazione su una classe \(X\), e \(Y \subseteq X\), si indica (con un abuso di notazione)
\begin{equation*}
R\restricted Y \coloneqq R\cap (Y\times Y)
\end{equation*}
\end{document}
