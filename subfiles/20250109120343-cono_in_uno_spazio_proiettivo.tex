% Intended LaTeX compiler: pdflatex
\documentclass[../main]{subfiles}


\begin{document}

Sia \(\K\) un \href{20241231112713-campo_algebricamente_chiuso.org}{campo algebricamente chiuso}, e si consideri \(\mathds{P}^{n}\) \href{20241231115051-spazio_proiettivo.org}{spazio proiettivo}.

Siano \(p \in \mathds{P}^{n}\) e \(X \subseteq \mathds{P}^{n}\setminus\set{p}\) una \href{20241231123223-varieta_algebrica_proiettiva.org}{varietà algebrica}.

Si definisce il cono su \(X\) di vertice \(p\):
\begin{equation*}
\operatorname{C}_{p}X \coloneqq\bigcup_{q \in X} \ell_{pq}
\end{equation*}
dove \(\ell_{pq}\) è la \href{20250109101841-retta_proiettiva.org}{retta} passante per \(p\) e \(q\).
\section{{\bfseries\sffamily TODO} Proposizione\hfill{}\textsc{matematica\_lm:geo\_alg}}
\label{sec:orgbcf62b5}
Questo cono è una varietà
\end{document}
