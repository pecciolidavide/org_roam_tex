% Intended LaTeX compiler: pdflatex
\documentclass[../main]{subfiles}


\begin{document}

\section{Push-Forward di campi vettoriali}
\label{sec:orge2747ff}
Siano \(M, N\) \href{20250113115909-struttura_differenziabile.org}{varietà differenziabili} e sia \(F:M\longrightarrow N\) una \href{20250113144722-funzioni_cinfinito_tra_varieta_differenziabili.org}{funzione \(C^{\infty}\)}, con \href{20250114111331-differenziale_di_una_funzione_tra_varieta_differenziabili.org}{differenziale}
\begin{equation*}
\restriction{F_{\star}}{p}: \operatorname{T}_{p}M \longrightarrow \operatorname{T}_{F(p)}N
\end{equation*}

Se \(F\) è un \href{20250113172924-diffeomorfismo_tra_varieta_differenziabili.org}{diffeomorfismo}, allora \(F\) induce \(F_{\star}\) funzione tra \href{20250115110259-campo_vettoriale_su_una_varieta_differenziabile.org}{campi vettoriali}
\begin{equation*}
F_{\star}:\Gamma(M)\longrightarrow \Gamma(N)
\end{equation*}
tale per cui, se \(p =F^{-1}(q)\)
\begin{equation*}
F_{\star}(X)_{q} \coloneqq \restriction{F_{\star}}{p}(X_{p})
\end{equation*}
\subsection{Proposizione}
\label{sec:org15e61b4}
\(F_{\star}\) è un \href{20250113125833-isomorfismo_tra_spazi_vettoriali.org}{isomorfismo di spazi vettoriali} (infatti \href{20250115122306-insieme_dei_campi_vettoriali_come_spazio_vettoriale_reale_e_come_modulo.org}{l'insieme dei campi vettoriali è uno spazio vettoriale reale}) e
\begin{equation*}
(F_{\star})^{-1} = (F^{-1})_{\star}
\end{equation*}
\end{document}
