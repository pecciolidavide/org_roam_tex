% Intended LaTeX compiler: pdflatex
\documentclass[../main]{subfiles}


\begin{document}

\def\D{\mathcal{D}}
\def\DD{\bm{\D}}
\def\U{\mathcal{U}}
\def\eq{{\rm eq}}
\def\Ueq{\U^\eq}
\def\L{\mathcal{L}}
\def\orbita{\mathcal{O}}
\def\Aut{\operatorname{Aut}}
\def\tc{\mid}
\def\tp{\operatorname{tp}}
\def\<{\langle}
\def\>{\rangle}
\def\b{\bm{b}}

\def\restricted#1{\,\mathord{\upharpoonright}{{\scriptstyle #1}}}
\def\equivalentover#1{\mathrel{\equiv_{ #1 }}}

%% NON FORKING
\def\nonforkSymbol{%
\mathbin{\raise1.8ex%
\rlap{\kern0.6ex\rule{0.6ex}{0.1ex}}%
\rlap{\kern1.1ex\rule{0.1ex}{1.9ex}}\raise-0.3ex\hbox{$\smile$}}}
\def\defaultnonforkmodel{M}
\def\nonfork{\nonforkSymbol}
\renewcommand{\nonfork}[1][\defaultnonforkmodel]{%
\mathrel{\nonforkSymbol_{#1}}}
\section{Sequenza di coeredi}
\label{sec:org98ae532}
Si utilizza la \href{20250612143636-notazione_teoria_dei_modelli.org}{Notazione della TEORIA DEI MODELLI}

Sia \(\L\) un \href{20250130162057-linguaggio_del_prim_ordine.org}{linguaggio}, \(T\) una \href{20250130114950-teoria_del_prim_ordine.org}{teoria} \href{20250131123151-teoria_completa.org}{completa} senza \href{20250131122945-modello_di_un_insieme_di_formule.org}{modelli} finiti e \(\U\) un \href{20250617095548-modello_lambda_saturo.org}{modello saturo} di \href{20241213101756-cardinalita.org}{cardinalità} \href{20250211123155-cardinale_limite_forte.org}{inaccessibile} \(\kappa>\card{\L}+ \omega\). \(\U\) è un \href{20250617102733-modello_mostro.org}{modello mostro}.

Sia \(p(x) \subseteq \L(\U)\) un \href{20250212164424-tipo_teoria_dei_modelli.org}{tipo}, e sia \(M \preceq \U\) un \href{20250131103035-struttura_del_prim_ordine.org}{modello}.

\begin{definizione}
Se \(p(x)\) è \href{20250212164424-tipo_teoria_dei_modelli.org}{finitamente soddisfacibile su \(M\)}\footnote{Ovvero per ogni \(\varphi(x)\) congiunzione finita di formule di \(p(x)\):
\begin{equation*}
\varphi(\U^{x})\cap M^{x}\neq \emptyset.
\end{equation*}}, una \href{20251026210840-sequenza_di_morley.org}{sequenza di Morley di \(p(x)\) su \(M\)} si dice \uline{sequenza di coeredi}.
\end{definizione}
\section{Caratterizzazione sequenza di coeredi}
\label{sec:org21bccaa}
%% Alcuni simboli
\def\nonforkSymbol{\mathbin{\raise1.8ex\rlap{\kern0.6ex\rule{0.6ex}{0.1ex}}\rlap{\kern1.1ex\rule{0.1ex}{1.9ex}}\raise-0.3ex\hbox{$\smile$} } }
\def\nonfork{\relax}
\renewcommand{\nonfork}[1][M]{\mathrel{\nonforkSymbol_{#1}}}
%% Leggibilità
%
\def\L{\mathcal{L}} % Il linguaggio
\def\U{\mathcal{U}} % Modello mostro
\def\orbita{\mathcal{O}} % Le orbite
\def\restricted{\upharpoonright} % Restrizione
\def\equivalentover#1{\mathrel{\equiv_{#1}}}

Si utilizza la \href{20250612143636-notazione_teoria_dei_modelli.org}{Notazione della TEORIA DEI MODELLI}

Sia \(\L\) un \href{20250130162057-linguaggio_del_prim_ordine.org}{linguaggio}, \(T\) una \href{20250130114950-teoria_del_prim_ordine.org}{teoria} \href{20250131123151-teoria_completa.org}{completa} senza \href{20250131122945-modello_di_un_insieme_di_formule.org}{modelli} finiti e \(\U\) un \href{20250617095548-modello_lambda_saturo.org}{modello saturo} di \href{20241213101756-cardinalita.org}{cardinalità} \href{20250211123155-cardinale_limite_forte.org}{inaccessibile} \(\kappa>\card{\L}+ \omega\). \(\U\) è un \href{20250617102733-modello_mostro.org}{modello mostro}.

\begin{lem}
LSASE:
\begin{enumerate}
\item \(\overline{c} = \langle c_{i} : i<\omega \rangle\) è \hyperref[sec:org98ae532]{sequenza di coeredi su \(M\)} (rispetto a qualche \(p(x) \in S(\U)\)\footnote{Con \(S(\U)\) si intende l'insieme dei tipi globali (notazione propria del \href{20250617102733-modello_mostro.org}{Modello MOSTRO})});
\item \(c_{i} \nonfork (c \restricted i)\)\footnote{La relazione \(\nonfork\) è la relazione di indipendenza; vedi ``\href{20251029154447-tuple_indipendenti_rispetto_ad_un_modello_monster_model.org}{Tuple indipendenti rispetto ad un modello (Monster Model)}''} e \(c_{i+1} \equivalentover{M, c \restricted i} c_{i}\)\footnote{La relazione \(\equivalentover{A}\) è quella di \href{20250212164424-tipo_teoria_dei_modelli.org}{equivalenza sull'insieme di parametri \(A\)}.};
\item \(c_{i} \nonfork (c \restricted i)\) e \(c_{j} \equivalentover{M, c \restricted i} c_{i}\) per ogni \(j>i\).
\end{enumerate}
\end{lem}
\end{document}
