% Intended LaTeX compiler: pdflatex
\documentclass[../main]{subfiles}


\begin{document}

\section{Divisore di una superficie di Riemann}
\label{sec:orgffa2f9a}
Sia \(X\) una \href{20260127112828-superficie_di_riemann.org}{superficie di Riemann}, e sia \(\Z^{X}\) il \href{20260201235333-gruppo_delle_funzioni_in_z.org}{gruppo delle funzioni a valori in \(\Z\)}.

\begin{definizione}
Un \textbf{divisore di \(X\)} è una \href{20250202170607-classe_relazione_binaria.org}{funzione} \(D \in \Z^{X}\) tale che il \href{20260201235333-gruppo_delle_funzioni_in_z.org}{supporto} \(\operatorname{supp} D\) sia un sottoinsieme \href{20250103145124-topologia.org}{chiuso} e \href{20260128123515-sottoinsieme_discreto.org}{discreto} di \(X\).
\end{definizione}

I divisori formano un \href{20241206143051-sottogruppo.org}{sottogruppo} di \(\Z^{X}\), indicato con \(\operatorname{Div}(X)\).

\begin{oss}
Se \(X\) è \href{20250103163701-spazio_topologico_compatto.org}{compatto}, allora i divisori sono tutte e sole le funzioni con supporto finito\footnote{Infatti, \href{20260128172415-sottoinsieme_discreto_in_un_compatto.org}{i sottoinsiemi discreti in un compatto sono finiti}}. Questo è lo \href{20241205141053-r_moduli.org}{\(\Z\)-modulo} \href{20241213094625-modulo_libero.org}{libero} \href{20241213095808-somma_diretta.org}{generato dai punti di \(X\)}: \(\operatorname{Div}(X) \cong \Z^{(X)}\).

In questo caso scriviamo \(D\) come una somma formale finita di punti:
\begin{equation*}
D = \sum_{p \in X} D(p) \cdot p,\qquad D(p) \in \Z.
\end{equation*}
\end{oss}

\uline{Notazione}: In generale, scriviamo tutti i divisori come somma formale (non finita).
\subsection{Grado di un divisore di una superficie di Riemann}
\label{sec:org08a2ada}
\begin{definizione}
Sia \(X\) una \href{20260127112828-superficie_di_riemann.org}{superficie di Riemann} \href{20250103163701-spazio_topologico_compatto.org}{compatta}, \(D \in \operatorname{Div}(X)\). Il \textbf{grado di \(D\)} è
\begin{equation*}
\deg D \coloneqq \sum_{p \in X} D(p) \in \Z.
\end{equation*}
\end{definizione}

\begin{oss}
La funzione \(\deg: \operatorname{Div}(X)\to \Z\) è un \href{20241206115531-morfismo_di_gruppi.org}{omomorfismo di gruppi}.
\end{oss}
\end{document}
