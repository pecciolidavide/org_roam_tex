% Intended LaTeX compiler: pdflatex
\documentclass[../main]{subfiles}

\usepackage[hyperref]{biblatex}
\date{}
\title{}
\begin{document}

\section{INBOX - Fleeting Notes}
\label{sec:orgbd00452}
\subsection{Note di lettura: ``Il Pendolo di Foucault''.}
\label{sec:org49ecaf6}
``perché quando lavori ad un testo in maniera ossessiva [\ldots{}] non ti puoi sottrarre all'universo di discorso in cui vivi''

Quanto è vero, mi capita ogni volta che studio con interesse una materia.
Ho paura che facendo un PhD mi specializzi troppo e poi non sia più capace ad uscire da quell'universo.
Chissà se manterrò la mia curiosità??
E la voglia di studiare COSE DIVERSE

Ora posso studiare corsi diversi, spaziare i miei interessi.
Ma poi?!
\subsection{Note di lettura: Il Grande Gatsby}
\label{sec:org916475a}
Questo romanzo è ambientato nel periodo a cavallo tra le due guerre mondiali a New York.
La ragione principale per cui ho letto questo libro è l'atmosfera e l'ambientazione dei primi del novecento, che mi proiettano in un tempo molto diverso dal nostro, in cui alcuni valori si rispettano più rispetto che al presente.

Ciò nonostante, questo romanzo in particolare mette in luce gli aspetti più corrotti della New York di quei tempi (nonostante non lo faccia in modo esplicito, bensì alludendo ad ``altro'').
L'aspetto forse più importante, invece, è quello psicologico: alcuni temi trattati sono incredibilmente comuni e ci toccano da vicino.

C'è il protagonista, Gatsby. Ha vissuto la sua vita con un unico obiettivo, un unico sogno, una utopia: poter amare Daisy.
Ma strada facendo, si rende conto di tenere più al sogno che all'amore di Daisy.
L'idea del suo amore, formatasi e consolidata si tramite i ricordi, completamente basata su un ``ritorno al passato'', che permetteva a Gatsby di andare avanti.

Quando egli si è reso conto, realizzando il suo ``sogno'', che quella idealizzazione era utopica, e che non sarebbe potuto tornare al passato, la sua intera vita è crollata.
E il climax di questa vicenda è stata la sua morte, indirettamente causata proprio da Daisy.
Daisy quindi ha ucciso sia moralmente che ``fisicamente'' Gatsby.

C'è poi il funerale di Gatsby, che rappresenta l'emblema dell'idea che la fama non porta amicizie e felicità:
Gatsby era popolarissimo, alle sue feste partecipavano a centinaia, ma al suo funerale non è andato nessuno —
\subsection{How to study well}
\label{sec:orgcab00ac}

\begin{enumerate}
\item Prepare EVERYTHING the night before, so that when you sit on your desk all you have to do is to take your pencil and STUDY !
\end{enumerate}

\noindent\rule{\textwidth}{0.5pt}
\begin{enumerate}
\item Put away phone

\item Don't allow yourself sit up untill you're done !
\end{enumerate}
\subsection{Metodo di Studio e Organizzazione}
\label{sec:org63cea94}
\subsubsection{Come studiare}
\label{sec:org1c61c38}

\begin{enumerate}
\item Dividere lo studio in 3 parti
\begin{itemize}
\item lettura / comprensione
\item rielaborazione
\item ripasso.
\end{itemize}

\item Interlacciare materie diverse
\begin{itemize}
\item allontanare le sedute di ripasso.
\end{itemize}
\end{enumerate}
\subsubsection{In particolare}
\label{sec:org5636417}

\begin{itemize}
\item sessione di studio iniziali
\begin{itemize}
\item 70\% lettura e comprensione
\item 30\% passaggio su org-roam.
\end{itemize}

\item sessione di studio di ripasso.
\begin{itemize}
\item utilizzo di tecniche tipo Flashcard tramite org-roam.
\end{itemize}
\end{itemize}
\subsubsection{Dubbi e Riflessioni}
\label{sec:orgeeabd26}

Ma siamo sicuri che sia il metodo migliore per ripassare ??

\begin{quote}
Potrei stampare le Flashcard
\end{quote}
\subsection{Salvataggio tools}
\label{sec:org19f97fb}
Dovrei salvare in org-mode gli strumenti per fare le cose: ad esempio E-DICE. Per quanto riguarda i programmi, potrei avere un link ad una dir sul server
\subsection{META}
\label{sec:org83c896c}

\href{20250515141706-da_finire.org}{INPUT}
\end{document}
