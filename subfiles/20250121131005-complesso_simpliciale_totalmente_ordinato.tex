% Intended LaTeX compiler: pdflatex
\documentclass[../main]{subfiles}


\begin{document}

Sia \(K\) un \href{20250121124147-complesso_simpliciale.org}{complesso simpliciale}, e sia \(K^{0}\) il suo \(0\)-\href{20250121124310-scheletro_di_un_complesso_simpliciale.org}{scheletro}. È ovvio verificare che per ogni \href{20250121121923-simplessi.org}{simplesso} \(\sigma \in K\):
\begin{equation*}
\sigma = [p_{0},\dots, p_{q}]
\end{equation*}
per qualche \(p_{0},\dots,p_{q} \in K^{0}\).

\begin{definizione}
\(K\) si dice \textbf{totalmente ordinato} se è stato fissato un ordine totale \(<\) su \(K^{0}\).
\end{definizione}

Quando \(K\) è totalmente ordinato ogni \(\sigma \in K\) è denotato da
\begin{equation*}
     \sigma = [p_{0},\dots,p_{q}]
\end{equation*}
con \(p_{0}<\dots<p_{q}\). \href{20250121124410-modulo_delle_catene_simpliciali.org}{Non è corretto scrivere \([p_{1},p_{0},\dots,p_{q}]\).}
\end{document}
