% Intended LaTeX compiler: pdflatex
\documentclass[../main]{subfiles}


\begin{document}

\section{Order Type del prodotto cartesiano di un cardinale è il cardinale stesso}
\label{sec:orgf578f88}
Contesto: \href{20250130104245-morse_kelly_set_theory.org}{Morse Kelly Set Theory}
\subsection{Teorema}
\label{sec:org69fc82e}
Sia \(\kappa\) una \href{20250203161341-cardinali.org}{cardinale} \href{20250205120448-classe_finita_e_infinita_mk.org}{infinito}, e sia \(\kappa\times\kappa\) il \href{20250131183735-prodotto_cartesiano_di_classi_mk.org}{prodotto cartesiano}.
\begin{enumerate}
\item Allora l'\href{20250203133527-insiemi_ben_ordinati_sono_isomorfi_ad_un_ordinale_unico.org}{order type} del prodotto cartesiano con il \href{20250203104134-buon_ordine_mk.org}{buon ordine} \href{20250205180911-buon_ordine_di_godel_per_ordxord.org}{di Godel} è
\begin{equation*}
 \operatorname{ot}\left(\kappa\times\kappa,<_{G}\right) = \kappa
\end{equation*}
\item \(|\kappa\times\kappa|=\kappa\) (vedi \href{20241213101756-cardinalita.org}{Cardinalità}).
\end{enumerate}
\subsubsection{Dimostrazione}
\label{sec:org5bdfbea}
La \href{20250202170607-classe_relazione_binaria.org}{funzione}
\begin{align*}
 \langle \kappa,<\rangle &\longrightarrow \langle\kappa\times\kappa,<_{G}\rangle \cong \operatorname{ot}(\kappa\times\kappa,<_{G})\\
\alpha &\longmapsto (\alpha,0)
\end{align*}
è una funzione tra ordinali crescente, e \href{20250203111003-ordinali.org}{dunque} \(\kappa\le \operatorname{ot}(\kappa\times\kappa,<_{G})\).

Si dimostra per \href{20250208172824-induzione_transfinita_per_le_relazioni_well_founded.org}{induzione} su \(\kappa\ge\omega\) che \(\operatorname{ot}(\kappa\times\kappa,<_{G})\le \kappa.\)
\begin{itemize}
\item Necessariamente \(\operatorname{ot}(\omega\times\omega,<_{G})\le\omega\).
\item Si assuma che per ogni ordinale \(\omega\le\alpha<\kappa\): \(\operatorname{ot}(\alpha\times\alpha,<_{G})\le \alpha\).

Sia per assurdo \(\delta\coloneqq \operatorname{ot}(\kappa\times\kappa,<_{G})\), \(\kappa<\delta\), e sia
\begin{equation*}
  f:\langle \kappa\times\kappa, <_{G}\rangle\to \delta=\operatorname{ot}(\kappa\times\kappa,<_{G})
\end{equation*}
l'isomorfismo. Allora esiste un unica coppia \((\lambda_{1},\lambda_{2}) \in \kappa\times\kappa\) tale che \(f(\lambda_{1},\lambda_{2}) =\kappa\).

\(\lambda_{1},\lambda_{2}<\kappa\), dunque \(\max(\lambda_{1},\lambda_{2})\eqqcolon \lambda<\kappa\).

Allora \(\kappa \subseteq f[\lambda\times\lambda]\), ma per ipotesi induttiva \(\card{\lambda\times\lambda}\le\lambda<\kappa\), dunque, siccome \(f\) è un isomorfismo
\begin{equation*}
  \kappa\embeds f[\lambda\times\lambda] \equipotenti \card{f[\lambda\times\lambda]} <\kappa
\end{equation*}
Assurdo.\qed
\end{itemize}

\url{https://math.stackexchange.com/a/2153155/1320017}
\subsection{Equipotenza degli insiemi di funzioni tra cardinali}
\label{sec:org6a4bf78}
Se \(\kappa,\lambda\) sono cardinali tali che
\begin{equation*}
2\le\kappa\le\lambda
\end{equation*}
e \(\lambda\) è infinito, allora
\begin{equation*}
\prescript{\lambda}{}{2}\asymp \prescript{\lambda}{}{\kappa}\asymp \prescript{\lambda}{}{\lambda}
\end{equation*}
(vedi \href{20250202192030-classe_delle_classi_funzioni.org}{Classe delle Classi-Funzioni} e \href{20250619101109-classi_equipotenti.org}{Classi equipotenti MK})
\subsubsection{Dimostrazione}
\label{sec:org0a144d8}
\href{20250203111003-ordinali.org}{Dal momento} che
\begin{equation*}
2 \subseteq \kappa \subseteq \lambda
\end{equation*}
ovviamente
\begin{equation*}
\prescript{\lambda}{}{2} \subseteq \prescript{\lambda}{}{\kappa} \subseteq \prescript{\lambda}{}{\lambda}
\end{equation*}
Inoltre \(\prescript{\lambda}{}{\lambda} \subseteq \parti{\lambda\times\lambda}\).

Inoltre, per il teorema \(|\lambda\times\lambda| = \lambda\), e \href{20241213101756-cardinalita.org}{pertanto}
\begin{equation*}
\lambda\times\lambda\asymp\lambda
\end{equation*}
(vedi \href{20250619101109-classi_equipotenti.org}{Classi equipotenti MK})

Questo induce una \href{20250104111707-funzione_biunivoca.org}{biiezione}: \(\parti{\lambda\times\lambda}\asymp\parti{\lambda}\).

Ma \(\parti{\lambda}\asymp \prescript{\lambda}{}{2}\), in quanto a ciascun \(\alpha \subseteq\lambda\) è associata la funzione \(f:\lambda\to 2\):
\begin{align*}
f(x) = \begin{cases}
\emptyset & x \notin\alpha\\
\set{\emptyset} & x \in \alpha
\end{cases}
\end{align*}
ricordando che \(2=\set{\set{\emptyset},\emptyset}\)

Si ha quindi la seguente, che dimostra la tesi:
\begin{equation*}
\prescript{\lambda}{}{2} \subseteq \prescript{\lambda}{}{\kappa} \subseteq \prescript{\lambda}{}{\lambda} \subseteq \parti{\lambda\times\lambda} \asymp \parti{\lambda} \asymp \prescript{\lambda}{}{2}
\end{equation*}
\end{document}
