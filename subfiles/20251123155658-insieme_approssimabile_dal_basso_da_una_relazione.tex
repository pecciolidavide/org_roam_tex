% Intended LaTeX compiler: pdflatex
\documentclass[../main]{subfiles}


\begin{document}

\section{Insieme approssimabile dal basso da una relazione}
\label{sec:org6aa2330}
\def\U{\mathcal{U}}
\def\L{\mathcal{L}}
%
\def\eq{{\rm eq}}
\def\Ueq{\U^\eq}
%
\def\orbita{\mathcal{O}}
\def\Aut{\operatorname{Aut}}
%
\def\tc{\mid}
\def\tp{\operatorname{tp}}
\def\EMtp{\operatorname{EM}\text{-}\operatorname{tp}}
\def\<{\langle}
\def\>{\rangle}
%
\def\restricted#1{\,\mathord{\upharpoonright}{{\scriptstyle #1}}}
\def\equivalentover#1{\mathrel{\equiv_{ #1 }}}

%% NON FORKING
\def\nonforkSymbol{%
\mathbin{\raise1.8ex%
\rlap{\kern0.6ex\rule{0.6ex}{0.1ex}}%
\rlap{\kern1.1ex\rule{0.1ex}{1.9ex}}\raise-0.3ex\hbox{$\smile$}}}
\def\defaultnonforkmodel{M}
\def\nonfork{\nonforkSymbol}
\renewcommand{\nonfork}[1][\defaultnonforkmodel]{%
\mathrel{\nonforkSymbol_{#1}}}

\def\D{\mathcal{D}}

Si utilizza la \href{20250612143636-notazione_teoria_dei_modelli.org}{Notazione della TEORIA DEI MODELLI}

Sia \(\L\) un \href{20250130162057-linguaggio_del_prim_ordine.org}{linguaggio}, \(T\) una \href{20250130114950-teoria_del_prim_ordine.org}{teoria} \href{20250131123151-teoria_completa.org}{completa} senza \href{20250131122945-modello_di_un_insieme_di_formule.org}{modelli} finiti e \(\U\) un \href{20250617095548-modello_lambda_saturo.org}{modello saturo} di \href{20241213101756-cardinalita.org}{cardinalità} \href{20250211123155-cardinale_limite_forte.org}{inaccessibile} \(\kappa>\card{\L}+ \omega\). \(\U\) è un \href{20250617102733-modello_mostro.org}{modello mostro}.

Una qualsiasi \href{20250131103317-formula_del_prim_ordine.org}{formula} \(\psi(x;z) \in \L(\U)\) si indende come relazione, vedendola come
\begin{equation*}
\psi(\U^{x};\U^{z}) \subseteq \U^{x} \times \U^{z}.
\end{equation*}

\textbf{Forse questo può essere generalizzato al di fuori del Modello Mostro}

\begin{definizione}
Un insieme \(\D\) è \uline{approssimabile da \(\varphi(x;z)\) dal basso} se vale una delle seguenti affermazioni equivalenti equivalenti:
\begin{itemize}
\item per ogni \(B \subseteq \U^{z}\) finito esiste \(a \in \U^{x}\) tale che\footnote{Vedi ``\href{20250131122913-soddisfazione_di_una_formula.org}{Insieme definito da una formula}''.}
\begin{equation*}
\D \cap B = \varphi(a;B) \coloneqq \varphi(a;\U^{z})\cap B
\end{equation*}
e \(\varphi(a;\U^{z}) \subseteq \D\);
\item per ogni \(B \subseteq \D\) finito esiste \(a \in \U^{x}\) tale che
\begin{equation*}
B \subseteq \varphi(a;\U^{z}) \subseteq \D.
\end{equation*}
\end{itemize}
\end{definizione}
\end{document}
