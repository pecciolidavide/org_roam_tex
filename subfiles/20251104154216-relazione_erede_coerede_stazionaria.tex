% Intended LaTeX compiler: pdflatex
\documentclass[../main]{subfiles}


\begin{document}

%% Alcuni simboli
\def\nonforkSymbol{\mathbin{\raise1.8ex\rlap{\kern0.6ex\rule{0.6ex}{0.1ex}}\rlap{\kern1.1ex\rule{0.1ex}{1.9ex}}\raise-0.3ex\hbox{$\smile$} } }
\def\defaultnonforkmodel{M}
\newcommand{\nonfork}[1][\defaultnonforkmodel]{\mathrel{\nonforkSymbol_{#1}}}
%% Leggibilità
%
\def\L{\mathcal{L}} % Il linguaggio
\def\U{\mathcal{U}} % Modello mostro
\def\orbita{\mathcal{O}} % Le orbite
\def\restricted{\upharpoonright} % Restrizione
\def\equivalentover #1 {\mathrel{\equiv_{#1}}}
\section{Relazione erede-coerede stazionaria}
\label{sec:orgea4a719}
Si utilizza la \href{20250612143636-notazione_teoria_dei_modelli.org}{Notazione della TEORIA DEI MODELLI}

Sia \(\L\) un \href{20250130162057-linguaggio_del_prim_ordine.org}{linguaggio}, \(T\) una \href{20250130114950-teoria_del_prim_ordine.org}{teoria} \href{20250131123151-teoria_completa.org}{completa} senza \href{20250131122945-modello_di_un_insieme_di_formule.org}{modelli} finiti e \(\U\) un \href{20250617095548-modello_lambda_saturo.org}{modello saturo} di \href{20241213101756-cardinalita.org}{cardinalità} \href{20250211123155-cardinale_limite_forte.org}{inaccessibile} \(\kappa>\card{\L}+ \omega\). \(\U\) è un \href{20250617102733-modello_mostro.org}{modello mostro}.

Sia \(M \preceq \U\) un \href{20250131103035-struttura_del_prim_ordine.org}{modello}.

\begin{definizione}
La \href{20251029154447-tuple_indipendenti_rispetto_ad_un_modello_monster_model.org}{relazione eredi-coeredi} \(\nonfork\) si dice \uline{stazionaria} se per ogni \(a,b \in \U^{x}\) tuple finite:
\begin{equation*}
a \equivalentover{M} x \nonfork b
\end{equation*}
è un \href{20250212164424-tipo_teoria_dei_modelli.org}{tipo} \href{20251029152405-tipo_completo.org}{completo} su \(M,b\).
\end{definizione}

\begin{definizione}
La \href{20251029154447-tuple_indipendenti_rispetto_ad_un_modello_monster_model.org}{relazione eredi-coeredi} \(\nonfork\) si dice \uline{\(n\)-stazionaria} se per \(\card{x}=n\) si ha:
\begin{equation*}
a \equivalentover{M} x \nonfork b
\end{equation*}
è un \href{20250212164424-tipo_teoria_dei_modelli.org}{tipo} \href{20251029152405-tipo_completo.org}{completo} su \(M,b\).
\end{definizione}

\begin{oss}
Se \(\nonfork\) è \(1\)-stazionaria e \(\card{x} = 1\) allora per ogni
\(a, a', b\):
\begin{equation*}
\big(a \nonfork b \quad \land \quad a' \equivalentover{M} a \big) %
\IMPLICA %
\big(a' \nonfork b \quad\land\quad a' \equivalentover{M,b} a\big).
\end{equation*}
\end{oss}
\begin{proof}
\href{20250515141706-da_finire.org}{DA FINIRE}
\end{proof}

\href{20250515141706-da_finire.org}{DA FINIRE}

\begin{prop}
Se per ogni \(\varphi(x) \in \L(\U)\) esiste \(\psi(x) \in \L(M)\) tale che
\begin{equation*}
\varphi(M^{x}) = \psi(M^{x}),
\end{equation*}
\uline{allora} \(\nonfork[M]\) è \hyperref[sec:orgea4a719]{stazionaria}, ovvero:
\begin{equation*}
a \equivalentover{M} x \nonfork[M] b.
\end{equation*}
è un \href{20251029152405-tipo_completo.org}{tipo completo} su \(M,b\).
\end{prop}
\begin{proof}
Sia \(\varphi(x) = \varphi(x;b)\) per \(\varphi(x;z) \in \L\) e \(b \in \U^{z}\).

Siano \(a_{1},a_{2} \in \U^{x}\) tali che
\(a_{1} \equivalentover{M} a_{2}\) e \(a_{1} \nonfork b\), \(a_{2} \nonfork b\).
Voglio mostrare che \(\varphi(a_{1},b) \iff \varphi (a_{2},b)\).

Sappiamo che \(\varphi(a_{i},b) \iff \psi(a_{i})\) perché \(\varphi(M^{x}) = \psi(M^{x})\).

Siccome \(a_{1}\equivalentover{M} a_{2}\) allora \(\psi(a_{1}) \iff \psi(a_{2})\).
\end{proof}

AGGIUNGERE Remark~14.12 di 
\end{document}
