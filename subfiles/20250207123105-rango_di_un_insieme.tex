% Intended LaTeX compiler: pdflatex
\documentclass[../main]{subfiles}


\begin{document}

\section{Rango di un insieme}
\label{sec:orgc29afb5}
Contesto: \href{20250130104245-morse_kelly_set_theory.org}{Morse Kelly Set Theory}
\begin{definizione}
Sia \(V\) la \href{20250203104513-classe_totale.org}{classe totale}, \href{20250203101604-ordine.org}{ordinata} con l'inclusione \(\in\). Si definisce il \uline{rango di \(x \in V\)} come il \href{20250207122234-rango_di_una_relazione_well_founded.org}{rango} di \(\in\) rispetto a \(V\): \(\bm{\varrho}_{\in,V}(x)\):\footnote{Si veda:
\begin{itemize}
\item \href{20250131155822-operazioni_insiemistiche_tra_classi_mk.org}{Unione generalizzata}
\item \href{20250203111003-ordinali.org}{Ordinali}
\end{itemize}}
\begin{equation*}
\operatorname{rank}(x)\coloneqq \bm{\varrho}_{\in,V}(x) = \bigcup\set{\operatorname{S}(\rank(y))\mid y \in x} \in \operatorname{Ord}
\end{equation*}
dove \(\operatorname{S}\) è il \href{20250202124648-successore_di_un_insieme_mk.org}{successore}.
\end{definizione}
\begin{prop}
Dalla definizione segue banalmente che
\begin{equation*}
x \in y \implies \operatorname{rank}(x)<\operatorname{rank}(y),\qquad x \subseteq y\implies \operatorname{rank}(x) \le \operatorname{rank}(y).
\end{equation*}
e per induzione si verifica che per ogni ordinale \(\alpha\): \(\operatorname{rank}(\alpha)=\alpha\). Inoltre:
\begin{enumerate}
\item \(\operatorname{rank}(\parti{x}) = \operatorname{S}(\operatorname{rank}(x))\)\footnote{\(\parti{x}\) indica l'\href{20250130104245-morse_kelly_set_theory.org}{insieme delle parti}, e \(\operatorname{S}\) il \href{20250202124648-successore_di_un_insieme_mk.org}{successore}.};
\item \(\operatorname{rank}\big(\bigcup x\big) = \sup\set{\operatorname{rank}(y) \mid y \in x}\)\footnote{Vedi ``\href{20250203102516-massimo_e_minimo.org}{Supremum}'' e ``\href{20250131155822-operazioni_insiemistiche_tra_classi_mk.org}{Unione generalizzata}''}.
\end{enumerate}
\end{prop}
\end{document}
