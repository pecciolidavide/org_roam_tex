% Intended LaTeX compiler: pdflatex
\documentclass[../main]{subfiles}


\begin{document}

\section{Morfismo tra complessi di catene induce morfismo tra moduli di omologia}
\label{sec:org424c75f}
Sia \(R\) un \href{20241205141119-anello.org}{anello} commutativo con unità. Siano \(\mathcal{C}_{\bullet}=\set{(C_{n},\partial_{n}^{C})}_{n \in \Z}, \mathcal{D}_{\bullet} \set{(D_{n},\partial_{n}^{D})}_{n \in \Z}\) due \href{20250120163114-complesso_di_catene.org}{complessi di catene}, e sia
\begin{equation*}
f_{\bullet}:\mathcal{C}_{\bullet}\longrightarrow \mathcal{D}_{\bullet}
\end{equation*}
un \href{20250120163759-categoria_complessi_di_catene.org}{morfismo},
\begin{equation*}
f_{\bullet} = \set{f_{n}: C_{n}\longrightarrow D_{n}}_{n \in \Z}
\end{equation*}

Allora \(\forall\, n \in \Z\) il morfismo \(f_{\bullet}\) induce
\begin{equation*}
\begin{tikzcd}[ampersand replacement=\&,column sep=large,row sep=tiny]
	{H_n(\mathcal{C}_{\bullet})} \& {H_n(\mathcal{C}_{\bullet}')} \\
	{z_n+B_n(\mathcal{C}_{\bullet})} \& {f_n(z_n)+B_n(\mathcal{C}_{\bullet}')}
	\arrow["{f_{\star}}", from=1-1, to=1-2]
	\arrow[maps to, from=2-1, to=2-2]
\end{tikzcd}
\end{equation*}
tra i \href{20250120164857-modulo_di_omologia_dei_complessi_di_catene.org}{moduli di omologia}.
\begin{prop}
Si ha che:
\begin{itemize}
\item \(f_{\star}\) è ben definita;
\item \(f_{\star}\) è un \href{20241206115416-morfismi_r_moduli.org}{morfismo di moduli}.
\end{itemize}
\end{prop}
\begin{proof}
Si consideri il seguente diagramma commutativo:
\begin{equation*}
\begin{tikzcd}
        \cdots & {C_{n+1}} & {C_n} & {C_{n-1}} & \cdots \\
        \cdots & {D_{n+1}} & {D_n} & {D_{n-1}} & \cdots
        \arrow[from=1-1, to=1-2]
        \arrow["{\partial_{n+1}^C}", from=1-2, to=1-3]
        \arrow["{f_{n+1}}"', from=1-2, to=2-2]
        \arrow["{\partial_n^C}", from=1-3, to=1-4]
        \arrow["{f_n}"', from=1-3, to=2-3]
        \arrow[from=1-4, to=1-5]
        \arrow["{f_{n-1}}", from=1-4, to=2-4]
        \arrow[from=2-1, to=2-2]
        \arrow["{\partial_{n+1}^D}"', from=2-2, to=2-3]
        \arrow["{\partial_n^D}"', from=2-3, to=2-4]
        \arrow[from=2-4, to=2-5]
\end{tikzcd}
\end{equation*}

Per mostrare la buona definizione è necessario mostrare che:
\begin{enumerate}
\item \uline{Se \(z_{n} \in Z_{n}(\mathcal{C}_{\bullet})\), allora \(f_{n}(z_{n}) \in Z_{n}(\mathcal{D}_{\bullet})\)}.

Sia quindi \(z_{n} \in Z_{n}(\mathcal{C}_{\bullet})\). Allora \(\partial_{n}^{C} z_{n} = 0\), e pertanto \(f_{n-1} \partial_{n}^{C} z_{n} = 0\). Per commutatività:
\begin{equation*}
 \partial_{n}^{D} f_{n}(z_{n}) = f_{n-1} \partial_{n}^{C} z_{n} = 0
\end{equation*}
e pertanto \(f_{n}(z_{n}) \in Z_{n}(\mathcal{D}_{\bullet})\).

\item \uline{Se \([z_{n}]=[z_{n}'] \in H_{n}(\mathcal{C}_{\bullet})\), allora \([f_{n}(z_{n})] = [f_{n}(z_{n}')]\)}.

Se \([z_{n}]=[z_{n}']\), allora esiste \(h \in C_{n+1}\) tale che
\begin{equation*}
 z_{n} = z_{n}' + \partial_{n+1}^{C} h.
\end{equation*}
Allora:
\begin{align*}
 f_{n}(z_{n}) &= f_{n}(z_{n}' + \partial_{n+1}^{C} h) = f_{n}(z_{n}') + f_{n}(\partial_{n+1}^{C} h)\\
 &= f_{n}(z_{n}') + \partial_{n+1}^{D} f_{n+1} (h)
\end{align*}
e quindi \([f_{n}(z_{n})] = [f_{n}(z_{n}')]\).
\end{enumerate}

Il fatto che sia un morfismo segue banalmente.
\end{proof}
\end{document}
