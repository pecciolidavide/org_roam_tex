% Intended LaTeX compiler: pdflatex
\documentclass[../main]{subfiles}


\begin{document}

\section{Morfismo di prefasci}
\label{sec:org0f2f1c0}
\begin{definizione}
Sia \(X\) \href{20250103145124-topologia.org}{spazio topologico} e siano \(\mathcal{F}, \mathcal{G}\) due \href{20250324165349-prefascio.org}{prefasci} su \(X\). Un \textbf{morfismo} di prefasci \(f:\mathcal{F}\to \mathcal{G}\) è, per ogni \(U \subseteq X\) aperto, un \href{20241126100904-categoria.org}{morfismo}
\begin{equation*}
f_{U} : \mathcal{F}(U)\to \mathcal{G}(U)
\end{equation*}
tali che, per ogni \(V \subseteq U \subseteq X\) aperti, il seguente diagramma commuti:
\begin{equation*}
\begin{tikzcd}[ampersand replacement=\&,cramped]
	{\mathcal{F}(U)} \& {\mathcal{G}(U)} \\
	{\mathcal{F}(V)} \& {\mathcal{G}(V)}
	\arrow["{f_U}", from=1-1, to=1-2]
	\arrow["{\rho^U_V}"', from=1-1, to=2-1]
	\arrow["{\rho_V^U}", from=1-2, to=2-2]
	\arrow["{f_V}"', from=2-1, to=2-2]
\end{tikzcd}
\end{equation*}
\end{definizione}

In particolare, considerata la \href{20250325192239-fascio_come_funtore.org}{definizione di prefascio come funtore}\footnote{Vedi ``\href{20241205131958-funtore.org}{Funtore}''}, un morfismo tra prefasci è una \href{20241205132705-trasformazioni_naturali.org}{trasformazione naturale} tra i due funtori.
\subsection{Morfismo identità tra prefasci}
\label{sec:org326496e}
\begin{definizione}
Sia \(X\) uno \href{20250103145124-topologia.org}{spazio topologico} e \(\mathcal{F}\) un \href{20250324165349-prefascio.org}{prefascio}. Il \hyperref[sec:org0f2f1c0]{morfismo} identità \(\Id_{\mathcal{F}}: \mathcal{F}\to \mathcal{F}\) è, per ogni \(U \subseteq X\) aperto, \((\Id_{\mathcal{F}})_{U} \coloneqq \Id_{U}\)\footnote{\(\Id_{U}\) è la \href{20250310111151-funzione_identita.org}{Funzione identità}.}.
\end{definizione}
\subsection{Composizione di morfismi di prefasci}
\label{sec:orgb41731f}
\begin{definizione}
Sia \(X\) uno \href{20250103145124-topologia.org}{spazio topologico} e siano \(\mathcal{F}, \mathcal{G}, \mathcal{H}\) \href{20250324165349-prefascio.org}{prefasci} su \(X\). Siano
\begin{align*}
f: \mathcal{F} &\to \mathcal{G}\\
g: \mathcal{G} &\to \mathcal{H}
\end{align*}
due \hyperref[sec:org0f2f1c0]{morfismi di prefasci}. Per ogni \(U \subseteq X\) aperto si hanno i \href{20241126100904-categoria.org}{morfismi}
\begin{equation*}
\begin{tikzcd}[ampersand replacement=\&,cramped]
	{\mathcal{F}(U)} \& {\mathcal{G}(U)} \& {\mathcal{H}(U)}
	\arrow["{f_U}", from=1-1, to=1-2]
	\arrow["{g_U}", from=1-2, to=1-3]
\end{tikzcd}
\end{equation*}
la cui composizione \(g_{U}\circ f_{U} \eqqcolon (g\circ f)_{U}\).

Questo definisce un \hyperref[sec:org0f2f1c0]{morfismo} \(g\circ f: \mathcal{F}\to \mathcal{H}\).
\end{definizione}
\subsection{Isomorfismo di prefasci}
\label{sec:orgc882df9}
\begin{definizione}
Sia \(X\) uno \href{20250103145124-topologia.org}{spazio topologico} e siano \(\mathcal{F},\mathcal{G}\) due \href{20250324165349-prefascio.org}{prefasci}. Un \hyperref[sec:org0f2f1c0]{morfismo} \(f:\mathcal{F}\to \mathcal{G}\) è detto \uline{isomorfismo} se esiste un \hyperref[sec:org0f2f1c0]{morfismo} \(g:\mathcal{G}\to \mathcal{F}\) tale che\footnote{Vedi:
\begin{itemize}
\item \hyperref[sec:orgb41731f]{Composizione di morfismi di prefasci}
\item \hyperref[sec:org326496e]{Morfismo identità tra prefasci}
\end{itemize}}
\begin{equation*}
f\circ g = \Id_{\mathcal{G}},\qquad g\circ f = \Id_{\mathcal{F}}
\end{equation*}
\end{definizione}

\begin{prop}
Un \hyperref[sec:org6b25c0e]{morfismo} di (\hyperref[sec:org0f2f1c0]{pre})\hyperref[sec:org6b25c0e]{fasci} \(f:\mathcal{F}\to\mathcal{G}\) è un \hyperref[sec:orgc882df9]{isomorfismo}  se e solo se per ogni \(U \subseteq X\) \href{20250103145124-topologia.org}{aperto}, \(f_{U} : \mathcal{F}(U)\to \mathcal{G}(U)\) è un \href{20241128162125-isomorfismo.org}{isomorfismo}.
\end{prop}
\section{Morfismo di fasci}
\label{sec:org6b25c0e}
\begin{definizione}
Se \(\mathcal{F}\) e \(\mathcal{G}\) sono due fasci, un \uline{morfismo di fasci} \(f:\mathcal{F}\to \mathcal{G}\) è il \hyperref[sec:org0f2f1c0]{morfismo tra prefasci} \(f:\mathcal{F}\to \mathcal{G}\).
\end{definizione}
\end{document}
