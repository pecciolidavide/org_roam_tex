% Intended LaTeX compiler: pdflatex
\documentclass[../main]{subfiles}


\begin{document}

\section{Varietà Algebrica Quasi Proiettiva QP}
\label{sec:orgde1fed3}
Sia \(\K\) un \href{20241231112713-campo_algebricamente_chiuso.org}{campo algebricamente chiuso}.
\subsection{Definizione}
\label{sec:org81fd829}
Una \textbf{varietà quasi proiettiva} è (sono fatti equivalenti):
\begin{enumerate}
\item un \href{20250103145124-topologia.org}{aperto} di una \href{20241231123223-varieta_algebrica_proiettiva.org}{varietà proiettiva};
\item un \href{20250107112308-insieme_localmente_chiuso.org}{localmente chiuso} (ovvero intersezione di un aperto e di un chiuso) di \(\mathds{P}^{n}\). (vedi \href{20241231115051-spazio_proiettivo.org}{Spazio Proiettivo})
\end{enumerate}
\subsubsection{Topologia di Zariski qp}
\label{sec:org524e00d}
La topologia di Zariski di una varietà qp è la \href{20250103163814-sottospazio_topologico.org}{topologia di sottospazio} rispetto alla \href{20250103180214-topologia_di_zariski_proiettiva.org}{topologia di Zariski proiettiva}.
\subsubsection{Osservazione}
\label{sec:org7c633a4}
Una varietà proiettiva è una varietà QP, ma anche una \href{20241231114256-varieta_algebrica_affine.org}{Varietà Algebrica Affine} è una varietà QP.
\end{document}
