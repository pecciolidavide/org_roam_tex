% Intended LaTeX compiler: pdflatex
\documentclass[../main]{subfiles}


\begin{document}

\section{Omologia Simpliciale per un complesso simpliciale generato}
\label{sec:org74e604b}
Sia \(R\) un \href{20241219112842-pid.org}{PID}.
Sia \(\sigma_{n}\) un qualsiasi \href{20250121121923-simplessi.org}{simplesso} in \(\R^{m}\), e sia \(K=K(\sigma_{n})\) il \href{20250121124147-complesso_simpliciale.org}{complesso simpliciale} \href{20250121125446-complesso_simpliciale_generato_da_un_simplesso.org}{generato} da \(\sigma_{n}\).

Allora, l'\href{20250121160827-omologia_simpliciale.org}{omologia simpliciale} di \(K\) è
\begin{equation*}
H_{q}(K) = \begin{cases}
0 & q>0\\
R\cdot e_{0} & q=0
\end{cases}
\end{equation*}
dove \(e_{0}\) indica lo \(0\) simplesso generato da un punto \href{20250121124310-scheletro_di_un_complesso_simpliciale.org}{di \(K^{0}\)}, \href{20250122100541-notazione_per_i_simplessi.org}{in accordo con la notazione}.
\end{document}
