% Intended LaTeX compiler: pdflatex
\documentclass[../main]{subfiles}


\begin{document}

\section{Esempi fondamentali di varietà topologiche}
\label{sec:org08e47ce}
\subsection{Rn è una varietà topologica}
\label{sec:org0c598a7}
Si ha che \(\R^{n}\) è una \href{20250111092123-varieta_topologica.org}{varietà topologica}. Infatti è \href{20250109155715-spazio_topologico_di_hausdorff.org}{T2}, a \href{20250111142303-spazio_topologico_a_base_numerabile.org}{base numerabile} (vedi \href{20250111142303-spazio_topologico_a_base_numerabile.org}{Topologia Euclidea è a base numerabile}) e
\begin{equation*}
\operatorname{Id}: \R^{n} \longrightarrow \R^{n}
\end{equation*}
è un \href{20250111142332-omeomorfismo.org}{omeomorfismo}.
\subsection{Aperti di Rn sono varietà topologica}
\label{sec:org3768235}
Nella topologia di sottospazio, \(U \subseteq \R^{n}\) aperto ereditano da \(\R^{n}\) essere T2 e a base numerabile. Inoltre
\begin{equation*}
\operatorname{Id}: U \longrightarrow U \subseteq \R^{n}
\end{equation*}
è un omeomorfismo.
\end{document}
