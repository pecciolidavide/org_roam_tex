% Intended LaTeX compiler: pdflatex
\documentclass[../main]{subfiles}


\begin{document}

\begin{prop}
Sia \(f:\R^{n}\to \R\) una \href{20250103103252-funzione_continua.org}{funzione continua} e \href{20250701140551-funzione_lineare_a_tratti.org}{lineare a tratti}. Allora esiste una \href{20250624155858-neurone_artificiale.org}{rete neurale} \href{20250624155858-neurone_artificiale.org}{feedforward}, con \href{20250624155858-neurone_artificiale.org}{neuroni} \href{20250624155858-neurone_artificiale.org}{ReLU} e \href{20250624155858-neurone_artificiale.org}{lineari}, che \href{20250708122736-exact_learning_machine_learning.org}{rappresenta esattamente} \(f\). Inoltre ha \(L\) \href{20250624155858-neurone_artificiale.org}{layers}, con
\begin{equation*}
L=\begin{cases}
2(k+1) & \text{se }n+1=2^{k}\\
2(\floor{\log_{2}(n+1)} + 2)&\text{altrimenti}
\end{cases}
\end{equation*}
\label{prop10.2.7}
\end{prop}
\end{document}
