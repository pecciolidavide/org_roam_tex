% Intended LaTeX compiler: pdflatex
\documentclass[../main]{subfiles}


\begin{document}

\section{Teorema di Church}
\label{sec:orgcd762c0}
Sia \(L_{Q}\) il \href{20250130162057-linguaggio_del_prim_ordine.org}{linguaggio} dell'\href{20250608093604-aritmetica.org}{aritmetica di Robinson} e sia \(\#\) una \href{20250609104647-buona_codifica_di_un_linguaggio.org}{buona codifica per \(L_{Q}\)}. Si utilizza la notazione de ``\href{20250609104647-buona_codifica_di_un_linguaggio.org}{Aritmetizzazione della sintassi}''
\subsection{Teorema}
\label{sec:org239774e}

Sia \(L\supset L_{Q}\). Allora l'insieme
\begin{equation*}
\operatorname{Val}^{\#}(L) \coloneqq \set{\termcode{\varphi}\mid \varphi\text{ è un }L\text{-enunciato valido}}
\end{equation*}
non è \href{20250216173925-insieme_ricorsivo.org}{ricorsivo}. Vedi ``\href{20250131123530-formula_valida.org}{Formula valida}''.
\end{document}
