% Intended LaTeX compiler: pdflatex
\documentclass[../main]{subfiles}


\begin{document}

\renewcommand{\L}{\mathcal{L}} % Il linguaggio
\newcommand{\U}{\mathcal{U}} % Modello mostro

Si utilizza la \href{20250612143636-notazione_teoria_dei_modelli.org}{Notazione della TEORIA DEI MODELLI}

Sia \(\L\) un \href{20250130162057-linguaggio_del_prim_ordine.org}{linguaggio}, \(T\) una \href{20250130114950-teoria_del_prim_ordine.org}{teoria} \href{20250131123151-teoria_completa.org}{completa} senza \href{20250131122945-modello_di_un_insieme_di_formule.org}{modelli} finiti e \(\U\) un \href{20250617095548-modello_lambda_saturo.org}{modello saturo} di \href{20241213101756-cardinalita.org}{cardinalità} \href{20250211123155-cardinale_limite_forte.org}{inaccessibile} \(\kappa>\card{\L}+ \omega\). \(\U\) è un \href{20250617102733-modello_mostro.org}{modello mostro}.

Sia \(A \subseteq \U\) piccolo.
\section{Tipo invariante su un insieme}
\label{sec:org7f8921b}
\begin{definizione}
Sia \(\varphi(x;z) \in \L\) una \href{20250131103317-formula_del_prim_ordine.org}{formula}.
Un \href{20251026161720-phi_formula_su_un_insieme.org}{\(\varphi\)-tipo} \(p(x) \subseteq \mathcal{L}_{\varphi}(\mathcal{U})\) si dice \uline{\(A\)-invariante} se per ogni \(\psi(x;\overline{b}) \in p(x)\) e per ogni \href{20250214120959-mappe_tra_strutture_del_prim_ordine.org}{automorfismo}\footnote{Vedi anche ``\href{20251020151126-insieme_degli_automorfismi_di_una_struttura_del_prim_ordine.org}{Gruppo degli automorfismi di una struttura del prim'ordine}''} \(f \in \operatorname{Aut}(\mathcal{U}/A)\):
\begin{equation*}
p(x)\vdash \psi(x;f\overline{b})
\end{equation*}
ovvero esistono \(\psi_{1}(x),\dots,\psi_{n}(x) \in p(x)\) tali che
\begin{equation*}
\U\vDash \forall x\ \big(\psi_{1}(x) \land \dots \land \psi_{n}(x) \implies \psi(x;f\overline{b})\big).
\end{equation*}
\end{definizione}

\uline{Notazione}: scriveremo \(p(x)\vdash fp(x)\).

\href{20250515141706-da_finire.org}{DA FINIRE}
\end{document}
