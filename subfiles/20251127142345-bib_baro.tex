% Intended LaTeX compiler: pdflatex
\documentclass[../main]{subfiles}


\begin{document}

\section{(bib) Neuman - Physiology and Medicine of Hyperbaric Oxygen Therapy, 2008}
\label{sec:orgcfba9a6}
Certamente. Di seguito, fornirò un \textbf{report} \textbf{\textbf{MOLTO APPROFONDITO}} sui capitoli specifici richiesti, con particolare attenzione agli aspetti rilevanti per la subacquea, l'ossigenoterapia iperbarica (HBOT) e la medicina subacquea.

Questo report è basato sul testo ``Physiology and Medicine of Hyperbaric Oxygen Therapy''.

---

\href{20250515141706-da_finire.org}{DA FINIRE}

\#\# Capitolo 4: The Physics of Hyperbaric Oxygen Therapy (La Fisica dell'Ossigenoterapia Iperbarica)

Questo capitolo stabilisce i principi fisici che sono fondamentali per comprendere sia l'immersione che l'HBOT.

\#\#\# Unità di Pressione e Misurazione
L'unità di pressione più utile e riconosciuta internazionalmente nella comunità iperbarica è l'atmosfera assoluta (\textbf{\textbf{ATA}}, \textbf{Atmospheres Absolute}).
\subsection{La pressione atmosferica al livello del mare è pari a \textbf{\textbf{1 ATA}} (che equivale a 14.7 psi o 101.32 kPa).}
\label{sec:org98260a2}
\subsection{La pressione assoluta (\(P_{\text{abs}}\)) è la somma della pressione atmosferica (\(P_{\text{atm}}\)) e della pressione misurata dal manometro (\(P_{\text{gauge}}\)): \(P_{\text{abs}} = P_{\text{atm}} + P_{\text{gauge}}\). Ad esempio, un trattamento a 2.4 ATA al livello del mare richiede una pressione manometrica di 1.4 ATA (1.0 ATA + 1.4 ATA = 2.4 ATA).}
\label{sec:org7134f86}

\#\#\# Pressione Parziale dei Gas (Partial Pressure)
Il concetto di \textbf{\textbf{pressione parziale}} di un singolo gas in una miscela è cruciale.
\subsection{È direttamente proporzionale alla percentuale di quel gas nel volume totale della miscela.}
\label{sec:org8cbd496}
\subsection{Determina direttamente la \textbf{\textbf{quantità di gas assorbito nei tessuti}}. Questo è fondamentale per la comprensione di condizioni come l'embolia gassosa arteriosa (AGE), la malattia da decompressione (DCS) e la tossicità da ossigeno (\textbf{oxygen toxicity}), come trattato nei Capitoli 13, 14 e 23.}
\label{sec:org4f6b761}
\subsection{Per esempio, a 6.0 ATA (165 fsw), la pressione parziale di ossigeno (\(P_{\text{O}_{2}}\)) è 1.26 ATA e quella dell'azoto (\(P_{\text{N}_{2}}\)) è 4.74 ATA, assumendo aria (21\% \(O_2\), 79\% \(N_2\)) [75t].}
\label{sec:orgdce8a78}

\#\# Capitolo 5: Clearance to Dive and Fitness for Work (Idoneità all'Immersione e al Lavoro Iperbarico)

Questo capitolo affronta la valutazione medica dei candidati subacquei e del personale iperbarico, un ruolo spesso ricoperto dai medici iperbarici.

\#\#\# Filosofia e Modelli di Valutazione
\subsection{\textbf{\textbf{Ruolo del Medico Iperbarico:}} I medici iperbarici esaminano spesso subacquei sportivi e commerciali per valutarne l'idoneità e, talvolta, per trattare lesioni legate all'immersione.}
\label{sec:orgcc5dc0d}
\subsection{\textbf{\textbf{Modelli Amministrativi:}} Esistono differenze regionali significative.}
\label{sec:org210576d}
\begin{itemize}
\item \textbf{\textbf{Subacquei Ricreativi:}} Il modello più diffuso prevede che il candidato compili un \textbf{\textbf{questionario di \textbf{screening}}} (come il Recreational Scuba Training Council [RSTC] Medical Statement). Se le risposte sono positive, è richiesta una consulenza medica.
\item \textbf{\textbf{Subacquei Commerciali/Occupazionali:}} È quasi sempre richiesta una \textbf{\textbf{valutazione medica obbligatoria}} da parte di un medico formato in medicina subacquea (ad esempio in Australia e Nuova Zelanda) [81f].
\end{itemize}
\subsection{\textbf{\textbf{Approccio di I Primi Principi (A Generic Approach):}} Per valutare l'idoneità in presenza di una condizione medica, si devono porre tre domande fondamentali:}
\label{sec:org0e108a1}
\begin{enumerate}
\item La condizione potrebbe predisporre a una \textbf{\textbf{malattia da immersione}} (\textbf{diving illness})?
\item La condizione potrebbe essere \textbf{\textbf{provocata dall'immersione}}?
\item La condizione potrebbe \textbf{\textbf{compromettere la sicurezza o le prestazioni}} del subacqueo sott'acqua?
\end{enumerate}
\subsection{Rispondere ``sì'' o ``non lo so'' a una di queste richiede una revisione più approfondita, e un ``sì'' definitivo richiede una valutazione rischio/beneficio.}
\label{sec:org6dbd832}

\#\#\# Condizioni Mediche Specifiche con Rilevanza Subacquea
\subsection{\textbf{\textbf{Forame Ovale Pervio (Patent Foramen Ovale, PFO):}} Vi è una comprovata sovra-rappresentazione di \textbf{shunt} destro-sinistro (come il PFO) tra i subacquei che hanno sofferto di DCS neurologica o vestibolococleare. Un \textbf{shunt} maggiore è stato riscontrato nel \textbf{\textbf{71\%}} dei casi vestibolococleari (rispetto al 12\% dei controlli).}
\label{sec:orgbdc79b4}
\begin{itemize}
\item \textbf{\textbf{Gestione PFO:}} Per un aspirante subacqueo con PFO ``largo'' o facilmente provocabile, l'immersione è sconsigliata. Per un PFO ``piccolo'' e con \textbf{shunting} minimo, il candidato può essere idoneo dopo consulenza sul rischio trascurabile.
\end{itemize}
\subsection{\textbf{\textbf{Asma:}} L'asma è comune nella popolazione subacquea. Il principale rischio è il \textbf{\textbf{barotrauma polmonare}} (PBT), che può portare ad AGE. È difficile dimostrare con certezza che i subacquei asmatici siano a rischio significativamente maggiore di infortuni da immersione. La valutazione può includere la spirometria o il test di provocazione bronchiale [102, 103f].}
\label{sec:org89455f4}
\subsection{\textbf{\textbf{Radiografia del Torace (CXR):}} La necessità di routine è controversa. Tuttavia, una CXR può rilevare anomalie polmonari (ad esempio, bolle enfisematose) che predispongono a PBT e AGE. Tali anomalie sono state associate a casi di PBT e AGE.}
\label{sec:org8654157}
\subsection{\textbf{\textbf{Esame Neurologico:}} È fondamentale documentare quantitativamente eventuali anomalie per avere un punteggio di riferimento (\textbf{baseline score}) nel caso in cui il subacqueo necessiti di trattamento per DCS neurologica.}
\label{sec:org667303a}

\#\# Capitolo 8: Pulmonary Gas Exchange, Oxygen Transport, and Tissue Oxygenation (Scambio Gassoso Polmonare, Trasporto di Ossigeno e Ossigenazione Tissutale)

Questo capitolo copre la fisiologia fondamentale del trasporto dell'ossigeno, essenziale per capire come la respirazione in ambiente iperbarico influenzi il corpo.

\#\#\# Fisiologia del Trasporto di Ossigeno
L'HBOT aumenta l'ossigenazione dei tessuti, in particolare grazie all'aumento della \(P_{\text{O}_{2}}\) arteriosa.
\subsection{L'O2 si lega all'emoglobina (Hb) per essere trasportato, ma in condizioni iperbariche, la quantità di \textbf{\textbf{ossigeno disciolto fisicamente}} nel plasma aumenta drasticamente (secondo la Legge di Henry, discussa nel Capitolo 4).}
\label{sec:org7f3fc6a}
\subsection{Durante la respirazione di O2 al livello del mare, la \(P_{\text{O}_{2}}\) nel sangue venoso giugulare raramente superava i 60 mmHg; la respirazione di ossigeno a \textbf{\textbf{3.5 ATA}} porta a profili di ossigenazione tissutale che riflettono un'alta differenza artero-venosa [177, 603f].}
\label{sec:org191a1d6}

\#\#\# Vasocostrizione Iperossica
L'iperossia (alta \(P_{\text{O}_{2}}\)) induce \textbf{\textbf{vasocostrizione arteriolarica}}.
\subsection{Questo effetto è più marcato nei vasi del \textbf{\textbf{cervello}}, della \textbf{\textbf{retina}} e del muscolo scheletrico.}
\label{sec:orgb9b5c12}
\subsection{Nonostante la riduzione del flusso sanguigno (ad esempio, cerebrale, che diminuisce l'apporto di ossigeno), l'enorme aumento di \(O_2\) disciolto garantisce comunque un aumento netto di ossigeno fornito ai tessuti.}
\label{sec:orgef5b338}
\subsection{Un'elevata \(P_{\text{aCO}_{2}}\) (anidride carbonica arteriosa) aumenta il flusso sanguigno cerebrale, il che a sua volta \textbf{\textbf{aumenta la tossicità dell'ossigeno per il sistema nervoso centrale (CNS)}} e il rischio di convulsioni.}
\label{sec:org9d5f3c6}

\#\# Capitolo 10: Pressure Effects on Human Physiology (Effetti della Pressione sulla Fisiologia Umana)

Questo capitolo distingue gli effetti della pressione (idrostatica o barometrica, \textbf{pressure per se}) dagli effetti delle pressioni parziali dei gas.

\#\#\# Range Fisiologico di Pressione
\subsection{Gli esseri umani, senza una cabina pressurizzata, possono tollerare un \textbf{range} di pressione che va da circa \textbf{\textbf{0.17 ATA}} (a 50.000 piedi di altitudine) a \textbf{\textbf{70 ATA}} (circa 2300 piedi sotto la superficie del mare). Questa variazione di circa 700 volte è tollerabile solo con l'uso di attrezzature specializzate per la respirazione.}
\label{sec:orga22fc85}

\#\#\# Effetti delle Pressioni Parziali dei Gas (Partial Pressure Effects)
\subsection{\textbf{\textbf{Narcosi da Azoto (Rapture of the Deep):}} Quando si respira aria in profondità, l'aumento della \(P_{\text{N}_{2}}\) provoca narcosi (come descritto da Paul Bert nel 1878 e da Cousteau 100 anni dopo). La narcosi da azoto è un problema neurologico che limita le immersioni in aria a circa 4 ATA (99 fsw) [214f, 220].}
\label{sec:orgfe4c731}
\subsection{\textbf{\textbf{Tossicità da Ossigeno del SNC (CNS Oxygen Toxicity):}} È il limite più pressante nell'uso dell'ossigeno iperbarico (in HBOT e immersione con Nitrox). Può verificarsi rapidamente a \(P_{\text{O}_{2}}\) di \textbf{\textbf{2 ATA o superiore}} e può portare a \textbf{\textbf{convulsioni violente}} con poco o nessun preavviso.}
\label{sec:org933a447}

\#\#\# Effetti della Pressione Idrostatica (Hydrostatic Pressure)
\subsection{\textbf{\textbf{HPNS (High-Pressure Nervous Syndrome):}} A profondità estreme (oltre 10–15 ATA), la compressione idrostatica diretta influenza il SNC, producendo un aumento dell'attività neurale che risulta in tremori.}
\label{sec:org7f28635}
\subsection{\textbf{\textbf{Effetti Moderati (<5 ATA):}} Anche i livelli moderati di pressione (quelli utilizzati in HBOT) possono avere effetti misurabili che non sono semplicemente dovuti alla \(P_{\text{O}_{2}}\): alterazioni nella diffusione dei gas polmonari, resistenza delle vie aeree, flusso sanguigno tissutale locale, e segnalazione neuronale.}
\label{sec:org9902bb8}

\#\# Capitolo 13: Arterial Gas Embolism (Embolia Gassosa Arteriosa)

L'embolia gassosa arteriosa (AGE) è una grave complicanza iatrogena o da immersione.

\#\#\# Eziologia e Rischio Subacqueo
\subsection{\textbf{\textbf{Causa Principale:}} Nella subacquea, l'AGE è quasi sempre una conseguenza del \textbf{\textbf{barotrauma polmonare}} (PBT). Il PBT si verifica quando l'aria rimane intrappolata nei polmoni durante la risalita (anche di soli 1 metro) e l'espansione del gas rompe gli alveoli, permettendo all'aria di entrare nel flusso sanguigno arterioso.}
\label{sec:org429a8f8}
\subsection{\textbf{\textbf{Sintomi:}} I sintomi sono estremamente variabili, ma sono riferibili all'occlusione del sistema vascolare del SNC. I segni suggestivi di AGE includono \textbf{\textbf{perdita di coscienza transitoria, cecità o disorientamento}} dopo l'emersione.}
\label{sec:orgfefba0b}
\subsection{\textbf{\textbf{Diagnosi:}} L'AGE è una \textbf{\textbf{diagnosi clinica}}. Qualsiasi paziente con una storia suggestiva di AGE richiede un esame neurologico dettagliato e una consulenza iperbarica, poiché le lesioni neurologiche sottili sono difficili da escludere in fase acuta e un ritardo nel trattamento può rendere le lesioni irreversibili.}
\label{sec:org2cecd85}

\#\#\# Trattamento Iperbarico (HBOT)
\subsection{\textbf{\textbf{Priorità:}} La \textbf{\textbf{ricompressione immediata}} è indicata per l'AGE confermata o sospetta.}
\label{sec:org37ae671}
\subsection{\textbf{\textbf{Protocolli:}} Il trattamento mira a ridurre meccanicamente la dimensione delle bolle (Legge di Boyle) e ad aumentare la \(P_{\text{O}_{2}}\) tissutale.}
\label{sec:org07da75f}
\begin{itemize}
\item Il \textbf{\textbf{Tavolo 6A della Marina USA (U.S. Navy Treatment Table 6A)}} è stato storicamente raccomandato per l'AGE, prevedendo un'escursione a \textbf{\textbf{6 ATA (165 fsw)}} per massimizzare la compressione delle bolle, seguita dalla porzione di respirazione a ossigeno identica al Tavolo 6 [59, 327f].
\item Tuttavia, alcuni studi mettono in discussione l'ulteriore vantaggio di una compressione superiore a \textbf{\textbf{2.82 ATA}} {[}327f].
\end{itemize}

\#\# Capitolo 14: Decompression Sickness (Malattia da Decompressione)

La DCS si verifica quando i gas inerti disciolti (principalmente azoto) formano bolle nei tessuti e nel sangue a seguito di una decompressione troppo rapida.

\#\#\# Patogenesi e Sintomatologia
\subsection{\textbf{\textbf{Fattori di Rischio:}} Il rischio è legato al carico di gas inerte (profilo tempo-profondità), alla velocità di risalita e alla presenza di \textbf{shunt} destro-sinistro (PFO), che permettono alle bolle venose (VGE) di raggiungere la circolazione arteriosa.}
\label{sec:orgc26e1e6}
\subsection{\textbf{\textbf{Sintomi Comuni:}} I sintomi sono estremamente variabili, ma i più comuni includono \textbf{\textbf{dolore}} (bends), parestesia, debolezza muscolare e affaticamento [314f]. La \textbf{\textbf{DCS cardiorespiratoria}} (\textbf{chokes}) è caratteristica di alti livelli di VGE.}
\label{sec:orgdab4247}
\subsection{\textbf{\textbf{Diagnosi:}} La DCS è una \textbf{\textbf{diagnosi clinica}} basata su storia e esame fisico. La differenziazione tra DCS neurologica e AGE si basa su fattori come il tempo di insorgenza e la presenza di alterazione della coscienza.}
\label{sec:org9e575ba}

\#\#\# Trattamento Iperbarico e Protocolli
\subsection{\textbf{\textbf{Ricompressione:}} Il trattamento primario è la ricompressione, per ridurre la dimensione delle bolle e favorirne la riassorbimento.}
\label{sec:orgccec069}
\subsection{\textbf{\textbf{Protocolli Standard:}} Il \textbf{\textbf{Tavolo 6 della Marina USA (U.S. Navy Table 6)}} (2.8 ATA o 60 fsw, respirando O2) è il \textbf{gold standard} per la DCS [323, 324t]. La raccomandazione iniziale per la DCS neurologica è la compressione a 18 msw (60 fsw) respirando O2 al 100\%.}
\label{sec:org2f4ef64}
\subsection{\textbf{\textbf{Uso di Gas Inerti Terapeutici:}} L'uso di un gas inerte diverso (come l'elio in una miscela Heliox o Nitrox) durante la ricompressione profonda (es. Tavolo 6A) può facilitare l'eliminazione delle bolle [322, 327f].}
\label{sec:orgd31c115}
\subsection{\textbf{\textbf{Trattamento in Acqua (In-Water Recompression, IWR):}} Può essere utilizzato in luoghi remoti quando non è disponibile una camera iperbarica. Richiede la somministrazione di ossigeno al 100\% tramite maschera a pieno facciale, protezione termica e un accompagnatore.}
\label{sec:org22bd5d7}
\subsection{\textbf{\textbf{Terapia Aggiuntiva (Adjunctive Therapy):}} La somministrazione di liquidi (cristalloidi) è considerata cruciale (Livello 1C) [341, 342t]. La lidocaina può essere usata come adiuvante per la DCS neurologica [342t, 375].}
\label{sec:org6b43e9f}
\subsection{\textbf{\textbf{Ritardo nel Trattamento (Delayed Treatment):}} Sebbene il beneficio diminuisca col ritardo, la ricompressione può essere comunque utile. Per i sintomi lievi, un ritardo non è associato a un peggioramento a lungo termine dell'esito.}
\label{sec:orgdfd8bba}

\#\# Capitolo 22: Effects of Pressure (Effetti della Pressione)

Questo capitolo si concentra sulle complicazioni dirette causate dalle variazioni di pressione, note come barotraumi.

\#\#\# Barotrauma dell'Orecchio Medio (Middle Ear Barotrauma, MEBT)
\subsection{\textbf{\textbf{Complicazione più Comune:}} Il MEBT è la complicanza più frequente dell'HBOT (incidenza 3.8 - 12 per 1000 esposizioni). È causato dall'incapacità di equilibrare la pressione tra l'orecchio medio e l'ambiente esterno, spesso dovuto alla disfunzione della Tuba di Eustachio (ET).}
\label{sec:orgeab7a09}
\subsection{\textbf{\textbf{Prevenzione e Trattamento:}} Si prevengono con l'educazione alle manovre di compensazione (come Valsalva). In caso di disfunzione cronica o incapacità di compensare, si può ricorrere alla \textbf{\textbf{miringotomia}} (incisione del timpano) o all'inserimento di un \textbf{\textbf{tubo di ventilazione}} (\textbf{tympanostomy tube}) [518, 519f, 585].}
\label{sec:org3b65aab}

\#\#\# Barotrauma dell'Orecchio Interno (Inner Ear Barotrauma, IEBT)
\subsection{\textbf{\textbf{Rischio Raro ma Serio:}} IEBT è raro ma può portare a perdita dell'udito e disfunzione vestibolare. Le lesioni includono rotture della membrana della finestra rotonda o della membrana di Reissner.}
\label{sec:orgcfc0efa}
\subsection{\textbf{\textbf{Chirurgia Otologica:}} La \textbf{\textbf{stapedectomia}} (intervento chirurgico per l'otosclerosi) è considerata una controindicazione relativa all'immersione e all'HBOT, a causa del potenziale rischio di penetrazione della protesi e danni all'orecchio interno dovuti ai cambiamenti di pressione. Si sconsiglia l'immersione anche dopo la timpanoplastica.}
\label{sec:org49fb44c}

\#\#\# Altri Barotraumi
\subsection{\textbf{\textbf{Barodontalgia:}} Dolore ai denti causato da variazioni di pressione. È stato segnalato in aviatori e subacquei, ma non nei pazienti sottoposti a HBOT.}
\label{sec:orgf3ba62d}
\subsection{\textbf{\textbf{Impianti Cocleari:}} Sebbene siano stati riportati casi di subacquei con impianti funzionanti, si raccomanda di consultare il produttore prima di iniziare l'HBOT per garantire la sicurezza del paziente e mantenere la garanzia.}
\label{sec:org205cbf6}

\#\# Capitolo 23: Oxygen Toxicity (Tossicità da Ossigeno)

L'esposizione a elevate pressioni parziali di ossigeno può causare tossicità chimica che colpisce principalmente il SNC, i polmoni e la retina [604f].

\#\#\# Tossicità del Sistema Nervoso Centrale (CNS Toxicity)
\subsection{\textbf{\textbf{Limite più Critico:}} È la limitazione più pressante dell'HBOT e dell'immersione con gas arricchiti.}
\label{sec:org0024863}
\subsection{\textbf{\textbf{Manifestazione:}} Si manifesta con \textbf{\textbf{convulsioni}} (crisi epilettiche). Possono essere precedute da sintomi prodromici (tremori, nausea, contrazioni muscolari, alterazioni visive), ma questi segnali non sono affidabili e non sempre si verificano prima dell'esordio convulsivo.}
\label{sec:orgdb2d1e6}
\subsection{\textbf{\textbf{Fattori di Rischio:}} La ritenzione di \(CO_{2}\) aumenta il rischio di convulsioni perché induce vasodilatazione cerebrale, che aumenta l'esposizione del cervello all'ossigeno.}
\label{sec:org659f9ef}
\subsection{\textbf{\textbf{Tolleranza:}} La tolleranza alla tossicità del SNC è stata definita utilizzando un indice obiettivo pre-convulsivo, come i decrementi progressivi nel rapporto \(T_{I}/T_{T}\) (componente temporale della ventilazione).}
\label{sec:org5e8a95d}

\#\#\# Tossicità Polmonare (Pulmonary Toxicity)
\subsection{\textbf{\textbf{Effetti:}} La tossicità polmonare è causata dal danno chimico all'epitelio alveolare e all'endotelio capillare [604f].}
\label{sec:org7577986}
\subsection{\textbf{\textbf{Misurazione:}} Viene misurata dal declino progressivo e significativo della \textbf{\textbf{Capacità Vitale (VC)}} {[}613, 614f]. Sono state generate curve iperboliche per prevedere la tolleranza polmonare all'ossigeno in base alla pressione e alla durata dell'esposizione [617f, 621f].}
\label{sec:org981c952}

\#\#\# Estensione della Tolleranza (Oxygen Tolerance Extension)
\subsection{\textbf{\textbf{Intermittent Exposure (Air Breaks):}} La tossicità polmonare e del SNC può essere ritardata in modo significativo \textbf{\textbf{alternando periodi di respirazione di ossigeno (O2) con periodi di respirazione di aria ambiente (air breaks)}}. Questo metodo è essenziale per la sicurezza nei protocolli HBOT prolungati, come il Tavolo 6 della Marina USA.}
\label{sec:org9378bb7}

\#\# Ossigeno Iperbarico nella Subacquea (Sintesi)

Per riassumere l'interesse per la subacquea nei capitoli analizzati:

\textbf{\textbf{Fisica e Pressione (Capitoli 4 \& 10):}} Il subacqueo deve comprendere come le pressioni parziali dei gas (in particolare \(P_{\text{N}_{2}}\) e \(P_{\text{O}_{2}}\)) influenzino la tossicità e la narcosi. La profondità (fsw) è direttamente correlata alla pressione assoluta (ATA). A profondità estreme (oltre 10-15 ATA), la limitazione diventa la \textbf{\textbf{HPNS}} (effetti diretti della pressione).

\textbf{\textbf{Medicina Subacquea (Capitoli 5, 13, 14, 22):}}
\subsection{\textbf{\textbf{Idoneità:}} La valutazione medica per l'immersione è cruciale, specialmente per condizioni come PFO o anamnesi di asma, che aumentano il rischio di DCS e PBT/AGE.}
\label{sec:orgddb5859}
\subsection{\textbf{\textbf{Incidenti:}} AGE (causata da PBT, spesso in risalita rapida) e DCS (formazione di bolle) sono le emergenze primarie.}
\label{sec:orgc227b95}
\subsection{\textbf{\textbf{Trattamento:}} La \textbf{\textbf{ricompressione immediata}} è il trattamento fondamentale, spesso utilizzando il Tavolo 6 della Marina USA (60 fsw/2.8 ATA \(O_2\)).}
\label{sec:org09cccbd}
\subsection{\textbf{\textbf{Barotrauma:}} La complicanza più comune per i subacquei è il \textbf{\textbf{barotrauma dell'orecchio medio}} (MEBT), ma l'IEBT è più grave; si deve prestare attenzione anche alla presenza di chirurgia otologica pregressa (es. stapedectomia).}
\label{sec:org3900345}

\textbf{\textbf{Limiti e Sicurezza (Capitoli 10 \& 23):}}
\subsection{La sicurezza in immersione è limitata dalla \textbf{\textbf{tossicità da ossigeno del SNC}} a \(P_{\text{O}_{2}} \geq 2\) ATA e dalla \textbf{\textbf{narcosi da azoto}} a \(P_{\text{N}_{2}} \geq 3-4\) ATA.}
\label{sec:orgdb52276}
\subsection{L'uso di \textbf{*interruzioni di aria (*air breaks})** è la tecnica standard per estendere la tolleranza all'ossigeno durante i trattamenti prolungati di ricompressione (come il Tavolo 6).}
\label{sec:org41d3d58}
\end{document}
