% Intended LaTeX compiler: pdflatex
\documentclass[../main]{subfiles}

\usepackage[hyperref]{biblatex}
\date{}
\title{}
\begin{document}

\section{Teoria delta-model-completa}
\label{sec:org2516e8e}
Si utilizza la \href{20250612143636-notazione_teoria_dei_modelli.org}{Notazione della TEORIA DEI MODELLI}
\begin{itemize}
\item Sia \(\mathcal{L}\) un \href{20250130162057-linguaggio_del_prim_ordine.org}{linguaggio del prim'ordine}
\item Sia \(\Delta\) un insieme di formule chiuso per sostituzione di variaibli e che contenga la formula ``\(x=y\)''.
\item Se \(C \subseteq \set{\forall , \exists , \land,\lor,\lnot}\) è un insieme di connettivi, \(C\Delta\) denota la chiusura di \(\Delta\) per i connettivi in \(C\). \(\Delta^{\pm} \coloneqq \set{\lnot}\Delta\).
\end{itemize}

\begin{definizione}
Una \href{20250130114950-teoria_del_prim_ordine.org}{teoria} \(T\) è \(\Delta\)-\emph{model}-completa se ogni \href{20250214120959-mappe_tra_strutture_del_prim_ordine.org}{\(\Delta\)-morfismo} \href{20250213105339-funzione_parziale.org}{totale} \(k:M\to N\) tra modelli di \(T\) è un \(\set{\exists ,\forall ,\land ,\lor}\!\Delta\)-morfismo.
\end{definizione}
\end{document}
