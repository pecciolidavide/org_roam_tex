% Intended LaTeX compiler: pdflatex
\documentclass[../main]{subfiles}


\begin{document}

\section{Sfera di Riemann biolomorfa alla retta proiettiva complessa}
\label{sec:org3da7995}
\begin{prop}
La \href{20260127112905-sfera_di_riemann.org}{sfera di Riemann \(\C_{\infty}\)} è \href{20260128144717-isomorfismo_tra_superfici_di_riemann.org}{biolomorfa} allo \href{20241231115051-spazio_proiettivo.org}{spazio proiettivo} \href{20260127112924-esempi_fondamentali_di_varieta_complesse.org}{complesso \(\mathds{P}^{1}_{\C}\)}.
\end{prop}
\begin{proof}
Si consideri
\begin{equation*}
F: \C_{\infty} \to \mathds{P}^{1}_{\C}:\qquad z \mapsto \begin{cases}
(z:1) & z \in \C\\
(1:0) & z = \infty
\end{cases}
\end{equation*}
\(F\) è ovviamente biunivoca. Consideriamo le carte:
\begin{equation*}
\begin{aligned}
A &\coloneqq \C \subseteq \C_{\infty}\\
B &\coloneqq \C_{\infty} \setminus \set{0}
\end{aligned}\qquad \begin{aligned}
U_{0} &= \set{z_{0}\neq 0} \subseteq \mathds{P}^{1}_{\C}(z_{0}:z_{1})\\
U_{1} &= \set{z_{1}\neq 0} \subseteq \mathds{P}^{1}_{\C}(z_{0}:z_{1})
\end{aligned}
\end{equation*}
Allora si hanno le seguenti composizioni:
\begin{equation*}
\begin{tikzcd}[ampersand replacement=\&,row sep=small]
	\& A \&\& {U_1} \&\& \C \\
	\&\&\& {(z_0\duepunti z_1)} \&\& {\frac{z_0}{z_1}} \\
	\& z \&\&\&\& z \\
	\\
	\\
	\\
	\\
	\C \& B \&\& {U_0} \&\& \C \\
	w \& \begin{array}{c} \begin{cases} 1/w & w\neq 0\\ \infty & w=0\end{cases} \end{array} \&\& {(z_0 \duepunti z_1)} \&\& {\frac{z_1}{z_0}} \\
	w \& \begin{array}{c} \begin{cases} 1/w & w\neq 0\\ \infty & w=0\end{cases} \end{array} \&\& \begin{array}{c} \begin{cases} (1/w\duepunti 1) & w\neq 0\\ (1\duepunti 0) & w=0\end{cases} \end{array} \&\& \begin{array}{c} \begin{cases} w & w\neq 0\\ 0 & w=0\end{cases} \end{array}
	\arrow["F", from=1-2, to=1-4]
	\arrow["\varphi", from=1-4, to=1-6]
	\arrow[maps to, from=2-4, to=2-6]
	\arrow["\Id"', maps to, from=3-2, to=3-6]
	\arrow[from=8-1, to=8-2]
	\arrow["F", from=8-2, to=8-4]
	\arrow[from=8-4, to=8-6]
	\arrow[maps to, from=9-1, to=9-2]
	\arrow[maps to, from=9-4, to=9-6]
	\arrow[maps to, from=10-1, to=10-2]
	\arrow["\Id"', maps to, bend right=30pt, from=10-1, to=10-6]
	\arrow[from=10-2, to=10-4]
	\arrow[from=10-4, to=10-6]
\end{tikzcd}
\end{equation*}
Entrambe sono l'identità, che è olomorfa. Anche le inverse sono olomorfe, quindi si ha la tesi.
\end{proof}
\end{document}
