% Intended LaTeX compiler: pdflatex
\documentclass[../main]{subfiles}


\begin{document}

Sia \(\K\) un \href{20241231112713-campo_algebricamente_chiuso.org}{campo algebricamente chiuso} e siano \(X \subseteq \A^{n}\), \(Y \subseteq \A^{m}\) \href{20241231114256-varieta_algebrica_affine.org}{varietà algebriche affini}. (Vedi \href{20241231114009-spazio_affine.org}{Spazio Affine}).
\section{Osservazione}
\label{sec:orgd5e7e38}
I \href{20250104110524-morfismo_tra_varieta_algebriche_affini.org}{morfismi} tra varietà algebriche affini non sono \href{20250104114559-funzione_chiusa.org}{chiusi}.
\subsection{Controesempio}
\label{sec:org36faf0d}
Consideriamo \(X=V(xy-1) \subseteq \A^{2}\) (vedi \href{20241231112823-radici_polinomiali.org}{Luogo di zeri} e \href{20241219113434-anello_dei_polinomi.org}{Anello-dei-polinomi}), e sia
\begin{align*}
F: \A^{2} &\longrightarrow \A^{1}\\
(x,y) &\longmapsto x
\end{align*}
Si ha che \(F(X)=\A^{1}\setminus\set{0}\), \href{20250104115050-spazio_affine_unidimensionale_senza_un_punto_non_e_chiuso.org}{che non è chiuso} in \(\A^{1}\).
\end{document}
