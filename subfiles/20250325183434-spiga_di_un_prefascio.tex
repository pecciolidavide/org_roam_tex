% Intended LaTeX compiler: pdflatex
\documentclass[../main]{subfiles}


\begin{document}

\section{Spiga di un fascio di gruppi}
\label{sec:org7f22005}
Sia \(X\) uno \href{20250103145124-topologia.org}{spazio topologico}, e sia \(\mathcal{F}\) un \href{20250324165349-prefascio.org}{prefascio} di \href{20241205141146-gruppo_abeliano.org}{gruppi}.
Sia \(p \in X\) fissato e sia
\begin{equation*}
I(p) \coloneqq \set{U \subseteq X\mid U\text{ intorno aperto di } p}
\end{equation*}
l'insieme degli \href{20250111142313-intorno.org}{intorni} \href{20250103145124-topologia.org}{aperti} di \(p\).

Posta l'\href{20250113175700-unione_disgiunta.org}{unione disgiunta} insiemistica:
\begin{equation*}
S\coloneqq \coprod_{U \in I(p)} \mathcal{F}(U)
\end{equation*}
si pone su \(S\) una \href{20250113110148-relazione_di_equivalenza.org}{relazione di equivalenza}: siano \(f,g \in S\); allora \(f \in \mathcal{F}(U), g \in \mathcal{F}(V)\) e \(p \in U\cap V\).
\begin{itemize}
\item \(f\sim g\) se e solo se esiste \(W \in I(p)\), \(W \subseteq U\cap V\) tale che \(\restriction{f}{W} = \restriction{g}{W}\) in \(\mathcal{F}(W)\).
\end{itemize}
\begin{definizione}
La \uline{spiga di \(\mathcal{F}\) in \(p\)} è l'insieme \href{20250114100810-quoziente_rispetto_a_relazione_di_equivalenza.org}{quoziente}:
\begin{equation*}
\mathcal{F}_{p} \coloneqq S/\sim
\end{equation*}
Per indicare la classe di equivalenza di \(f \in S\) dentro a \(\mathcal{F}_{p}\) si scriverà \([f]_{p}\): questo è il \uline{germe} di \(f\) in \(p\).
\end{definizione}
\begin{oss}
La spiga eredita una naturale struttura di \href{20241205141146-gruppo_abeliano.org}{gruppo}: se \(s_{1},s_{2} \in \mathcal{F}_{p}\), allora esiste \(U \in I(p)\) ed esistono \(f_{1},f_{2} \in \mathcal{F}(U)\) tali che
\begin{equation*}
s_{1}=[f_{1}]_{p},\qquad s_{2}=[f_{2}]_{p}
\end{equation*}

Si definisce: \(s_{1}+s_{2}\coloneqq [f_{1}+f_{2}]_{p}\).
\end{oss}

\begin{prop}
Per ogni \(U \in I(p)\), la seguente mappa è un \href{20241206115531-morfismo_di_gruppi.org}{omomorfismo di gruppi}:
\begin{align*}
\mathcal{F}(U) &\longrightarrow \mathcal{F}_{p}\\
f &\longmapsto [f]_{p}
\end{align*}
\end{prop}
\begin{prop}
Si ha che \([f]_{p} = 0\) se e solo se per qualche \(W \in I(p)\) si ha \(\restriction{f}{W} = 0\).
\end{prop}
\begin{prop}
Se \(\mathcal{F}\) è un \href{20250324174728-fascio.org}{fascio}, \(U \subseteq X\) aperto, \(f \in \mathcal{F}(U)\).
\begin{quote}
\(f=0\) se e solo se, per ogni \(p \in U\), \([f]_{p} = 0\).
\end{quote}
\end{prop}

\begin{proof}
(\(\Rightarrow\)): siccome \(f\mapsto [f]_{p}\) è omomorfismo di gruppi, segue la tesi.

(\(\Leftarrow\)):
Sia \(f \in \mathcal{F}(U)\) tale che, per ogni \(p \in U\), \([f]_{p} =0\). Allora, per ogni \(p \in U\), esiste \(V_{p}' \in I(p)\) tale che \(\restriction{f}{V_{p}'} = 0\).

Allora, posto \(V_{p} \coloneqq V_{p}'\cap U\), \(\set{V_{p}}_{p \in U}\) è un \href{20250103164252-ricoprimento.org}{ricoprimento} aperto di \(U\), e \(\restriction{f}{V_{p}} = 0\). Per l'\href{20250324174728-fascio.org}{assioma di unicità di fascio}, \(f = 0\) in \(\mathcal{F}(U)\).
\end{proof}
\end{document}
