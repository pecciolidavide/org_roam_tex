% Intended LaTeX compiler: pdflatex
\documentclass[../main]{subfiles}


\begin{document}

\section{Pullback di un divisore tra superfici di Riemann}
\label{sec:org1712818}
Siano \(X,Y\) \href{20260127112828-superficie_di_riemann.org}{superfici di Riemann}, \(F:X\to Y\) \href{20260128143822-funzione_olomorfa_su_una_superficie_di_riemann.org}{olomorfa} non costante. Si indichi con \(\operatorname{Div}(X), \operatorname{Div}(Y)\) il \href{20260201235551-divisore_di_una_superficie_di_riemann.org}{gruppo dei divisori}.

\begin{definizione}
Per ogni \(q \in Y\), si definisce il \href{20260201235551-divisore_di_una_superficie_di_riemann.org}{divisore}:\footnote{\(\mathrm{mult}_{p}\) è la \href{20260129104215-forma_normale_locale_per_superfici_di_riemann.org}{molteplicità}}
\begin{equation*}
     F^{*}(q) \coloneqq \sum_{p \in F^{-1}(q)} (\mathrm{mult}_{p}\,F)\cdot p
\end{equation*}
Questo ha \href{20260201235333-gruppo_delle_funzioni_in_z.org}{supporto} su \(F^{-1}(q)\), che è \href{20260128123515-sottoinsieme_discreto.org}{discreto} per il \href{20260128144016-principio_di_identita_per_funzioni_olomorfe_su_superfici_di_riemann.org}{principio d'identità}.
\end{definizione}

\begin{oss}
Se \(X, Y\) sono \href{20250103163701-spazio_topologico_compatto.org}{compatte}, allora il \href{20260201235551-divisore_di_una_superficie_di_riemann.org}{grado del divisore} è uguale al \href{20260129171444-teorema_del_grado_per_olomorfismi_tra_superfici_di_riemannn.org}{grado di \(F\)}:
\begin{equation*}
\deg F^{*}(q) = \deg F.
\end{equation*}
\end{oss}

\begin{definizione}
Sia \(D = \sum_{q \in Y} m_{q}\cdot q \in \operatorname{Div}(Y)\). Il \textbf{pullback di \(D\) lungo \(F\)} è
\begin{equation*}
F^{*}D \coloneqq \sum_{q \in Y} m_{q} F^{*}q \in \operatorname{Div}(X).
\end{equation*}
\end{definizione}

\begin{oss}
Se \(X, Y\) sono compatte, allora
\begin{equation*}
\deg F^{*}D = (\deg F)(\deg D).
\end{equation*}
Infatti, detto \(D=\sum_{i=1}^{r} m_{i} q_{i}\):
\begin{align*}
F^{*}D &= \sum_{i=1}^{r} m_{i} F^{*}(q_{i}) = \sum_{i=1}^{r} m_{i} \cdot \left( \sum_{p \in F^{-1}(q_{i})} (\mathrm{mult}_{p} F) \cdot p \right)\\[4ex]
\deg F^{*} D &=
\sum_{i=1}^{r} m_{i} \cdot \parentesi{\deg F^{*}(q_{i})}{\left( \sum_{p \in F^{-1}(q_{i})} (\mathrm{mult}_{p} F)\right)} \\
& = \sum_{i=1}^{r} m_{i}\cdot \deg F^{*}(q_{i}) = \deg F \cdot \left(\sum_{i=1}^{r} m_{i}\right) = (\deg F)\cdot (\deg D).
\end{align*}
\end{oss}
\subsection{Pullback come omomorfismo}
\label{sec:org2c754ca}

\begin{oss}
Si ha che \(F^{*}: \operatorname{Div}(Y) \to \operatorname{Div}(X)\) è \href{20241206115531-morfismo_di_gruppi.org}{omomorfismo di gruppi}.
\end{oss}

\begin{lem}
Sia \(f \in \mathcal{M}(Y)\setminus\set{0}\)\footnote{Con \(\mathcal{M}\) indichiamo il \href{20260128144151-fascio_delle_funzioni_meromorfe_su_una_superficie_di_riemann.org}{fascio delle funzioni meromorfe a valori complessi}.}. Allora\footnote{Con \(\operatorname{div}f\) si indica il \href{20260202095650-divisore_di_una_funzione_meromorfa.org}{divisore di \(f\)}.}
\begin{equation*}
F^{*}(\operatorname{div} f) = \operatorname{div}(f\circ F).
\end{equation*}
\end{lem}
\begin{proof}
Sia \(p \in X\).
\begin{itemize}
\item Il coefficiente di \(p\) in \(\operatorname{div}(f\circ F)\) è l'\href{20260128144450-ordine_di_una_funzione_meromorfa_su_una_superficie_di_riemann.org}{ordine} \(\mathrm{ord}_{p} (f\circ F)\);
\item il coefficiente di \(q\coloneqq F(p)\) in \(\operatorname{div}(f)\) è \(\mathrm{ord}_{q}f\), e pertanto il coefficiente di \(p\) in \(F^{*}(\operatorname{div} f)\) è \((\mathrm{ord}_{p} f) \cdot (\mathrm{mult}_{p}\,F)\).
\end{itemize}

Inoltre si ha che \(\mathrm{ord}_{p} (f\circ F) = (\mathrm{ord}_{p} f) \cdot (\mathrm{mult}_{p}\,F)\).

Infatti, localmente,
\begin{itemize}
\item detto \(n\coloneqq \mathrm{ord}_{p} (f)\), \(f\) si scrive come \(z^{n} \cdot g(z)\), con \(g(z)\) olomorfa e mai nulla;
\item per il Teorema di Forma Normale, detto \(m \coloneqq \mathrm{mult}_{p}\, F\), \(F\) si scrive come \(z^{m}\).
\end{itemize}

Componendo localmente: \(f\circ F\) è
\begin{equation*}
(z^{m})^{n} \cdot g(z) = z^{m\cdot n} \cdot g(z)
\end{equation*}
e quindi, per le \href{20260128144433-ordine_di_una_funzione_meromorfa.org}{proprietà dell'ordine}, si ha che
\begin{equation*}
\mathrm{ord}_{p} (f\circ F) = m\cdot n %
= (\mathrm{ord}_{p} f) \cdot (\mathrm{mult}_{p}\,F).
\qedhere
\end{equation*}
\end{proof}

\begin{oss}
Quindi il pullback di un \href{20260202095650-divisore_di_una_funzione_meromorfa.org}{divisore principale} è principale, e \(F^{*}\) passa al \href{20250127093819-quoziente_di_gruppo_e_sottogruppo.org}{quoziente}, definendo un omomorfismo tra i gruppi di Picard:
\begin{equation*}
F^{*}:\operatorname{Pic}(Y) \to \operatorname{Pic}(X).
\end{equation*}
\end{oss}
\end{document}
