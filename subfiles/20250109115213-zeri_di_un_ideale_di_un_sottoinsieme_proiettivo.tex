% Intended LaTeX compiler: pdflatex
\documentclass[../main]{subfiles}

\begin{document}

\section{Zeri di un ideale di un sottoinsieme proiettivo}
\label{sec:org9f526aa}
Sia \(\K\) un \href{20241231112713-campo_algebricamente_chiuso.org}{campo algebricamente chiuso}.
\subsection{Proposizione}
\label{sec:org2a6be94}
Sia \(Y \subseteq \mathds{P}^{n}\) un sottoinsieme, e sia \(I(Y)\) il suo \href{20250103144124-ideale_di_un_sottoinsieme.org}{ideale}. Allora, considerando la \href{20250103180214-topologia_di_zariski_proiettiva.org}{topologia di Zariski}, si ha che la \href{20241231123223-varieta_algebrica_proiettiva.org}{varietà algebrica proiettiva} (vedi \href{20250102183523-luogo_di_zeri_di_un_ideale_omogeneo.org}{Luogo di zeri di un ideale omogeneo})
\begin{equation*}
V\left(I(Y)\right)=\overline{Y}
\end{equation*}
(vedi \href{20241231112823-radici_polinomiali.org}{Luogo di zeri}) dove con \(\overline{Y}\) si intende la \href{20250103144944-chiusura_topologica.org}{chiusura} di \(Y\) nella \href{20250103145124-topologia.org}{topologia} citata.
\subsubsection{Osservazione}
\label{sec:orgfacd40d}
Se \(Y\) è una varietà algebrica, allora
\begin{equation*}
Y=\overline{Y} = V\left(I(Y)\right)
\end{equation*}
Inoltre, per la definizione di \href{20250103144124-ideale_di_un_sottoinsieme.org}{ideale di un sottoinsieme} in ambito proiettivo,
\begin{equation*}
Y=V\left(I(Y)^{h}\right)
\end{equation*}
(vedi \href{20241231121125-polinomi_omogenei.org}{Polinomi Omogenei})
\end{document}
