% Intended LaTeX compiler: pdflatex
\documentclass[../main]{subfiles}

\usepackage[hyperref]{biblatex}
\date{}
\title{}
\begin{document}

\section{Funtore Controvariante}
\label{sec:orgc41a692}
\begin{definizione}
Siano \(\mathcal{C},\mathcal{C'}\) due \href{20241126100904-categoria.org}{categorie}. Un \textbf{funtore controvariante}
\begin{equation*}
F:\mathcal{C}\longrightarrow \mathcal{C'}
\end{equation*}
è una funzione tale che
\begin{enumerate}
\item per ogni \(X \in \operatorname{Ob}(\mathcal{C})\) si ha che \(F(X) \in \operatorname{Ob}(\mathcal{C'})\);
\item per ogni \(f \in \operatorname{Hom}_{\mathcal{C}} (X,Y)\) si ha che
\begin{equation*}
F(f)\in \operatorname{Hom}_{\mathcal{C'}}\left(F(Y),F(X)\right)
\end{equation*}
\end{enumerate}

tali che
\begin{enumerate}
\item \(F(f\circ g) = F(g)\circ F(f)\)
\item \(F(\1_X)=\1_{F(X)}\)
\end{enumerate}
\end{definizione}
\begin{esempio}
Consideriamo il funtore \(\cat{Top} \longrightarrow \cat{Vect_{\R}}\)\footnote{Vedi:
\begin{itemize}
\item \href{20241205115600-categoria_top.org}{Categoria-Top}
\item \href{20241205115727-categoria_vectr.org}{Categoria-VectR}
\end{itemize}} tale che
\begin{equation*}
X \longrightarrow C^0(X) \coloneqq \set{\varphi: X \longrightarrow \R \text{ continua}}
\end{equation*}
e tale che \(X \dfreccia{f} Y\) venga mandata in
\begin{align*}
C^0(X)  &\longleftarrow C^0(Y) \\
(X \dfreccia{\varphi\circ f} \R) &\longmapsfrom (Y \dfreccia{\varphi} \R)
\end{align*}
\end{esempio}
\uline{Notazione}.
Se \(F\)  è un funtore \textbf{controvariante} e \(X \dfreccia{f} Y\) allora
\begin{equation*}
F(X)\sfreccia{F(f)}F(Y)
\end{equation*}
e si indica con \(f^* \coloneqq F(f)\). \(f^*\) si dice \textbf{pull back} di \(f\) tramite \(F\).
\end{document}
