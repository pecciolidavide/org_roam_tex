% Intended LaTeX compiler: pdflatex
\documentclass[../main]{subfiles}


\begin{document}

\section{Ipotesi del continuo generalizzata}
\label{sec:org5662b1f}
Contesto: \href{20250130104245-morse_kelly_set_theory.org}{Morse Kelly Set Theory}
\begin{definizione}
\begin{description}
\item[{Assumento \href{20250206171508-axiom_of_choiche.org}{AC}:}] L'\uline{ipotesi del continuo generalizzata} è\footnote{Vedi:
\begin{itemize}
\item \href{20250203111003-ordinali.org}{Ordinali}
\item \href{20250207123015-aritmentica_per_gli_ordinali.org}{Aritmetica per gli ordinali}
\item \href{20250207122627-funzione_aleph.org}{Funzione Aleph}
\item \href{20250612151505-aritmetica_dei_cardinali.org}{Elevamento a potenza di cardinali}
\end{itemize}}
\begin{equation*}
  \forall\alpha \in \operatorname{Ord}\ (2^{\aleph_{\alpha}} = \aleph_{\alpha\dotplus 1}).
\end{equation*}

Equivalentemente, significa che\footnote{Vedi:
\begin{itemize}
\item \href{20241213101756-cardinalita.org}{Cardinalità}
\end{itemize}}
\begin{equation*}
  \forall X \subseteq \parti{\aleph_{\alpha}}\ (\card{X} \le \aleph_{\alpha} \lor \card{X} = \card{\parti{\aleph_{\alpha}}})
\end{equation*}

\item[{Senza \href{20250206171508-axiom_of_choiche.org}{AC}:}] L'ipotesi del continua generalizzata diventa:\footnote{Vedi:
\begin{itemize}
\item \href{20250130104245-morse_kelly_set_theory.org}{Insieme delle parti per MK}
\item ``\(\embeds\)'': \href{20241219101956-funzione_iniettiva.org}{Classe si inietta}
\item ``\(\equipotenti\)'': \href{20250619101109-classi_equipotenti.org}{Classi equipotenti MK}
\end{itemize}}
\begin{equation*}
  \forall X\ \forall A \subseteq \parti{X}\ (\card{A} \embeds X \lor A \equipotenti \parti{X})
\end{equation*}
e quando descritta in questo modo \textbf{implica \href{20250206171508-axiom_of_choiche.org}{AC}}.
\end{description}
\end{definizione}
\end{document}
