% Intended LaTeX compiler: pdflatex
\documentclass[../main]{subfiles}


\begin{document}

\section{Estensione di un simplesso singolare in uno spazio stellato}
\label{sec:org189bc5b}
Sia \(X \subseteq \R^{N}\) \href{20250122154613-insieme_stellato.org}{stellato} rispetto a \(x_{0} \in X\).

Ogni \href{20250122133435-simplesso_singolare.org}{simplesso singolare} \(\sigma:\Delta_{q}\longrightarrow X\)\footnote{Dove \(\Delta_{q}\) è il \href{20250121122324-simplesso_standard.org}{\(q\)-simplesso standard}.} può essere esteso ad
\begin{equation*}
J_{\sigma}: \Delta_{q+1}\longrightarrow X
\end{equation*}
tramite il \uline{joint a \(x_{0}\)}: infatti, siccome un generico punto di \(\Delta_{q+1}\) può essere scritto come:
\begin{equation*}
t e_{q+1} + (1-t)\cdot p,\qquad t \in [0,1]
\end{equation*}
con \(p \in \Delta_{q} \subseteq \Delta_{q+1}\), allora
\begin{align*}
J_{\sigma}: \Delta_{q+1} &\longrightarrow X\\
(t e_{q+1} + (1-t)\cdot p) &\longmapsto (t e_{q+1} + (1-t)\cdot \sigma(p))
\end{align*}

\begin{prop}
Se \(\varepsilon_{i}^{q+1}\) sono gli \href{20250122133535-operatori_di_facciata_del_simplesso_standard.org}{operatori di facciata}, allora
\begin{equation*}
J_{\sigma}\circ \varepsilon_{q+1}^{q+1} = \sigma
\end{equation*}
ed inoltre, per ogni \(i=0,\dots,q\) si ha che
\begin{equation*}
J_{\sigma}\circ\varepsilon_{i}^{q+1} = J_{\sigma\circ\varepsilon^{q}_{i}}
\end{equation*}
\end{prop}
\begin{definizione}
Si definisce la \href{20241206115416-morfismi_r_moduli.org}{mappa} \(s_{q}: S_{q}(X) \to S_{q+1}(X)\)\footnote{\(S_{q}(X)\) è il \href{20250122133435-simplesso_singolare.org}{modulo delle catene singolari}, ed è la \href{20241213095808-somma_diretta.org}{somma diretta \(R^{(\Sigma_{q})}\)}, con \href{20241213094625-modulo_libero.org}{base} proprio \(\Sigma_{q}\).}, che per ogni \(\sigma \in \Sigma_{q}\):
\begin{equation*}
s_{q}(\sigma) \coloneqq (-1)^{q+1} J_{\sigma}.
\end{equation*}
\end{definizione}
\begin{lem}
Per ogni \(q\) si ha e per ogni \(\sigma \in \Sigma_{q}\):\footnote{I \(\partial_{q}\) sono i \href{20250122133614-mappa_di_bordo_tra_moduli_di_catene_singolari.org}{morfismi di bordo}.}
\begin{equation*}
\partial_{q+1} (s_{q}\sigma) = \sigma - s_{q-1}(\partial_{q}\sigma).
\end{equation*}
\end{lem}
\begin{proof}
Per definizione si ha
\begin{equation*}
\partial_{q+1} (s_{q}\sigma)  = \sum_{i=0}^{q+1} (-1)^{i} s_{q}(\sigma) \circ \varepsilon_{i}^{q+1} = \sum_{i=0}^{q+1} (-1)^{i} (-1)^{q+1} J_{\sigma} \circ \varepsilon_{i}^{q+1}.
\end{equation*}
Per la proposizione precedente, \(J_{\sigma}\circ \varepsilon_{q+1}^{q+1}=\sigma\) e \(J_{\sigma} \circ \varepsilon_{i}^{q+1} = J_{\sigma\circ \varepsilon_{i}^{q}}\) altrimenti.
\begin{align*}
\partial_{q+1} (s_{q}\sigma) &= (-1)^{q+1}(-1)^{q+1} \sigma - \sum_{i=0}^{q} (-1)^{i} (-1)^{q}J_{\sigma\circ \varepsilon_{i}^{q}} = \sigma - \sum_{i=0}^{q} (-1)^{i} (-1)^{q}J_{\sigma\circ \varepsilon_{i}^{q}}\\
&= \sigma - \sum_{i=0}^{q}(-1)^{i} s_{q-1}(\sigma\circ \varepsilon_{i}^{q}) = \sigma - s_{q-1} \left(\sum_{i=0}^{q}(-1)^{i} \sigma \circ \varepsilon_{i}^{q}\right) \\
&= \sigma - s_{q-1} (\partial_{q} \sigma).%
\aedhere
\end{align*}
\end{proof}
\end{document}
