% Intended LaTeX compiler: pdflatex
\documentclass[../main]{subfiles}


\begin{document}

\section{Spazio topologico contraibile è aciclico}
\label{sec:org3e3a9d0}
Sia \(R\) un \href{20241219112842-pid.org}{PID}.
\begin{prop}
Se uno \href{20250103145124-topologia.org}{spazio topologico} \(X\) è \href{20250122155640-spazio_topologico_contraibile.org}{contraibile}, allora è \href{20250122154451-spazio_topologico_aciclico.org}{aciclico}.
\end{prop}
\begin{cor}
Se \(X\) è uno spazio topologico contraibile, allora per ogni \(q\), l'\href{20250122133631-omologia_singolare.org}{omologia singolare} di \(X\) è
\begin{equation*}
     \forall\,q>0\quad H_{q}(X) = 0, \qquad H_{0}(X) = R
\end{equation*}
\end{cor}
\begin{proof}
Segue dall'\href{20250122154153-calcolo_dell_omologia_del_punto.org}{omologia singolare del punto} tramite questo \href{20250122155528-teorema_dell_invarianza_per_omotopia.org}{corollario}.
\end{proof}
\end{document}
