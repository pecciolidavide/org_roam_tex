% Intended LaTeX compiler: pdflatex
\documentclass[../main]{subfiles}

\usepackage[hyperref]{biblatex}
\date{}
\title{}
\begin{document}

\section{Funzioni ricorsive per minimizzazione su un predicato}
\label{sec:org57b958b}
In questa sezione si identificano i \href{20250131155822-operazioni_insiemistiche_tra_classi_mk.org}{sottinsiemi} di \(\N^{k}\) (vedi \href{20250202130045-insieme_dei_numeri_naturali_mk.org}{Insieme dei numeri naturali MK}) con i \href{20250131103317-formula_del_prim_ordine.org}{predicati} \(k\)-ari (ovvero con \(k\) \href{20250131103429-variabile_libera_di_una_formula.org}{variabili libere}), per mezzo degli \href{20250131122913-soddisfazione_di_una_formula.org}{insiemi di verità}.

Scriveremo indifferentemente \((x_{1},\dots,x_{k}) \in P\) oppure \(P(x_{1},\dots,x_{k})\) per dire che \(\N\vDash P(x_{1},\dots,x_{k})\).
\subsection{Proposizione}
\label{sec:org125e579}

Se \(P \subseteq \N^{k+1}\) è un \href{20250216173925-insieme_ricorsivo.org}{predicato ricorsivo}, la \href{20250202170607-classe_relazione_binaria.org}{funzione} \(f: \N^{k}\to \N\) definita tramite l'\href{20250215151440-operatore_di_minimizzazione_non_limitato.org}{operatore di minimizzazione}:
\begin{equation*}
f(\oldvec{x}) \coloneqq \minim{y}{P(\oldvec{x},y)}
\end{equation*}
è \uline{ricorsiva}. Questa non è necessariamente una funzione \href{20250213105339-funzione_parziale.org}{totale}. \footnote{Infatti possono esistere degli \(\oldvec{x}\) tali che per nessun \(y \in \N\) si abbia \(P(\oldvec{x},y)\).}
\subsubsection{Dimostrazione}
\label{sec:org7fbb2cc}
\begin{equation*}
f(\oldvec{x}) = \minim{y}{\overline{\operatorname{sgn}}\left(\chi_{P}(\oldvec{x},y)\right)=0}
\end{equation*}
Vedi: \href{20250215141024-funzioni_primitive_ricordive.org}{Esempi di funzioni primitive ricorsive} e \href{20250215151458-funzioni_ricorsive.org}{Funzioni ricorsive}
\end{document}
