% Intended LaTeX compiler: pdflatex
\documentclass[../main]{subfiles}


\begin{document}

\section{Fascio delle funzioni olomorfe su una superficie di Riemann}
\label{sec:org6687dc0}
\begin{definizione}
Sia \(X\) una \href{20260127112828-superficie_di_riemann.org}{superficie di Riemann}. Il \uline{\href{20250324174728-fascio.org}{fascio} delle funzioni olomorfe} su \(X\), denotato con \(\mathcal{O}_{X}\), è definito come:
\begin{itemize}
\item per ogni \(U \subseteq X\): \(\mathcal{O}_{X}(U) = \set{f:U\subseteq X \to \C \mid f\text{ olomorfa}}\) l'\href{20260128143822-funzione_olomorfa_su_una_superficie_di_riemann.org}{insieme delle funzioni olomorfe}
\item le \href{20250205170515-restrizione_di_una_classe.org}{restrizioni ovvie}.
\end{itemize}
\end{definizione}
\end{document}
