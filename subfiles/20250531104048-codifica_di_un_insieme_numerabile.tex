% Intended LaTeX compiler: pdflatex
\documentclass[../main]{subfiles}

\usepackage[hyperref]{biblatex}
\date{}
\title{}
\begin{document}

\section{Codifica di un insieme numerabile}
\label{sec:org05206c1}
\begin{definizione}
Sia \(D\) un \href{20250130104331-insieme_mk.org}{insieme} \href{20250111143651-insieme_numerabile.org}{numerabile}. Una \uline{codifica} di \(D\) è una \href{20250202170607-classe_relazione_binaria.org}{funzione} \href{20241219101956-funzione_iniettiva.org}{iniettiva}
\begin{equation*}
\varphi: D\to \N.
\end{equation*}

Una codifica si dice \uline{ricorsiva} se il suo \href{20250202173528-dominio_range_e_campo_di_una_classe_relazione.org}{range} \(\operatorname{rng}(\varphi)\) è un \href{20250216173925-insieme_ricorsivo.org}{sottoinsieme ricorsivo} di \(\N\).
\end{definizione}

Inoltre, se \(D\) non è finito, \href{20250520143216-insieme_ricorsivo_come_range_di_funzione_ricorsiva_totale_crescente.org}{allora} esiste una \href{20250104111707-funzione_biunivoca.org}{biiezione} ricorsiva \(f:\operatorname{rng}(\varphi)\to \N\). Pertanto si può supporre che \(\operatorname{rng}(\varphi)=\N\).
\end{document}
