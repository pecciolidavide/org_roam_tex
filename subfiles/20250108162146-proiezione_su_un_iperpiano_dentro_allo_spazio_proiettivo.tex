% Intended LaTeX compiler: pdflatex
\documentclass[../main]{subfiles}


\begin{document}

\section{Proiezione su un iperpiano dentro allo spazio proiettivo}
\label{sec:org742d1c8}
Sia \(\K\) un \href{20241231112713-campo_algebricamente_chiuso.org}{campo algebricamente chiuso}. Si consideri \(\mathds{P}^{n}\) \href{20241231115051-spazio_proiettivo.org}{spazio proiettivo}, e sia \(H \subseteq \mathds{P}^{n}\) un \href{20250102144306-iperpiano_proeittivo.org}{iperpiano}, ovvero il \href{20241231112823-radici_polinomiali.org}{luogo degli zeri} di un \href{20241231112750-polinomio.org}{polinomio} di primo grado:
\begin{equation*}
H=V(F)
\end{equation*}

Sia \(p \in \mathds{P}^{n}\setminus H\) fissato. Allora per ogni \(q \in \mathds{P}^{n}\setminus\set{p}\) si definisce \(\ell_{pq}\) la \href{20250109101841-retta_proiettiva.org}{retta} che passa tra \(p\) e \(q\).
La \textbf{proiezione} da \(p\) su \(H\) è l'applicazione
\begin{align*}
\pi_{p}: \mathds{P}^{n}\setminus \set{p} &\longrightarrow H\\
q &\longmapsto \ell_{pq}\cap H
\end{align*}
\(\pi_{p}\) è un \href{20250104120600-morfismo_tra_varieta_algebriche_proiettive.org}{morfismo}, poiché è polinomiale. Infatti, scegliendo a dovere le coordinate, si ponga
\begin{align*}
p&=[0:\dots:0:1]\\
H&=V(x_{n})
\end{align*}
dunque, per ogni \(q =[q_{0}:\dots:q_{n}]\neq p\) si ha che
\begin{equation*}
\pi_{p}(q) = [q_{0}:\dots:q_{n-1}:0] \in H
\end{equation*}

Infatti, siccome la retta \(\ell_{pq}\) si può parametrizzare come
\begin{equation*}
\lambda \ p + \mu\ q,\qquad [\lambda:\mu] \in \mathds{P}^{1}
\end{equation*}
oppure con
\begin{equation*}
q+t\ p=[q_{0}:\dots:q_{n-1}:q_{n}+t],\qquad t \in \K
\end{equation*}
e il punto di \(\ell_{pq}\) con \(x_{n}=0\) ha coordinate \([q_{0}:\dots:q_{n-1}:0]\)

Se \(X \subseteq \mathds{P}^{n}\) e \(p\notin X\), allora \(\restriction{\pi_{p}}{X}\) è ancora un morfismo
\subsubsection{{\bfseries\sffamily TODO} Proiezione da uno spazio lineare\hfill{}\textsc{matematica\_lm:geo\_alg}}
\label{sec:org8d9ede8}
\end{document}
