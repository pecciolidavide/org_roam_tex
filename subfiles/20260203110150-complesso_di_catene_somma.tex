% Intended LaTeX compiler: pdflatex
\documentclass[../main]{subfiles}


\begin{document}

\section{Somma e intersezione di complessi di catene}
\label{sec:orgb0a909e}
Sia \(R\) un \href{20241205141119-anello.org}{anello} commutativo con unità, e sia \(\mathcal{C}_{\bullet}\) un \href{20250120163114-complesso_di_catene.org}{complesso di catene di \(R\)-moduli}.

\begin{definizione}
Siano \(\mathcal{S}_{\bullet}\) e \(\mathcal{T}_{\bullet}\) due \href{20250128151459-sottocomplesso_di_catene.org}{sottocomplessi}  di \(\mathcal{C}_{\bullet}\).
Si definiscono la \textbf{\textbf{somma}} \(\mathcal{S}_{\bullet} + \mathcal{T}_{\bullet}\) e l'\textbf{\textbf{intersezione}} \(\mathcal{S}_{\bullet} \cap \mathcal{T}_{\bullet}\) come i complessi di catene dati termine a termine dalla \href{20241206142802-sottomoduli.org}{somma} e \href{20241206142802-sottomoduli.org}{intersezione} degli \(R\)-moduli:
\begin{align*}
(S + T)_{n} &\coloneqq S_{n} + T_{n} \\
(S \cap T)_{n} &\coloneqq S_{n} \cap T_{n}
\end{align*}
I differenziali sono quelli indotti (ristretti) da \(\mathcal{C}_{\bullet}\), dato che la somma e l'intersezione di sottomoduli preservano la condizione di stabilità rispetto al bordo.
\end{definizione}
\end{document}
