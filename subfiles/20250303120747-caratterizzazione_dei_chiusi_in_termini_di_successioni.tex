% Intended LaTeX compiler: pdflatex
\documentclass[../main]{subfiles}

\usepackage[hyperref]{biblatex}
\date{}
\title{}
\begin{document}

\section{Caratterizzazione dei chiusi in termini di successioni}
\label{sec:org5608d0c}
\subsection{Proposizione}
\label{sec:org6bcb21b}
Sia \(X\) uno \href{20250103145124-topologia.org}{spazio topologico} \href{20250301193254-spazio_topologico_primo_numerabile.org}{primo numerabile}, e sia \(C \subseteq X\). Sono fatti equivalenti:
\begin{enumerate}
\item \(C\) è \href{20250103145124-topologia.org}{chiuso}.
\item Per ogni \href{20250115100904-successione.org}{successione} \(\set{a_{n}}_{n \in \N} \subseteq C\), se questa \href{20250115100930-convergenza_per_una_successione.org}{converge} a \(p \in X\), allora \(p \in C\).
\end{enumerate}
\subsubsection{Dimostrazione}
\label{sec:org417ac66}
Questo segue da
\begin{itemize}
\item \href{20250103144944-chiusura_topologica.org}{Caratterizzazione dei chiusi in termini di chiusura}
\item \href{20250303121451-caratterizzazione_della_chiusura_in_termini_di_successioni.org}{Caratterizzazione della chiusura in termini di successioni}
\end{itemize}
\end{document}
