% Intended LaTeX compiler: pdflatex
\documentclass[../main]{subfiles}

\def\mult#1{\mathrm{mult}_{#1}\,}


\begin{document}

\section{Formula di Hurwitz (Superfici di Riemann)}
\label{sec:org5df4cb0}
Siano \(X,Y\) \href{20260127112828-superficie_di_riemann.org}{superfici di Riemann} \href{20250103163701-spazio_topologico_compatto.org}{compatte}, e sia \(F:X\to Y\) \href{20260128143822-funzione_olomorfa_su_una_superficie_di_riemann.org}{olomorfismo} non costante.

\begin{thm}
Le \href{20260129171755-caratteristica_di_eulero_poincare.org}{caratteristiche di Eulero di \(X\) e \(Y\)} sono legate da questa formula:\footnote{Con \(\mult{p} F\) si indica la \href{20260129104215-forma_normale_locale_per_superfici_di_riemann.org}{molteplicità}.}
\begin{equation*}
\chi(X) = (\deg F) \cdot \chi(Y) - \parentesi{\mathrm{K}\coloneqq}{\sum_{p \in X} (\mult{p}F-1)}
\end{equation*}
dove \(\deg F\) è il \href{20260129171444-teorema_del_grado_per_olomorfismi_tra_superfici_di_riemannn.org}{grado di \(F\)}. Indicando invece con \(g(X), g(Y)\) i generi topologici, si ha equivalentemente:
\begin{equation*}
2g(X)-2 = (\deg F)\big(2g(Y)-2\big) + \sum_{p \in X} (\mult{p}F-1).
\end{equation*}
\end{thm}

\begin{proof}


La dimostrazione di articola in due passi. Si indichi con \(d\coloneqq\deg F\).
\begin{enumerate}
\item \uline{Costruzione della triangolazione}

Costruiamo una \href{20260130154758-triangolazione_di_una_superficie_topologica.org}{triangolazione} di \(Y\) avente come vertici i \href{20260129104340-punto_di_ramificazione_per_una_funzione_tra_superfici_di_riemann.org}{punti di diramazione di \(F\) in \(Y\)}: siccome \(Y\) è compatto, sono in numero finito. Sappiamo che
\begin{equation*}
 \chi(Y) = \mathrm{V} - \mathrm{L} + \mathrm{F} = 2-2g(Y).
\end{equation*}

\item \uline{Sollevare la triangolazione}

Portiamo la triangolazione su \(Y\) ad una su \(Y\) tramite \(F\).

\begin{itemize}
\item Vertici: i vertici sono le controimmagini dei vertici in \(Y\);

\item Lati: se \(L \subseteq Y\) è lato aperto, allora \(L \mathrel{\overset{omeo}{\approx}} (0,1)\)\footnote{Ovvero sono \href{20250111142332-omeomorfismo.org}{omeomorfi}.}, e non contiene punti di diramazione. \href{20260130172938-funzione_olomorfa_tra_superfici_di_riemann_compatte_induce_rivestimento_topologico.org}{Allora}
\begin{align*}
\restriction{F}{F^{-1}(L)}: F^{-1}(L) &\longrightarrow L\\
\end{align*}
è un \href{20250113175231-rivestimento.org}{rivestimento topologico} di grado \(d = \deg F\). Pertanto è una \href{20250113175700-unione_disgiunta.org}{unione disgiunta}
 \begin{equation*}
F^{-1}(L) = L_{1} \sqcup L_{2} \sqcup \dots \sqcup L_{d}
 \end{equation*}
dove \(\restriction{F}{L_{i}} : L_{i} \to L\) è un omeomorfismo.

\item Facce: Se \(T \subseteq Y\) è l'interno del triangolo, allora \(T\) non contiene punti di diramazione, e
\begin{align*}
\restriction{F}{F^{-1}(T)}: F^{-1}(T) &\longrightarrow T\\
\end{align*}
è un rivestimento topologico di grado \(d = \deg F\). Pertanto è una \href{20250113175700-unione_disgiunta.org}{unione disgiunta}
 \begin{equation*}
F^{-1}(T) = T_{1} \sqcup T_{2} \sqcup \dots \sqcup T_{d}
 \end{equation*}
dove \(\restriction{F}{T_{i}} : T_{i} \to T\) è un omeomorfismo.
\end{itemize}

\item \uline{Calcolo di \(\chi(X)\)}-

Indicati con \(\mathrm{V}',\mathrm{L}', \mathrm{F}'\) i numeri della triangolazione di \(X\), si ha
\begin{equation*}
 \mathrm{L}' = d\cdot \mathrm{L},\qquad \mathrm{F}'=d\cdot \mathrm{F},\qquad \mathrm{V}' < d\cdot \mathrm{V}.
\end{equation*}
L'uguaglianza per lati e facce segue dal fatto che su tutto \(X\setminus F^{-1}(\set{\text{vertici}})\) la funzione \(F\) sia un rivestimento topologico di grado \(d\).

Se \(q \in Y\) è un vertice, allora la \href{20241213101756-cardinalita.org}{cardinalità} della \href{20250113175231-rivestimento.org}{fibra} è:
\begin{align*}
 \card{F^{-1}(q)} &= %
 \sum_{ p \in F^{-1}(q)} 1 = \sum_{p \in F^{-1}(q)} 1 -d + d \\
 &= \sum_{p \in F^{-1}(q)} 1 - \sum_{p \in F^{-1}(q)} \mathrm{mult}_{p}\, F + d =%
 d - \sum_{p \in F^{-1}(q)} (\mathrm{mult}_{p}\, F - 1)
\end{align*}
applicando il \href{20260129171444-teorema_del_grado_per_olomorfismi_tra_superfici_di_riemannn.org}{Teorema del Grado}. Per costruzione, il numero di vertici è:
\begin{align*}
 \mathrm{V}' &= \sum_{q\text{ vertici}} \card{F^{-1}(q)} = %
 \sum_{q\text{ vertici}} \Bigg[ d - \sum_{p \in F^{-1}(q)} (\mathrm{mult}_{p}\, F - 1) \Bigg] \\[2ex]
 &= d \cdot \mathrm{V} - \sum_{\parbox{3em}{\scriptsize\centering\(p\) vertice in \(X\)}} (\mathrm{mult}_{p}\, F - 1) = d \cdot \mathrm{V} - \parentesi{\mathrm{K}\coloneqq}{\sum_{p \in X} (\mathrm{mult}_{p}\, F - 1)}
\end{align*}
dove l'ultima uguaglianza segue dal fatto che se \(p \in X\) non è \href{20260129104340-punto_di_ramificazione_per_una_funzione_tra_superfici_di_riemann.org}{di ramificazione}, allora la molteplicità \(\mathrm{mult}_{p}\, F = 1\).
\end{enumerate}

Concludendo, si ha che
\begin{align*}
2 g(X) - 2 &= -\chi(X) = - \mathrm{V}' +\mathrm{L}' - \mathrm{F}' \\
&= - d \cdot (\mathrm{V}-\mathrm{L} + \mathrm{F}) + \mathrm{K} = - d \cdot \chi(Y) + \mathrm{K}\\
&= + (\deg F) \cdot \big(2g(Y)-2\big) + \mathrm{K}.
\qedhere
\end{align*}
\end{proof}
\end{document}
