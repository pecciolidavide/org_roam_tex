% Intended LaTeX compiler: pdflatex
\documentclass[../main]{subfiles}


\begin{document}

\def\U{\mathcal{U}}
\def\L{\mathcal{L}}
%
\def\eq{{\rm eq}}
\def\Ueq{\U^\eq}
%
\def\orbita{\mathcal{O}}
\def\Aut{\operatorname{Aut}}
%
\def\tc{\mid}
\def\tp{\operatorname{tp}}
\def\EMtp{\operatorname{EM}\text{-}\operatorname{tp}}
\def\<{\langle}
\def\>{\rangle}
%
\def\restricted#1{\,\mathord{\upharpoonright}{{\scriptstyle #1}}}
\def\equivalentover#1{\mathrel{\equiv_{ #1 }}}

%% NON FORKING
\def\nonforkSymbol{%
\mathbin{\raise1.8ex%
\rlap{\kern0.6ex\rule{0.6ex}{0.1ex}}%
\rlap{\kern1.1ex\rule{0.1ex}{1.9ex}}\raise-0.3ex\hbox{$\smile$}}}
\def\defaultnonforkmodel{M}
\def\nonfork{\nonforkSymbol}
\renewcommand{\nonfork}[1][\defaultnonforkmodel]{%
\mathrel{\nonforkSymbol_{#1}}}

\def\c{\overline{c}}
\def\b{\overline{b}}
\def\x{\overline{x}}
\section{Estensione di sequenza di indiscernibili su un insieme ad un Modello}
\label{sec:org227f8e0}
Si utilizza la \href{20250612143636-notazione_teoria_dei_modelli.org}{Notazione della TEORIA DEI MODELLI}

Sia \(\L\) un \href{20250130162057-linguaggio_del_prim_ordine.org}{linguaggio}, \(T\) una \href{20250130114950-teoria_del_prim_ordine.org}{teoria} \href{20250131123151-teoria_completa.org}{completa} senza \href{20250131122945-modello_di_un_insieme_di_formule.org}{modelli} finiti e \(\U\) un \href{20250617095548-modello_lambda_saturo.org}{modello saturo} di \href{20241213101756-cardinalita.org}{cardinalità} \href{20250211123155-cardinale_limite_forte.org}{inaccessibile} \(\kappa>\card{\L}+ \omega\). \(\U\) è un \href{20250617102733-modello_mostro.org}{modello mostro}.

Sia \(A \subseteq \U\) un insieme.

\begin{prop}
Sia \(\c = \langle c_{i} : i <\omega \rangle\) \href{20250206170922-sequenze_e_stringhe.org}{sequenza} \href{20251026210908-sequenza_di_indiscernibili.org}{di indiscernibili su \(A\)}. Allora esiste un \href{20250131103035-struttura_del_prim_ordine.org}{modello} \(M\supseteq A\) tale che \(\c\) è una sequenza di indiscernibili su \(M\).
\end{prop}
\begin{proof}
Sia \(M\) un modello arbitrario.
Consideriamo  il \href{20251107125412-tipo_di_ehrenfeucht_mostowski.org}{Tipo di Ehrenfeucht-Mostowski} \(\EMtp(\c/A)\) \href{20251029152405-tipo_completo.org}{completo}, e definiamo
\begin{equation*}
p(\x)\coloneqq \EMtp(\c / M)\supseteq \EMtp(\c/A)
\end{equation*}
Per il \href{20251107125444-teorema_di_ehrenfeucht_mostowski.org}{Teorema di Ehrenfeucht-Mostowski} esiste \(\overline{b}\vDash p(\overline{x})\) sequenza di indiscernibili su \(M\).
Siccome \(\overline{b}\vDash \EMtp(\overline{c}/A)\) allora
\begin{equation*}
\b \equivalentover{A} \c
\end{equation*}
e \href{20250618103257-gruppo_degli_automorfismi_che_fissano_un_sottoinsieme_e_tipo_di_un_elemento.org}{pertanto} esiste \(f \in \Aut(\U/A)\) tale che \(f(\b) = \c\), ovvero \(\c\) è indiscernibile su \(f[M] \supseteq A\).
\end{proof}
\end{document}
