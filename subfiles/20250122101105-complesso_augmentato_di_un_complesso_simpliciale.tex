% Intended LaTeX compiler: pdflatex
\documentclass[../main]{subfiles}


\begin{document}

Sia \(R\) un \href{20241219112842-pid.org}{PID} e sia \(K\) un \href{20250121124147-complesso_simpliciale.org}{complesso simpliciale} \href{20250121131005-complesso_simpliciale_totalmente_ordinato.org}{totalmente ordinato}. Sia
\begin{equation*}
\mathcal{C}_{\bullet}(K) \coloneqq \set{\left(C_{q}(K),\partial_{q}\right)}_{q}
\end{equation*}
il \href{20250121160600-complesso_di_catene_simpliciali.org}{complesso delle catene simpliciali} di \(K\).\footnote{Si hanno:
\begin{itemize}
\item \(C_{q}(K)\) è il \href{20250121124410-modulo_delle_catene_simpliciali.org}{Modulo delle catene simpliciali}
\item \(\partial_{q}\) la \href{20250121132856-mappe_di_bordo_tra_moduli_di_catene_simpliciali.org}{mappa di bordo}
\end{itemize}}
\begin{definizione}
Si definisce \textbf{complesso augmentato di \(K\)} il \href{20250120163114-complesso_di_catene.org}{complesso di catene}
\begin{equation*}
\tilde{\mathcal{C}}_{\bullet}(K) \coloneqq\set{\left(\tilde{C}_{q}(K), \tilde{\partial}_{q}\right)}_{q}
\end{equation*}
dove si ha:
\begin{itemize}
\item per \(q\ge 0\): \(\tilde{C}_{q}(K) \coloneqq C_{q}(K)\), e
\begin{equation*}
  \tilde{C}_{-1}(K) \coloneqq  R
\end{equation*}
\item per \(q\ge 1\): \(\tilde{\partial}_{q} \coloneqq \partial_{q}\), e
\begin{align*}
\tilde{\partial}_{0} : C_{0}(K) &\longrightarrow R\\
\sum a_{i} [p_{i}]&\longmapsto \sum a_{i}.
\end{align*}
\end{itemize}
come illustrato in Figura~\ref{Fig:augm_complex}.
\end{definizione}

\begin{figure}
\begin{equation*}
\begin{tikzcd}[row sep=tiny]
	\cdots & {C_q(K)} & \cdots & {C_1(K)} & {C_0(K)} & \textcolor{rgb,255:red,214;green,92;blue,92}{{\tilde{C}_{-1}(K) \coloneqq R}} \\
	&&&& \textcolor{rgb,255:red,214;green,92;blue,92}{{\sum a_i[p_i]}} & \textcolor{rgb,255:red,214;green,92;blue,92}{{\sum a_i}}
	\arrow[from=1-1, to=1-2]
	\arrow[from=1-2, to=1-3]
	\arrow[from=1-3, to=1-4]
	\arrow["{\partial_1}", from=1-4, to=1-5]
	\arrow["{\tilde{\partial}_0}", color={rgb,255:red,214;green,92;blue,92}, from=1-5, to=1-6]
	\arrow[color={rgb,255:red,214;green,92;blue,92}, maps to, from=2-5, to=2-6]
\end{tikzcd}
\end{equation*}
\caption{\label{Fig:augm_complex}Il Complesso Augmentato di \(K\).}
\end{figure}
\begin{prop}
\(\tilde{\mathcal{C}}_{\bullet}(K)\) è un \href{20250120163114-complesso_di_catene.org}{complesso di catene}.
\end{prop}
\begin{proof}
È sufficiente dimostrare che \(\tilde{\partial}_{0}\circ\partial_{1}=0\), e lo si mostra solo sugli elementi della base. Sia \([p,q] \in K^{1}\)\footnote{\(K^{1}\) è l'\href{20250121124310-scheletro_di_un_complesso_simpliciale.org}{1-scheletro di \(K\)}.}:
\begin{equation*}
\tilde{\partial}_{0}\circ\partial_{1}\left([p,q]\right) = \tilde{\partial}_{0}\left([q]-[p]\right) = 1-1 = 0 %
\qedhere
\end{equation*}
\end{proof}
\end{document}
