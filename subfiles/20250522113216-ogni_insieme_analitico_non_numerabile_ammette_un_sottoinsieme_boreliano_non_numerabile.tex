% Intended LaTeX compiler: pdflatex
\documentclass[../main]{subfiles}


\begin{document}

Inoltre, ogni insieme analitico \(A\) non numerabile ammette un sottoinsieme boreliano \(B\) non numerabile, in quanto:
\begin{itemize}
\item siccome \(A\) è analitico, allora \(A\) ha la PSP (per il Teorema 3.4.1);
\item siccome \(A\) è non numerabile, allora esiste
\begin{equation*}
  \iota: 2^{\omega}\to A
\end{equation*}
una immersione topologica, ovvero \(\iota\) continua e iniettiva;
\item pertanto, per il Corollario 3.2.7, \(B\coloneqq\iota(2^{\omega}) \subseteq T\) è boreliano (poiché \(2^{\omega} \in \bm{{\operatorname{Bor}}}(2^{\omega})\)) ed è ovviamente non numerabile, poiché ha cardinalità \(2^{\omega}>\omega\).

\href{20250515141706-da_finire.org}{DA FINIRE}
\end{itemize}
\end{document}
