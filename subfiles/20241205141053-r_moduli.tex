% Intended LaTeX compiler: pdflatex
\documentclass[../main]{subfiles}


\begin{document}

\section{R-Moduli}
\label{sec:org63d4e93}
\begin{definizione}
Sia \(R\) un \href{20241205141119-anello.org}{anello} commutativo con unità. Un \(R\)-modulo è un \href{20250127093245-gruppo_abeliano.org}{gruppo abeliano} \((M,+)\) con un'azione di \(R\) su \(M\):
\begin{align*}
\cdot: R\times M &\longrightarrow M\\
(r,m)&\longmapsto rm
\end{align*}
tale che, se \(\1\) è l'identità di \(R\)
\begin{align*}
\forall\,r,s \in R,\ m &\in M & r(sm)&=(rs)m\\
\forall\,r,s \in R,\ m &\in M & (r+s)m&=rm+sm\\
\forall\,r \in R,\ m_1,m_2 &\in M & r(m_1+m_2)&=rm_1+rm_{2}\\
\forall\,m &\in M & \1m&=m.
\end{align*}
\end{definizione}
\begin{esempio}
\begin{enumerate}
\item Se \(R\) è un \href{20241205142049-campo.org}{campo}, allora gli \(R\) moduli sono tutti e soli gli \href{20241205142027-spazio_vettoriale.org}{spazi vettoriali} su \(R\).
\item Se \(R= \Z\) allora gli \(\Z\)-moduli sono tutti e soli i \href{20241205141146-gruppo_abeliano.org}{gruppi abeliani}.
\end{enumerate}
\end{esempio}
\uline{Notazione}.
Sia \(M\) un \(R\)-modulo e sia \(e \in M\). Il modulo \textbf{generato} da \(e\) si denota con
\begin{equation*}
R\cdot e \coloneqq  \set{re: r\in R}.
\end{equation*}
\end{document}
