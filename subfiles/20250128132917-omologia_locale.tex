% Intended LaTeX compiler: pdflatex
\documentclass[../main]{subfiles}


\begin{document}

\section{Omologia Locale}
\label{sec:org5a49b8e}
Sia \(R\) un \href{20241219112842-pid.org}{PID} e sia \(X\) uno \href{20250103145124-topologia.org}{spazio topologico}.
\begin{definizione}
Sia \(x \in X\). Si definisce l'\textbf{omologia locale} di \(X\) attorno ad \(x\) come l'\href{20250122154903-omologia_singolare_relativa.org}{omologia singolare relativa} della \href{20250122154728-coppia_topologica.org}{coppia topologica} \((X,X\setminus\set{x})\):
\begin{equation*}
H_{n}(X \mid x) \coloneqq H_{n}(X,\ X\setminus\set{x})
\end{equation*}
\end{definizione}
\begin{prop}
Per ogni intorno chiuso \(U\) di \(x\) si ha che
\begin{equation*}
H_{n}(X \mid x)\cong H_{n}(U \mid x)
\end{equation*}
\end{prop}
\begin{proof}
Banale applicazione del \href{20250126223310-teorema_di_escissione.org}{Teorema di Escissione}:
\begin{align*}
H_{n}(X \mid x) &= H_{n}(X,X\setminus\set{x})\cong H_{n}\bigg(X\setminus\big(X\setminus U\big);\  \big(X\setminus\set{x}\big)\setminus\big(X\setminus U\big)\bigg)\\
& = H_{n}(U;\ U\setminus \set{x}) = H_{n}(U \mid x).%
\qedhere
\end{align*}
\end{proof}
\end{document}
