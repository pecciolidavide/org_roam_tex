% Intended LaTeX compiler: pdflatex
\documentclass[../main]{subfiles}

\usepackage[hyperref]{biblatex}
\date{}
\title{}
\begin{document}

\section{Lemma del numero di Lebesgue}
\label{sec:org63cb10c}
\begin{lem}
Siano \((X,d)\) uno \href{20250301193511-spazio_metrico.org}{spazio metrico} \href{20250103163701-spazio_topologico_compatto.org}{compatto} e \(\mathcal{A}\) un \href{20250103164252-ricoprimento.org}{ricoprimento} \href{20250103145124-topologia.org}{aperto} di \(X\).

Allora esiste \(\delta\) tale che per ogni \href{20250103145124-topologia.org}{aperto} \(B\) in \(X\) di \href{20250327131547-diametro_di_un_insieme.org}{diametro} minore di \(\delta\), esiste \(U \in \mathcal{A}\) tale che $\backslash$(B \subseteq U.
\end{lem}

\begin{definizione}
\(\delta\) si chiama \uline{numero di Lebesgue} del ricoprimento \(\mathcal{A}\).
\end{definizione}
\end{document}
