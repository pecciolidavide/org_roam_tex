% Intended LaTeX compiler: pdflatex
\documentclass[../main]{subfiles}


\begin{document}

\newcommand{\U}{\mathcal{U}} % Modello mostro
\newcommand{\acl}{\operatorname{acl}}
\newcommand{\dcl}{\operatorname{dcl}}
\newcommand{\tp}{\operatorname{tp}}
\newcommand{\eq}{\text{eq}}
\newcommand{\Aut}{\operatorname{Aut}}
\newcommand{\equivalentover}[1]{\mathrel{\equiv_{#1}}}
\newcommand{\Shequivalentover}[1]{\mathrel{\overset{\text{Sh}}{\equiv}_{#1}}}
\newcommand{\orbita}{\mathcal{O}}

\begin{prop}
Prove that following are equivalent for every \(A\subseteq\mathcal{U}\)
\begin{enumerate}
\item \(\operatorname{acl}^{\text{eq}} A=\operatorname{dcl}^{\text{eq}}\big(\operatorname{acl} A\big)\)
\item \(\operatorname{Aut}(\mathcal{U}/\operatorname{acl}^{\text{eq}} A)= \operatorname{Aut}(\mathcal{U}/\operatorname{acl} A)\)
\item \(c\mathrel{\equiv_{\operatorname{acl}A}}b\oldiff c\mathrel{\overset{\text{Sh}}{\equiv}_{A}} b\) for all \(c,b \in \U^{\lambda}\).
\end{enumerate}
\end{prop}
\begin{proof}
(\(1.\Rightarrow 2.\)):
Siccome in generale \(\acl A \subseteq \acl^{\eq} A\) allora
\begin{equation*}
\Aut(\U/\acl^\eq A) \subseteq \Aut(\U/\acl A).
\end{equation*}

Viceversa, sia \(f \in \Aut(\U/\acl A)\) e sia \(\mathcal{A} \in \acl^{\eq} A\).
Per \(1.\) allora \(\mathcal{A} \in \dcl^{\eq}\big(\acl A\big)\) e per la caratterizzazione di \(\dcl^{\eq}\) allora esiste \(\psi(x) \in \mathcal{L}(\acl A)\) tale che \(\mathcal{A}=\psi(\U^{x})\).

In particolare esiste \(\psi(x;z) \in \mathcal{L}\), con \(\card{z} = \card{\acl A}\) tale che, detta \(a\) una enumerazione di \(\acl A\):
\begin{equation*}
\psi(x) = \psi(x;a)\quad\oldimplies\quad \mathcal{A}=\psi(\U^{x};a).
\end{equation*}
Pertanto:
\begin{equation*}
f\mathcal{A}=f[\psi(\U^{x};a)] = \psi(\U^{x};fa) = \psi(\U^{x};a) = \mathcal{A}.
\end{equation*}

Per arbitrarietà di \(\mathcal{A}\) si ha \(f\mathcal{A}=\mathcal{A}\) per ogni \(\mathcal{A} \in \acl^{\eq} A\): \(f \in \Aut(\U/\acl^{\eq} A)\).

(\(2.\Rightarrow 3.\)): È noto che per ogni \(c,b \in \U^{\lambda}\)
\begin{equation*}
c \Shequivalentover{A} b\quad\oldiff\quad c\equivalentover{\acl^\eq A} b.
\end{equation*}
È sufficiente quindi dimostrare che 2. implichi
\begin{equation*}
c\equivalentover{\acl A}b
\quad\oldiff\quad
c\equivalentover{\acl^\eq A} b
\end{equation*}
per ogni \(b,c \in\U^{\lambda}\) per ottenere la tesi.

Ma \(c\equivalentover{\acl A}b\) se e solo se\footnote{Il verso \(c\equivalentover{\acl A}b\ \oldimplies\ \orbita (c/\acl A) = \orbita(b/\acl A)\)
è ovvio, dal momento che \(\orbita(b/\acl A) = p(\U^{x})\), dove \(p(x) = \tp(b/\acl A)\).
Viceversa WLOG si supponga per assurdo che vi sia \(\varphi(x) \in \tp(b/\acl A)\) tale che \(\varphi(x) \notin \tp(c/\acl A)\). Allora non vale \(\varphi(c)\) e quindi \(c\notin \orbita(b/\acl A)\). Assurdo poiché \(c\in \orbita(c/\acl A)\).} \(\orbita (c/\acl A) = \orbita(b/\acl A)\).

Per \(2.\) questo è vero se e solo se \(\orbita(c/\acl^{\eq}A) = \orbita(b/\acl^{\eq}A)\) e, come sopra, questo avviene se e solo se \(c\equivalentover{\acl^{\eq} A} b\).

(\(3.\Rightarrow 1.\)):
Vale sempre \(\dcl^{\eq} (\acl A) \subseteq \acl^{\eq} A\). Infatti, sia \(\mathcal{B} \in \dcl^{\eq}(\acl A)\). Allora esiste \(\psi(x;z) \in \mathcal{L}\), \(\card{z}=n\), ed esistono \(a_{1},\dots,a_{n} \in \acl A\) tali che
\begin{equation*}
\mathcal{B} = \psi(\U^{x};a_{1},\dots,a_{n}).
\end{equation*}
Inoltre esistono \(\varphi_{1},\dots,\varphi_{n} \in \mathcal{L}(A)\) tali che \(\varphi_{i}(a_{i})\) e che testimoniano \(a_{i} \in \acl A\). Quindi la formula
\begin{equation*}
\Phi(\mathcal{X}):\quad
\exists z_{1},\dots,z_{n}\
\big(\mathcal{X} = \psi(\U^{x}; z_{1},\dots,z_{n}) \land \bigwedge_{i=1}^{n} \varphi_{i}(z_{i})\big)
\end{equation*}
ha parametri unicamente in \(A\) ed è realizzata da un numero finito di \(\mathcal{X}\) (poiché ciascuna \(\varphi_{i}\) ha un numero finito di realizzazioni), e inoltre \(\Phi(\mathcal{B})\). Pertanto \(\Phi\) testimonia \(\mathcal{B} \in \acl^{\eq} A\).

Resta da dimostrare che \(3.\) implica \(\acl^{\eq} A \subseteq \dcl^{\eq}(\acl A)\).
Sia quindi \(a\in \acl^{\eq} A\).

\begin{itemize}
\item Si noti che
\(\orbita(a/\acl^{\eq}A) = \set{a}\)
poiché \(a \in \acl^{\eq}A\).
\item Inoltre,
\begin{align*}
  \orbita(a/\acl A) &= \set{b \in (\U^{\eq})^{\card{a}}\mid a\equivalentover{\acl A} b}\\
  \orbita(a/\acl^{\eq} A) &= \set{b \in (\U^{\eq})^{\card{a}}\mid a\equivalentover{\acl^{\eq} A} b}.
\end{align*}
\item L'ipotesi \(3.\) implica che
\begin{equation*}
a\equivalentover{\acl A}b
\quad\oldiff\quad
a\equivalentover{\acl^\eq A} b
\end{equation*}
per ogni \(c,b\).
\end{itemize}

Utilizzando i tre fatti di cui sopra, si ottiene che
\begin{equation*}
\orbita(a/\acl A) = \orbita(a/\acl^{\eq}A) = \set{a}
\end{equation*}
ovvero \(a\) invariante su \(\acl A\). Per la caratterizzazione \(a \in \dcl^{\eq}(\acl A)\).
\end{proof}
\end{document}
