% Intended LaTeX compiler: pdflatex
\documentclass[../main]{subfiles}


\begin{document}

\section{Categoria}
\label{sec:org18e0ab2}
\begin{definizione}
Una \uline{categoria} \(\mathcal{C}\) è composta dai seguenti oggetti:
\begin{enumerate}
\item una \href{20250130104320-classe_mk.org}{classe} \(\operatorname{Ob}(\mathcal{C})\) i cui elementi sono detti \uline{oggetti} di \(\mathcal{C}\);
\item per ogni \href{20250131162451-coppia_ordinata_mk.org}{coppia} di oggetti \((X,Y)\) in \(\operatorname{Ob}(\mathcal{C})\): un \href{20250130104331-insieme_mk.org}{insieme} \(\operatorname{Hom}_{\mathcal{C}}(X,Y)\) dei \uline{morfismi} (o delle frecce);
\item per ogni \href{20250206170922-sequenze_e_stringhe.org}{tripla} di oggetti \((X,Y,Z)\) in \(\operatorname{Ob}(\mathcal{C})\): una mappa
\begin{align*}
   \operatorname{Hom}_{\mathcal{C}}(X,Y) \times \operatorname{Hom}_{\mathcal{C}}(Y,Z) &\longrightarrow \operatorname{Hom}_{\mathcal{C}}(X,Z)\\
   (f,g) &\mapsto g\circ f;
\end{align*}
\end{enumerate}

soggetti alle seguenti proprietà:
\begin{enumerate}
\item se \(X,Y,X',Y' \in \operatorname{Ob}(\mathcal{C})\) e \((X,Y)\neq (X',Y')\), allora
\begin{equation*}
 \operatorname{Hom}_{\mathcal{C}}(X,Y)\cap \operatorname{Hom}_{\mathcal{C}}(X',Y') = \emptyset
\end{equation*}

\item per ogni quadrupla \((X,Y,Z,W)\) di oggetti in \(\operatorname{Ob}(\mathcal{C})\) e per ogni \(f \in \operatorname{Hom}_{\mathcal{C}}(X,Y)\),  \(g \in \operatorname{Hom}_{\mathcal{C}}(Y,Z)\),  \(h \in \operatorname{Hom}_{\mathcal{C}}(Z,W)\) valga
\begin{equation*}
   h\circ (g\circ f) = (h\circ g)\circ f;
\end{equation*}
\item per ogni oggetto \(X\) in \(\operatorname{Ob}(\mathcal{C})\) esiste \(\1_X \in \operatorname{Hom}_{\mathcal{C}} (X,X)\) tale che
\begin{enumerate}
\item per ogni \(Y\)  in \(\operatorname{Ob}(\mathcal{C})\) e per ogni  \(g \in \operatorname{Hom}_{\mathcal{C}}(X,Y)\)
\begin{equation*}
  g \circ \1_X = g
\end{equation*}
\item per ogni \(Y\)  in \(\operatorname{Ob}(\mathcal{C})\) e per ogni  \(h \in \operatorname{Hom}_{\mathcal{C}}(Y,X)\)
\begin{equation*}
  \1_X\circ h = h
\end{equation*}
\end{enumerate}
\end{enumerate}
\end{definizione}

\uline{Notazione}:
In luogo di \(f\in \operatorname{Hom}_{\mathcal{C}}(X,Y)\), quando la categoria è ben specificata, spesso si scriverà
\begin{equation*}
X \dfreccia{f} Y;\qquad f: X\longrightarrow Y
\end{equation*}
\begin{prop}
Sia \(\mathcal{C}\) una categoria. Per ogni \(X \in \operatorname{Ob}(\mathcal{C})\) esiste un unica \(\1_X\)
\end{prop}

\begin{proof}
Sia \(\1_X'\) un'altra unità. Allora
\begin{equation*}
    \1_X = \1_X\circ\1_X' = \1_X'\qedhere
\end{equation*}
\end{proof}


Vedi degli \href{20241205113007-categorie_esempi.org}{esempi di categorie}.
\end{document}
