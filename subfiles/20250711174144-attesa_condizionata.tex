% Intended LaTeX compiler: pdflatex
\documentclass[../main]{subfiles}


\begin{document}

\section{Attesa condizionata}
\label{sec:org2471a03}
\begin{definizione}
Sia \((\Omega, \mathcal{F}, \mathds{P})\) uno \href{20250711175559-spazio_di_probabilita.org}{spazio di probabilità}, e sia \(X \in L^{2}(\Omega, \mathcal{F}, \mathds{P})\)\footnote{Vedi ``\href{20250624162220-spazi_lp.org}{Spazi Lp}''}. Sia \(\mathcal{G}\subseteq \mathcal{F}\) una sotto-\href{20250526100313-sigma_algebra.org}{\(\sigma\)-algebra}.

Si definisce l'\uline{attesa condizionata} di \(X\) dato \(\mathcal{G}\)
\begin{equation*}
Z = \media [X\mid\mathcal{G}]
\end{equation*}
\begin{enumerate}
\item \(Z\) è \(\mathcal{G}\)-\href{20250704104947-funzione_misurabile.org}{misurabile};
\item per ogni \(\Lambda \in \mathcal{G}\)
\begin{equation*}
 \int_{\Lambda} Z\dif\mathds{P} = \int_{\Lambda} X\dif\mathds{P}
\end{equation*}
o, equivalentemente, detta \(\mathds{1}_{\Lambda}\) la \href{20250215160218-funzione_caratteristica.org}{funzione caratteristica} di \(\Lambda\)
\begin{equation*}
 \int \mathds{1}_{\Lambda} Z\dif\mathds{P} = \int \mathds{1}_{\Lambda} X\dif\mathds{P}.
\end{equation*}
\end{enumerate}
\end{definizione}

Se \(Y\) è una \href{20250711175937-variabile_aleatoria.org}{variabile aleatoria}, detta \(\sigma(Y) \coloneqq \set{Y^{-1}(B)\mid B \in \bm{{\operatorname{Bor}}}(\R)}\)\footnote{Si indica con \(\bm{{\operatorname{Bor}}}(\R)\) l'insieme dei \href{20250419121450-gerarchia_di_borel.org}{Boreliani} di \(\R\).}, si denota con
\begin{equation*}
\media[X\mid Y] \coloneqq \media[X\mid\sigma(Y)].
\end{equation*}
\section{Probabilità condizionata}
\label{sec:org2d1aa55}
Se \(A,B \in  \mathcal{F}\), con \(\mathds{P}(B)\neq \emptyset\), la \uline{probabilità condizionata} è
\begin{equation*}
\mathds{P}(A\mid B)\coloneqq\frac{\mathds{P}(A\cap B)}{\mathds{P}(B)}.
\end{equation*}

Si generalizza questo concetto, e si definisce la \uline{probabilità condizionata}, per \(A \in  \mathcal{F}\) e \(Y\) \href{20250711175937-variabile_aleatoria.org}{variabile aleatoria}
\begin{equation*}
\mathds{P}(A\mid Y) \coloneqq \media[\mathds{1}_{A}\mid Y].
\end{equation*}
\begin{prop}
Sia \((\Omega, \mathcal{F}, \mathds{P})\) uno \href{20250711175559-spazio_di_probabilita.org}{spazio di probabilità}. Se \(X\) è una \href{20250711175937-variabile_aleatoria.org}{variabile aleatoria discreta} e \(A \in  \mathcal{F}\) allora
\begin{equation*}
\mathds{P}(A\mid X) = \sum_{n=1}^{+\infty} \mathds{P}(A\mid X=x_{n})\cdot\mathds{1}_{(X=x_{n})}.
\end{equation*}
\end{prop}
\begin{oss}
Si noti quindi che \(\mathds{P}(A\mid X) = f(X)\), poiché
\begin{equation*}
\mathds{P}(A\mid X) = \begin{cases}
\mathds{P}(A\mid X=x_{1}) &\text{se }X=x_{1}\\
\mathds{P}(A\mid X=x_{2}) &\text{se }X=x_{2}\\
\mathds{P}(A\mid X=x_{3}) &\text{se }X=x_{3}\\
\vdots
\end{cases}
\end{equation*}
ed inoltre \(\mathds{P}(A\mid X = x) = f(x)\).
\end{oss}
\end{document}
