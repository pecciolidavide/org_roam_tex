% Intended LaTeX compiler: pdflatex
\documentclass[../main]{subfiles}


\begin{document}

Contesto: \href{20250130104245-morse_kelly_set_theory.org}{Morse Kelly Set Theory} (\uline{senza AC})
\section{Teorema di Cantor}
\label{sec:orge2586fe}

Se \(X\) è un \href{20250130104331-insieme_mk.org}{insieme}, allora non esiste alcuna \href{20241213105600-funzione_suriettiva.org}{mappa suriettiva}:\footnote{Vedi ``\href{20250130104245-morse_kelly_set_theory.org}{Insieme delle parti per MK}''}
\begin{equation*}
X\surjects \parti{X}
\end{equation*}
\section[Corollario]{Corollario\footnote{Vedi:
\begin{itemize}
\item \href{20250612151505-aritmetica_dei_cardinali.org}{Elevamento a potenza di cardinali}
\end{itemize}}}
\label{sec:org329e925}
Se vale \href{20250206171508-axiom_of_choiche.org}{AC}, allora:

Sia \(I\) un \href{20250130104331-insieme_mk.org}{insieme}. Allora
\begin{equation*}
\card{I}<2^{\card{I}}
\end{equation*}
dove \(\card{I}\) è la \href{20241213101756-cardinalita.org}{cardinalità} di \(I\).
\subsection{Dimostrazione}
\label{sec:org9cd85f7}
Si noti che\footnote{Vedi: \href{20250211145622-proprieta_di_prodotto_e_somma_generalizzata_di_cardinali.org}{Proprietà di prodotto e somma generalizzata di cardinali}}
\begin{align*}
\card{I} &= \sum_{i \in \card{I}} 1\\
2^{\card{I}} &= \prod_{i \in \card{I}} 2
\end{align*}
e \(1<2\), e pertanto, applicando il \href{20250211104106-somma_di_cardinali_e_minore_del_prodotto_di_cardinali.org}{Teorema di Konig}, si ha la tesi.
\end{document}
