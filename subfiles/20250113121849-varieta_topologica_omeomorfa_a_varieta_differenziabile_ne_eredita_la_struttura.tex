% Intended LaTeX compiler: pdflatex
\documentclass[../main]{subfiles}

\usepackage[hyperref]{biblatex}
\date{}
\title{}
\begin{document}

\section{Varietà Topologica omeomorfa a varietà differenziabile ne eredità la struttura}
\label{sec:orgf4ca25d}
\subsection{Proposizione}
\label{sec:org543a0b4}
Sia \(M\) una \href{20250111092123-varieta_topologica.org}{varietà topologica}, sia \(N\) una \href{20250113115909-struttura_differenziabile.org}{varietà differenziabile}, e sia
\begin{equation*}
F: M\longrightarrow N
\end{equation*}
un \href{20250111142332-omeomorfismo.org}{omeomorfismo}.

\(M\) eredita una struttura di varietà differenziabile; se \(\set{(U_{\alpha},\varphi_{\alpha})}\) è un \href{20250113103136-atlante_topologico_differenziabile.org}{atlante differenziabile} di \(N\), allora
\begin{equation*}
\set{\left(F^{-1}(U_{\alpha}), \varphi_{\alpha}\circ F\right)}
\end{equation*}
è un atlante differenziabile su \(M\).
\subsubsection{{\bfseries\sffamily TODO} Dimostrazione\hfill{}\textsc{matematica\_lm:geo\_diff}}
\label{sec:org615a839}
\end{document}
