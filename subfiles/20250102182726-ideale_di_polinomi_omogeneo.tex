% Intended LaTeX compiler: pdflatex
\documentclass[../main]{subfiles}


\begin{document}

Sia (\K) un \href{20241231112713-campo_algebricamente_chiuso.org}{campo algebricamente chiuso}, e sia \(S=\K[x_{0},\dots,x_{n}]\) l'\href{20241219113434-anello_dei_polinomi.org}{anello dei polinomi}. Siano \(S_{d}\) \href{20241231121125-polinomi_omogenei.org}{i sottogruppi omogenei di \(S\)}.
\section{Definizione}
\label{sec:org149e990}
\(I\) \href{20241219112955-ideale.org}{Ideale} di \(S\) si dice \textbf{omogeneo} se
\begin{equation*}
I=\bigoplus_{d\ge 0} (S_{d}\cap I)
\end{equation*}
ovvero se, dato \(F \in I\) e
\begin{equation*}
F=F_{0}+F_{1}+\dots+F_{d}
\end{equation*}
con \(F_{i}\) omogenei, si ha che \(F_{0},F_{1},\dots,F_{d}\in I\).
\section{Proposizione}
\label{sec:orgaf8da98}
Un ideale è omogeneo se e solo se è \href{20241219113154-ideale_generato.org}{generato} da \href{20241231121125-polinomi_omogenei.org}{Polinomi Omogenei}.
\end{document}
