% Intended LaTeX compiler: pdflatex
\documentclass[../main]{subfiles}


\begin{document}

\def\L{\mathcal{L}} % Il linguaggio
\def\U{\mathcal{U}} % Modello mostro
\section{phi-formula su un insieme}
\label{sec:orgdd40782}
Si utilizza la \href{20250612143636-notazione_teoria_dei_modelli.org}{Notazione della TEORIA DEI MODELLI}

Sia \(\L\) un \href{20250130162057-linguaggio_del_prim_ordine.org}{linguaggio}, \(T\) una \href{20250130114950-teoria_del_prim_ordine.org}{teoria} \href{20250131123151-teoria_completa.org}{completa} senza \href{20250131122945-modello_di_un_insieme_di_formule.org}{modelli} finiti e \(\U\) un \href{20250617095548-modello_lambda_saturo.org}{modello saturo} di \href{20241213101756-cardinalita.org}{cardinalità} \href{20250211123155-cardinale_limite_forte.org}{inaccessibile} \(\kappa>\card{\L}+ \omega\). \(\U\) è un \href{20250617102733-modello_mostro.org}{modello mostro}.

Sia \(A \subseteq \U\) piccolo.

\begin{definizione}
Sia \(\varphi(x;z) \in \L\) una \href{20250131103317-formula_del_prim_ordine.org}{formula}. Una \uline{\(\varphi\)-formula} su \(A\) è\footnote{Con questa notazione (\(\set{\land,\lor, \lnot}\! \Delta\)) si indica la \href{20251021120818-chiusura_di_un_insieme_di_formule_rispetto_a_connettivi_logici.org}{chiusura di \(\Delta\)} rispetto a \(\set{\land,\lor, \lnot}\)}
\begin{equation*}
\psi(x) \in \set{\land,\lor, \lnot}\! \Delta,\qquad \Delta = \set{\varphi(x;b)\mid b \in \U^{z}}
\end{equation*}
tale che \(\psi\) sia \href{20251026161128-formula_invariante_su_un_insieme.org}{invariante su \(A\)}.

Con \(\mathcal{L}_{\varphi}(A)\) si intende l'insieme delle \(\varphi\)-formule su \(A\).
\end{definizione}
\subsection{phi-tipo su un insieme}
\label{sec:orgdc1180f}
\begin{definizione}
Un \(\varphi\)-tipo è un insieme di \hyperref[sec:orgdd40782]{\(\varphi\)-formule}.

Un \(\varphi\)-tipo \uline{globale} è un insieme \(\subseteq\)-\href{20250203102516-massimo_e_minimo.org}{massimale} \href{20250212164424-tipo_teoria_dei_modelli.org}{finitamente consistent}e di \(\varphi\)-formule su \(\U\).
\(S_{\varphi}(\U)\) è l'insieme di tutti i \(\varphi\)-tipi globali.
\end{definizione}

\begin{oss}
Se \(p(x) \in S_{\varphi}(\mathcal{U})\) allora a meno di equivalenza
\begin{equation*}
p(x) \subseteq \set{\varphi(x;b),\lnot\varphi(x;b)\mid b \in \mathcal{U}^{z}}.
\end{equation*}

\href{20250515141706-da_finire.org}{DA FINIRE}
\end{oss}
\end{document}
