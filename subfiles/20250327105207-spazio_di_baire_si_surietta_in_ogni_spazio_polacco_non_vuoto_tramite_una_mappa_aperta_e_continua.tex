% Intended LaTeX compiler: pdflatex
\documentclass[../main]{subfiles}


\begin{document}

\section{Proposizione}
\label{sec:orgd876d8a}
Show that for every nonempty \href{20250301194013-spazio_polacco.org}{Polish space} \(X\) there is a \href{20250103103252-funzione_continua.org}{continuous} \href{20250104114559-funzione_chiusa.org}{open} \href{20241213105600-funzione_suriettiva.org}{surjection} \(f: \omega^\omega \rightarrow X\).

{[}Hint. First show that if \(X\) is a \href{20250301193511-spazio_metrico.org}{metric space}, then \href{20250327132216-ogni_aperto_di_uno_spazio_metrico_ammette_un_ricoprimento_numerabile_di_diametro_arbitrariamente_piccolo.org}{for every open \(U\) and every \(\varepsilon \in \R^{+}\) there is a countable covering \(\left(U_n\right)_{n \in \omega}\) of \(U\) such that \(\operatorname{Cl}\left(U_n\right) \subseteq U\) and \(\operatorname{diam}\left(U_n\right)<\varepsilon\), for all \(n \in \omega\)}. Use this to build an appropriate \$\(\omega\)\$-scheme inducing the function \(f\).]
\subsection{Soluzione}
\label{sec:org0e0d0bf}

Sia \(d\) una \href{20250301193511-spazio_metrico.org}{metrica} \href{20250301194153-spazio_metrico_completo.org}{completa} fissata su \(X\).

Si costruisce per induzione su \(\operatorname{lh}(s)\) un \(\omega\)-schema \(\mathcal{S}\) su \((X,d)\). Sia \(B_{\langle\,\rangle} = X\).

Sia ora \(s \in \omega^{<\omega}\) tale per cui \(B_{s}\) è definito. Sia \((U_{n})_{n \in \omega}\) il \href{20250103164252-ricoprimento.org}{ricoprimento} \href{20250103145124-topologia.org}{aperto} \href{20250111143651-insieme_numerabile.org}{numerabile} di \(B_{s}\) che esiste per il \href{20250327132216-ogni_aperto_di_uno_spazio_metrico_ammette_un_ricoprimento_numerabile_di_diametro_arbitrariamente_piccolo.org}{claim}. Questo è tale che:
\begin{itemize}
\item \(\bigcup_{n \in \omega}  U_{n} = B_{s}\)
\item gli \(U_{n}\) sono aperti, per ogni \(n \in \omega\);
\item \(U_{n} \subseteq \operatorname{Cl}(U_{n}) \subseteq B_{s}\);
\item \(\operatorname{diam}(U_{n})\le 2^{-\operatorname{lh}(s)}\);
\item senza perdità di generalità, e possibile supporre che ciascun \(U_{n}\neq \emptyset\) (si sostituisce nel ricoprimento a ciascun \(U_{n}=\emptyset\) il primo \(U_{m}\neq \emptyset\));
\end{itemize}
Si pone quindi, per ogni \(a \in \omega\): \(B_{s\concat a} \coloneqq U_{a}\).

Si è in questo modo definito un \(\omega\)-schema su \(X\) che induce una funzione
\begin{equation*}
f:D_{\mathcal{S}}\to X.
\end{equation*}
Per il lemma 1.3.6, \(f\) è tale che
\begin{enumerate}
\item \(f\) è continua per il punto (a);
\item \(f\) è suriettiva per il punto (d);
\item \(D_{\mathcal{S}} = \omega^{\omega}\) per il punto (e);
\item \(f\) è aperta per il punto (d).
\end{enumerate}
\end{document}
