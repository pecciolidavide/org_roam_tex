% Intended LaTeX compiler: pdflatex
\documentclass[../main]{subfiles}


\begin{document}

\section{Unione}
\label{sec:org5458193}
\subsection{Unione generalizzata}
\label{sec:org76db5cc}
\section{Intersezione}
\label{sec:org426f2da}
\section{Sottrazione insiemistica}
\label{sec:org299d85f}
\section{Differenza simmetrica tra due insiemi}
\label{sec:org898c240}
\section{Sottoinsieme}
\label{sec:org03a0556}
\section{Generalizzazione nell'ambito MK}
\label{sec:orgc2d0a72}

Contesto: \href{20250130104245-morse_kelly_set_theory.org}{Morse Kelly Set Theory}

Siano \(A,B\) due classi. Prr l'assioma di Comprehension, possiamo scrivere
\subsection{Unione di classi MK}
\label{sec:orgede3c82}
\begin{equation*}
A\cup B \coloneqq \set{x\,|\, x \in A \,\land\,  x \in B}
\end{equation*}

Per l'Axiom of Extensionality, è commutativa.
\subsubsection{Classe Unione Generalizzata MK}
\label{sec:orgecf548b}
Sia \(A\) una \href{20250130104320-classe_mk.org}{classe}.
Si definisce la classe unione:
\begin{equation*}
\bigcup A = \bigcup_{x \in A} x \coloneqq \set{y\,|\, \exists\,x (x \in A \,\land\, y \in x)}
\end{equation*}
Questa è una classe per l'Axiom of Comprehension

Se \(x,y\) sono insiemi, allora per l'Axiom of pairing \(\set{x,y}\) è un insieme, e si ha che
\begin{equation*}
x\cup y = \bigcup\set{x,y}
\end{equation*}
\subsection{Intersezione di classi MK}
\label{sec:org8eb3ce4}
\begin{equation*}
A\cap B \coloneqq \set{x\,|\, x \in A \,\lor\, x \in B}
\end{equation*}

Per l'Axiom of Extensionality, è commutativa.
\subsubsection{Classe Intersezione Generalizzata}
\label{sec:orgf6111d0}
Sia \(A\) una \href{20250130104320-classe_mk.org}{classe}.
Si definisce la classe intersezione
\begin{itemize}
\item se \(A\neq \emptyset\) (vedi \href{20250131161811-insieme_vuoto_mk.org}{Insieme vuoto MK})
\end{itemize}
\begin{equation*}
\bigcap A = \bigcap_{x \in A} x \coloneqq \set{y\, |\, \forall\,x\ (x \in A \,\implies y \in x)}
\end{equation*}
tale classe esiste per l'Axiom of Comprehension.
\begin{itemize}
\item se \(A=\emptyset\) allora \(\bigcap A = \emptyset\).
\end{itemize}

La classe intersezione è sempre un \href{20250130104331-insieme_mk.org}{insieme}, \href{20250131160822-ogni_sottoclasse_di_un_insieme_e_un_insieme_mk.org}{poiché} per ogni \(x \in A\) si ha che
\begin{equation*}
\bigcap A \subseteq x
\end{equation*}
e \(x\) è un insieme.
\subsection{Sottrazione di classi MK}
\label{sec:orgd08eb9f}
\begin{equation*}
A\setminus B \coloneqq \set{x\,|\, x \in A \,\land\,  x \notin B}
\end{equation*}

Per l'Axiom of Extensionality, è commutativa.
\subsection{Differenza simmetrica di classi MK}
\label{sec:org1ba90f6}
\begin{equation*}
A\mathrel{\triangle} B\coloneqq (A\setminus B)\cup(B\setminus A)
\end{equation*}

Per l'Axiom of Extensionality, è commutativa.
\subsection{Sottoclasse MK}
\label{sec:orge7be09b}
Scriveremo \(A \subseteq B\) (e diremo che \(A\) è una sottoclasse di \(B\)) se e solo se
\begin{equation*}
\forall\,x \left(x \in A\,\implies\, x \in B\right)
\end{equation*}
Se \(A \neq B\) allora diremo che \(A\) è una sottoclasse propria di \(B\), e scriveremo \(A \subset B\).
\end{document}
