% Intended LaTeX compiler: pdflatex
\documentclass[../main]{subfiles}


\begin{document}

\begin{thm}
\lipsum[1]
\end{thm}
\begin{cor}
Sia \((\mathcal{X}, \norma{\cdot})\) uno \href{20241205142027-spazio_vettoriale.org}{spazio vettoriale} \href{20250625123506-spazio_normato.org}{normato}, e sia \(\mathcal{U} \subseteq \mathcal{X}\) un \href{20250114103118-sottospazio_vettoriale.org}{sottospazio vettoriale} non \href{20250301193045-sottoinsieme_denso.org}{denso}.

Allora esiste un funzionale \href{20250114101949-funzione_lineare.org}{lineare} \href{20250704145518-funzione_limitata.org}{limitato}
\begin{equation*}
L:\mathcal{X}\to \R
\end{equation*}
tale che \(L\neq 0\) e la \href{20250205170515-restrizione_di_una_classe.org}{restrizione} \(\restriction{L}{\mathcal{U}} = 0\).
\label{lem:9.3.2}
\end{cor}
\end{document}
