% Intended LaTeX compiler: pdflatex
\documentclass[../main]{subfiles}


\begin{document}

\section{Teorema}
\label{sec:org852105c}
Sia \(X\) uno \href{20250103145124-topologia.org}{Spazio topologico} \href{20250103152646-spazio_topologico_noetheriano.org}{noetheriano}. Allora ogni \href{20250103163814-sottospazio_topologico.org}{sottospazio} \(Y \subseteq X\) è \href{20250103163701-spazio_topologico_compatto.org}{compatto}.
\subsection{Dimostrazione}
\label{sec:org2c75a61}
Sia \(\mathcal{U}=\set{U_{\alpha}}_{{\alpha \in A}}\) un \href{20250103164252-ricoprimento.org}{ricoprimento} \href{20250103145124-topologia.org}{aperto} di \(Y\). Sia \(\mathscr{F}\):
\begin{equation*}
\mathscr{F}\coloneqq\set{U_{\alpha_{1}}\cup\dots\cup U_{\alpha_{n}}: \alpha_{i} \in A, n \in \N}
\end{equation*}
la famiglia di unioni finite di elementi di \(\mathcal{U}\). Siccome \(X\) è noetheriano, per il \href{20250103155209-caratterizzazione_topologia_noetheriana.org}{teorema di caratterizzazione}, \(\mathscr{F}\) ha un elemento massimale \(U\),
\begin{equation*}
U = U_{\beta_{1}}\cup \dots\cup U_{\beta_{k}}
\end{equation*}

Si ha che \(\set{U_{\beta_{i}}}_{i=1,\dots,k}\) è un ricorprimento aperto finito di \(Y\), e pertanto \(Y\) è compatto. Infatti \(Y \subseteq U\).

Se per assurdo \(Y \not\subseteq U = U_{\beta_{1}}\cup \dots\cup U_{\beta_{k}}\), sia \(p \in Y\setminus U\). Allora esiste \(U_{\gamma} \in \mathscr{F}\) tale che \(p \in U_{\gamma}\). Inoltre
\begin{equation*}
U\cup U_{\gamma} \subsetneqq U
\end{equation*}
e pertanto \(U\) non è massimale. Assurdo.
\subsection{Corollario}
\label{sec:org496601a}
Se \(X\) è uno \href{20250103145124-topologia.org}{spazio topologico} \href{20250103152646-spazio_topologico_noetheriano.org}{noetheriano}, allora \(X\) è \href{20250103163701-spazio_topologico_compatto.org}{compatto}.
\end{document}
