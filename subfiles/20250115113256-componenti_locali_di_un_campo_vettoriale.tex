% Intended LaTeX compiler: pdflatex
\documentclass[../main]{subfiles}

\usepackage[hyperref]{biblatex}
\date{}
\title{}
\begin{document}

\section{Componenti locali di un campo vettoriale}
\label{sec:org6b8ab7c}
Sia \(M\) una \href{20250113115909-struttura_differenziabile.org}{varietà differenziabile} di dimensione \(n\), e sia \(X \in \Gamma(M)\) (vedi \href{20250115110259-campo_vettoriale_su_una_varieta_differenziabile.org}{Campo vettoriale su una varietà differenziabile}); sia \((U,\varphi)\) \href{20250111092123-varieta_topologica.org}{carta} di \(M\) con \(\varphi=(x^{1},\dots,x^{n})\), \(p \in U\).

Dunque \(X_{p} \in \operatorname{T}_{p}M\) (vedi \href{20250114102823-spazio_tangente_ad_un_punto_di_una_varieta_differenziabile.org}{Spazio tangente ad un punto di una varietà differenziabile}), e
\begin{equation*}
\set{\restriction{\dpd{}{{x^{1}}}}{p},\dots,\restriction{\dpd{}{{x^{n}}}}{p} }
\end{equation*}
è una \href{20250102163502-base_di_uno_spazio_vettoriale.org}{base} di \(\operatorname{T}_{p}M\) (vedi \href{20250114103339-teorema_sulla_base_dello_spazio_tangente_ad_un_punto_di_una_varieta_differenziabile.org}{Teorema sulla base dello spazio tangente ad un punto di una varietà differenziabile}), e quindi
\begin{equation*}
X_{p} = X_{p}^{r} \restriction{\dpd{}{{x^{r}}}}{p}
\end{equation*}
con \(X_{p}^{r} \in \R\) (usando la \href{20250114105957-notazione_di_einstein.org}{notazione di Einstein}).



Si hanno quindi \(n\) funzioni
\begin{align*}
X^{r}: U &\longrightarrow \R\\
p &\longmapsto X_{p}^{r}
\end{align*}
che sono le \textbf{componenti del campo rispetto alla carta \((U,\varphi)\)}.

In particolare, se \(\tilde{\varphi}\) è la mappa di \(\operatorname{T}M\) indotta da \(\varphi\) e le \(\pi_{j}\) sono
\begin{align*}
\pi_{j}: \R^{2n} &\longrightarrow \R\\
(x_{1},\dots,x_{2n}) &\longmapsto x_{j}
\end{align*}
allora \(X^{r} = \varphi_{n+j}\circ \tilde{\varphi} \circ X\).
\subsection{{\bfseries\sffamily TODO} Scrittura come \(X=X^{r}\pd{}{{x^{r}}}\)\hfill{}\textsc{matematica\_lm:geo\_diff}}
\label{sec:orge7dfeb1}
\end{document}
