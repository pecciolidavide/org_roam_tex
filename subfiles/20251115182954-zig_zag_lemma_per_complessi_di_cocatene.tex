% Intended LaTeX compiler: pdflatex
\documentclass[../main]{subfiles}


\begin{document}

\section{Zig-Zag Lemma (per complessi di cocatene)}
\label{sec:orge466d71}
\begin{prop}
Sia
\begin{equation*}
\begin{tikzcd}
0 \arrow[r] & A^\bullet \arrow[r, "F^\bullet"] & B^\bullet \arrow[r, "G^\bullet"] & C^\bullet \arrow[r] & 0 \qquad (*)
\end{tikzcd}
\end{equation*}
una \href{20251115182707-sec_di_complessi_di_cocatene.org}{SEC} di \href{20251115182320-complesso_di_cocatene.org}{complessi di cocatene}. Allora esiste \(\delta\), detto ``morfismo di connessione'', tale che\footnote{Nota:
\begin{equation*}
H^k(C^\bullet) = \frac{\ker d_C^k}{\operatorname{Im} d_C^{k-1}}
\end{equation*}
Questa è la \href{20251115182537-coomologia_di_un_complesso_di_cocatene.org}{Coomologia di un complesso di cocatene}.}
\begin{equation*}
\begin{tikzcd}[row sep=large, column sep=large]
H^k(A^\bullet) \arrow[r, "{F^{*}}"] & H^k(B^\bullet) \arrow[r, "{G^{*}}"]
    \arrow[phantom, d, ""{coordinate, name=Z}]
    & H^k(C^\bullet) \arrow[dll, "\delta", out=-30, in=150, overlay] \\
H^{k+1}(A^\bullet) \arrow[r, "{F^{*}}"] & H^{k+1}(B^\bullet) \arrow[r, "{G^{*}}"] & H^{k+1}(C^\bullet)
\end{tikzcd}
\end{equation*}
è una \href{20251115182707-sec_di_complessi_di_cocatene.org}{successione esatta lunga}.
\end{prop}


\begin{proof}
La dimostrazione si svolge in due fasi:
\begin{enumerate}
\item Devo trovare \(\delta\) (buona def + applicazione lineare).
\item Esattezza della successione (solo \(\ker F^* = \operatorname{Im} \delta\)).
\end{enumerate}

\textbf{FASE 1.}

Se \(*\) è SEC, allora il seguente è diagramma commutativo a righe esatte.
\begin{equation*}
\begin{tikzcd}[sep=small]
	& \vdots && \vdots && \vdots \\
	0 & {A^{k+1}} && {B^{k+1}} && {C^{k+1}} & 0 \\
	&&&& {(1)} \\
	0 & {A^k} && {B^k} && {C^k} & 0 \\
	& \vdots && \vdots && \vdots
	\arrow[from=2-1, to=2-2]
	\arrow[from=2-2, to=1-2]
	\arrow["{F_{k+1}}", from=2-2, to=2-4]
	\arrow[from=2-4, to=1-4]
	\arrow["{G_{k+1}}", from=2-4, to=2-6]
	\arrow[from=2-6, to=1-6]
	\arrow[from=2-6, to=2-7]
	\arrow[from=4-1, to=4-2]
	\arrow["{d_A^k}", from=4-2, to=2-2]
	\arrow["{F_k}", from=4-2, to=4-4]
	\arrow["{d_B^k}"', from=4-4, to=2-4]
	\arrow["{G_k}", from=4-4, to=4-6]
	\arrow["{d_C^k}", from=4-6, to=2-6]
	\arrow["{G_k}", from=4-6, to=4-7]
	\arrow[from=5-2, to=4-2]
	\arrow[from=5-4, to=4-4]
	\arrow[from=5-6, to=4-6]
\end{tikzcd}
\end{equation*}

Sia quindi \([c] \in H^k(C^\bullet)\) (i.e. \(c \in C^k\) t.c. \(dc = 0\)).

Poiché la seconda riga è esatta allora \(G^k\) è \href{20241213105600-funzione_suriettiva.org}{suriettiva}, e pertanto esiste \(b \in B^k\) t.c.
\begin{equation*}
G b = c.
\end{equation*}

Considero ora \(\operatorname{d}_B^k(b)\) (abbreviato \(\dif b\)).
\begin{equation*}
G \dif b = \dif G b =\dif c = 0
\end{equation*}
\emph{(per commutatività di (1))}

Segue quindi che
\begin{equation*}
\dif b \in \ker G^{k+1} = \operatorname{Im} F^{k+1}
\end{equation*}
dove l'ultima uguaglianza sussiste per esattezza della prima riga.
Pertanto esiste \(a \in A^{k+1}\) t.c.
\begin{equation*}
F a = b' \coloneqq \dif b
\end{equation*}

Questo \(a\) è unico poiché \(F\) iniettiva (per esattezza della I\textsuperscript{a} riga).
Inoltre
\begin{equation*}
F \dif a = \dif Fa = \dif\dif b = 0 \IMPLICA \dif a = 0.
\end{equation*}

Definisco quindi
\begin{equation*}
\boxed{ \delta([c]) := [a] }
\end{equation*}

Abbiamo fatto:
\begin{equation*}
\begin{tikzcd}
	a & {\dif b} \\
	& b & c
	\arrow["F", maps to, from=1-1, to=1-2]
	\arrow["{\operatorname{d}}", maps to, from=2-2, to=1-2]
	\arrow["G", maps to, from=2-2, to=2-3]
\end{tikzcd}
\end{equation*}

Resta da dimostrare:
\begin{enumerate}
\item Che l'unica scelta fatta (ovvero \(b\)) non influisce sulla definizione.
\item Che \(\delta\) è ben definita (rispetto al quoziente).
\item Che \(\delta\) è lineare.
\end{enumerate}

\textbf{1.} Siano \(b, b' \in B^k\) tali che
\begin{equation*}
G^k(b) = G^k(b') = c.
\end{equation*}
e sia \(a, a' \in A^{k+1}\) t.c. \(F a = \dif b\), \(F a' = \dif b'\). Allora
\begin{equation*}
0 = G(b) - G(b') = G(b - b')
\end{equation*}
e pertanto (per esattezza della seconda riga) esiste \(\tilde{a} \in A^k\) t.c.
\begin{align*}
b - b' &= F \tilde{a}\\
\operatorname{d}(b - b') &= \dif F \tilde{a}\\
\dif b - \dif b' &= F \dif \tilde{a}
\end{align*}
Pertanto, ricordando che \(Fa = \dif b\):
\begin{align*}
Fa %
	&= \dif b = \dif b' + F \dif \tilde{a}\\
	&= F a' + F\dif \tilde{a}\\
	&= F (a' +\dif \tilde{a})
\end{align*}
e, poiché \(F\) è iniettiva, segue che:
\begin{equation*}
a = a' + \dif \tilde{a} \IMPLICA [a] = [a'].
\end{equation*}

\textbf{2.} Sia \(c \in \operatorname{Im} d^{k-1}\). Dimostriamo \(a \in \operatorname{Im} d^k\), dove la nomenclatura è la solita:
\begin{equation*}
\begin{tikzcd}
	a & {\dif b} \\
	& b & c
	\arrow["F", maps to, from=1-1, to=1-2]
	\arrow["{\operatorname{d}}", maps to, from=2-2, to=1-2]
	\arrow["G", maps to, from=2-2, to=2-3]
\end{tikzcd}
\end{equation*}

Sia quindi \(c = \dif \gamma\). Per esattezza \(G\) suriettiva, e quindi \(\gamma = G \beta\).
\begin{equation*}
c = \dif G \beta = G (\dif \beta).
\end{equation*}

Per il punto precedente, la scelta di \(b\) è ininfluente, pertanto pongo
\begin{equation*}
b := \dif \beta, \quad \text{e } G(\dif \dif\beta) = c
\end{equation*}
La situazione quindi è la seguente:
\begin{equation*}
\begin{tikzcd}
	a & {\dif b} \\
	& b & c \\
	& \beta & \gamma
	\arrow["F", maps to, from=1-1, to=1-2]
	\arrow["{\operatorname{d}}", maps to, from=2-2, to=1-2]
	\arrow["G", maps to, from=2-2, to=2-3]
	\arrow["{\operatorname{d}}", maps to, from=3-2, to=2-2]
	\arrow["G"', maps to, from=3-2, to=3-3]
	\arrow["{\operatorname{d}}"', maps to, from=3-3, to=2-3]
\end{tikzcd}
\end{equation*}

Segue che \(F(a) = \dif b = \dif\dif \beta = 0\) e dunque, per iniettività, \(a=0\).
Siccome \(0 \in \operatorname{Im} d^k\), la tesi.

\textbf{3.} \(\delta\) è lineare poiché ?

\textbf{FASE 2.}

Resta da dimostrare l'esattezza. Noi dimostriamo soltanto
\begin{equation*}
\ker F^* = \operatorname{Im} \delta.
\end{equation*}

(\(\supseteq\)): Sia \([a] = \delta^*[c] \in \operatorname{Im} \delta^*\):
\begin{equation*}
\begin{tikzcd}
	a & {\dif b} \\
	& b & c
	\arrow["F", maps to, from=1-1, to=1-2]
	\arrow["{\operatorname{d}}", maps to, from=2-2, to=1-2]
	\arrow["G", maps to, from=2-2, to=2-3]
\end{tikzcd}
\end{equation*}

Quindi si ha che
\begin{equation*}
Fa = \dif b \IMPLICA [Fa] = 0
\end{equation*}
Ma \(0 = [Fa] \eqqcolon F^*[a]\) e quindi \([a] \in \ker F^*\).

(\(\subseteq\)): Sia \([a]\) t.c. \(F^*[a] = 0\), \([a] \in \ker F^{*}\).
i.e. \(Fa = \dif b\).

Detto quindi \(c = G b\)
\begin{equation*}
\dif c = \dif G b = G \dif b = G F a = 0
\end{equation*}
\emph{(righe esatte: \(GF = 0\))}

Pertanto, è possibile utilizzare la costruzione di prima:
\begin{equation*}
\begin{tikzcd}
	a & {\dif b} \\
	& b & c
	\arrow["F", maps to, from=1-1, to=1-2]
	\arrow["{\operatorname{d}}", maps to, from=2-2, to=1-2]
	\arrow["G", maps to, from=2-2, to=2-3]
\end{tikzcd}
\end{equation*}
ottenendo \(\delta[c]=[a]\). Segue che \([a] \in\operatorname{Im}\delta\).
\end{proof}
\end{document}
