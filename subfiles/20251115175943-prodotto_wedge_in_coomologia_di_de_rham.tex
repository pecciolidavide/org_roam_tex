% Intended LaTeX compiler: pdflatex
\documentclass[../main]{subfiles}

\def\HdRham{\operatorname{H}_{\text{dR}}}


\begin{document}

\section{Prodotto wedge in Coomologia di De Rham}
\label{sec:orgfc45a6b}
Sia \(M\) una \href{20250113115909-struttura_differenziabile.org}{varietà differenziabile} e sia \(\HdRham^{\bullet}(M)\) la \href{20241213095808-somma_diretta.org}{somma diretta} dei \href{20251115172442-gruppo_di_coomologia_di_de_rham.org}{gruppi di coomologia di De Rham}.
\begin{definizione}
Si definisce il \uline{prodotto cup} (o \uline{prodotto wedge})
\begin{align*}
\wedge: \HdRham^{\bullet}(M)\times \HdRham^{\bullet}(M) &\longrightarrow \HdRham^{\bullet}(M)\\
([\omega],[\eta]) &\longmapsto [\omega \wedge \eta]
\end{align*}
dove \(\omega\wedge\eta\) è il \href{20251115155511-forma_differenziale_in_un_punto.org}{prodotto wedge tra forme differenziali}.
\end{definizione}

\begin{oss}
Questo prodotto rende \(\HdRham^{\bullet}(M)\) un'\href{20250110175552-algebra_su_un_campo.org}{algebra (reale)} \href{20251201160758-isomorfismo_tra_algebre_graduate.org}{graduata} \href{20250110175552-algebra_su_un_campo.org}{associativa} e \href{20250110175552-algebra_su_un_campo.org}{anticommutativa}.
\end{oss}
\end{document}
