% Intended LaTeX compiler: pdflatex
\documentclass[../main]{subfiles}

\usepackage[hyperref]{biblatex}
\date{}
\title{}
\begin{document}

\section{Spazio affine unidimensionale senza un punto non è chiuso}
\label{sec:orge49e4f6}
\subsection{Proposizione}
\label{sec:orgcf4be71}
Sia \(\K\) un \href{20241231112713-campo_algebricamente_chiuso.org}{Campo Algebricamente Chiuso}, e sia \(\A^{1}\) lo \href{20241231114009-spazio_affine.org}{Spazio Affine}. Allora per ogni \(p \in \A^{1}\), \(\A^{1}\setminus\set{p}\) non è chiuso nella \href{20250103101459-topologia_di_zariski_affine.org}{topologia di Zariski}.
\subsubsection{Dimostrazione}
\label{sec:org1c2fc2a}
\(\A^{1}\setminus\set{p}\) è chiuso, per definizione, se è una \href{20241231114256-varieta_algebrica_affine.org}{Varietà Algebrica Affine}.

Supponiamo per assurdo che lo sia. \href{20250103100601-varieta_algebriche_luogo_di_zeri_di_finiti_polinomi.org}{Allora} esistono \(f_{1},\dots,f_{k} \in \K[x]\) (vedi \href{20241219113434-anello_dei_polinomi.org}{anello dei polinomi}) tali che
\begin{equation*}
\A^{1}\setminus\set{p} = V(f_{1},\dots,f_{k})=V(f_{1})\cap \dots\cap V(f_{k})
\end{equation*}
(vedi \href{20241231112823-radici_polinomiali.org}{Luogo di zeri}).

Per il \href{20250102154204-teorema_fondamentale_dell_algebra.org}{Teorema Fondamentale dell'Algebra}, i polinomi di \(\K[x]\) di grado \(d\) hanno al massimo \(d\) zeri, e dunque \(V(f_{i})\) è un insieme finito, così come \(V(f_{1},\dots,f_{k})\).

Ma \(\K\) è algebricamente chiuso, \href{20250102103618-infinitezza_campi_alg_chiusi.org}{dunque \(\K\) è infinito}, e quindi anche \(\A^{1}\) e \(\A^{1}\setminus\set{p}\) lo sono. Assurdo.
\end{document}
