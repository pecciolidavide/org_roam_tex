% Intended LaTeX compiler: pdflatex
\documentclass[../main]{subfiles}


\begin{document}

\section{Mappa di Segre}
\label{sec:org540a763}
Sia \(\K\) un \href{20241231112713-campo_algebricamente_chiuso.org}{campo algebricamente chiuso}.
\subsection{{\bfseries\sffamily TODO} Versione Baby\hfill{}\textsc{matematica\_lm:geo\_alg}}
\label{sec:org15bc291}

Si consideri l'applicazione
\begin{align*}
\sigma: \mathds{P}^{1}\times \mathds{P}^{1} &\longrightarrow \mathds{P}^{3}_{[z_{0}:\dots:z_{3}]}\\
\left([x_{0}:x_{1}], [y_{0}:y_{1}]\right) &\longmapsto [x_{0}y_{0}:x_{0}y_{1}:x_{1}y_{0}:x_{1}y_{1}]
\end{align*}
(vedi \href{20241231115051-spazio_proiettivo.org}{Spazio Proiettivo}).

Sia \(Q=\sigma(\mathds{P}^{1}\times \mathds{P}^{1})\). È possibile dimostrare che
\begin{equation*}
Q=V(z_{0}z_{3}-z_{1}z_{2})
\end{equation*}
(vedi \href{20241231112823-radici_polinomiali.org}{Luogo di zeri})
\subsection{Morfismo di Segre.}
\label{sec:orgb05155c}
Siano \(m,n\ge 0\), e sia \(N=(n+1)(m+1)-1\), e si consideri la seguente applicazione:
\begin{align*}
\sigma_{n,m}: \mathds{P}^{n}_{x}\times \mathds{P}^{m}_{y} &\longrightarrow \mathds{P}^{N}_{z=z_{ij}}\\
\left([x_{0}:\dots:x_{n}],[y_{0}:\dots:y_{m}]\right) &\longmapsto [\dots:x_{i}y_{j}:\dots]
\end{align*}
Segue da banali calcoli che \(\sigma_{n,m}\) sia ben definita. Sia \(\Sigma_{m,n}\coloneqq \sigma_{n,m}(\mathds{P}^{n}\times \mathds{P}^{m})\).

La mappa \(\sigma_{n,m}\) è descritta dalle equazioni \(z_{ij}=x_{i}y_{j}\), ovvero
\begin{equation*}
A\coloneqq\begin{bmatrix}
z_{00} & z_{01} & \dots & z_{0m}\\
\vdots\\
z_{n0} & x_{z_{n1}} & \dots & z_{nm} \end{bmatrix}
 = \begin{bmatrix}
x_{0}\\
\vdots\\
x_{n}\end{bmatrix} \ \begin{bmatrix}
y_{0} & \dots & y_{m}
\end{bmatrix}
\end{equation*}
Notiamo che la \href{20250104111539-spazio_delle_matrici.org}{matrice} \(A\) è di \href{20250104170945-rango_di_una_matrice.org}{rango} \(1\), poiché ogni riga è multipla di \([y_{0}\ \dots \ y_{m}]\).
Inoltre, la condizione \(\rank A = 1\) è polinomiale: tutti i determinanti dei minori \(2\times 2\) sono nulli.

Dunque: \(\Sigma_{n,m} \subseteq V(\rank A=1)\) (vedi \href{20241231112823-radici_polinomiali.org}{Luogo di zeri}).
\subsubsection{Dimostriamo l'uguaglianza}
\label{sec:orgbdc866b}
Sia \(p \in V(\rank A =1)\), \(p=[\dots:p_{ij}:\dots]\).

Supponiamo che \(p_{00}\neq 0\). È dunque possibile porre \(p_{00}=1\). Considerando l'equazione, presente tra quelle che definiscono \(\rank A=0\):
\begin{equation*}
z_{00}\ z_{ij} = z_{i0}\ z_{0j}
\end{equation*}
e applicandola a \(p\), si ottiene che \(p_{ij}=p_{i0}\ p_{0j}\).

Dunque, posti
\begin{align*}
A&= [1:p_{10}:p_{20}:\dots:p_{n0}] \in \mathds{P}^{n}\\
B&=[1:p_{01}:p_{02}:\dots:p_{0m}] \in \mathds{P}^{m}
\end{align*}
si ha che \(p=\sigma_{n,m}(A,B)\).
\subsubsection{La mappa di Segre è \href{20241219101956-funzione_iniettiva.org}{iniettiva}}
\label{sec:org622dcaa}

Siano \((A,B), (A',B') \in \mathds{P}^{n}\times\mathds{P}^{m}\) tali che
\begin{equation*}
\sigma_{n,m}(A,B)\sigma_{n,m}(A',B')=P
\end{equation*}
con \(P=[\dots:p_{ij}:\dots]\).

Supponiamo che \(p_{00}\neq 0\). È possibile porre quindi \(p_{00}=1\). Segue che, se
\begin{align*}
A&=[a_{0}:\dots:a_{n}]\\
B&=[b_{0}:\dots:b_{m}]\\
A'&=[a_{0}':\dots:a_{n}']\\
B'&=[b_{0}':\dots:b_{m}']
\end{align*}
allora
\begin{enumerate}
\item \(a_{0}' b_{0}'\neq 0\) e pertanto, poiché \(\K\) è un campo, \(a_{0}'\neq 0\) e \(b_{0}'\neq 0\). È dunque possibile porre entrambi \(=1\), e si ottiene che
\end{enumerate}
\begin{align*}
A'&= [1:a_{1}':\dots:a_{n}']\\
B' &= [1:b_{1}':\dots:b_{m}']
\end{align*}
\begin{enumerate}
\item \(a_{0} b_{0}\neq 0\) e pertanto, poiché \(\K\) è un campo, \(a_{0}\neq 0\) e \(b_{0}\neq 0\). È dunque possibile porre entrambi \(=1\), e si ottiene che
\end{enumerate}
\begin{align*}
A&= [1:a_{1}:\dots:a_{n}]\\
B &= [1:b_{1}:\dots:b_{m}]
\end{align*}

A questo punto si sfrutta il fatto che:
\begin{align*}
a_{i}&=a_{i} b_{0} = p_{i0} = a_{i}' b_{0}' = a_{i}' & \forall\, i&=1,\dots,n\\
b_{j}&=a_{0} b_{j} = p_{0j} = a_{0}' b_j' = b_{j}' & \forall\, j&=1,\dots,m
\end{align*}
e pertanto \(A=A'\) e \(B=B'\).
\subsection{Prodotto di spazi proiettivi come varietà proiettiva}
\label{sec:orgdea66f8}
Si definisce \(\mathds{P}^{n}\times \mathds{P}^{m}\) come \href{20241231123223-varieta_algebrica_proiettiva.org}{varietà proiettiva} con la struttura indotta da \(\sigma_{n,m}\),
\begin{equation*}
\Sigma_{n,m} \subseteq \mathds{P}^{N}
\end{equation*}
\end{document}
