% Intended LaTeX compiler: pdflatex
\documentclass[../main]{subfiles}


\begin{document}

\section{Classe funzione Beth}
\label{sec:orgb7a6453}
Contesto: \href{20250130104245-morse_kelly_set_theory.org}{Morse Kelly Set Theory} + \href{20250206171508-axiom_of_choiche.org}{AC}

\begin{definizione}
Consideriamo le due classi proprie \(\operatorname{Ord}\) e \(\operatorname{Card}\). Su definisce la \href{20250202170607-classe_relazione_binaria.org}{classe-funzione}
\begin{equation*}
\beth:\operatorname{Ord}\to \operatorname{Card}
\end{equation*}
per \href{20250207121906-teorema_di_ricorsione.org}{ricorsione}:
\begin{equation*}
\beth_{0} = \omega; \quad \beth_{\alpha+1} = 2^{\beta_\alpha};\quad \beth_{\lambda} = \operatorname{sup}_{\alpha<\lambda} 2^{\beth_{\alpha}}\text{ per }\lambda\text{ limite}
\end{equation*}

Riferimenti\footnote{Vedi:
\begin{itemize}
\item \href{20250203111003-ordinali.org}{Ordinali}
\item \href{20250203161341-cardinali.org}{Cardinali}
\item \href{20250203161110-numeri_naturali_sono_ordinali.org}{Ordinale omega},
\item \href{20250612151505-aritmetica_dei_cardinali.org}{Elevamento a potenza di Cardinali},
\item \href{20250203102516-massimo_e_minimo.org}{Infimum e supremum}
\item \href{20250203161341-cardinali.org}{Supremum di un insieme di cardinali è un cardinale},
\item \href{20250203161132-ordinale_limite.org}{Ordinale limite}
\item \href{20250207123015-aritmentica_per_gli_ordinali.org}{Aritmentica per gli ordinali}
\end{itemize}}
\end{definizione}
\end{document}
