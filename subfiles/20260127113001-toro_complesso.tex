% Intended LaTeX compiler: pdflatex
\documentclass[../main]{subfiles}


\begin{document}

\section{Toro complesso}
\label{sec:org08b66d4}
\subsection{Costruzione come quoziente di gruppi}
\label{sec:org55416af}

In \(\C^{n}\) si considerino \(2n\) vettori \(\set{w_{1},\dots,w_{2n}}\), \href{20241212142019-insiemi_linearmente_indipendenti.org}{linearmente indipendenti} su \(\R\).

Si consideri ora il \href{20241206143051-sottogruppo.org}{sottogruppo} additivo di \(\C^{n}\) \href{20260115123905-sottogruppo_generato.org}{generato} da \(\set{w_{1},\dots,w_{2n}}\):
\begin{equation*}
\Gamma \coloneqq \set{m_{1}w_{1} + \dots + m_{2n}w_{2n} \mid m_{i} \in \Z}
\end{equation*}
Alcune proprietà:
\begin{itemize}
\item \(\Gamma\) è \href{20241206115531-morfismo_di_gruppi.org}{isomorfo} a \(\Z^{2n}\) come \href{20241205141146-gruppo_abeliano.org}{gruppo},
\item \(\Gamma \subseteq \C^{n}\) è chiuso;
\item \(\Gamma\) ha la \href{20250317165247-topologia_discreta.org}{topologia discreta}, ovvero ogni suo punto è \href{20250403131856-punto_isolato.org}{isolato}.
\end{itemize}
\(\Gamma\) prende il nome di \uline{reticolo}.

Sia quindi \(X\coloneqq \C^{n}/\Gamma\) il \href{20250127093819-quoziente_di_gruppo_e_sottogruppo.org}{gruppo quoziente}, con proiezione suriettiva:
\begin{equation*}
\pi: \C^{n}\to X.
\end{equation*}
Si dota quindi \(X\) della \href{20250129155316-spazio_topologico_quoziente.org}{topologia quoziente}.

\begin{itemize}
\item \textbf{Topologia di \(X\)}:

Definiamo la mappa \(\phi: \C^{n}\to \R^{2n}:\ w_{i}\mapsto e_{i}\), dove \(\set{e_{i}}\) è la \href{20250102163502-base_di_uno_spazio_vettoriale.org}{base} canonica di \(\R^{2n}\). Questa è \(\R\)-\href{20250114101949-funzione_lineare.org}{lineare} ed è un \href{20250111142332-omeomorfismo.org}{omeomorfismo}.

Quindi è una \href{20250104114559-funzione_chiusa.org}{mappa chiusa}, e, siccome \(\Gamma\) è chiuso, \href{20250104114559-funzione_chiusa.org}{anche} \(\restriction{\phi}{\Gamma}\) è chiusa. Inoltre\footnote{Vedi ``\href{20250202173528-dominio_range_e_campo_di_una_classe_relazione.org}{Range di una funzione}''}
\begin{equation*}
  \phi(\Gamma) = \Z^{2n}
\end{equation*}
e \href{20250104114559-funzione_chiusa.org}{pertanto} \(\phi\) è un omeomorfismo tra \(\Gamma\) e \(\Z^{2n}\).

Inoltre, ovviamente, \(\phi\) induce un isomorfismo tra i gruppi:
\begin{itemize}
\item \(\R^{2n}\) e \(\C^{n}\);
\item \(\Gamma\) e \(\Z^{2n}\).
\end{itemize}

Pertanto\footnote{Questa cosa non mi è per nulla chiara}
\begin{equation*}
  X \mathrel{\overset{\text{omeo}}{\approx}} \frac{\R^{2n}}{\Z^{2n}} \mathrel{\overset{\text{omeo}}{\approx}} (\mathds{S}^{1})^{2n}
\end{equation*}
e quindi \(X\) è una \href{20250111092123-varieta_topologica.org}{varietà topologica} \href{20250103163701-spazio_topologico_compatto.org}{compatta}, orientabile e di dimensione pari.

\item \textbf{Mappa \(\pi\)}

Siccome \(X\) è il \href{20260128105613-azione_di_gruppo_su_uno_spazio_topologico.org}{quoziente per l'azione di \(\Gamma\) su \(\C^{n}\)}, allora è una \href{20250104114559-funzione_chiusa.org}{mappa aperta}.
\end{itemize}

Possiamo ora costruire un atlante complesso per \(X\).

\begin{enumerate}
\item Siccome \(0 \in \Gamma\) e \(\Gamma\) è discreto, esiste un intorno aperto \(U\) di \(0\) in \(\C\) tale che \(\Gamma\cap U = \set{0}\). Ovvero
\begin{equation*}
\exists \varepsilon >0\ \tc\quad \forall w \in \Gamma\setminus\set{0}\ \norma{w} > 2\varepsilon.
\end{equation*}

Allora, per ogni \(z_{0} \in \C^{n}\) fissato, detto
\begin{equation*}
D(z_{0};\varepsilon) \coloneqq \set{z \in \C^{n} \mid \norma{z-z_{0}}<\varepsilon}
\end{equation*}
la mappa
\begin{equation*}
\restriction{\pi}{D(z_{0};\varepsilon)}: D(z_{0};\varepsilon) \to X
\end{equation*}
è \href{20241219101956-funzione_iniettiva.org}{iniettiva}: se \(z_{1},z_{2} \in D\) sono tali che \(\pi(z_{1})=\pi(z_{2})\) allora
\begin{equation*}
z_{1}-z_{2} \in \Gamma:\qquad \norma{z_{1}-z_{2}} \le \norma{z_{1}-z_{0}} + \norma{z_{0}-z_{2}} < 2\varepsilon
\end{equation*}
e quindi \(z_{1}-z_{2} = 0\).

\item Sia quindi \(U_{z_{0}} \coloneqq \pi[D(z_{0};\varepsilon)]\). Siccome \(\pi\) è una mappa aperta, \(U_{z_{0}} \subseteq X\) è un aperto. Inoltre
\begin{equation*}
 \restriction{\pi}{D(z_{0};\varepsilon)}:D(z_{0};\varepsilon) \to U_{z_{0}}
\end{equation*}
è un \href{20250111142332-omeomorfismo.org}{omeomorfismo}, poiché è una funzione continua, biiettiva e aperta\footnote{Infatti \href{20250104114559-funzione_chiusa.org}{restrizione di mappe aperte ad un aperto sono ancora aperte}.}.

Ammette quindi un inverso \(\varphi_{z_{0}} \coloneqq \big(	\restriction{\pi}{D(z_{0};\varepsilon)}\big)^{-1}\):
\begin{equation*}
 \varphi_{z_{0}}:U_{z_{0}}\to D(z_{0};\varepsilon).
\end{equation*}

Quindi \((U_{z_{0}};\varphi_{z_{0}})\) è una carta locale.
\end{enumerate}

Si propone \(\set{(U_{z_{0}};\varphi_{z_{0}})}_{z_{0} \in \C^{n}}\) come \href{20260127112715-atlante_complesso.org}{atlante complesso}.
\begin{itemize}
\item Sicuramente \(\displaystyle X = \bigcup_{z_{0} \in \C^{n}} U_{z_{0}}\).
\item Si dimostra la compatibilità: siano \(z_{0},z_{1} \in \C^{n}\) tali che \(U_{z_{0}}\cap U_{z_{1}} \neq \emptyset\):
\begin{equation*}
\begin{tikzcd}[ampersand replacement=\&]
        \&\& {U_{z_0}\cap U_{z_1}} \\
        \\
        {D(z_0;\varepsilon)\supseteq A_0} \&\&\&\& {A_1 \subseteq D(z_1;\varepsilon)}
        \arrow["{\varphi_{z_0}}", from=1-3, to=3-1]
        \arrow["{\varphi_{z_1}}"', from=1-3, to=3-5]
        \arrow["\pi", shift left=3, from=3-1, to=1-3]
        \arrow["h"', from=3-1, to=3-5]
        \arrow["\pi"', shift right=3, from=3-5, to=1-3]
\end{tikzcd}
\end{equation*}

Per ogni \(z \in A_{0}\) si ha
\begin{equation*}
  \pi(z) = \pi(h(z))
\end{equation*}
e pertanto \(z-h(z) \in \Gamma\).

Se consideriamo l'inclusione \(\iota: A_{0} \subseteq \C^{n} \hookarrow \C^{n}\) si ha che
\begin{equation*}
  i-h : A_{0} \to \C
\end{equation*}
ha immagine in \(\Gamma\) insieme discreto. Inoltre \(i-h\) è continua, e pertanto è \href{20250325153824-funzione_localmente_costante.org}{localmente costante}.

\href{20250325154046-funzione_localmente_costante_sse_costante_sulle_componenti_connesse.org}{Pertanto}, \(i-h\) è costante sulle \href{20250325160128-componente_connessa_di_uno_spazio_topologico.org}{componenti connesse} di \(A_{0}\), ovvero per ogni \(\Omega \subseteq A_{0}\) componente connessa (aperta) esiste \(\omega_{\Omega} \in \Gamma\) tale che
\begin{equation*}
  \forall z \in \Omega:\qquad h(z) = z+\omega_{\Omega}.
\end{equation*}

Pertanto \(\restriction{h}{\Omega}\) è olomorfa, e quindi lo è \(h\).
\end{itemize}
\end{document}
