% Intended LaTeX compiler: pdflatex
\documentclass[../main]{subfiles}


\begin{document}

\section{\href{20241231112750-polinomio.org}{Polinomi}}
\label{sec:orgff8e387}

\subsection{Osservazione}
\label{sec:orgae8f597}
Se \(T \subseteq S \subseteq \K[x_{1},\dots,x_{n}]\) (vedi \href{20241219113434-anello_dei_polinomi.org}{Anello-dei-polinomi}), allora
\begin{equation*}
V(S) \subseteq V(T).
\end{equation*}
Infatti, se \(p \in V(S)\) allora per ogni \(f \in S\) si ha che \(f(p)=0\) e, in particolare, per ogni \(g \in T \subseteq S\) si ha che \(g(p)=0\) e, pertanto, \(p \in V(T)\).
\section{\href{20241231121125-polinomi_omogenei.org}{Polinomi omogenei}}
\label{sec:org36b31e8}

\section{\href{20241219112955-ideale.org}{Ideali}}
\label{sec:orgc2f1218}

\section{\href{20250102182726-ideale_di_polinomi_omogeneo.org}{Ideali omogenei}}
\label{sec:org68b06b3}
\end{document}
