% Intended LaTeX compiler: pdflatex
\documentclass[../main]{subfiles}


\begin{document}

\section{Teorema di Löwenheim-Skolem all'insù}
\label{sec:orgc27d231}
Sia \(\mathcal{L}\) un \href{20250130162057-linguaggio_del_prim_ordine.org}{linguaggio del prim'ordine}.
\subsection{Teorema}
\label{sec:org94ef627}
Per ogni \(\mathcal{L}\)-\href{20250131103035-struttura_del_prim_ordine.org}{struttura} \(M\) \href{20250205120448-classe_finita_e_infinita_mk.org}{infinita}, e per ogni \href{20250203161341-cardinali.org}{cardinale} \(\lambda\), esiste una \(\mathcal{L}\)-\href{20250131103035-struttura_del_prim_ordine.org}{struttura} \(N\) tale che
\begin{enumerate}
\item \(M\preceq N\) (vedi \href{20250212102253-sottostruttura_elementare.org}{Sottostruttura elementare});
\item \(\card{N}>\lambda\) (vedi \href{20241213101756-cardinalita.org}{Cardinalità} e \href{20250203111003-ordinali.org}{Relazione d'ordine sugli ordinali}).
\end{enumerate}
\subsubsection{Dimostrazione}
\label{sec:orgc56029f}
Sia \(x=\langle x_{i}\ |\ i<\lambda\rangle\) una \href{20250206170922-sequenze_e_stringhe.org}{sequenza} di \href{20250130162057-linguaggio_del_prim_ordine.org}{variabili} distinte.
Il \href{20250212164424-tipo_teoria_dei_modelli.org}{tipo}
\begin{equation*}
p(x) = \set{x_{i}\neq x_{j}\ |\ i<j<\lambda}
\end{equation*}
è \href{20250212164424-tipo_teoria_dei_modelli.org}{finitamente soddisfacibile} su \(M\), in quanto struttura infinita\footnote{Infatti, se la struttura è infinita allora esiste un numero arbitrariamente grande di elementi a due a due distinti.}.

Applicando il \href{20250212171043-teorema_di_compattezza_per_tipi.org}{Teorema di Compattezza per tipi}, si ottiene che esiste \(N\) tale che \(M\preceq N\) e tale che \href{20250212164424-tipo_teoria_dei_modelli.org}{soddisfa} \(p(x)\).

Pertanto, \(\card{N}\ge \lambda\), in quanto contiene almeno \(\lambda\) elementi distinti.
\end{document}
