% Intended LaTeX compiler: pdflatex
\documentclass[../main]{subfiles}

\usepackage[hyperref]{biblatex}
\date{}
\title{}
\begin{document}

\section{Quoziente, resto, MCD e mcm sono funzioni ricorsive primitive}
\label{sec:org2522c1c}
Vedi: \href{20250202130045-insieme_dei_numeri_naturali_mk.org}{Insieme dei numeri naturali MK}
\subsection{Quoziente e resto}
\label{sec:org6279b5c}
Sono \href{20250215141024-funzioni_primitive_ricordive.org}{funzioni ricorsive primitive}:
\begin{itemize}
\item la funzione \(\operatorname{Quoz}(m,n)\) che calcola il \href{20250108174027-divisione.org}{quoziente della divisione intera} di \(m\) per \(n\) quando \(n>0\), e \(0\) altrimenti;
\item la funzione \(\operatorname{Res}(m,n)\) che calcola il resto della divisione intera di \(m\) e \(n\) quando \(n>0\), e \(m\) altrimenti.
\end{itemize}

Per ogni \(m, n \in \N\) vale
\begin{equation*}
m = \operatorname{Quoz}(m,n) \cdot n + \operatorname{Res}(m,n)
\end{equation*}
\subsection{MCD mcm}
\label{sec:org3dbb021}
Il \href{20250215174227-mcd.org}{massimo comun divisore} e il \href{20250215174234-mcm.org}{minimo comune multiplo} sono funzioni ricorsive primitive.
\end{document}
