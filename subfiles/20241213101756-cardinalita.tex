% Intended LaTeX compiler: pdflatex
\documentclass[../main]{subfiles}

\def\S{\operatorname{S}}
\def\Ord{\operatorname{Ord}}


\begin{document}

\section{Cardinalità}
\label{sec:org8012113}
Contesto: \href{20250130104245-morse_kelly_set_theory.org}{Morse Kelly Set Theory}
\begin{definizione}
Se \(X\) è un \href{20250130104331-insieme_mk.org}{insieme} \href{20250203161431-classe_ben_ordinabile_mk.org}{ben ordinabile}, la \textbf{cardinalità} di \(X\) è il più \href{20250203111003-ordinali.org}{piccolo} \href{20250203111003-ordinali.org}{ordinale} \(|X|\) in \href{20250104111707-funzione_biunivoca.org}{biiezione} con \(X\).
\end{definizione}

\begin{oss}
In particolare, se \(\alpha \in \Ord\), allora \(|\alpha|\) è il più piccolo ordinale \href{20250619101109-classi_equipotenti.org}{\(\beta \asymp \alpha\)}: \(|\alpha|\le\alpha\).
\end{oss}
\begin{oss}
La cardinalità di un insieme, quando esiste, è un \href{20250203161341-cardinali.org}{cardinale}.
\end{oss}
\begin{prop}
Siano \(\alpha,\beta\) due \href{20250203161110-numeri_naturali_sono_ordinali.org}{ordinali} e si consideri l'\href{20250203111003-ordinali.org}{ordinale} \href{20250203161110-numeri_naturali_sono_ordinali.org}{omega}: \(\omega\)
\begin{enumerate}
\item Se \(\alpha\ge\omega\) (vedi \href{20250203111003-ordinali.org}{Relazione d'ordine sugli ordinali}), allora
\begin{equation*}
 |\alpha|=|\operatorname{S}(\alpha)|
\end{equation*}
(vedi \href{20250202124648-successore_di_un_insieme_mk.org}{Successore di un insieme MK})
\item Se \(|\alpha|\le\beta\le\alpha\) allora \(|\alpha|=|\beta|\).
\item \(|\alpha|=|\beta|\) se e solo se \(\alpha\asymp\beta\) (vedi \href{20250619101109-classi_equipotenti.org}{Classi equipotenti MK})
\item \(|\alpha|<|\beta|\) se e solo se \(\alpha\preceq\beta\).
\end{enumerate}

In particolare, se \(X,Y\) sono due \href{20250130104320-classe_mk.org}{classi} \href{20250203161431-classe_ben_ordinabile_mk.org}{ben ordinabili}, allora
\begin{itemize}
\item \(|X|\le|Y|\) se e solo se \(X\preceq Y\) (vedi \href{20241219101956-funzione_iniettiva.org}{Classe si inietta MK})
\item \(|X|=|Y|\) se e solo se \(X\asymp Y\) (vedi \href{20250619101109-classi_equipotenti.org}{Classi equipotenti MK})
\end{itemize}
\end{prop}
\begin{proof}
\begin{enumerate}
\item Consideriamo
\begin{equation*}
f:\operatorname{S}(\alpha) \to \alpha: \qquad \beta \mapsto\begin{cases}
\operatorname{S}(\beta) & \beta < \omega\\
\beta & \omega\le \beta< \alpha\\
0 & \beta = \alpha
\end{cases}
\end{equation*}

Questa è una biiezione, e pertanto \(\alpha\asymp\S (\alpha)\) (vedi \href{20250619101109-classi_equipotenti.org}{Classi equipotenti MK})
\href{20250203161341-cardinali.org}{e quindi} \(|\alpha|=|\S (\alpha)|\)

\item Sia \(f:\alpha\to|\alpha|\) una biiezione. Dunque, siccome \(|\alpha|\le \beta\)
\begin{equation*}
f:\alpha\to\beta
\end{equation*}
è \href{20241219101956-funzione_iniettiva.org}{iniettiva}. Inoltre se \(\beta\le\alpha\) allora \(\beta \subseteq \alpha\) (vedi \href{20250203111003-ordinali.org}{Proprietà degli ordinali}) e dunque l'identità
\begin{equation*}
i:\beta \to \alpha
\end{equation*}
è \href{20241219101956-funzione_iniettiva.org}{iniettiva}. Per il \href{20250205150457-teorema_di_cantor_bernstein_schroder.org}{Teorema di Cantor-Bernstein-Schröder}, \(\alpha\asymp\beta\).

\item Ovvia conseguenza delle \href{20250203161341-cardinali.org}{proprietà dei cardinali}.

\item Ovvia conseguenza delle \href{20250203161341-cardinali.org}{proprietà dei cardinali}.
\end{enumerate}
\end{proof}
\subsection{Definizione senza AC}
\label{sec:orgc3cdabe}
La definizione precedente è valida per ogni insieme se si assume \href{20250206171508-axiom_of_choiche.org}{AC}, \href{20250210104534-ac_e_classi_ben_ordinabili.org}{poiché AC implica che ogni insieme sia ben ordinabile}..

\uline{Senza assumere} \href{20250206171508-axiom_of_choiche.org}{AC} si può definire la cardinalità di un insieme \(X\) come segue: \href{20250515141706-da_finire.org}{DA FINIRE}
\end{document}
