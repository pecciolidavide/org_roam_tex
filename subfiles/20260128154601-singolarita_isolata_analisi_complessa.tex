% Intended LaTeX compiler: pdflatex
\documentclass[../main]{subfiles}


\begin{document}

\section{Singolarità isolata (Analisi Complessa)}
\label{sec:org4461344}
\subsection{Per superfici di Riemann}
\label{sec:org0797acc}

Sia \(X\) una \href{20260127112828-superficie_di_riemann.org}{superficie di Riemann}, \(p_{0} \in X\) e \(U\) \href{20250111142313-intorno.org}{intorno} \href{20250103145124-topologia.org}{aperto} di \(p_{0}\).

\begin{definizione}
Se \(f: U \setminus\set{p_{0}}\to \C\) è \href{20260128143822-funzione_olomorfa_su_una_superficie_di_riemann.org}{olomorfa}, allora si dice che \uline{\(f\) ha una singolarità isolata in \(p_{0}\)}
\end{definizione}

\begin{thm}
Ci sono esattamente tre possibilità:
\begin{enumerate}
\item esiste finito il \href{20250625110412-limite_analisi_matematica.org}{limite}
\begin{equation*}
 \lim_{p\to p_{0}} |f(p)|
\end{equation*}
e in questo caso si dice che \emph{\(f\) ha una singolarità eliminabile in \(p_{0}\)};
\item esiste il \href{20250625110412-limite_analisi_matematica.org}{limite}
\begin{equation*}
 \lim_{p\to p_{0}} |f(p)| = +\infty
\end{equation*}
e in questo caso si dice che \emph{\(f\) ha un polo in \(p_{0}\)};
\item non esiste il \href{20250625110412-limite_analisi_matematica.org}{limite}
\begin{equation*}
 \lim_{p\to p_{0}} |f(p)|
\end{equation*}
e in questo caso si dice che \emph{\(f\) ha una singolarità essenziale in \(p_{0}\)}.
\end{enumerate}
\end{thm}

\begin{oss}
Se \(\varphi :U_{1}\to V \subseteq \C\) è una carta locale in \(p_{0}\), con \(z_{0}\coloneqq\varphi(p_{0})\), allora \(f\circ \varphi^{-1}\) ha una singolarità isolata in \(z_{0}\) che sarà eliminabile / polo / essenziale in base a che tipo di singolarità è \(p\) per \(f\).
\end{oss}
\end{document}
