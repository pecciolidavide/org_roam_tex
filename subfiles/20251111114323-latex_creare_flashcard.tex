% Intended LaTeX compiler: pdflatex
\documentclass[../main]{subfiles}

\usepackage[hyperref]{biblatex}
\date{}
\title{}
\begin{document}

\section{\LaTeX{} - Creare Flashcard}
\label{sec:org08cfc99}
Utilizzando il pacchetto \texttt{flashcard} è possibile creare delle flashcard in
\LaTeX
(si rimanda alla documentazione\footnote{\url{https://ctan.mirror.garr.it/mirrors/ctan/macros/latex/contrib/flashcards/flashcards.pdf}}): si consigliano le seguenti impostazioni
\begin{verbatim}
   \documentclass[avery5388, frame]{flashcards}
   \renewcommand{\cardpaper}{a4paper}
\end{verbatim}

In particolare un MWE di Header
\begin{verbatim}
   \documentclass[avery5388, frame]{flashcards}
   \renewcommand{\cardpaper}{a4paper}

   % pacchetti fondamentali
   \usepackage[T1]{fontenc}
   \usepackage[utf8]{inputenc}
   \usepackage[italian]{babel}
   \usepackage[babel]{csquotes}
   \usepackage{microtype}

   % pacchetti utili
   \usepackage{lipsum}
   \usepackage{amsmath}

   % Sostituisce il comando \section
   \let\oldsection\section
   \makeatletter
   \def\section#1{\end{flashcard}\begin{flashcard}{#1}}
   \makeatother

   % Riscrive begin{document} e end{document}
   % così da ovviare allo sfasamento
   \let\olddocument\document
   \let\oldenddocument\enddocument
   \renewenvironment{document}%
   {%
     \olddocument\begin{flashcard}{Prima}\tiny\lipsum[1]
   }{%
     \end{flashcard}\oldenddocument
   }
\end{verbatim}
può essere aggiunto in un file sufficientemente ben formato e sostituisce le sezioni del documento con delle flashcard.
\end{document}
