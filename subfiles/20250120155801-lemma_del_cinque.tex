% Intended LaTeX compiler: pdflatex
\documentclass[../main]{subfiles}


\begin{document}

Sia \(R\) un \href{20241205141119-anello.org}{anello} commutativo con unità, e siano \(H,H',L,L',M,M',N,N',P,P'\) degli \href{20241205141053-r_moduli.org}{\(R\)-moduli}.

\begin{lem}
Si consideri il \href{https://q.uiver.app/\#q=WzAsMTAsWzAsMCwiSCJdLFsxLDAsIkwiXSxbMiwwLCJNIl0sWzMsMCwiTiJdLFs0LDAsIlAiXSxbMCwxLCJIJyJdLFsxLDEsIkwnIl0sWzIsMSwiTSciXSxbMywxLCJOJyJdLFs0LDEsIlAnIl0sWzAsMSwiXFxhbHBoYSJdLFsxLDIsIlxcYmV0YSJdLFsyLDMsIlxcZ2FtbWEiXSxbMyw0LCJcXGRlbHRhIl0sWzUsNiwiXFxhbHBoYSciLDJdLFs2LDcsIlxcYmV0YSciLDJdLFs3LDgsIlxcZ2FtbWEnIiwyXSxbOCw5LCJcXGRlbHRhJyIsMl0sWzAsNSwiXFxldGEiLDAseyJzdHlsZSI6eyJoZWFkIjp7Im5hbWUiOiJlcGkifX19XSxbMSw2LCJcXHZhcnBoaSJdLFsyLDcsIlxccHNpIl0sWzMsOCwiXFxjb25nIl0sWzQsOSwiXFxyaG8iLDIseyJzdHlsZSI6eyJ0YWlsIjp7Im5hbWUiOiJob29rIiwic2lkZSI6InRvcCJ9fX1dLFsxLDYsIlxcY29uZyIsMl0sWzMsOCwiXFx0aGV0YSIsMl1d}{seguente} diagramma commutativo a righe \href{20250120125004-successione_di_r_moduli_esatta.org}{esatte}.
\[\begin{tikzcd}[ampersand replacement=\&]
	H \& L \& M \& N \& P \\
	{H'} \& {L'} \& {M'} \& {N'} \& {P'}
	\arrow["\alpha", from=1-1, to=1-2]
	\arrow["\eta", two heads, from=1-1, to=2-1]
	\arrow["\beta", from=1-2, to=1-3]
	\arrow["\varphi", from=1-2, to=2-2]
	\arrow["\cong"', from=1-2, to=2-2]
	\arrow["\gamma", from=1-3, to=1-4]
	\arrow["\psi", from=1-3, to=2-3]
	\arrow["\delta", from=1-4, to=1-5]
	\arrow["\cong", from=1-4, to=2-4]
	\arrow["\theta"', from=1-4, to=2-4]
	\arrow["\rho"', hook, from=1-5, to=2-5]
	\arrow["{\alpha'}"', from=2-1, to=2-2]
	\arrow["{\beta'}"', from=2-2, to=2-3]
	\arrow["{\gamma'}"', from=2-3, to=2-4]
	\arrow["{\delta'}"', from=2-4, to=2-5]
\end{tikzcd}
\]
Se \(\eta\) è \href{20241213105600-funzione_suriettiva.org}{suriettiva}, \(\rho\) è \href{20241219101956-funzione_iniettiva.org}{iniettiva} e \(\varphi,\theta\) sono \href{20241206115416-morfismi_r_moduli.org}{isomorfismi}, allora \(\psi\) è un \href{20241206115416-morfismi_r_moduli.org}{isomorfismo}.
\end{lem}
\begin{proof}
(per \emph{diagram chasing}).
\begin{itemize}
\item \uline{Suriettività.}

Sia \(m' \in M'\). Allora, siccome \(\theta\) è un isomorfismo, esiste un unico \(n \in N\) tale che
\begin{equation*}
\theta(n) = \gamma'(m')
\end{equation*}

Consideriamo \(\rho\left(\delta(n)\right) \in P'\). Per la commutatività del diagramma
\begin{equation*}
\rho\left(\delta(n)\right) = \delta'\left(\theta(n)\right) = \delta'\left(\gamma'(m')\right) = 0
\end{equation*}
dove l'ultima uguaglianza viene dal fatto che \(\operatorname{ker}\delta'=\operatorname{Im}\gamma'\) per l'esattezza delle righe.

Ma \(\rho\) è iniettivo per ipotesi, pertanto \(\delta(n)= 0\) e quindi \(n \in \operatorname{ker}\delta = \operatorname{Im}\gamma\).

Quindi esiste \(m \in M\) tale che \(n = \gamma(m)\).

Considero ora \(m'-\psi(m)\), e calcolo
\begin{align*}
\gamma'\left(m'-\psi(m)\right) &= \gamma'(m) - \gamma'\psi(m)\\
&= \gamma'(m) - \theta\gamma (m)
\end{align*}
ma \(n = \gamma(m)\) e \(\theta( n) = \gamma'(m')\), e pertanto questo termine vale \(0\).

Dunque \(m'-\psi(m) \in \operatorname{ker}\gamma' = \operatorname{Im}\beta'\), e quindi esiste \(l' \in L'\) tale che
\begin{equation*}
\beta'(l') = m'-\psi(m)
\end{equation*}
e, siccome \(\varphi\) è un isomorfismo, esiste un unico \(l \in L\) tale che
\begin{equation*}
\varphi(l)=l'
\end{equation*}

Pertanto
\begin{equation*}
m'-\psi(m) = \beta'\circ\varphi(l) = \psi\beta(l)
\end{equation*}
e dunque \(m' = \psi(m) + \psi\beta(l)\) e quindi
\begin{equation*}
m' = \psi \left(m + \beta(l)\right)
\end{equation*}
e quindi \(\psi\) suriettiva.

\item \uline{Iniettività}.

Sia \(m \in M\) tale che \(\psi(m)=0\).
\begin{equation*}
0 = \gamma'\psi(m) = \theta\gamma(m)
\end{equation*}

Siccome \(\theta\) è un isomorfismo, allora \(\gamma(m) = 0\) e quindi
\begin{equation*}
m \in \operatorname{ker}\gamma = \operatorname{Im}\beta
\end{equation*}
ed esiste \(\ell \in L\) tale che \(m = \beta(\ell)\).
\begin{equation*}
\beta'\varphi(\ell) = \psi\beta(\ell) = \psi(m) = 0
\end{equation*}
e dunque \(\varphi(\ell) \in \operatorname{ker}\beta' = \operatorname{Im}\alpha'\) e dunque esiste \(h' \in H'\) tale che
\begin{equation*}
\alpha'(h') = \varphi(\ell)
\end{equation*}

Siccome \(\eta\) è suriettiva, allora esiste \(h \in H\) tale che \(\eta(h) = h'\).
\begin{equation*}
\varphi(\ell)=\alpha'(h') = \alpha'\eta(h) = \varphi\alpha(h)
\end{equation*}
e siccome \(\varphi\) è un isomorfismo, allora \(\ell=\alpha(h)\)

Inoltre \(m = \beta(\ell) = \beta\alpha(h)\) ma \(\operatorname{ker}(\beta) = \operatorname{Im}(\alpha)\) e quindi \(m = 0\).
\qedhere
\end{itemize}
\end{proof}
\end{document}
