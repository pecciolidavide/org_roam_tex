% Intended LaTeX compiler: pdflatex
\documentclass[../main]{subfiles}

\usepackage[hyperref]{biblatex}
\date{}
\title{}
\begin{document}

\section{Biiezione canonica tra N e prodotti cartesiani di N}
\label{sec:org70b001f}
Vedi: \href{20250202130045-insieme_dei_numeri_naturali_mk.org}{Insieme dei numeri naturali MK}
\subsection{Caso base per n=2}
\label{sec:orgc94d1e8}
La biiezione
\begin{align*}
{\bm{J}}: \N^{2} &\longrightarrow \N\\
(x,y) &\longmapsto \frac{1}{2}(x+y)(x+y+1) + x
\end{align*}
è \href{20250215141024-funzioni_primitive_ricordive.org}{ricorsiva primitiva}, poiché lo \href{20250215141024-funzioni_primitive_ricordive.org}{sono somma e prodotto}.

Siano inoltre \((\cdot)_{0},(\cdot)_{1} : \N\to \N\) le \href{20250202170607-classe_relazione_binaria.org}{inverse}, ovvero tali che:
\begin{equation*}
\begin{aligned}
\left({\bm{J}}(x,y)\right)_{0} &= x\\
\left({\bm{J}}(x,y)\right)_{1} &= y
\end{aligned}\qquad \bm{J}\left((\xi)_{0},(\xi)_{1}\right)=\xi
\end{equation*}

Queste sono \href{20250215141024-funzioni_primitive_ricordive.org}{funzioni ricorsive primitive} \footnote{È ovvio che \(\bm{J}\) sia ricorsiva primitiva ,\href{20250215171731-quoziente_e_resto_sono_funzioni_ricorsive_primitive.org}{sicché \(\operatorname{Quoz}\) è una funzione ricorsiva primitiva}.. Ora, si consideri a titolo esplicativo \((\cdot)_{0}\).
\((\xi)_0 = x\) se e solo se \(\bm{J}\left(x,(\xi)_{1}\right) = \xi\), se e solo se \(\exists\,y\ \bm{J}(x,y)=\xi\). Infatti: (\(\Rightarrow\)) ovvio; (\(\Leftarrow\)) se esiste \(y_{0}\) tale che \(\bm{J}(x,y_{0})=\xi\), allora per definizione di funzione inversa
\begin{equation*}
\bm{J}(x,y_{0}) = \xi = \bm{J}\left((\xi)_{0},(\xi)_{1}\right)
\end{equation*}
Poiché \(\bm{J}\) è biiettiva (quindi iniettiva) allora \(x=(\xi)_{0}\).
Pertanto
\begin{equation*}
(\xi)_{0}=\minim{x\le \xi}{\minim{y\le \xi}{\bm{J}(x,y)=\xi}}.
\end{equation*}
e quindi è ricorsiva primitiva, poiché questa classe di funzioni \href{20250519112500-proprieta_di_chiusura_delle_funzioni_primitive_ricorsive.org}{è chiusa per minimizzazioni limitiate}.}.
\begin{oss}
Si noti che \({\bm{J}}(x,y)\ge x,y\), ma in particolare, se \((x,y)\neq (0,0)\) e \((x,y) \neq(0,1)\) si ha
\begin{equation*}
{\bm{J}}(x,y)>x,y
\end{equation*}
\end{oss}
\subsection{Generalizzazione}
\label{sec:orgc51b871}
Si definiscono le seguenti biiezioni, per ogni \(k \in \N^{+}\):
\begin{equation*}
{\bm{J}}^{k} : \N^{k}\to \N
\end{equation*}
come:
\begin{align*}
{\bm{J}}^{1}(x_{1}) = x_{1}\\
{\bm{J}}^{k+1}(x_{1},\dots,x_{k+1}) &\coloneqq {\bm{J}}\left(x_{1}, {\bm{J}}^{k}(x_{2},\dots,x_{k+1})\right)
\end{align*}

Si hanno le conseguenti inverse \((\cdot)^{k}_{i}\) per \(1\le i\le k\), definite da, per ogni \(x \in \N\):
\begin{equation*}
{\bm{J}}^{k}\left((x)^{k}_{1}, (x)^{k}_{2},\dots, (x)^{k}_{k}\right)= x
\end{equation*}

È possibile dimostrare che queste siano \href{20250215141024-funzioni_primitive_ricordive.org}{funzioni ricorsive primitive}.

Si noti, inoltre, che \href{20250216162903-funzioni_primitive_ricorsive_in_piu_dimensioni.org}{espanendo la definizione di funzione ricorsiva primitiva},
\begin{equation*}
(\bm{J}^{k})^{-1}: \N\to \N^{k}
\end{equation*}
è una funzione ricorsiva primitiva.
\end{document}
