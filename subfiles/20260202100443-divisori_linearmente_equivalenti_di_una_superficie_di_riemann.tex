% Intended LaTeX compiler: pdflatex
\documentclass[../main]{subfiles}


\begin{document}

\section{Divisori linearmente equivalenti di una superficie di Riemann}
\label{sec:org373cfc8}
Sia \(X\) una \href{20260127112828-superficie_di_riemann.org}{superficie di Riemann}, e si indichi con \(\operatorname{Div}(X)\) il \href{20260201235551-divisore_di_una_superficie_di_riemann.org}{gruppo dei divisori} e con \(\operatorname{PDiv}(X)\) il \href{20241206143051-sottogruppo.org}{sottogruppo} dei \href{20260202095650-divisore_di_una_funzione_meromorfa.org}{divisori principali}.

\begin{definizione}
Due divisori \(D_{1},D_{2} \in \operatorname{Div}(X)\) si dicono \textbf{linearmente equivalenti} (e si scrive \(D_{1}\sim D_{2}\)) se \(D_{1}-D_{2}\) è un \href{20260202095650-divisore_di_una_funzione_meromorfa.org}{divisore principale}.
\end{definizione}

\begin{oss}
Questa definizione si estende naturalmente ai punti di \(X\), tramite il divisore, per ogni \(p_{0} \in X\):
\begin{equation*}
D: X\mapsto \Z: p_{0} \neq p \mapsto 0: p_{0}\mapsto 1.
\end{equation*}
Questo si scrive (come somma formale), come \(D=p_{0}\).
\end{oss}

\begin{prop}
L'equivalenza lineare è una \href{20250113110148-relazione_di_equivalenza.org}{relazione di equivalenza} su \(\operatorname{Div}(X)\), le cui classi di equivalenza sono i \href{20241206143051-sottogruppo.org}{laterali} di \(\operatorname{PDiv}(X)\) in \(\operatorname{Div}(X)\).
\end{prop}
\end{document}
