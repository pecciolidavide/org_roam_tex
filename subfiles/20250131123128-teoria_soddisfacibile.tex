% Intended LaTeX compiler: pdflatex
\documentclass[../main]{subfiles}


\begin{document}

\section{Teoria soddisfacibile}
\label{sec:orga596e0b}
Una teoria \(T\) si dice \uline{soddisfacibile} se esiste un \href{20250131122945-modello_di_un_insieme_di_formule.org}{modello} di \(T\).
\section{Teoria finitamente soddisfacibile}
\label{sec:org0abc71c}
Una teoria \(T\) si dice \uline{finitamente soddisfacibile} se ogni suo sottoinsieme finito è soddisfacibile.
\section{Teoria massimamente soddisfacibile}
\label{sec:org528a553}
Una teoria \(T\) si dice \uline{massimamente soddisfacibile} se è soddisfacibile e non esiste alcuna teoria \(S\) soddisfacibile t.c. \(T \subsetneqq S\).
\subsection{Teorie massimamente soddisfacibili coincidono con la loro chiusura logica}
\label{sec:orgb823102}
Se \(T\) è massimamente soddisfacibile, detta \(T'=\operatorname{ccl}(T)\) la sua \href{20250612110627-chiusura_logica_di_una_teoria.org}{chiusura logica}, si ha
\begin{equation*}
T=T'.
\end{equation*}
\end{document}
