% Intended LaTeX compiler: pdflatex
\documentclass[../main]{subfiles}


\begin{document}

Contesto: \href{20250130104245-morse_kelly_set_theory.org}{Morse Kelly Set Theory}
\section{Ordinale Successore}
\label{sec:org19f38d5}
Un \href{20250203111003-ordinali.org}{ordinale} \(\alpha\) si dice \textbf{successore} se
\begin{equation*}
\alpha = \operatorname{\beta}
\end{equation*}
per qualche \(\beta\). (vedi \href{20250202124648-successore_di_un_insieme_mk.org}{Successore di un insieme MK}).
\subsection{Osservazione}
\label{sec:org082cd49}
Chiaramente \(\alpha < \operatorname{S}(\alpha)\) (vedi \href{20250203111003-ordinali.org}{Relazione d'ordine sugli ordinali}), e non esiste alcun ordinale \(\beta\) tale che
\begin{equation*}
\alpha<\beta<\operatorname{S}(\alpha)
\end{equation*}
\section{Ordinale Limite}
\label{sec:orgabb5a00}
Un \href{20250203111003-ordinali.org}{ordinale} si dice \textbf{limite} se non è un \hyperref[sec:org19f38d5]{ordinale successore} e non è \href{20250202130045-insieme_dei_numeri_naturali_mk.org}{zero}.
\end{document}
