% Intended LaTeX compiler: pdflatex
\documentclass[../main]{subfiles}


\begin{document}

\section{PID Sottomoduli di moduli libero sono liberi}
\label{sec:orgfd9b758}
Sia \(R\) un \href{20241205141119-anello.org}{anello commutativo con unità}.

\begin{thm}
Se \(R\) è un \href{20241219112842-pid.org}{PID} e \(M\) è un \href{20241205141053-r_moduli.org}{\(R\)-modulo} \href{20241213094625-modulo_libero.org}{libero}, allora per ogni \(N \subsetneqq_R M\) \href{20241206142802-sottomoduli.org}{sottomodulo} proprio, \(N\) è \textbf{libero}.

Inoltre, se \(M\) è \href{20241213100845-modulo_finitamente_generato.org}{finitamente generato}, allora anche \(N\) è  \href{20241213100845-modulo_finitamente_generato.org}{finitamente generato}, e vale\footnote{: Vedi \href{20241213101621-teorema_della_base.org}{Teorema-della-Base} e \href{20260109173733-rango_di_un_modulo.org}{Rango di un modulo libero}}:
\begin{equation*}
\operatorname{rg}N<\operatorname{rg}M
\end{equation*}
\end{thm}

\begin{cor}
Se \(R\) è un \href{20241219112842-pid.org}{PID} ogni \href{20241205141053-r_moduli.org}{\(R\)-modulo} è \href{20241206142802-sottomoduli.org}{quoziente} di due \href{20241205141053-r_moduli.org}{\(R\)-moduli} \href{20241213094625-modulo_libero.org}{liberi}.
\end{cor}

\begin{cor}
Se \(R\) è un \href{20241219112842-pid.org}{PID} ogni \href{20241205141053-r_moduli.org}{\(R\)-modulo} \href{20241212141101-generatori_modulo.org}{finitamente generato} è \href{20241206142802-sottomoduli.org}{quoziente} di due \href{20241205141053-r_moduli.org}{\(R\)-moduli} \href{20241213094625-modulo_libero.org}{liberi} \href{20241212141101-generatori_modulo.org}{finitamente generati}.
\end{cor}

I corollari seguono rispettivamente da ``\href{20241213102346-moduli_quoziente_mod_liberi.org}{Ogni modulo è quoziente di modulo libero}'' e ``\href{20241213110005-caratterizzazione_moduli_fg.org}{Caratterizzazione moduli finitamente generati}''.
\end{document}
