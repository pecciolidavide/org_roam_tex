% Intended LaTeX compiler: pdflatex
\documentclass[../main]{subfiles}


\begin{document}

\section{Funzione di scelta}
\label{sec:orgbd1f941}
\begin{enumerate}
\item Sia \(\mathcal{A}\) un insieme non vuoto tale che \(\forall\, A \in \mathcal{A}\ (A\neq \emptyset)\).

Una \uline{funzione di scelta} \textbf{\textbf{per}} \(\mathcal{A}\) è
\begin{equation*}
 f:\mathcal{A} \to \bigcup\mathcal{A}
\end{equation*}
tale che per ogni \(A \in \mathcal{A}\), \(f(A) \in A\).

\item Una \uline{funzione di scelta} \textbf{\textbf{su}} un insieme \(X\neq\emptyset\) è una funzione
\begin{equation*}
 F:\parti{X}\to X
\end{equation*}
tale che \href{20250205170515-restrizione_di_una_classe.org}{\(f\upharpoonright \parti{X}\setminus\set{\emptyset}\)} sia una funzione di scelta per \(\parti{X}\).
\end{enumerate}
\subsection{Funzione di scelta su una classe propria}
\label{sec:orgb4155c4}
Contesto: \href{20250130104245-morse_kelly_set_theory.org}{Morse Kelly Set Theory}

È possibile ampliare il concetto di cui sopra definendo una funzionr di scelta per una \href{20250130104320-classe_mk.org}{classe propria} \(X\) come una \href{20250202170607-classe_relazione_binaria.org}{classe funzione} \(F\) con \href{20250202173528-dominio_range_e_campo_di_una_classe_relazione.org}{dominio}
\begin{equation*}
\set{y\ |\ \emptyset\neq y\subseteq X}
\end{equation*}
(vedi \href{20250131161811-insieme_vuoto_mk.org}{Insieme vuoto MK} e \href{20250131155822-operazioni_insiemistiche_tra_classi_mk.org}{Sottoclasse MK}) e tale che \(F(y) \in y\) per ogni \(y \in \operatorname{dom}F\).
\end{document}
