% Intended LaTeX compiler: pdflatex
\documentclass[../main]{subfiles}


\begin{document}

\section{Fascio delle funzioni meromorfe su una superficie di Riemann}
\label{sec:org4627713}
\begin{definizione}
Sia \(X\) una \href{20260127112828-superficie_di_riemann.org}{superficie di Riemann}. Il \uline{\href{20250324174728-fascio.org}{fascio} delle funzioni meromorfe} su \(X\), denotato con \(\mathcal{M}_{X}\), è definito come:
\begin{itemize}
\item per ogni \(U \subseteq X\): \(\mathcal{M}_{X}(U) = \set{f\text{ meromorfa su }U}\) l'\href{20260128144105-funzione_meromorfa_su_una_superficie_di_riemann.org}{insieme delle funzioni meromorfe}
\item le \href{20250205170515-restrizione_di_una_classe.org}{restrizioni ovvie}.
\end{itemize}
\end{definizione}

\begin{oss}
Per ogni \(U \subseteq X\), \(\mathcal{M}_{X}(U)\) è una \(C\)-algebra.
\end{oss}

\begin{oss}
Il \href{20260128143847-fascio_delle_funzioni_olomorfe_su_una_superficie_di_riemann.org}{fascio \(\mathcal{O}_{X}\)} è un \href{20250325150647-sottoprefascio.org}{sottofascio} di \(\mathcal{M}_{X}\).
\end{oss}

\begin{prop}
Se \(U\) è connesso, allora \(\mathcal{M}_{X}(U)\) è un \href{20241205142049-campo.org}{campo}.
\end{prop}

\begin{proof}
Infatti, sia \(\mathcal{M}_{X}(U)\), \(f\not\equiv 0\). Si definisca
\begin{align*}
Z&\coloneqq \set{\text{zeri di \(f\)}}\\
S&\coloneqq \set{\text{poli di \(f\)}}
\end{align*}

Allora \(f: U \setminus (Z\cup S) \to \C\) è olomorfa mai nulla, e pertanto
\begin{equation*}
\frac{1}{f}:  U \setminus (Z\cup S) \to \C
\end{equation*}
è olomorfa, con \(Z\cup S\) discreto.

Allora \(\frac{1}{f}\) ha zeri in \(S\) e poli in \(Z\)\footnote{Questo lo si vede con la \href{20260128130443-comportamento_locale_di_una_funzione_analitica.org}{scrittura locale in serie di potenze}.}, e quindi è meromorfa.
\end{proof}
\end{document}
