% Intended LaTeX compiler: pdflatex
\documentclass[../main]{subfiles}


\begin{document}

\section{Somma diretta di complessi di catene}
\label{sec:org4107c28}
Sia \(R\) un \href{20241205141119-anello.org}{anello commutativo con unità}.

\begin{definizione}
Sia \(\{\mathcal{C}_{\bullet}^{\alpha}\}_{\alpha \in A}\) una famiglia di \href{20250120163114-complesso_di_catene.org}{complessi di catene} di \href{20241205141053-r_moduli.org}{\(R\)-moduli}, dove \(\mathcal{C}_{\bullet}^{\alpha} = \set{(C_{n}^{\alpha}, \partial_{n}^{\alpha})}_{n \in \Z}\).
La \textbf{\textbf{somma diretta}} dei complessi, denotata con \(\bigoplus_{\alpha \in A} \mathcal{C}_{\bullet}^{\alpha}\), è il complesso di catene \(\mathcal{S}_{\bullet} = \set{(S_{n}, \partial_{n})}_{n \in \Z}\) definito come segue:
\begin{itemize}
\item L'\(n\)-esimo modulo è la \href{20241213095808-somma_diretta.org}{somma diretta dei moduli} della famiglia:
\begin{equation*}
S_{n} = \bigoplus_{\alpha \in A} C_{n}^{\alpha}
\end{equation*}
\item Il differenziale \(\partial_{n}: S_{n} \longrightarrow S_{n-1}\) agisce componente per componente:
\begin{equation*}
\partial_{n}((x_{\alpha})_{\alpha}) = (\partial_{n}^{\alpha}(x_{\alpha}))_{\alpha}
\end{equation*}
\end{itemize}
Nel caso di due complessi \(\mathcal{A}_{\bullet}\) e \(\mathcal{B}_{\bullet}\), la somma diretta si denota con \(\mathcal{A}_{\bullet} \oplus \mathcal{B}_{\bullet}\).
\end{definizione}
\subsection{Somma diretta di morfismi tra complessi di catene}
\label{sec:orgcd6ad94}
\begin{definizione}
Siano \(\{\mathcal{C}_{\bullet}^{\alpha}\}_{\alpha \in I}\) e \(\{\mathcal{D}_{\bullet}^{\alpha}\}_{\alpha \in I}\) due famiglie di \href{20250120163114-complesso_di_catene.org}{complessi di catene} di \href{20241206115416-morfismi_r_moduli.org}{\(R\)-moduli} indicizzate da un insieme \(I\).
Sia \(\{F_{\bullet}^{\alpha}: \mathcal{C}_{\bullet}^{\alpha} \longrightarrow \mathcal{D}_{\bullet}^{\alpha}\}_{\alpha \in I}\) una famiglia di \href{20250120163759-categoria_complessi_di_catene.org}{morfismi di complessi di catene}.

Il morfismo somma diretta, denotato con \(\bigoplus_{\alpha \in I} F_{\bullet}^{\alpha}\), è la mappa tra le \hyperref[sec:org4107c28]{somme dirette dei complessi}:
\begin{equation*}
\Phi_{\bullet} = \bigoplus_{\alpha \in I} F_{\bullet}^{\alpha} : \bigoplus_{\alpha \in I} \mathcal{C}_{\bullet}^{\alpha} \longrightarrow \bigoplus_{\alpha \in I} \mathcal{D}_{\bullet}^{\alpha}
\end{equation*}
definita, per ogni grado \(n \in \Z\), come la \href{20241213095808-somma_diretta.org}{somma diretta dei morfismi} di moduli \(n\)-esimi:
\begin{equation*}
\Phi_{n} = \bigoplus_{\alpha \in I} F_{n}^{\alpha} : \bigoplus_{\alpha \in I} C_{n}^{\alpha} \longrightarrow \bigoplus_{\alpha \in I} D_{n}^{\alpha}
\end{equation*}
\end{definizione}

Questo definisce un morfismo di complessi poiché, per la definizione di differenziale nella somma diretta, si ha la commutatività con i differenziali componente per componente.
\end{document}
