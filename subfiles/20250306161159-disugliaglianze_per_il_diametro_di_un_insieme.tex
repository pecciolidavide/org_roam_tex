% Intended LaTeX compiler: pdflatex
\documentclass[../main]{subfiles}


\begin{document}

Sia \((X,d)\) uno spazio metrico, e si definisca il diametro per ogni \(B \subseteq X\)
\begin{equation*}
\operatorname{diam}(B) \coloneqq \begin{cases}
\sup\set{d(x,y)\ |\ x,y \in B} & B\neq \emptyset\\
0 & B=\emptyset
\end{cases}
\end{equation*}
\section{Proposizione}
\label{sec:org2933ce4}
Siano \(A, B \subseteq X\).
\begin{itemize}
\item Se \(A \subseteq B\) allora \(\operatorname{diam}(A)\le \operatorname{diam}(B)\).
\item Se \(A \subseteq \operatorname{Cl}(B)\) allora \(\operatorname{diam}(A)\le \operatorname{diam}(B)\). (vedi \href{20250103144944-chiusura_topologica.org}{Chiusura Topologica})
\end{itemize}
\subsection{Dimostrazione}
\label{sec:orgbb72ddb}
\end{document}
