% Intended LaTeX compiler: pdflatex
\documentclass[../main]{subfiles}


\begin{document}

\section{Sistema diretto}
\label{sec:orgb601304}
Sia \(\mathcal{C}\) una categoria, e sia \(I\) un \href{20250203101604-ordine.org}{insieme ordinato} \href{20250202184517-ordine_superiormente_diretto.org}{diretto superiorimente}, ovvero
\begin{equation*}
\forall i,j \in I\ \exists k \in I\quad\big(i\le k \land  j\le k\big).
\end{equation*}

\begin{definizione}
Un \textbf{sistema diretto} in \(\mathcal{C}\) è una famiglia \(\set{G_{i}, \uprho_{ij}}\) tale che:
\begin{itemize}
\item \(G_{i} \in \operatorname{Ob}(\mathcal{C})\);
\item per ogni \(j\ge i\): \(\uprho_{ij} : G_{i} \longrightarrow G_{j}\)
\end{itemize}
e inoltre, per ogni \(i \le j \le k\) il seguente diagramma commuta:
\begin{equation*}
\begin{tikzcd}[ampersand replacement=\&]
	{G_i} \&\& {G_k} \\
	\& {G_j}
	\arrow["{\uprho_{ik}}", from=1-1, to=1-3]
	\arrow["{\uprho_{ij}}"', from=1-1, to=2-2]
	\arrow["{\uprho_{jk}}"', from=2-2, to=1-3]
\end{tikzcd}%
\hspace{5em} \uprho_{jk} \circ \uprho_{ij} = \uprho_{ik}.
\end{equation*}
\end{definizione}
\section{Limite diretto}
\label{sec:org0802a13}
\begin{definizione}
Dato \(\set{G_{i}, \uprho_{ij}}\) sistema diretto il \textbf{limite diretto dei \(G_{i}\)} è
\begin{itemize}
\item un oggetto \(G = \varinjlim G_{i}\);
\item dotato di morfismi \(\uprho_{j}:G_{j} \to G\) tali che \(\uprho_{j} \circ \uprho_{ij} = \uprho_{i}\);
\end{itemize}
dotato della seguente proprietà universale:
\begin{itemize}
\item per ogni \(Z \in \operatorname{Ob}(\mathcal{C})\) dotato di mappe \(f_{i}: G_{i}\to Z\) tali che \(f_{j} \circ \uprho_{ij} = f_{i}\), esiste un unico \(f:G\to Z\) tale che \(f\circ\uprho_{j} = f_{j}\).
\end{itemize}
\begin{equation*}
\begin{tikzcd}[ampersand replacement=\&]
	{G_i} \&\&\&\& {G_j} \\
	\&\& G \\
	\\
	\\
	\&\& Z
	\arrow["{\uprho_{ij}}", from=1-1, to=1-5]
	\arrow["{\uprho_i}"{description}, from=1-1, to=2-3]
	\arrow["{f_i}"', from=1-1, to=5-3]
	\arrow["{\uprho_j}"{description}, from=1-5, to=2-3]
	\arrow["{f_j}", from=1-5, to=5-3]
	\arrow["{\exists!\ f}"{description}, dashed, from=2-3, to=5-3]
\end{tikzcd}
\end{equation*}
\end{definizione}

\begin{esempio}
Nella categoria \href{20241205115600-categoria_top.org}{\(\mathcal{C}=\cat{Top}\) degli spazi topologici}, si consideri
\begin{equation*}
X_{i} \subseteq X_{j},\qquad \uprho_{ij}: X_{i} \hookrightarrow X_{j}.
\end{equation*}
Allora \(\varinjlim X_{i} = \bigcup X_{i}\) dotato della ``topologia debole'', ovvero
\begin{equation*}
C \subseteq \bigcup X_{i}\ \text{chiuso} %
\IFF %
\forall i\ C\cap X_{i}\ \text{chiuso}
\end{equation*}
\end{esempio}
\end{document}
