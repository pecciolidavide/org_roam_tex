% Intended LaTeX compiler: pdflatex
\documentclass[../main]{subfiles}

\usepackage[hyperref]{biblatex}
\date{}
\title{}
\begin{document}

\section{Deomogenizzazione}
\label{sec:orgc48bc4d}
Sia \(\K\) un \href{20241231112713-campo_algebricamente_chiuso.org}{Campo Algebricamente Chiuso}, e sia \(F \in \K[x_{0},\dots,x_{n}]\) (vedi \href{20241219113434-anello_dei_polinomi.org}{Anello-dei-polinomi}) un \href{20241231121125-polinomi_omogenei.org}{polinomio omogeneo}. Si definisce una funzione \(\beta_{i}\) che \textbf{deomoegenizza} sulla \(i\)-esima coordinata, e che associa ad \(F\) il polinomio \textbf{non omogeneo}
\begin{equation*}
f(y_{1},\dots,y_{n}) \coloneqq F(y_{1},\dots,1,\dots,y_{n})
\end{equation*}
dove l'\(i\)-esima coordinata di \(F\) è sostituita con l'elemento \(1\) del \href{20241205142049-campo.org}{Campo}
\end{document}
