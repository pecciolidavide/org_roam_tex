% Intended LaTeX compiler: pdflatex
\documentclass[../main]{subfiles}


\begin{document}

\section{Metodo di studio proposto}
\label{sec:org309bf66}
\subsection{Struttura giornaliera in blocchi Pomodoro (50/10)}
\label{sec:org4f2a993}
\begin{itemize}
\item \textbf{Prima mattina}: ripasso (refresh dei concetti già studiati)
\item \textbf{Metà mattina – metà pomeriggio}: studio attivo (definizioni, teoremi, dimostrazioni)
\item Tardo pomeriggio*: recall giornaliero in Org-Roam (ricostruzione a memoria, integrazione)
\end{itemize}
\subsection{Giorni full-focus}
\label{sec:orgf85d04f}
\begin{itemize}
\item Sessioni dedicate esclusivamente a esercizi e recall
\end{itemize}
\section{Criticità e ottimizzazioni}
\label{sec:orgef7e159}

\subsection{Equilibrio e stanchezza}
\label{sec:org95c5c52}

\begin{itemize}
\item \textbf{Rischio}: calo di attenzione nel blocco tardo pomeriggio di 2 ore continue
\item \textbf{Soluzione}: spezzare in 2 × 40–50 min con pause attive
\end{itemize}
\subsection{Spaced repetition integrata}
\label{sec:orgc41eb6d}
\begin{itemize}
\item Quick review a 24h
\item Medium review a 3–4 giorni
\item Review settimanale e super‑review nei giorni full‑focus
\end{itemize}
\subsection{Varietà di task}
\label{sec:org3d02d2e}
\begin{itemize}
\item Alternare esercizi “a freddo” e teach‑back (registrazione audio/video)
\item Macro‑planning settimanale con obiettivi e peer‑review
\end{itemize}
\subsection{Micro-obiettivi e transizioni}
\label{sec:org89c2fa0}
\begin{itemize}
\item Micro-obiettivo per ogni sessione Org‑Roam (es. completare una dimostrazione specifica)
\item Mini-routine di 2 min (respirazione/stretching) per segnalare i cambi di fase
\end{itemize}
\end{document}
