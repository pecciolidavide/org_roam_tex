% Intended LaTeX compiler: pdflatex
\documentclass[../main]{subfiles}


\begin{document}

\section{Forma differenziale chiusa}
\label{sec:orgea7ae62}
\def\d{\operatorname{d}}
\def\H{\operatorname{H}_{\text{dR}}}

Sia \(M\) una \href{20250113115909-struttura_differenziabile.org}{varietà differenziabile}, e sia \(\Omega^{k}(M)\) lo \href{20241205142027-spazio_vettoriale.org}{spazio vettoriale} delle \href{20251115155511-forma_differenziale_in_un_punto.org}{\(k\)-forme}. Si indichi con \(\d\) il \href{20251115160537-differenziale_di_una_forma.org}{differenziale di forme}.

\begin{definizione}
Una forma \(\omega \in \Omega^{k}(M)\) è \uline{chiusa} se \(\dif \omega = 0\)
\end{definizione}
\section{Forma differenziale esatta}
\label{sec:org693f5e9}
\def\d{\operatorname{d}}
\def\H{\operatorname{H}_{\text{dR}}}

Sia \(M\) una \href{20250113115909-struttura_differenziabile.org}{varietà differenziabile}, e sia \(\Omega^{k}(M)\) lo \href{20241205142027-spazio_vettoriale.org}{spazio vettoriale} delle \href{20251115155511-forma_differenziale_in_un_punto.org}{\(k\)-forme}. Si indichi con \(\d\) il \href{20251115160537-differenziale_di_una_forma.org}{differenziale di forme}.

\begin{definizione}
Una forma \(\omega \in \Omega^{k}(M)\) è \uline{esatta} se esiste \(\eta \in \Omega^{k-1}(M)\) tale che \(\dif \eta = \omega\).
\end{definizione}
\end{document}
