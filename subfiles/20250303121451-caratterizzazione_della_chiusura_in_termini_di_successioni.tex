% Intended LaTeX compiler: pdflatex
\documentclass[../main]{subfiles}


\begin{document}

\section{Caratterizzazione della chiusura in termini di successioni}
\label{sec:org00d1235}
\subsection{Teorema}
\label{sec:org013504a}
Sia \(X\) uno \href{20250103145124-topologia.org}{spazio topologico} \href{20250301193254-spazio_topologico_primo_numerabile.org}{primo numerabile}. Siano \(Y \subseteq X\) e \(p \in X\). Sono fatti equivalenti:
\begin{enumerate}
\item esiste una \href{20250115100904-successione.org}{successione} in \(Y\) \href{20250115100930-convergenza_per_una_successione.org}{convergente} a \(p\);
\item \(p\) è un \href{20250303121635-punto_di_accumulazione_di_una_successione.org}{punto di accumulazione} per una successione in \(Y\);
\item \(p \in \operatorname{Cl}_{X}(Y)\) (vedi \href{20250103144944-chiusura_topologica.org}{Chiusura Topologica})
\end{enumerate}
\subsection{Teorema}
\label{sec:orged97d5d}
Sia \(X\) uno \href{20250103145124-topologia.org}{spazio topologico}. Siano \(Y \subseteq X\) e \(p \in X\).
\begin{itemize}
\item Se esiste una \href{20250115100904-successione.org}{successione} in \(Y\) \href{20250115100930-convergenza_per_una_successione.org}{convergente} a \(p\) allora \(p\) è un \href{20250303121635-punto_di_accumulazione_di_una_successione.org}{punto di accumulazione} per una successione in \(Y\).
\item Se \(p\) è un \href{20250303121635-punto_di_accumulazione_di_una_successione.org}{punto di accumulazione} per una successione in \(Y\) allora \(p \in \operatorname{Cl}_{X}(Y)\) (vedi \href{20250103144944-chiusura_topologica.org}{Chiusura Topologica})
\end{itemize}
\end{document}
