% Intended LaTeX compiler: pdflatex
\documentclass[../main]{subfiles}


\begin{document}

Sia \(R\) un \href{20241205141119-anello.org}{anello commutativo con unità}, sia \(X\) lo \href{20250103145124-topologia.org}{spazio topologico} composto da un unico punto \(X=\set{p}\), e sia \(H_{q}(X)\) l'\href{20241205141053-r_moduli.org}{\(R\)-modulo} di \href{20250122133631-omologia_singolare.org}{omologia singolare di \(X\)}.

\begin{prop}
Si ha che
\begin{equation*}
H_{q}(X) = \begin{cases}
R & q = 0\\
0 & q>0
\end{cases}
\end{equation*}
\end{prop}

\begin{proof}
Per ogni \(q \ge 0\), esiste un unico \href{20250122133435-simplesso_singolare.org}{simplesso singolare}:\footnote{Con \(\Delta_{q}\) si indica il \(q\)-\href{20250121122324-simplesso_standard.org}{simplesso standard}.}
\begin{align*}
\sigma: \Delta_{q} &\longrightarrow \set{p}\\
x &\longmapsto p
\end{align*}
che indicheremo con \(\sigma_{n}:\Delta_{q}\to X\). Pertanto, per ogni \(q\), \href{20250122133435-simplesso_singolare.org}{il modulo delle catene singolari} è la \href{20241213095808-somma_diretta.org}{somma diretta}:
\begin{equation*}
S_{q}(X) = R \cdot \sigma_{q} \cong R.
\end{equation*}

Il \href{20250122133614-mappa_di_bordo_tra_moduli_di_catene_singolari.org}{morfismo di bordo}, invece, \(\partial_{q}: S_{q}(X) \to S_{q-1}(X)\), per \(q>0\):
\begin{align*}
\partial_{q} \sigma_{q} = \sum_{i=0}^{q} (-1)^{i} \parentesi{\sigma_{q-1}}{\sigma_{q} \circ \varepsilon_{i}^{q}} = \sum_{i=0}^{q} (-1)^{i} \sigma_{q-1}
\end{align*}
e quindi:
\begin{equation*}
\partial_{q} \sigma_{q} = \begin{cases}
0 & q \text{ dispari}\\
\sigma_{q-1} & q \text{ pari}
\end{cases}
\end{equation*}
La situazione è la seguente:
\begin{equation*}
\begin{tikzcd}
	{S_{2q}(X)} & {S_{2q-1}} & \cdots & {S_2(X)} & {S_1(X)} & {S_0(X)} & 0 \\
	R & R & \cdots & R & R & R & 0
	\arrow["{\partial_{2q}}", from=1-1, to=1-2]
	\arrow["{\partial_{2q-1}}", from=1-2, to=1-3]
	\arrow["{\partial_3}", from=1-3, to=1-4]
	\arrow["{\partial_2}", from=1-4, to=1-5]
	\arrow["{\partial_1}", from=1-5, to=1-6]
	\arrow["{\partial_0}", from=1-6, to=1-7]
	\arrow["\cong"', from=2-1, to=2-2]
	\arrow["0"', from=2-2, to=2-3]
	\arrow["0"', from=2-3, to=2-4]
	\arrow["\cong"', from=2-4, to=2-5]
	\arrow["0"', from=2-5, to=2-6]
	\arrow["0"', from=2-6, to=2-7]
\end{tikzcd}
\end{equation*}
e pertanto\footnote{Vedi:
\begin{itemize}
\item \href{20241206142802-sottomoduli.org}{Quoziente di moduli}
\item \href{20241213105201-kernel.org}{Kernel}
\item \href{20250202190147-immagine_punto_a_punto_di_due_classi.org}{Immagine e retroimmagine tramite una funzione}
\end{itemize}}
\begin{align*}
H_{0}(X) &= \frac{\ker \partial_{0}}{\operatorname{Im} \partial_{1}} = \frac{R}{\set{0}} \cong R\\
H_{2q}(X) &= \frac{\ker \partial_{2q}}{\operatorname{Im} \partial_{2q+1}} = \frac{\set{0}}{\set{0}} \cong 0\\
H_{2q-1}(X) &= \frac{\ker \partial_{2q-1}}{\operatorname{Im} \partial_{2q}} = \frac{R}{R}\cong 0.
\qedhere
\end{align*}
\end{proof}
\end{document}
