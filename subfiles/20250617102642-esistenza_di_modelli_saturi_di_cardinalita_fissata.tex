% Intended LaTeX compiler: pdflatex
\documentclass[../main]{subfiles}

\usepackage[hyperref]{biblatex}
\date{}
\title{}
\begin{document}

\section{Esistenza di modelli saturi di cardinalità fissata}
\label{sec:org99b9bc3}
Si utilizza la \href{20250612143636-notazione_teoria_dei_modelli.org}{Notazione della TEORIA DEI MODELLI}
\subsection{Teorema}
\label{sec:org95f507c}

Sia \(\lambda\) un \href{20250203161341-cardinali.org}{cardinale} tale che \(\lambda=\lambda^{<\lambda}\). Sia \(\mathcal{L}\) un \href{20250130162057-linguaggio_del_prim_ordine.org}{linguaggio del prim'ordine} tale che \(\card{\mathcal{L}}<\lambda\).

Allora per ogni \(\mathcal{L}\)-\href{20250131103035-struttura_del_prim_ordine.org}{struttura} \(M\) di \href{20241213101756-cardinalita.org}{cardinalità} \(\le\lambda\) esiste una \(\mathcal{L}\)-struttura \(N\) tale che:
\begin{itemize}
\item \(M\preceq N\) è \href{20250212102253-sottostruttura_elementare.org}{sottostruttura elementare};
\item \(\card{N} = \lambda\);
\item \(N\) è \href{20250617095548-modello_lambda_saturo.org}{satura}.
\end{itemize}
\end{document}
