% Intended LaTeX compiler: pdflatex
\documentclass[../main]{subfiles}


\begin{document}

\section{Teorema di Hindman}
\label{sec:orgf92400e}
\begin{thm}
Per ogni \href{20251111150142-colorazione.org}{colorazione finita} di \href{20250202130045-insieme_dei_numeri_naturali_mk.org}{\(\N\)} esiste un \href{20250130104331-insieme_mk.org}{insieme} \href{20250205120448-classe_finita_e_infinita_mk.org}{infinito} \(H \subseteq \N\) tale che
\begin{equation*}
\set{\sum_{a \in A} a \mid A \subseteq H\text{ finito}}
\end{equation*}
è \href{20251111150142-colorazione.org}{monocromatico}.
\end{thm}
\begin{oss}
Questo teorema dice ``quasi'' che esiste un insieme monocromatico chiuso per somma. Infatti, mancano le somme di lunghezza arbitraria dello stesso elemento.
\end{oss}
\def\U{\mathcal{U}}
\def\L{\mathcal{L}}
%
\def\eq{{\rm eq}}
\def\Ueq{\U^\eq}
%
\def\orbita{\mathcal{O}}
\def\Aut{\operatorname{Aut}}
%
\def\tc{\mid}
\def\tp{\operatorname{tp}}
\def\EMtp{\operatorname{EM}\text{-}\operatorname{tp}}
\def\<{\langle}
\def\>{\rangle}
%
\def\restricted#1{\,\mathord{\upharpoonright}{{\scriptstyle #1}}}
\def\equivalentover#1{\mathrel{\equiv_{ #1 }}}

%% NON FORKING
\def\nonforkSymbol{%
\mathbin{\raise1.8ex%
\rlap{\kern0.6ex\rule{0.6ex}{0.1ex}}%
\rlap{\kern1.1ex\rule{0.1ex}{1.9ex}}\raise-0.3ex\hbox{$\smile$}}}
\def\defaultnonforkmodel{M}
\def\nonfork{\nonforkSymbol}
\renewcommand{\nonfork}[1][\defaultnonforkmodel]{%
\mathrel{\nonforkSymbol_{#1}}}

\def\defaultnonforkmodel{\N}
\def\fp{\operatorname{fp}}
\begin{proof}
Si utilizza la ``\href{20251104154427-estensione_di_un_semigruppo_ad_un_modello_mostro.org}{Estensione semigruppo a Modello Mostro}''

Sia
\begin{equation*}
\mathcal{A} \coloneqq \set{a \in \U\mid a\neq 0}
\end{equation*}
che è definibile e idem-potente. Quindi contiene \(b_{0}\) tale che \(\orbita(b_{0}/\N)\) è idempotente.

Sia \(\overline{b} = \langle b_{i} : i<\omega \rangle\) sequenza di coeredi su \(N\).

Si noti che \(\fp(\overline{b})\) è monocromatica (ha colore 1). Infatti
\begin{equation*}
b_{0}\cdot b_{1} \equivalentover{\N} b_{0},\qquad
b_{0}\cdot b_{1} \equivalentover{\N} b_{i}
\end{equation*}
poiché \(b_{1}\nonfork b_{0}\).

Definiamo per induzione una sequenza \(\overline{a} = \langle a_{i}: i<\omega\rangle\) tale che
\begin{equation*}
\fp(\overline{a} \restricted{n}, b_{1},b_{0})
\end{equation*}
ha colore \(1\). (definibile da \(\varphi(a\restricted{n}, b_{1},b_{0})\))

Lo assumo per ipotesi induttiva. Voglio trovare \(a_{n}\). Siccome \(b_{1}\nonfork b_{0}\) allora esiste \(a_{n} \in \N\) tale che
\begin{equation*}
\varphi(a \restricted{n}, a_{n}, b_{0})
\end{equation*}
e quindi \(\fp(a \restricted{n}, a_{n}, b_{0})\) ha colore 1.
Voglio arrivare a mostrare che \(\fp(a \restricted{n}, a_{n}, b_{1}, b_{0})\) ha colore 1.

Siccome \(b_{1}\equivalentover{\N} b_{0}\), sicuramente
\begin{equation*}
a_{i_{1}}\cdot \dots a_{i_{n}} \cdot b_{0} \text{%
ha lo stesso colore di %
}a_{i_{1}}\cdot \dots a_{i_{n}} \cdot b_{1}
\end{equation*}
Se per assurdo \(a_{i_{1}}\cdot \dots a_{i_{n}} \cdot b_{1}\cdot b_{0}\) ha colore diverso, siccome \(b_{1}\nonfork b_{0}\) ho
\begin{equation*}
b_{1}\cdot b_{0} \equivalentover{\N} b_{0}
\end{equation*}
per idempotenza quindi anche
\(a_{i_{1}}\cdot \dots a_{i_{n}} \cdot b_{0}\)
ha un colore diverso. Assurdo.
\end{proof}
\end{document}
