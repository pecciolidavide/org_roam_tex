% Intended LaTeX compiler: pdflatex
\documentclass[../main]{subfiles}


\begin{document}

\section{Lemma del diagramma elementare}
\label{sec:org723003d}
\def\tp{\operatorname{tp}}

Si utilizza la \href{20250612143636-notazione_teoria_dei_modelli.org}{Notazione della TEORIA DEI MODELLI}

Vedi Lemma~5.6 di 

\begin{cor}
Sia \(M\) è un \href{20250131103035-struttura_del_prim_ordine.org}{modello} infinito, \(c\) è una sua \href{20250203133527-insiemi_ben_ordinati_sono_isomorfi_ad_un_ordinale_unico.org}{enumerazione}. Se\footnote{notazione:
\begin{itemize}
\item con ``\(\vDash\)'' si indica la ``\href{20250212164424-tipo_teoria_dei_modelli.org}{Soddisfazione di un tipo}'';
\item con \(\tp_{M}(c)\) si indica il ``\href{20250212164424-tipo_teoria_dei_modelli.org}{Tipo di un elemento di una struttura}''
\end{itemize}}
\begin{equation*}
M,d \vDash q(z) \coloneqq \tp_{M}(c)
\end{equation*}
allora, detta \(N = \operatorname{rng} d\), si ha che \(N\) è \href{20250212102253-sottostruttura_elementare.org}{sottostruttura elementare} di \(M\).
\end{cor}
\end{document}
