% Intended LaTeX compiler: pdflatex
\documentclass[../main]{subfiles}


\begin{document}

\section{Sezioni globali del fascio associato ad un divisore su una superficie di Riemann}
\label{sec:org862ec5f}
Sia \(X\) una \href{20260127112828-superficie_di_riemann.org}{superficie di Riemann}, e sia \(\operatorname{Div}(X)\) il \href{20260201235551-divisore_di_una_superficie_di_riemann.org}{gruppo dei divisori di \(X\)}. Fissiamo \(D \in \operatorname{Div}(X)\), e si consideri il \href{20250324174728-fascio.org}{fascio} \(\mathcal{O}_{X}(D)\) \href{20260202101128-fascio_associato_ad_un_divisore_su_una_superficie_di_riemann.org}{associato} a \(D\).

\begin{definizione}
Si denota \(\mathcal{O}_{X}(D)(X)\) la \href{20250324165349-prefascio.org}{sezione globale} di \(\mathcal{O}_{X}(D)\) con
\begin{equation*}
L(D) = \mathcal{O}_{X}(D)(X) %
= \mathcal{O}(D)(X) %
= H^{0}\big(X, \mathcal{O}_{X}(D)\big)
\end{equation*}
\end{definizione}

\begin{oss}
Si ha che \(L(D)\) è un \(\C\)-sottospazio vettoriale di \(\mathcal{M}_{X}(X)\)\footnote{Con \(\mathcal{M}_{X}\) si indica il \href{20260128144151-fascio_delle_funzioni_meromorfe_su_una_superficie_di_riemann.org}{fascio delle funzioni meromorfe a valori in \(\C\)}.}
\end{oss}
\subsection{Dimensione della sezione globale del fascio associato ad un divisore su una superficie di Riemann compatta}
\label{sec:org23f843a}
\begin{lem}
Sia \(X\) \href{20250103163701-spazio_topologico_compatto.org}{compatta}, \(D \in \operatorname{Div}(X)\) di \href{20260201235551-divisore_di_una_superficie_di_riemann.org}{grado} \(\deg D < 0\). Allora \(L(D) = \set{0}\).
\label{lem:finitezza}
\end{lem}
\begin{proof}
Sia per assurdo \(f \in L(D)\) diversa da 0. Allora \(f \in \mathcal{M}_{X}(X)\setminus \set{0}\) è tale che\footnote{con ``\(\ge 0\)'' si intende che è un \href{20260202100903-divisore_effettivo_di_una_superficie_di_riemann.org}{divisore effettivo}.}
\begin{equation*}
\operatorname{div}(f) + D \ge 0 \IMPLICA \deg\bigg(\operatorname{div}(f) + D\bigg)\ge 0
\end{equation*}
Ma il \href{20260201235551-divisore_di_una_superficie_di_riemann.org}{grado} è omomorfismo di gruppi, e quindi
\begin{equation*}
0 \le \deg\bigg(\operatorname{div}(f) + D\bigg) = \deg \operatorname{div} (f) + \deg D
\end{equation*}
e \href{20260202100511-gruppo_di_picard_di_una_superficie_di_riemann.org}{inoltre \(\deg \operatorname{div} f = 0\)}.  Quindi \(\deg D \ge 0\).
\end{proof}

\begin{oss}
Se \(X\) è compatto, \href{20260202173818-successione_esatta_corta_di_fasci_data_da_un_divisore_e_un_punto_di_una_superficie_di_riemann.org}{si ottiene} una \href{20250120131527-sec.org}{sequenza esatta corta di \(\C\)-spazi vettoriali}:
\begin{equation*}
\begin{tikzcd}[ampersand replacement=\&]
	0 \& {L(D-p)} \& {L(D)} \& \C \& 0
	\arrow[from=1-1, to=1-2]
	\arrow[hook, from=1-2, to=1-3]
	\arrow["\alpha", from=1-3, to=1-4]
	\arrow[from=1-4, to=1-5]
\end{tikzcd}
\end{equation*}
Questo implica che \(\ker\alpha \cong L(D-p)\).

Poiché \(\dim_{\C}\C = 1\) e \(\alpha\) è \(\C\)-lineare, ci sono solo due possibilità per l'\href{20250202190147-immagine_punto_a_punto_di_due_classi.org}{immagine di \(\alpha\)}:
\begin{itemize}
\item \(\dim\operatorname{Im}\alpha = 0\): per il teorema di nullità più rango,
\begin{equation*}
  L(D) \cong L(D-p).
\end{equation*}
\item \(\dim\operatorname{Im}\alpha = 1\): allora \(\alpha\) è suriettivo, e pertanto, per il \href{20250120155457-morfismo_iniettivo_di_r_moduli_induce_isomorfismo.org}{primo teorema di isomorfismo}, il \href{20251121143644-quoziente_di_spazi_vettoriali.org}{quoziente} è \href{20250113125833-isomorfismo_tra_spazi_vettoriali.org}{isomorfo}
\begin{equation*}
  \frac{L(D)}{L(D-p)} \cong \C
\end{equation*}
\end{itemize}
In entrambi i casi, questo implica che \(L(D)\) ha \(\C\)-dimensione finita se e solo se \(L(D-p)\) ha \(\C\)-dimensione finita\footnote{Il caso in cui \(L(D) \cong L(D-p)\) è ovvio. Inoltre, \href{20260202101128-fascio_associato_ad_un_divisore_su_una_superficie_di_riemann.org}{siccome} \(L(D-p) \subseteq L(D)\) è \href{20250114103118-sottospazio_vettoriale.org}{sottospazio vettoriale}, ovviamente se \(\dim L(D) <+\infty\) allora \(\dim L(D-p)<+\infty\). Viceversa, se \(L(D)/L(D-p) \cong \C\), e \(L(D-p)\) ha dimensione finita, allora necessariamente \(L(D)\) ha dimensione finita. \href{20251121143644-quoziente_di_spazi_vettoriali.org}{Altrimenti}, \(\dim \C = \infty\), assurdo.}.
\end{oss}

\begin{cor}
Data \(X\) superficie di Riemann compatta, ci sono solo due possibilità:
\begin{enumerate}
\item per ogni \(D \in \operatorname{Div}(X)\) lo spazio \(L(D)\) ha \(\C\)-dimensione finita;
\item per ogni \(D \in \operatorname{Div}(X)\) lo spazio \(L(D)\) ha \(\C\)-dimensione infinita.
\end{enumerate}
\end{cor}
\begin{proof}
Supponiamo per assurdo esistano \(D\) e \(D_{\infty}\) divisori di \(X\) tali che
\begin{equation*}
\dim L(D) < \infty,\qquad \dim L(D_{\infty}) = \infty.
\end{equation*}

Poiché \(X\) è compatto, allora \(\operatorname{supp}D\) e \(\operatorname{supp}D_{\infty}\), insiemi \href{20260128123515-sottoinsieme_discreto.org}{discreti}, \href{20260128172415-sottoinsieme_discreto_in_un_compatto.org}{sono finiti}. Pertanto \(D_{\infty}\) si può ottenere da \(D\) aggiungendo e togliendo un numero finito di punti. Per l'osservazione precedente, allora, \(D_{\infty}\) deve avere dimensione finita.
\end{proof}

\begin{thm}
Se \(X\) è compatto, per ogni \(D \in \operatorname{Div}(X)\) si ha che
\begin{equation*}
\dim_{\C} L(D) < + \infty
\end{equation*}
\end{thm}
\begin{proof}
Per il Lemma~\ref{lem:finitezza}, esistono divisori di \(X\) tali che \(\dim L(D) < \infty\). Pertanto, per il corollario precedente, per ogni \(D \in \operatorname{Div}(X)\): \(\dim L(D) < \infty\).
\end{proof}
\end{document}
