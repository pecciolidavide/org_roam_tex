% Intended LaTeX compiler: pdflatex
\documentclass[../main]{subfiles}


\begin{document}

\section{Esempi di morfismi di prefasci}
\label{sec:orgb421919}
\subsection{Inclusioni di sottoprefasci sono morfismi di fasci}
\label{sec:org79855c3}
\begin{prop}
Se \(\mathcal{F}, \mathcal{G}\) sono \href{20250324165349-prefascio.org}{prefasci} e \(\mathcal{F} \subseteq \mathcal{G}\) è \href{20250325150647-sottoprefascio.org}{sottoprefascio}, allora l'inclusione \(i:\mathcal{F} \to \mathcal{G}\) è un \href{20250325180613-morfismo_di_prefasci.org}{morfismo}.
\end{prop}
\subsection{Differenziale di una forma è morfismo di fasci}
\label{sec:orgfbbb09b}
\begin{prop}
Se \(X\) è \href{20250113115909-struttura_differenziabile.org}{varietà differenziabile} e \(\mathcal{A}_{X}^{p}\), \(\mathcal{A}_{X}^{p+1}\) sono i \href{20250324174728-fascio.org}{fasci} \href{20250324175845-esempi_di_prefasci.org}{delle \(p\)-forme differenziali su \(X\)}\footnote{Vedi ``\href{20251115155511-forma_differenziale_in_un_punto.org}{Forma differenziale}''}, allora il \href{20251115160537-differenziale_di_una_forma.org}{differenziale \(\dif\)} è un \href{20250325180613-morfismo_di_prefasci.org}{morfismo di fasci}: per ogni \(U \subseteq X\):
\begin{align*}
\mathrm{d}_{U}: \mathcal{A}_{X}^{p}(U) &\longrightarrow \mathcal{A}_{X}^{p+1}(U)\\
\omega &\longmapsto \dif\omega.
\end{align*}
\end{prop}

\begin{oss}
Se \(p=0\), allora il \href{20250327114922-fascio_nucleo.org}{fascio nucleo} \(\ker\mathrm{d} = \uline{\R}\) \href{20250325171002-fascio_di_gruppo_localmente_costante.org}{fascio localmente costante}.

Se \(p\ge 1\), allora il fascio nucleo è composto dalle \href{20251115155511-forma_differenziale_in_un_punto.org}{\(p\)-forme} \href{20251115172517-forma_differenziale_chiusa.org}{chiuse} su \(X\).
\end{oss}
\subsection{Valutazione in un punto}
\label{sec:org37d2e53}

Sia \(X\) spazio topologico, e si considerino:
\begin{itemize}
\item \(\mathcal{C}_{X}\) il fascio delle funzioni continue a valori in \(\R\)\footnote{Si trova in ``\href{20250324175845-esempi_di_prefasci.org}{Esempi di Fasci}''}.
\item \(\R_{p}\) il \href{20250325171249-fascio_grattacielo.org}{fascio grattacielo concentrato in \(p \in X\)}.
\end{itemize}

\begin{prop}
La mappa \(e_{p}: \mathcal{C}_{X} \to \R_{p}\), definita come, per ogni \(U \subseteq X\) aperto:
\begin{itemize}
\item se \(p \notin U\) allora \((e_{p})_{U} : \mathcal{C}_{X}(U) \to \set{0}\) mappa nulla;
\item se \(p \in U\) allora
\begin{align*}
(e_{p})_{U}: \mathcal{C}_{X}(U) &\longrightarrow \R\\
f &\longmapsto f(p)
\end{align*}
\end{itemize}
è detta \uline{valutazione in \(p\)} ed è un \href{20250325180613-morfismo_di_prefasci.org}{morfismo di fasci}.
\end{prop}

\begin{oss}
Il \href{20250327114922-fascio_nucleo.org}{nucleo} di questo morfismo è
\begin{equation*}
\ker e_{p} = \set{f: X\to \R \mid f(p) = 0,\ f\text{ continua}}.
\end{equation*}
\end{oss}
\subsection{Mappa esponenziale tra fasci di funzioni olomorfe}
\label{sec:org8ebe275}

Vedi \href{20260127095157-kkk.org}{Mappa esponenziale tra fasci di funzioni olomorfe}
\end{document}
