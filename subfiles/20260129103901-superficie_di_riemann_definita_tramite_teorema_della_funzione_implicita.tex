% Intended LaTeX compiler: pdflatex
\documentclass[../main]{subfiles}


\begin{document}

\section{Superficie di Riemann definita tramite Teorema della funzione implicita}
\label{sec:org744975b}
Sia \(\Omega \subseteq \C^{2}_{z,w}\) un \href{20250103145124-topologia.org}{aperto}, e sia \(F:\Omega\to \C\) \href{20260126110551-funzione_olomorfa.org}{olomorfa}. Si definisce
\begin{equation*}
X\coloneqq \set{p \in \Omega \mid F(p) = 0}.
\end{equation*}
Sia \(p_{0} = (z_{0},w_{0}) \in X\) tale che \(\pd{F}{w}(p_{0}) \neq 0\)\footnote{Vedi ``\href{20250114103236-derivata_parziale.org}{Derivata parziale}''}.

\begin{prop}
Se
\begin{enumerate}
\item \(X\) è connesso;
\item per ogni \(p \in X\), il \href{20250624171244-gradiente_di_una_funzione.org}{gradiente} \(\nabla F (p) \neq \bm{0}\);
\end{enumerate}
allora \(X\) è una \href{20260127112828-superficie_di_riemann.org}{superficie di Riemann}.
\end{prop}

\begin{proof}
Per ogni \(p \in X\), per la condizione 2., almeno una delle proiezioni sugli assi è \href{20250111142332-omeomorfismo.org}{omeo} locale, che dà una \href{20260127112715-atlante_complesso.org}{carta locale}. I cambiamenti di coordinate sono dati da:
\begin{itemize}
\item la funzione identità;
\item \(w=f(z)\) \href{20260126110551-funzione_olomorfa.org}{olomorfismo}.
\qedhere
\end{itemize}
\end{proof}
\end{document}
