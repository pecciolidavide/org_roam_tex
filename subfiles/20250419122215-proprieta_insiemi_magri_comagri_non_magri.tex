% Intended LaTeX compiler: pdflatex
\documentclass[../main]{subfiles}


\begin{document}

\section{Proprietà insiemi magri, comagri, non magri}
\label{sec:orgbd04b86}
\subsection{Proposizione}
\label{sec:orgf33bb8f}
Prove that for every topological space \(X\), the following are equivalent:
\begin{enumerate}
\item Every nonempty open subset of \(X\) is non-meager.
\item Every comeager set in \(X\) is dense.
\item The intersection of countably many dense open subsets of \(X\) is dense.
\end{enumerate}
\subsubsection{Dimostrazione}
\label{sec:org1004caa}

\paragraph{a. implica b.}
\label{sec:org7367909}

Sia \(A \subseteq X\) un insieme comagro: pertanto \(X\setminus A\) è magro. Se per assurdo \(A\) non è denso, allora esiste \(U \subseteq X\) aperto tale che \(A\cap U = \emptyset\), ovvero \(U \subseteq X\setminus A\).

Dunque \(U\) è sottoinsieme di un magro, e pertanto è magro. Assurdo.
\paragraph{b. implica c.}
\label{sec:org11c898e}

Sia \(\set{U_{n}\mid n \in \omega}\) una collezione di aperti densi di \(X\), e sia, per ogni \(n \in\omega\), \(F_{n} \coloneqq X\setminus U_{n}\).

Per la caratterizzazione di cui sopra, gli \(F_{n}\) sono mai densi, e pertanto \(\bigcup_{n \in \omega} F_{n}\) è magro per definizione

Siccome
\[
X \setminus \bigcap_{n \in \omega} U_n = \bigcup_{n \in \omega} F_n
\]
allora \(\bigcap_{n \in \omega} U_{n}\) è comagro, e quindi è denso.
\paragraph{c. implica a.}
\label{sec:orgd4c6ff1}

Sia \(U \subseteq X\) aperto, magro. Per definizione, allora, esistono, per ogni \(n \in\omega\), \(A_{n} \subseteq X\) mai densi tali che
\begin{equation*}
U = \bigcup_{n \in \omega} A_{n}.
\end{equation*}

Sia quindi \(B_{n} \coloneqq X\setminus \operatorname{Cl}_{X}(A_{n})\): questo è aperto poiché complementare di un chiuso, ed è denso, in quanto il suo complementare ha interno vuoto (per la caratterizzazione dell'esercizio precedente).

Pertanto \(\bigcap_{n \in \omega} B_{n}\) è denso. Inoltre
\begin{equation*}
\bigcap_{n \in \omega} B_{n} = X\setminus \bigcup_{n \in\omega} \operatorname{Cl}_{X}(A_{n}) \subseteq X\setminus A_{n} = X\setminus U
\end{equation*}

Pertanto \(U\cap \left(\bigcap_{n \in \omega} B_{n}\right) = \emptyset\). Siccome \(\bigcap_{n \in\omega} B_{n}\) è denso, allora \(U=\emptyset\).

Segie che ogni aperto non vuoto di \(X\) è non magro.\qed
\end{document}
