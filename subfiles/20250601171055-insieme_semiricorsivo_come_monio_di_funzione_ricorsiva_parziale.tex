% Intended LaTeX compiler: pdflatex
\documentclass[../main]{subfiles}

\usepackage[hyperref]{biblatex}
\date{}
\title{}
\begin{document}

\section{Insieme semiricorsivo come dominio di funzione ricorsiva parziale}
\label{sec:org36f5b95}
In questa sezione si identificano i \href{20250131155822-operazioni_insiemistiche_tra_classi_mk.org}{sottinsiemi} di \(\N^{k}\) (vedi \href{20250202130045-insieme_dei_numeri_naturali_mk.org}{Insieme dei numeri naturali MK}) con i \href{20250131103317-formula_del_prim_ordine.org}{predicati} \(k\)-ari (ovvero con \(k\) \href{20250131103429-variabile_libera_di_una_formula.org}{variabili libere}), per mezzo degli \href{20250131122913-soddisfazione_di_una_formula.org}{insiemi di verità}.

Scriveremo indifferentemente \((x_{1},\dots,x_{k}) \in P\) oppure \(P(x_{1},\dots,x_{k})\) per dire che \(\N\vDash P(x_{1},\dots,x_{k})\).
\subsection{Proposizione}
\label{sec:org07ada07}
Un \href{20250130104331-insieme_mk.org}{insieme} \(P \subseteq \N^{k}\) è \href{20250520113238-insieme_semiricorsivo.org}{semiricorsivo} se e solo se è \href{20250202173528-dominio_range_e_campo_di_una_classe_relazione.org}{dominio} di qualche \href{20250207104855-funzione_ricorsiva.org}{funzione ricorsiva} \href{20250213105339-funzione_parziale.org}{parziale} \(f:\N^{k}\to \N\):
\begin{equation*}
P=\operatorname{dom}(f).
\end{equation*}
\end{document}
