% Intended LaTeX compiler: pdflatex
\documentclass[../main]{subfiles}


\begin{document}

\begin{definizione}
Sia \(K\) un \href{20250121124147-complesso_simpliciale.org}{complesso simpliciale} \href{20250121131005-complesso_simpliciale_totalmente_ordinato.org}{totalmente ordinato} e sia \(R\) un \href{20241219112842-pid.org}{PID}.

Allora, per ogni \(q \in \N\), se \(K^{q}\) è il \(q\)-\href{20250121124310-scheletro_di_un_complesso_simpliciale.org}{scheletro} di \(K\), si definisce il \href{20241205141053-r_moduli.org}{modulo} delle catene simpliciali:\footnote{Vedi ``\href{20241213095808-somma_diretta.org}{Somma Diretta}''}
\begin{equation*}
C_{q}(K) = R^{(K^{q})} = \set{\sum a_{i}\ \sigma_{i}: a_{i} \in R, \sigma_{i} \in K^{q}}
\end{equation*}
dove le somma sono formali:
\begin{equation*}
\sum a_{i}\sigma_{i} = (a_{i})_{\sigma_{i} \in K^{q}},
\end{equation*}
\end{definizione}

\uline{Notazione}.
Se \(\tau\) è una \href{20250120123610-permutazione.org}{permutazione} degli elementi \((0,\dots,q)\) allora si indica con:
\begin{equation*}
[p_{\tau(0)},\dots,p_{\tau(q)}] \coloneqq \operatorname{sgn}(\tau)\, [p_{0},\dots,p_{q}]
\end{equation*}
\end{document}
