% Intended LaTeX compiler: pdflatex
\documentclass[../main]{subfiles}

\usepackage[hyperref]{biblatex}
\date{}
\title{}
\begin{document}

\section{Differenziale di una funzione tra varietà differenziabili}
\label{sec:orgc8e23ef}
Siano \(M, N\) due \href{20250113115909-struttura_differenziabile.org}{varietà differenziabili}, e sia \(F:M\longrightarrow N\) una \href{20250113144722-funzioni_cinfinito_tra_varieta_differenziabili.org}{funzione \(C^{\infty}\)}, e sia \(p \in M\).
\subsection{Definizione}
\label{sec:org9e113e5}
Il \textbf{differenziale} di \(F\) in \(p\) è l'applicazione (vedi \href{20250114102823-spazio_tangente_ad_un_punto_di_una_varieta_differenziabile.org}{Spazio tangente ad un punto di una varietà differenziabile})
\begin{equation*}
\restriction{F_{\star}}{p}: \operatorname{T}_{p}M\longrightarrow \operatorname{T}_{F(p)}N
\end{equation*}
che per ogni \(f_{F(p)} \in C^{\infty}_{F(p)}(N)\) (vedi \href{20250114100254-germi_di_funzioni.org}{Germi di funzioni}) e per ogni \(\bm{v} \in \operatorname{T}_{p} M\) (vedi \href{20250114101437-derivazione_su_una_varieta_differenziabile.org}{Derivazione su una varietà differenziabile}):
\begin{equation*}
\restriction{F_{\star}}{p}(\bm{v})\left(f_{F(p)}\right) \coloneqq v\left[(f\circ F)_{p}\right]
\end{equation*}
\subsection{Proposizione}
\label{sec:orgacbff01}
\(\restriction{F_{\star}}{p}\) è ben definito e \href{20250114101949-funzione_lineare.org}{lineare}.
\subsubsection{{\bfseries\sffamily TODO} Dimostrazione\hfill{}\textsc{matematica\_lm:geo\_diff}}
\label{sec:org136ce52}
\subsection{{\bfseries\sffamily TODO} Osservazione\hfill{}\textsc{matematica\_lm:geo\_diff}}
\label{sec:org50d0dd9}
Se \(f:M\longrightarrow M\) è l'identità, allora il suo differenziale
\begin{equation*}
\restriction{f_{\star}}{p}:\operatorname{T}_{p}M\longrightarrow \operatorname{T}_{p}M
\end{equation*}
è l'identità
\end{document}
