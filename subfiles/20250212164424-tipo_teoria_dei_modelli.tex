% Intended LaTeX compiler: pdflatex
\documentclass[../main]{subfiles}


\begin{document}

\section{Tipo - Teoria dei Modelli}
\label{sec:org0ee03ef}
Sia \(\mathcal{L}\) un \href{20250130162057-linguaggio_del_prim_ordine.org}{linguaggio del prim'ordine}. Si utilizzerà la \href{20250612143636-notazione_teoria_dei_modelli.org}{Notazione TEORIA DEI MODELLI}.

Un \textbf{tipo} è un insieme di \(\mathcal{L}\)-\href{20250131103317-formula_del_prim_ordine.org}{formule}. Si denota con \(p(x)\), dove \(x\) indicano le \href{20250130162057-linguaggio_del_prim_ordine.org}{variabili} che potrebbero \href{20250131103429-variabile_libera_di_una_formula.org}{occorrere} tra le formule.

Sia \(M\) una \(\mathcal{L}\)-\href{20250131103035-struttura_del_prim_ordine.org}{struttura}.
\subsection{Soddisfazione di un tipo}
\label{sec:orge621b33}
Se per qualche \(a \in M^{x}\) si ha che per ogni \(\varphi(x) \in p(x)\) vale \(M\vDash \varphi(a)\) (vedi \href{20250131122913-soddisfazione_di_una_formula.org}{Soddisfazione di una formula}) si scive una delle seguenti:
\begin{align*}
M&\vDash p(a)\\
M, (a) &\vDash p(x)\\
a &\vDash p(x)
\end{align*}
Diremo che \(a\) è una \uline{soluzione} o una \uline{realizzazione} di \(p(x)\).
\subsection{Tipo soddisfacibile}
\label{sec:orgb69db67}
\begin{itemize}
\item Diremo che \(p(x)\) è \uline{soddisfacibile in \(M\)} (consistente in \(M\)) se ha una soluzione in \(M\).
\item Diremo che \(p(x)\) è \uline{soddisfacibile} (consistente) se è realizzato per qualche \(\mathcal{L}\)-struttura
\end{itemize}
\subsection{Tipo finitamente soddisfacibile}
\label{sec:org15af1f1}
\begin{itemize}
\item Diremo che \(p(x)\) è \uline{finitamente soddisfacibile in \(M\)} (o finitamente consistente) se tutti i sottoinsiemi finiti \(q(x) \subseteq p(x)\) sono soddisfacibili in \(M\) (o, equivalentemente, se la congiunzione arbitraria di formule di \(p(x)\) sono \href{20250131123540-formula_soddisfacibile.org}{soddisfacibili in \(M\)}).

\item Diremo che \(p(x)\) è \uline{finitamente soddisfacibile} se tutti i sottoinsiemi finiti \(q(x) \subseteq p(x)\) sono soddisfacibili.
\end{itemize}
\subsection{Tipo di un elemento di una struttura}
\label{sec:orgb9e78f3}
Sia \(a \in M^{\lambda}\) per qualche \href{20250203161341-cardinali.org}{cardinale} \(\lambda\), e sia \(x\) una variabile di lunghezza \(\lambda\).

Si definisce il \uline{tipo di \(a\) in \(M\)} come
\begin{equation*}
\operatorname{tp}_{M}(a) \coloneqq \set{
\varphi(x)\ \mathcal{L}\text{-formula}\ |\ M\vDash \varphi(a)
}
\end{equation*}
(vedi \href{20250131122913-soddisfazione_di_una_formula.org}{Soddisfazione di una formula})

Generalizzando, si scrive \(\operatorname{tp}_{M}(a/A)\), per \(A \subseteq M\) come
\begin{equation*}
\operatorname{tp}_{M}(a/A) \coloneqq \set{
\varphi(x)\ \mathcal{L}(A)\text{-formula}\ |\ M\vDash \varphi(a)
}
\end{equation*}
Questo è un \href{20251029152405-tipo_completo.org}{tipo completo su \(A\)}.

Se \(\Delta\) è un insieme di formule (chiuso per sostituzione di variabili) si definisce il \uline{\(\Delta\)-tipo di \(a\) in \(M\)} come
\begin{equation*}
\operatorname{tp}_{M}(a) \coloneqq \set{
\varphi(x) \in \Delta\ |\ M\vDash \varphi(a)
}
\end{equation*}
\subsubsection{Tuple elementarmente equivalenti su un insieme di parametri}
\label{sec:orgf1511c4}
\begin{definizione}
Se \(a,b \in M^{\lambda}\) e \(A \subseteq M\), si scrive
\begin{equation*}
a \mathrel{\equiv_{A}} b
\end{equation*}
(e si dice che \(a\) e \(b\) sono equivalenti su \(A\)) se
\begin{equation*}
\operatorname{tp}_{M}(a/A) = \operatorname{tp}_{M}(b/A).
\end{equation*}
\end{definizione}
\subsection{Insieme definito da un tipo}
\label{sec:org3b9e43b}
Sia \(p(x)\) un tipo, \(p(x) = \set{\varphi_{\alpha}(x)\mid \alpha <\lambda}\).

L'insieme definito da \(p\) in \(M\), denotato con \(p(M^{x})\) è
\begin{equation*}
p(M^{x}) \coloneqq \set{a \in M^{x}\mid M\vDash \varphi_{\alpha}[a],\text{ per ogni }\alpha<\lambda} = \bigcap_{\alpha<\lambda} \varphi_{\alpha}(M^{x})
\end{equation*}
dove \(\varphi_{\alpha}(M^{x})\) è l'\href{20250131122913-soddisfazione_di_una_formula.org}{insieme definito} da \(\varphi_{\alpha}\).

L'insieme \(p(M^{x})\) viene detto \uline{tipo-definibile}.
\end{document}
