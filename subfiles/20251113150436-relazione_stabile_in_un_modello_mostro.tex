% Intended LaTeX compiler: pdflatex
\documentclass[../main]{subfiles}


\begin{document}

\section{Relazione stabile in un Modello Mostro}
\label{sec:orge771f76}
\def\U{\mathcal{U}}
\def\L{\mathcal{L}}
%
\def\eq{{\rm eq}}
\def\Ueq{\U^\eq}
%
\def\orbita{\mathcal{O}}
\def\Aut{\operatorname{Aut}}
%
\def\tc{\mid}
\def\tp{\operatorname{tp}}
\def\EMtp{\operatorname{EM}\text{-}\operatorname{tp}}
\def\<{\langle}
\def\>{\rangle}
%
\def\restricted#1{\,\mathord{\upharpoonright}{{\scriptstyle #1}}}
\def\equivalentover#1{\mathrel{\equiv_{ #1 }}}

%% NON FORKING
\def\nonforkSymbol{%
\mathbin{\raise1.8ex%
\rlap{\kern0.6ex\rule{0.6ex}{0.1ex}}%
\rlap{\kern1.1ex\rule{0.1ex}{1.9ex}}\raise-0.3ex\hbox{$\smile$}}}
\def\defaultnonforkmodel{M}
\def\nonfork{\nonforkSymbol}
\renewcommand{\nonfork}[1][\defaultnonforkmodel]{%
\mathrel{\nonforkSymbol_{#1}}}

Si utilizza la \href{20250612143636-notazione_teoria_dei_modelli.org}{Notazione della TEORIA DEI MODELLI}

Sia \(\L\) un \href{20250130162057-linguaggio_del_prim_ordine.org}{linguaggio}, \(T\) una \href{20250130114950-teoria_del_prim_ordine.org}{teoria} \href{20250131123151-teoria_completa.org}{completa} senza \href{20250131122945-modello_di_un_insieme_di_formule.org}{modelli} finiti e \(\U\) un \href{20250617095548-modello_lambda_saturo.org}{modello saturo} di \href{20241213101756-cardinalita.org}{cardinalità} \href{20250211123155-cardinale_limite_forte.org}{inaccessibile} \(\kappa>\card{\L}+ \omega\). \(\U\) è un \href{20250617102733-modello_mostro.org}{modello mostro}.

Una qualsiasi \href{20250131103317-formula_del_prim_ordine.org}{formula} \(\psi(x;z) \in \L(\U)\) si indende come relazione, vedendola come
\begin{equation*}
\psi(\U^{x};\U^{z}) \subseteq \U^{x} \times \U^{z}.
\end{equation*}


\begin{prop}
Sia \(\psi(x;z) \in \L(A)\). LSASE:
\begin{enumerate}
\item \(\psi(x;z)\) è \href{20251113150354-relazione_stabile.org}{stabile};
\item non esistono \href{20251113150354-relazione_stabile.org}{scalette di lunghezza \(\omega\)};
\item per ogni \(\< a_{i}, b_{i} \mid i <\omega \>\) \href{20250206170922-sequenze_e_stringhe.org}{sequenza} di \href{20251026210908-sequenza_di_indiscernibili.org}{\(A\)-indiscernibili}
\begin{equation*}
 \psi(a_{0},b_{1}) \iff \psi(a_{1}, b_{0}).
\end{equation*}
\end{enumerate}
\end{prop}

\begin{thm}
Sia \(\varphi(x;z) \in \L(\U)\). LSASE:
\begin{enumerate}
\item \(\varphi(x;z)\) è \href{20251113150354-relazione_stabile.org}{stabile};
\item ogni \(\D \subseteq \U^{z}\) \href{20251113175327-insieme_esternamente_definibile_in_un_modello_mostro.org}{esternamente definibile} è \href{20250131122913-soddisfazione_di_una_formula.org}{definibile}
\item esistono sono \(\le\kappa\) insiemi esternamente definibili da \(\varphi(x;z)\), ovvero \(\card{S_{\varphi}(\U)}\le \kappa\)\footnote{Con \(S_{\varphi}(\U)\) si intende l'\href{20251026161720-phi_formula_su_un_insieme.org}{insieme dei \(\varphi\)-tipi globali}.}.
\item ci sono \(<2^{k}\) insiemi esternamente definibile da \(\varphi(x;z)\).
\end{enumerate}
\end{thm}

\begin{thm}
Sia \(\varphi(x;z) \in \L(\U)\). LSASE:
\begin{enumerate}
\item \(\varphi(x;z)\) \href{20251113150354-relazione_stabile.org}{stabile};
\item il \href{20251124090935-rango_binario_di_shelah_di_una_relazione.org}{rango binario} di \(\varphi(x;z)\) è finito.
\end{enumerate}
\end{thm}
\end{document}
