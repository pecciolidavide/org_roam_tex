% Intended LaTeX compiler: pdflatex
\documentclass[../main]{subfiles}


\begin{document}

\section{Teorema di Hurewicz}
\label{sec:org5488ea1}
\begin{thm}
Sia \(X\) \href{20250103145124-topologia.org}{spazio topologico} \href{20250113103025-spazio_topologico_connesso_per_archi.org}{connesso per archi}, \(x_{0} \in X\).

Sia \(\Pi_{1}(X,x_{0})\) il \href{20241204223712-gruppo_fondamentale.org}{gruppo fondamentale} di \((X,x_{0})\) e \(H_{1}(X)\) lo \href{20241205141053-r_moduli.org}{\(\Z\)-modulo} di \href{20250122133631-omologia_singolare.org}{omologia singolare} di \(X\) (ovvero un \href{20250127093245-gruppo_abeliano.org}{gruppo abeliano}).
Allora esiste un \href{20241206115531-morfismo_di_gruppi.org}{morfismo}, detto Mappa di Hurewicz
\begin{equation*}
\varphi:\Pi_{1}(X,x_{0}) \longrightarrow H_{1}(X)
\end{equation*}
\href{20241213105600-funzione_suriettiva.org}{suriettivo} tale che\footnote{\(\ker\) è il \href{20241213105201-kernel.org}{Kernel}.} \(\operatorname{ker}(\varphi)\) sia l'\href{20250127093652-gruppo_abelianizzato.org}{abelianizzato} di \(\Pi_{1}(X,x_{0})\):
\begin{equation*}
\operatorname{ker}(\varphi)=\left[\Pi_{1}(X,x_{0}),\Pi_{1}(X,x_{0})\right].
\end{equation*}
\end{thm}
\begin{cor}
Per il \href{20250120155457-morfismo_iniettivo_di_r_moduli_induce_isomorfismo.org}{primo teorema di isomorfismo}
\begin{equation*}
H_{1}(X)\cong \Pi_{1}(X,x_{0})/\operatorname{ker}(\varphi) = \Pi_{1}(X,x_{0})^{\operatorname{ab}}
\end{equation*}
\end{cor}
\section{Trasformazione naturale indotta dal Teorema di Hurewicz}
\label{sec:orgab97ce7}
Sia \(R=\Z\), siano \(X,Y\) \href{20250103145124-topologia.org}{spazi topologici} \href{20250113103025-spazio_topologico_connesso_per_archi.org}{cpa}. Sia \(f:X\to Y\) funzione continua, \(x_{0} \in X\), \(y_{0}\coloneqq f(x_{0}) \in Y\). Siano:
\begin{align*}
\varphi_{X}: \Pi_{1}(X,x_{0}) &\longrightarrow H_{1}(X)\\
\varphi_{Y}: \Pi_{1}(Y,y_{0}) &\longrightarrow H_{1}(Y )
\end{align*}
le mappe del \hyperref[sec:org5488ea1]{Teorema di Hurewicz} dal \href{20241204223712-gruppo_fondamentale.org}{gruppo fondamentale} di \(X\) e \(Y\) nel primo gruppo di \href{20250122133631-omologia_singolare.org}{omologia singolare}. Si indichi con \(f_{\star}\) l'immagine di \(f\) sia tramite \href{20250114095222-funtore_del_gruppo_fondamentale.org}{il funtore \(\Pi_{1}\)} che tramite il \href{20250123115927-funtore_di_omologia_singolare.org}{funtore \(H_{1}\)}.

\begin{prop}
Il seguente diagramma commuta:
\begin{equation*}
\begin{tikzcd}[ampersand replacement=\&]
	{\Pi_1(X,x_0)} \&\& {H_1(X)} \\
	\\
	{\Pi_1(Y,y_0)} \&\& {H_1(Y)}
	\arrow["{\varphi_X}", from=1-1, to=1-3]
	\arrow["{f_\star}"', from=1-1, to=3-1]
	\arrow["{f_\star}", from=1-3, to=3-3]
	\arrow["{\varphi_Y}"', from=3-1, to=3-3]
\end{tikzcd}
\end{equation*}
e quindi \(\set{\varphi_{X}}\) è una \href{20241205132705-trasformazioni_naturali.org}{trasformazione naturale} tra i \href{20241205131958-funtore.org}{funtori}:
\begin{itemize}
\item del \href{20250114095222-funtore_del_gruppo_fondamentale.org}{gruppo fondamentale}, \(\Pi_{1}\);
\item di \href{20250123115927-funtore_di_omologia_singolare.org}{omologia singolare}, \(H_{1} \circ U\)\footnote{\(U\) è il \href{20241205131958-funtore.org}{funtore} dimenticante da \href{20241205115614-categoria_top.org}{spazi topologici puntati} a \href{20241205115600-categoria_top.org}{spazi topologici} \(\cat{Top*}\to \cat{Top}\)}
\end{itemize}
\end{prop}
\begin{proof}
Infatti, sia \([\gamma]_{\Pi} \in \Pi_{1}(X,x_{0})\):
\begin{align*}
\varphi_{Y}\circ f_{\star}[\gamma]_{\Pi} &= \varphi_{Y} [f\circ\gamma]_{\Pi}%
= [f\circ\gamma]_{H}\\
f_{\star}\circ\varphi_{X}[\gamma]_{\Pi} &= f_{\star} [\gamma]_{H} = [f\circ\gamma]_{H}.
\qedhere
\end{align*}
\end{proof}
\end{document}
