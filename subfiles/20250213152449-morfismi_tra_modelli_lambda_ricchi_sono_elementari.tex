% Intended LaTeX compiler: pdflatex
\documentclass[../main]{subfiles}


\begin{document}

Si utilizza la \href{20250612143636-notazione_teoria_dei_modelli.org}{Notazione TEORIA DEI MODELLI}.

Sia \(\mathcal{M} = \mathcal{M}_{\text{ob}}\cup \mathcal{M}_{\text{hom}}\) una \href{20241126100904-categoria.org}{categoria} \href{20250213142026-categorie_di_modelli_e_morfismi_parziali.org}{di modelli e morfismi parziali} di \href{20250130162057-linguaggio_del_prim_ordine.org}{linguaggio} \(\mathcal{L}\) tale che
\begin{enumerate}
\item per ogni \(M \in \mathcal{M}_{\text{ob}}\) e per ogni \(A \subseteq M\), \(\id_{A} : M\partialto M\) è \(\id_{A} \in \mathcal{M}_{\text{hom}}\);
\item per ogni \(M,N \in \mathcal{M}_{\text{ob}}\), e \(k:M\partialto N\) \href{20250213105339-funzione_parziale.org}{funzione parziale}, se per ogni \(k' \subseteq k\) \href{20250205120448-classe_finita_e_infinita_mk.org}{finito} si ha \(k' \in \mathcal{M}_{\text{hom}}\), allora \(k \in \mathcal{M}_{\text{hom}}\);
\item per ogni \(k \in \mathcal{M}_{\text{hom}}\), \(k\) è \href{20250111142446-funzione_inversa.org}{invertibile} e la sua \href{20250111142446-funzione_inversa.org}{inversa} \(k^{-1} \in \mathcal{M}_{\text{hom}}\);
\item gli elementi di \(\mathcal{M}_{\text{hom}}\) \href{20250214120959-mappe_tra_strutture_del_prim_ordine.org}{preservano la verità} delle \href{20250131103317-formula_del_prim_ordine.org}{formule atomiche};
\item se \(M \in \mathcal{M}_{\text{ob}}\) e \(N\) è una \href{20250131103035-struttura_del_prim_ordine.org}{struttura} \href{20250131123208-teorie_elementarmente_equivalente.org}{elementarmente equivalente} a \(M\), \(M\equiv N\), allora \(N \in \mathcal{M}_{\text{ob}}\).
\item per ogni \(M, N \in \mathcal{M}_{\text{ob}}\), se \(h:M\partialto N\) è una \href{20250214120959-mappe_tra_strutture_del_prim_ordine.org}{mappa elementare}, allora \(h \in \mathcal{M}_{\text{hom}}\).
\end{enumerate}

Sia \(\lambda\) un \href{20250203161341-cardinali.org}{cardinale} tale che \(\card{\mathcal{L}}\le\lambda\)
\section{Teorema}
\label{sec:orgeec6ee2}
Siano \(M,N \in \mathcal{M}_{\text{ob}}\) dei modelli \(\lambda\)-ricchi. Allora, per ogni \(k \in \mathcal{M}_{\text{hom}}(M,N)\), \(k\) è un \href{20250214120959-mappe_tra_strutture_del_prim_ordine.org}{morfismo elementare}.
\subsection{Dimostrazione}
\label{sec:orgdf4f915}
Sia \(k \in \mathcal{M}_{\text{hom}}(M,N)\).
\begin{itemize}
\item Se tutte le restrizioni finite di \(k\) sono elementari, allora \(k\) è elementare: si supponga ad esempio che \(M\vDash \varphi[a_{1},\dots,a_{n}]\); poiché \(k\upharpoonright\set{a_{1},\dots,a_{n}}\) è elementare, allora
\begin{equation*}
  N\vDash \varphi\left[k\upharpoonright\set{a_{1},\dots,a_{n}}(a_{1}),\dots,k\upharpoonright\set{a_{1},\dots,a_{n}}(a_{n})\right]
\end{equation*}
ovvero \(N\vDash \varphi[k(a_{1}),\dots,k(a_{n})]\).

Pertanto, WLOG, si dimostra l'enunciato per \(k\) finita.
\item Si costruiscono due \href{20250212102253-sottostruttura_elementare.org}{sottostrutture elementari} (che per 5. sono in \(\mathcal{M}_{\text{ob}}\)): \(\operatorname{dom}k \subseteq M'\preceq M\), \(N'\preceq N\) ed una mappa \(h \in \mathcal{M}_{\text{hom}}(M,N)\) tale che \(k \subseteq h\) e \(h:M'\to N'\) sia biiettiva.

Per le ipotesi 3. e 4. \(h\) è un \href{20250214120959-mappe_tra_strutture_del_prim_ordine.org}{isomorfismo} (e quindi in particolare è una \href{20250214120959-mappe_tra_strutture_del_prim_ordine.org}{mappa elementare}) e quindi, per l'ipotesi 6. \(h \in \mathcal{M}_{\text{hom}}\).
\item Siano ora \(\varphi\) una \(\mathcal{L}\)-formula, e siano \(a_{1},\dots,a_{n} \in \operatorname{dom}(k)\) tali che \(M\vDash \varphi[a_{1},\dots,a_{n}]\). Allora \(a_{1},\dots,a_{n} \in M'\) e pertanto (siccome \(M'\preceq M\))
\begin{equation*}
  M'\vDash \varphi[a_{1},\dots,a_{n}]
\end{equation*}
e poiché \(h\) è un isomorfismo:
\begin{equation*}
  N'\vDash \varphi[ha_{1},\dots,ha_{n}]
\end{equation*}
ma \(N'\preceq N\) e \(h\upharpoonright \operatorname{dom}k = k\) e pertanto
\begin{equation*}
  N\vDash \varphi[ha_{1},\dots,ha_{n}]\quad\leadsto\quad N\vDash \varphi[ka_{1},\dots,ka_{n}].
\end{equation*}
\end{itemize}

Per ricorsione si costruisce una famiglia \(\langle h_{i}: i<\lambda\rangle\) tale che \(\card{h_{i}}<\lambda\) ; \href{20250205181254-order_type_del_prodotto_cartesiano_di_un_cardinale_e_il_cardinale_stesso.org}{si fissi \(\pi:\lambda^{2}\to \lambda\)} una \href{20250104111707-funzione_biunivoca.org}{biiezione} tale che \(j,k\le\pi(j,k)\):
\begin{itemize}
\item si pone \(h_{0}\coloneqq k \in \mathcal{M}_{\text{hom}}\);
\item ai passi \href{20250203161132-ordinale_limite.org}{limite} si considera l'unione; anche questa ha cardinalità \(<\lambda\), e inoltre è in \(\mathcal{M}_{\text{hom}}\) per la proprietà 2.;
\item per i passi \href{20250203161132-ordinale_limite.org}{successori} \(i+1\), si suppongia sia costruita \(h_{j}\) per ogni \(j<i+1\).

\begin{itemize}
\item Siccome \(\operatorname{dom}(h_{j})\) ha cardinalità \(<\lambda\) allora anche \(\card{\mathcal{L}(\operatorname{dom}h_{j})}<\lambda\): sia dunque
\begin{equation*}
	\langle \varphi_{j,k}(x)\mid k<\lambda\rangle
\end{equation*}
una \href{20250203133527-insiemi_ben_ordinati_sono_isomorfi_ad_un_ordinale_unico.org}{enumerazione} delle \href{20250131103317-formula_del_prim_ordine.org}{formule} \href{20250212144403-formula_consistente.org}{consistenti} di \(\mathcal{L}(\operatorname{dom}h_{j})\) in \(M\).

Siano dunque \(j,k<i\) tali che \(\pi(j,k)=i\), e sia \(b \in M\) testimone di \(\varphi_{j,k}\) in \(M\) (ovvero \(M\vDash \varphi_{j,k}[b]\)). Poiché \(\card{h_{i}}<\lambda\), siccome \(N\) è \(\lambda\)-ricco, esiste \(c \in N\) tale che
\begin{equation*}
h_{i}\cup\set{(b,c)} \in \mathcal{M}_{\text{hom}}.
\end{equation*}

Si pone dunque \(h_{i+1/2} \coloneqq h_{i}\cup\set{(b,c)}\). Dunque \(\card{h_{i+1/2}}<\lambda\).

\item Siccome \(\operatorname{rng}(h_{j})\) ha cardinalità \(<\lambda\) allora anche \(\card{\mathcal{L}(\operatorname{rng}h_{j})}<\lambda\): sia dunque
\begin{equation*}
	\langle \psi_{j,k}(x)\mid k<\lambda\rangle
\end{equation*}
una \href{20250203133527-insiemi_ben_ordinati_sono_isomorfi_ad_un_ordinale_unico.org}{enumerazione} delle \href{20250131103317-formula_del_prim_ordine.org}{formule} \href{20250212144403-formula_consistente.org}{consistenti} di \(\mathcal{L}(\operatorname{rng}h_{j})\) in \(N\).

Siano dunque \(j,k < i\) tali che \(\pi(j,k) = i\), e sia \(b \in N\) testimone di \(\psi_{j,k}\) in \(N\) (ovvero \(N\vDash\psi_{j,k}[b]\)). Poiché \(\card{(h_{i+1/2})^{-1}}<\lambda\) e \((h_{i+1/2})^{-1} \in \mathcal{M}_{\text{hom}}\) per l'ipotesi 3., allora siccome \(M\) è \(\lambda\)-ricco, esiste \(c \in N\) tale che
\begin{equation*}
	(h_{i+1/2})^{-1}\cup \set{(b,c)} \in \mathcal{M}_{\text{hom}}.
\end{equation*}

Si pone quindi \(h_{i+1}\coloneqq\left((h_{i+1/2})^{-1}\cup \set{(b,c)}\right)^{-1}\).
\end{itemize}
\end{itemize}

Dunque sia \(h\coloneqq\bigcup_{i<\lambda} h_{i}\) e siano
\begin{equation*}
M' = \operatorname{dom}h,\qquad N': \operatorname{rng}h.
\end{equation*}
\begin{itemize}
\item \(k \subseteq h\), \(h \in \mathcal{M}_{\text{hom}}\) poiché ogni sua restrizione finita lo è (ad ogni passo \(i<\omega\le \lambda\) si ha che \(h_{i} \in \mathcal{M}_{\text{hom}}\)), e inoltre \(h\) è iniettiva, pertanto \(h\upharpoonright M' = h\upharpoonright \operatorname{dom}h\) è una biiezione.
\item \(M'\preceq M\): si utilizza il \href{20250212113245-criterio_di_tarski_vaught.org}{Criterio di Tarski-Vaught}; sia \(\varphi(x)\) una \(\mathcal{L}(M')\)-formula consistente in \(M\). Allora \(\varphi(x)\) è una \(\mathcal{L}(\operatorname{dom}h_{i})\) per qualche \(i<\lambda\) (poiché ogni formula è una stringa finita), e pertanto \(\varphi=\varphi_{i,k}\) per qualche \(k<\lambda\). Quindi, per costruzione, esiste \(b \in \operatorname{dom}h_{\pi(i,k)+1} \subseteq M'\) tale che \(M\vDash \varphi[b]\).
\item \(N'\preceq N\): si utilizza il \href{20250212113245-criterio_di_tarski_vaught.org}{Criterio di Tarski-Vaught}; sia \(\psi(x)\) una \(\mathcal{L}(N')\)-formula consistente in \(N\). Allora \(\varphi(x)\) è una \(\mathcal{L}(\operatorname{rng}h_{i})\) per qualche \(i<\lambda\) (poiché ogni formula è una stringa finita), e pertanto \(\psi=\psi_{i,k}\) per qualche \(k<\lambda\). Quindi, per costruzione, esiste \(b \in \operatorname{rng}h_{\pi(i,k)+1} \subseteq N'\) tale che \(N\vDash \psi[b]\).\qed
\end{itemize}
\end{document}
