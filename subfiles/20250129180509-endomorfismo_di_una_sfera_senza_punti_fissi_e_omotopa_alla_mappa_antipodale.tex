% Intended LaTeX compiler: pdflatex
\documentclass[../main]{subfiles}


\begin{document}

\section{Endomorfismo di una sfera senza punti fissi è omotopa alla mappa antipodale}
\label{sec:orgdea2df5}
\begin{prop}
Sia \(f: \mathds{S}^{n} \longrightarrow \mathds{S}^{n}\) una \href{20250103103252-funzione_continua.org}{funzione continua} senza \href{20250129160423-punto_fisso.org}{punti fissi}.
Allora \(f\) è \href{20250121094654-omotopia_tra_funzioni_continue.org}{omotopa} alla funzione:
\begin{align*}
a: \mathds{S}^{n} &\longrightarrow \mathds{S}^{n}\\
x &\longmapsto -x
\end{align*}
\end{prop}

\begin{proof}
Se \(f\) non ha punti fissi, allora \(\forall p \in \mathds{S}^{n}\), \(f(p) \neq p\). Si costruisce l'omotopia \href{20250625123506-spazio_normato.org}{normalizzata}:\footnote{\textbf{NOTA BENE}: questa mappa altro non è che il cammino standard tra \(-p\) e \(f(p)\) in \(\R^{n+1}\), rinormalizzato per stare dentro \(\mathds{S}^{n}\).}
\begin{equation*}
H(p,t) \coloneqq \frac{t\cdot f(p) - (1-t)\,p}{\norma{t\cdot f(p) - (1-t)\,p}}
\end{equation*}
Se è ben definita, allora è continua, e \(H(p, 0) = -p\), \(H(p,1) = f(p)\).

Supponiamo quindi per assurdo che esistano \(( t_{0}, p_0)\) (WLOG \(t_{0}\neq 0\)) tali che
\begin{equation*}
\norma{ t_{0}\cdot f( p_0) - (1- t_{0})\, p_0} = 0%
\IFF%
 t_{0}\cdot f( p_0) - (1- t_{0})\, p_0 = 0
\end{equation*}
ovvero sse \(\displaystyle f( p_0) = \frac{1- t_{0}}{ t_{0}}\  p_0\). Quindi
\begin{equation*}
\norma{f( p_0)} = \norma{\frac{1- t_{0}}{ t_{0}}\  p_0} = \frac{1- t_{0}}{ t_{0}}\ \norma{ p_0}
\end{equation*}
e quindi \(\frac{1- t_{0}}{ t_{0}} = 1\), ovvero \(t_{0}=0\). Assurdo.
\end{proof}

\begin{cor}
Se \(f:\mathds{S}^{n} \longrightarrow \mathds{S}^{n}\) \href{20250103103252-funzione_continua.org}{continua} ha \href{20250129170910-grado_di_un_endomorfismo_della_sfera.org}{grado}
\begin{equation*}
\operatorname{deg}f\neq (-1)^{n+1}
\end{equation*}
allora \(f\) ha \href{20250129160423-punto_fisso.org}{punti fissi}.
\end{cor}
\begin{proof}
È sufficiente considerare il \href{20250129170910-grado_di_un_endomorfismo_della_sfera.org}{grado della mappa antipodale}.
\end{proof}
\end{document}
