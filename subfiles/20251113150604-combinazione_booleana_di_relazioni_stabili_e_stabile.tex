% Intended LaTeX compiler: pdflatex
\documentclass[../main]{subfiles}


\begin{document}

\def\U{\mathcal{U}}
\def\L{\mathcal{L}}
\def\X{\mathcal{X}}
\def\Z{\mathcal{Z}}
%
\def\eq{{\rm eq}}
\def\Ueq{\U^\eq}
%
\def\orbita{\mathcal{O}}
\def\Aut{\operatorname{Aut}}
%
\def\tc{\mid}
\def\tp{\operatorname{tp}}
\def\EMtp{\operatorname{EM}\text{-}\operatorname{tp}}
\def\<{\langle}
\def\>{\rangle}
%
\def\restricted#1{\,\mathord{\upharpoonright}{{\scriptstyle #1}}}
\def\equivalentover#1{\mathrel{\equiv_{ #1 }}}

%% NON FORKING
\def\nonforkSymbol{%
\mathbin{\raise1.8ex%
\rlap{\kern0.6ex\rule{0.6ex}{0.1ex}}%
\rlap{\kern1.1ex\rule{0.1ex}{1.9ex}}\raise-0.3ex\hbox{$\smile$}}}
\def\defaultnonforkmodel{M}
\def\nonfork{\nonforkSymbol}
\renewcommand{\nonfork}[1][\defaultnonforkmodel]{%
\mathrel{\nonforkSymbol_{#1}}}
\section{Combinazione booleana di relazioni stabili è stabile}
\label{sec:org088e740}
\begin{prop}
Combinazioni Booleane di \href{20251113150354-relazione_stabile.org}{relazioni stabili} sono stabili.
\end{prop}
\begin{proof}
Siano \(\X, \Z\) due \href{20250130104331-insieme_mk.org}{insiemi} arbitrari, e siano \(\pi,\sigma \subseteq \X\times \Z\) due \href{20250202170607-classe_relazione_binaria.org}{relazioni} binaria. Si scriverà indifferentemente
\begin{equation*}
(x,z) \in \pi,\qquad \pi(x,z).
\end{equation*}

(\(\lnot\)): Se \(\pi\) è stabile allora \(\lnot\pi \coloneqq \X\times\Z\setminus\pi\) è stabile.

Infatti, se  \(\< (a_{i},\, b_{i}) \mid i<m \>\) è una scaletta per \(\pi\)
\begin{equation*}
i<j<m \IMPLICA \pi(a_{i}, b_{j}) \land \lnot\pi(a_{j}, b_{i}). %
\end{equation*}
allora \(\< (a_{m-i+1},\, b_{m-i+1}) \mid i<m \>\) è una scaletta per \(\lnot \pi\).

(\(\land\)):
\begin{itemize}
\item \uline{Caso facile}: Con la \href{20250612143636-notazione_teoria_dei_modelli.org}{Notazione della TEORIA DEI MODELLI} in un modello mostro

Sia \(\L\) un \href{20250130162057-linguaggio_del_prim_ordine.org}{linguaggio}, \(T\) una \href{20250130114950-teoria_del_prim_ordine.org}{teoria} \href{20250131123151-teoria_completa.org}{completa} senza \href{20250131122945-modello_di_un_insieme_di_formule.org}{modelli} finiti e \(\U\) un \href{20250617095548-modello_lambda_saturo.org}{modello saturo} di \href{20241213101756-cardinalita.org}{cardinalità} \href{20250211123155-cardinale_limite_forte.org}{inaccessibile} \(\kappa>\card{\L}+ \omega\). \(\U\) è un \href{20250617102733-modello_mostro.org}{modello mostro}.

Siano \(\X=\U^{x}\) e \(\Z=\U^{z}\) e siano \(\pi(x;z), \sigma(x;z) \in \L(A)\).
Si dimostra che
\begin{equation*}
\pi(x;z), \sigma(x;z) \text{ stabili} \IMPLICA
\pi(x;z)\land\sigma(x;z)\text{ stabile}
\end{equation*}
ovvero
\begin{equation*}
\pi(x;z)\land\sigma(x;z)\text{ instabile} \IMPLICA
\pi(x;z) \text{ instabile}\lor \sigma(x;z) \text{ instabile}.
\end{equation*}

Si supponga che \(\< a_{i}, b_{i} \mid i<\omega \>\) sequenza di \(A\) indiscernibili tali che
\begin{equation*}
\pi(a_{0},b_{1}) \land \sigma(a_{0},b_{1}) \land
\lnot\big[\pi(a_{1},b_{0}) \land \sigma(a_{1},b_{0})\big]
\end{equation*}
ovvero
\begin{equation*}
\pi(a_{0},b_{1}) \land \sigma(a_{0},b_{1}) \land
\big[\lnot\pi(a_{1},b_{0}) \lor \lnot\sigma(a_{1},b_{0})\big]
\end{equation*}
\begin{itemize}
\item Se \(\lnot\pi(a_{1},b_{0})\) è vera, allora \(\pi\) è instabile;
\item se \(\lnot\sigma(a_{1},b_{0})\) è vera, allora \(\sigma\) è instabile.
\end{itemize}

\item \uline{Caso generale}. Siano \(\pi,\sigma\) relazioni arbitrarie.
Si vuole dimostrare
\begin{equation*}
\pi(x;z)\land\sigma(x;z)\text{ instabile} \IMPLICA
\pi(x;z) \text{ instabile}\lor \sigma(x;z) \text{ instabile}.
\end{equation*}
Sia \(\< a_{i}, b_{i} \mid i<k \>\) scaletta per
\(\pi(x;z)\land\sigma(x;z)\),
con \(k\) ``abbastanza grande'', ovvero se \(i<j<k\) si ha
\begin{equation*}
\pi(a_{i},b_{j}) \land \sigma(a_{i},b_{j}) \land
\big[\lnot\pi(a_{j},b_{i}) \lor \lnot\sigma(a_{j},b_{i})\big].
\end{equation*}
\begin{itemize}
\item Coloro \(i<j\) di verde se \(\lnot\pi(a_{j},b_{i})\);
\item Coloro \(i<j\) di blu se \(\lnot\sigma(a_{j},b_{i})\).
\end{itemize}

Per il \href{20251029171516-teorema_di_ramsey.org}{Teorema di Ramsey} per ogni \(m\) esiste \(k\) (ovvero quel \(k\) abbastanza grande) tale che per ogni \href{20251111150142-colorazione.org}{colorazione} di \(\set{0,\dots,k-1}^{(2)}\) esiste \(H \subseteq \set{0,\dots,k-1}\) tale che \(\card{H}\ge m\) e \(H^{(2)}\) è \href{20251111150142-colorazione.org}{monocromatico}.
\begin{itemize}
\item Se \(H^{(2)}\) è blu, allora \(\sigma\) ammette scaletta di lunghezza \(m\);
\item se \(H^{(2)}\) è verde, allora \(\pi\) ammette scaletta di lunghezza \(m\).
\end{itemize}
Segue che almeno una dei due è instabile.
Infatti, se \(\sigma\) non ammette scalette di lunghezza \(m'\), allora non ammette scalette di lunghezza \(m>m'\).
Pertanto, per ogni \(m>m'\) si ha che \(\pi\) ammette scaletta di lunghezza \(m\).
Segue che \(\pi\) ammette scalette di lunghezza \(m\) per ogni \(m<\omega\).
\qedhere
\end{itemize}
\end{proof}
\end{document}
