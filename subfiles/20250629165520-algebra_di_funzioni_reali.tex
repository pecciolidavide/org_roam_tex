% Intended LaTeX compiler: pdflatex
\documentclass[../main]{subfiles}


\begin{document}

\begin{definizione}
Sia \(\mathcal{F}\) un insieme di \href{20250202170607-classe_relazione_binaria.org}{funzioni} con lo stesso \href{20250202173528-dominio_range_e_campo_di_una_classe_relazione.org}{dominio}, a \href{20250202173528-dominio_range_e_campo_di_una_classe_relazione.org}{valori} in \(\R\). \(\mathcal{F}\) si dice un'\uline{algebra di funzioni reali} se, per ogni \(f,g \in \mathcal{F}\) e \(c \in \R\):
\begin{enumerate}
\item \(f+g \in \mathcal{F}\);
\item \(c\,f \in \mathcal{F}\);
\item \(f\,g \in \mathcal{F}\).
\end{enumerate}
\end{definizione}

Questa è la definizione di \href{20250110175552-algebra_su_un_campo.org}{\(\R\)-algebra}, specializzata in questo ambito.
\begin{esempio}
Sia \(\mathcal{A}\) l'insieme di tutte le serie di Fourier finite su \([0,2\pi]\):
\begin{equation*}
\mathcal{A} \coloneqq \set{f(x) = c_{0}+ \sum_{k=1}^{N} (a_{k}\cos kx + b_{k}\sin kx)\mid c_{0},a_{k},b_{k} \in \R, N \in\N}.
\end{equation*}
Ovviamente \(\mathcal{A}\) è chiuso per combinazioni lineari. Utilizzando il fatto che
\begin{equation*}
\cos(mx)\cos(nx) = \frac{1}{2}\left[\cos((m+n)x) + \cos((m-n)x)\right]
\end{equation*}
segue che \(\mathcal{A}\) è anche chiuso per prodotti. Pertanto è una \(\R\)-algebra.
\end{esempio}
\end{document}
