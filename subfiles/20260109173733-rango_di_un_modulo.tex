% Intended LaTeX compiler: pdflatex
\documentclass[../main]{subfiles}

\usepackage[hyperref]{biblatex}
\date{}
\title{}
\begin{document}

\section{Rango di un modulo}
\label{sec:org0335572}
\subsection{Rango di un modulo libero}
\label{sec:org72d79eb}
Sia \(R\) un \href{20241205141119-anello.org}{anello commutativo con unità}.

\begin{definizione}
Sia \(M\) un \href{20241205141053-r_moduli.org}{modulo} \href{20241213094625-modulo_libero.org}{libero}, e sia \(E\) una sua \href{20241213094625-modulo_libero.org}{base}. La \href{20241213101756-cardinalita.org}{cardinalità} di \(E\) si dice \textbf{rango} di \(M\) e si indica con \(\operatorname{rg}M\).
\end{definizione}

\begin{oss}
La definizione è ben posta in virtù del \href{20241213101621-teorema_della_base.org}{Teorema della Base}.
\end{oss}

È possibile estendere la definizione di cui sopra come segue:
\subsection{Rango di un modulo FG}
\label{sec:org5fd7a22}
Sia \(R\) un \href{20241219112842-pid.org}{PID}, e sia \(M\) un \href{20241205141053-r_moduli.org}{\(R\)-modulo} \href{20241213100845-modulo_finitamente_generato.org}{finitamente generato}. \href{20250120103205-modulo_di_torsione_finitamente_generato_e_libero.org}{Allora}\footnote{Vedi ``\href{20241220000402-torsione_moduli.org}{Torsione di un modulo}''} \(M'\coloneqq M/\operatorname{Tor}M\) è \href{20241213094625-modulo_libero.org}{libero}. È pertanto ben definito il suo \hyperref[sec:org72d79eb]{rango}.

\begin{definizione}
Si definisce il rango di \(M\) come
\begin{equation*}
\operatorname{rg} M \coloneqq \operatorname{rg}M'
\end{equation*}
\end{definizione}
\end{document}
