% Intended LaTeX compiler: pdflatex
\documentclass[../main]{subfiles}


\begin{document}

\section{Calcolo dell'omologia simpliciale per il complesso di catene generato da un 2-simplesso}
\label{sec:org8c1b5fe}
Sia \(\sigma=[p_{0},p_{1},p_{2}]\) un 2-\href{20250121121923-simplessi.org}{simplesso}, e sia \(K = K[\sigma]\) il \href{20250120163114-complesso_di_catene.org}{complesso simpliciale} \href{20250121125446-complesso_simpliciale_generato_da_un_simplesso.org}{generato} da \(\sigma\).
Sia \(R\) un \href{20241219112842-pid.org}{PID}.

Denotando con
\begin{equation*}
\simplesso{i_{0},\dots,i_{k}} = [p_{i_{0}},\dots,p_{i_{k}}]
\end{equation*}
si ha che i \href{20250121160600-complesso_di_catene_simpliciali.org}{complessi di catene simpliciali} sono i seguenti \href{20241213094625-modulo_libero.org}{moduli liberi}\footnote{Vedi:
\begin{itemize}
\item \href{20241213095808-somma_diretta.org}{Somma Diretta}
\item \href{20241213095808-somma_diretta.org}{Somma diretta dell'anello è modulo libero}
\end{itemize}}:
\begin{align*}
C_{0}(K) &= R^{(\set{\simplesso{0},\simplesso{1},\simplesso{2}})}\\
C_{1}(K) &= R^{(\set{\simplesso{01}, \simplesso{02}, \simplesso{12}})}\\
C_{2}(K) &= R^{(\set{\simplesso{012}})} = R\cdot \simplesso{012}
\end{align*}

Le \href{20250121132856-mappe_di_bordo_tra_moduli_di_catene_simpliciali.org}{mappe di bordo} sono le seguenti:
\begin{equation*}
\begin{tikzcd}[ampersand replacement=\&,row sep=tiny]
	0 \& {C_2(K)} \& {C_1(K)} \& {C_0(K)} \& 0 \\
	\& {\simplesso{012}} \& {\simplesso{12}-\simplesso{02}+\simplesso{01}} \\
	\&\& {\simplesso{01}} \& {\simplesso{1}-\simplesso{0}} \\
	\&\& {\simplesso{02}} \& {e_2-\simplesso{0}} \\
	\&\& {\simplesso{12}} \& {e_2-\simplesso{1}}
	\arrow["{\partial_3}", from=1-1, to=1-2]
	\arrow["{\partial_2}", from=1-2, to=1-3]
	\arrow["{\partial_1}", from=1-3, to=1-4]
	\arrow["{\partial_0}", from=1-4, to=1-5]
	\arrow[maps to, from=2-2, to=2-3]
	\arrow[maps to, from=3-3, to=3-4]
	\arrow[maps to, from=4-3, to=4-4]
	\arrow[maps to, from=5-3, to=5-4]
\end{tikzcd}
\end{equation*}

Si vogliono calcolare i \href{20241205141053-r_moduli.org}{moduli} di \href{20250121160827-omologia_simpliciale.org}{omologia}.

\begin{itemize}
\item \textbf{Modulo di omologia 2}

Si ha che \(\partial_{2}\) è una mappa iniettiva\footnote{Se \(\partial_{2}(r\simplesso{012}) = 0\), allora
\begin{equation*}
r\simplesso{12} - r\simplesso{02} + r\simplesso{01} = 0
\end{equation*}
con \(\set{\simplesso{01}, \simplesso{02}, \simplesso{12}}\) \href{20241213094625-modulo_libero.org}{base di \(C_{1}(K)\)}: quindi \(r=0\).}, e quindi \(\operatorname{ker}\partial_{2} = \{0\}\). Inoltre, ovviamente, \(\partial_{3}\equiv 0\), e dunque \(\operatorname{Im}\partial_{3} = \set{0}\).
Dunque
\begin{equation*}
H_{2}(K) = \frac{\operatorname{ker}\partial_{2}}{\operatorname{Im}\partial_{3}} = \set{0}/\set{0} = 0
\end{equation*}

\item \textbf{Modulo di omologia 1}

Per il \href{20250120155457-morfismo_iniettivo_di_r_moduli_induce_isomorfismo.org}{primo teorema di isomorfismo}, \(\operatorname{Im}\partial_{2} \cong C_{2}(K) / \set{0}\), con base \(\set{\simplesso{12}-\simplesso{02}+\simplesso{01}}\).

Consideriamo invece un generico elemento di \(C_{1}(K)\), ovvero, fissati \(a,b,c \in R\),
\begin{equation*}
a\simplesso{01} + b\simplesso{02}+c\simplesso{12}.
\end{equation*}
Per calcolare il \(\operatorname{ker}\partial_{1}\) imponiamo
\begin{equation*}
  \begin{aligned}
  	0 &= \partial_{1}(a\simplesso{01}+b\simplesso{02}+c\simplesso{12})\\
  	&= a \simplesso{1}-a\simplesso{0} +b\simplesso{2}-b\simplesso{0} +c\simplesso{2}-c\simplesso{1}\\
  	&= \simplesso{1}(a-c) + \simplesso{0} (-a-b) + \simplesso{2}(b+c)
  \end{aligned}%
\IFF%
  \begin{cases}
  	a-c=0\\
  	-a-b=0\\
  	b+c=0
  \end{cases}
\end{equation*}
dove la doppia implicazione vale perché \(\simplesso{0},\set{\simplesso{1},\simplesso{2}}\) è base di \(C_{0}(K)\).

Dunque, \(x \in \operatorname{ker}\partial_{1}\) se e solo se \(x = c(\simplesso{01}-\simplesso{02}+\simplesso{12})\) e dunque
\begin{equation*}
\operatorname{ker}\partial_{1} = R\cdot (\simplesso{01}-\simplesso{02}+\simplesso{12})
\end{equation*}

Pertanto
\begin{equation*}
H_{1}(K) = \frac{R\cdot(\simplesso{01}-\simplesso{02}+\simplesso{12})}{R\cdot (\simplesso{01}-\simplesso{02}+\simplesso{12})
} = 0
\end{equation*}

\item \textbf{Modulo di omologia 0}

Siccome \(\partial_{0} \equiv 0\), allora \(\operatorname{ker}\partial_{0} = C_{0}(K) = R^{\set{\simplesso{0},\simplesso{1},\simplesso{2}}}\).
Viceversa, \(\operatorname{Im}\partial_{1} = R^{\set{\simplesso{1}-\simplesso{0},\simplesso{2}-\simplesso{0},\simplesso{2}-\simplesso{1}}}\), e dunque
\begin{equation*}
H_{0}(K) = \frac{R^{\set{\simplesso{0},\simplesso{1},\simplesso{2}}}}{R^{\set{\simplesso{1}-\simplesso{0},\simplesso{2}-\simplesso{0},\simplesso{2}-\simplesso{1}}}} = R\cdot \simplesso{0}
\end{equation*}
\end{itemize}
\end{document}
