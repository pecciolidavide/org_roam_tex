% Intended LaTeX compiler: pdflatex
\documentclass[../main]{subfiles}


\begin{document}

\begin{thm}
Sia \(R\) un \href{20241205141119-anello.org}{anello} commutativo con unità, e siano \(M, N, P, 0\) degli \href{20241205141053-r_moduli.org}{\(R\)-moduli}, e siano \(f,g\) dei \href{20241206115416-morfismi_r_moduli.org}{morfismi}.

\begin{itemize}
\item La \href{20250120125644-successione_di_r_moduli.org}{successione}
\begin{equation*}
\begin{tikzcd}[ampersand replacement=\&]
        0 \& M \& N
        \arrow[from=1-1, to=1-2]
        \arrow["f", from=1-2, to=1-3]
\end{tikzcd}
\end{equation*}
è \href{20250120125004-successione_di_r_moduli_esatta.org}{esatta} se e solo se \(f\) è \href{20241219101956-funzione_iniettiva.org}{iniettiva}.

\item La \href{20250120125644-successione_di_r_moduli.org}{successione}
\begin{equation*}
\begin{tikzcd}[ampersand replacement=\&]
        M \& N \& 0
        \arrow["f", from=1-1, to=1-2]
        \arrow[from=1-2, to=1-3]
\end{tikzcd}
\end{equation*}
è \href{20250120125004-successione_di_r_moduli_esatta.org}{esatta} se e solo se \(f\) è \href{20241213105600-funzione_suriettiva.org}{suriettiva}

\item La \href{20250120125644-successione_di_r_moduli.org}{successione}
\begin{equation*}
\begin{tikzcd}[ampersand replacement=\&]
        0 \& M \& N \& 0
        \arrow[from=1-1, to=1-2]
        \arrow["f", from=1-2, to=1-3]
        \arrow[from=1-3, to=1-4]
\end{tikzcd}
\end{equation*}
è \href{20250120125004-successione_di_r_moduli_esatta.org}{esatta} se e solo se \(f\) è \href{20250104111707-funzione_biunivoca.org}{biiettiva} (ovvero è un \href{20241206115416-morfismi_r_moduli.org}{isomorfismo})

\item La \href{20250120125644-successione_di_r_moduli.org}{successione}
\begin{equation*}
\begin{tikzcd}[ampersand replacement=\&]
        0 \& M \& N \& P
        \arrow[from=1-1, to=1-2]
        \arrow["f", from=1-2, to=1-3]
        \arrow["g", from=1-3, to=1-4]
\end{tikzcd}
\end{equation*}
è \href{20250120125004-successione_di_r_moduli_esatta.org}{esatta} se e solo se \(f\) è \href{20241219101956-funzione_iniettiva.org}{iniettiva} e \(\operatorname{ker}g = \operatorname{Im}f \cong M\)\footnote{Vedi: ``\href{20241206115416-morfismi_r_moduli.org}{Isomorfismo tra R-Moduli}''} se e solo se, per il \href{20250120155457-morfismo_iniettivo_di_r_moduli_induce_isomorfismo.org}{primo teorema di isomorfismo}\footnote{Vedi ``\href{20241206142802-sottomoduli.org}{Quoziente di Moduli}''}
\begin{equation*}
\operatorname{Im}g \cong N/M
\end{equation*}

\item La \href{20250120125644-successione_di_r_moduli.org}{successione}
\begin{equation*}
\begin{tikzcd}[ampersand replacement=\&]
        0 \& M \& N \& P \& 0
        \arrow[from=1-1, to=1-2]
        \arrow["f", from=1-2, to=1-3]
        \arrow["g", from=1-3, to=1-4]
        \arrow[from=1-4, to=1-5]
\end{tikzcd}
\end{equation*}
è \href{20250120125004-successione_di_r_moduli_esatta.org}{esatta} se e solo se \(\operatorname{Im} g \cong N/M\), \(f\) \href{20241219101956-funzione_iniettiva.org}{iniettiva} e \(g\) \href{20241213105600-funzione_suriettiva.org}{suriettiva}, se e solo se \(P\cong N/M\).
\end{itemize}
\end{thm}
\end{document}
