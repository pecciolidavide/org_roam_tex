% Intended LaTeX compiler: pdflatex
\documentclass[../main]{subfiles}

\usepackage[hyperref]{biblatex}
\date{}
\title{}
\begin{document}

\section{Insieme analitico}
\label{sec:orgdbe5a9b}
\subsection{Definizione}
\label{sec:org3078f9d}

Sia \(X\) uno spazio topologico metrizzabile e separabile.
\begin{itemize}
\item \(A \subseteq X\) è detto \uline{analitico} se esiste uno spazio Polacco \(Y\) e una funzione continua
\begin{equation*}
f:Y\to X
\end{equation*}
tale che \(f(Y) = A\).
\item \(C \subseteq X\) è detto \uline{coanalitico} se \(X\setminus C\) è analitico.
\item \(B \subseteq X\) è detto \uline{bianalitico} se \(B\) e \(X\setminus B\) sono analitici.
\end{itemize}

La collezione dei sottoinsiemi di \(X\) analitici, coanalitici e bianalitici è indicata, rispettivamente, da
\begin{equation*}
\bm{\Sigma}_{1}^{1}(X),\quad \bm{\Pi}_{1}^{1}(X),\quad \bm{\Delta}_{1}^{1}(X)
\end{equation*}
\subsection{Proposizione}
\label{sec:org9d2f46b}

Sia \(X\) uno spazio Polacco, e sia \(\emptyset \neq A \subseteq X\). Sono fatti equivalenti:
\begin{enumerate}
\item \(A\) è analitico;
\item \(A\) è immagine continua di \(\omega^{\omega}\);
\item \(A= \pi_{X}(C)\) per qualche \(C \in \bm{\Pi}_{1}^{0}\left(X\times \omega^{\omega}\right)\), \(C\neq \emptyset\), dove \(\pi_{X}\) è la proiezione su \(X\)
\item \(A = \pi_{X}(C)\) per qualche spazio Polacco \(Y\) e per qualche \(C \in \bm{{\operatorname{Bor}}}(X\times Y)\), \(C\neq \emptyset\);
\item \(A=f(C)\) per qualche spazio Polacco \(Y\), per qualche \(C \in \bm{{\operatorname{Bor}}}(X\times Y)\) e per qualche funzione Boreliana \(f:Y\to X\).
\end{enumerate}
\end{document}
