% Intended LaTeX compiler: pdflatex
\documentclass[../main]{subfiles}


\begin{document}

\def\U{\mathcal{U}}
\def\L{\mathcal{L}}
\def\X{\mathcal{X}}
\def\Z{\mathcal{Z}}
\def\D{\mathcal{D}}
%
\def\eq{{\rm eq}}
\def\Ueq{\U^\eq}
%
\def\orbita{\mathcal{O}}
\def\Aut{\operatorname{Aut}}
%
\def\tc{\mid}
\def\tp{\operatorname{tp}}
\def\EMtp{\operatorname{EM}\text{-}\operatorname{tp}}
\def\<{\langle}
\def\>{\rangle}
%
\def\restricted#1{\,\mathord{\upharpoonright}{{\scriptstyle #1}}}
\def\equivalentover#1{\mathrel{\equiv_{ #1 }}}

%% NON FORKING
\def\nonforkSymbol{%
\mathbin{\raise1.8ex%
\rlap{\kern0.6ex\rule{0.6ex}{0.1ex}}%
\rlap{\kern1.1ex\rule{0.1ex}{1.9ex}}\raise-0.3ex\hbox{$\smile$}}}
\def\defaultnonforkmodel{M}
\def\nonfork{\nonforkSymbol}
\renewcommand{\nonfork}[1][\defaultnonforkmodel]{%
\mathrel{\nonforkSymbol_{#1}}}
\section{Insieme approssimabile da una relazione in un Modello Mostro}
\label{sec:orgf07e90f}
Si utilizza la \href{20250612143636-notazione_teoria_dei_modelli.org}{Notazione della TEORIA DEI MODELLI}

Sia \(\L\) un \href{20250130162057-linguaggio_del_prim_ordine.org}{linguaggio}, \(T\) una \href{20250130114950-teoria_del_prim_ordine.org}{teoria} \href{20250131123151-teoria_completa.org}{completa} senza \href{20250131122945-modello_di_un_insieme_di_formule.org}{modelli} finiti e \(\U\) un \href{20250617095548-modello_lambda_saturo.org}{modello saturo} di \href{20241213101756-cardinalita.org}{cardinalità} \href{20250211123155-cardinale_limite_forte.org}{inaccessibile} \(\kappa>\card{\L}+ \omega\). \(\U\) è un \href{20250617102733-modello_mostro.org}{modello mostro}.

Sia \(A \subseteq \U\) piccolo.

Una qualsiasi \href{20250131103317-formula_del_prim_ordine.org}{formula} \(\psi(x;z)\) si indende come \href{20250202170607-classe_relazione_binaria.org}{relazione} binaria, vedendola come
\begin{equation*}
\psi(\U^{x};\U^{z}) \subseteq \U^{x} \times \U^{z}.
\end{equation*}

\begin{prop}
Sia \(\psi(x;z) \in \L(A)\), \(\D \subseteq \U^{z}\). LSASE:
\begin{enumerate}
\item \(\D\) \href{20251113150652-insieme_approssimabile_da_una_relazione.org}{approssimabile} da \(\psi\);
\item \(\D\) è \href{20251113175327-insieme_esternamente_definibile_in_un_modello_mostro.org}{esternamente definibile} da \(\psi\).
\end{enumerate}
\end{prop}
\begin{proof}
(\(2.\Rightarrow 1.\)):
Se \(\D=\D_{p,\psi}\), siano allora
\(B \subseteq \U^{z}\) finito e
\(a\vDash p(x)\restricted{A,B}\)\footnote{Vedi ``\href{20251029160457-restrizione_di_un_tipo_ad_un_insieme_di_parametri.org}{Restrizione di un tipo ad un insieme di parametri}''}. Questo esiste per \href{20250617095548-modello_lambda_saturo.org}{saturazione}.

Allora \(\pi(a, B) = \D\cap B\), in quanto
\begin{equation*}
\forall b \in B\ %
\big[\pi(a,b) \implies b \in \D\big]%
\land %
\big[\lnot\pi(a,b) \implies b \notin \D\big]
\end{equation*}
Questo è ovvio poiché \(p\) è completo e \(a\vDash p(x) \restricted{A,B}\).

(\(1.\Rightarrow 2.\)): Sia\footnote{Con \(S_{\psi}(\U)\) si intende l'\href{20251026161720-phi_formula_su_un_insieme.org}{insieme dei \(\psi\)-tipi globali}.}
\begin{equation*}
S_{\psi}(\U)\ni p(x) \supseteq \set{\pi(x;b) \mid b \in \D} \cup \set{\lnot\pi(x;b) \mid b \notin \D}.
\end{equation*}
Sia \(B \subseteq \U^{z}\) finito. Allora
\begin{equation*}
p(x) \restricted{B} \supseteq \set{\pi(x;b) \mid b \in \D\cap B} \cup \set{\lnot\pi(x;b) \mid b \in B\setminus\D}
\end{equation*}
Ma se \(a\) è tale che \(\pi(a,B) = \D\cap B\), allora \(a\vDash p(x) \restricted{B}\), e quindi \(p(x)\) è finitamente consistente.
\end{proof}
\end{document}
