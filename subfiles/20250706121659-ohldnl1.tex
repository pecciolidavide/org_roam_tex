% Intended LaTeX compiler: pdflatex
\documentclass[../main]{subfiles}


\begin{document}

\begin{prop}
Sia \(\sigma:\R\to\R\) una \href{20250202170607-classe_relazione_binaria.org}{funzione} \href{20250625105528-funzione_discriminatoria_per_una_misura_di_baire_sul_cubo_unitario.org}{discriminatoria in senso \(L^{1}\)}. Allora l'insieme
\begin{equation*}
\set{
G(\bm{x}) = \sum_{j=1}^{N}\alpha_{j}\sigma(\bm{w}_{j}\cdot \bm{x} + \theta_{j})\mid N \in \N, \bm{w}_{j} \in \R^{n}, \alpha_{j},\theta_{j} \in \R
}
\end{equation*}
è \href{20250301193045-sottoinsieme_denso.org}{denso} in \(L^{1}(I_{n})\)\footnote{Con \(I_{n}\) si indica il cubo
\begin{equation*}
I_{n} = [0,1]\times \dots \times [0,1] = [0,1]^{n}.
\end{equation*}
Si vedano ``\href{20250624162220-spazi_lp.org}{Spazi Lp}''\label{org7d673b2}}.
\label{prop9.3.17}
\end{prop}
Siccome ogni \href{20250202170607-classe_relazione_binaria.org}{funzione} \href{20250625110110-funzione_sigmoidale.org}{sigmoidale}, \href{20250704104947-funzione_misurabile.org}{misurabile} e \href{20250704145518-funzione_limitata.org}{limitata} è \href{20250625105528-funzione_discriminatoria_per_una_misura_di_baire_sul_cubo_unitario.org}{discriminatoria in senso \(L^{1}\)}, segue il corollario.

\begin{cor}
Sia \(\sigma:\R\to\R\) una \href{20250202170607-classe_relazione_binaria.org}{funzione} \href{20250625110110-funzione_sigmoidale.org}{sigmoidale}, \href{20250704104947-funzione_misurabile.org}{misurabile} e \href{20250704145518-funzione_limitata.org}{limitata}. Allora l'insieme
\begin{equation*}
\set{
G(\bm{x}) = \sum_{j=1}^{N}\alpha_{j}\sigma(\bm{w}_{j}\cdot \bm{x} + \theta_{j})\mid N \in \N, \bm{w}_{j} \in \R^{n}, \alpha_{j},\theta_{j} \in \R
}
\end{equation*}
è \href{20250301193045-sottoinsieme_denso.org}{denso} in \(L^{1}(I_{n})\)\textsuperscript{\ref{org7d673b2}}.
\label{thm9.3.18}
\end{cor}
\end{document}
