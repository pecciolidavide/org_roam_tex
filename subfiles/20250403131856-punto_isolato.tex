% Intended LaTeX compiler: pdflatex
\documentclass[../main]{subfiles}


\begin{document}

\section{Punto isolato}
\label{sec:orga429bc6}
\begin{definizione}
Sia \(X\) uno \href{20250103145124-topologia.org}{spazio topologico} e sia \(S \subseteq X\). \(x_{0} \in S\) si dice \textbf{punto isolato di \(S\)} se esiste \(U\) \href{20250111142313-intorno.org}{intorno} \href{20250103145124-topologia.org}{aperto} di \(x_{0}\) in \(X\) tale che \(S\cap U = \set{x_{0}}\).
\end{definizione}

Equivalentemente, \(x_{0}\) è un punto isolato di \(S\) se \(\set{x_{0}}\) è aperto nella \href{20250103163814-sottospazio_topologico.org}{topologia di sottospazio} di \(S\).

\begin{prop}
Sia \(x_{0} \in S\). Sono fatti equivalenti:
\begin{enumerate}
\item \(x_{0}\) è un punto isolato di \(S\);
\item \(x_{0}\) non è un \href{20260128120739-punto_di_accumulazione.org}{punto di accumulazione} per \(S\).
\end{enumerate}
\end{prop}

\begin{proof}
(\(2.\Rightarrow 1.\)): Sia \(U\) l'intorno aperto di \(x_{0}\) in \(X\) tale che
\begin{equation*}
U\cap S = \set{x_{0}}.
\end{equation*}
Allora \((U\setminus\set{x_{0}})\cap S = \emptyset\), e pertanto \(x_{0}\) non è un punto di accumulazione.

(\(2.\Rightarrow 1.\)): Siccome \(x_{0}\) non è un punto di accumulazione per \(S\), allora esiste \(U\) intorno aperto di \(x_{0}\) tale che
\begin{equation*}
(U\setminus\set{x_{0}})\cap S = \emptyset
\end{equation*}
Siccome \(x_{0} \in U\) e \(x_{0} \in S\), allora
\begin{equation*}
U\cap S = \set{x_{0}}.\qedhere
\end{equation*}
\end{proof}

\begin{definizione}
Il punto \(x_{0}\) si dice \textbf{punto isolato di \(X\)} se \(\set{x_{0}}\) è un aperto di \(X\).
\end{definizione}
\end{document}
