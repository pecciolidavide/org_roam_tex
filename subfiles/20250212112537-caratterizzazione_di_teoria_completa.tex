% Intended LaTeX compiler: pdflatex
\documentclass[../main]{subfiles}


\begin{document}

Sia \(\mathcal{L}\) un \href{20250130162057-linguaggio_del_prim_ordine.org}{linguaggio del prim'ordine}, e sia \(T\) una \(\mathcal{L}\)-\href{20250130114950-teoria_del_prim_ordine.org}{teoria}.
\section{Teorema}
\label{sec:org0d8a89d}
Sono fatti equivalenti:
\begin{enumerate}
\item \(T\) è \href{20250131123151-teoria_completa.org}{completa};
\item esiste un'unica \href{20250130114950-teoria_del_prim_ordine.org}{teoria} \href{20250131123128-teoria_soddisfacibile.org}{massimamente consistente} \(S\) tale che \(T \subseteq S\);
\item \(T\) è {[}BROKEN LINK: dee43b09-d515-4385-b1fa-6652a8da5773] e per ogni \href{20250131122945-modello_di_un_insieme_di_formule.org}{modello} \(M\vDash T\), \(T\vdash \operatorname{Th}(M)\) (vedi \href{20250131123011-conseguenza_logica.org}{Conseguenza logica});
\item \(T\) è {[}BROKEN LINK: dee43b09-d515-4385-b1fa-6652a8da5773] e per ogni \(\mathcal{L}\)-\href{20250131103446-enunciato_del_prim_ordine.org}{enunciato} \(\varphi\), \(T\vdash \varphi\) oppure \(T\vdash \lnot\,\varphi\) (vedi \href{20250131123011-conseguenza_logica.org}{Conseguenza logica});
\item \(T\) è {[}BROKEN LINK: dee43b09-d515-4385-b1fa-6652a8da5773] e per ogni coppia \(M, N\) di \href{20250131122945-modello_di_un_insieme_di_formule.org}{modelli} di \(T\), \(M\equiv N\) (vedi \href{20250131123208-teorie_elementarmente_equivalente.org}{Strutture elementarmente equivalenti}).
\end{enumerate}
\end{document}
