% Intended LaTeX compiler: pdflatex
\documentclass[../main]{subfiles}

\usepackage[hyperref]{biblatex}
\date{}
\title{}
\begin{document}

\section{Varietà Algebriche luogo di zeri di finiti polinomi}
\label{sec:org77c447b}
Sia \(\K\) un \href{20241231112713-campo_algebricamente_chiuso.org}{campo algebricamente chiuso}.
\subsection{\href{20241231114256-varieta_algebrica_affine.org}{Varietà algebriche affini}}
\label{sec:orge895cf9}
Una \href{20241231114256-varieta_algebrica_affine.org}{Varietà Algebrica Affine} è il \href{20241231112823-radici_polinomiali.org}{Luogo di zeri} di un numero finito di \href{20241231112750-polinomio.org}{polinomi}.
\subsubsection{Dimostrazione}
\label{sec:orge73ec7c}
Vedi il corollario di ``\href{20250102180050-zeri_di_un_ideale_generato_in_uno_spazio_affine.org}{Zeri di un ideale generato in uno spazio affine}''
\subsection{\href{20241231123223-varieta_algebrica_proiettiva.org}{Varietà Algebriche proiettive}}
\label{sec:org1cc011a}
Una \href{20241231123223-varieta_algebrica_proiettiva.org}{Varietà Algebrica Proiettiva} è il \href{20241231112823-radici_polinomiali.org}{Luogo di zeri} di un numero finito di \href{20241231121125-polinomi_omogenei.org}{Polinomi Omogenei}.
\subsubsection{Dimostrazione}
\label{sec:org0973dab}
Si è visto che ogni varietà algebrica proiettiva è luogo di zeri di un ideale omogeneo. Per la \href{20250102182726-ideale_di_polinomi_omogeneo.org}{caratterizzazione degli ideali omogenei}, ogni ideale omogeneo è \href{20241219113154-ideale_generato.org}{generato} da polinomi omogenei, ma siccome l'\href{20241219113434-anello_dei_polinomi.org}{Anello-dei-polinomi} è \href{20250102115942-anello_noetheriano.org}{noetheriano} (grazie al \href{20250102165420-teorema_della_base_di_hilbert.org}{Teorema della Base di Hilbert}), si ha che ogni ideale generato è \href{20241219113154-ideale_generato.org}{finitamente generato} (per il \href{20250102165143-ideali_generati_di_un_anello_noetheriano.org}{teorema sugli ideali generati in un anello noetheriano}).
\end{document}
