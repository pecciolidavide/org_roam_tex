% Intended LaTeX compiler: pdflatex
\documentclass[../main]{subfiles}


\begin{document}

\section{Spazi topologici omotopicamente equivalenti}
\label{sec:orgb5afe09}
Due \href{20250103145124-topologia.org}{spazi topologici} \(X,Y\) si dicono \textbf{omotopicamente equivalenti} se esistono due \href{20250103103252-funzione_continua.org}{funzioni continue}
\begin{equation*}
\begin{tikzcd}[ampersand replacement=\&,cramped,column sep=3.15em]
	X \& Y
	\arrow["f", shift left=2, from=1-1, to=1-2]
	\arrow["g", shift left=2, from=1-2, to=1-1]
\end{tikzcd}
\end{equation*}
tali che le loro composizioni siano \href{20250121094654-omotopia_tra_funzioni_continue.org}{omotope} all'identità:
\begin{equation*}
g\circ f\sim \operatorname{Id}_{X},\qquad f\circ g \sim \operatorname{Id}_{Y}
\end{equation*}
ovvero esistono continue
\begin{align*}
G: X\times(0,1) &\longrightarrow X & H: Y\times(0,1)&\longrightarrow Y\\
(x,0) &\longmapsto g\circ f(x) & (y,0) & \longmapsto f\circ g(y)\\
(x,1) &\longmapsto x & (y,1) & \longmapsto y.
\end{align*}

Le due funzioni\(f,g\) si dicono equivalenze omotopiche
\end{document}
