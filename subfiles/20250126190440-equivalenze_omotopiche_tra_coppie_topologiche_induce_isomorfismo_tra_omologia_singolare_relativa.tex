% Intended LaTeX compiler: pdflatex
\documentclass[../main]{subfiles}

\usepackage[hyperref]{biblatex}
\date{}
\title{}
\begin{document}

\section{Equivalenza omotopica tra coppie topologiche induce isomorfismo tra omologia singolare relativa}
\label{sec:orgf2c2d8d}
Sia \(R\) un \href{20241219112842-pid.org}{PID}.
\begin{prop}
Siano \((X,A),(Y,B)\) \href{20250122154728-coppia_topologica.org}{coppie topologiche}, e sia
\begin{equation*}
f:(X,A)\longrightarrow(Y,B)
\end{equation*}
un \href{20250122160145-categoria_topp.org}{morfismo} tale che
\begin{align*}
f:X&\longrightarrow Y\\
\restriction{f}{A}:A&\longrightarrow B
\end{align*}
siano \href{20250124155008-spazi_topologici_omotopicamente_equivalenti.org}{equivalenze omotopiche}.

Allora applicando il \href{20241205131958-funtore.org}{funtore} di \href{20250126191208-funtore_da_topp_a_rmod_di_omologia.org}{omologia}, si ottiene che per ogni \(n\)
\begin{equation*}
\overline{f_{\star}}: H_{n}(X,A) \longrightarrow H_{n}(Y,B)
\end{equation*}
è un \href{20241206115416-morfismi_r_moduli.org}{isomorfismo}.
\end{prop}
\begin{proof}
Si hanno le seguenti successioni \href{20250120125004-successione_di_r_moduli_esatta.org}{esatte}:\footnote{La commutatività del diagramma è data da ``\href{20250124163710-diagramma_commutativo_dell_omologia_di_due_coppie_topologiche.org}{Morfismo tra coppie topologiche induce diagramma commutativo in omologia}''. Vedi anche:
\begin{itemize}
\item \href{20250122133631-omologia_singolare.org}{Omologia Singolare}
\item \href{20250122154903-omologia_singolare_relativa.org}{Omologia Singolare Relativa}
\end{itemize}}
\begin{equation*}
\begin{tikzcd}[ampersand replacement=\&,cramped]
	{H_n(A)} \& {H_n(X)} \& {H_n(X,A)} \& {H_{n-1}(A)} \& {H_{n-1}(X)} \\
	{H_n(B)} \& {H_n(Y)} \& {H_n(Y,B)} \& {H_{n-1}(B)} \& {H_{n-1}(Y)}
	\arrow[from=1-1, to=1-2]
	\arrow[from=1-2, to=1-3]
	\arrow[from=1-3, to=1-4]
	\arrow[from=1-4, to=1-5]
	\arrow[from=2-1, to=2-2]
	\arrow[from=2-2, to=2-3]
	\arrow[from=2-3, to=2-4]
	\arrow[from=2-4, to=2-5]
\end{tikzcd}
\end{equation*}

Per il \href{20250122155528-teorema_dell_invarianza_per_omotopia.org}{corollario} del \href{20250122155528-teorema_dell_invarianza_per_omotopia.org}{teorema di invarianza per omotopia}, le funzioni \(f_{\star}, (\restriction{f}{A})_{\star}\)\footnote{Ottenute applicando il \href{20250123115927-funtore_di_omologia_singolare.org}{funtore di omologia singolare}.} sono \href{20241206115416-morfismi_r_moduli.org}{isomorfismi}, e pertanto il diagramma seguente è commutativo e a righe esatte:
\begin{equation*}
\begin{tikzcd}[ampersand replacement=\&,cramped,column sep=large,row sep=huge]
	{H_n(A)} \& {H_n(X)} \& {H_n(X,A)} \& {H_{n-1}(A)} \& {H_{n-1}(X)} \\
	{H_n(B)} \& {H_n(Y)} \& {H_n(Y,B)} \& {H_{n-1}(B)} \& {H_{n-1}(Y)}
	\arrow[from=1-1, to=1-2]
	\arrow["{(\restriction{f}{A})_{\star}}"', from=1-1, to=2-1]
	\arrow["\cong", draw=none, from=1-1, to=2-1]
	\arrow[from=1-2, to=1-3]
	\arrow["{f_{\star}}"', from=1-2, to=2-2]
	\arrow["\cong", draw=none, from=1-2, to=2-2]
	\arrow[from=1-3, to=1-4]
	\arrow["{\overline{f_{\star}}}"', from=1-3, to=2-3]
	\arrow[from=1-4, to=1-5]
	\arrow["{(\restriction{f}{A})_{\star}}", from=1-4, to=2-4]
	\arrow["\cong"', draw=none, from=1-4, to=2-4]
	\arrow["{f_{\star}}", from=1-5, to=2-5]
	\arrow["\cong"', draw=none, from=1-5, to=2-5]
	\arrow[from=2-1, to=2-2]
	\arrow[from=2-2, to=2-3]
	\arrow[from=2-3, to=2-4]
	\arrow[from=2-4, to=2-5]
\end{tikzcd}
\end{equation*}
e dunque, applicando il \href{20250120155801-lemma_del_cinque.org}{lemma del cinque}, \(\overline{f_{\star}}\) è un isomorfismo.
\end{proof}
\end{document}
