% Intended LaTeX compiler: pdflatex
\documentclass[../main]{subfiles}

\def\Im{\operatorname{Im}}


\begin{document}

\section{Successione di spazi vettoriali esatta}
\label{sec:orgc80a1d7}
\begin{definizione}
Siano \(V,W,U\) degli spazi vettoriali, e siano \(f:V\to W\) e \(g:W\to U\) due funzioni lineari. La successione
\begin{equation*}
V \xrightarrow{f} W \xrightarrow{g} U
\end{equation*}
si dice \uline{esatta in \(W\)} se\footnote{Vedi:
\begin{itemize}
\item \href{20250202173528-dominio_range_e_campo_di_una_classe_relazione.org}{Range di una funzione}
\item \href{20251121143525-kernel_di_una_funzione_tra_spazi_vettoriali.org}{Kernel di una funzione tra spazi vettoriali}
\end{itemize}}
\begin{equation*}
\Im f = \ker g
\end{equation*}
\end{definizione}

\begin{oss}
Se
\begin{equation*}
\begin{tikzcd}
	0 & V & W & U & 0
	\arrow[from=1-1, to=1-2]
	\arrow["f", from=1-2, to=1-3]
	\arrow["g", from=1-3, to=1-4]
	\arrow[from=1-4, to=1-5]
\end{tikzcd}
\end{equation*}
è una sucessione esatta in ogni suo termine, si chiama \uline{successione esatta corta} e si ha che
\begin{itemize}
\item \(f\) è iniettiva;
\item \(g\) è suriettiva;
\item \(\Im f = \ker g\).
\end{itemize}
\end{oss}
\begin{oss}
Data una successione di spazi vettoriali (di dimensione finita) esatta \(\big\langle (V^{k}, d^{k}) \big\rangle\), si ha che (dal teorema di nullità più rango), le \href{20241205142027-spazio_vettoriale.org}{dimensioni} rispettano:
\begin{equation*}
\sum (-1)^{i} \dim V^{i} = 0.
\end{equation*}
\end{oss}
\end{document}
