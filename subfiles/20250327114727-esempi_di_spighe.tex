% Intended LaTeX compiler: pdflatex
\documentclass[../main]{subfiles}


\begin{document}

\section{Spighe del fascio delle funzioni Cinfinito}
\label{sec:org241c92c}
\begin{prop}
Se \(X\) è una \href{20250113115909-struttura_differenziabile.org}{varietà differenziabile} e \(\mathcal{C}_{X}^{\infty}\) è il \href{20250324174728-fascio.org}{fascio} \href{20250324175845-esempi_di_prefasci.org}{delle funzioni \(\mathcal{C}^{\infty}\)}, allora per ogni \(p \in Z\) la \href{20250325183434-spiga_di_un_prefascio.org}{spiga} \((\mathcal{C}_{X}^{\infty})_{p}\) è\footnote{Vedi ``\href{20250114100254-germi_di_funzioni.org}{Germi di funzioni}''}
\begin{equation*}
(\mathcal{C}_{X}^{\infty})_{p} = \set{\text{germi delle funzioni \(C^{\infty}\) in \(p\)}}.
\end{equation*}
\end{prop}

Questo ci dice che il concetto di \href{20250114100254-germi_di_funzioni.org}{germe di funzione} è un caso particolare del \href{20250325183434-spiga_di_un_prefascio.org}{germe di una spiga}.
\section{Spighe del fascio localmente costante}
\label{sec:orga88525b}
Sia \(X\) una \href{20250111092123-varieta_topologica.org}{varietà topologica}, e sia \(\uline{\R}\) il \href{20250325171002-fascio_di_gruppo_localmente_costante.org}{fascio delle funzioni localmente costanti}.

\begin{prop}
Per ogni \(p \in X\), si ha \href{20241206115531-morfismo_di_gruppi.org}{l'isomorfismo} (di gruppi)
\begin{equation*}
\uline{\R}_{p} \cong \R
\end{equation*}
dove \(\uline{\R}_{p}\) è la \href{20250325183434-spiga_di_un_prefascio.org}{spiga di \(\uline{\R}\) in \(p\)}.
\end{prop}
\section{Spighe del fascio grattacielo}
\label{sec:org447744c}
Sia \(X\) una \href{20250111092123-varieta_topologica.org}{varietà topologica}, e \(\mathcal{F}\) il \href{20250325171249-fascio_grattacielo.org}{fascio grattacielo} centrato in \(p \in X\) con gruppo \(\R\).

\begin{prop}
Per ogni \(q \in X\), la \href{20250325183434-spiga_di_un_prefascio.org}{spiga} \(\mathcal{F}_{q}\) è
\begin{equation*}
\mathcal{F}_{q} = \begin{cases}
0 & q\neq p\\
\R & q=p.
\end{cases}
\end{equation*}
\end{prop}
\section{Spieghe del fascio delle funzioni olomorfe}
\label{sec:orgfbbfe24}
\begin{prop}
Se \(\mathcal{O}_{\C}\) è il \href{20250324175845-esempi_di_prefasci.org}{fascio delle funzioni olomorfe in \(\C\)}, e \(z_{0} \in \C\), allora la spiga \(\mathcal{O}_{\C,z_{0}}\) è isomorfa all'anello delle serie di potenze convergenti in \((z-z_{0})\).
\end{prop}
\end{document}
