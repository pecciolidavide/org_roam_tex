% Intended LaTeX compiler: pdflatex
\documentclass[../main]{subfiles}

\usepackage[hyperref]{biblatex}
\date{}
\title{}
\begin{document}

\section{Generalizzazione schema di Ricorsione tramite la funzione memoria}
\label{sec:org7ba976b}
\subsection{Proposizione}
\label{sec:orgf91423a}

Siano \(g:\N^{k}\to \N\) e \(h:\N^{k+2}\to \N\) \href{20250202170607-classe_relazione_binaria.org}{funzioni} \href{20250207104855-funzione_ricorsiva.org}{ricorsive} (\href{20250215141024-funzioni_primitive_ricordive.org}{primitive}). Allora la funzione \(f:\N^{k+1}\to \N\) definita da
\begin{equation*}
\begin{cases}
f(\bm{x},0) = g(\bm{x})\\
f(\bm{x},y+1) = h\left(\bm{x}, y, f^{\text{m}}(\bm{x},y)\right)
\end{cases}
\end{equation*}
è ricorsiva (primitiva), dove \(f^{\text{m}}\) è la \href{20250601160026-funzione_memoria.org}{funzione memoria}.
\end{document}
