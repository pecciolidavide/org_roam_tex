% Intended LaTeX compiler: pdflatex
\documentclass[../main]{subfiles}


\begin{document}

\section{Somma ordini di una funzione meromorfa su una superficie di Riemann è nulla}
\label{sec:org3bb1daf}
\begin{prop}
Sia \(X\) una \href{20260127112828-superficie_di_riemann.org}{superficie di Riemann} \href{20250103163701-spazio_topologico_compatto.org}{compatta} e sia \(f \in \mathcal{M}_{X}(X)\)\footnote{Vedi
\begin{itemize}
\item \href{20260128144151-fascio_delle_funzioni_meromorfe_su_una_superficie_di_riemann.org}{Fascio delle funzioni meromorfe su una superficie di Riemann}.
\end{itemize}} una \href{20260128144105-funzione_meromorfa_su_una_superficie_di_riemann.org}{funzione meromorfa} e non nulla. Allora la somma degli \href{20260128144450-ordine_di_una_funzione_meromorfa_su_una_superficie_di_riemann.org}{ordini}:
\begin{equation*}
\sum_{p \in X} \mathrm{ord}_{p}\, f = 0
\end{equation*}
\end{prop}

\begin{proof}
La funzione \(f\) \href{20260128183731-corrispondenza_funzioni_meromorfe_su_una_superficie_di_riemann_e_funzioni_olomorfe_sulla_sfera_di_riemann.org}{induce} la funzione \href{20260128143822-funzione_olomorfa_su_una_superficie_di_riemann.org}{olomorfa}\footnote{Vedi
\begin{itemize}
\item \href{20260127112905-sfera_di_riemann.org}{Sfera di Riemann}
\item \href{20260128181135-sfera_di_riemann_biolomorfa_al_piano_proiettivo_complesso.org}{Sfera di Riemann biolomorfa al piano proiettivo complesso}
\item \href{20260127112924-esempi_fondamentali_di_varieta_complesse.org}{Spazio proiettivo complesso è una varietà complessa}.
\end{itemize}
Vedi anche
\begin{itemize}
\item \href{20260128154601-singolarita_isolata_analisi_complessa.org}{Polo (Analisi Complessa)}.
\end{itemize}}
\begin{align*}
F: X &\longrightarrow \C_{\infty} \cong \mathds{P}^{1}_{\C}\\
x &\longmapsto \big(f(x):1\big)\\
x\text{ polo} &\longmapsto (1:0)
\end{align*}
\begin{itemize}
\item Se \(x_{0}\) è uno zero di \(f\), allora: \(F(x_{0}) \in U_{1} = \set{z_{1} \neq 0}\): per \(\varphi\) carta locale di \(X\) centrata in \(x_{0}\):
\begin{equation*}
\begin{tikzcd}[ampersand replacement=\&,row sep=tiny]
        \C \&\& X \&\& {U_1 \subseteq \mathds{P}^1_{\C}} \&\& \C \\
        {0} \&\& {x_0} \&\& {\big(f(x_0)\duepunti 1\big)} \&\& {f(x_0)}
        \arrow["\varphi"', from=1-3, to=1-1]
        \arrow["F", from=1-3, to=1-5]
        \arrow["{(z_0\duepunti z_1) \mapsto \frac{z_0}{z_1}}", from=1-5, to=1-7]
        \arrow[maps to, from=2-1, to=2-3]
        \arrow[maps to, from=2-3, to=2-5]
        \arrow[maps to, from=2-5, to=2-7]
\end{tikzcd}
\end{equation*}
e pertanto \(f(x_{0}) = f\circ\varphi^{-1}(0)\) e in un intorno di \(0\):
\begin{equation*}
  f\circ\varphi^{-1}(z) = (z)^{\mathrm{ord}_{x_{0}} f} \cdot g(z)
\end{equation*}
per \(g(z)\) olomorfa e mai nulla.

Ma questa è una scrittura locale per \(F\), e quindi\footnote{Vedi:
\begin{itemize}
\item \href{20260129104215-forma_normale_locale_per_superfici_di_riemann.org}{Molteplicità di una funzione tra superfici di Riemann}
\item \href{20260129104215-forma_normale_locale_per_superfici_di_riemann.org}{Forma normale locale per superfici di Riemann}
\end{itemize}} \(\mathrm{mult}_{x_{0}} F = \mathrm{ord}_{x_{0}} f\).

\item Se \(x_{1}\) è un polo di \(f\), allora \(F(x_{1}) = (1:0)\): esiste quindi un intorno \(U\) di \(x_{1}\) tale che:
\begin{itemize}
\item per ogni \(x \in U\): \(F(x) \in U_{0} \coloneqq \set{z_{0}\neq 0}\);

\item \(f\) è ben definita e non nulla su \(U\setminus\set{x_{1}}\)\footnote{Infatti \href{20260128144105-funzione_meromorfa_su_una_superficie_di_riemann.org}{zeri e molteplicità sono un insieme discreto}.}
\end{itemize}

e quindi, se \(g(x)\) \href{20260128143822-funzione_olomorfa_su_una_superficie_di_riemann.org}{olomorfa} è espressione locale di \(F\) in \(U_{0}\), per ogni \(x \in U\):
\begin{equation*}
  F(x) = \big(1:g(x)\big)
\end{equation*}
In particolare, \(g\) è olomorfa e \(g: X\to \C\), e \(g(x_{1}) = 0\). Pertanto, per il punto precedente
\begin{equation*}
  \mathrm{mult}_{x_{1}} F = \mathrm{ord}_{x_{1}} g.
\end{equation*}

Su \(U \setminus \set{x_{1}}\):
\begin{equation*}
  F(x) = \big(1:g(x)\big) = \big(f(x):1\big)
\end{equation*}
e quindi \(g(x) = \frac{1}{f(x)}\). Quindi \(g(x)\) è meromorfa in \(U\), e per ogni \(x \in U\):
\begin{equation*}
  \mathrm{ord}_{x}\, g = -\mathrm{ord}_{x}\, f.
\end{equation*}

In particolare \(\mathrm{mult}_{x_{1}} F = -\mathrm{ord}_{x_{1}}\, f\)
\end{itemize}

Calcolando ora la somma, e applicando il \href{20260129171444-teorema_del_grado_per_olomorfismi_tra_superfici_di_riemannn.org}{Teorema del Grado}:
\begin{equation*}
\sum_{p \in X} \mathrm{ord}_{p}\, f =
\sum_{x_0\text{ zero}} \mathrm{ord}_{x_{0}}\, f
	+ \sum_{x_{1}\text{ polo}} \mathrm{ord}_{x_{1}}\, f = %
\parentesi{=\deg F}{%
	\sum_{x_0\text{ zero}} \mathrm{mult}_{x_{0}}\, F} %
-  \parentesi{=\deg F}{%
	\sum_{x_{1}\text{ polo}} \mathrm{mult}_{x_{1}}\, F} = 0 %
\qedhere
\end{equation*}
\end{proof}
\end{document}
