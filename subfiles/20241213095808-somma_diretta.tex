% Intended LaTeX compiler: pdflatex
\documentclass[../main]{subfiles}


\begin{document}

\section{Somma Diretta di moduli}
\label{sec:org3af185c}
Sia \(R\) un \href{20241205141119-anello.org}{anello} commutativo con unità, e sia \(\set{M_{i}}_{i \in I}\) una famiglia di \href{20241205141053-r_moduli.org}{\(R\)-moduli}.

La \textbf{somma diretta} degli \(M_{i}\) è l'\(R\)-modulo dato da:
\begin{equation*}
M \coloneqq\bigoplus_{i \in I} M_{i} =
\set{(x_{i})_{i \in I} \mid x_{i} = 0 \text{ salvo un numero finito di indici}}
\end{equation*}
(dove con \((x_{i})_{i \in I}\) si intende una \href{20250115100904-successione.org}{successione}).
\begin{itemize}
\item La struttura di \href{20241205141146-gruppo_abeliano.org}{gruppo} di \(M\) è la seguente: se \((x_{i})_{i \in I},(y_{i})_{i \in I} \in M\), si definisce
\begin{equation*}
(x_{i})_{i \in I} + (y_{i})_{i \in I} \coloneqq (x_{i}+y_{i})_{i \in I}
\end{equation*}

\item La struttura di \(R\)-modulo, invece, è dato dall'azione, per ogni \(a \in R\)
\begin{equation*}
  a(x_{i})_{i \in I} \coloneqq (ax_{i})_{i \in I}.
\end{equation*}
\end{itemize}

Spesso la successione \((x_{i})_{i \in I}\) si scrive come somma formale \(\sum_{i \in I}x_{i}\).

\uline{Notazione}: quando per ogni \(i \in I\), \(M_{i} = A\), allora si denota:
\begin{equation*}
\bigoplus_{i \in I} M_{i} \eqqcolon A^{(I)}.
\end{equation*}

\uline{Notazione}: se \(I=\set{i}\), allora \(R^{(I)} \eqqcolon R\cdot i\).

\begin{oss}
Si noti che la somma diretta è associativa e commutativa (\url{https://it.wikipedia.org/wiki/Somma\_diretta})
\end{oss}
\subsection{Somma diretta dell'anello è modulo libero}
\label{sec:org4c63cfe}
\begin{prop}
Sia \(R\) un \href{20241205141119-anello.org}{anello commutativo con unità}. Allora, per ogni \href{20250130104331-insieme_mk.org}{insieme} non vuoto \(I\), la \hyperref[sec:org3af185c]{somma diretta}\footnote{\href{20260112124828-ogni_anello_ha_struttura_di_modulo_su_se_stesso.org}{Ogni anello ha struttura di modulo su sé stesso}.}
\begin{equation*}
R^{(I)}
\end{equation*}
è un \href{20241213094625-modulo_libero.org}{modulo libero} di \href{20260109173733-rango_di_un_modulo.org}{rango} \(\card{I}\)\footnote{Vedi ``\href{20241213101756-cardinalita.org}{Cardinalità}''}.
\end{prop}

\begin{proof}
Consideriamo, per ogni \(i \in I\), l'elemento \(e_{i} \in R^{(I)}\),
\begin{equation*}
e_{i} = \big(e_{i}(j)\big)_{j \in I}
\end{equation*}
tale che \(e_{i}(j) = \delta_{ij}\)\footnote{con \(\delta_{ij}\) si indica la \href{20260109154336-delta_di_kronecher.org}{Delta di Kronecher}} per ogni \(j \in I\). Allora ogni \((x_{i})_{i \in I} \in R^{(I)}\) può essere scritto come
\begin{equation*}
(x_{i})_{i \in I} = \sum_{\substack{i \in I\\ x_{i}\neq 0}} x_{i}e_{i}.
\end{equation*}
L'insieme \(\set{e_{i}\mid i \in I}\) è una base di \(R^{(I)}\), detta \uline{base canonica}.
\end{proof}

\begin{oss}
È quindi possibile considerare come base di \(R^{(I)}\) proprio l'insieme \(I\), che è possibile considerare come sottoinsieme \(I \subseteq R^{(I)}\).
\end{oss}
\subsection{Somma diretta di morfismi di moduli}
\label{sec:org58c77c4}
\begin{definizione}
Siano \(\{M_{\alpha}\}_{\alpha \in I}\) e \(\{N_{\alpha}\}_{\alpha \in I}\) due famiglie di \href{20241205141053-r_moduli.org}{\(R\)-moduli} indicizzate da un insieme \(I\) di cardinalità arbitraria.
Sia data una famiglia di \href{20241206115416-morfismi_r_moduli.org}{morfismi} \(\{f_{\alpha}: M_{\alpha} \longrightarrow N_{\alpha}\}_{\alpha \in I}\).

La \textbf{\textbf{somma diretta della famiglia di morfismi}}, denotata con \(\bigoplus_{\alpha \in I} f_{\alpha}\), è l'unico \href{20241206115416-morfismi_r_moduli.org}{morfismo di moduli}:
\begin{equation*}
\bigoplus_{\alpha \in I} f_{\alpha} : \bigoplus_{\alpha \in I} M_{\alpha} \longrightarrow \bigoplus_{\alpha \in I} N_{\alpha}
\end{equation*}
definito sugli elementi della somma diretta (ovvero le famiglie \((x_{\alpha})_{\alpha \in I}\) tali che \(x_{\alpha} \neq 0\) solo per un numero finito di indici) come:
\begin{equation*}
\left(\bigoplus_{\alpha \in I} f_{\alpha}\right)\left((x_{\alpha})_{\alpha \in I}\right) = (f_{\alpha}(x_{\alpha}))_{\alpha \in I}
\end{equation*}
\end{definizione}

La definizione è ben posta poiché, se \(x_{\alpha} = 0\) per quasi tutti gli \(\alpha\), allora anche \(f_{\alpha}(x_{\alpha}) = 0\) per quasi tutti gli \(\alpha\) (essendo \(f_{\alpha}\) morfismo, manda 0 in 0).
\end{document}
