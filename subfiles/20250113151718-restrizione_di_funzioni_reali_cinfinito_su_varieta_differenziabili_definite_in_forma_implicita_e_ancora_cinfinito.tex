% Intended LaTeX compiler: pdflatex
\documentclass[../main]{subfiles}

\usepackage[hyperref]{biblatex}
\date{}
\title{}
\begin{document}

\section{Restrizione di funzioni reali Cinfinito su varietà differenziabili definite in forma implicita è ancora Cinfinito}
\label{sec:orgd6e2556}
Sia \(M_{a} \coloneqq F^{-1}(a)\) una \href{20250113115909-struttura_differenziabile.org}{varietà differenziabile} definita in \href{20250113125903-teorema_della_funzione_implicita.org}{forma implicita}, ovvero esiste
\begin{equation*}
F: \Omega \subseteq \R^{m+n}\longrightarrow \R^{m}
\end{equation*}
\href{20250113125602-classe_c_di_una_funzione.org}{funzione \(C^{\infty}\) (come funzione tra aperti di \(\R^{N}\))}, ed esiste \(a \in F(\Omega)\) \href{20250113130125-valore_regolare_di_una_funzione.org}{valore regolare}.

Sia \(G: \Omega \longrightarrow \R^{k}\) una funzione \(C^{\infty}\) (come funzione tra aperti di \(\R^{N}\)).

Allora \(\restriction{G}{M_{a}}: M_{a}\longrightarrow \R^{k}\) è \href{20250113144722-funzioni_cinfinito_tra_varieta_differenziabili.org}{una funzione \(C^{\infty}\) tra varietà differenziabili.}
\end{document}
