% Intended LaTeX compiler: pdflatex
\documentclass[../main]{subfiles}

\usepackage[hyperref]{biblatex}
\date{}
\title{}
\begin{document}

\section{Operatori di facciata del simplesso standard}
\label{sec:org6e2359f}
Siano \(\Delta_{q}\) e \(\Delta_{q-1}\) i \href{20250121122324-simplesso_standard.org}{simplessi standard}:
\begin{equation*}
\Delta_{q-1} = [0, e_{1},\dots,e_{q-1}] \subseteq \R^{q},\qquad \Delta_{q} = [0, e_{1},\dots,e_{q}] \subseteq \R^{q+1}
\end{equation*}

\begin{definizione}
Si definiscono gli operatori di facciata opposta all'\(i\)-esimo vertice di \(\Delta_{q}\):
\begin{equation*}
\varepsilon_{0}^{q},\dots,\varepsilon_{i}^{q},\dots,\varepsilon_{q}^{q} : \Delta_{q-1}\longrightarrow \Delta_{q}
\end{equation*}
definito sui vertici come (ponendo per convenzione \(e_{0}=0\))
\begin{equation*}
\varepsilon_{j}^{q}(e_{i}) \coloneqq \begin{cases}
e_{i} &i<j\\
 e_{i+1} &i\ge j
\end{cases}
\end{equation*}
ed esteso per linearità a tutti gli altri punti.
\end{definizione}
\end{document}
