% Intended LaTeX compiler: pdflatex
\documentclass[../main]{subfiles}


\begin{document}

\section{Teoria decidibile e coerente ha estesione decidibile coerente e completa}
\label{sec:orgf5065ce}
Si adotta la notazione di ``\href{20250609104647-buona_codifica_di_un_linguaggio.org}{Aritmetizzazione della sintassi}''

Sia \(L\) un \href{20250130162057-linguaggio_del_prim_ordine.org}{linguaggio} \href{20250111143651-insieme_numerabile.org}{numerabile}. Sia \(\#\) una \href{20250609104647-buona_codifica_di_un_linguaggio.org}{buona codifica per \(L\)}.
\subsection{Teorema}
\label{sec:orgb4464bc}

Ogni \(L\)-\href{20250130114950-teoria_del_prim_ordine.org}{teoria} \href{20250610134232-teoria_decidibile.org}{decidibile} e \href{20250609162657-teoria_coerente.org}{coerente} \(T\) ha un'\href{20250130114950-teoria_del_prim_ordine.org}{estensione} \href{20250610134232-teoria_decidibile.org}{decidibile}, \href{20250609162657-teoria_coerente.org}{coerente} e \href{20250131123151-teoria_completa.org}{completa}.
\subsubsection{Idea di dimostrazione}
\label{sec:orgc963033}

Si costruisce per induzione \(T_{n}\):
\begin{itemize}
\item \(T_{0} = T\);
\item \(T_{n+1} = T_{n} \cup\set{\sigma}\) se \(\termcode{\sigma} = n+1\) e \(T_{n}\not\vdash\lnot\sigma\).
\end{itemize}

Posto \(T_{\infty} \coloneqq \bigcup_{n \in \N} T_{n}\), questa è coerente, ed è inoltre completa.

\textbf{\textbf{NON}} dimostriamo che \(T\) sia decidibile.
\end{document}
