% Intended LaTeX compiler: pdflatex
\documentclass[../main]{subfiles}

\usepackage[hyperref]{biblatex}
\date{}
\title{}
\begin{document}

\section{Strutture elementarmente equivalenti}
\label{sec:orga737b19}
Sia \(\mathcal{L}\) un \href{20250130162057-linguaggio_del_prim_ordine.org}{linguaggio del prim'ordine}.
\subsection{Definizione}
\label{sec:orgc4ba368}
Due \(\mathcal{L}\)-\href{20250131103035-struttura_del_prim_ordine.org}{strutture} \(M, N\) si dicono \texttt{elementarmente equivalenti} se
\begin{equation*}
M\vDash \varphi\,\iff\, N\vDash\varphi
\end{equation*}
per ogni \(\mathcal{L}\)-\href{20250131103446-enunciato_del_prim_ordine.org}{enunciato} \(\varphi\). (vedi \href{20250131122913-soddisfazione_di_una_formula.org}{Enunciato vero in un modello del prim'ordine}). In tal caso si scrive \(M\equiv N\).
\subsection{Definizione}
\label{sec:org9c9cf94}
Si dice che \(M, N\) sono \texttt{elementarmente equivalenti su \textbackslash{}(A\textbackslash{})} se:
\begin{enumerate}
\item \(A \subseteq M\cap N\);
\item \(M\vDash \varphi\,\iff\, N\vDash\varphi\) per ogni \href{20250212102927-enunciato_con_parametri.org}{enunciato} \(\varphi \in \mathcal{L}(A)\).
\end{enumerate}
\end{document}
