% Intended LaTeX compiler: pdflatex
\documentclass[../main]{subfiles}


\begin{document}

\section{Sottoinsieme denso}
\label{sec:orgcc883f7}
\subsection{Sottoinsieme codenso}
\label{sec:orgb0f3c42}
Sia \(X\) uno spazio topologico, \(A \subseteq X\) si dice \textbf{codenso} se \(X\setminus A\) è denso.
\subsection{Proprietà di base}
\label{sec:orge0870ed}
\subsubsection{Proprietà 1}
\label{sec:org6a318a3}
Se \(A \subseteq B \subseteq C\) \href{20250103145124-topologia.org}{spazi topologici} e \(A\) è denso in \(C\) allora \(A\) è denso in \(B\).
\subsubsection{Proprietà 2}
\label{sec:org30f9667}
Se \(A \subseteq B \subseteq C\) \href{20250103145124-topologia.org}{spazi topologici} e \(A\) è denso in \(C\) allora \(B\) è denso in \(C\).
\subsection{Caratterizzazione di un insieme denso per intorni}
\label{sec:org0a4f46e}
\end{document}
