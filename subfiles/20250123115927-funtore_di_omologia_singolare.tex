% Intended LaTeX compiler: pdflatex
\documentclass[../main]{subfiles}

\usepackage[hyperref]{biblatex}
\date{}
\title{}
\begin{document}

\section{Funtore da Top a RMod - di omologia singolare}
\label{sec:orgea87734}
Sia \(R\) un \href{20241219112842-pid.org}{PID}. Si considerino le categorie
\begin{itemize}
\item \href{20241205115600-categoria_top.org}{Categoria \(\cat{Top}\)};
\item \href{20250120163759-categoria_complessi_di_catene.org}{Categoria \(\cat{Ch}_{R}\)};
\item \href{20241206115740-categoria_degli_r_moduli.org}{Categoria \(R\text{-}\cat{Mod}\)}.
\end{itemize}

È definito il funtore
\begin{equation*}
\begin{tikzcd}[ampersand replacement=\&,row sep=tiny]
	{\cat{Top}} \& {\cat{Ch}_R} \& {R\text{-}\cat{Mod}} \\
	X \& {\mathcal{S}_{\bullet}(X)} \& {H_q(X)} \\
	{X\xrightarrow{f}Y} \& {\mathcal{S}_{\bullet}(X)\xrightarrow{f_{\diesis}} \mathcal{S}_{\bullet}(Y)} \& {H_q(X)\xrightarrow{\left((f_\diesis)_\star\right)_q} H_q(Y)}
	\arrow[color={rgb,255:red,214;green,92;blue,92}, from=1-1, to=1-2]
	\arrow[color={rgb,255:red,214;green,92;blue,92}, from=1-2, to=1-3]
	\arrow[color={rgb,255:red,214;green,92;blue,92}, maps to, from=2-1, to=2-2]
	\arrow[color={rgb,255:red,214;green,92;blue,92}, maps to, from=2-2, to=2-3]
	\arrow[color={rgb,255:red,214;green,92;blue,92}, maps to, from=3-1, to=3-2]
	\arrow[color={rgb,255:red,214;green,92;blue,92}, maps to, from=3-2, to=3-3]
\end{tikzcd}
\end{equation*}
come composizione del \href{20250122154136-funtorialita_dell_omologia_singolare.org}{funtore \((\mathcal{S}_{\bullet}, \diesis)\)} e del \href{20250120165029-funtore_tra_chr_e_rmod.org}{funtore \((H_{q}, \star)\)}.
Ponendo \(f_{\star}\coloneqq \left((f_{\diesis})_{\star}\right)_{q}\), si ha
\begin{equation*}
(H_{q},\star) : \cat{Top}\longrightarrow R\text{-}\cat{Mod}
\end{equation*}
\subsection{Omologia singolare di spazi topologici omeomorfi}
\label{sec:orgdc9d5d3}
La funtorialità garantisce che spazi topologici omeomorfi abbiano omologia singolare isomorfa.
\end{document}
