% Intended LaTeX compiler: pdflatex
\documentclass[../main]{subfiles}


\begin{document}

\def\D{\mathcal{D}}
\def\C{\mathcal{C}}
\def\B{\mathcal{C}}
\def\X{\mathcal{X}}
\def\DD{\bm{\D}}
\def\CC{\bm{\C}}
\def\BB{\bm{\B}}
\def\U{\mathcal{U}}
\def\eq{{\rm eq}}
\def\Ueq{\U^\eq}
\def\L{\mathcal{L}}
\def\orbita{\mathcal{O}}
\def\Aut{\operatorname{Aut}}
\def\tc{\mid}
\def\EMtp{\operatorname{EM}\text{-}\operatorname{tp}}
\def\tp{\operatorname{tp}}
\def\<{\langle}
\def\>{\rangle}
\def\b{\bm{b}}
\def\c{\bm{c}}
\def\d{\bm{d}}
\def\z{\overline{z}}
\def\restricted#1{\,\mathord{\upharpoonright}{{\scriptstyle #1}}}
\def\equivalentover#1{\mathrel{\equiv_{ #1 }}}

%% NON FORKING
\def\nonforkSymbol{%
\mathbin{\raise1.8ex%
\rlap{\kern0.6ex\rule{0.6ex}{0.1ex}}%
\rlap{\kern1.1ex\rule{0.1ex}{1.9ex}}\raise-0.3ex\hbox{$\smile$}}}
\def\defaultnonforkmodel{M}
\def\nonfork{\nonforkSymbol}
\renewcommand{\nonfork}[1][\defaultnonforkmodel]{%
\mathrel{\nonforkSymbol_{#1}}}

\begin{prop}
Let \(A \subseteq \U\) and let \(\<\D_i:i<\omega\>\) be an \href{20251026210908-sequenza_di_indiscernibili.org}{\(A\)-indiscernible sequence} in \href{20251028160605-eq_espansione_di_un_modello_mostro.org}{\(\U^\eq\)}.
Prove that there is a \href{20250131103317-formula_del_prim_ordine.org}{formula} \(\phi(x\,;z)\in \L\) and an \(A\)-indiscernible sequence \(\< b_i:i<\omega \>\) in \(\U^z\) such that \(\D_i=\phi(\U^{x}\,;b_i)\).
\end{prop}

\def\defaultnonforkmodel{A}

\begin{proof}
Si suppone che ciascun \(\D_{i} \subseteq \U^{x}\) per \(x\) fissata.
\begin{itemize}
\item Siccome \(\D_{i} \in \U^{\eq}\), allora esiste \(b_{i}' \in \U^{z_{i}}\) ed esiste \(\sigma_{i}(x;z_{i}) \in \L\) tale che
\begin{equation*}
  \D_{i} = \sigma_{i} (\U^{x}; b_{i}')
\end{equation*}
e, ponendo \(z\) tale che \(\card{z} = \sup_{i \in \omega} \card{z_{i}}\):
\begin{equation*}
  \D_{i} = \sigma_{i} (\U^{x}; b_{i}'), \qquad b_{i}' \in \U^{z}.
\end{equation*}
\item Il fatto che \(\DD \coloneqq \< \D_{i} : i <\omega \>\) sia una sequenza di indiscernibili su \(A\) significa che per ogni \(n \in \N\), \(I,J \in \omega^{(n)}\)
\begin{equation}
  \DD \restricted{ I } \equivalentover{A} \DD \restricted{ J }. %
  \label{eq:Dseqind}
\end{equation}
\end{itemize}

Siccome si ha
\begin{equation*}
\exists z\ \forall x\ \big[x \in \D_{0} \iff \sigma_{0}(x;z)\big]
\end{equation*}
allora per la~\eqref{eq:Dseqind} si ha che per ogni \(i \in \omega\)
(ponendo \(I = \< 0 \>\) e \(J = \< j \>\))
\begin{equation*}
\exists z\ \forall x\ \big[x \in \D_{j} \iff \sigma_{0}(x;z)\big]
\end{equation*}
ovvero, se \(\phi(x;z) \coloneqq \sigma_{0}(x;z)\) vale:
\begin{equation*}
\exists z\ \forall x\ \big[x \in \D_{j} \iff \phi(x;z)\big]
\end{equation*}

Pertanto per ogni \(\D_{i}\) esiste \(d_{i} \in \U^{z}\) tale che
\begin{equation}
\D_{i} = \phi(\U^{x}; d_{i}). %
\label{eq:Dphi}
\end{equation}
Sia \(\d\coloneqq \< d_{i} \mid i<\omega \>\) la sequenza associata.
Si consideri \(\EMtp(\d / A)\). Per il \href{20251107125444-teorema_di_ehrenfeucht_mostowski.org}{Teorema di Ehrenfeucht-Mostowski} esiste una sequenza di indiscernibili
\(\c = \< c_{i} \mid i<\omega \>\) tale che
\begin{equation*}
\EMtp(\c / A) \supseteq \EMtp(\d / A)
\end{equation*}
Sia quindi \(\C_{i} \coloneqq \phi(\U^{x}; c_{i})\) e
\(\CC \coloneqq \< \C_{i} \mid i<\omega \>\).

Si dimostra che \(\CC \equivalentover{A} \DD\).
Infatti, sia \(\psi(\X) \in \L(A)\) tale che
\begin{equation}
\vDash \psi(\CC) \IMPLICA \vDash \psi \big( \phi(\U^{x};c_{i_{1}}),\dots, \phi(\U^{x};c_{i_{n}})\big). %
\label{eq:formulasub}
\end{equation}
Se per assurdo \(\vDash \lnot\psi(\DD)\) poiché \(\DD\) è una sequenza di indiscernibili, per ogni \(j_{1}<\dots<j_{n}<\omega\)
\begin{equation*}
\vDash \lnot\psi(\D_{j_{1}},\dots,\D_{j_{n}}).
\end{equation*}
In particolar modo, quindi, per ogni \(j_{1}<\dots<j_{n}<\omega\)
\begin{equation*}
\vDash %
\lnot\psi\big( \phi(\U^{x};d_{j_{1}}),\dots, \phi(\U^{x};d_{j_{n}})\big)
\end{equation*}
e pertanto
\(\lnot\psi\big( \phi(\U^{x};z_{1}),\dots, \phi(\U^{x};z_{n})\big) \in \EMtp(\d / A)\).
Siccome \(\c \vDash \EMtp(\d / A)\) allora
\begin{equation*}
\c \vDash %
\lnot\psi\big( \phi(\U^{x};z_{1}),\dots, \phi(\U^{x};z_{n})\big).
\end{equation*}
Assurdo, per la~\eqref{eq:formulasub}.
Quindi per ogni \(\psi(\X) \in \L(A)\):
\begin{equation*}
\CC \vDash \psi(\X) \IMPLICA %
\DD \vDash \psi(\X).
\end{equation*}

Poiché \(\L(A)\) è chiuso per negazioni, questo implica che
\(\CC \equivalentover{A} \DD\):
\begin{equation*}
\orbita(\CC / A) = \orbita(\DD / A) \ni \DD
\end{equation*}
e pertanto esiste \(F \in \Aut(\U / A)\) tale che \(F\CC = \DD\).
Per ogni \(i<\omega\):
\begin{equation*}
\D_{i} = F \C_{i} = F\big[\phi(\U^{x}; c_{i})\big] = \phi(\U^{x}; F c_{i}) = \phi(\U^{x}; b_{i}).
\end{equation*}
avendo posto \(b_{i}\coloneqq Fc_{i}\).
Siccome \(F\) è un automorfismo,
e \(\< c_{i} \mid i<\omega\>\) è sequenza di indiscernibili
allora anche \(\< b_{i} \mid i<\omega\>\) lo è.
\end{proof}
\end{document}
