% Intended LaTeX compiler: pdflatex
\documentclass[../main]{subfiles}

\use{upgreek}
\def\tau{\uptau}


\begin{document}

\section{Sottospazio Topologico}
\label{sec:org0677bc3}
\begin{definizione}
Sia \((X, \tau)\) uno \href{20250103145124-topologia.org}{spazio topologico} e sia \(Y \subseteq X\) un \href{20250131155822-operazioni_insiemistiche_tra_classi_mk.org}{sottoinsieme} di \(X\).
La \textbf{\textbf{topologia di sottospazio}} (o topologia indotta) su \(Y\) è la collezione \(\tau_Y\) definita come:
\begin{equation*}
\tau_Y = \{ U \cap Y \mid U \in \tau \}
\end{equation*}
Gli elementi di \(\tau_Y\) sono detti \textbf{\textbf{aperti}} in \(Y\).
La coppia \((Y, \tau_Y)\) è detta \textbf{\textbf{sottospazio topologico}} di \(X\).
\end{definizione}

\begin{oss}
Analogamente per i chiusi: un sottoinsieme \(C \subseteq Y\) è \textbf{\textbf{chiuso}} nella topologia di sottospazio se e solo se è l'intersezione di \(Y\) con un insieme chiuso di \(X\).
\begin{equation*}
C \text{ chiuso in } Y \IFF \exists K \text{ chiuso in } X \text{ tale che } C = K \cap Y
\end{equation*}
\end{oss}
\subsection{Proprietà Transitive}
\label{sec:org35a697e}

\begin{prop}
Sia \(Y\) un sottospazio di \(X\). Valgono le seguenti proprietà di transitività:

\begin{enumerate}
\item \textbf{\textbf{Transitività dell'apertura}}: Se \(Y\) è un insieme \textbf{\textbf{aperto}} in \(X\), allora ogni sottoinsieme aperto in \(Y\) è anche aperto in \(X\).
\begin{equation*}
(Y \text{ aperto in } X) \land (A \text{ aperto in } Y) \IMPLICA A \text{ aperto in } X
\end{equation*}

\item \textbf{\textbf{Transitività della chiusura}}: Se \(Y\) è un insieme \textbf{\textbf{chiuso}} in \(X\), allora ogni sottoinsieme chiuso in \(Y\) è anche chiuso in \(X\).
\begin{equation*}
(Y \text{ chiuso in } X) \land (C \text{ chiuso in } Y) \IMPLICA C \text{ chiuso in } X
\end{equation*}
\end{enumerate}
\end{prop}
\end{document}
