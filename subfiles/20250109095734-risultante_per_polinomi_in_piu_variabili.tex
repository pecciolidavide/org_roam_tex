% Intended LaTeX compiler: pdflatex
\documentclass[../main]{subfiles}

\usepackage[hyperref]{biblatex}
\date{}
\title{}
\begin{document}

\section{Risultante per polinomi in più variabili}
\label{sec:org153f73d}
Sia \(A\) un \href{20250108173116-ufd.org}{UFD}. Allora anche \(A[x_{0},\dots,x_{\ell-{1}}]\) (vedi \href{20241219113434-anello_dei_polinomi.org}{Anello-dei-polinomi}) è un UFD, per induzione, dato che se \(A\) ufd \href{20250108173116-ufd.org}{allora} \(A[x]\) ufd.

Sia dunque \(\K\) un \href{20241205142049-campo.org}{campo} \href{20241231112713-campo_algebricamente_chiuso.org}{algebricamente chiuso}, e consideriamo \(\K[x_{0},\dots,x_{\ell}]=\K[x_{0},\dots,x_{\ell-1}][x_{\ell}]\).

Infatti per \(f \in \K[x_{0},\dots,x_{\ell}]\), si può pensare a
\begin{equation*}
f(x_{0},\dots,x_{\ell}) = \sum f_{i}(x_{0},\dots,x_{\ell-1}) x_{\ell}^{i},\qquad f_{i} \in \K[x_{0},\dots,x_{\ell-1}]
\end{equation*}

Dunque, se \(f,g \in \K[x_{0},\dots,x_{\ell}]\) si ha che (vedi \href{20250108173056-fattori_non_costanti_comuni_tra_polinomi.org}{Risultante} e \href{20250108173056-fattori_non_costanti_comuni_tra_polinomi.org}{Fattori non costanti comuni tra polinomi}) \(f\) e \(g\) hanno \href{20250108173845-fattore_comune.org}{fattori} non costanti (ovvero un \href{20241231112750-polinomio.org}{polinomio} che contiene \(x_{\ell}\)) in comune se e solo se
\begin{equation*}
\operatorname{Res}_{x_{\ell}}(f,g)=0
\end{equation*}
dove con \(\operatorname{Res}_{x_{\ell}}\) si intende il risultante calcolato con i coefficienti rispetto alla variabile \(x_{\ell}\). Infatti \(\K\) è un UFD, e dunque il risultato vale.
\subsection{Proposizione}
\label{sec:org8f02432}
Si ha che \(\operatorname{Res}_{x_{\ell}}(f,g) \in \K[x_{0},\dots,x_{\ell-1}]\) è un \href{20241231121125-polinomi_omogenei.org}{polinomio omogeneo}. Se \(m=\deg f, n=\deg G\) (vedi \href{20241231124742-grado_polinomi.org}{grado di un polinomio}) allora il grado del risultante è
\begin{equation*}
m\deg_{x_{\ell}} G + n\deg_{x_{\ell}} F - \deg_{x_{\ell}} F \cdot \deg_{x_{\ell}}G
\end{equation*}
\end{document}
