% Intended LaTeX compiler: pdflatex
\documentclass[../main]{subfiles}

\usepackage[hyperref]{biblatex}
\date{}
\title{}
\begin{document}

\section{Caratterizzazione topologia noetheriana}
\label{sec:org201ddf2}
\subsection{Lemma}
\label{sec:org29f6f61}
Uno \href{20250103145124-topologia.org}{spazio topologico} \(X\) è \href{20250103152646-spazio_topologico_noetheriano.org}{noetheriano} se e solo se ogni famiglia non vuota di \href{20250103145124-topologia.org}{chiusi} ha un \href{20250203102516-massimo_e_minimo.org}{elemento minimale} (rispetto a \(\subseteq\)). Equivalentemente, se ogni famiglia non vuota di aperti ha un elemento massimale.
\subsubsection{Dimostrazione}
\label{sec:orgfc88351}
\paragraph{``\(\implies\)''}
\label{sec:orgbe2e7e6}

Sia \(X\) uno spazio topologico noetheriano, e sia \(\mathscr{F}=\set{Y_{i}}_{i \in I}\) una famiglia di chiusi. Se per assurdo \(\mathscr{F}\) non ha elementi minimali, sia \(Y_{1} \in \mathscr{F}\).

Allora esiste \(Y_{2} \in \mathscr{F}\) tale che \(Y_{2} \subsetneqq Y_{1}\), esiste \(Y_{3} \in \mathscr{F}\) tale che \(Y_{3}\subsetneqq Y_{2}\) e così via.

Si è quindi creata una catena \(Y_{1}\supseteq Y_{2} \supseteq Y_{3} \supseteq\dots\) non stazionaria. Assurdo, poiché \(X\) noetheriano per hp.
\paragraph{``\(\impliedby\)''}
\label{sec:orgdd5b1ac}

Si consideri una catena discendente
\begin{equation*}
Y_{1} \supseteq Y_{2} \supseteq Y_{3} \supseteq \dots
\end{equation*}
di chiusi. La famiglia di chiusi \(\mathscr{F}\coloneqq\set{Y_{i}}\) ha un elemento minimale, e dunque la catena è stazionaria.
\end{document}
