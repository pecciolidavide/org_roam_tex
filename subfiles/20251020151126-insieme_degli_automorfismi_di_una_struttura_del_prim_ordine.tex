% Intended LaTeX compiler: pdflatex
\documentclass[../main]{subfiles}


\begin{document}

Sia \(\mathcal{L}\) un \href{20250130162057-linguaggio_del_prim_ordine.org}{linguaggio del prim'ordine}, e \(N\) una \(\mathcal{L}\)-\href{20250131103035-struttura_del_prim_ordine.org}{struttura}.
\section{Gruppo degli automorfismi di una struttura del prim'ordine}
\label{sec:org721a565}
\begin{definizione}
Il gruppo degli \uline{\href{20250214120959-mappe_tra_strutture_del_prim_ordine.org}{automorfismi} di \(N\)} si indica con \(\operatorname{Aut}(N)\).

Dato \(A \subseteq N\) si denota con \(\operatorname{Aut}(N/A)\) il \href{20241205141146-gruppo_abeliano.org}{gruppo}:
\begin{equation*}
\operatorname{Aut}(N/A)\coloneqq \set{f \in \operatorname{Aut}(N)\mid \forall a \in A\ fa=a}.
\end{equation*}
\end{definizione}
\section{Orbite rispetto all'azione del gruppo degli automorfismi in una struttura del prim'ordine}
\label{sec:orgecc5c73}
\begin{definizione}
L'\href{20260201185059-azione_di_gruppo.org}{orbita} di \(a \in N\) sotto l'azione di gruppo di \(\operatorname{Aut}(N)\) si indica con
\begin{equation*}
o_{N}(a) \coloneqq \set{fa\mid f \in \operatorname{Aut}(N)}.
\end{equation*}

L'orbita di \(a \in N\) sotto l'azione di gruppo di \(\operatorname{Aut}(N/A)\) si indica con
\begin{equation*}
o_{N}(a/A) \coloneqq \set{fa\mid f \in \operatorname{Aut}(N/A)}.
\end{equation*}
\end{definizione}
\end{document}
