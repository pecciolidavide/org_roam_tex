% Intended LaTeX compiler: pdflatex
\documentclass[../main]{subfiles}


\begin{document}

\section{Reverse Mathematics [CORSO SEL2025]}
\label{sec:orgbc86574}
\subsection{Introduzione}
\label{sec:org0f75d5e}

Partiamo dal Teorema di Pitagora

\begin{thm}
Dato un triangolo rettangolo di lati \(a,b,c\), si ha \(a^{2}+b^{2}=c^{2}\).
\end{thm}

Questo è dimostrato a partire dagli \uline{assiomi di Euclide}.

L'idea generale della matematica è risolvere questo problema:
\begin{equation*}
\text{Assiomi} \implies \text{Teoremi}
\end{equation*}

L'obiettivo della Reverse Mathematics è fare il processo inverso, ovvero rispondere a:
\begin{description}
\item[{(D1)}] Dato un teorema \(\varphi\), trovare gli assiomi minimi/minimali per dimostrarlo.
\end{description}

Una situazione ottimale è che esiste un gruppo di assiomi minimo \uline{equivalente} al teorema preso in questione. Questo può succedere per teoremi diversi: ad esempio, il Teorema di Pitagora è equivalente ad un gruppo degli assiomi di Euclide, e questo gruppo è equivalente al Teorema di Euclide, quindi Pitagora è equivalente ad Euclide.

Quindi, la seconda domanda è:

\begin{description}
\item[{(D2)}] Dato un teorema \(\varphi\), trovare i teoremi equivalenti.
\end{description}

Formalmente, fissato un teorema \(A\), si fissa una teoria base tale che \(T\not\vdash A\). Si cerca un insieme di assiomi \(S\) tali che
\begin{equation*}
T+S\vdash A,\qquad T+A\vdash S
\end{equation*}
e quindi \(T\vdash T\iff S\).
\subsection{Aritmetica del second'ordine \(\mathcal{L}_{2}\) e \(Z_{2}\).}
\label{sec:org5d1bbfa}

Si consideri il linguaggio del prim'ordine
\begin{equation*}
\mathcal{L}_{2} \coloneqq \set{0,S,+,\cdot,=,\le,\in}
\end{equation*}
con due tipi di variabili:
\begin{itemize}
\item variabili numeriche: \(x,y,z,n,m,\dots\).
\item variabili insiemi: \(X,Y,Z\).
\end{itemize}

\uline{Assiomi}:
\begin{itemize}
\item assiomi di base (ovvero l'Aritmetica di Robinson)
\begin{equation*}
  0+x=x,\ 0\cdot x=0,\ 0\le x\ x+y=y+x
\end{equation*}
\item assioma di induzione
\begin{description}
\item[{\(\mathrm{Ind}\)}] \(0 \in X\land \forall n [n \in X\implies S(n) \in X]\implies \forall  n (n \in X)\)
\end{description}
\item schema di induzione: per ogni formula \(\varphi \in \mathcal{L}_{2}\)
\begin{description}
\item[{\(\mathrm{Ind-Schema}\)}] \(\varphi(0)\land \forall  n[\varphi(n)\implies \varphi\big(S(n)\big)]\implies \forall n\ \varphi(n).\)
\end{description}
\item schema di comprensione: per ogni formula \(\varphi \in \mathcal{L}_{2}\)
\begin{description}
\item[{\(\mathrm{CA}\)}] \(\exists X\ \forall  n\ [n \in X\iff \varphi(n)]\)
\end{description}
\end{itemize}

\begin{esempio}
Gli assiomi di base + l'assioma di comprensione dimostrano che l'assioma di induzione è equivalente allora schema di induzione
\end{esempio}

Quindi \(Z_{2}\) è dato dagli assiomi di base + \(\mathrm{Ca}\) + \(\mathrm{Ind}\). Questa teoria però è \uline{troppo forte}, quindi si considerano, per un insieme\(\Gamma \subseteq \mathcal{L}_{2}\), gli assiomi:
\begin{description}
\item[{\(\Gamma-\mathrm{Ind}\)}] \(\varphi(0)\land \forall  n[\varphi(n)\implies \varphi\big(S(n)\big)]\implies \forall n\ \varphi(n)\) per ogni \(\varphi \in \Gamma\);
\item[{\(\Gamma-\mathrm{CA}\)}] \(\exists X\ \forall  n\ [n \in X\iff \varphi(n)]\) per ogni \(\varphi \in \Gamma\).
\end{description}

\begin{definizione}
Quantificatori limitati: \href{20250603170559-complessita_di_una_formula_del_modello_standard.org}{Quantificatore limitato nel linguaggio dell'aritmetica}
\end{definizione}

\begin{definizione}
Famiglia \(\Sigma_{n}^{0}, \Pi_{n}^{0}\): \href{20250603170559-complessita_di_una_formula_del_modello_standard.org}{Complessità di una formula nel linguaggio dell'aritmetica}
\end{definizione}

\begin{esempio}
Alcune formule
\begin{align*}
x\cdot y = y\cdot x &\in \Sigma_{0}^{0}\\
\mathrm{Pr}(t):\quad t>1 \land \forall  n\le t\ \forall  m\le t\ (t=n\cdot m \implies n=1\lor m=1) &\in \Sigma_{0}^{0}\\
\forall n\ \exists p\ (p>n \land \mathrm{Pr}(p)) \in \Pi_{2}^{0}\\
\forall n\ \big[n \in X\implies \exists m \le n\ (n=m+m)\big]&\in \Pi^{0}_{1}
\end{align*}
\end{esempio}

\begin{definizione}
Si definiscono le famiglie \(\Sigma_{n}^{1},\Pi_{n}^{1}\):
\begin{itemize}
\item \(\Sigma_{0}^{1} = \Pi_{0}^{1}\coloneqq \bigcup_{n} \Sigma_{n}^{1} = \bigcup_{n} \Pi_{n}^{1}\).
\item \(\Sigma_{n}^{1} =\)
\item \(\Pi_{n}^{1} =\) (definite come al solito)
\end{itemize}
\end{definizione}
\subsection{``Big Five''}
\label{sec:org38831a4}

L'idea è che quasi tutti i teoremi sono equivalenti a uno dei seguenti gruppi di assiomi
\subsubsection{RCA\textsubscript{0}}
\label{sec:org7d12d59}

Questa lista di assiomi è data da: Assiomi di base + \(\Sigma_{1}^{0}-\mathrm{Ind}\) + \(\Delta_{1}^{0}-\mathrm{CA}\) dove
\begin{description}
\item[{(\(\Delta_{1}^{0}-\mathrm{CA}\))}] \(\forall n\ (\varphi(n)\iff\psi(n))\implies \exists X\ \forall n [n \in X\iff \varphi(n)]\) per ogni \(\varphi \in \Sigma_{1}^{0}\) e \(\psi \in \Pi_{1}^{0}\).
\end{description}

L'idea è che, a meno di equivalenza logica, \(\Delta_{n}^{0} = \Sigma_{n}^{0}\cap \Pi_{n}^{0}\).

\begin{prop}
Sono fatti equivalenti
\begin{itemize}
\item \(\varphi(n)\in\Delta_{1}^{0}\)
\item \(\varphi(n)\) è decidibile.
\end{itemize}
\end{prop}

\begin{thm}
RCA\textsubscript{0} prova:
\begin{itemize}
\item \(\R\) non è numerabile
\item Teorema di Bolzano
\item Ogni campo numerabile ha una chiusura algebrica
\end{itemize}
e non prova:
\begin{itemize}
\item ogni anello ha un ideale primo
\item ogni sequenza non limitata di reali ha limite
\item non unicità.
\end{itemize}
\end{thm}
\subsubsection{WKL\textsubscript{0}}
\label{sec:org8d2b195}

Questa lista di assiomi è data da RCA\textsubscript{0} + WKL dove:
\begin{description}
\item[{\textbf{\textbf{WKL}}}] Sia \(T\) un albero binario infinito. Allora \(T\) ha un cammino.
\end{description}

\begin{thm}
In RCA\textsubscript{0} sono fatti equivalenti:
\begin{itemize}
\item WKL
\item ogni anello numerabile ha un ideale primo;
\item ogni funzione continua in \([0,1]\) è limitata
\item Teorema di completezza di Godel
\item Ogni campo numerabile ha un'unica chiusura algebrica
\end{itemize}
\end{thm}
\subsubsection{ACA\textsubscript{0}}
\label{sec:org3994536}

Questa lista di assiomi è data da RCA\textsubscript{0} + \(\mathrm{Ind}\) + \(\Pi^{1}_{0}-\mathrm{CA}\).

\begin{thm}
In RCA\textsubscript{0} sono fatti equivalenti:
\begin{itemize}
\item ACA\textsubscript{0}
\item Lemma di Koning
\item Ogni anello numerabile ha un ideale
\end{itemize}
\end{thm}

Idea: ACA\textsubscript{0} è equivalente a PA
\subsubsection{ATR\textsubscript{0}}
\label{sec:orga6600ae}
\subsubsection{\(\Pi_{1}^{1}-\mathrm{CA}_{0}\)}
\label{sec:orgd051a04}

\subsubsection{Riferimenti:}
\label{sec:org45ffd1f}

\begin{itemize}
\item Simpson, Subsystems of second order aritmetic
\item Stillwell, Reverse Mathematics
\item Hitschelot, Slicing the truth
\item Dzhafarov and Mummert, Reverse Mathematics
\end{itemize}
\subsection{Esempio}
\label{sec:org1ab8547}

\begin{thm}
In RCA\textsubscript{0} sono fatti equivalenti:
\begin{enumerate}
\item ACA\textsubscript{0}
\item \(\Sigma^{0}_{1}-\mathrm{CA}\)
\item per ogni \(f:\N\to \N\) iniettiva esiste \(\operatorname{rng}(f)\) come insieme, ovvero
\begin{equation*}
 \exists X\ \forall n\ \big[n \in X\iff \exists m(f(m=n))\big]
\end{equation*}
\end{enumerate}
\end{thm}
\end{document}
