% Intended LaTeX compiler: pdflatex
\documentclass[../main]{subfiles}

\def\mult#1{\mathrm{mult}_{#1}\,}


\begin{document}

\section{Teorema del Grado per olomorfismi tra superfici di Riemannn}
\label{sec:orgfcccf50}
Siano \(X,Y\) \href{20260127112828-superficie_di_riemann.org}{superfici di Riemann} \href{20250103163701-spazio_topologico_compatto.org}{compatte}, e sia \(F:X\to Y\) \href{20260128143822-funzione_olomorfa_su_una_superficie_di_riemann.org}{olomorfismo} non costante.

\begin{definizione}
Per ogni \(y \in Y\) si definisce\footnote{La somma è finita per il \href{20260128144016-principio_di_identita_per_funzioni_olomorfe_su_superfici_di_riemann.org}{Principio di identità} (poiché \href{20260128172415-sottoinsieme_discreto_in_un_compatto.org}{sottinsiemi discreti in un compatto sono finiti}), mentre \(\mathrm{mult}_{p}\, F\) è la \href{20260129104215-forma_normale_locale_per_superfici_di_riemann.org}{molteplicità}.}
\begin{equation*}
\operatorname{d}(y) \coloneqq \sum_{p \in F^{-1}(y)} \mathrm{mult}_{p}\, F
\end{equation*}
\end{definizione}

\begin{thm}
\(\operatorname{d}(y)\) non dipende da \(y\).
\end{thm}
\begin{proof}
Si fissi \(y_{0} \in Y\), e sia
\begin{equation*}
F^{-1}(y_{0}) \coloneqq \set{x_{1},\dots,x_{h}},\qquad m_{i} \coloneqq \mult{x_{i}} F.
\end{equation*}
Allora \(\operatorname{d}(y_{0}) = m_{1}+\dots+m_{h}\).

Consideriamo ora:
\begin{itemize}
\item su \(Y\) la carta locale \(w\), centrata in \(y_{0}\)\footnote{Ovvero \(w\) è definita su un intorno aperto di \(y_{0}\), e \(w(y_{0})=0\).};
\item su \(X\), le carte locali \(z_{i}\), centrate in \(x_{i}\), tali per cui l'espressione locale di \(F\) è
\begin{equation*}
  w = z_{i}^{m_{i}}.
\end{equation*}
A meno di restringermi, considero tutti i domini disgiunti.
\end{itemize}
Queste esistono per il \href{20260129104215-forma_normale_locale_per_superfici_di_riemann.org}{Teorema di Forma Normale}.

Si consideri ora, per ogni \(\varepsilon>0\), l'insieme
\begin{equation*}
D(0;\varepsilon) \coloneqq \set{|z| < \varepsilon} \subseteq \C
\end{equation*}
e si definiscano:
\begin{align*}
\Delta_{\varepsilon}&\coloneqq w^{-1}\big(D(0;\varepsilon)\big) \subseteq Y\\
\Delta_{x_{i}} &\coloneqq z_{i}^{-1}\bigg(D(0;\varepsilon^{1/m_{i}})\bigg) \subseteq X
\end{align*}

A meno di restringere \(\varepsilon\), si può supporre che \(w\) e \(z_{i}\) siano degli omeomorfismi su \(\Delta_{\varepsilon}\) e \(\Delta_{x_{i}}\).

Si ha che, per ogni \(i\), \(F(\Delta_{x_{i}}) = \Delta_{\varepsilon}\)
\begin{itemize}
\item Per ogni \(y \in \Delta_{\varepsilon} \setminus \set{y_{0}}\), ci sono \(m_{i}\) \href{20250202190147-immagine_punto_a_punto_di_due_classi.org}{retroimmagini} in \(\Delta_{x_{i}}\), ciascuna di \href{20260129104215-forma_normale_locale_per_superfici_di_riemann.org}{molteplicità} 1.

Infatti, ci sono esattamente \(m_{i}\) radici \(\eta_{1},\dots,\eta_{m_{i}} \in D(0;\varepsilon^{1/m_{i}})\) tali che
\begin{equation*}
(\eta_{\ell})^{m_{i}} = w(y) \in D(0; \varepsilon)
\end{equation*}
e siccome \(z_{i}\) è un omeomorfismo, ne esistono altrettanti in \(\Delta_{x_{i}}\).

I punti sono di molteplicità \(1\) poiché la funzione \(z\mapsto z^{m}\) \href{20260130094844-funzione_z_m.org}{è iniettiva in un intorno di ogni punto \(z\neq 0\)}, e questa cosa si trasmette ad \(F\).
\end{itemize}

Per le proprietà delle retroimmagini si ha la seguente inclusione
\begin{equation*}
\bigcup_{i=1}^{h} \Delta_{x_{i}} \subseteq F^{-1}(\Delta_{\varepsilon}).
\end{equation*}
\begin{itemize}
\item Se vale l'uguaglianza: \(\bigcup_{i=1}^{h} \Delta_{x_{i}} = F^{-1}(\Delta_{\varepsilon})\), allora per ogni \(y \in \Delta_{\varepsilon}\setminus \set{y_{0}}\)
\begin{equation*}
  \operatorname{d}(y) = m_{1}+\dots+m_{h} = \operatorname{d}(y_{0})
\end{equation*}
e quindi \(\operatorname{d}\) è \href{20250325153824-funzione_localmente_costante.org}{localmente costante}, con \(Y\) \href{20250103165325-spazio_topologico_connesso.org}{connesso}: \href{20250325154046-funzione_localmente_costante_sse_costante_sulle_componenti_connesse.org}{vale che \(\operatorname{d}\) è costante}.

\item \uline{A meno di restringere \(\varepsilon\), si ottiene l'uguaglianza}.

Per assurdo, supponiamo che per ogni \(\varepsilon > 0\) ammissibile\footnote{Si erano poste delle restrizioni sul valore massimo che potesse assumere \(\varepsilon\).}, si abbia \(\bigcup_{i=1}^{h} \Delta_{x_{i}} \subsetneqq F^{-1}(\Delta_{\varepsilon})\).

Allora, esiste un certo \(n_{0}\) tale per cui, per ogni \(n>n_{0}\), esiste \(y_{n} \in \Delta_{1/n}\) tale che
\begin{equation*}
  F^{-1}(y_{n}) \setminus \bigcup_{i=1}^{h} \Delta_{x_{i}} \neq \emptyset
\end{equation*}
e sia \(x_{n} \in F^{-1}(y_{n}) \setminus \bigcup_{i=1}^{h} \Delta_{x_{i}}\).

Si hanno due \href{20250115100904-successione.org}{successioni} \(\set{x_{n}}\), \(\set{y_{n}}\) tali per cui \(F(x_{n}) = y_{n}\), e inoltre \(y_{n} \to y\) in \(Y\)\footnote{Vedi ``\href{20260130103009-successione_in_uno_spazio_topologico.org}{Successione in uno spazio topologico}''}.

Poiché \(X\) è compatto, \(x_{n}\) ha una \href{20250115100916-sottosuccessione.org}{sottosuccessione} \(\set{x_{n_{k}}}\) convergente ad \(\tilde{x} \in X\), e siccome \(F\) è continua (e \href{20250304142114-funzione_continua_e_continua_per_successioni.org}{quindi} \href{20250310112816-funzione_continua_per_successioni.org}{continua per successioni})
\begin{equation*}
  F(\tilde{x}) = y_{0}
\end{equation*}
ovvero \(\tilde{x} \in F^{-1}(y_{0}) = \set{x_{1},\dots,x_{h}}\).

Quindi, se \(\tilde{x} = x_{j}\), allora esiste \(k_{0}\) tale che per ogni \(k>k_{0}\): \(x_{n_{k}} \in \Delta_{x_{j}}\). Questo contraddice il fatto che
\begin{equation*}
  x_{n_{k}} \in F^{-1}(y_{n_{k}}) \setminus \bigcup_{i=1}^{h} \Delta_{x_{i}} \IMPLICA x_{n_{k}} \notin \Delta_{x_{j}}.
\end{equation*}
Assurdo.\qedhere
\end{itemize}
\end{proof}

\begin{definizione}
Si definisce il \textbf{grado di \(F\)} come:
\begin{equation*}
\deg F \coloneqq \operatorname{d}(y),\qquad y \in Y.
\end{equation*}
\end{definizione}
\end{document}
