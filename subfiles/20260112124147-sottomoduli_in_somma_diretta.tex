% Intended LaTeX compiler: pdflatex
\documentclass[../main]{subfiles}


\begin{document}

Sia \(R\) un \href{20241205141119-anello.org}{anello commutativo con unità}.

\begin{thm}
Sia \(M\) un \href{20241205141053-r_moduli.org}{\(R\)-modulo}, e siano \(U,V \subseteq M\) \href{20241206142802-sottomoduli.org}{sottomoduli}. Sono fatti equivalenti:
\begin{enumerate}
\item per ogni \(m \in M\), esistono unici \(u \in U\), \(v \in V\) tali che
\begin{equation*}
 m=u+v
\end{equation*}
\item \(U+V = M\)\footnote{Vedi ``\href{20241206142802-sottomoduli.org}{Somma di sottomoduli}''} e \(U\cap V = \set{0}\);
\item \(M \cong U \oplus V\).
\end{enumerate}
\end{thm}
\begin{proof}
\begin{description}
\item[{(\(1.\Rightarrow 2.\)):}] L'esistenza di \(u,v\) garantiscono che \(U+V = M\). Se per assurdo si avesse \(0\neq x \in U\cap V\), allora
\begin{equation*}
  0 = 0 + 0 = (x) + (-x)
\end{equation*}
negando l'unicità della scrittura.

\item[{(\(2.\Rightarrow 3.\)):}] Si consideri il morfismo:
\begin{align*}
\phi: U\oplus V &\longrightarrow M\\
(u,v) &\longmapsto u+v
\end{align*}
Allora:
\begin{itemize}
\item \(\phi\) è suriettivo, poiché \(U+V = M\);

\item \(\phi\) è iniettivo: se \(\varphi(u,v) = u + v = 0\), allora \(u = -v\), e quindi \(u,v \in U\cap V = \set{0}\), e \((u,v) = 0\).
\end{itemize}

Quindi \(\phi\) è un \href{20241206115416-morfismi_r_moduli.org}{isomorfismo}.

\item[{(\(3.\Rightarrow 1.\)):}] Considerando l'isomorfismo \(\phi\) del punto precedente, allora, se \((u,v) = \phi^{-1}(m)\) con \(u \in U\) e \(v \in V\), allora
\begin{equation*}
  u+v = m
\end{equation*}
e questa scrittura è unica.
\qedhere
\end{description}
\end{proof}
\begin{definizione}
Quando vale una delle condizioni equivalenti, si dice che \(U\) e \(V\) sono in somma diretta, e si scrive (con un abuso di notazione), \(M  = U \oplus V\).
\end{definizione}
\end{document}
