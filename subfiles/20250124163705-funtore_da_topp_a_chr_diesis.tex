% Intended LaTeX compiler: pdflatex
\documentclass[../main]{subfiles}

\usepackage[hyperref]{biblatex}
\date{}
\title{}
\begin{document}

\section{Funtore da TopP a ChR - diesis}
\label{sec:orge820216}
Si definisce il \href{20241205131958-funtore.org}{funtore}
\begin{equation*}
(\mathcal{S}_{\bullet}, \overline{\diesis}) : \cat{TopP}\to \cat{Ch}_{R}
\end{equation*}
tra la \href{20250122160145-categoria_topp.org}{categoria \(\cat{TopP}\)} e la \href{20250120163759-categoria_complessi_di_catene.org}{categoria \(\cat{Ch}_{R}\)}.
\begin{itemize}
\item \textbf{Oggetti}:
Alla \href{20250122154728-coppia_topologica.org}{coppia topologica} \((X,A)\) si associa il \href{20250120163114-complesso_di_catene.org}{complesso di catene} \href{20250122154809-complesso_di_catene_singolari_relative.org}{singolare relativo} \(\mathcal{S}_{\bullet}(X,A)\).
\item \textbf{Morfismi}:
Al \href{20250122160145-categoria_topp.org}{morfismo} \((X,A) \xrightarrow{\hphantom{int}f\hphantom{int}} (Y,B)\) si associa il \href{20250120163759-categoria_complessi_di_catene.org}{morfismo} \(\overline{f_{\diesis}} = \set{(\overline{f_{\diesis}})_{q} : S_{q}(X,A)\to S_{q}(Y,B)}\):
\begin{align*}
(\overline{f_{\diesis}})_{q}: S_{q}(X,A) &\longrightarrow S_{q}(Y,B)\\
c_{q} + S_{q}(A) &\longmapsto f_{\diesis} (c_{q}) + S_{q}(B)
\end{align*}
dove \(f_{\diesis}\) si ottiene da \(f\) applicando il \href{20250122154136-funtorialita_dell_omologia_singolare.org}{funtore \((\mathcal{S}_{\bullet},\diesis): \cat{Top}\to \cat{Ch}_{R}\)}.
\end{itemize}

(È da dimostrare che \(\overline{f_{\star}}\) sia ben definita e definsica un morfismo di complessi di catene.)
\end{document}
