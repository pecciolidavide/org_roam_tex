% Intended LaTeX compiler: pdflatex
\documentclass[../main]{subfiles}


\begin{document}

\section{Successione esatta corta di fasci data da un divisore e un punto di una superficie di Riemann}
\label{sec:org142394c}
Sia \(X\) una \href{20260127112828-superficie_di_riemann.org}{superficie di Riemann}, e sia \(\operatorname{Div}(X)\) il \href{20260201235551-divisore_di_una_superficie_di_riemann.org}{gruppo dei divisori di \(X\)}. Siano:
\begin{itemize}
\item per ogni \(D \in \operatorname{Div}(X)\), \(\mathcal{O}_{X}(D)\) il \href{20260202101128-fascio_associato_ad_un_divisore_su_una_superficie_di_riemann.org}{fascio associato a \(D\)};
\item \(\mathcal{O}_{X}\) il \href{20260128143847-fascio_delle_funzioni_olomorfe_su_una_superficie_di_riemann.org}{fascio delle funzioni olomorfe};
\item per ogni \(p \in X\), \(\C_{p}\) il \href{20250325171249-fascio_grattacielo.org}{fascio grattacielo centrato in \(p\)}.
\end{itemize}

\begin{oss}
Per \(p \in X\):
\begin{itemize}
\item \(\mathcal{O}_{X}(-p)\) è il fascio delle funzioni olomorfe nulle in \(p\), ed è pertanto sottofascio di \(\mathcal{O}_{X}\):
\begin{equation*}
\mathcal{O}_{X}(-p) \hookrightarrow \mathcal{O}_{X}
\end{equation*}
\item Inoltre, si ha la mappa di valutazione in \(p\), morfismo di fasci \(\operatorname{ev}_{p}: \mathcal{O}_{X}\to \C_{p}\): se \(p \in U\):
\begin{align*}
\operatorname{ev}_{p,U}: \mathcal{O}_{X}(U) &\longrightarrow \C_{p}(U)\\
f &\longmapsto f(p)
\end{align*}
se \(p\notin U\) allora \(\operatorname{ev}_{p,U}=0\).

È possibile dimostrare che il \href{20250327114922-fascio_nucleo.org}{fascio nucleo} \(\ker \operatorname{ev}_{p} = \mathcal{O}_{X}(-p)\) e che \(\operatorname{ev}_{p}\) è \href{20250325180613-morfismo_di_prefasci.org}{morfismo di fasci} \href{20250327115214-morfismo_di_fasci_suriettivo.org}{suriettivo}.
\end{itemize}
Si ottiene quindi la \href{20250327150404-successione_di_fasci_esatta.org}{SEC} di \href{20250324174728-fascio.org}{fasci} di \href{20241205142027-spazio_vettoriale.org}{spazi vettoriali}:
\begin{equation*}
\begin{tikzcd}[ampersand replacement=\&]
	0 \& {\mathcal{O}_X(-p)} \& {\mathcal{O}_X} \& {\C_p} \& 0
	\arrow[from=1-1, to=1-2]
	\arrow[hook, from=1-2, to=1-3]
	\arrow["{\operatorname{ev}_p}", from=1-3, to=1-4]
	\arrow[from=1-4, to=1-5]
\end{tikzcd}
\end{equation*}
\end{oss}

\begin{prop}
Per ogni \(D \in \operatorname{Div}(X)\) e \(p \in X\) esistono \(i\) e \(\varphi\) tali che
\begin{equation*}
\begin{tikzcd}[ampersand replacement=\&]
	0 \& {\mathcal{O}_X(D-p)} \& {\mathcal{O}_X(D)} \& {\C_p} \& 0
	\arrow[from=1-1, to=1-2]
	\arrow["i", from=1-2, to=1-3]
	\arrow["\varphi", from=1-3, to=1-4]
	\arrow[from=1-4, to=1-5]
\end{tikzcd}
\end{equation*}
sia una \href{20250327150404-successione_di_fasci_esatta.org}{SEC} di \href{20250324174728-fascio.org}{fasci} di \href{20241205142027-spazio_vettoriale.org}{spazi vettoriali}.
\end{prop}
\begin{proof}
Siccome \(D-p \le D\), \href{20260202101128-fascio_associato_ad_un_divisore_su_una_superficie_di_riemann.org}{allora} \(\mathcal{O}_{X}(D-p) \subseteq \mathcal{O}_{X}(D)\) è \href{20250325150647-sottoprefascio.org}{sottofascio}, e pertanto è possibile prendere come \(i\) l'inclusione.

È necessario quindi costruire \(\mathcal{O}_{X}(D)\xrightarrow{\varphi} \C_{p}\) \href{20250327115214-morfismo_di_fasci_suriettivo.org}{suriettiva} il cui \href{20250327114922-fascio_nucleo.org}{nucleo} sia \(\mathcal{O}_{X}(D-p)\). Si fissi \(U_{p} \subseteq X\) \href{20250111142313-intorno.org}{intorno} \href{20250103145124-topologia.org}{aperto} e \href{20250103165325-spazio_topologico_connesso.org}{connesso} di \(p\), e \(z_{p}:U_{p}\to \C\) \href{20260127112715-atlante_complesso.org}{carta locale} tale che \(z_{p}(p) = 0\).

Si definisce \(\varphi\). Sia ora \(U \subseteq X\) aperto.
\begin{itemize}
\item Se \(p\notin U\), allora si pone \(\varphi_{U} \coloneqq 0\).
\item Se \(p \in U\), sia \(f \in \mathcal{O}_{X}(D)(U)\). Allora l'\href{20260128144450-ordine_di_una_funzione_meromorfa_su_una_superficie_di_riemann.org}{ordine}
\begin{equation*}
  \mathrm{ord}_{p} f \ge - D (p)
\end{equation*}
\begin{itemize}
\item se \(f\) è nulla in un intorno di \(p\), si pone \(\varphi_{U}(f) = 0\);
\item se \(f\) è non nulla in un intorno di \(p\), allora \(f\) nella carta locale ha uno sviluppo in \href{20260128163831-serie_di_laurent.org}{serie di Laurent}:
\begin{equation*}
f = \sum_{n\ge -D(p)} a_{n}\,z_{p}^{n}
\end{equation*}
Si pone \(\varphi_{U}(f) \coloneqq a_{-D(p)} \in \C\). Si noti che
\begin{align*}
a_{-D(p)} \neq 0 & \IFF \mathrm{ord}_{p} f = -D(p)\\
a_{-D(p)} = 0 & \IFF \mathrm{ord}_{p} f > -D(p)\\
&\IFF \mathrm{ord}_{p} f \ge -D(p) + 1\\
&\IFF f \in \mathcal{O}_{X}(D-p) \subseteq \mathcal{O}_{X}(D).
\end{align*}
\end{itemize}
\end{itemize}

Si è dimostrato anche che \(\ker \varphi = \mathcal{O}_{X}(D-p)\).

Resta da dimostrare che \(\varphi\) sia un \href{20250327115214-morfismo_di_fasci_suriettivo.org}{morfismo di fasci suriettivo}. Sia \(U \subseteq X\) aperto.
\begin{itemize}
\item Se \(p \notin U\), allora \(\varphi_{U}(U) = 0 = \C_{p}(U)\), e rispetta la suriettività.
\item Se \(p \in U\), sia \(\lambda \in \C = \C_{p}(U)\). Vogliamo dimostrare che per ogni \(q \in U\) esiste \(W \subseteq U\) intorno aperto di \(q\), ed esiste \(F \in \mathcal{O}_{X}(D)(W)\) tale che \(\restriction{\lambda}{W} = \varphi_{W} (F)\).

Per ogni \(q \neq p\) questo è ovvio: è sufficiente prendere un intorno \(W\) di \(q\) che non contenga \(p\), ottenendo che \(0 \in \mathcal{O}_{X}(D)(W)\) e \(\varphi_{W}(0) = 0 = \restriction{\lambda}{W}\).

Sia quindi \(U_{p}\) l'intorno di \(p\) di cui sopra: \(z_{p}:U_{p}\to \C\) carta locale tale che \(z_{p}(p) = 0\). A meno di restringere \(U_{p}\), è possibile supporre che
\begin{equation*}
  U_{p}\cap \operatorname{supp}D \subseteq\set{p}
\end{equation*}
in quanto il supporto è insieme discreto in uno spazio di Haussdorf. Allora \(\restriction{D}{U_{p}} = D(p)\cdot p\).

Sia \(g \in \mathcal{M}_{X}(U_{p})\) data da \(g \coloneqq \lambda z_{p}^{-D(p)}\).
\begin{itemize}
\item \(g\) è olomorfa su \(U_{p}\setminus\set{p}\);

\item \(\mathrm{ord}_{p}\,g = - D(p)\)
\end{itemize}
e pertanto \(\operatorname{div}g + \restriction{D}{U_{p}}\ge 0\): \(g \in \mathcal{O}_{X}(D)(U_{p})\) e \(\varphi_{U_{p}}(g) = \lambda\).
\qedhere
\end{itemize}
\end{proof}
\end{document}
