% Intended LaTeX compiler: pdflatex
\documentclass[../main]{subfiles}


\begin{document}

\section{Magrezza dentro ad un polacco tramite gioco di Banach-Mazur}
\label{sec:org939e042}
\subsection{Teorema}
\label{sec:org8183b57}

Sia \(X\) uno spazio polacco con una metrica fissata e sia \(\mathcal{W}\) una base debole di X.

Dato \(F \subseteq X\times \omega^{\omega}\) si consideri il \href{20250513111844-gioco_di_banach_mazur.org}{**-gioco}: \(G^{**}_{\text{u}}(F)\). Indicato con \(A\coloneqq \pi_{X}(F)\):
\begin{enumerate}
\item se I ha una \href{20250513171520-giochi_di_gale_stewart.org}{strategia} \href{20250513171520-giochi_di_gale_stewart.org}{vincente} in \(G^{**}_{\text{u}}(F)\), allora \(A\) è magro in un aperto non vuoto di \(X\);
\item se II ha una \href{20250513171520-giochi_di_gale_stewart.org}{strategia} \href{20250513171520-giochi_di_gale_stewart.org}{vincente} in \(G^{**}_{\text{u}}(F)\) allora \(A\) è comagro.
\end{enumerate}
\subsubsection{Dimostrazione}
\label{sec:orgfe54698}

\begin{enumerate}
\item Sia \(\sigma\) una \href{20250513171520-giochi_di_gale_stewart.org}{strategia} \href{20250513171520-giochi_di_gale_stewart.org}{vincente} per I, e sia \(U_{0}\) la prima mossa. Si mostra che \(A\) è \href{20250419122752-insieme_magro.org}{magro} in \(U_{0}\).

Per ogni \(a \in \omega\) e per ogni \(p \in \sigma\) della forma:
\begin{equation*}
 p=\langle
 	U_{0},({y}_{0}, V_{0}), \dots, U_{n-1}, ({y}_{n-1}, V_{n-1}), U_{n}
 \rangle
\end{equation*}
si definisce \(F_{p,a} \subseteq U_{0}\):
\begin{align*}
 F_{p,a} = \{&z \in U_{n}\mid \text{per ogni mossa legale }(a, V_{n})\\
 &\text{se } U_{n+1}\text{ è l'unico elemento di }\mathcal{W}\text{ tale che}\\
 &p \concat \langle(a,V_{n}), U_{n+1}\rangle \in \sigma \text{ allora } z \notin U_{n+1}\}
\end{align*}
\begin{itemize}
\item L'insieme \(F_{p,a}\) è mai denso, poiché chiuso e con interno vuoto \footnote{Esempio 1.5.2 (2) di LMR ``Notes on descriptive set theory''}. Infatti, se per assurdo \(\operatorname{Int}(F_{p,a}) \neq \emptyset\), allora esiste \(W \in \mathcal{W}\) tale che
\begin{equation*}
   W \subseteq \operatorname{Int}(F_{p,a}), \quad \operatorname{diam}(W)<2^{-n}
\end{equation*}
pertanto se II gioca \(V_{n} \coloneqq W\) allora I dovrà giocare \(U_{n+1} \subseteq V_{n} \subseteq F_{p,a}\). Ma per definizione \(U_{n+1}\cap F_{p,a} = \emptyset\). Assurdo.

Inoltre, se \(\eta \in U_{n}\setminus F_{p,a}\), allora esiste una sequenza
\begin{equation*}
   p\concat\langle (a,V_{n}), U_{n+1}\rangle \in \sigma
\end{equation*}
con \(\eta \in U_{n+1}\); siccome \(U_{n+1} \cap F_{p,a} = \emptyset\) segue
\begin{equation*}
   \eta \in U_{n+1} \subseteq U_{n}\setminus F_{p,a} \subseteq X\setminus F_{p,a}
\end{equation*}
ovvero \(F_{p,a}\) chiuso.

\item Siccome \(\sigma\) e \(\omega\) sono insiemi numerabili allora
\begin{equation*}
   \bigcup_{p \in \sigma', a \in \omega} F_{p,a}
\end{equation*}
è un insieme magro, dove \(\sigma' \subseteq \sigma\) è l'insieme delle sequenze di lunghezze dispari.
\end{itemize}

Sia ora \(x \in A\cap U_{0}\). Allora esiste \(y \in \omega^{\omega}\), \(y=(y_{i})_{i \in\omega}\) tale che \((x,y) \in F\).

Una posizione \(p \in \sigma'\):
\begin{equation*}
     p=\langle
     	U_{0},({y}_{0}, V_{0}), \dots, U_{n-1}, ({y}_{n-1}, V_{n-1}), U_{n}
     \rangle
\end{equation*}
è \uline{buona} per \((x,{y})\) se \(x \in U_{n}\). Siccome \(\sigma\) è una strategia vincente per il giocatore I, allora esiste una posizione \(p_{(x,y)} \in \sigma\) buona per \((x,y)\) e massimale, ovvero ogni estensione di \(p_{(x,y)}\) \uline{non è buona}. Ma allora, se
\begin{equation*}
 p_{(x,y)} = \langle U_{0}, (y_{0},V_{0}),\dots, U_{n}\rangle
\end{equation*}
si ha che \(x \in F_{p_{(x,y)}, y_{n}}\).

Pertanto \(A\cap U_{0} \subseteq \bigcup_{p \in \sigma', a \in \omega} F_{p,a}\) è magro.

\item Se II ha una strategia vincente per \(G^{**}_{\text{u}}(F)\), allora ha una strategia vincente \href{20250513111844-gioco_di_banach_mazur.org}{in \(G^{**}(A)\)}. Per il \href{20250514174717-teorema_di_caratterizzazione_dei_comagri_tramite_il_gioco_di_banach_mazur.org}{teorema precedente}, \(A\) è comagro.\qed
\end{enumerate}
\end{document}
