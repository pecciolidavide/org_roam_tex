% Intended LaTeX compiler: pdflatex
\documentclass[../main]{subfiles}


\begin{document}

\begin{prop}
Sia \((\Omega,  \mathcal{F}, \mathds{P})\) uno \href{20250711175559-spazio_di_probabilita.org}{spazio di probabilità}. Siano \(X_{1},\dots,X_{n}\) delle \href{20250711175937-variabile_aleatoria.org}{variabili aleatorie}, e sia
\begin{equation*}
Y\coloneqq \max\set{X_{1},\dots,X_{n}}.
\end{equation*}
Dette \(\sigma(X_{i}),\sigma(Y)\) le \href{20250526100313-sigma_algebra.org}{\(\sigma\)-algebre} \href{20250714154501-sigma_algebra_generata_da_una_variabile_aleatoria.org}{generate dalle v.a.}, si ha che
\begin{equation*}
\sigma(Y) \subseteq \bigcap \sigma(X_{i}).
\end{equation*}
\end{prop}

\begin{proof}
Sia \(b \in \R\).
\begin{align*}
Y^{-1}(-\infty, b] &= \set{\omega \in\Omega\mid Y(\omega)\le b}\\
&= \set{\omega \in \Omega\mid
\max\set{X_{1}(\omega),\dots,X_{n}(\omega)}\le b
}\\
&=\set{\omega \in\Omega\mid X_{1}(\omega)\le b \land \dots \land X_{n}(\omega)\le b}\\
&= \set{\omega \in\Omega\mid X_{1}(\omega)\le b}\cap \dots \cap \set{\omega \in\Omega\mid X_{n}(\omega)\le b}\\
&= \bigcap X^{-1}_{i}(-\infty, b]
\end{align*}
ma per ogni \(i\) si ha che \(X^{-1}_{i}(-\infty, b] \in \sigma(X_{i})\), dunque
\begin{equation*}
Y^{-1}(-\infty, b] \in \bigcap\sigma(X_{i}).\qedhere
\end{equation*}
\end{proof}
\end{document}
