% Intended LaTeX compiler: pdflatex
\documentclass[../main]{subfiles}


\begin{document}

\section{Punto Critico di una funzione reale}
\label{sec:org50dda46}
\begin{definizione}
Sia \(f: A \subseteq \R^{n}\to \R\) \href{20250702101532-funzione_reale_differenziabile.org}{differenziabile}, con \(A\) \href{20250103145124-topologia.org}{aperto}. \(c \in A\) si dice \uline{punto critico} (o \uline{punto stazionario}) di \(f\) se\footnote{Vedi ``\href{20250624171244-gradiente_di_una_funzione.org}{Gradiente di una funzione}''}
\begin{equation*}
\nabla f(x) = 0
\end{equation*}
\end{definizione}
\end{document}
