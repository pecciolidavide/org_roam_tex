% Intended LaTeX compiler: pdflatex
\documentclass[../main]{subfiles}


\begin{document}

Uno \href{20250103145124-topologia.org}{spazio topologico} si dice \textbf{a base numerabile} se ammette almeno una \href{20250111142837-base_di_una_topologia.org}{base} con una \href{20241213101756-cardinalita.org}{cardinalità} \href{20250111143651-insieme_numerabile.org}{numerabile}.
\section{Topologia Euclidea è a base numerabile}
\label{sec:orgc039da5}
Una base numerabile per la \href{20250103103232-topologia_euclidea.org}{Topologia Euclidea} di \(\R^{n}\) è
\begin{equation*}
\mathscr{B} = \set{B_{q} \left(\frac{1}{m}\right): q \in \Q^{n}, m \in \N}
\end{equation*}
che è numerabile perché \(\Q^{n}\) è numerabile e \(\N\) è numerabile.
\end{document}
