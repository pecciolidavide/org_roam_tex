% Intended LaTeX compiler: pdflatex
\documentclass[../main]{subfiles}


\begin{document}

\section{Differenziale di una forma}
\label{sec:org270d8c7}
\def\d{\operatorname{d}}

Sia \(M\) una \href{20250113115909-struttura_differenziabile.org}{varietà differenziabile}, e si indichi con \(\Omega^{k}(M)\) l'insieme delle \href{20251115155511-forma_differenziale_in_un_punto.org}{\(k\)-forme differenziali} su \(M\).

\begin{definizione}
Il \uline{differenziale} è una funzione
\(\dif :\Omega^{k}(M) \to \Omega^{k+1}(M)\) definita come segue:
\begin{itemize}
\item se \(k = 0\), allora \(\Omega^{0}(M) = C^{\infty}(M)\)\footnote{Questo è l'\href{20250113144722-funzioni_cinfinito_tra_varieta_differenziabili.org}{anello delle funzioni \(C^{\infty}\) da una varietà ai reali}.} e quindi, localmente,
\begin{equation*}
  \dif f = \dpd{f}{{x^{i}}}\dif x^{i}
\end{equation*}
dove si è utilizzata la \href{20250114105957-notazione_di_einstein.org}{notazione di Einstein};
\item se \(k > 0\), allora localmente \(\omega = f_{I} \dif x^{I}\), e si definisce
\begin{equation*}
  \dif \omega = \dpd{f_{I}}{{x^{j}}} \dif x^{j} \wedge \dif x^{I}.
\end{equation*}
dove \(I=(i_{1},\dots,i_{k}) \in \set{1,\dots, n}^{k}\) è un multi-indice di elementi diversi, e
\begin{equation*}
  \dif x^{I} \coloneqq %
  \dif x^{i_{1}} \wedge \dif x^{i_{2}} \wedge \dots \wedge \dif x^{i_{k}}.
\end{equation*}
\end{itemize}
\end{definizione}
\section{Proprietà del differenziale di forme}
\label{sec:org2e5f1a0}
\def\d{\operatorname{d}}

\begin{prop}
Valgono le seguenti proprietà:
\begin{enumerate}
\item \(\d\) è lineare;
\item \(\displaystyle  \d(\Omega^{k}(M)) \subseteq \Omega^{k+1}(M)\);
\item \(\d\circ\d = 0\);
\item per \(\omega \in \Omega^{k}(M)\) e \(\varphi \in \Omega^{p}(M)\):
\(\d(\omega\wedge\varphi) = %
   (\dif\omega) \wedge \varphi + (-1)^{k}\, \omega \wedge (\dif\varphi)\)
\end{enumerate}
\end{prop}
\begin{oss}
Se \(M\) è una varietà \(n\)-dimensionale e \(\omega \in \Omega^{n}(M)\), allora \(\dif \omega= 0\).
\end{oss}
\end{document}
