% Intended LaTeX compiler: pdflatex
\documentclass[../main]{subfiles}


\begin{document}

\section{Teorema di Milliken-Taylor}
\label{sec:orge44e108}
\def\fp{\operatorname{fp}}
\begin{thm}
For every finite coloring of \(\N^{(2)}\) there is an infinite sequence \(\overline{a} = \langle a_i:i<\omega\rangle\) in \(\N\) such that \(\operatorname{ocfp}(\overline{a}) \subseteq \fp(\overline{a})^{(2)}\) is monochromatic, where
\begin{equation*}
 \operatorname{ocfp}(\overline{a}) \coloneqq \set{%
 	\set{a_{i_{1}}+\dots+a_{i_{n}}, a_{j_{1}}+\dots+a_{j_{m}}} %
 		\in \fp(\overline{a})^{(2)} %
 	\mid i_{1}<\dots<i_{n}<j_{1}<\dots<j_{m} %
 }.
\end{equation*}
\end{thm}
\def\U{\mathcal{U}}
\def\L{\mathcal{L}}
\def\A{\mathcal{A}}
%
\def\eq{{\rm eq}}
\def\Ueq{\U^\eq}
\def\a{\overline{a}}
%
\def\orbita{\mathcal{O}}
\def\Aut{\operatorname{Aut}}
%
\def\tc{\mid}
\def\tp{\operatorname{tp}}
\def\EMtp{\operatorname{EM}\text{-}\operatorname{tp}}
\def\<{\langle}
\def\>{\rangle}
%
\def\restricted#1{\,\mathord{\upharpoonright}{{\scriptstyle #1}}}
\def\equivalentover#1{\mathrel{\equiv_{ #1 }}}

%% NON FORKING
\def\nonforkSymbol{%
\mathbin{\raise1.8ex%
\rlap{\kern0.6ex\rule{0.6ex}{0.1ex}}%
\rlap{\kern1.1ex\rule{0.1ex}{1.9ex}}\raise-0.3ex\hbox{$\smile$}}}
\def\defaultnonforkmodel{M}
\def\nonfork{\nonforkSymbol}
\renewcommand{\nonfork}[1][\defaultnonforkmodel]{%
\mathrel{\nonforkSymbol_{#1}}}

\def\defaultnonforkmodel{\N}
\def\fp{\operatorname{fp}}
\def\ocfp{\operatorname{ocfp}}
\def\b{\overline{b}}

\begin{proof}
Si dimostra la seguende affermazione:
\begin{quote}
For every finite coloring of \(\N^{(2)}\) there is an infinite sequence \(\overline{a} = \langle a_i:i<\omega\rangle\) in \(\N\) such that \(\operatorname{ocfp}(\overline{a}) \subseteq \fp(\overline{a})^{(2)}\) is monochromatic, where
\begin{equation*}
 \operatorname{ocfp}(\overline{a}) \coloneqq \set{%
 	\set{a_{i_{1}}+\dots+a_{i_{n}}, a_{j_{1}}+\dots+a_{j_{m}}} %
 		\in \fp(\overline{a})^{(2)} %
 	\mid i_{1}<\dots<i_{n}<j_{1}<\dots<j_{m} %
 }.
\end{equation*}
\end{quote}

Sia \(\U\succeq \N\) modello saturo in un linguaggio \(\L\) che contiene un simbolo di relazione per ogni sottoinsieme proprio di \(\N\) e per ogni colore di \(\N^{(2)}\).
Segue che \(\nonfork\) è 1-stazionaria

Sia \(\A \coloneqq \set{a \in \U \mid a \neq 0}\). Questo è un insieme definibile ed è idempotente, pertanto esiste \(b_{0} \in \A\) tale che
\(\orbita(b_{0}/\N)\)
sia idempotente.

Sia \(\b = \langle b_{i} \mid i <\omega \rangle\) una sequenza di coeredi su \(\N\).
In particolare si ha:
\begin{enumerate}
\item per ogni \(i<\omega\): \(b_{i} \nonfork b\restricted{i}\) (e quindi \(b_{1} \nonfork b_{0}\) e \(b_{2} \nonfork b_{1},b_{0}\));
\item per ogni \(i<\omega\): \(b_{i+1} \equivalentover{\N, b\restricted{i}} b_{i}\) (e quindi \(b_{1}\equivalentover{\N}b_{0}\));
\item \(b_{1}+b_{0} \equivalentover{\N} b_{0}\), e
\(b_{2}+b_{1} \equivalentover{\N, b_{0}} b_{1}\).
\end{enumerate}

Si costruisce induttivamente una sequenza \(\a=\langle a_{i} \mid i < \omega \rangle\) che soddisfi la tesi.

Per ipotesi induttiva al passo \(h\) si supponga di aver costruito
\(\a\restricted{h} = \langle a_{i} \mid i < h \rangle\) tale che
\begin{equation*}
\ocfp(\a\restricted{h}, b_{1}, b_{0})\text{ è monocromatico verde}.
\end{equation*}
\uline{Claim}: Questo implica che anche
\(\ocfp(\a\restricted{h}, b_{2}, b_{1}, b_{0})\) sia monocromatico verde.

Esiste una formula \(\varphi_{h}(\a\restricted{h}, b_{2}, b_{1},b_{0}) \in \L\) che afferma questa cosa.

Siccome per 1. si ha che \(b_{2} \nonfork b_{1},b_{0}\), allora esiste \(a_{h} \in \N\) tale che
\begin{equation*}
\U\vDash\varphi_{h}(\a\restricted{h}, a_{h}, b_{1}, b_{0}).
\end{equation*}
ovvero \(\ocfp(\a\restricted{h+1}, b_{1}, b_{0})\) ha solo elementi verdi.

Si è quindi costruita \(\a\) tale che \(\ocfp(\a,b_{1},b_{0})\) è monocromatico, e pertanto anche \(\ocfp(\a) \subseteq \fp(\a)^{(2)}\) lo~è.
\end{proof}

\uline{Claim}: \(\ocfp(\a\restricted{h}, b_{1}, b_{0})\) monocromatico verde implica che anche
\(\ocfp(\a\restricted{h}, b_{2}, b_{1}, b_{0})\) sia monocromatico verde.

\uline{Dimostrazione del Claim}: Infatti
\begin{align}
&\ocfp(\a\restricted{h},b_{2},b_{1},b_{0}) = \ocfp(\a\restricted{h},b_{1},b_{0}) \cup \notag\\
	&\cup \set{\set{
		a_{i_{1}} + \dots + a_{i_{n}}, a_{j_{1}} + \dots + a_{j_{m}} + b_{2} + b_{1} + b_{0}
	} \mid i_{1}<\dots<i_{n}<j_{1}<\dots<j_{m}<h}%
	\label{eq:1_elementi}\\
	&\cup \set{\set{
		a_{i_{1}} + \dots + a_{i_{n}}, a_{j_{1}} + \dots + a_{j_{m}} + b_{2} + b_{1}
	} \mid i_{1}<\dots<i_{n}<j_{1}<\dots<j_{m}<h}%
	\label{eq:2_elementi}\\
	&\cup \set{\set{
		a_{i_{1}} + \dots + a_{i_{n}}, a_{j_{1}} + \dots + a_{j_{m}} + b_{2}  + b_{0}
	} \mid i_{1}<\dots<i_{n}<j_{1}<\dots<j_{m}<h}%
	\label{eq:3_elementi}\\
&\cup \set{\set{
		a_{i_{1}} + \dots + a_{i_{n}}, b_{2}  + b_{1} +b_{0}
	} \mid i_{1}<\dots<i_{n}<j_{1}<\dots<j_{m}<h}%
	\label{eq:4_elementi}\\
&\cup \set{\set{
		a_{i_{1}} + \dots + a_{i_{n}} + b_{2}, b_{1} +b_{0}
	} \mid i_{1}<\dots<i_{n}<j_{1}<\dots<j_{m}<h}%
	\label{eq:5_elementi}\\
&\cup \set{\set{
		a_{i_{1}} + \dots + a_{i_{n}} + b_{2} + b_{1}, b_{0}
	} \mid i_{1}<\dots<i_{n}<j_{1}<\dots<j_{m}<h}%
	\label{eq:6_elementi}\\
&\cup \set{\set{
		a_{i_{1}} + \dots + a_{i_{n}}, b_{2}  + b_{1}
	} \mid i_{1}<\dots<i_{n}<j_{1}<\dots<j_{m}<h}%
	\label{eq:7_elementi}\\
&\cup \set{\set{
		a_{i_{1}} + \dots + a_{i_{n}} + b_{2}, b_{1}
	} \mid i_{1}<\dots<i_{n}<j_{1}<\dots<j_{m}<h}%
	\label{eq:8_elementi}\\
&\cup \set{\set{
		a_{i_{1}} + \dots + a_{i_{n}}, b_{2} +b_{0}
	} \mid i_{1}<\dots<i_{n}<j_{1}<\dots<j_{m}<h}%
	\label{eq:9_elementi}\\
&\cup \set{\set{
		a_{i_{1}} + \dots + a_{i_{n}} + b_{2}, b_{0}
	} \mid i_{1}<\dots<i_{n}<j_{1}<\dots<j_{m}<h}%
	\label{eq:10_elementi}\end{align}
\begin{itemize}
\item Per gli elementi in
\eqref{eq:1_elementi},
\eqref{eq:2_elementi},
\eqref{eq:4_elementi},
\eqref{eq:6_elementi},
\eqref{eq:7_elementi},
per 3. \(b_{2}+b_{1} \equivalentover{\N, b_{0}} b_{1}\) e pertanto è possibile ricondursi ad un elemento di \(\ocfp(\a\restricted{h},b_{1},b_{0})\), insieme monocromatico verde.
\item Per gli elementi in
\eqref{eq:3_elementi},
\eqref{eq:9_elementi},
\eqref{eq:10_elementi},
per 2. \(b_{2} \equivalentover{\N,b_{0}} b_{1}\) e pertanto è possibile ricondursi ad un elemento di \(\ocfp(\a\restricted{h},b_{1},b_{0})\), insieme monocromatico verde.
\item Per \eqref{eq:8_elementi}, siccome \(\b\) è una sequenza di coeredi, allora è una sequenza di indiscernibili, e pertanto
\((b_{2},b_{1}) \equivalentover{\N} (b_{1},b_{0})\)
e dunque è possibile ricondursi ad un elemento di \(\ocfp(\a\restricted{h},b_{1},b_{0})\), insieme monocromatico verde.
\item Sia
\(\set{a_{i_{1}} + \dots + a_{i_{n}} + b_{2}, b_{1} +b_{0}}\)
un elemento di \eqref{eq:5_elementi}.
Siccome, per 1., \(b_{2} \nonfork b_{1},b_{0}\), allora \(b_{2} \nonfork b_{1}+b_{0}, b_{0}\) (questo segue banalmente).
Inoltre, per 3., si ha che \(b_{1}+b_{0} \equivalentover{\N} b_{0}\).

Per il Lemma~14.8(5) \emph{non-splitting}, allora
\(b_{1}+b_{0} \equivalentover{\N,b_{2}} b_{0}\).
Pertanto l'elemento considerato ha lo stesso colore di
\begin{equation*}
  \set{a_{i_{1}} + \dots + a_{i_{n}} + b_{2}, b_{0}}
\end{equation*}
che a sua volta, poiché \((b_{2},b_{0}) \equivalentover{\N} (b_{1},b_{0})\) come argomentato al punto precedente, ha lo stesso colore di
\begin{equation*}
  \set{a_{i_{1}} + \dots + a_{i_{n}} + b_{1}, b_{0}} \in%
  \ocfp(\a\restricted{h},b_{1},b_{0})
\end{equation*}
che è verde per ipotesi.
\end{itemize}
\end{document}
