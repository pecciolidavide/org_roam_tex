% Intended LaTeX compiler: pdflatex
\documentclass[../main]{subfiles}


\begin{document}

Si utilizza la notazione di ``\href{20250609104647-buona_codifica_di_un_linguaggio.org}{Aritmetizzazione della sintassi}''
\section{Teoria decidibile}
\label{sec:org4f5ae71}
Sia \(L\) un \href{20250130162057-linguaggio_del_prim_ordine.org}{linguaggio} \href{20250111143651-insieme_numerabile.org}{numerabile}; una \(L\)-\hyperref[sec:org4f5ae71]{teoria} \(T\) si dice \uline{decidibile} se \(\operatorname{Teor}_{T}^{\#} \subseteq \N\) è \href{20250216173925-insieme_ricorsivo.org}{ricorsivo}.

In caso contrario, \(T\) si dice \uline{indecidibile}
\section{Teoria decidibile è ricorsivamente assiomatizzabile}
\label{sec:orge17530d}
Si noti che se \(T\) è decidibile, allora è anche \href{20250609135250-teoria_ricorsivamente_assiomatizzabile.org}{ricorsivamente assiomatizzabile}: infatti \(\operatorname{Teor}_{T}\) è un insieme di assiomi per \(T\).
\section{Teoria essenzialmente indecidibile}
\label{sec:orgead4b60}
Una teoria \(T\) è \uline{essenzialmente indecidibile} se ogni \href{20250130114950-teoria_del_prim_ordine.org}{estensione} \href{20250609162657-teoria_coerente.org}{coerente} di \(T\) è indecidibile.

Si noti che se \(T\) è \href{20250131123151-teoria_completa.org}{completa}, allora \(T\) è essenzialmente indecidibile se e solo se è indecidibile.

Infatti, essendo completa, ogni estensione ``propria'' di \(T\) è incoerente.
\end{document}
