% Intended LaTeX compiler: pdflatex
\documentclass[../main]{subfiles}


\begin{document}


\maketitle
\section{Lezione 2 - \textit{<2025-09-24 Wed>}}
\label{sec:org4573ec1}

\begin{itemize}
\item \uline{Definizione}: Vector space over \(\C\).
\item Examples of vector spaces.
\item \uline{Definizione}: Linear independent vectors.
\item \uline{Definizione}: Vector subspace.
\item \uline{Definizione}: Span and basis.
\item \uline{Definizione}: Quotient space.
\item \uline{Definizione}: Direct sum of vector spaces.
\item \uline{Definizione}: Cartesian product of vector spaces.
\item \uline{Definizione}: Tensor product of vector spaces.
\item \uline{Definizione}: Inner product on a vector space.
\item \uline{Definizione}: Orthogonal vectors and orthonormal/orthogonal basis.
\item \uline{Definizione}: Kronecker delta.
\item \uline{Proposizione}: Graham-Schmidt.
\item \uline{Proposizione}: Cauchy-Schwartz inequality.
\item \uline{Definizione}: Norm and Norm defined by a scalar product.
\item \uline{Definizione}: Distance and Distance induced by a norm.
\item \uline{Proposizione}: Law of parallelogram and inner product induced by a norm.
\item \uline{Definizione}: Linear map and vector space of linear maps.
\item \uline{Definizione}: Isometric map (Isometry).
\end{itemize}
\section{Lezione 3 - \textit{<2025-09-26 Fri>}}
\label{sec:org6b1b2b5}

Appunti mandati da Elisa
\section{Lezione 4 - \textit{<2025-09-30 Tue>}}
\label{sec:org73b4175}

\begin{itemize}
\item \uline{Definizione}: Operatore idempotente
\item \uline{Definizione}: Operazione di Proiezione
\item \uline{Proposizione}: Caratterizzazione somma di proiettori ortogonali
\item \emph{Completeness relation}
\item Algebra commutativa e rappresentazione di un'algebra in uno spazio vettoriale
\item \uline{Definizione}: equivalenza di rappresentazioni
\item Rappresentazione somma diretta
\item Cambio di base

\ldots{} Dal libro hassaniMathematicalPhysicsModern2013
\end{itemize}
\section{Lezione 5 - \textit{<2025-10-01 Wed>}}
\label{sec:org1b27c29}

Dal libro di Hassani\footnote{\url{https://link.springer.com/book/10.1007/978-3-319-01195-0}} (in ordine sparso):
\begin{itemize}
\item Definition 6.1.4
\item Lemma 6.1.5 (lui ha fatto la dimostrazione, assente sul libro)
\item Theorem 6.1.6
\item Theorem 6.2.10 + proof
\item Introduzione del paragrafo 6.3 (prima delle Proposition 6.3.1)
\item Definition 6.4.1
\item Proposition 6.4.2
\item Theorem 6.4.3 + proof
\item Proposition 6.4.4 + proof
\item Theorem 6.4.7 + proof
\item Theorem 6.4.8 + proof
\item Corollary 6.4.9
\item Corollary 6.4.13
\item Definition 6.4.16
\item Lemma 6.4.17 + proof
\item Theorem 6.4.18 (solo enunciato, farà la dimostrazione la prossima volta)
\end{itemize}
\end{document}
