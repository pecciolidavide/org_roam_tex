% Intended LaTeX compiler: pdflatex
\documentclass[../main]{subfiles}

\usepackage[hyperref]{biblatex}
\date{}
\title{}
\begin{document}

\section{Proprietà insieme parzialmente trasversale per una relazione di equivalenza in uno spazio polacco}
\label{sec:orgc94c96a}
\subsection{Esercizio 2}
\label{sec:orgc9f7205}

Let \(E\) be an equivalence relation on a Polish space \(X\). A \textbf{partial transversal} for \(E\) is a set \(T \subseteq X\) meeting each \(E\)-equivalence class in at most one point. Show that the following are equivalent:
\begin{enumerate}
\item \(E\) admits an uncountable analytic partial transversal;
\item \(E\) admits an uncountable Borel partial transversal;
\item there is a Borel function \(f: \R \to X\) such that \(f(r_0) \not\mathrel{E} f(r_1)\) for all distinct \(r_0, r_1 \in \R\).
\end{enumerate}
\subsubsection{Soluzione}
\label{sec:orgf9d7dd3}

\paragraph{a. implica b.}
\label{sec:orgbc3538f}

\uline{Osservazione}: se \(T \subseteq X\) è un insieme trasversale parziale, allora ogni \(T' \subseteq T\) è ancora un insieme trasversale parziale.

Inoltre, \href{20250522113216-ogni_insieme_analitico_non_numerabile_ammette_un_sottoinsieme_boreliano_non_numerabile.org}{ogni insieme analitico \(A\) non numerabile ammette un sottoinsieme boreliano \(B\) non numerabile}, in quanto:
\begin{itemize}
\item siccome \(A\) è analitico, allora \(A\) ha la PSP (per il Teorema 3.4.1);
\item siccome \(A\) è non numerabile, allora esiste
\begin{equation*}
  \iota: 2^{\omega}\to A
\end{equation*}
una immersione topologica, ovvero \(\iota\) continua e iniettiva;
\item pertanto, per il Corollario 3.2.7, \(B\coloneqq\iota(2^{\omega}) \subseteq T\) è boreliano (poiché \(2^{\omega} \in \bm{{\operatorname{Bor}}}(2^{\omega})\)) ed è ovviamente non numerabile, poiché ha cardinalità \(2^{\omega}>\omega\).
\end{itemize}

Pertanto l'insieme analitico trasversale parziale \(T\) ammette un sottoinsieme boreliano non numerabile \(T' \subseteq T\), e per l'Osservazione iniziale, \(T'\) è un insieme trasversale parziale.
\paragraph{b. implica a.}
\label{sec:orgae03cc4}

Questo è ovvio, poiché \(\bm{{\operatorname{Bor}}}(X) \subseteq \bm{\Sigma}_{1}^{1}(X)\) per il Corollario 3.1.4.
\paragraph{b. implica c.}
\label{sec:org61e6a7d}

Sia \(T' \subseteq X\) un insieme boreliano trasversale parziale. Allora, per il Corollario 3.2.7 esiste un chiuso \(F \subseteq \omega^{\omega}\) e una funzione continua e iniettiva
\begin{equation*}
g: F \subseteq \omega^{\omega}\to X
\end{equation*}
tale che \(g(F)=T'\).

Inoltre, per il Teorema 1.3.17, esiste una biiezione continua
\begin{equation*}
h: F \subseteq \omega^{\omega}\to \R.
\end{equation*}
In particolare, per il Corollario 3.2.6, \(h\) è un Borel-isomorfismo, e pertanto \(h^{-1}: \R\to F\) è una funzione Boreliana.

Si pone quindi \(f\coloneqq g\circ h^{-1}\). Questa è una funzione boreliana iniettiva (poiché composizione di funzioni iniettive)
\begin{equation*}
f: \R\to X.
\end{equation*}

Siano dunque \(r_{0}\neq r_{1} \in \R\). Allora \(f(r_{0})\neq f(r_{1})\), e \(f(r_{0}), f(r_{1}) \in T'\). Se per assurdo
\begin{equation*}
f(r_{0})\mathrel{E} f(r_{1})
\end{equation*}
si avrebbe che \(T'\) contiene due elementi distinti della stessa classe di \(E\)-equivalenza. Assurdo.

Pertanto, se \(r_{0}\neq r_{1} \in \R\), allora \(f(r_{0})\not\mathrel{E}f(r_{1})\).
\paragraph{c. implica b.}
\label{sec:org497694f}

La funzione \(f\) è necessariamente \uline{iniettiva}, poiché se per assurdo esistessero \(r_{0}\neq r_{1} \in \R\) tali che \(f(r_{0})=f(r_{1})\), allora per la \uline{riflessività} di \(E\):
\begin{equation*}
f(r_{0})\mathrel{E}f(r_{1})
\end{equation*}
e questo contraddice l'ipotesi.

Si consideri dunque \(A \subseteq \R\) non numerabile, \(A \in \bm{{\operatorname{Bor}}}(\R)\): allora \(f(A) \subseteq X\) è boreliano per il Corollario 3.2.7, ed è inoltre un insieme trasversale parziale per \(E\): infatti se per assurdo vi fossero \(x\neq y \in f(A)\) tali che \(x\mathrel{E}y\) allora, siccome \(f\) è iniettiva, esistono \(x_{0}\neq y_{0} \in A\) tali che \(x=f(x_{0})\), \(y=f(y_{0})\), ovvero
\begin{equation*}
f(x_{0})\mathrel{E}f(y_{0}).
\end{equation*}
Questo contraddice l'ipotesi.\qed
\end{document}
