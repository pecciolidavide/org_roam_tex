% Intended LaTeX compiler: pdflatex
\documentclass[../main]{subfiles}


\begin{document}

\maketitle

Sia \(\mathcal{L}\) un \href{20250130162057-linguaggio_del_prim_ordine.org}{linguaggio del prim'ordine}, e siano \(\mathcal{M},\mathcal{N}\) due \(\mathcal{L}\)-\href{20250131103035-struttura_del_prim_ordine.org}{strutture} di \href{20250131103035-struttura_del_prim_ordine.org}{domini}, rispettivamente, \(M\) ed \(N\).
\section{Proprietà delle mappe tra strutture del prim'ordine}
\label{sec:org8eb146b}

\subsection{Delta-Morfismi tra strutture del prim'ordine}
\label{sec:org19fac10}
\begin{definizione}
Sia \(\Delta \subseteq \mathcal{L}\) un insieme di \href{20250131103317-formula_del_prim_ordine.org}{formule}. Una \href{20250202170607-classe_relazione_binaria.org}{funzione} (che sia \href{20250213105339-funzione_parziale.org}{totale} o \href{20250213105339-funzione_parziale.org}{parziale}) \(F:M\to N\) tra i \href{20250131103035-struttura_del_prim_ordine.org}{domini} delle strutture è detta \uline{\(\Delta\)-morfismo} se per ogni \(\varphi(x) \in \Delta\) e per ogni \(a \in (\dom F)^{x}\)\footnote{Con ``\(\vDash\)'' si intende la \href{20250131122913-soddisfazione_di_una_formula.org}{Soddisfazione di una formula}}
\begin{equation*}
M\vDash\varphi(a)\quad\implies\quad N\vDash\varphi(Fa).
\end{equation*}
\end{definizione}

In questo caso diciamo che \(F\) \uline{preserva la verità di tutte le formule di \(\Delta\)}.
\subsection{Mappa elementare}
\label{sec:orgf622959}
\begin{definizione}
Una \href{20250202170607-classe_relazione_binaria.org}{funzione} (che sia \href{20250213105339-funzione_parziale.org}{totale} o \href{20250213105339-funzione_parziale.org}{parziale}) \(F:M\to N\) tra i \href{20250131103035-struttura_del_prim_ordine.org}{domini} delle strutture è detta \uline{mappa elementare} se per ogni \href{20250131103317-formula_del_prim_ordine.org}{formula} \(\varphi(x)\)
e per ogni \(a \in (\dom F)^{x}\) nel \href{20250202173528-dominio_range_e_campo_di_una_classe_relazione.org}{dominio} di \(F\), si ha
\begin{equation*}
M\vDash \varphi(a) \quad\implies\quad N\vDash \varphi(Fa)
\end{equation*}
ovvero se \(F\) è un \hyperref[sec:org19fac10]{\(\Delta\)-morfismo} per \(\Delta=\mathcal{L}\) l'\href{20250131103317-formula_del_prim_ordine.org}{insieme di tutte le formule}.
\end{definizione}
\subsubsection{Immersione elementare}
\label{sec:org9a3f231}

CFR: \hyperref[sec:org6fc2c5a]{Immersione elementare}
\section{Mappe TOTALI tra strutture del prim'ordine}
\label{sec:org326fea0}
\subsection{Morfismo tra strutture del prim'ordine}
\label{sec:orgf592ca2}
\begin{definizione}
Una \href{20250202170607-classe_relazione_binaria.org}{funzione} \href{20250213105339-funzione_parziale.org}{totale} \(F:M\to N\) tra i \href{20250131103035-struttura_del_prim_ordine.org}{domini delle due strutture} è detta \uline{morfismo} se
\begin{itemize}
\item per ogni \href{20250130162057-linguaggio_del_prim_ordine.org}{simbolo di relazione} \(R\) di \href{20250130162057-linguaggio_del_prim_ordine.org}{arietà} \(n\) e per ogni \(a_{1},\dots,a_{n} \in M\),
\begin{equation*}
  (a_{1},\dots,a_{n}) \in R^{M}\,\implies\, \left(F(a_{1}),\dots,F(a_{n})\right) \in R^{N}
\end{equation*}
\item per ogni \href{20250130162057-linguaggio_del_prim_ordine.org}{simbolo di funzione} \(g\) di \href{20250130162057-linguaggio_del_prim_ordine.org}{arietà} \(n\) e per ogni \(a_{1},\dots,a_{n} \in M\),
\begin{equation*}
  F\left[g^{M}(a_{1},\dots,a_{n})\right] = g^{N}\left[F(a_{1}),\dots, F(a_{n})\right]
\end{equation*}
\end{itemize}
\end{definizione}

\begin{prop}
Sia \(F:M\to N\) funzione totale. Sono fatti equivalenti:
\begin{enumerate}
\item \(F\) è un morfismo
\item \(F\) è \hyperref[sec:org19fac10]{\(\Delta\)-morfismo} per \(\Delta=\mathcal{L}_{\text{at}}\) l'\href{20250131103317-formula_del_prim_ordine.org}{insieme delle formule atomiche}, ovvero per ogni \href{20250131103317-formula_del_prim_ordine.org}{formula atomica} \(\varphi(x)\) e per ogni \(a\in M^{x}\) si ha
\begin{equation*}
 M\vDash \varphi(a) \quad\implies\quad N\vDash \varphi(Fa)
\end{equation*}
\end{enumerate}
\end{prop}
\subsection{Morfismo pieno tra strutture del prim'ordine}
\label{sec:org99b66fa}
\begin{definizione}
Una \href{20250202170607-classe_relazione_binaria.org}{funzione} \href{20250213105339-funzione_parziale.org}{totale} \(F:M\to N\) tra i \href{20250131103035-struttura_del_prim_ordine.org}{domini delle due strutture} è detta \uline{morfismo pieno} se
\begin{itemize}
\item per ogni \href{20250130162057-linguaggio_del_prim_ordine.org}{simbolo di relazione} \(R\) di \href{20250130162057-linguaggio_del_prim_ordine.org}{arietà} \(n\) e per ogni \(a_{1},\dots,a_{n} \in M\),
\begin{equation*}
  (a_{1},\dots,a_{n}) \in R^{M}\,\textcolor{red}{\iff}\, \left(F(a_{1}),\dots,F(a_{n})\right) \in R^{N}
\end{equation*}
\item per ogni \href{20250130162057-linguaggio_del_prim_ordine.org}{simbolo di funzione} \(g\) di \href{20250130162057-linguaggio_del_prim_ordine.org}{arietà} \(n\) e per ogni \(a_{1},\dots,a_{n} \in M\),
\begin{equation*}
  F\left[g^{M}(a_{1},\dots,a_{n})\right] = g^{N}\left[F(a_{1}),\dots, F(a_{n})\right]
\end{equation*}
\end{itemize}
\end{definizione}

\begin{prop}
\textbf{Caratterizzazione come \(\Delta\)-morfismo} \href{20250515141706-da_finire.org}{DA FINIRE}
\end{prop}
\subsection{Immersione tra strutture del prim'ordine}
\label{sec:org5fc7307}
\begin{definizione}
Una \href{20250202170607-classe_relazione_binaria.org}{funzione} \href{20250213105339-funzione_parziale.org}{totale} \href{20241219101956-funzione_iniettiva.org}{iniettiva} \(F:M\to N\) tra i \href{20250131103035-struttura_del_prim_ordine.org}{domini delle due strutture} è detta \uline{immersione} se
\begin{itemize}
\item per ogni \href{20250130162057-linguaggio_del_prim_ordine.org}{simbolo di relazione} \(R\) di \href{20250130162057-linguaggio_del_prim_ordine.org}{arietà} \(n\) e per ogni \(a_{1},\dots,a_{n} \in M\),
\begin{equation*}
  (a_{1},\dots,a_{n}) \in R^{M}\,\iff\, \left(F(a_{1}),\dots,F(a_{n})\right) \in R^{N}
\end{equation*}
\item per ogni \href{20250130162057-linguaggio_del_prim_ordine.org}{simbolo di funzione} \(g\) di \href{20250130162057-linguaggio_del_prim_ordine.org}{arietà} \(n\) e per ogni \(a_{1},\dots,a_{n} \in M\),
\begin{equation*}
  F\left[g^{M}(a_{1},\dots,a_{n})\right] = g^{N}\left[F(a_{1}),\dots, F(a_{n})\right]
\end{equation*}
\end{itemize}
\end{definizione}

Una immersione è un morfismo pieno iniettivo.

\begin{prop}
Sia \(F:M\to N\) una funzione totale. Sono fatti equivalenti:
\begin{enumerate}
\item \(F:M\to N\) è una immersione;
\item per ogni \(\varphi(x) \in \mathcal{L}_{\text{qf}}\) (ovvero per \(\varphi(x)\) \href{20250131103317-formula_del_prim_ordine.org}{formula senza quantificatori}) e per ogni \(a \in M^{x}\)
\begin{equation*}
 M\vDash \varphi(a)\quad\textcolor{red}{\iff}\quad N\vDash \varphi(Fa)
\end{equation*}
ovvero sia \(F\) che \(F^{-1}\) (eventualmente \href{20250213105339-funzione_parziale.org}{parziale}) sono dei \hyperref[sec:org19fac10]{\(\Delta\)-morfismi} per \(\Delta=\mathcal{L}_{qf}\).
\item per ogni \(\varphi(x) \in \mathcal{L}_{\text{at}}\) (ovvero per \(\varphi(x)\) \href{20250131103317-formula_del_prim_ordine.org}{formula atomica}) e per ogni \(a \in M^{x}\)
\begin{equation*}
 M\vDash \varphi(a)\quad\textcolor{red}{\iff}\quad N\vDash \varphi(Fa)
\end{equation*}
ovvero sia \(F\) che \(F^{-1}\) (eventualmente \href{20250213105339-funzione_parziale.org}{parziale}) sono dei \hyperref[sec:org19fac10]{\(\Delta\)-morfismi} per \(\Delta=\mathcal{L}_{at}\).
\end{enumerate}

\href{20250515141706-da_finire.org}{DA FINIRE}: Siamo sicuri che l'equivalenza regga?? Sono sicuro di 1->2->3 (1->2 note di LMR Logica 1)
\end{prop}
\subsubsection{Immersione elementare}
\label{sec:org6fc2c5a}
\begin{definizione}
Una \href{20250202170607-classe_relazione_binaria.org}{funzione} \href{20250213105339-funzione_parziale.org}{totale} \(F:M\to N\) tra i \href{20250131103035-struttura_del_prim_ordine.org}{domini delle due strutture} è detta \uline{immersione elementare} se
\begin{itemize}
\item \(F\) è una \hyperref[sec:org5fc7307]{immersione};
\item il range \(\operatorname{ran}F = F[M]\), visto come \href{20250131103212-sottostruttura_del_prim_ordine.org}{sottostruttura} di \(\mathcal{N}\), è \href{20250212102253-sottostruttura_elementare.org}{sottostruttura elementare}
\end{itemize}
\end{definizione}

\href{20250515141706-da_finire.org}{DA FINIRE}:
\begin{itemize}
\item Aggiungere il fatto che
\begin{quote}
Una \href{20250202170607-classe_relazione_binaria.org}{funzione} \(F:M\to N\) tra i \href{20250131103035-struttura_del_prim_ordine.org}{domini delle due strutture} è detta \uline{immersione elementare} se per ogni \href{20250131103317-formula_del_prim_ordine.org}{formula} \(\varphi(x_{1},\dots,x_{n})\)
e per ogni \(a_{1},..,a_{n} \in M\) si ha
\begin{equation*}
	M\vDash \varphi[a_{1},\dots,a_{n}] \quad\implies\quad N\vDash \varphi[F(a_{1}),\dots,F(a_{n})]
\end{equation*}
\end{quote}
\item Vedi: \href{20250212185030-immersione_elementare_induce_isomorfismo.org}{Immersione elementare induce isomorfismo}
\end{itemize}
\subsection{Isomorfismo tra strutture del prim'ordine}
\label{sec:org85b7474}
\begin{definizione}
Una \href{20250202170607-classe_relazione_binaria.org}{funzione} \href{20250213105339-funzione_parziale.org}{totale} \href{20250104111707-funzione_biunivoca.org}{biiettiva} \(F:M\to N\) tra i \href{20250131103035-struttura_del_prim_ordine.org}{domini delle due strutture} è detta \uline{isomorfismo} se sia \(F\) che la sua \href{20250111142446-funzione_inversa.org}{inversa} \(F^{-1}\) sono \hyperref[sec:orgf592ca2]{morfismi tra le due strutture}.
\end{definizione}

\begin{prop}
Sia \(F:M\to N\) una funzione totale \uline{biiettiva}. Sono fatti equivalenti:
\begin{enumerate}
\item \(F:M\to N\) è un isomorfismo;
\item per ogni \href{20250131103317-formula_del_prim_ordine.org}{formula} \(\varphi(x) \in \mathcal{L}\) e per ogni \(a \in M^{x}\)
\begin{equation*}
 M\vDash \varphi(a)\quad\textcolor{red}{\iff}\quad N\vDash \varphi(Fa)
\end{equation*}
ovvero sia \(F\) che \(F^{-1}\) sono dei \hyperref[sec:org19fac10]{\(\Delta\)-morfismi} per \(\Delta=\mathcal{L}\).
\item per ogni \(\varphi(x) \in \mathcal{L}_{\text{at}}\) (ovvero per \(\varphi(x)\) \href{20250131103317-formula_del_prim_ordine.org}{formula atomica}) e per ogni \(a \in M^{x}\)
\begin{equation*}
 M\vDash \varphi(a)\quad\textcolor{red}{\iff}\quad N\vDash \varphi(Fa)
\end{equation*}
ovvero sia \(F\) che \(F^{-1}\) sono dei \hyperref[sec:org19fac10]{\(\Delta\)-morfismi} per \(\Delta=\mathcal{L}_{at}\).
\end{enumerate}

\href{20250515141706-da_finire.org}{DA FINIRE}: Siamo sicuri che l'equivalenza regga?? Sono sicuro di 1->2->3 (1->2 note di LMR Logica 1)
\end{prop}
\section{Mappe PARZIALI tra strutture del prim'ordine}
\label{sec:org9d1157d}

\subsection{Morfismo parziale tra strutture del prim'ordine}
\label{sec:orgbe23eab}

\subsection{Isomorfismo parziale tra strutture del prim'ordine}
\label{sec:org4ccde1a}
\end{document}
