% Intended LaTeX compiler: pdflatex
\documentclass[../main]{subfiles}


\begin{document}

Sia \(\lambda\) un \href{20250203161341-cardinali.org}{cardinale} \href{20250205120448-classe_finita_e_infinita_mk.org}{infinito}, e sia \(T\) una \href{20250130114950-teoria_del_prim_ordine.org}{teoria del prim'ordine}.
\begin{definizione}
Diremo che \(T\) è \(\lambda\)-categorica se, ogni due \href{20250131122945-modello_di_un_insieme_di_formule.org}{modelli} di \(T\) di \href{20241213101756-cardinalita.org}{cardinalità} \(\lambda\) sono \href{20250214120959-mappe_tra_strutture_del_prim_ordine.org}{isomorfi}, ovvero:
se \(M, N\) sono \href{20250131122945-modello_di_un_insieme_di_formule.org}{modelli} di \(T\) e \(\card{M}=\card{N} = \lambda\), allora \(M\cong N\).
\end{definizione}
\begin{definizione}
\(T\) si dice \uline{totalmente categorica} se è \(\lambda\)-categorica per ogni \(\lambda\).
\end{definizione}
\end{document}
