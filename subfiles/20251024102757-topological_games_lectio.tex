% Intended LaTeX compiler: pdflatex
\documentclass[../main]{subfiles}


\begin{document}

\section{Giochi di Schmidt}
\label{sec:orge53254b}

\begin{definizione}
\(\bm{B} \subseteq \R\) è l'insieme dei numeri \uline{badly approximable}. (\(x \in \R\) è badly approximable sse ???)
\end{definizione}

\(\bm{B}\) è di misura di Lebesgue nulla e magro.
\subsection{Alcuni giochi}
\label{sec:org73fb87f}

Per \(T \subseteq \R\) e \(0<\alpha,\beta<1\), \(G_{\text{Sch}}(\alpha,\beta;T)\) è il gioco logico dove i giocatori \textbf{I} e \textbf{II} giocano intervalli chiusi non vuoti \(I_{0}\supseteq I_{1}\supseteq \dots \supseteq I_{i}\supseteq \dots\) di \(\R\) tali che
\begin{align*}
\operatorname{diam}(I_{2n+1}) &= \alpha \, \operatorname{diam}(I_{2n})\\
\operatorname{diam}(I_{2n+2}) &= \beta\, \operatorname{diam}(I_{2n+1})
\end{align*}
per ogni \(n \in \N\).

Il giocatore \textbf{II} vince sse \(\bigcap_{n \in \N} I_{n} \subseteq T\). \textbf{I} altrimenti.

\begin{definizione}
Generalizzazione per \(X\) spazio metrico completo.
\end{definizione}

L'insieme dei \(T \subseteq \R\) tali che \(I\) ha una strategia vincente è un \(\sigma\)-filtro.
\subsection{Definizioni di TDI}
\label{sec:org69e5d46}

Un sottoinsieme \(Y\) di uno spaio topologico \(X\) è
\begin{itemize}
\item \uline{mai denso} in \(X\) se \(\Int(\Cl(Y)) = \emptyset\)
\item \uline{magro} in \(X\) se è unione numerabile di insieme mai densi;
\item \uline{comagro} in \(X\) se \(X\setminus Y\) è magro.
\end{itemize}

\(\operatorname{Mgr}(X)\) è la famiglia dei sottoinsiemi magri di \(X\), chiuso sotto sottoinsiemi e unione numerabile.

Alcuni argomenti accennati
\begin{itemize}
\item Spazi di Baire.
\item Se \(X\) di Baire e \(U \subseteq X\) aperto allora \(U\) di Baire.
\item Giochi di Choquet e spazi di Choquet. (Vedi libro di Oxtoby\footnote{Oxtoby, John C. - \emph{Measure and category.} ISBN: 0-387-90508-1})
\item Strong Choquet Game e spazi di Strong Choquet.
\item Proprietà di Baire
\item Banach-Mazur Game
\item Spazio di Cantor
\item Lo spazio \(2^{\N}\) è compatto e omeomorfi all'insieme di tutti i percorsi attraverso ogni \uline{finitely branching tree} senza nodi terminali.
\item \(2^{\N}\) è omeomorfo all'insieme di Cantor.
\item \(\N^{\N}\) è omeomorfo a \(\R\setminus D\), dove \(D \subseteq \R\) è un insieme numerabile e denso.
\item Spazio Polacco
\end{itemize}

\begin{definizione}
Un insieme \(P \subseteq X\) è \uline{perfetto} se \(P\) è chiuso in \(X\) e senza punti isolati.
\end{definizione}

\begin{thm}
(Teorema di Cantor-Bendixson).
Se \(X\) è uno spazio polacco e \(C \subseteq X\) è chiuso, allora esistono \(U,P\) unici tali che \(U\cup P = C\); \(U\cap P = \emptyset\) e
\begin{itemize}
\item \(U\) è numerabile (possibilmente vuoto), e aperto in \(C\);
\item \(P\) è perfetto (possibilmente vuoto).
\end{itemize}
\end{thm}

\uline{The perfect set game \(G^{*}(A)\)}

\begin{itemize}
\item Gale-Stewart Games
\item Alberi e alberi potati
\item Se \(X\) ha la topologia discreta e \(X^{\omega}\) ha la topologia prodotto, allora i sottoinsiemi chiusi di \(X^{\omega}\) sono esattamente gli insiemi \([T]\), con \(T\) albero potato su \(X\).
\item Teorema di Gale-Stewart
\end{itemize}

\begin{thm}
(Teorema di Martin).
Se \(B \subseteq \omega^{\omega}\) è boreliano, allora \(G(B)\) è determinato.

Assumendo AC, questo vale per ogni \(B \subseteq X^{\omega}\).
\end{thm}

\begin{definizione}
Si definisce l'\uline{assioma di determinatezza per giochi su \(X\)}:
\begin{description}
\item[{\(\mathrm{AD}_{X}\):}] \(G(A)\) è determinato per ogni \(A \subseteq X^{\omega}\).
\end{description}
\end{definizione}

Se \(X\supseteq Y\) allora \(\mathrm{AD}_{X}\oldimplies \mathrm{AD}_{Y}\).

Con \(\mathrm{AD}\) si intende \(\mathrm{AD}_{\omega}\) (equivalente anche a \(\mathrm{AD}_{2}\)).

???

\begin{thm}
(Teorema di Martin-Woodin).
Assumendo \(\mathrm{AD}+\mathrm{DC}\): \(\mathrm{AD}_{\R}\) sse \uline{Unif}.
\end{thm}

\uline{Unif} è la \emph{uniformization property} in \(\R^{2}\).
\end{document}
