% Intended LaTeX compiler: pdflatex
\documentclass[../main]{subfiles}


\begin{document}

\section{Complesso di catene Nucleo e Immagine}
\label{sec:org9f049d1}
Sia \(R\) un \href{20241205141119-anello.org}{anello} commutativo con unità, e siano
\begin{align*}
\mathcal{C}_{\bullet} &\coloneqq \set{(C_{n},\partial^{C}_{n})}_{n \in \Z}\\
\mathcal{D}_{\bullet} &\coloneqq \set{(D_{n},\partial^{D}_{n})}_{n \in \Z}\\
f_{\bullet} &\coloneqq \set{C_{n}\xrightarrow{f_{n}}D_{n}}_{n \in \Z}
\end{align*}
due \href{20250120163114-complesso_di_catene.org}{complessi di catene di \(R\)-moduli} e un \href{20250120163759-categoria_complessi_di_catene.org}{morfismo} tra loro

\begin{definizione}
Il \textbf{\textbf{Kernel}} (o Nucleo) di \(f_{\bullet}\), denotato con \(\ker(f_{\bullet})\), è il \href{20250128151459-sottocomplesso_di_catene.org}{sottocomplesso} di \(\mathcal{C}_{\bullet}\) definito termine a termine dal \href{20241213105201-kernel.org}{kernel}:
\begin{equation*}
(\ker f)_{n} = \ker(f_{n}) = \{ x \in C_{n} \mid f_{n}(x) = 0 \}
\end{equation*}

L'\textbf{\textbf{Immagine}} di \(f_{\bullet}\), denotata con \(\operatorname{Im}(f_{\bullet})\), è il \href{20250128151459-sottocomplesso_di_catene.org}{sottocomplesso} di \(\mathcal{D}_{\bullet}\) definito termine a termine dall'\href{20250202190147-immagine_punto_a_punto_di_due_classi.org}{immagine}:
\begin{equation*}
(\operatorname{Im} f)_{n} = \operatorname{Im}(f_{n}) = \{ f_{n}(x) \mid x \in C_{n} \}
\end{equation*}
\end{definizione}
\end{document}
