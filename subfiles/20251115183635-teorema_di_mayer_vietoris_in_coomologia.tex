% Intended LaTeX compiler: pdflatex
\documentclass[../main]{subfiles}


\begin{document}

\section{Teorema di Mayer-Vietoris (in coomologia)}
\label{sec:org83fa8da}
Sia \(M\) una \href{20250113115909-struttura_differenziabile.org}{varietà differenziabile}, e sia \(\set{U_{0},U_{1}}\) un suo \href{20250103164252-ricoprimento.org}{ricoprimento} \href{20250103145124-topologia.org}{aperto}. Si considerino le seguenti inclusioni:
\begin{equation*}
\begin{tikzcd}
	& {U_0} \\
	{U_0\cap U_1} && {U_0\cup U_1 = M } \\
	& {U_1}
	\arrow["{\jmath_0}", from=1-2, to=2-3]
	\arrow["{\iota_0}", from=2-1, to=1-2]
	\arrow["{\iota_1}"', from=2-1, to=3-2]
	\arrow["{\jmath_1}"', from=3-2, to=2-3]
\end{tikzcd}
\end{equation*}
È possibile farne il \href{20251115174001-pullback_di_una_funzione_tra_varieta_differenziabili.org}{pullback} alle \href{20251115155511-forma_differenziale_in_un_punto.org}{forme differenziali} per ogni \(k \in \N\), ottenendo:
\begin{equation*}
\begin{tikzcd}
	& {A^k(U_0)} \\
	{A^k(U_0\cap U_1)} && {A^k(M )} \\
	& {A^k(U_1)}
	\arrow["{\iota_0^*}"', from=1-2, to=2-1]
	\arrow["{\jmath_0^*}"', from=2-3, to=1-2]
	\arrow["{\jmath_1^*}", from=2-3, to=3-2]
	\arrow["{\iota_1^*}", from=3-2, to=2-1]
\end{tikzcd}
\end{equation*}
Si \href{20251223102452-pullback_di_una_inclusione_tra_varieta_differenziabili.org}{ricorda} che il pullback di una inclusione è la \href{20251201155413-restrizione_di_una_forma_ad_una_sottovarieta.org}{restrizione della forma all'aperto considerato}, ovvero:
\begin{align*}
\iota_{0}^{*}\omega &= \restriction{\omega}{U_{0}\cap U_{1}}\\
\iota_{1}^{*}\omega &= \restriction{\omega}{U_{0}\cap U_{1}}\\
\jmath_{0}^{*}\omega &= \restriction{\omega}{U_{0}}\\
\jmath_{1}^{*}\omega &= \restriction{\omega}{U_{1}}
\end{align*}
Si consideri ora la \href{20241213095808-somma_diretta.org}{somma diretta tra spazi vettoriali} \(A^{k}(U_{0}) \oplus A^{k}(U_{1})\): si definiscono le seguenti funzioni:
\begin{align*}
\jmath^{*}: A^{k}(M) &\longrightarrow A^{k}(U_{0}) \oplus A^{k}(U_{1})\\
\omega &\longmapsto (\jmath_{1}^{*}\omega, \jmath_{2}^{*}\omega) = ( \restriction{\omega}{U_{0}},  \restriction{\omega}{U_{1}})\\[1em]
\iota_{1}^{*}-\iota_{0}^{*} : A^{k}(U_{0}) \oplus A^{k}(U_{1}) &\longrightarrow A^{k}(U_{0}\cap U_{1})\\
(\eta,\tau) &\longmapsto \iota_{1}^{*}\eta - \iota_{0}^{*}\tau = \restriction{\eta}{U_{0}\cap U_{1}} - \restriction{\tau}{U_{0}\cap U_{1}}.
\end{align*}
Si ottiene quindi, considerando queste funzioni per ogni \(k \in \N\) (e ricordando che la somma diretta è commutativa e associativa):
\begin{equation*}
\begin{tikzcd}[row sep=small]
	0 & {A^{\bullet}(M)} & {A^{\bullet}(U_0)\oplus A^{\bullet}(U_1)} & {A^{\bullet}(U_0\cap U_1)} & 0 \\
	& \omega & {( \restriction{\omega}{U_{0}},  \restriction{\omega}{U_{1}})} \\
	&& {(\eta,\tau)} & {\restriction{\eta}{U_{0}\cap U_{1}} - \restriction{\tau}{U_{0}\cap U_{1}}}
	\arrow[from=1-1, to=1-2]
	\arrow["{\jmath^*}", from=1-2, to=1-3]
	\arrow["{\iota_1^*-\iota_0^*}", from=1-3, to=1-4]
	\arrow[from=1-4, to=1-5]
	\arrow[maps to, from=2-2, to=2-3]
	\arrow[maps to, from=3-3, to=3-4]
\end{tikzcd}
\end{equation*}

\begin{thm}
Questa \href{20251115182133-successione_di_spazi_vettoriali_esatta.org}{successione è esatta}.
\end{thm}
\begin{proof}
La dimostrazione si articola in tre fasi:
\begin{enumerate}
\item \uline{\(\jmath^{*}\) è \href{20241219101956-funzione_iniettiva.org}{iniettiva}.}

Sia \(\eta \in A^{\bullet}(M)\) tale che \(\jmath^{*}\eta = 0\). Allora
\begin{equation*}
\restriction{\eta}{U_{0}} = 0,\qquad \restriction{\eta}{U_{1}} = 0
\end{equation*}
e \href{20251201155413-restrizione_di_una_forma_ad_una_sottovarieta.org}{pertanto} \(j = 0\) in \(M=U_{0}\cup U_{1}\).

\item \uline{\(\operatorname{Im}\jmath^{*} = \ker (\iota_{1}^{*}-\iota_{0}^{*})\)}\footnote{Vedi
\begin{itemize}
\item \href{20250202173528-dominio_range_e_campo_di_una_classe_relazione.org}{Range di una funzione}
\item \href{20251121143525-kernel_di_una_funzione_tra_spazi_vettoriali.org}{Kernel di una funzione tra spazi vettoriali}
\end{itemize}}

(\(\subseteq\)): è ovvio.

(\(\supseteq\)): sia \((\eta,\tau) \in \ker (\iota_{1}^{*}-\iota_{0}^{*})\). Allora in \(U_{0}\cap U_{1}\) si ha che \(\eta \equiv \tau\). Definendo quindi
\begin{equation*}
\omega = %
\begin{cases}
\eta & \text{su } U_{0}\\
\tau & \text{su } U_{1}
\end{cases}
\end{equation*}
questa è ben definita e \(\jmath^{*} \omega = (\eta,\tau)\).

\item \uline{\((\iota_{1}^{*}-\iota_{0}^{*})\) è \href{20241213105600-funzione_suriettiva.org}{suriettiva}}.

Si utilizza una partizione dell'unità subordinata al ricoprimento \(\mathcal{U} = \set{U_{0}, U_{1}}\): esistono \href{20250113144722-funzioni_cinfinito_tra_varieta_differenziabili.org}{due funzioni} \(\rho_{0},\rho_{1} \in C^{\infty}(M) = A^{0}(M)\)  tali che i loro \href{20250701115005-supporto_di_una_funzione.org}{supporti}
\begin{align*}
 \operatorname{supp} \rho_{0} &= \overline{\set{
 	x \in M \mid \rho_{0}(x) \neq 0
 }} \subseteq U_{0}\\
 \operatorname{supp} \rho_{1} &= \overline{\set{
 	x \in M \mid \rho_{1}(x) \neq 0
 }} \subseteq U_{1}
\end{align*}
e per ogni \(x \in M\): \(\rho_{0}(x) + \rho_{1}(x) = 1\).

Sia quindi \(\eta \in A^{\bullet}(U_{0}\cap U_{1})\). Si definiscono:
\begin{align*}
 \omega &= -\rho_{1}\eta\quad\text{ su }U_{0}\\
 \tau &= \rho_{0}\eta\quad\text{ su }U_{1}
\end{align*}
(estese a \(0\) dove non sono definite), e si ottiene che
\begin{align*}
 (\iota_{1}^{*}-\iota_{0}^{*})(\omega,\tau) &=%
 	\rho_{0}\eta+\rho_{1}\eta \\
 	&= (\rho_{0}+\rho_{1})\eta = \eta.%
 	\qedhere
\end{align*}
\end{enumerate}
\end{proof}

\begin{oss}
È possibile vedere questa come una \href{20251115182707-sec_di_complessi_di_cocatene.org}{successione} di \href{20251115182320-complesso_di_cocatene.org}{complessi di cocatene}
\end{oss}

\begin{oss}
Tramite lo \href{20251115182954-zig_zag_lemma_per_complessi_di_cocatene.org}{Zig-Zag Lemma}, otteniamo una \href{20251115182133-successione_di_spazi_vettoriali_esatta.org}{successione esatta} in \href{20251115182537-coomologia_di_un_complesso_di_cocatene.org}{coomologia} (\href{20250122122650-quoziente_di_somma_diretta_di_moduli.org}{somma diretta commuta con quoziente}):
\begin{equation*}
\begin{tikzcd}
	\dots \\
	{H^k(M)} & {H^k(U_0)\oplus H^k(U_1)} & {H^k(U_0\cap U_1)} \\
	\\
	{H^{k+1}(M)} & {H^{k+1}(U_0)\oplus H^{k+1}(U_1)} & \dots
	\arrow[from=1-1, to=2-1]
	\arrow["{\jmath^*}", from=2-1, to=2-2]
	\arrow["{\iota_1^*-\iota_0^*}", from=2-2, to=2-3]
	\arrow["{\partial^*}", from=2-3, to=4-1]
	\arrow["{\jmath^*}"', from=4-1, to=4-2]
	\arrow[from=4-2, to=4-3]
\end{tikzcd}
\end{equation*}
ed è possibile calcolare esplicitamente il morfismo di connessione \(\partial^{*}\) (\href{20251115182904-morfismo_tra_complessi_di_cocatene_induce_morfismo_in_coomologia.org}{oltre che gli altri morfismi}), seguendo la costruizione: dato \([\eta] \in H^{k}(U_{0}\cap U_{1})\),
\begin{equation*}
\eta = (\iota_{1}^{*}-\iota_{0}^{*})(-\rho_{1} \eta, \rho_{0} \eta)
\end{equation*}
Facendo il differenziale di \((-\rho_{1} \eta, \rho_{0} \eta)\) su ambo le componenti, si \href{20251115160537-differenziale_di_una_forma.org}{ottiene}
\begin{equation*}
(- \dif \rho_{1} \wedge \eta, \dif \rho_{0} \wedge \eta)
\end{equation*}
in quanto \(\dif\eta = 0\)\footnote{Vedi forme \href{20251115172517-forma_differenziale_chiusa.org}{chiuse} ed \href{20251115172517-forma_differenziale_chiusa.org}{esatte}.}. Ponendo ora
\begin{equation*}
\omega = %
\begin{cases}
-\dif \rho_{1} \wedge \eta & \text{su } U_{0}\\
\dif \rho_{0} \wedge \eta & \text{su } U_{1}
\end{cases}
\end{equation*}
che è ben definita in quanto in \(p \in U_{0}\cap U_{1}\), se le coordinate in un intorno sono \((x^{1},\dots,x^{n})\) si ha
\begin{equation*}
-\dif \rho_{1} \wedge \eta
	= - \dpd{\rho_{1}}{{x^{\mu}}} \dif x^{\mu} \wedge \eta %
	= - \dpd{}{{x^{\mu}}} (1-\rho_{0}) \dif x^{\mu} \wedge \eta
	= \dpd{\rho_{0}}{{x^{\mu}}} \dif x^{\mu} \wedge \eta = \dif \rho_{0} \wedge \eta
\end{equation*}
e si ottiene per definizione che \(\jmath^{*} \omega = (- \dif \rho_{1} \wedge \eta, \dif \rho_{0} \wedge \eta)\).

Quindi \(\partial^{*}[\eta] = [\omega]\), con
\begin{equation*}
\omega = %
\begin{cases}
-\dif \rho_{1} \wedge \eta & \text{su } U_{0}\\
\dif \rho_{0} \wedge \eta & \text{su } U_{1}
\end{cases}
\end{equation*}
\end{oss}
\end{document}
