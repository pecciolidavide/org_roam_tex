% Intended LaTeX compiler: pdflatex
\documentclass[../main]{subfiles}


\begin{document}

\section{Cardinali}
\label{sec:org6bb7238}
Contesto: \href{20250130104245-morse_kelly_set_theory.org}{Morse Kelly Set Theory}
\subsection{Definizione}
\label{sec:org17ac02a}
Un \textbf{cardinale} è un \href{20250203111003-ordinali.org}{ordinale} \(\kappa\) non \href{20250619101109-classi_equipotenti.org}{in biiezione} con nessun \(\alpha < \kappa\).

Si indica con \(\operatorname{Card}\) la \href{20250130104320-classe_mk.org}{classe} dei cardinali.
\subsection{Numeri naturali sono cardinali}
\label{sec:org95e30ce}
Ogni \href{20250202130045-insieme_dei_numeri_naturali_mk.org}{numero naturale} è un cardinale (vedi \href{20250203161110-numeri_naturali_sono_ordinali.org}{Numeri naturali sono ordinali}), e \(\omega\) è il primo cardinale infinito (vedi \href{20250203161110-numeri_naturali_sono_ordinali.org}{Ordinale omega} e \href{20250203161110-numeri_naturali_sono_ordinali.org}{Ordinale omega è il più piccolo ordinale limite})
\subsection{Proprietà dei cardinali}
\label{sec:orga50d7f2}
Se \(\kappa, \lambda\) sono \hyperref[sec:org6bb7238]{cardinali}, allora
\begin{enumerate}
\item \(\kappa=\lambda\) se e solo se \(\kappa \asymp\lambda\) (vedi \href{20250619101109-classi_equipotenti.org}{Classi equipotenti MK});
\item \(\kappa\le \lambda\) (vedi \href{20250203111003-ordinali.org}{Relazione d'ordine sugli ordinali}) se e solo se \(\kappa\preceq \lambda\) (vedi \href{20241219101956-funzione_iniettiva.org}{Classe si inietta MK}).
\end{enumerate}
\subsection{Estremo superiore di un insieme  di cardinali è un cardinale}
\label{sec:orgac99adb}
Sia \(X\) un \href{20250130104331-insieme_mk.org}{insieme} di \hyperref[sec:org6bb7238]{cardinali}. Allora \(\operatorname{sup}(X)\) è un \hyperref[sec:org6bb7238]{cardinale} (vedi \href{20250203111003-ordinali.org}{Relazione d'ordine sugli ordinali} e \href{20250203102516-massimo_e_minimo.org}{Infimum e supremum})
\subsubsection{Dimostrazione}
\label{sec:org0165e24}
Se \(X\) è un insieme di cardinali, allora \(X\) è un insieme di \href{20250203111003-ordinali.org}{ordinali}, e pertanto
\begin{equation*}
\operatorname{sup}(X) = \bigcup X\eqqcolon \lambda
\end{equation*}
(vedi \href{20250131155822-operazioni_insiemistiche_tra_classi_mk.org}{Classe Unione Generalizzata}) ed è un ordinale.

Se per assurdo \(\lambda\) non fosse un cardinale, \href{20241213101756-cardinalita.org}{allora}  \(|\lambda|<\lambda\) e \href{20250203111003-ordinali.org}{pertanto} per qualche \(\kappa \subseteq \lambda\) (ovvero \(\kappa \in X\))
\begin{equation*}
|\lambda|<\kappa\le \lambda
\end{equation*}
e \href{20241213101756-cardinalita.org}{quindi} \(|\kappa|=|\lambda|\). Quindi \(|\kappa|<\kappa\) e \href{20241213101756-cardinalita.org}{pertanto} \(\kappa\) non è un cardinale. Assurdo (per ipotesi \(\kappa \in X\) e \(X\) insieme di cardinali, dunque \(\kappa\) è un cardinale).
\subsection{Corollario}
\label{sec:orgf65cd84}
La \href{20250130104320-classe_mk.org}{classe} \(\operatorname{Card}\) è una \href{20250130104320-classe_mk.org}{classe propria}.
\subsubsection{Dimostrazione}
\label{sec:orgbdfecf8}
Supponiamo che \(\operatorname{Card} \subseteq \operatorname{Ord}\) sia un \href{20250130104331-insieme_mk.org}{insieme}. Allora \(\sup \operatorname{Card} \in \operatorname{Card}\) (per il Teorema). Se consideriamo il \href{20250205152531-numeri_di_hartogs.org}{numero di Hartogs} di \(\sup\operatorname{Card}\) si ha che
\begin{equation*}
\sup \operatorname{Card} < \operatorname{Hrtg}\left(\sup\operatorname{Card}\right) \in \operatorname{Card}
\end{equation*}
(vedi \href{20250205180824-numero_di_hartogs_di_un_ordinale.org}{Numero di Hartogs di un ordinale}) Assurdo.
\end{document}
