% Intended LaTeX compiler: pdflatex
\documentclass[../main]{subfiles}

\usepackage[hyperref]{biblatex}
\date{}
\title{}
\begin{document}

\section{Automorfismo e chiusura algebrica in un modello mostro}
\label{sec:org7c279f8}
Si utilizza la \href{20250612143636-notazione_teoria_dei_modelli.org}{notazione della Teoria dei Modelli}.

Sia \(\mathcal{L}\) un \href{20250130162057-linguaggio_del_prim_ordine.org}{linguaggio del prim'ordine} fissato, \(T\) una \(\mathcal{L}\)-\href{20250130114950-teoria_del_prim_ordine.org}{teoria} \href{20250131123151-teoria_completa.org}{completa} senza \href{20250131122945-modello_di_un_insieme_di_formule.org}{modelli} finiti, ed un modello \(\mathcal{U}\) di \href{20241213101756-cardinalita.org}{cardinalità} \(\kappa>\card{\mathcal{L}}\), con \(\kappa\) \href{20250211123155-cardinale_limite_forte.org}{inaccessibile}. Si lavora nell'ambito di un \href{20250617102733-modello_mostro.org}{MODELLO MOSTRO}.
\subsection{Proposizione}
\label{sec:orgf2eaffd}

Se \(f \in \operatorname{Aut}\mathcal{U}\) è un \href{20250214120959-mappe_tra_strutture_del_prim_ordine.org}{automorfismo}, allora per ogni \(A \subseteq \mathcal{U}\) l'\href{20250202173528-dominio_range_e_campo_di_una_classe_relazione.org}{immagine} commuta con la \href{20250618095344-elementi_algebrici_e_definibili_teoria_dei_modelli.org}{chiusura algebrica}, ovvero:
\begin{equation*}
f[\operatorname{acl}(A)]= \operatorname{acl}\left(f[A]\right)
\end{equation*}
\end{document}
