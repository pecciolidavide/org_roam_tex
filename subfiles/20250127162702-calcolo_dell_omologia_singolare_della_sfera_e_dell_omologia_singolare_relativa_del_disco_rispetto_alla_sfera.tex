% Intended LaTeX compiler: pdflatex
\documentclass[../main]{subfiles}


\begin{document}

Sia \(R\) un \href{20241219112842-pid.org}{PID}. Siano considerino \href{20250115150754-sfera_n_dimensionale.org}{le sfere} e \href{20250127170831-disco_n_dimensionale.org}{i dischi} \(n\)-dimensionali:
\begin{align*}
\mathds{D}^{n}&=\set{x \in \R^{n} : \norma{x}\le 1}\\
\mathds{S}^{n-1}&= \set{x \in \R^{n}: \norma{x}=1}
\end{align*}

Si vogliono calcolare i \href{20241205141053-r_moduli.org}{moduli} di \href{20250122133631-omologia_singolare.org}{omologia singolare} \(H_{q}(\mathds{S}^{n})\) e di \href{20250122154903-omologia_singolare_relativa.org}{omologia singolare relativa} \(H_{q}(\mathds{D}^{n},\mathds{S}^{n-1})\).

\begin{prop}
Si ha che,\footnote{``\(\oplus\)'' indica la \href{20241213095808-somma_diretta.org}{somma diretta}}
\begin{equation*}
H_{q}(\mathds{S}^{0}) = \begin{cases}
R\oplus R & q= 0\\
0 &\text{altrimenti}
\end{cases}
\end{equation*}
mentre, per ogni \(n\ge 1\):
\begin{equation*}
H_{q}(\mathds{S}^{n}) = \begin{cases}
R & q = 0, n\\
0 & \text{altrimenti}
\end{cases}\qquad
H_{q}(\mathds{D}^{n};\mathds{S}^{n-1}) = \begin{cases}
R & q=n\\
0 & \text{altrimenti}.
\end{cases}
\end{equation*}
\end{prop}
\begin{proof}
Vedi \emph{Dimostrazione 2}.
\end{proof}
\end{document}
