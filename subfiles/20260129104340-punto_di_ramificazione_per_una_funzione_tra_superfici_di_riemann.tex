% Intended LaTeX compiler: pdflatex
\documentclass[../main]{subfiles}


\begin{document}

\section{Punto di ramificazione per una funzione tra superfici di Riemann}
\label{sec:org0cb8da2}
Siano \(X,Y\) \href{20260127112828-superficie_di_riemann.org}{superfici di Riemann}, \(F:X\to Y\) \href{20260126110551-funzione_olomorfa.org}{olomorfa} non costante.

\begin{prop}
L'insieme\footnote{Vedi ``\href{20260129104215-forma_normale_locale_per_superfici_di_riemann.org}{Molteplicità di una funzione tra superfici di Riemann}''}
\begin{equation*}
R=\set{p \in X \mid \mathrm{molt}_{p}\,F > 1}
\end{equation*}
è un insieme \href{20250103145124-topologia.org}{chiuso} e \href{20260128123515-sottoinsieme_discreto.org}{discreto}.
\end{prop}

\begin{proof}
\(R\) è un insieme chiuso in quanto \(R = X\setminus \set{p \in X \mid \mathrm{molt}_{p}\,F = 1}\).

Sia ora \(p \in R\). Per il \href{20260129104215-forma_normale_locale_per_superfici_di_riemann.org}{Teorema di Forma Normale}, siano
\begin{align*}
\varphi_{1}: U_{1} &\to V_{1}\\
\varphi_{2}: U_{2} &\to V_{2}
\end{align*}
le carte centrate in \(p\) e \(F(p)\) tali che
\begin{equation*}
\varphi_{2}\circ F\circ \varphi_{1}^{-1}(z) = z^{m}
\end{equation*}
con \href{20260129104215-forma_normale_locale_per_superfici_di_riemann.org}{\(m= \mathrm{molt}_{p}\,F\)}.

Osserviamo che la \href{20260126110551-funzione_olomorfa.org}{funzione olomorfa} \(h(z) = z^{m}\) è iniettiva nell'intorno di ogni \(z_{0} \neq 0\).

Siccome \(\varphi_{1},\varphi_{2}\) sono biiettiva, allora \(F\) è localmente iniettiva nell'intorno di ogni \(q \in U_{1}\), con \(q \neq 0\), e \href{20260129104215-forma_normale_locale_per_superfici_di_riemann.org}{quindi}
\begin{equation*}
\forall q \in U_{1}\setminus \set{p}:\qquad \mathrm{molt}_{p}\,F = 1
\end{equation*}
e pertanto \(U_{1} \cap R = \set{p}\). Allora \(p\) è un \href{20250403131856-punto_isolato.org}{punto isolato}.
\end{proof}

\begin{definizione}
Un punto \(p \in X\) tale che \(\mathrm{molt}_{p}\,F > 1\) si dice \textbf{punto di ramificazione per \(F\)}, mentre \(q\coloneqq F(p) \in Y\) si dice \textbf{punto di diramazione}.
\end{definizione}

\begin{oss}
Se \(X\) è \href{20250103163701-spazio_topologico_compatto.org}{compatto}, \href{20260128172415-sottoinsieme_discreto_in_un_compatto.org}{\(F\) ha un numero finito} di punti di ramificazione e di diramazione.
\end{oss}

\begin{oss}
\uline{Come si trovano i punti di ramificazione?}

Sia \(f:U \subseteq \C\to V \subseteq \C\) l'espressione di \(F\) in alcune carte locali. Si fissano le seguenti nomenclature:
\begin{equation*}
\begin{tikzcd}[ampersand replacement=\&,row sep=tiny]
	{U \subseteq \C} \& {A \subseteq X} \\
	{z_0 \in U} \& {p\in A} \\
	{f(z_0) \in V} \& {F(p)}
	\arrow[tail reversed, from=1-1, to=1-2]
	\arrow[tail reversed, from=2-1, to=2-2]
	\arrow[tail reversed, from=3-1, to=3-2]
\end{tikzcd}
\end{equation*}

Applicando una traslazione, si ottiene che, per costruzione\footnote{Vedi ``\href{20260128124105-ordine_di_una_funzione_olomorfa.org}{Ordine di una funzione olomorfa}''}
\begin{equation*}
\mathrm{mult}_{p}\, F = \mathrm{ord}_{z_{0}} \big(f-f(z_{0})\big)
\end{equation*}
Siccome \(f-f(z_{0})\) è una funzione olomorfa, allora \href{20260128124105-ordine_di_una_funzione_olomorfa.org}{per costruzione}
\begin{equation*}
\mathrm{mult}_{p}\, F = \mathrm{ord}_{z_{0}} \big(f-f(z_{0})\big) =  \mathrm{ord}_{z_{0}} \left(\dod{}{z}\big(f-f(z_{0})\big)\right) +1 = [\mathrm{ord}_{z_{0}} f'] + 1
\end{equation*}
Segue che
\begin{equation*}
\mathrm{mult}_{p}\, F > 1 \IFF [\mathrm{ord}_{z_{0}} f'] > 0 \IFF f'(z_{0}) = 0
\end{equation*}
e pertanto i punti di ramificazione di \(F\) in \(A\) corrispondono agli zeri di \(f'(z)\) in \(U\).
\end{oss}
\end{document}
