% Intended LaTeX compiler: pdflatex
\documentclass[../main]{subfiles}


\begin{document}

\section{Coomologia di De Rham di \(\R\).}
\label{sec:org881e885}

\def\d{\operatorname{d}}
\def\H{\operatorname{H}_{\text{dR}}}

\(\R\) è una \href{20250113115909-struttura_differenziabile.org}{varietà differenziabile} di dimensione \(1\). Pertanto esistono \href{20251115155511-forma_differenziale_in_un_punto.org}{\(k\)-forme} solo per \(k=0,1\). Pertanto dobbiamo calcolare solo la \href{20251115172442-gruppo_di_coomologia_di_de_rham.org}{Coomologia di De Rham}:
\begin{equation*}
\H^{0}(\R),\qquad \H^{1}(\R)
\end{equation*}
mentre tutti gli altri saranno lo \href{20241205142027-spazio_vettoriale.org}{spazio vettoriale} banale \(O\).
\begin{itemize}
\item \(\boxed{\H^{0}(\R)}\)

Si ha che
\begin{equation*}
  \H^{0}(\R) = \frac{\ker \d^{0} }{O} \cong \ker \d^{0}
\end{equation*}
Sia quindi \(f \in \Omega^{0}(\R) = C^{\infty}(\R)\)\footnote{Questo è l'\href{20250113144722-funzioni_cinfinito_tra_varieta_differenziabili.org}{anello delle funzioni \(C^{\infty}\)}.} tale che \(\dif f = 0\), \(\dif f = 0\) sse \(f' = 0\) sse \(f\) è costante (poiché il dominio è connesso).

Pertanto \(\ker\d^{0} \cong \R\), e \(\H^{0}(\R) \cong \R\).
\item \(\boxed{\H^{1}(\R)}\)

Se \(\omega \in \Omega^{1}(\R)\), allora \(\omega= f \dif x\) per \(f \in C^{\infty}(\R)\).
\begin{itemize}
\item Sicuramente \(\dif \omega = \d ( f\dif x) = f (\d\circ \d)(x) = 0\), e pertanto \(\Omega^{1}(\R) = \ker \d^{1}\).

\item Invece, se considero \(g = \int_{0}^{x} f(\tau) \dif \tau \in C^{\infty}(\R)\), allora
\begin{equation*}
\dif g = \omega
\end{equation*}
e pertanto \(\operatorname{rng}\d^{0}=\Omega^{1}(\R)\).
\end{itemize}
Segue che
\begin{equation*}
  \H^{1}(\R) = \frac{\ker \d^{1}}{\operatorname{rng}\d^{0}} = \frac{\Omega^{1}(\R)}{\Omega^{1}(\R)} = O.
\end{equation*}
\end{itemize}

Pertanto, in definitiva, si ha che
\begin{equation*}
\H^{k}(\R) = \begin{cases}
\R & k = 0\\
O & k \ge 1.
\end{cases}
\end{equation*}
\end{document}
