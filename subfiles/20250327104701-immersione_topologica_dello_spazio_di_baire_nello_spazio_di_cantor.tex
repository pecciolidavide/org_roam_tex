% Intended LaTeX compiler: pdflatex
\documentclass[../main]{subfiles}


\begin{document}

\section{Immersione topologica dello spazio di Baire nello spazio di Cantor}
\label{sec:org1057981}
\subsection{Proposizione}
\label{sec:org12c90c6}
Prove that the map
\[
\omega^\omega \rightarrow 2^\omega, \quad x \mapsto \underbrace{0 \ldots 0}_{x(0)} 1 \underbrace{0 \ldots 0}_{x(1)} 1 \underbrace{0 \ldots 0}_{x(2)} 1 \ldots
\]
is a (topological) \href{20250331122739-immersione_topologica.org}{embedding}, and argue that this provides an alternative proof of the fact that
\[
\left\{x \in 2^\omega \mid x(n)=1 \text { for infinitely many } n \in \omega\right\}
\]
is a \href{20250301193045-sottoinsieme_denso.org}{dense} \href{20250301194013-spazio_polacco.org}{Polish} \href{20250103163814-sottospazio_topologico.org}{subspace} of \(2^\omega\). In contrast, show that \(2^\omega\) cannot be embedded as a dense subset in \(\omega^\omega\).

{[}Hint. Use compactness.]
\subsubsection{Dimostrazione}
\label{sec:org615062a}

\paragraph{Prima parte}
\label{sec:org65f679d}
Sia \(f:\omega^{\omega}\to 2^{\omega}\) la funzione descritta. Sia, per ogni \(s \in \omega^{<\omega}\):
\begin{equation*}
\bm{N}_{s} \coloneqq \set{x \in \omega^{\omega}\mid x\upharpoonright s =s }
\end{equation*}
Si costruisce quindi l'\$\(\omega\)\$-schema associato ad \(f\):
\begin{align*}
\mathcal{S} &\coloneqq \set{B_{s}\mid s \in \omega^{<\omega}}\\
B_{s} &\coloneqq f(\bm{N}_{s})
\end{align*}

Si applica il lemma 1.3.9.
\begin{itemize}
\item Si ha che \(B_{s}\) è aperto in \(2^{\omega}\), per definizione di \href{20250109154723-topologia_prodotto.org}{topologia prodotto} (sono insiemi con un numero finito di componenti fissate).
\item Sia \(s \in \omega^{<\omega}\) e siano \(x,y \in \omega\), \(x\neq y\). Supponiamo per assurdo che \(B_{s\concat x} \cap B_{s\concat y} \neq \emptyset\), ovvero
\begin{equation*}
  f(\bm{N}_{s\concat x}) \cap f(\bm{N}_{s\concat y})\neq \emptyset.
\end{equation*}
Dunque esistono \(a \in \bm{N}_{s\concat x}\), \(b \in \bm{N}_{s\concat y}\) tali che \(f(a)=f(b)\). Siccome \(f\) è \href{20241219101956-funzione_iniettiva.org}{iniettiva}, \(a=b\), ma \(\bm{N}_{s\concat x}\cap \bm{N}_{s\concat y}=\emptyset\). Assurdo.
\end{itemize}
Per il punto (c) si ha che \(f\) è una immersione topologica.
\paragraph{Seconda parte}
\label{sec:org92bf21e}
Sia
\begin{equation*}
A\coloneqq \set{x \in 2^{\omega}\mid x(n)=1\text{ per un numero infinito di }n \in\omega}
\end{equation*}

Si ha che \(A=f(\omega^{\omega})\). Infatti, l'inclusione ``\(\supseteq\)'' è ovvia. Per il viceversa, si definisce per ogni \(y \in A\), l'insieme
\begin{equation*}
Z_{y} \coloneqq \set{n \in \omega\mid y(n) = 1} \subseteq \omega
\end{equation*}
Si ordina \(Z_{y}\) in maniera crescente, \(Z_{y} \coloneqq (y_{i})_{i \in \omega}\).

Si definisce \(x \in \omega^{\omega}\) come:
\begin{equation*}
x(i) \coloneqq\begin{cases}
y_{0} & i=0\\
y_{i} - y_{i-1} - 1& i>0
\end{cases}
\end{equation*}
e si ottiene che \(y=f(x)\).

Questo dimostra che \(A\) è un sottospazio polacco di \(2^{\omega}\), \href{20250303115241-proprieta_di_chiusura_degli_spazi_polacchi.org}{poiché} \href{20250111142332-omeomorfismo.org}{omeomorfo} a \(\omega^{\omega}\) spazio polacco.

Inoltre, sia \(y \in 2^{\omega}\), e sia, per ogni \(n \in \omega\):
\begin{equation*}
x_{n}(i) \coloneqq \begin{cases}
y(i) & i\le n\\
1 & i >n
\end{cases}
\end{equation*}

Allora \(x_{n} \in A\) e \(x_{n}\to y\), in quanto, considerata la distanza su \(2^{\omega}\):
\begin{equation*}
d(\eta,\tau) \coloneqq \begin{cases}
0 & \eta=\tau\\
2^{-(n+1)} & \eta\neq \tau \text{ e }n\text{ è il più piccolo t.c. }\eta(n)\neq \tau(n)
\end{cases}
\end{equation*}
per ogni \(\varepsilon>0\) esiste \(N \in \omega\), \(N\coloneqq \lceil -\ln_{2}(\varepsilon)-1\rceil\) tale che per ogni \(n>N\): \(d(x_{n}, y)<\varepsilon\). Quindi \(A\) è denso in \(2^{\omega}\), per la \href{20250303121451-caratterizzazione_della_chiusura_in_termini_di_successioni.org}{caratterizzazione della chiusura per successioni}.
\paragraph{Terza parte}
\label{sec:org9ceb583}
Lo spazio \(\omega^{\omega}\) è \href{20250301193401-spazio_topologico_metrizzabile.org}{metrizzabile}, e pertanto \href{20250109155715-spazio_topologico_di_hausdorff.org}{T2}.

Supponiamo che per assurdo esista \(\iota: 2^{\omega}\to \omega^{\omega}\) immersione topologica tale che \(\iota(2^{\omega})\eqqcolon X\) sia denso in \(\omega^{\omega}\).

Siccome \(2^{\omega}\) è compatto, allora \(X\) è compatto, ed inoltre \(\operatorname{Cl}(X) = \omega^{\omega}\). Ma in uno spazio di Hausdorff \href{20250331174140-compatto_in_un_haussdorf_e_chiuso.org}{i compatti sono chiusi}, e \href{20250103144944-chiusura_topologica.org}{pertanto}
\begin{equation*}
X=\operatorname{Cl}(X)=\omega^{\omega}
\end{equation*}

Questo è un assurdo, poiché \(\omega^{\omega}\) non è compatto.
\end{document}
