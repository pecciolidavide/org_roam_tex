% Intended LaTeX compiler: pdflatex
\documentclass[../main]{subfiles}


\begin{document}

\section{Categoria dei Complessi di Catene di R-Moduli}
\label{sec:org398026e}
Sia \(R\) un \href{20241205141119-anello.org}{anello} commutativo con unità. Si definisce la \href{20241126100904-categoria.org}{categoria} \(\cat{Ch}_{R}\) dei complessi di catene di \(R\)-moduli come segue:
\subsection{Oggetti}
\label{sec:org49970fe}
I \href{20250120163114-complesso_di_catene.org}{complessi di catene di \(R\)-moduli} \(\mathcal{C}_{\bullet}\);
\subsection{Morfismi tra complessi di catene}
\label{sec:orgb002d64}
Se \(\mathcal{C}_{\bullet} = \set{(C_{n},\partial_{n}^{C})}_{n \in \Z}\) e \(\mathcal{D}_{\bullet} = \set{(D_{n},\partial_{n}^{D})}_{z \in \Z}\) sono due complessi di catene, allora un morfismo
\begin{equation*}
f_{\bullet}: \mathcal{C}_{\bullet}\longrightarrow \mathcal{D}_{\bullet}
\end{equation*}
è una collezione di \href{20241206115416-morfismi_r_moduli.org}{morfismi tra \(R\)-moduli}
\begin{equation*}
\set{f_{n}:C_{n}\longrightarrow D_{n}}_{n \in \Z}
\end{equation*}
tali che, per ogni \(n\),
\begin{equation*}
f_{n-1}\circ \partial_{n}^{C} = \partial_{n}^{D}\circ f_{n}
\end{equation*}
ovvero commuti il seguente diagramma
\begin{equation*}
\begin{tikzcd}[ampersand replacement=\&,sep=large]
	\cdots \& {C_{n+2}} \& {C_{n+1}} \& {C_n} \& {C_{n-1}} \& {C_{n-2}} \& \cdots \\
	\cdots \& {D_{n+2}} \& {D_{n+1}} \& {D_n} \& {D_{n-1}} \& {D_{n-2}} \& \cdots
	\arrow["{\partial^C_{n+3}}", from=1-1, to=1-2]
	\arrow["{\partial^C_{n+2}}", from=1-2, to=1-3]
	\arrow["{f_{n+2}}"', from=1-2, to=2-2]
	\arrow["{\partial^C_{n+1}}", from=1-3, to=1-4]
	\arrow["{f_{n+1}}"', from=1-3, to=2-3]
	\arrow["{\partial^C_{n}}", from=1-4, to=1-5]
	\arrow["{f_{n}}"', from=1-4, to=2-4]
	\arrow["{\partial^C_{n-1}}", from=1-5, to=1-6]
	\arrow["{f_{n-1}}"', from=1-5, to=2-5]
	\arrow["{\partial^C_{n-2}}", from=1-6, to=1-7]
	\arrow["{f_{n-2}}"', from=1-6, to=2-6]
	\arrow["{\partial^D_{n+3}}"', from=2-1, to=2-2]
	\arrow["{\partial^D_{n+2}}"', from=2-2, to=2-3]
	\arrow["{\partial^D_{n+1}}"', from=2-3, to=2-4]
	\arrow["{\partial^D_{n}}"', from=2-4, to=2-5]
	\arrow["{\partial^D_{n-1}}"', from=2-5, to=2-6]
	\arrow["{\partial^D_{n-2}}"', from=2-6, to=2-7]
\end{tikzcd}
\end{equation*}

\begin{definizione}
Sia \(f_{\bullet}: \mathcal{C}_{\bullet} \longrightarrow \mathcal{D}_{\bullet}\) un morfismo di complessi di catene.
\begin{itemize}
\item Il morfismo \(f_{\bullet}\) si dice \textbf{\textbf{iniettivo}} (o monomorfismo) se ogni componente \(f_{n}: C_{n} \to D_{n}\) è un \href{20241206115416-morfismi_r_moduli.org}{omomorfismo} \href{20241219101956-funzione_iniettiva.org}{iniettivo} di moduli.
\item Il morfismo \(f_{\bullet}\) si dice \textbf{\textbf{suriettivo}} (o epimorfismo) se ogni componente \(f_{n}: C_{n} \to D_{n}\) è un \href{20241206115416-morfismi_r_moduli.org}{omomorfismo} \href{20241213105600-funzione_suriettiva.org}{suriettivo} di moduli.
\item Il morfismo \(f_{\bullet}\) si dice \textbf{\textbf{biiettivo}} (o isomorfismo) se è sia iniettivo che suriettivo (ovvero se ogni \(f_{n}\) è un \href{20241206115416-morfismi_r_moduli.org}{isomorfismo} di moduli).
\end{itemize}
\end{definizione}
\end{document}
