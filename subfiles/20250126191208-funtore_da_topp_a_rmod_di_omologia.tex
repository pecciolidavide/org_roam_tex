% Intended LaTeX compiler: pdflatex
\documentclass[../main]{subfiles}


\begin{document}

Sia \(R\) un \href{20241219112842-pid.org}{PID}. Si considerino i \href{20241205131958-funtore.org}{funtori} tra \href{20241126100904-categoria.org}{categorie}:
\begin{enumerate}
\item dalla \href{20250122160145-categoria_topp.org}{categoria \(\cat{TopP}\)} alla \href{20250120163759-categoria_complessi_di_catene.org}{categoria \(\cat{Ch}_{R}\)}, \href{20250124163705-funtore_da_topp_a_chr_diesis.org}{diesis};
\item dalla \href{20250120163759-categoria_complessi_di_catene.org}{categoria \(\cat{Ch}_{R}\)} alla \href{20241206115740-categoria_degli_r_moduli.org}{categoria \(R\text{-}\cat{Mod}\)}, \href{20250120165029-funtore_tra_chr_e_rmod.org}{di omologia}.
\end{enumerate}

Componendoli, si ottiene un funtore
\begin{equation*}
(H_{n},\overline{\star}): \cat{TopP}\longrightarrow R\text{-}\cat{Mod}
\end{equation*}
che alle \href{20250122154728-coppia_topologica.org}{coppie topologiche \((X,A)\)} assegna il \href{20241205141053-r_moduli.org}{modulo} di \href{20250122154903-omologia_singolare_relativa.org}{omologia singolare relativa}
\begin{equation*}
H_{n}(X,A)
\end{equation*}
e ai morfismi
\begin{equation*}
f:(X,A)\longrightarrow (Y,B)
\end{equation*}
assegna
\begin{equation*}
\overline{f_{\star}}:H_{n}(X,A) \longrightarrow H_{n}(Y,B)
\end{equation*}

Spesso si indicherà \(\overline{f_{\star}}\) con \(f_{\star}\), rendendo il funtore:
\begin{equation*}
(H_{n},\star): \cat{TopP}\longrightarrow R\text{-}\cat{Mod}
\end{equation*}
\end{document}
