% Intended LaTeX compiler: pdflatex
\documentclass[../main]{subfiles}


\begin{document}

\begin{prop}
Sia \(0=x_{0}<x_{1}<\dots<x_{N}=1\) una partizione di \([0,1]\), e sia\footnote{Con \(\chi_{A}\) si intende la \href{20250215160218-funzione_caratteristica.org}{funzione caratteristica} di \(A\).}
\begin{equation*}
c(x) = \sum_{i=0}^{N-1}\alpha_{i}\chi_{[x_{i},x_{i+1})}
\end{equation*}
Allora \(c(x)\) può essere scritto come combinazione lineare di \href{20250624161413-funzione_di_heaviside.org}{Funzioni di Heaviside}:
\begin{align*}
\chi_{[x_{i},x_{i+1})} &= H(x-x_{i})-H(x-x_{i+1})
;\\[1ex]
c(x) &= \sum_{i=0}^{N-1}\left(\alpha_{i}H(x-x_{i})-\alpha_{i} H(x-x_{i+1})\right)\\
&=\sum_{i=0}^{N} c_{i} H(x-x_{i}).
\end{align*}
\end{prop}
\begin{prop}
Se \(c(x)\) è come sopra, segue che la \href{20250701080039-derivata_distribuzioni.org}{derivata distribuzionale} di \(c\), \(c'(x)\), è\footnote{Con \(\delta(x)\) si intende la \href{20250625100133-delta_di_dirac.org}{Delta di Dirac}, e con \(\delta_{a}(x)\) si intende la Delta di Dirac centrata in \(a\).}
\begin{equation*}
c'(x) = \sum_{i=0}^{N} c_{i} H'(x-x_{i}) = \sum_{i=0}^{N} c_{i} \delta(x-x_{i}) = \sum_{i=0}^{N} c_{i} \delta_{x_{i}}(x)
\end{equation*}
\end{prop}
\end{document}
