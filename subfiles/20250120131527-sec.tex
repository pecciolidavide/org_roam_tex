% Intended LaTeX compiler: pdflatex
\documentclass[../main]{subfiles}


\begin{document}

\section{SEC di Moduli}
\label{sec:org8279e22}
Sia \(R\) un \href{20241205141119-anello.org}{anello} commutativo con unità, e siano \(M, N, P, 0\) \href{20241205141053-r_moduli.org}{\(R\)-moduli}.

\begin{definizione}
La \href{20250120125644-successione_di_r_moduli.org}{successione} \href{20250120125004-successione_di_r_moduli_esatta.org}{esatta}
\begin{equation*}
\begin{tikzcd}[ampersand replacement=\&]
	0 \& M \& N \& P \& 0
	\arrow[from=1-1, to=1-2]
	\arrow["f", from=1-2, to=1-3]
	\arrow["g", from=1-3, to=1-4]
	\arrow[from=1-4, to=1-5]
\end{tikzcd}
\end{equation*}
si dice \textbf{successione esatta corta} (SEC)
\end{definizione}

\begin{oss}
Per l'appropriato \href{20250120130155-caratterizzazione_di_alcune_successioni_esatte_di_r_moduli.org}{teorema di caratterizzazione}, si ha che
\begin{equation*}
P\cong N/M
\end{equation*}
\end{oss}
\end{document}
