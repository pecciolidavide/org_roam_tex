% Intended LaTeX compiler: pdflatex
\documentclass[../main]{subfiles}


\begin{document}

\section{Ipotesi del continuo}
\label{sec:org2c3ea3b}
Contesto: \href{20250130104245-morse_kelly_set_theory.org}{Morse Kelly Set Theory}

\begin{definizione}
\begin{description}
\item[{Assumento \href{20250206171508-axiom_of_choiche.org}{AC}:}] L'\uline{ipotesi del continuo} è\footnote{Vedi:
\begin{itemize}
\item \href{20250207122627-funzione_aleph.org}{Funzione Aleph}
\item \href{20250612151505-aritmetica_dei_cardinali.org}{Elevamento a potenza di cardinali}
\end{itemize}}
\begin{equation*}
  2^{\aleph_{0}} = \aleph_{1}.
\end{equation*}
ovvero \(\R\asymp\omega_{1}\) (infatti \(\R\asymp\parti{\omega}\)\footnote{Vedi ``\href{20250621113243-cardinalita_dei_reali.org}{Cardinalità dei reali}''} e \(\omega_{1}=\aleph_{1}\)).

Equivalentemente, significa che\footnote{Vedi:
\begin{itemize}
\item \href{20241213101756-cardinalita.org}{Cardinalità}
\end{itemize}}
\begin{equation*}
  \forall\, X \subseteq \R\ (\card{X} \le \aleph_{0} \,\lor\, \card{X} = \card{R})
\end{equation*}

\item[{Senza \href{20250206171508-axiom_of_choiche.org}{AC}:}] L'ipotesi del continua diventa:\footnote{Vedi:
\begin{itemize}
\item \href{20250130104245-morse_kelly_set_theory.org}{Insieme delle parti per MK}
\item ``\(\embeds\)'': \href{20241219101956-funzione_iniettiva.org}{Classe si inietta}
\item ``\(\equipotenti\)'': \href{20250619101109-classi_equipotenti.org}{Classi equipotenti MK}
\end{itemize}}
\begin{equation*}
  \forall A \subseteq \parti{\omega}\ (A\embeds\omega \lor A\equipotenti \parti{\omega})
\end{equation*}
\end{description}
\end{definizione}
\end{document}
