% Intended LaTeX compiler: pdflatex
\documentclass[../main]{subfiles}


\begin{document}

\section{Curva Piana proiettiva liscia complessa}
\label{sec:orga109991}
Sia \(F \in \C[x_{0}:x_{1}:x_{2}]\)\footnote{Vedi ``\href{20241219113434-anello_dei_polinomi.org}{Anello-dei-polinomi}''} un \href{20241231121125-polinomi_omogenei.org}{polinomio omogeneo}, e se ne consideri il \href{20241231112823-radici_polinomiali.org}{luogo degli zeri omogeneo}:\footnote{Vedi ``\href{20241231115051-spazio_proiettivo.org}{Spazio Proiettivo}''}
\begin{equation*}
X\coloneqq \set{(x_{0}:x_{1}:x_{2}) \in \mathds{P}^{2}_{\C} \mid F(x_{0}:x_{1}:x_{2}) = 0} = V(F) \subseteq \mathds{P}^{2}_{\C}
\end{equation*}

\begin{prop}
Se per ogni \(p \in X\) il \href{20250624171244-gradiente_di_una_funzione.org}{gradiente} \(\nabla F(p) \neq \bm{0}\)\footnote{In questo caso si dice che \(X\) è \uline{liscia}.}, allora \(X\) è una \href{20260127112828-superficie_di_riemann.org}{superficie di Riemann} \href{20250103163701-spazio_topologico_compatto.org}{compatta}.
\end{prop}

\begin{definizione}
Si definisce il grado di \(X\) come il \href{20241231124742-grado_polinomi.org}{grado di \(F\)}:
\begin{equation*}
\deg X \coloneqq \deg F
\end{equation*}
\end{definizione}
\end{document}
