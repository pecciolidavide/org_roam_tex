% Intended LaTeX compiler: pdflatex
\documentclass[../main]{subfiles}


\begin{document}

Sia \(x\) un \href{20250130104331-insieme_mk.org}{insieme}.

Si definisce il \textbf{successore di \(x\)} come l'insieme
\begin{equation*}
\operatorname{S}(x) \coloneqq x\cup\set{x}
\end{equation*}
\section{Successore di un numero naturale}
\label{sec:orge74096c}
Ricordando la \href{20250202130045-insieme_dei_numeri_naturali_mk.org}{definizione di numeri naturali} nella teoria degli insiemi, è naturale la definizione di successore:
\begin{align*}
\operatorname{S}: \N &\longrightarrow \N\\
n &\longmapsto n+1
\end{align*}
\section{Formalizzazione in MK}
\label{sec:org8bf20be}

Contesto: \href{20250130104245-morse_kelly_set_theory.org}{Morse Kelly Set Theory}

Per l'Axiom of Comprehension, se \(x\) è un insieme \(\set{x}\) è un insieme e \(x\cup \set{x}\) è un insieme.
\subsection{Osservazione}
\label{sec:org46d73d6}
Per ogni insieme \(x\), \(x\neq \operatorname{S}(x)\). Infatti \(x \in S(x)\), ma \(x\notin x\) (vedi \href{20250131180704-nessun_insieme_appartiene_a_se_stesso.org}{Nessuna classe appartiene a se stessa}) e pertanto, per l'Axiom of Extentionality, \(x\neq \operatorname{S}(x)\).
\end{document}
