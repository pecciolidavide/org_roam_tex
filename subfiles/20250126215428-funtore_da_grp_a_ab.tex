% Intended LaTeX compiler: pdflatex
\documentclass[../main]{subfiles}


\begin{document}

Consideriamo le \href{20241126100904-categoria.org}{categorie}
\begin{itemize}
\item \href{20241205115631-categoria_grp.org}{categoria \(\cat{Grp}\)} dei \href{20241205141146-gruppo_abeliano.org}{gruppi};
\item \href{20250127095321-categoria_ab.org}{categoria \(\cat{Ab}\)} dei \href{20250127093245-gruppo_abeliano.org}{gruppi abeliani}.
\end{itemize}

Si definisce un funtore
\begin{equation*}
(\cdot)^{\operatorname{ab}}:\cat{Grp}\longrightarrow\cat{Ab}
\end{equation*}
che
\begin{itemize}
\item associa a ciascun \href{20241205141146-gruppo_abeliano.org}{gruppo} \(G\) il suo \href{20250127093652-gruppo_abelianizzato.org}{abelianizzato} \(G^{\operatorname{ab}}\), \href{20250127093245-gruppo_abeliano.org}{gruppo abeliano};
\item per ciascun \href{20241206115531-morfismo_di_gruppi.org}{morfismo} \(f:G \longrightarrow H\), si considerano le proiezioni al \href{20250127093819-quoziente_di_gruppo_e_sottogruppo.org}{quoziente} \(\pi_{G},\pi_{H}\):
\begin{equation*}
\begin{tikzcd}[ampersand replacement=\&,cramped,column sep=large,row sep=huge]
  G \& H \\
  {G^{\operatorname{ab}}} \& {H^{\operatorname{ab}}}
  \arrow["f", from=1-1, to=1-2]
  \arrow["{\pi_G}"', from=1-1, to=2-1]
  \arrow["{\pi_H}", from=1-2, to=2-2]
  \arrow[dashed, from=2-1, to=2-2]
\end{tikzcd}
\end{equation*}
 Allora \(\pi_{H}\circ f : G \longrightarrow H^{\operatorname{ab}}\) è un morfismo da un gruppo ad un gruppo abeliano, e \href{20250127093602-ogni_funzione_da_un_gruppo_ad_un_gruppo_abeliano_fattorizza_tramite_il_gruppo_abelianizzato.org}{pertanto} esiste un unico morfismo, che chiameremo \(f^{\operatorname{ab}}\) tale che il diagramma commuti.
Il funtore associa ad \(f:G \longrightarrow H\) il morfismo \(f^{\operatorname{ab}}:G^{\operatorname{ab}}\longrightarrow H^{\operatorname{ab}}\).
\end{itemize}
\end{document}
