% Intended LaTeX compiler: pdflatex
\documentclass[../main]{subfiles}


\begin{document}

\section{Insiemi ricorsivi tramite codifica}
\label{sec:org2bddc2b}
\subsection{Definizione}
\label{sec:orga5d6e9a}
Siano \(\varphi: D\to \N\) e \(\psi:E\to \N\) due codifiche ricorsive per gli insiemi numerabili \(E,D\).
\begin{itemize}
\item \(A \subseteq D\) si dice (\uline{semi}) \uline{ricorsivo} rispetto alla codifica \(\varphi\), se \(\varphi(A) \subseteq \N\) è (\href{20250520113238-insieme_semiricorsivo.org}{semi}) \href{20250216173925-insieme_ricorsivo.org}{ricorsivo}.
\item Una \href{20250202170607-classe_relazione_binaria.org}{funzione} \(f:D^{k}\to E\) si dice ricorsiva (\href{20250213105339-funzione_parziale.org}{parziale}) rispetto alle codifiche \(\varphi,\psi\) se è \href{20250215151458-funzioni_ricorsive.org}{ricorsiva} la funzione (\href{20250213105339-funzione_parziale.org}{parziale})
\begin{align*}
f_{\varphi,\psi}: \N^{k} &\longrightarrow \N\\
(x_{1},\dots,x_{k}) &\longmapsto \psi\left[f\left(\varphi^{-1}(x_{1}),\dots,\varphi^{-1}(x_{k})\right)\right]
\end{align*}
\end{itemize}
\subsection{Esempi}
\label{sec:org48d4bbe}
\begin{enumerate}
\item Le funzioni \(\bm{J}^{k}\) forniscono una codifica ricorsiva dell'insieme \(\N^{k}\), che coincide con la nozione di ricorsività.
\end{enumerate}
\end{document}
