% Intended LaTeX compiler: pdflatex
\documentclass[../main]{subfiles}


\begin{document}

\section{Omologia Simpliciale Relativa}
\label{sec:org02dcce3}
Sia \(R\) un \href{20241219112842-pid.org}{PID}, \(K\) un \href{20250121124147-complesso_simpliciale.org}{complesso simpliciale} \href{20250121131005-complesso_simpliciale_totalmente_ordinato.org}{totalmente ordinato} e \(L \subseteq K\) un \href{20250121124213-sottocomplesso_simpliciale.org}{sottocomplesso}.
Si consideri il \href{20250122111953-complesso_di_catene_relative.org}{complesso di catene relative} \(\mathcal{C}_{\bullet}(K,L)\).
\begin{definizione}
L'omologia simpliciale relativa della coppia \((K,L)\) è \href{20250120164857-modulo_di_omologia_dei_complessi_di_catene.org}{l'omologia} del \href{20250120163114-complesso_di_catene.org}{complesso di catene} relative.
\begin{equation*}
H_{q}(K,L) \coloneqq H_{q}\left(\mathcal{C}_{\bullet}(K,L)\right)
\end{equation*}
\end{definizione}
\end{document}
