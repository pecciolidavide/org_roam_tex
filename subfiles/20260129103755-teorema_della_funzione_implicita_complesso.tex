% Intended LaTeX compiler: pdflatex
\documentclass[../main]{subfiles}


\begin{document}

\section{Teorema della funzione implicita complesso}
\label{sec:org9a933ea}
\begin{thm}
Sia \(\Omega \subseteq \C^{2}_{z,w}\) un \href{20250103145124-topologia.org}{aperto}, e sia \(F:\Omega\to \C\) \href{20260126110551-funzione_olomorfa.org}{olomorfa}. Si definisce
\begin{equation*}
X\coloneqq \set{p \in \Omega \mid F(p) = 0}.
\end{equation*}
Sia \(p_{0} = (z_{0},w_{0}) \in X\) tale che \(\pd{F}{w}(p_{0}) \neq 0\)\footnote{Vedi ``\href{20250114103236-derivata_parziale.org}{Derivata parziale}''}.

Allora esiste \(\Omega_{0} \subseteq \Omega\) \href{20250111142313-intorno.org}{intorno} \href{20250103145124-topologia.org}{aperto} di \(p_{0}\) tale che \(X\cap \Omega_{0}\) è il \href{20250104112443-grafico_di_una_funzione.org}{grafico} di una \href{20260126110551-funzione_olomorfa.org}{funzione olomorfa}
\begin{equation*}
w = f(z)
\end{equation*}
definita in un intorno di \(z_{0}\).\qedhere
\end{thm}
\end{document}
