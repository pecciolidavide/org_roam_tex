% Intended LaTeX compiler: pdflatex
\documentclass[../main]{subfiles}

\usepackage[hyperref]{biblatex}
\date{}
\title{}
\begin{document}

\section{Immagine e retroimmagine tramite una funzione}
\label{sec:org337d157}
\subsection{Generalizzazione in MK}
\label{sec:orgdd9904f}

Contesto: \href{20250130104245-morse_kelly_set_theory.org}{Morse Kelly Set Theory}
\subsubsection{Immagine punto a punto di due classi MK}
\label{sec:orgae58971}
Siano \(F\) e \(A\) due \href{20250130104320-classe_mk.org}{classi}. L'immagine punto a punto di \(A\) tramite \(F\) è la classe
\begin{equation*}
F[A] = F``A = \set{y\,|\, \exists\, x \in A\ \left((x,y) \in F\right)}
\end{equation*}
\subsubsection{Preimmagine di una classe MK}
\label{sec:orgb74ad46}
Siano \(F, A\) due \href{20250130104320-classe_mk.org}{classi} arbitrarie. La \textbf{preimmagine} di \(A\) tramite \(F\) è la classe
\begin{equation*}
F^{-1}[A] \coloneqq \set{x\,|\, \exists y \in A\ (x,y) \in F}
\end{equation*}
\end{document}
