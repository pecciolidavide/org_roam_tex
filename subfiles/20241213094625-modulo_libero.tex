% Intended LaTeX compiler: pdflatex
\documentclass[../main]{subfiles}


\begin{document}

\section{Modulo Libero}
\label{sec:org418abab}
Sia \(R\) un \href{20241205141119-anello.org}{anello abeliano con unità}.

\begin{definizione}
Sia \(M\) un \href{20241205141053-r_moduli.org}{\(R\)-modulo}. \(E \subseteq M\) si dice \textbf{base di \(M\)} se
\begin{itemize}
\item \(E\) è \href{20241212142019-insiemi_linearmente_indipendenti.org}{linearmente indipendente}
\item \(E\) è un \href{20241212141101-generatori_modulo.org}{insieme di generatori}.
\end{itemize}
\end{definizione}

\begin{definizione}
Un \href{20241205141053-r_moduli.org}{\(R\)-modulo} si dice \textbf{libero} se ammette una base.
\end{definizione}

\begin{prop}
Sia \(M\) un \href{20241205141053-r_moduli.org}{\(R\)-modulo}. Sono fatti equivalenti:
\begin{enumerate}
\item \(E \subseteq M\) è base di \(M\).
\item ogni \(m \in M\) si può scrivere \textbf{in modo unico} come
\begin{equation*}
m = \sum_{i=1}^k a_i\,e_i\qquad a_i \in R,\ m_i \in E
\end{equation*}
\item \(M\) è isomorfo alla \href{20241213095808-somma_diretta.org}{somma diretta}\footnote{Vedi ``\href{20241212141101-generatori_modulo.org}{Insieme di Generatori di un Modulo}''} \# TODO unire i concetti di modulo libero e somma diretta
\begin{equation*}
M \cong \bigoplus_{e \in E} R \cdot e \cong R^{(E)}.
\end{equation*}
\end{enumerate}
\end{prop}

\begin{esempio}
\(R\) \href{20260112124828-ogni_anello_ha_struttura_di_modulo_su_se_stesso.org}{considerato come \(R\)-modulo} è libero.
\end{esempio}
\section{Estensione morfismo di R-Moduli definito su una base}
\label{sec:orgc6d5a76}
Sia \(R\) un \href{20241205141119-anello.org}{anello abeliano con unità}.

\begin{oss}
Se \(M, N\) sono due \(R\)-moduli, e \(E\) è un base di \(M\), allora è possibile definire un qualsiasi \href{20241206115416-morfismi_r_moduli.org}{morfismo} \(\alpha: M\longrightarrow N\) dichiarandone l'azione solo sugli elementi di una base \(E\) di \(M\), e poi imponendo che sia un morfismo: per ogni \(m \in M\):
\begin{equation*}
\alpha(m) \underset{\dagger}{=} \alpha \big(\sum_{i=1}^{k} r_{i} e_{i}\big) =  \sum_{i = 1}^{k} r_{i} \alpha (e_{i})\qquad r_{i} \in R, e_{i} \in E
\end{equation*}
dove \(\dagger\) è per la caratterizzazione della base.
\end{oss}
\end{document}
