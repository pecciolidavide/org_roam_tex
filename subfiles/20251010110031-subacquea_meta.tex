% Intended LaTeX compiler: pdflatex
\documentclass[../main]{subfiles}

\usepackage[hyperref]{biblatex}
\date{}
\title{}
\begin{document}

\section{Subacquea [META]}
\label{sec:org98164f0}
\subsection{Gestione immersione}
\label{sec:org261dead}

\begin{itemize}
\item \href{20251201092513-sicurezza_in_immersione.org}{Immersione Subacquea}
\item \href{20251201092909-attrezzatura_subacquea.org}{Attrezzatura subacquea}
\item \href{20251201093908-subacquea_lancio_del_pallone_sparabile.org}{Subacquea - Lancio del pallone sparabile}
\end{itemize}
\subsection{Fisiologia e fisica}
\label{sec:org4b0c5ce}

\begin{itemize}
\item \href{20251201094746-decompressione_gas_in_immersione.org}{Decompressione Gas in Immersione}
\end{itemize}
\subsection{Esercizi}
\label{sec:orgfa95019}

\begin{itemize}
\item \href{20251127172859-esercizio_nuoto_in_superficie.org}{Esercizio: Nuoto in superficie}
\item \href{20251127172906-esercizio_apnea_dinamica_senza_attrezzi.org}{Esercizio: Apnea dinamica senza attrezzi}
\item \href{20251127172912-esercizio_salvamento_di_un_apneista_incosciente_sul_fondo.org}{Esercizio: Salvamento di un apneista incosciente sul fondo}
\item \href{20251127172918-esercizio_capovolte_in_raccolta.org}{Esercizio: Capovolte in raccolta}
\item \href{20251127172923-esercizio_equipaggiamento_sul_fondo.org}{Esercizio: Equipaggiamento sul fondo}
\item \href{20251127172928-esercizio_capovolte_con_attrezzatura.org}{Esercizio: Capovolte con attrezzatura}
\end{itemize}
\subsection{Nuoto in superficie - Descrizione}
\label{sec:org3c43d86}

\begin{itemize}
\item Al via dell'istruttore l'allievo parte con il tuffo dal blocco di partenza
\item 200 m in superficie, massimo in 8 minuti, alternando stile/rana ad ogni virata;
\item Al termine del percorso, l'allievo si sostenta fino allo stop dell'istruttore.
\end{itemize}
\subsection{Nuoto in superficie - Tecniche componenti}
\label{sec:orge733615}

\begin{enumerate}
\item Efficacia del tuffo
\item Efficacia degli stili
\item Efficacia delle virate
\item Andatura costante e corretto ritmo respiratorio
\end{enumerate}
\subsection{Apnea dinamica senza attrezzi - Descrizione}
\label{sec:orgfea37c8}

\begin{itemize}
\item Entrata per rotazione;
\item 25 metri + virata, con partenza alla parete;
\item quota costante, mai affiorare, pause evidenti;
\item il compagno segue l'esercizio in superficie, e se necessario si immerge in capovolta per segnalarne il termine;
\item riemersione: atti respiratori e poi \textbf{dice il numero di passate}.
\end{itemize}
\subsection{Apena dinamica senza attrezzi - Tecniche componenti}
\label{sec:org387a542}

\begin{enumerate}
\item Spinta dal bordo e virata corretti
\item Efficacia della rana (braccia e gambe)
\item Pausa evidente
\item Mantenimento della quota
\item Lucidità nell'emersione
\end{enumerate}
\subsection{Salvamento di un apneista incosciente sul fondo - Descrizione}
\label{sec:orgb11d7da}

\begin{itemize}
\item Soccorritore: 50 metri in superficie con la testa fuori (<60s)
\item Infortunato: (+3kg di zavorra) scende sulla verticale al segnale dell'istruttore
\item Soccorritore: a 8 metri dall'infortunato capovolta, poi:
\begin{itemize}
\item toglie la zavorra all'infortunato
\item toglie la \textbf{propria} zavorra
\item risalita (chiudendo le vie aeree all'infortunato)
\item toglie maschera e due ventilazioni
\end{itemize}
\item Trasporto (totale <180s)
\item Estrazione e primo soccorso
\end{itemize}
\subsection{Salvamento di un apneista incosciente sul fondo - Tecniche componenti}
\label{sec:org435e2a9}

\begin{enumerate}
\item Pinneggiata e capovolta
\item Rispetto dei tempi (60s, 180s)
\item Ordine di sgancio della zavorra
\item Corretta presa per risalire (chiudendo le vie aeree)
\item Corretto trasporto, svestizione e ventilazione
\item Corretta estrazione
\item Primo soccorso
\end{enumerate}
\subsection{Capovolte in raccolta - Descrizione}
\label{sec:orgeebc934}

\begin{itemize}
\item Al pronti, l'allievo si stacca dal bordo e si sostenta.
\item 3 capovolte + raccordo, un solo atto respiratorio tra le due.
\item L'allievo attende sostentandosi lo stop dell'istruttore.
\end{itemize}

Il compagno controlla dalla superficie con maschera, pinne e areatore.
\subsection{Capovolte in raccolta - Tecniche componenti}
\label{sec:org63b0450}

\begin{enumerate}
\item Correttezza ed efficacia delle capovolte
\item Raccordo
\item Un solo atto respiratorio
\item Mantenimento del punto fisso
\end{enumerate}
\subsection{Equipaggiamento sul fondo - Descrizione}
\label{sec:orgb5e2d61}

\begin{itemize}
\item L'allievo al \textbf{pronto} tuffo in piedi per affondare e deposita il pacchetto sul fondo.
\item Ventilazioni.
\item Capovolta in raccolta.
\item Vestizione (Pinne, Maschera e Aereatore).
\item Svuotamento maschera \textbf{prima di riemergere}
\item Indossa boccaglio.
\end{itemize}
\subsection{Equipaggiamento sul fondo - Tecniche componenti}
\label{sec:org89b716d}

\begin{enumerate}
\item Capovolta corretta ed efficace
\item Giusta successione di indossamento
\item Totale svuotamento maschera con minima perdita d'aria
\item Rispetto del punto
\item Tranquillità
\end{enumerate}
\subsection{Capovolte con attrezzatura - Descrizione}
\label{sec:org661d4d7}

\begin{itemize}
\item Due allievi insieme.
\item Tuffo per affondare e indosso le pinne prima di riemergere.
\item 4 capovolte: squadra~-~forbice~-~squadra~-~forbice.
\item L'allievo abbandona il boccaglio e risale senza espirare.
\end{itemize}
\subsection{Capovolte con attrezzatura - Tecniche componenti}
\label{sec:orgc7e8583}

\begin{enumerate}
\item Efficacia delle orizzontalizzazioni
\item Posizione di attesa (sulla verticale, sostentandosi con le mani)
\item Correttezza ed efficacia delle capovolte
\item Alternanza
\item Abbandono del boccaglio
\end{enumerate}
\end{document}
