% Intended LaTeX compiler: pdflatex
\documentclass[../main]{subfiles}


\begin{document}

\section{PID}
\label{sec:orgb8c27c5}
\begin{definizione}
Un \href{20241205141119-anello.org}{anello commutativo con unità} \(R\) è detto \textbf{dominio ad \href{20241219112955-ideale.org}{ideali} principali} (PID) se
\begin{enumerate}
\item \(\forall\,a,a' \in R\), se \(aa' = 0\) allora \(a=0\) oppure \(a'=0\) (ovvero \(R\) è un \href{20250103143950-dominio_di_integrita.org}{Dominio di integrità});
\item per ogni \href{20241219112955-ideale.org}{ideale} \(I\) di \(R\) esiste \(a \in R\) tale che \(I=(a)\)\footnote{Vedi anche \href{20241219113154-ideale_generato.org}{Ideale-Generato}}
\end{enumerate}
\end{definizione}
\begin{esempio}
Sono PID i \href{20241205142049-campo.org}{campi}, \(\Z\) e \(\K[x]\)\footnote{Vedi anche \href{20241219113434-anello_dei_polinomi.org}{Anello-dei-polinomi} e \href{20250102115740-anello_dei_polinomi_ad_ideali_principali.org}{Anello dei polinomi ad ideali principali}} con la divisione euclidea
\end{esempio}
\end{document}
