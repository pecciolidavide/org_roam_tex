% Intended LaTeX compiler: pdflatex
\documentclass[../main]{subfiles}


\begin{document}

\section{Catena elementare di modelli}
\label{sec:orge472b83}
Sia \(\mathcal{L}\) un \href{20250130162057-linguaggio_del_prim_ordine.org}{linguaggio del prim'ordine}.
\subsection{Definizione}
\label{sec:org2d6e9f0}

Una \uline{catena elementare} \(\langle M_{i}\mid i \in \lambda\rangle\) per qualche \href{20250203161341-cardinali.org}{cardinale} \(\lambda\) è una \href{20250206170922-sequenze_e_stringhe.org}{sequenza} di \(\mathcal{L}\)-\href{20250131103035-struttura_del_prim_ordine.org}{strutture} tali che per ogni \(i<j<\lambda\): \(M_{i}\preceq M_{j}\) è \href{20250212102253-sottostruttura_elementare.org}{sottostruttura elementare}.
\subsection{Unione di una catena elementare}
\label{sec:org534aa79}
L'\uline{unione} (o il \uline{limite}) di una catena elementare \(\langle M_{i}\mid i \in \lambda\rangle\) è la \(\mathcal{L}\)-struttura fatta come segue:
\begin{itemize}
\item il \href{20250131103035-struttura_del_prim_ordine.org}{dominio} è l'unione \(\bigcup_{i<\lambda} M_{i}\);
\item le funzioni e le relazioni sono l'unione delle funzioni e delle relazioni degli \(M_{i}\).
\end{itemize}
\subsubsection{Lemma}
\label{sec:orgd676c9c}
Sia \(N\) il limite della catena elementare \(\langle M_{i}\mid i \in \lambda\rangle\). Allora, per ogni \(i<\lambda\), \(M_{i}\) è \href{20250212102253-sottostruttura_elementare.org}{sottostruttura elementare} di \(N\): \(M_{i}\preceq N\).
\end{document}
