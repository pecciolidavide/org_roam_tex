% Intended LaTeX compiler: pdflatex
\documentclass[../main]{subfiles}


\begin{document}

\section{Modulo di omologia dei complessi di catene}
\label{sec:org751e5fe}
Sia \(R\) un \href{20241205141119-anello.org}{anello} commutativo con unità.

\begin{definizione}
Sia \(\mathcal{C}_{\bullet} = \set{(C_{n},\partial_{n})}_{n \in \Z}\) un \href{20250120163114-complesso_di_catene.org}{complesso di catene di \(R\)-moduli}. Fissato \(n \in \Z\) si definiscono
\begin{itemize}
\item il \href{20241205141053-r_moduli.org}{modulo} degli \(n\)-cicli\footnote{Vedi ``\href{20241213105201-kernel.org}{Kernel}''}:
\begin{equation*}
  Z_{n}(\mathcal{C}_{\bullet})\coloneqq \operatorname{ker}(\partial_{n})
\end{equation*}
\item il \href{20241205141053-r_moduli.org}{modulo} degli \(n\)-bordi:\footnote{Vedi ``\href{20250202173528-dominio_range_e_campo_di_una_classe_relazione.org}{Range di una funzione}''}
\begin{equation*}
  B_{n}(\mathcal{C}_{\bullet}) \coloneqq \operatorname{Im}(\partial_{n+1})
\end{equation*}
\end{itemize}

Per la definizione di complesso di catene si ha che \(B_{n} \subseteq Z_{n}\) e pertanto si definisce l'\(n\)-esimo modulo di omologia di \(\mathcal{C}_{\bullet}\) come\footnote{Vedi ``\href{20241206142802-sottomoduli.org}{Quoziente di Moduli}''}
\begin{equation*}
H_{n}(\mathcal{C}_{\bullet}) \coloneqq Z_{n}(\mathcal{C}_{\bullet})/B_{n}(\mathcal{C}_{\bullet}).
\end{equation*}
Se \(z_{n} \in Z_{n}(\mathcal{C}_{\bullet})\) allora gli elementi di \(H_{n}(\mathcal{C}_{\bullet})\)  sono
\begin{equation*}
[z_{n}] = z_{n} + B_{n}(\mathcal{C}_{\bullet})
\end{equation*}
\end{definizione}
\end{document}
