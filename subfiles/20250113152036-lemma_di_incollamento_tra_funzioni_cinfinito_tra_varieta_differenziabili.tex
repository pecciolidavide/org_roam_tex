% Intended LaTeX compiler: pdflatex
\documentclass[../main]{subfiles}


\begin{document}

\section{Lemma di incollamento tra funzioni Cinfinito tra varietà differenziabili}
\label{sec:org04ad8b2}
\subsection{Lemma di incollamento}
\label{sec:orgb30436f}
Siano \(M, N\) due \href{20250113115909-struttura_differenziabile.org}{varietà differenziabili}, e sia \(\mathcal{A}\coloneqq\set{(U_{\alpha},\varphi_{\alpha})}\) un \href{20250113103136-atlante_topologico_differenziabile.org}{atlante} di \(M\).
Se per ogni \(\alpha\) esiste una \href{20250113144722-funzioni_cinfinito_tra_varieta_differenziabili.org}{funzione \(C^{\infty}\)} (tra varietà differenziabili, poiché \href{20250113152503-aperti_di_una_varieta_differenziabile_sono_varieta_differenziabili.org}{aperti di una varietà differenziabile sono varietà differenziabili})
\begin{equation*}
F_{\alpha}: U_{\alpha}\longrightarrow N
\end{equation*}
tale che se \(U_{\alpha}\cap U_{\beta} \neq \emptyset\) allora
\begin{equation*}
\restriction{F_{\alpha}}{U_{\alpha}\cap U_{\beta}}\equiv \restriction{F_{\beta}}{U_{\alpha}\cap U_{\beta}}
\end{equation*}
allora la funzione ben definita
\begin{align*}
F: M &\longrightarrow N\\
p &\longmapsto F_{\alpha}(p)\text{ se }p \in U_{\alpha}
\end{align*}
è una funzione \(C^{\infty}\)
\end{document}
