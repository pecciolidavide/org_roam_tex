% Intended LaTeX compiler: pdflatex
\documentclass[../main]{subfiles}

\usepackage[hyperref]{biblatex}
\date{}
\title{}
\begin{document}

\section{Analisi non-standard}
\label{sec:org8c34613}
Sia \(\mathcal{L}\) il \href{20250130162057-linguaggio_del_prim_ordine.org}{linguaggio del prim'ordine} che contiene:
\begin{itemize}
\item per ogni \href{20250203161110-numeri_naturali_sono_ordinali.org}{\(n \in \omega\)} e per ogni \(X \subseteq \R^{n}\), un \href{20250130162057-linguaggio_del_prim_ordine.org}{simbolo di relazione} \(X\) di \href{20250130162057-linguaggio_del_prim_ordine.org}{arietà} \(n\);
\item per ogni \(n \in \omega\) e per ogni \(f:\R^{n}\to \R\), un \href{20250130162057-linguaggio_del_prim_ordine.org}{simbolo di funzione} \(f\) di \href{20250130162057-linguaggio_del_prim_ordine.org}{arietà} \(n\).
\end{itemize}

Il \uline{modello standard} dell'analisi \(\R\) è la \(\mathcal{L}\)-\href{20250131103035-struttura_del_prim_ordine.org}{struttura} di \href{20250131103035-struttura_del_prim_ordine.org}{dominio} \(\R\), e in cui ciascun simbolo di \(\mathcal{L}\) è intepretato nel modo naturale.
\subsection{Teorema}
\label{sec:org772db7e}

Esiste una \href{20250212102253-sottostruttura_elementare.org}{estensione elementare} di \(\R\), detta \(\starR\), tale che \(\R\neq \starR\).

Inoltre \(\starR\) è un \href{20250611115646-campo_ordinato.org}{campo ordinato}, e viene quindi detto \uline{campo degli iperreali}
\subsection{Notazione}
\label{sec:org59c8cf4}

Se \(X\) è un simbolo di relazione di \(\mathcal{L}\), si denoterà con \(\nonstandard{X}\) la sua interpretazione in \(\starR\); se \(f\) è un simbolo di funzione di \(\mathcal{L}\), si denoterà con \(\nonstandard{f}\) la sua interpretazione in \(\starR\).

Quando sarà chiaro dal contesto, si ometterà l'asterisco.
\subsection{Numeri iperreali}
\label{sec:org8fbff16}
Gli elementi di \(\starR\) sono detti \uline{numeri iperreali}; gli elementi di \(R \subseteq \starR\) sono detti \uline{iperreali standard}, mentre gli elementi di \(\starR\setminus \R\) sono detti \uline{iperreali nonstandard}.

Sia \(c \in \starR\):
\begin{itemize}
\item \(c\) è detto \uline{infinitesimo} se per ogni \(\varepsilon \in \R\), \(\varepsilon>0\): \href{20250131122913-soddisfazione_di_una_formula.org}{\(\starR\vDash |c|<\varepsilon\)};
\item \(c\) è detto \uline{infinito} se per ogni \(k \in \R\): \(\starR\vdash k<|c|\), ed è detto \uline{finito} altrimenti.
\end{itemize}

Se \(c \in \starR\) è infinito, allora \(c^{-1}\) è infinitesimale.

Tutti i reali standard sono finiti, e \(0\) è l'unico infinitesimo standard.
\subsection{Proprietà}
\label{sec:org4d613cd}

\subsubsection{Lemma 1}
\label{sec:org7bd144e}

Gli infinitesimi sono chiusi rispetto a somma, prodotto e moltiplicazione per un numero reale.
\subsubsection{Lemma 2}
\label{sec:orgcfc53fd}

\begin{enumerate}
\item Esistono iperreali infiniti e infinitesimi diversi da \(0\).

\item Inoltre, per ogni iperreale finito \(c\) esiste un unico iperreale standard \(b\) tale che \(b-c\) è infinitesimo.
\end{enumerate}
\paragraph{Dimostrazione}
\label{sec:org72bf894}

2 implica 1; infatti preso \(c \in \starR\setminus\R\), allora ci sono due possibilità
\begin{itemize}
\item \(c\) è infinito, allora \(c^{-1}\) è infinitesimo; inoltre \(c^{-1}\neq 0\), poiché
\begin{equation*}
  \R\vDash \forall\,x\ (x\neq 0);
\end{equation*}
\item \(c\) è finito, e allora per 2. esiste un unico \(b \in \R\) tale che \(b-c\) sia infinitesimo; inoltre \(b-c\neq 0\), poiché
\begin{equation*}
  \R\vDash \forall\,x\, \forall\,y (x=y\iff x-y=0)
\end{equation*}
Inoltre, \((b-c)^{-1}\), che esiste poiché \(\R\vDash \forall\,x\ (x\neq 0 \implies \exists\,y\ (xy=1))\), è infinito.
\end{itemize}

Per 2., si ponga per \(c \in \starR\) finito:
\begin{equation*}
b\coloneqq \inf\set{a \in \R\mid c<a}.
\end{equation*}
\subsubsection{Campo degli iperreali non è archimedeo}
\label{sec:org35faff9}
L'esistenza di iperreali infiniti mostra che \(\starR\) non è un \href{20250320150051-gruppo_archimedeo.org}{campo archimedeo}.
\subsubsection{Caratterizzazione dei limiti per gli iperreali}
\label{sec:org01f6b38}
Sia \(f:\R\to \R\) e siano \(a,\ell \in \R\). Allora:
\begin{enumerate}
\item \(\lim_{x\to + \infty} f(x)=+\infty\) se e solo se \(\nonstandard{f}(c)\) è positivo e infinito per ogni \(c>0\) infinito.
\item \(\lim_{x\to+\infty} f(x) = \ell\) se e solo se \(\nonstandard{f}(c)\approx \ell\) per ogni \(c>0\) infinito.
\item \(\lim_{x\to a} f(x) = + \infty\) se e solo se per ogni \(a\neq c\approx a\) si ha che \(\nonstandard{f}(c)>0\) è infinito.
\item \(\lim_{x\to a}(fx) = \ell\) se e solo se per ogni \(a\neq c\approx a\), \(f(c)\approx \ell\).
\end{enumerate}
\subsubsection{Caratterizzazione funzioni continue tramite gli iperreali.}
\label{sec:org18021e4}
Sia \(f:\R\to \R\). Sono equivalenti:
\begin{enumerate}
\item \(f\) è \href{20250103103252-funzione_continua.org}{continua};
\item per ogni \(a,c \in \starR\) finiti, se \(a\approx c\) allora \(\nonstandard{f}(a)\approx \nonstandard{f}(c)\).
\end{enumerate}
\subsubsection{Caratterizzazione funzioni uniformemente contiune tramite gli iperreali}
\label{sec:orge592f41}
Sia \(f:\R\to \R\). Sono equivalenti:
\begin{enumerate}
\item \(f\) è \href{20250611135127-funzione_uniformemente_continua.org}{uniformemente continua};
\item per ogni \(a,c \in \starR\), se \(a\approx c\) allora \(\nonstandard{f}(a)\approx \nonstandard{f}(c)\).
\end{enumerate}
\subsection{Altre definizioni}
\label{sec:org39f8081}

\subsubsection{Monade di un iperreale}
\label{sec:org16f6a99}
In virtà del \ref{sec:org7bd144e}, è possibile definire una \href{20250113110148-relazione_di_equivalenza.org}{relazione di equivalenza} su \(\starR\): scriveremo \(a\approx b\) se \(|a-b|\) è infinitesimo. La \href{20250114100810-quoziente_rispetto_a_relazione_di_equivalenza.org}{classe di equivalenza} di \(c\) è chiamata \uline{monade di \(c\)}.

Per il \ref{sec:orgcfc53fd}, se \(c\) è un iperreale finito, allora vi è un unico numero reale nella monade di \(c\); questo è detto \uline{parte standard} di \(c\), ed è denotato con \(\operatorname{st}(c)\).
\end{document}
