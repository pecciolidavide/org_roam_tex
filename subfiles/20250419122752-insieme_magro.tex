% Intended LaTeX compiler: pdflatex
\documentclass[../main]{subfiles}


\begin{document}

Sia \(X\) uno \href{20250103145124-topologia.org}{spazio topologico}
\section{Definizione}
\label{sec:org670adae}

\begin{itemize}
\item Un \href{20250131155822-operazioni_insiemistiche_tra_classi_mk.org}{sottoinsieme} \(A \subseteq X\) è \uline{magro} se può essere scritto come \href{20250131155822-operazioni_insiemistiche_tra_classi_mk.org}{unione} \href{20250111143651-insieme_numerabile.org}{numerabile} di insiemi \href{20250417180515-insieme_mai_denso.org}{mai densi}.
\item Un sottoinsieme \(A \subseteq X\) è \uline{comagro} se il suo \href{20250317100425-complementare_di_un_insieme.org}{complementare} è magro.

Equivalentemente, \(A\) è \uline{comagro} se e solo se contiene l'\href{20250131155822-operazioni_insiemistiche_tra_classi_mk.org}{intersezione} di una quantità numerabile di aperti \href{20250301193045-sottoinsieme_denso.org}{densi}.
\end{itemize}
\section{Definizione}
\label{sec:orga9e1beb}

Sia \(U \subseteq X\) un aperto.
\begin{itemize}
\item Un sottoinsieme \(A \subseteq X\) si dice \uline{magro in \(U\)} se \(A\cap U\) è magro;
\item Un sottoinsieme \(A \subseteq X\) è \uline{comagro in \(U\)} se \(U\setminus A\) è magro, ovvero se esiste una sequenza di aperti \href{20250301193045-sottoinsieme_denso.org}{densi} in \(U\) la cui intersezione è contenuta in \(A\).
\end{itemize}

Si noti che queste definizioni sono equivalenti a richiedere che \(U\cap A\) sia, rispettivamente, magro e comagro nella topologia di sottospazio di \(U\).
\end{document}
