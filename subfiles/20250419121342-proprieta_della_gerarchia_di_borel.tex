% Intended LaTeX compiler: pdflatex
\documentclass[../main]{subfiles}

\usepackage[hyperref]{biblatex}
\date{}
\title{}
\begin{document}

\section{Proprietà della gerarchia di Borel}
\label{sec:org6e9ccab}
\subsection{Proposizione}
\label{sec:org180f893}

Let \(X\) be a metrizable topological space. Prove by induction on \(1 \leq \alpha < \omega_1\) that:
\begin{enumerate}
\item \(\bm{\Sigma}^0_\alpha(X)\) is closed under countable unions and finite intersections;
\item \(\bm{\Pi}^0_\alpha(X)\) is closed under countable intersections and finite unions;
\item \(\bm{\Delta}^0_\alpha(X)\) is a Boolean algebra, i.e., it is closed under complements, finite unions, and finite intersections.
\end{enumerate}
\subsubsection{Dimostrazione}
\label{sec:org5128c13}

\paragraph{Caso base: \(\alpha=1\)}
\label{sec:orgd7ec588}

\begin{enumerate}
\item Unione di aperti è aperta e intersezione finita di aperti è aperta.
\item Intersezione di chiusi è chiusa e unione finita di chiusi è chiusa.
\item Il complementare di un clopen è ancora un clopen, così come unioni e intersezioni finite.
\end{enumerate}
\paragraph{Passo induttivo}
\label{sec:org2d49624}

Sia l'enunciato vero per ogni \(\beta<\alpha\).
\begin{enumerate}
\item Classi additive
\label{sec:org678f13d}

\begin{itemize}
\item Siano, per ogni \(n \in\omega\), \(A_{n} \in \bm{\Sigma}^0_\alpha(X)\). Per definizione, per ogni \(n \in\omega\), esistono degli \(A_{n}^{m} \in \bm{\Pi}^0_{\beta_{n}^{m}}(X)\), con \(\beta_{n}^{m}<\alpha\), tali che
\begin{equation*}
  A_{n} = \bigcup_{m \in \omega} A_{n}^{m}
\end{equation*}
Allora si ha che
\begin{equation*}
  \bigcup_{n \in \omega} A_{n} = \bigcup_{n,m \in \omega} A_{n}^{m}
\end{equation*}
che è ancora una unione numerabile, ed è quindi un elemento di \(\bm{\Sigma}^0_\alpha(X)\).

\item Siano \(U, V \in \bm{\Sigma}^0_\alpha(X)\). Per definizione esistono degli \(U_{n} \in \bm{\Pi}^0_{\beta^{U}_{n}}(X)\) e degli \(V_{m} \in \bm{\Pi}^0_{\beta^{V}_{m}}(X)\), con \(\beta_{n}^{U}, \beta_{m}^{V} <\alpha\) tali che
\begin{equation*}
  U = \bigcup_{n \in \omega} U_{n},\quad V = \bigcup_{m \in \omega} V_{m}
\end{equation*}

Detto \(\beta_{n,m} \coloneqq \max\set{\beta_{n}^{V},\beta_{m}^{U}}<\alpha\), si ha che
\begin{equation*}
  U\cap V = \left(\bigcup_{n \in\omega} U_{n}\right)\cap\left(\bigcup_{m \in \omega} V_{m}\right) = \bigcup_{n,m \in \omega} (U_{n}\cap V_{m})
\end{equation*}
Per ipotesi induttiva, per ogni \(n,m\) si ha \(U_{n}\cap V_{m} \in \bm{\Pi}^0_{\beta_{n,m}}(X)\) e pertanto \(U\cap V \in \bm{\Sigma}^0_\alpha(X)\)
\end{itemize}
\item Classi moltiplicative
\label{sec:org7edf96b}

\begin{itemize}
\item Siano, per ogni \(n \in\omega\), \(A_{n} \in \bm{\Pi}^0_\alpha(X)\). Per definizione, per ogni \(n \in\omega\), esistono degli \(A_{n}^{m} \in \bm{\Sigma}^0_{\beta_{n}^{m}}(X)\), con \(\beta_{n}^{m}<\alpha\), tali che
\begin{equation*}
  A_{n} = \bigcap_{m \in \omega} A_{n}^{m}
\end{equation*}
Allora si ha che
\begin{equation*}
  \bigcap_{n \in \omega} A_{n} = \bigcap_{n,m \in \omega} A_{n}^{m}
\end{equation*}
che è ancora una intersezione numerabile, ed è quindi un elemento di \(\bm{\Pi}^0_\alpha(X)\).

\item Siano \(U, V \in \bm{\Pi}^0_\alpha(X)\). Allora \((X\setminus U), (X\setminus V) \in \bm{\Sigma}^0_\alpha(X)\)
\begin{align*}
  X\setminus (U\cup V) = (X\setminus U)\cap (X\setminus V)
\end{align*}
e siccome \(\bm{\Sigma}^0_\alpha(X)\) è chiuso per intersezioni finite, allora \(X\setminus (U\cup V)\) è un elemento di \(\bm{\Sigma}^0_\alpha(X)\), ovvero
\begin{equation*}
  U\cup V \in \bm{\Pi}^0_\alpha(X).
\end{equation*}
\end{itemize}
\item Classi ambigue
\label{sec:orgb5011d2}

\begin{itemize}
\item Sia \(U \in \bm{\Delta}^0_\alpha(X)\). Allora \(U \in \bm{\Sigma}^0_\alpha(X)\cap \bm{\Pi}^0_\alpha(X)\), ovvero esistono
\begin{equation*}
  A_{n} \in \bm{\Pi}^0_{\beta_{n}}(X),\qquad B_{m} \in \bm{\Sigma}^0_{\beta^{m}}(X)
\end{equation*}
con \(\beta_{n},\beta^{m} <\alpha\) tali che
\begin{equation*}
  U=\bigcup_{n \in \omega} A_{n},\qquad U = \bigcap_{m \in \omega} B_{m}.
\end{equation*}

Pertanto si ha che
\begin{align*}
  X\setminus U &= X \setminus \left(\bigcup_{n \in \omega} A_{n}\right) = \bigcap_{n \in\omega} (X\setminus A_{n})\\
  X\setminus U &= X\setminus \left(\bigcap_{m \in \omega} B_{m}\right) = \bigcup_{m \in \omega} (X\setminus B_{m})
\end{align*}

Se \(A_{n} \in \bm{\Pi}^0_{\beta_{n}}(X)\) allora \(X\setminus A_{n} \in \bm{\Sigma}^0_{\beta_{n}}(X)\), e pertanto \(X\setminus U \in \bm{\Pi}^0_\alpha(X)\).

Se \(B_{m} \in \bm{\Sigma}^0_{\beta^{m}}(X)\) allora \(X\setminus B_{m} \in \bm{\Pi}^0_{\beta^{m}}(X)\), e pertanto \(X\setminus U \in \bm{\Sigma}^0_\alpha(X)\).

Dunque \(X\setminus U \in \bm{\Sigma}^0_\alpha(X)\cap \bm{\Pi}^0_\alpha(X) = \bm{\Delta}^0_\alpha(X)\).

\item Siccome sia \(\bm{\Pi}^0_\alpha(X)\) che \(\bm{\Sigma}^0_\alpha(X)\) sono chiusi per unioni e intersezioni finite, allora
\[
  \bm{\Pi}^0_\alpha(X)\cap \bm{\Sigma}^0_\alpha(X) = \bm{\Delta}^0_\alpha(X)
  \]
è chiuso per unioni e intersezioni finite.\qed
\end{itemize}
\end{enumerate}
\end{document}
