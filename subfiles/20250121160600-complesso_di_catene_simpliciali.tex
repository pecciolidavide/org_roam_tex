% Intended LaTeX compiler: pdflatex
\documentclass[../main]{subfiles}


\begin{document}

Sia \(R\) un \href{20241219112842-pid.org}{PID}. Sia \(K\) un \href{20250121124147-complesso_simpliciale.org}{complesso simpliciale} \href{20250121131005-complesso_simpliciale_totalmente_ordinato.org}{totalmente ordinato}, e siano, per ogni \(q\):
\begin{itemize}
\item \(C_{q}(K)\) il \href{20250121124410-modulo_delle_catene_simpliciali.org}{modulo delle catene simpliciali};
\item \(\partial_{q}: C_{q}(K)\to C_{q-1}(K)\) la \href{20250121132856-mappe_di_bordo_tra_moduli_di_catene_simpliciali.org}{mappa di bordo}.
\end{itemize}
Allora
\begin{equation*}
\mathcal{C}_{\bullet}(K)\coloneqq\set{(C_{q}(K), \partial_{q})}_{q}
\end{equation*}
è un \href{20250120163114-complesso_di_catene.org}{complesso di catene}, detto \textbf{complesso di catene simpliciali}.
\end{document}
