% Intended LaTeX compiler: pdflatex
\documentclass[../main]{subfiles}

\usepackage[hyperref]{biblatex}
\date{}
\title{}
\begin{document}

\section{Complesso di catene contraibile}
\label{sec:org090b843}
Sia \(R\) un \href{20241205141119-anello.org}{anello} commutativo con unità e sia \(\mathcal{C}_{\bullet}\) un \href{20250120163114-complesso_di_catene.org}{complesso di catene},
\begin{equation*}
\mathcal{C}_{\bullet}=\set{(C_{n},\partial_{n})}_{n \in \Z}
\end{equation*}

Siano \(\1_{\mathcal{C}_{\bullet}}, \mathds{O}_{\mathcal{C}_{\bullet}} : \mathcal{C}_{\bullet}\to \mathcal{C}_{\bullet}\) due \href{20250120163759-categoria_complessi_di_catene.org}{morfismi}:
\begin{align*}
\1_{\mathcal{C}_{\bullet}} &= \set{\Id_{n}:C_{n}\to C_{n}: c\mapsto c}_{n \in \Z}\\
\mathds{O}_{\mathcal{C}_{\bullet}} &= \set{\mathds{O}_{n}: C_{n}\to C_{n}: c\mapsto 0}_{n \in \Z}
\end{align*}
\begin{definizione}
\(\mathcal{C}_{\bullet}\) si dice contraibile se i \href{20250120163759-categoria_complessi_di_catene.org}{morfismi} \(\1_{\mathcal{C}_{\bullet}}\) e \(\mathds{O}_{\mathcal{C}_{\bullet}}\) sono \href{20250121094935-omotopia_tra_morfismi_di_complessi_di_catene.org}{omotopi},
\begin{equation*}
\1_{\mathcal{C}_{\bullet}}\sim \mathds{O}_{\mathcal{C}_{\bullet}}
\end{equation*}
\end{definizione}
\end{document}
