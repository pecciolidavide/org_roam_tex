% Intended LaTeX compiler: pdflatex
\documentclass[../main]{subfiles}


\begin{document}

%% Leggibilità
\def\D{\mathcal{D}}
\def\DD{\bm{\D}}
\def\U{\mathcal{U}}
\def\eq{{\rm eq}}
\def\Ueq{\U^\eq}
\def\L{\mathcal{L}}
\def\orbita{\mathcal{O}}
\def\Aut{\operatorname{Aut}}
\def\tc{\mid}
\def\tp{\operatorname{tp}}
\def\<{\langle}
\def\>{\rangle}
\def\b{\bm{b}}

\def\restricted#1{\,\mathord{\upharpoonright}{{\scriptstyle #1}}}
\def\equivalentover#1{\mathrel{\equiv_{ #1 }}}

%% NON FORKING
\def\nonforkSymbol{%
\mathbin{\raise1.8ex%
\rlap{\kern0.6ex\rule{0.6ex}{0.1ex}}%
\rlap{\kern1.1ex\rule{0.1ex}{1.9ex}}\raise-0.3ex\hbox{$\smile$}}}
\def\defaultnonforkmodel{M}
\def\nonfork{\nonforkSymbol}
\renewcommand{\nonfork}[1][\defaultnonforkmodel]{%
\mathrel{\nonforkSymbol_{#1}}}
\section{Estensione semigruppo a Modello Mostro}
\label{sec:org4ed5446}
\def\defaultnonforkmodel{S}

Si utilizza la \href{20250612143636-notazione_teoria_dei_modelli.org}{notazione della Teoria dei Modelli}.
\begin{itemize}
\item Sia \(S\) un \href{20251104163738-semigruppo.org}{semigruppo}, \(\L\) un \href{20250130162057-linguaggio_del_prim_ordine.org}{linguaggio} che estende il linguaggio dei gruppi \(\set{\cdot}\), con \(\cdot\) funzione binaria.
\item Sia \(\U \succeq S\) un \(\L\)-\href{20250617102733-modello_mostro.org}{modello mostro}, ovvero \(\U\) un \href{20250617095548-modello_lambda_saturo.org}{modello saturo} di \href{20241213101756-cardinalita.org}{cardinalità} \href{20250211123155-cardinale_limite_forte.org}{inaccessibile} \(\kappa>\card{\L}+ \omega\).
\item Pertanto anche \(\U\) è un semigruppo.
\end{itemize}
\subsection{Operazione di semigruppo}
\label{sec:orgcd26ec3}

\begin{definizione}
Per due insiemi \(A,B \subseteq \U\) si definisce
\begin{equation*}
A*B \coloneqq \set{a\cdot b \tc a \in A,\, b \in B,\, a \nonfork b}.
\end{equation*}
dove \(\nonfork\) è la \href{20251029154447-tuple_indipendenti_rispetto_ad_un_modello_monster_model.org}{relazione eredi-coeredi}.
\end{definizione}

\uline{Notazione}: per \(a,b \in \U\), ricordando la definizione di \href{20251020151126-insieme_degli_automorfismi_di_una_struttura_del_prim_ordine.org}{orbita} \(\orbita(a/S)\):
\begin{equation*}
\orbita(a/S) = \set{fa \tc f \in \Aut(\U/S)}
\end{equation*}
si scrive
\begin{align*}
A*b &\coloneqq A*\orbita(b/S)\\
a*B &\coloneqq \orbita(a/S)*B\\
a*b &\coloneqq \orbita(a/S)* \orbita(b/S)
\end{align*}

\begin{oss}
Se \(A' \subseteq A\), \(B' \subseteq B\), allora
\begin{equation*}
A' * B' \subseteq A * B.
\end{equation*}
Inoltre, se \(B\) è invariante su \(S\) allora per ogni \(b \in B\) si ha
\begin{equation*}
\orbita(b/S) \subseteq B
\end{equation*}
e quindi
\begin{equation*}
A * b \subseteq A * B.
\end{equation*}
\end{oss}

\begin{prop}
Siano \(A,B \subseteq \U\) piccoli, \(b \in \U\).
\begin{itemize}
\item Se \(A,B\) sono sono \href{20251020150308-insieme_invariante_su_un_insieme_in_un_modello_mostro.org}{invarianti su \(S\)}, allora \(A*B\) è invariante su \(S\).
\item Se \(A\) è \href{20250212164424-tipo_teoria_dei_modelli.org}{tipo-definibile (su \(S\))} allora \(A*b\) è tipo-definibile.
\end{itemize}
\label{prop:prodottiprop}
\end{prop}
\begin{lem}
Se \(A,B,C\) sono invarianti su \(S\) allora
\begin{equation*}
A*(B*C) \subseteq (A*B)*C.
\end{equation*}
\end{lem}
\begin{proof}
Sia \(a\cdot b\cdot c \in A*(B*C)\). Allora
\begin{equation*}
b \nonfork c, \qquad
a \nonfork b\cdot c
\end{equation*}

Per il \href{20251029154447-tuple_indipendenti_rispetto_ad_un_modello_monster_model.org}{Lemma(4)} trovo \(a'\) tale che\footnote{Con ``\(\equivalentover{S,\, b \cdot c}\)'' si intende ``\href{20250212164424-tipo_teoria_dei_modelli.org}{Tuple elementarmente equivalenti su un insieme di parametri}''}
\begin{equation}
a \equivalentover{S,b\cdot c} a',\quad %
a' \nonfork b\cdot c,\, b,c %
\label{eq:aprimo}
\end{equation}
\begin{itemize}
\item Siccome \(A\) è \href{20251020150308-insieme_invariante_su_un_insieme_in_un_modello_mostro.org}{\(S\)-invariante}, allora \(a' \in A\).

Infatti se \(a' \equivalentover{S} a\) allora
\begin{equation*}
  a' \in \orbita(a'/S) = \orbita(a/S)
\end{equation*}
ed in particolare \(a' = fa\) per qualche \(f \in \Aut(\U/S)\).

\item La \eqref{eq:aprimo} implica che \(a'\nonfork b,c\); unito al fatto che \(b \nonfork c\) per il \href{20251029154447-tuple_indipendenti_rispetto_ad_un_modello_monster_model.org}{Lemma(3)} si ottiene che \(a',b \nonfork c\). Questo implica che\footnote{Infatti sia \(\varphi(x;y) \in\L(S)\) tale che
\(\U\vDash \varphi(a'\cdot b, c)\).
Allora si ha che
\begin{equation*}
\psi(w;z;y) \coloneqq \varphi(w\cdot z, y) \in \L(S),\qquad %
\U\vDash \psi(a',b,c).
\end{equation*}
Pertanto esistono \(a'',b' \in S\) tali che \(\U \vDash \psi(a'',b',c)\), ovvero
\begin{equation*}
\U \vDash \varphi(a'' \cdot b', c).
\end{equation*}}
\begin{equation}
  a'\cdot b \nonfork c.%
\label{eq:ipotesi1}
\end{equation}

\item Inoltre la \eqref{eq:aprimo} implica anche che
\begin{equation}
  a' \nonfork b %
\label{eq:ipotesi2}
\end{equation}
\end{itemize}

Siccome \(a \equivalentover{S,b\cdot c} a'\) allora \(a\cdot b\cdot c \equivalentover{S} a'\cdot b \cdot c\).

Infatti, se \(\varphi(y) \in \L(S)\) è tale che
\(\vDash \varphi(a \cdot b\cdot c)\) allora, detta
\begin{equation*}
\psi(x) \coloneqq \varphi(x \cdot b \cdot c) \in \L(S,\, b\cdot c)
\end{equation*}
si ottiene che \(\vDash \psi(a)\) e, per ipotesi, \(\vDash \psi(a')\), ovvero
\begin{equation*}
\vDash \varphi(a' \cdot b \cdot c).
\end{equation*}
Per simmetria si ha la tesi, ovvero
\begin{equation}
a\cdot b\cdot c \equivalentover{S} a'\cdot b \cdot c. %
\label{eq:ipotesi3}
\end{equation}

Quindi, combinando la~\eqref{eq:ipotesi1} e la~\eqref{eq:ipotesi2} si ottiene che
\begin{equation*}
a' \cdot b \cdot c \in (A*B)*C.
\end{equation*}
Inoltre, per la~\eqref{eq:ipotesi3} si ha che
\begin{equation*}
\orbita(a' \cdot b \cdot c/S) = \orbita(a \cdot b \cdot c/S) \ni a \cdot b \cdot c
\end{equation*}
e pertanto
\begin{equation*}
a \cdot b \cdot c = g (a' \cdot b \cdot c)
\end{equation*}
per qualche \(g \in \Aut(\U/S)\). Siccome \((A*B)*C\) è \(S\)-invariante e \(a' \cdot b \cdot c \in (A*B)*C\) allora
\begin{equation*}
a \cdot b \cdot c \in (A*B)*C. %
\qedhere
\end{equation*}
\end{proof}

\begin{prop}
Sia \(\nonfork\) 1-\href{20251104154216-relazione_erede_coerede_stazionaria.org}{stazionaria}.
\begin{enumerate}
\item Se \(a \nonfork b\) e \(a' \equivalentover{S} a\) allora
\(a\cdot b \equivalentover{S} a' \cdot b\).
\item Se \(a \nonfork b\), \(a' \nonfork b'\),
\(a' \equivalentover{S} a\), \(b' \equivalentover{S} b\)
allora
\(a\cdot b \equivalentover{S} a'\cdot b'\).
\item Per ogni \(a,b \in \U\) tali che \(a \nonfork b\) si ha che
\begin{equation*}
 a*b = \orbita(a \cdot b / S).
\end{equation*}
\end{enumerate}
\label{prop:lemmastazionaria}
\end{prop}

\begin{proof}
Per definizione
\begin{equation*}
a \equivalentover{S} x \nonfork b
\end{equation*}
è un \href{20251029152405-tipo_completo.org}{tipo completo su \(S\)}.
\begin{enumerate}
\item Se \(a \nonfork b\) e \(a' \equivalentover{S} a\), allora
\(a \equivalentover{S,b} a'\), e pertanto \(a,b \equivalentover{S} a',b\). Segue la tesi.
\item Supponiamo per assurdo che
\(a \cdot b \not\equivalentover{S} a' \cdot b'\).

Sia \(f \in \Aut(\U/S)\) tale che \(fb' = b\) (poiché \(b'\equivalentover{S} b\) allora \(b' \in \orbita(b/S)\)).

Allora \(a\cdot b \not\equivalentover{S} f(a'\cdot b')\) in quanto
\begin{equation*}
 \orbita\big(f(a'\cdot b') / S\big) = \orbita(a'\cdot b / S) \neq \orbita(a\cdot b / S).
\end{equation*}
Ma siccome \(f\) è morfismo di semigruppi allora
\(f(a'\cdot b') = f(a') \cdot f(b') = a'' \cdot b\)
per qualche
\(a'' \in \orbita(a' / S) = \orbita(a / S)\), \(a'' \equivalentover{S} a\).

Quindi si ottiene che \(a \cdot b \not\equivalentover{S} a'' \cdot b\). Questo è assurdo in quanto contraddice il punto 1.
\item Si deve dimostrare che
\begin{equation*}
 \orbita(a \cdot b / S) = a*b %
 \coloneqq \orbita(a / S) *  \orbita(b / S)
\end{equation*}
dove, per definizione di \(*\):
\begin{equation*}
 \orbita(a / S) *  \orbita(b / S) %
 = \set{a' \cdot b' \mid
 	a' = fa,\,
 	b' = gb,\,
 	a' \nonfork b',\,
 	f,g \in \Aut(\U/S)}.
\end{equation*}
(\(\subseteq\)): Se \(d \in \orbita(a\cdot b / S)\) allora
\(d = f(a \cdot b)\) per \(d \in \Aut(\U/S)\). Posto \(a'=fa\), \(b'=fb\) si ha \(d \in a*b\).

(\(\supseteq\)): Sia \(a'\cdot b' \in a*b\). Allora
\begin{equation*}
 a' \equivalentover{S} a, \quad b' \equivalentover{S} b,\quad a' \nonfork b'
\end{equation*}
e pertanto, per il punto 2., si ha che \(a'\cdot b' \equivalentover{S} a \cdot b\), ovvero
\(a' \cdot b' \in \orbita(a\cdot b /S)\).\qedhere
\end{enumerate}
\end{proof}

\begin{thm}
Se \(\nonfork\) è 1-stazionaria, e \(A,B,C\) sono invarianti su \(S\) allora
\begin{equation*}
A*(B*C) = (A*B)*C.
\end{equation*}
\label{thm:assoperorbite}
\end{thm}
\begin{proof}
Considero
\begin{equation*}
A= \orbita(a/S); \quad %
B= \orbita(b/S); \quad %
C= \orbita(c/S).
\end{equation*}
Allora si ha che \(a \nonfork b\cdot c\) e \(b \nonfork c\).

\href{20250515141706-da_finire.org}{DA FINIRE}
\end{proof}
\subsection{Semigruppo di Ellis}
\label{sec:orgc5e435f}
\href{20250515141706-da_finire.org}{DA FINIRE}

Vedi anche \href{20250922171928-teoria_dei_modelli_corso.org}{Teoria dei modelli [CORSO]​}/Lezione 15
\subsection{Insiemi idempotenti e Teorema di Ellis}
\label{sec:org685b50b}

\begin{definizione}
Sia \(\emptyset \neq A \subseteq \U\). \(A\) è \uline{idempotente} se
\begin{equation*}
A*A \subseteq A.
\end{equation*}
\end{definizione}
\begin{oss}
Se \(A\) è idempotente e \(B \subseteq A\) allora \(A*B\) è idempotente:
\begin{equation*}
(A*B) * (A*B) \subseteq (A*A) * (A*A) \subseteq A * A \subseteq A.
\end{equation*}
\end{oss}
\def\defaultnonforkmodel{S}
\begin{thm}
(Ellis).
Si suppona \(\nonfork\) \href{20251104154216-relazione_erede_coerede_stazionaria.org}{1-stazionaria}.
Se \(A\) è \href{20250212164424-tipo_teoria_dei_modelli.org}{tipo-definibile} su \(S\) e idempotente, allora esiste \(b \in A\) tale che \(\orbita(b/S)\) è idempotente.
\end{thm}
\begin{proof}
Sia \(B \subseteq A\) un \href{20250203102516-massimo_e_minimo.org}{\(\subseteq\)-minimale} tra gli idempotenti tipo-definibili su \(S\)\footnote{Questo esiste per il \href{20250210104633-lemma_di_zorn.org}{lemma di Zorn}. Infatti, sia
\begin{equation*}
\set{\emptyset \neq B_{i} \subseteq A \mid i<\alpha}
\end{equation*}
una catena di idempotenti tipo-definibili, \(B_{i} \coloneqq p_{i}(\U)\) per qualche \href{20250203111003-ordinali.org}{ordinale} \(\alpha\).
Allora \(\bigcap_{i <\alpha} B_{i}\) è un idempotente tipo-definibile contenuto in ogni \(B_{i}\) e pertanto, per il lemma di Zorn, esiste un elemento \(\subseteq\)-minimale.
\begin{itemize}
\item \(\bigcap_{i<\alpha} B_{i}\) è tipo definibile, definito da
\begin{equation*}
  p(x) \coloneqq \bigcup_{i<\alpha} p_{i}(x)
\end{equation*}
tipo piccolo.
\item \(\bigcap_{i<\alpha} B_{i} \neq \emptyset\) poiché, per saturazione di \(\U\), \(p(\U) \neq \emptyset\).
\item Siano \(a_{1}\cdot a_{2} \in \big(\bigcap_{i<\alpha} B_{i}\big)*\big(\bigcap_{i<\alpha} B_{i}\big)\),
\(a_{1} \nonfork a_{2}\).
In particolare, per ogni \(i<\alpha\):
\begin{equation*}
a_{1} \in B_{i}, a_{2} \in B_{i} \IMPLICA a_{1}\cdot a_{2} \in B_{i}*B_{i} \subseteq B_{i}
\end{equation*}
e dunque \(a_{1}\cdot a_{2} \in B_{i}\) per ogni \(i<\alpha\).
\end{itemize}}.

Prendiamo \(b \in B\). Allora:
\begin{itemize}
\item \(B*b\) è tipo definibile su \(S\) per la Proposizione~\ref{prop:prodottiprop};
\item Siccome \(B\) è tipo-definibile su \(S\), \href{20251106123927-insieme_tipo_definibile_su_un_insieme_e_invariante.org}{allora} \(B\) è \href{20251020150308-insieme_invariante_su_un_insieme_in_un_modello_mostro.org}{invariante su \(S\)},  e pertanto
\begin{equation*}
  B*b \subseteq B* B \subseteq B.
\end{equation*}
\item Inoltre
\begin{equation*}
  (B*b) * (B*b) \subseteq B * (B*b) = (B*B)*b \subseteq B*b
\end{equation*}
e pertanto \(B*b\) idempotente.
(L'uguaglianza è in virtù del Teorema~\ref{thm:assoperorbite})
\end{itemize}

Per minimalità di \(B\) si ha che \(B*b=B\).

Sia ora
\begin{equation*}
B'\coloneqq
\set{a \in B \mid a \nonfork b,\, a \cdot b \equivalentover{S} b} \subseteq B.
\end{equation*}
Questo insieme è tipo-definibile su \(S,b\)\footnote{Se
\begin{align*}
p(x) &= x \nonfork b \subseteq \L(S,b)\\
q(y) &= \tp(b/S)
\end{align*}
allora \(B'\) è definito da \(p(x) \cup q(x \cdot b)\).}, ed è invariante su \(S\): sia \(a \in B'\) e \(f \in \Aut(\U/S)\):
\begin{itemize}
\item \(fa \in B\) in quanto \(B\) è \(S\)-invariante;
\item ovviamente \(fa \equivalentover{S} a\), e inoltre \(a \nonfork b\), quindi per la Proposizione~\ref{prop:lemmastazionaria}(1)
\begin{equation*}
  fa \cdot b \equivalentover{S} a \cdot b \equivalentover{S} b;
\end{equation*}
\item \(fa \nonfork b\) poiché \href{20251104154216-relazione_erede_coerede_stazionaria.org}{\(\nonfork\) è 1-stazionaria} e \(fa \equivalentover{S} a\) e \(a \nonfork b\).
\end{itemize}
quindi \(fa \in B'\), ovvero \(B'\) è \(S\)-invariante. \href{20251020150315-caratterizzazione_insiemi_invarianti_su_un_insieme_in_un_modello_mostro.org}{Pertanto} \(B'\) è tipo-definibile su \(S\).

Si ha che \(B' \neq \emptyset\):
\begin{itemize}
\item siccome \(B*b = B\), allora in particolare esiste \(f \in \Aut(\U/S)\) ed esiste \(a \in B\) tale che
\begin{equation*}
  a \cdot fb = b,\qquad a \nonfork fb
\end{equation*}
ovvero, per \(f^{-1} \in \Aut(\U/B)\) e il \href{20251029154447-tuple_indipendenti_rispetto_ad_un_modello_monster_model.org}{Lemma(1)}:
\begin{equation*}
  f^{-1}a \cdot b = f^{-1} b \equivalentover{S} b,\qquad f^{-1} a \nonfork b
\end{equation*}
\item inoltre, \(f^{-1}a \in B\), in quanto \(B\) è \(A\)-invariante;
\item quindi \(f^{-1}(a) \in B'\).
\end{itemize}

Inoltre \(B'\) è idempotente. Infatti, sia \(a_{1}\cdot a_{2} \in B'*B'\):
\begin{equation*}
a_{1} \nonfork a_{2}, \qquad%
a_{1} \cdot b \equivalentover{S} b \equivalentover{S} a_{2} \cdot b, \qquad %
a_{1} \nonfork b, \qquad %
a_{2} \nonfork b.
\end{equation*}
\begin{itemize}
\item Siccome \(a_{1} \nonfork a_{2}\) allora per il \href{20251029154447-tuple_indipendenti_rispetto_ad_un_modello_monster_model.org}{Lemma(4)} esiste
\(a_{1}' \equivalentover{S, a_{2}} a_{1}\) tale che
\begin{equation*}
  	a_{1}' \nonfork a_{2}, b.
\end{equation*}
Poiché \href{20251104154216-relazione_erede_coerede_stazionaria.org}{\(\nonfork\) è 1-stazionaria} ed in particolare \(a_{1}' \equivalentover{S} a_{1}\) allora
\begin{equation*}
  	a_{1} \nonfork a_{2},b.
\end{equation*}
Per ipotesi inoltre \(a_{2} \nonfork b\) e dunque, per il \href{20251029154447-tuple_indipendenti_rispetto_ad_un_modello_monster_model.org}{Lemma(3)}
\(a_{1},a_{2} \nonfork b\) e pertanto
\begin{equation*}
  	a_{1}\cdot a_{2} \nonfork b.
\end{equation*}

\item Si consideri pertanto la seguente uguaglianza, valida per il Teorema~\ref{thm:assoperorbite}
\begin{equation*}
\orbita(a_{1}/S) * \big(\orbita(a_{2}/S) * \orbita(b/S)\big) = \big(\orbita(a_{1}/S) * \orbita(a_{2}/S) \big)* \orbita(b/S)
\end{equation*}
Espandendo la parte a sinistra dell'uguale si ottiene:
\begin{align*}
\orbita(a_{1}/S) * \big(\orbita(a_{2}/S) * \orbita(b/S)\big) %
	&= \orbita(a_{1}/S) * \orbita(a_{2}\cdot b / S) %
		& &\text{poiché } a_{2} \nonfork b\\
	&= \orbita(a_{1}/S) * \orbita(b / S ) %
		& &\text{poiché } a_{2}\cdot b \equivalentover{S} b\\
	&= \orbita(a_{1} \cdot b / S) %
		& &\text{poiché } a_{2} \nonfork b\\
	&= \orbita(b/S) %
		& &\text{poiché } a_{1}\cdot b \equivalentover{S} b\\
\end{align*}
dove si è fortemente utilizzata la
Proposizione~\ref{prop:lemmastazionaria}(3).

Espandendo invece la parte a destra dell'uguale si ottiene:
\begin{align*}
\big(\orbita(a_{1}/S) * \orbita(a_{2}/S) \big)* \orbita(b/S) %
        &= \orbita(a_{1}\cdot a_{2} / S) * \orbita(b / S) %
                & &\text{poiché } a_{1} \nonfork a_{2}\\
        &= \orbita(a_{1} \cdot a_{2} \cdot b / S) %
                & &\text{poiché } a_{1} \cdot a_{2} \nonfork b.
\end{align*}

Pertanto si ha che
\(\orbita(b/S) = \orbita(a_{1}\cdot a_{2}\cdot b /S)\)
ovvero che
\begin{equation*}
a_{1} \cdot a_{2} \cdot b \equivalentover{S} b.
\end{equation*}

\item I due punti precedenti affermano che \(a_{1}\cdot a_{2} \in B'\), ovvero che
\begin{equation*}
  B' * B' \subseteq B'.
\end{equation*}
\end{itemize}



Quindi, per minimalità di \(B\) si ha \(B'=B\), ovvero che \(b \in B'\):
\begin{equation*}
b \nonfork b,\qquad b\cdot b \equivalentover{S} b
\end{equation*}
e la Proposizione~\ref{prop:lemmastazionaria}(3) si ha che
\begin{equation*}
\orbita(b/S)*\orbita(b/S) = \orbita(b\cdot b/S) = \orbita(b/S)
\end{equation*}
Pertanto \(\orbita(b/S)\) è idempotente.
\end{proof}
\subsection{Gruppo di Ellis}
\label{sec:org7f34b07}
\href{20250515141706-da_finire.org}{DA FINIRE}

Vedi anche \href{20250922171928-teoria_dei_modelli_corso.org}{Teoria dei modelli [CORSO]​}/Lezione 15
\end{document}
