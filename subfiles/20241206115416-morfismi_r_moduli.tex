% Intended LaTeX compiler: pdflatex
\documentclass[../main]{subfiles}


\begin{document}

\section{Morfismo tra R-Moduli}
\label{sec:orgf784bf6}
\begin{definizione}
Siano \(M,N\) due \href{20241205141053-r_moduli.org}{\(R\)-Moduli}. Un \textbf{morfismo} da \(M\) a \(N\) è un \href{20241206115531-morfismo_di_gruppi.org}{omomorfismo di gruppi} \(f:M \longrightarrow N\) tale che
\begin{equation*}
\forall\,a \in R,\ m \in M:\quad f(a \cdot m) = a \cdot f(m).
\end{equation*}
\end{definizione}
\section{Isomorfismo tra R-Moduli}
\label{sec:org5d7bff5}
\begin{definizione}
Siano \(M,N\) due \href{20241205141053-r_moduli.org}{\(R\)-Moduli}. Se esiste \(f: M \to N\) morfismo tra moduli biettivo e tale che \(f^{-1}:N\to M\) è morfismo tra moduli, allora diciamo:
\begin{itemize}
\item \(M \cong N\) sono isomorfi;
\item \(f\) è un isomorfismo.
\end{itemize}
\end{definizione}
\begin{prop}
Se \(f:M\to N\) è un morfismo tra moduli biiettivo, allora \(f\) è un isomorfismo.
\end{prop}
\end{document}
