% Intended LaTeX compiler: pdflatex
\documentclass[../main]{subfiles}

\usepackage[hyperref]{biblatex}
\date{}
\title{}
\begin{document}

\section{Modello lambda saturo sse lambda ricco}
\label{sec:orgf4629f2}
Si utilizza la \href{20250612143636-notazione_teoria_dei_modelli.org}{Notazione della TEORIA DEI MODELLI}
\subsection{Teorema}
\label{sec:org177ed9c}

Sia \(\lambda\) un \href{20250203161341-cardinali.org}{cardinale}, e sia \(\mathcal{L}\) un \href{20250130162057-linguaggio_del_prim_ordine.org}{linguaggio del prim'ordine} tale che \(\card{\mathcal{L}}<\lambda\).

Sia \(\mathcal{M}\) la \href{20250213142026-categorie_di_modelli_e_morfismi_parziali.org}{categoria} che consiste delle \(\mathcal{L}\)-\href{20250131103035-struttura_del_prim_ordine.org}{strutture} e delle \href{20250214120959-mappe_tra_strutture_del_prim_ordine.org}{mappe elementari} tra di loro.

Allora per ogni \(N \in \mathcal{M}_{\text{ob}}\) sono fatti equivalenti:
\begin{enumerate}
\item \(N\) è un modello \(\lambda\)-saturo;
\item \(N\) è un modello \(\lambda\)-ricco;
\end{enumerate}
\end{document}
