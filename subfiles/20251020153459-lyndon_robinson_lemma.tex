% Intended LaTeX compiler: pdflatex
\documentclass[../main]{subfiles}

\usepackage[hyperref]{biblatex}
\date{}
\title{}
\begin{document}

\section{Lyndon-Robinson Lemma}
\label{sec:org3ac5c92}
Si utilizza la \href{20250612143636-notazione_teoria_dei_modelli.org}{Notazione della TEORIA DEI MODELLI}

Sia \(\mathcal{L}\) un \href{20250130162057-linguaggio_del_prim_ordine.org}{linguaggio del prim'ordine}, \(T\) una \(\mathcal{L}\)-\href{20250130114950-teoria_del_prim_ordine.org}{teoria} senza \href{20250131122945-modello_di_un_insieme_di_formule.org}{modelli} finiti.

Sia \(\Delta\) un insieme di formule chiuso per sostituzione di variaibli e che contenga la formula ``\(x=y\)''.

Se \(C \subseteq \set{\forall , \exists , \land,\lor,\lnot}\) è un insieme di connettivi, \(C\Delta\) denota la chiusura di \(\Delta\) per i connettivi in \(C\). \(\Delta^{\pm} \coloneqq \set{\lnot}\Delta\).
\begin{thm}
(Lyndon-Robinson Lemma). Sia \(\varphi(x) \in \mathcal{L}\). Sono fatti equivalenti:
\begin{enumerate}
\item \(\varphi(x)\) è equivalente su \(T\) ad una \(\set{\land ,\lor }\Delta\)-formula;
\item \(\varphi(x)\) è preservata da \(\Delta\)-morfismi.
\end{enumerate}
\end{thm}
\end{document}
