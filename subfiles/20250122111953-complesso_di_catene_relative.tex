% Intended LaTeX compiler: pdflatex
\documentclass[../main]{subfiles}


\begin{document}

Sia \(R\) un \href{20241219112842-pid.org}{PID} e sia \(K\) un \href{20250121124147-complesso_simpliciale.org}{complesso simpliciale} \href{20250121131005-complesso_simpliciale_totalmente_ordinato.org}{totalmente ordinato.}
Si consideri il \href{20250121160600-complesso_di_catene_simpliciali.org}{complesso di catene simpliciali}:
\begin{equation*}
\mathcal{C}_{\bullet}(K) = \set{\left(C_{q}(K), \partial_{q}\right)}_{q}
\end{equation*}
\section{Modulo del complesso simpliciale relativo}
\label{sec:org203ed03}
Sia \(L \subseteq K\) un \href{20250121124213-sottocomplesso_simpliciale.org}{sottocomplesso}. Allora per ogni \(q\), gli \href{20250121124310-scheletro_di_un_complesso_simpliciale.org}{scheletri} \(L^{q} \subseteq K^{q}\) e, pertanto:\footnote{Vedi:
\begin{itemize}
\item \href{20241205141053-r_moduli.org}{Modulo}
\item \href{20241213095808-somma_diretta.org}{Somma Diretta}
\end{itemize}}
\begin{equation*}
R^{L^{q}} \subseteq R^{K^{q}}
\end{equation*}

il \href{20241205141053-r_moduli.org}{modulo} \href{20241213094625-modulo_libero.org}{libero} \href{20241212141101-generatori_modulo.org}{generato} da \(L^{q}\) è \href{20241206142802-sottomoduli.org}{sottomodulo} di quello generato da \(K^{q}\):
\begin{equation*}
C_{q}(L) = R^{L^{q}}\mathrel{\subseteq_{R}} R^{K^{q}} = C_{q}(K)
\end{equation*}

\href{20250122122650-quoziente_di_somma_diretta_di_moduli.org}{Quindi} il modulo \href{20241206142802-sottomoduli.org}{quoziente} \(R^{K^{q}}/R^{L^{q}}\) è \href{20241213094625-modulo_libero.org}{libero}, ed è \href{20241206115416-morfismi_r_moduli.org}{isomorfo} a \(R^{K^{q}\setminus L^{q}}\).

\begin{definizione}
Si definisce il \textbf{modulo del complesso simpliciale relativo \((K,L)\)}:
\begin{equation*}
C_{q}(K,L) \coloneqq \frac{C_{q} (K)}{C_{q}(L)}
\end{equation*}
\end{definizione}
\section{Mappa di bordo tra moduli di complessi simpliciali relativi}
\label{sec:orgc9da138}
Si definisce, per ogni \(q\), la mappa di bordo
\begin{align*}
\overline{\partial}_{q}: C_{q}(K,L) &\longrightarrow C_{q-1}(K,L)\\
c_{q} + C_{q}(L) &\longmapsto \partial_{q} c_{q} + C_{q-1}(L)
\end{align*}
dove \(\partial_{q}\) è la \href{20250121132856-mappe_di_bordo_tra_moduli_di_catene_simpliciali.org}{mappa di bordo per \(C_{q}(K)\)}.

\begin{prop}
Per ogni \(q\),
\begin{enumerate}
\item \(\overline{\partial}_{q}\) è ben definita;
\item \(\overline{\partial}_{q+1}\circ\overline{\partial}_{q} \equiv 0\).
\end{enumerate}
\end{prop}

\begin{proof}
\begin{enumerate}
\item È sufficiente mostrare che se, per \(\ell_{q} \in C_{q}(L)\):
\begin{equation*}
 c_{q} = c_{q}' + \ell_{q} \IFF [c_{q}] = [c_{q}']
\end{equation*}
allora \([\partial_{q}c_{q}] = [\partial_{q}c_{q}']\). In particolare, si mostra che \(\partial_{q} \ell_{q} \in C_{q-1}(L)\) per ogni \(\ell_{q} \in C_{q}(L)\).

È sufficiente farlo per un elemento \(\sigma=[p_{0},\dots,p_{q}] \in L^{q}\) \href{20241213094625-modulo_libero.org}{base} di \(C_{q}(L)\):
\begin{equation*}
 \partial_{q}[p_{0},\dots,p_{q}] = \sum (-1)^{j} \parentesi{in L^{q-1}}{[p_{0},\dots,\hat{p_{j}},\dots,p_{q}]}
\end{equation*}
e quindi \(\partial_{q}\sigma \in C^{q-1}(L)\).

\item Sia \(c+C_{q}(L) \in C_{q}(K,L)\). Allora
\begin{align*}
\overline{\partial}_{q-1}\circ \overline{\partial}_{q} \left(c+ C_{q}(K)\right) &= \overline{\partial}_{q-1}\left(\partial_{q}c + C_{q-1}(L)\right)\\
&= \partial_{q-1}\partial_{q} c + C_{q-2}(L) = 0
\end{align*}
dove l'ultima uguaglianza si ha perché \href{20250121132856-mappe_di_bordo_tra_moduli_di_catene_simpliciali.org}{\(\partial_{q-1}\partial_{q} = 0\) per le mappe di bordo}.
\qedhere
\end{enumerate}
\end{proof}
\section{Complesso di catene relative}
\label{sec:org68957ca}
Mantenendo le notazioni:
\begin{definizione}
Il \textbf{complesso di catene relative} è il \href{20250120163114-complesso_di_catene.org}{complesso di catene}:
\begin{equation*}
\mathcal{C}_{\bullet}(K,L) \coloneqq \set{\left(C_{q}(K,L),\overline{\partial}_{q}\right)}_{q}
\end{equation*}
\end{definizione}
\end{document}
