% Intended LaTeX compiler: pdflatex
\documentclass[../main]{subfiles}

\usepackage[hyperref]{biblatex}
\date{}
\title{}
\begin{document}

\section{Teorema della funzione implicita}
\label{sec:org7edbea9}
\subsection{Teorema della funzione implicita}
\label{sec:org6591623}
Sia \(\Omega \subseteq \R^{n+m}\) \href{20250103145124-topologia.org}{aperto} e \(F:\Omega \longrightarrow \R^{m}\) una \href{20250113125602-classe_c_di_una_funzione.org}{funzione di classe \(C^{\infty}\)}. Sia \(a \in F(\Omega)\) un \href{20250113130125-valore_regolare_di_una_funzione.org}{valore regolare} per \(F\) (ovvero \(\restriction{F_{\star}}{p}\) è \href{20241213105600-funzione_suriettiva.org}{suriettivo} per ogni \(p \in F^{-1}(a)\)).

Allora \(M_{a}\coloneqq F^{-1}(a)\) ha una naturale struttura di \href{20250113115909-struttura_differenziabile.org}{varietà differenziabile.}
\subsubsection{{\bfseries\sffamily TODO} Dimostrazione\hfill{}\textsc{matematica\_lm:geo\_diff}}
\label{sec:org7a2459c}
In questa dimostrazione si assumerà per vero il \href{20250113125429-teorema_dell_inversa_locale.org}{Teorema dell'inversa locale}.
\end{document}
