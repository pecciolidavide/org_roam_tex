% Intended LaTeX compiler: pdflatex
\documentclass[../main]{subfiles}


\begin{document}

\section{Definizione}
\label{sec:org4fdae6d}
Un gruppo abeliano è un \href{20250130104331-insieme_mk.org}{insieme} \(G\) dotato di una \href{20250206171120-operazione_su_una_classe_mk.org}{operazione} \(+:G\times G\to G\) tale che
\begin{itemize}
\item per ogni \(a,b \in G\): \(a*b=b*a\);
\item dati \(a,b,c \in G\): \((a*b)*c = a*(b*c)\);
\item esiste \(e \in G\) tale che per ogni \(a \in G\): \(a*e=e*a=a\);
\item per ogni \(a \in G\) esiste \(a' \in G\) tale che \(a*a'=a'*a=e\).
\end{itemize}

Si denota con \(\langle G, *\rangle\) oppure con \(\langle G,*,e\rangle\).
\subsection{Osservazione}
\label{sec:orgf108576}
Un gruppo abeliano è un \href{20241205141146-gruppo_abeliano.org}{gruppo} in cui l'operazione è commutativa.
\end{document}
