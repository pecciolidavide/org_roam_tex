% Intended LaTeX compiler: pdflatex
\documentclass[../main]{subfiles}

\usepackage[hyperref]{biblatex}
\date{}
\title{}
\begin{document}

\section{Fascio come funtore}
\label{sec:orgbc8e91b}
\subsection{Prefascio come funtore}
\label{sec:org151e3b5}
\subsubsection{Prefascio di gruppi come funtore}
\label{sec:org536c536}

\begin{prop}
Dare un \uline{\href{20250324165349-prefascio.org}{prefascio di gruppi}} \(\mathcal{F}\) sullo spazio topologico \(X\) è equivalente a dare un \href{20241204223502-funtore_controvariante.org}{\uline{funtore controvariante}} \(\mathcal{G}\) tra la \href{20250324170250-categoria_degli_aperti_di_uno_spazio_topologico.org}{categoria \(\cat{A}_{X}\) degli aperti di \(X\)} è la \href{20241205115631-categoria_grp.org}{categoria \(\cat{Grp}\) dei gruppi} tale che \(\mathcal{G}(\emptyset) = \set{0}\).
\end{prop}
\begin{proof}
Si deve dimostrare che ogni prefascio sia un funtore controvariante, ed il viceversa.

\begin{itemize}
\item \uline{Ogni prefascio definisce un funtore}.

Sia \(\mathcal{G}:\cat{A}_{X}\to \cat{Grp}\) un funtore definito come segue:
\begin{itemize}
\item per ogni \(U \in \cat{A}_{X}\), \(\mathcal{G}(U) \coloneqq\mathcal{F}(U) \in \cat{Grp}\).
\item per ogni \(U, V \in \cat{A}_{X}\) tali che \(U \subseteq V\) con \(i:U\hookrightarrow V\) l'inclusione, si ponga
\begin{equation*}
     \mathcal{G}(i) \coloneqq \rho_{U}^{V}: \mathcal{F}(V)\to \mathcal{F}(U)
\end{equation*}
\end{itemize}

Allora
\begin{enumerate}
\item \(\mathcal{G}(\emptyset) = \mathcal{F}(\emptyset) = \set{0}\);
\item se \(U \subseteq V \subseteq W\) aperti, \(i:U\hookrightarrow V\), \(j:V\hookrightarrow W\) allora
\begin{equation*}
     \mathcal{G}(j\circ i) = \rho_{U}^{W} = \rho_{U}^{V}\circ \rho_{V}^{W} = \mathcal{G}(i)\circ \mathcal{G}(j)
\end{equation*}
\item per ogni \(U\), si considerì l'identità \(\Id_{U}:U\to U\); allora
\begin{equation*}
     \mathcal{G}(\Id_{U}) = \rho_{U}^{U} = \id_{\mathcal{F}(U)} = \id_{\mathcal{G}(U)}.
\end{equation*}
\end{enumerate}

\item \uline{Ogni funtore definisce un prefascio}.

Sia \(\mathcal{G} : \cat{A}_{X}\to \cat{Grp}\) un funtore controvariante tale che \(\mathcal{G}(\emptyset) = \set{0}\). Si definisce un prefascio \(\mathcal{F}\) di gruppi su \(X\) come segue:
\begin{enumerate}
\item per ogni \(U \subseteq X\) aperto, si pone \(\mathcal{F}(U) \coloneqq \mathcal{G}(U)\);
\item per ogni \(U \subseteq V \subseteq X\) aperti, detta \(i:U\hookrightarrow V\) l'inclusione, si pone
\begin{equation*}
     \rho_{U}^{V}: \mathcal{F}(V)\to \mathcal{F}(U)
\end{equation*}
come \(\rho_{U}^{V} \coloneqq \mathcal{G}(i)\).
\end{enumerate}

Allora
\begin{itemize}
\item \(\mathcal{F}(\emptyset) = \mathcal{G}(\emptyset) = \set{0}\);
\item per ogni \(U \subseteq X\) aperto: \(\rho_{U}^{U} = \mathcal{G}(\id_{U}) = \id_{\mathcal{G}(U)} = \id_{\mathcal{F}(U)}\);
\item per ogni \(W \subseteq V \subseteq U \subseteq X\) aperti:
\begin{equation*}
\begin{tikzcd}[cramped]
      W & V & {U}
      \arrow["i", hook, from=1-1, to=1-2]
      \arrow["j", hook, from=1-2, to=1-3]
\end{tikzcd}
\end{equation*}
si ha:
\begin{equation*}
      \rho_{W}^{U} = \mathcal{G}(j\circ i) = \mathcal{G}(i)\circ \mathcal{G}(j) = \rho_{W}^{V} \circ \rho_{V}^{U}.
      \qedhere
\end{equation*}
\end{itemize}
\end{itemize}
\end{proof}
\end{document}
