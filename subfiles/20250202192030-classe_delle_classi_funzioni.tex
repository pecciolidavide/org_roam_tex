% Intended LaTeX compiler: pdflatex
\documentclass[../main]{subfiles}


\begin{document}

Se \(A,B\) sono due \href{20250130104331-insieme_mk.org}{insiemi}, si indica con
\begin{equation*}
A^{B}, \null^{B}\!A \coloneqq \set{f:A\to B\text{ funzioni}}
\end{equation*}
\section{Generalizzazione in MK}
\label{sec:org09c31b0}

Contesto: \href{20250130104245-morse_kelly_set_theory.org}{Morse Kelly Set Theory}

Se invece \(A,B\) sono due \href{20250130104320-classe_mk.org}{classi} qualsiasi,
\begin{equation*}
A^{B}=\null^{B}\!A \coloneqq \set{F:A\to B\text{ classi-funzioni}}
\end{equation*}
vedi: \href{20250202170607-classe_relazione_binaria.org}{Classe-Funzione}
\subsection{Proprietà}
\label{sec:orga025c5b}

Se \(A\) e \(B\) sono \href{20250130104331-insieme_mk.org}{insiemi}, allora \(A^{B}\) è un insieme.
\end{document}
