% Intended LaTeX compiler: pdflatex
\documentclass[../main]{subfiles}


\begin{document}

\section{Nullstellensatz}
\label{sec:org041b35e}
Sia \(\K\) un \href{20241231112713-campo_algebricamente_chiuso.org}{campo algebricamente chiuso}, e sia \(J \subseteq \K[x_{1},\dots,x_{n}]\) un \href{20241219112955-ideale.org}{ideale}. Sia  \(V(J) \subseteq \A^{n}\) (vedi \href{20241231114009-spazio_affine.org}{Spazio Affine} e \href{20241231112823-radici_polinomiali.org}{Luogo di zeri}). Allora
\begin{equation*}
I\left(V(J)\right) = \sqrt{J}
\end{equation*}
dove \(I\left(V(J)\right)\) è \href{20250103144124-ideale_di_un_sottoinsieme.org}{l'ideale} del \href{20241231112823-radici_polinomiali.org}{luogo di zeri} di \(J\) (vedi \href{20241231114256-varieta_algebrica_affine.org}{Varietà Algebrica Affine}), e \(\sqrt{J}\) è il \href{20250110125225-radicale_di_un_ideale.org}{radicale} di \(J\).
\end{document}
