% Intended LaTeX compiler: pdflatex
\documentclass[../main]{subfiles}


\begin{document}

\begin{definizione}
Un complesso simpliciale \(K\) in \(\R^{n+1}\) è un insieme
\begin{equation*}
K = \set{\sigma_{i}}_{i \in I}
\end{equation*}
dove \(\sigma_{i}\) sono \href{20250121121923-simplessi.org}{simplessi} di \(\R^{n+1}\) tali che
\begin{enumerate}
\item Se \(\sigma \in K\) e \(\tau \preceq \sigma\) è una \href{20250121122621-faccia_di_un_simplesso.org}{faccia} allora \(\tau \in K\);
\item Se \(\tau_{1},\tau_{2} \in K\) e \(\sigma=\tau_{1}\cap \tau_{2}\neq \emptyset\), allora \(\sigma \preceq\tau_{1}\) e \(\sigma\preceq \tau_{2}\)\footnote{Ovvero tutti i simplessi di \(K\) possono intersecarsi solo sulle loro facce.}
\end{enumerate}
\end{definizione}
\end{document}
