% Intended LaTeX compiler: pdflatex
\documentclass[../main]{subfiles}


\begin{document}

\section{Materiale}
\label{sec:org4365c08}

\begin{itemize}
\item \href{20250301192908-spazio_topologico_separabile.org}{Spazio topologico separabile}
\item \href{20250301193254-spazio_topologico_primo_numerabile.org}{Spazio topologico primo numerabile}
\item \href{20250111142303-spazio_topologico_a_base_numerabile.org}{Spazio topologico secondo numerabile}
\item \href{20250301193401-spazio_topologico_metrizzabile.org}{Spazio topologico metrizzabile}
\item \href{20250301194013-spazio_polacco.org}{Spazio Polacco}
\item \href{20250303114357-teorema_di_completamento_di_spazi_metrici.org}{Teorema di completamento di spazi metrici}
\item \href{20250303115241-proprieta_di_chiusura_degli_spazi_polacchi.org}{Proprietà di chiusura degli Spazi Polacchi}
\item \href{20250310121944-esempi_di_spazi_polacchi.org}{Esempi di spazi polacchi} *
\item \href{20250303115241-proprieta_di_chiusura_degli_spazi_polacchi.org}{Sottoinsiemi aperti di spazi polacchi sono polacchi}
\item \href{20250304152026-sottoinsiemi_gdelta_e_fsigma.org}{Insiemi Gdelta e Fsigma}
\item \href{20250306135124-intersezione_di_gdelta_densi_e_densa_in_un_polacco.org}{Intersezione di Gdelta densi è densa in un polacco}
\item \href{20250306142322-chiuso_in_uno_spazio_metrizzabile_e_gdelta.org}{Chiuso in uno spazio metrizzabile è Gdelta}
\item \href{20250306134632-caratterizzazione_dei_sottoinsiemi_polacchi_di_uno_spazio_polacco.org}{Caratterizzazione dei sottoinsiemi polacchi di uno spazio polacco}
\item \href{20250306135302-oscillazione_di_una_funzione_in_uno_spazio_metrico.org}{Oscillazione di una funzione in uno spazio metrico}
\item \href{20250306141326-estensione_di_una_funzione_ad_un_dominio_gdelta.org}{Estensione di una funzione ad un dominio Gdelta}
\end{itemize}
\section{{\bfseries\sffamily DONE} Esercizi}
\label{sec:orgc7e27a2}

\subsection{Foglio di esercizi 1}
\label{sec:org77da9d5}

\begin{itemize}
\item \href{20250313194245-sottospazi_di_spazi_polacchi_fsigma_densi_e_codensi_non_sono_gdelta.org}{Sottospazi di spazi polacchi Fsigma densi e codensi non sono Gdelta}
\item \href{20250310121944-esempi_di_spazi_polacchi.org}{Esempi di spazi polacchi - Esercizi TDI}
\item \href{20250313193059-proprieta_di_base_di_un_ultrametrica.org}{Proprietà di base di un ultrametrica}
\end{itemize}

OUTPUT: {[}BROKEN LINK: 41478265-ff20-444c-8a9c-e37fd954d64a]
\subsection{Foglio di esercizi 2}
\label{sec:org6272445}

\begin{itemize}
\item \href{20250327104701-immersione_topologica_dello_spazio_di_baire_nello_spazio_di_cantor.org}{Immersione topologica dello spazio di Baire nello spazio di Cantor} (Esercizio 1)
\item \href{20250327104804-insieme_limitato_dello_spazio_di_baire.org}{Insieme limitato dello spazio di Baire} (Esercizio 2)
\item \href{20250327104743-caratterizzazione_dei_compatti_dello_spazio_di_baire.org}{Caratterizzazione dei compatti dello spazio di Baire} (Esercizio 2)
\item \href{20250304152026-sottoinsiemi_gdelta_e_fsigma.org}{Insiemi Ksigma} (Esercizio 3)
\item \href{20250327104804-insieme_limitato_dello_spazio_di_baire.org}{Insieme definitivamente limitato dello spazio di Baire} (Esercizio 3)
\item \href{20250327105051-caratterizzazione_dei_sigma_compatti_nello_spazio_di_baire.org}{Caratterizzazione dei sigma-compatti nello spazio di Baire} (Esercizio 3)
\item \href{20250327105207-spazio_di_baire_si_surietta_in_ogni_spazio_polacco_non_vuoto_tramite_una_mappa_aperta_e_continua.org}{Spazio di Baire si surietta in ogni spazio polacco non vuoto tramite una mappa aperta e continua} (Esercizio 4)
\item \href{20250403093420-rango_di_cantor_bendixson.org}{Esempi di spazi polacchi con rango di Cantor-Bendixson arbitrario} (Esercizio 5)
\end{itemize}

OUTPUT: {[}BROKEN LINK: c7293a8f-9247-4c87-a182-c67988d8ae7d]
\subsection{Foglio di esercizi 3}
\label{sec:orgc62e674}

\begin{itemize}
\item \href{20250417125850-caratterizzazione_insieme_mai_denso.org}{Caratterizzazione insieme mai denso} (Esercizio 1)
\item \href{20250419122215-proprieta_insiemi_magri_comagri_non_magri.org}{Proprietà insiemi magri, comagri, non magri} (Esercizio 2)
\item \href{20250419121342-proprieta_della_gerarchia_di_borel.org}{Proprietà della gerarchia di Borel} (Esercizio 3)
\item \href{20250419122543-classi_ambigue_di_un_sottospazio_polacco_nella_gerarchia_di_borel.org}{Classi ambigue di un sottospazio polacco nella gerarchia di Borel} (Esercizio 4)
\item \href{20250419122722-classe_di_borel_dell_insieme_dei_punti_di_derivabilita_di_una_funzione_reale.org}{Classe di Borel dell'insieme dei punti di derivabilità di una funzione reale} (Esercizio 5)
\end{itemize}
\subsection{Foglio di esercizi 4}
\label{sec:org23614e0}
\begin{itemize}
\item \href{20250505103058-caratterizzazione_dei_punti_non_isolati_di_uno_spazio_polacco.org}{Caratterizzazione dei punti non isolati di uno spazio polacco} (Esercizio 1)
\item \href{20250505103212-caratterizzazione_dei_sottoinsiemi_chiusi_ma_non_aperti_di_un_polacco.org}{Caratterizzazione dei sottoinsiemi chiusi ma non aperti di un polacco} (Esercizio 2)
\item \href{20250505103416-esempi_di_sottoinsimi_pi03_completi.org}{Esempi di sottoinsiemi pi03 completi} (Esercizio 3)
\item \href{20250505103631-esempi_di_sottoinsiemi_sigma02_completi.org}{Esempi di sottoinsiemi sigma02 completi} (Esercizio 4)
\item \href{20250505103829-esempi_di_sottoinsiemi_analitici.org}{Esempi di sottoinsiemi analitici} (Esercizio 5)
\end{itemize}
\subsection{Foglio di esercizi 5}
\label{sec:org424192c}
\begin{itemize}
\item \href{20250521105415-separazione_tramite_boreliani_di_insiemi_invarianti_per_una_relazione_di_equivalenza.org}{Separazione tramite boreliani di insiemi invarianti per una relazione di equivalenza} (Esercizio 1)
\item \href{20250521105548-insieme_parzialmente_trasversale_per_una_relazione_di_equivalenza.org}{Insieme parzialmente trasversale per una relazione di equivalenza}
\item \href{20250521105603-insieme_trasversale_per_una_relazione_di_equivalenza.org}{Insieme trasversale per una relazione di equivalenza}
\item \href{20250521105653-selettore_per_una_relazione_di_equivalenza.org}{Selettore per una relazione di equivalenza}
\item \href{20250521110737-proprieta_insieme_parzialmente_trasversale_per_una_relazione_di_equivalenza_in_uno_spazio_polacco.org}{Proprietà insieme parzialmente trasversale per una relazione di equivalenza in uno spazio polacco} (Esercizio 2)
\item \href{20250521110811-proprieta_insieme_trasversale_e_selettore_per_una_relazione_di_equivalenza_in_uno_spazio_polacco.org}{Proprietà insieme trasversale e selettore per una relazione di equivalenza in uno spazio polacco} (Esercizio 3)
\item \href{20250521110928-coanalitici_sono_unione_di_omega1_boreliani.org}{Coanalitici sono unione di omega1 boreliani} (Esercizio 4)
\end{itemize}
\end{document}
