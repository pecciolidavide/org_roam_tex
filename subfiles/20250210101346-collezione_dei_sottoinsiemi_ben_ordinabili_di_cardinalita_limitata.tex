% Intended LaTeX compiler: pdflatex
\documentclass[../main]{subfiles}


\begin{document}

\section{Collezione dei sottoinsiemi ben ordinabili di cardinalità limitata}
\label{sec:org1cbed08}
Contesto: \href{20250130104245-morse_kelly_set_theory.org}{Morse Kelly Set Theory}
\begin{definizione}
Sia \(X\) un \href{20250130104331-insieme_mk.org}{insieme} e sia \(\kappa\) un \href{20250203161341-cardinali.org}{cardinale}. Si definisce
\begin{equation*}
\parti[\kappa]{X}\coloneqq\set{Y \subseteq X\ |\ |Y|<\kappa}
\end{equation*}
dove \(|Y|\) è la \href{20241213101756-cardinalita.org}{cardinalità} del \href{20250131155822-operazioni_insiemistiche_tra_classi_mk.org}{sottoinsieme}\(Y \subseteq X\), che esiste soltanto quando \(Y\) è \href{20250203161431-classe_ben_ordinabile_mk.org}{ben ordinabile}.

Pertanto questa è la collezione di tutti i sottoinsiemi di \(X\) ben ordinabili di cardinalità minore di \(\kappa\).
\end{definizione}
\subsection{Applicazione ai cardinali}
\label{sec:orgac251ab}

Notiamo che, considerando la \href{20250205182017-insieme_dei_sottoinsiemi_con_order_type_fissato.org}{definzione} di \([\lambda]^{<\kappa}\) per i \href{20250203161341-cardinali.org}{cardinali} \(\lambda\) e \(\kappa\) si ha che\footnote{Vedi ``\href{20250612151505-aritmetica_dei_cardinali.org}{Elevamento a potenza di Cardinali}''}
\begin{equation*}
\parti[\kappa]{\lambda} = [\lambda]^{<\kappa}
\end{equation*}
e, assumendo \href{20250206171508-axiom_of_choiche.org}{AC}, si ha che
\begin{equation*}
\card{\parti[\kappa]{\lambda}} = \lambda^{<\kappa} \le \lambda^{k}
\end{equation*}
dove
\begin{equation*}
\lambda^{<\kappa} \coloneqq \sup\set{\lambda^{\nu}\mid \nu \in \operatorname{Card}\,\land\, \nu\le\kappa}
\end{equation*}
\end{document}
