% Intended LaTeX compiler: pdflatex
\documentclass[../main]{subfiles}


\begin{document}

\section{Varietà Topologica}
\label{sec:org1401724}
\begin{definizione}
Uno \href{20250103145124-topologia.org}{spazio topologico} \(M\) si dice \textbf{varietà topologica \(n\)-dimensionale} se:
\begin{enumerate}
\item \(M\) è uno \href{20250109155715-spazio_topologico_di_hausdorff.org}{spazio topologico di Hausdorff}
\item \(M\) è uno \href{20250111142303-spazio_topologico_a_base_numerabile.org}{spazio topologico a base numerabile}
\item \(\forall\, p \in M\) esiste \(U\) \href{20250111142313-intorno.org}{intorno} \href{20250103145124-topologia.org}{aperto} di \(p\) in \(M\) ed esiste
\begin{equation*}
  \Phi: U \longrightarrow V
\end{equation*}
\href{20250111142332-omeomorfismo.org}{omeomorfismo}. \((U,\Phi)\) si dice \textbf{carta topologica}.
\end{enumerate}
\end{definizione}
\end{document}
