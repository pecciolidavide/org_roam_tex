% Intended LaTeX compiler: pdflatex
\documentclass[../main]{subfiles}

\usepackage[hyperref]{biblatex}
\date{}
\title{}
\begin{document}

\section{Kernel}
\label{sec:org9dac8af}
\begin{definizione}
Siano \(A,B\) \href{20241205141146-gruppo_abeliano.org}{gruppi} con \(0_{B} \in B\) l'elemento neutro. Se \(f:A\to B\), si definisce il \uline{kernel di \(f\)} come la \href{20250202190147-immagine_punto_a_punto_di_due_classi.org}{retroimmagine}:
\begin{equation*}
\ker f \coloneqq f^{-1}[\set{0_{B}}].
\end{equation*}
\end{definizione}

\begin{prop}
Se \(f\) è un \href{20241206115531-morfismo_di_gruppi.org}{morfismo di gruppi} ed è \href{20241219101956-funzione_iniettiva.org}{iniettiva}, allora \(\ker f = \set{0_{A}}\).
\end{prop}
\subsection{Invarianza del kernel per iniezioni}
\label{sec:org97460b9}
\begin{lem}
Siano \(f: A \to B\) e \(g: B \to C\) due \href{20241206115531-morfismo_di_gruppi.org}{morfismi di gruppi}. Se \(g\) è \textbf{\href{20241219101956-funzione_iniettiva.org}{iniettiva}}, allora:
\begin{equation*}
\ker(f) = \ker(g \circ f)
\end{equation*}
\end{lem}
\end{document}
