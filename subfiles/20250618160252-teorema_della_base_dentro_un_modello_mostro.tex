% Intended LaTeX compiler: pdflatex
\documentclass[../main]{subfiles}


\begin{document}

\section{Teorema della Base dentro un modello mostro}
\label{sec:org680a05f}
Si utilizza la \href{20250612143636-notazione_teoria_dei_modelli.org}{notazione della Teoria dei Modelli}.

Sia \(\mathcal{L}\) un \href{20250130162057-linguaggio_del_prim_ordine.org}{linguaggio del prim'ordine} fissato, \(T\) una \(\mathcal{L}\)-\href{20250130114950-teoria_del_prim_ordine.org}{teoria} \href{20250131123151-teoria_completa.org}{completa} senza \href{20250131122945-modello_di_un_insieme_di_formule.org}{modelli} finiti, ed un modello \(\mathcal{U}\) di \href{20241213101756-cardinalita.org}{cardinalità} \(\kappa>\card{\mathcal{L}}\), con \(\kappa\) \href{20250211123155-cardinale_limite_forte.org}{inaccessibile}. Si lavora nell'ambito di un \href{20250617102733-modello_mostro.org}{MODELLO MOSTRO}.

Sia \(T\) \href{20250618153446-struttura_minimale.org}{fortemente minimale}.
\subsection{Teorema}
\label{sec:orgeee75af}

Per \(C \subseteq \mathcal{U}\):
\begin{enumerate}
\item per ogni \href{20250618095344-elementi_algebrici_e_definibili_teoria_dei_modelli.org}{insieme indipendente} \(B \subseteq C\) esiste una \href{20250618155810-base_di_un_insieme_dentro_un_modello_mostro.org}{base} \(B'\) di \(C\) tale che \(B \subseteq B'\);
\item tutte le basi di \(C\) hanno la stessa \href{20241213101756-cardinalita.org}{cardinalità}.
\end{enumerate}

Questo teorema garantisce che la seguente definizione sia ben posta:
\subsection{Dimensione di un insieme dentro un modello mostro}
\label{sec:org7df4a55}
La \uline{dimensione di \(C \subseteq \mathcal{U}\)}, denotata con \(\dim C\), è la cardinalità di una base di \(C\).
\end{document}
