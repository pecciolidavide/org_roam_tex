% Intended LaTeX compiler: pdflatex
\documentclass[../main]{subfiles}

\usepackage[hyperref]{biblatex}
\date{}
\title{}
\begin{document}

\section{Funzione beta di Godel}
\label{sec:org36848e6}
\begin{definizione}
Si definisce la funzione \(\beta: \N^{2}\to \N\) come segue:
\begin{equation*}
\beta(m,i) \coloneqq \operatorname{Res}\left((m)_{0}, (i+1)\cdot (m)_{1}+1\right)
\end{equation*}
dove \((\cdot)_{0}, (\cdot)_{1}\) sono le \href{20250215151413-biiezione_canonica_tra_n_e_n2.org}{inverse della biiezione} \(\bm{J}: \N^{2}\to \N\), e il \href{20250108174027-divisione.org}{resto}:
\begin{equation*}
\operatorname{Res}(a,b) \coloneqq \begin{cases}
a & b=0\\
a \mathrel{\operatorname{mod}} b & b>0
\end{cases}
\end{equation*}
\end{definizione}

Questa è una \href{20250215141024-funzioni_primitive_ricordive.org}{funzione ricorsiva primitiva}, poiché \href{20250215141024-funzioni_primitive_ricordive.org}{composizione} \href{20250215141024-funzioni_primitive_ricordive.org}{di funzioni ricorsive primitive}.

Si osservi inoltre che per ogni \(m,i \in \N\) si ha
\begin{equation*}
\beta(m,i)\le(m)_{0}\le m
\end{equation*}
ed inoltre, se \(m\neq 0\)
\begin{equation*}
\beta(m,i)< m.
\end{equation*}
\end{document}
