% Intended LaTeX compiler: pdflatex
\documentclass[../main]{subfiles}


\begin{document}

\section{Assolutezza delle formule tra un insieme transitivo e un modello di MK}
\label{sec:org60d9455}
Contesto: si lavora nella \href{20250130104245-morse_kelly_set_theory.org}{Morse Kelly Set Theory}, con il linguaggio del prim'ordine \(\mathcal{L}=\set{\in}\); \(\operatorname{V}\) è la \href{20250130104320-classe_mk.org}{classe} \href{20250203104513-classe_totale.org}{totale}
\begin{prop}
Sia \(M\neq\emptyset\) un \href{20250130104331-insieme_mk.org}{insieme} \href{20250203110714-classe_transitiva.org}{transitivo}.
\begin{enumerate}
\item Tutte le \href{20250131103317-formula_del_prim_ordine.org}{formule} \href{20250131103317-formula_del_prim_ordine.org}{senza quantificatori} sono \href{20250207123940-formula_assoluta.org}{assolute tra \(M\) e \(V\)};
\item Tutte le \href{20250620163542-complessita_di_una_formula_nel_linguaggio_della_teoria_degli_insiemi.org}{formule \(\Delta_{0}\)} sono \href{20250207123940-formula_assoluta.org}{assolute tra \(M\) e \(V\)};
\item Le \href{20250620163542-complessita_di_una_formula_nel_linguaggio_della_teoria_degli_insiemi.org}{formule \(\Sigma_{1}\)} sono \href{20250207123940-formula_assoluta.org}{assolute verso l'alto tra \(M\) e \(V\)}; le \href{20250620163542-complessita_di_una_formula_nel_linguaggio_della_teoria_degli_insiemi.org}{formule \(\Pi_{1}\)} sono \href{20250207123940-formula_assoluta.org}{assolute verso il basso tra \(M\) e \(V\)}.
\end{enumerate}
\end{prop}
\end{document}
