% Intended LaTeX compiler: pdflatex
\documentclass[../main]{subfiles}


\begin{document}

\textbf{\textbf{\uline{NOTA}: quando si parla di \uline{classi}, se ne parla nell'ambito della \href{20250130104245-morse_kelly_set_theory.org}{Morse Kelly Set Theory}; quando si parla di insiemi, il discorso ha validità più generale.}}

Una \uline{sequenza} è un modo per indicare una \href{20250202170607-classe_relazione_binaria.org}{funzione} \(f:I\to X\): si scriverà, in luogo di \(f\):
\begin{equation*}
\langle a_{i}\mid i \in I\rangle = \langle a_{i}: i \in I\rangle
\end{equation*}
dove, quindi \(a_{i} =f(i)\).
(si osservi che questa notazione viene utilizzata anche quando \(X\) è una \href{20250130104320-classe_mk.org}{classe}).

Con un abuso di notazione, anche quando \(f\) è una \href{20250202170607-classe_relazione_binaria.org}{Classe-Funzione}, si utilizza la stessa scrittura.
\section{Stringa}
\label{sec:orgf877da8}
Sia \(X\) un \href{20250130104331-insieme_mk.org}{insieme} (o una \href{20250130104320-classe_mk.org}{classe}).

Una sequenza indicizzata da \(n \in \N\), \(s:n \to X\) \href{20250202130045-insieme_dei_numeri_naturali_mk.org}{(dove \(n = \set{0,1,\dots,n-1} \in \N\))} viene detta \uline{stringa} (oppure sequenza finita). Spesso si indica:
\begin{equation*}
s=\langle a_{i}\mid i \in n\rangle = \langle a_{0},a_{1},\dots,a_{n-1}\rangle.
\end{equation*}

Questa quindi è la \href{20250202173528-dominio_range_e_campo_di_una_classe_relazione.org}{funzione}
\begin{align*}
s: n\subseteq \N &\longrightarrow X\\
i &\longmapsto a_{i}
\end{align*}

Il \href{20250202173528-dominio_range_e_campo_di_una_classe_relazione.org}{dominio} di \(s\) viene detto \uline{lunghezza della stringa}, e si indica con \(\operatorname{lh}(s)\).

Vedi anche Section 3.E of Andretta
\subsection{Insieme delle sequenze finite}
\label{sec:orge4ab3b0}
Per ogni \uline{\href{20250130104320-classe_mk.org}{classe} \(X\)}, si definisce la classe:\footnote{Vedi \href{20250131155822-operazioni_insiemistiche_tra_classi_mk.org}{Sottoclasse MK} e \href{20250202173528-dominio_range_e_campo_di_una_classe_relazione.org}{Range di una funzione}}
\begin{equation*}
X^{<\omega} \coloneqq \set{s\mid s\text{ è una sequenza finita e } \operatorname{ran}(s) \subseteq X}
\end{equation*}

\(X^{<\omega}\) è un \href{20250130104331-insieme_mk.org}{insieme} se e solo se \(X\) è un \href{20250130104331-insieme_mk.org}{insieme}.
\end{document}
