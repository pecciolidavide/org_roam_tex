% Intended LaTeX compiler: pdflatex
\documentclass[../main]{subfiles}

\usepackage[hyperref]{biblatex}
\date{}
\title{}
\begin{document}

\section{Ogni chiuso di uno spazio metrico secondo numerabile ammette un ricoprimento numerabile di diametro arbitrariamente piccolo}
\label{sec:org074b89e}
\subsection{Proposizione}
\label{sec:org6f870e9}
Se \(X\) è uno \href{20250301193511-spazio_metrico.org}{spazio metrico} \href{20250111142303-spazio_topologico_a_base_numerabile.org}{secondo numerabile}, allora per ogni \(U \subseteq X\) \href{20250103145124-topologia.org}{aperto} e per ogni \(\varepsilon \in \R^{+}\), esiste un \href{20250103164252-ricoprimento.org}{ricoprimento} \href{20250103145124-topologia.org}{aperto} \href{20250111143651-insieme_numerabile.org}{numerabile} \((U_{n})_{n \in \omega}\) di \(U\) tale che \(\operatorname{Cl}(U_{n}) \subseteq U\) e \(\operatorname{diam}(U_{n})< \varepsilon\), per ogni \(n \in \omega\).
\subsubsection{Dimostrazione}
\label{sec:org1e920db}
Sia \((X,d)\) lo \href{20250301193511-spazio_metrico.org}{spazio metrico} in considerazione. Si denotino con
\begin{equation*}
B_{d}(x,r) \coloneqq \set{y \in X\mid d(x,y)< r}.
\end{equation*}

Siccome \(X\) è secondo numerabile \href{20250331095811-spazio_topologico_secondo_numerabile_implica_separabile.org}{allora} \(X\) è \href{20250301192908-spazio_topologico_separabile.org}{separabile}, e \href{20250301192908-spazio_topologico_separabile.org}{pertanto} \(U \subseteq X\) è \href{20250301192908-spazio_topologico_separabile.org}{separabile}. Sia quindi \(C\) sottoinsieme \href{20250301193045-sottoinsieme_denso.org}{denso} di \(U\), numerabile. Allora, per ogni \(c \in C\) esiste \(0<r_{c}<\varepsilon\) tale che \(B_{d}(c,r_{c}) \subseteq U\), poiché \(U\) aperto \href{20250317093153-insieme_aperto_sse_intorno_di_ogni_suo_punto.org}{e quindi} intorno di ogni suo punto. In particolare, si richiede che
\begin{equation*}
r_{c}\coloneqq \sup \set{ r \in \left(0,\frac{\varepsilon}{2}\right)\mid B_{d}(c,r) \subseteq U}.
\end{equation*}

Si consideri quindi:
\begin{equation*}
\mathcal{B}_{U} \coloneqq \set{B_{d}\left(c,\frac{r_{c}}{2}\right)\mid c \in C}
\end{equation*}
\begin{itemize}
\item Sia ora \(c \in C\) fissato, e sia \((x_{n})_{n \in \omega}\subseteq B_{d}\left(c,\frac{r_{c}}{2}\right)\) \href{20250115100904-successione.org}{successione} \href{20250115100930-convergenza_per_una_successione.org}{convergente} a \(x\). Allora, siccome \(d(x,x_{n})<\frac{r_{c}}{2}\) \href{20250304141512-proprieta_vere_definitivamente.org}{definitivamente}:
\begin{align*}
d(x,c) &\le d(x,x_{n}) + d(x_{n}, c)\\
&< \frac{r_{c}}{2} + \frac{r_{c}}{2} < r_{c}
\end{align*}
e pertanto \(x \in U\). Quindi, per la \href{20250303121451-caratterizzazione_della_chiusura_in_termini_di_successioni.org}{caratterizzazione della chiusura per successioni}, \(\operatorname{cl}\left(B_{d}\left(c,\frac{r_{c}}{2}\right)\right) \subseteq U\).
\item Infine, si ha che \href{20250131155822-operazioni_insiemistiche_tra_classi_mk.org}{l'unione} \(\bigcup \mathcal{B}_{U} = U\). Infatti, se \(y \in U \setminus C\) allora esiste \(\delta<\frac{\varepsilon}{2}\) tale che \(B_{d}(y,\delta) \subseteq U\). In particolare, esiste \(c \in B_{d}(y,\delta/2)\cap C\) (poiché \(C\) è denso in \(U\)). Si ha quindi che \(y \in B_{d}(c, \delta/2) \subseteq U\): infatti, se per assurdo esistesse \(x \in B_{d}(c,\delta/2)\setminus U\) allora
\begin{align*}
  d(x,y) &\le d(x,c) + d(c,y)\\
  	&< \frac{\delta}{2}+\frac{\delta}{2} < \delta
\end{align*}
e pertanto \(x \in B_{d}(y,\delta) \subseteq U\). Assurdo.

Dunque \(\delta/2<\varepsilon/2\) e \(B_{d}(c,\delta/2) \subseteq U\), e dunque \(r_{c}\ge \delta/2\) per massimalità. Pertanto \(B_{d}(c,r_{c})\supseteq B_{d}(c,\delta/2)\ni y\), e quindi \(y \in \bigcup\mathcal{B}_{U}\).
\end{itemize}
\subsection{NON Proposizione}
\label{sec:org71b74ca}
If \(X\) is a \href{20250301193511-spazio_metrico.org}{metric space}, then for every \href{20250103145124-topologia.org}{open} \(U\) and every \(\varepsilon \in \R^{+}\) there is a \href{20250111143651-insieme_numerabile.org}{countable} \href{20250103164252-ricoprimento.org}{covering} \(\left(U_n\right)_{n \in \omega}\) of \(U\) such that \(\operatorname{Cl}\left(U_n\right) \subseteq U\) (vedi \href{20250103144944-chiusura_topologica.org}{Chiusura Topologica}) and \(\operatorname{diam}\left(U_n\right)<\varepsilon\) (vedi \href{20250327131547-diametro_di_un_insieme.org}{Diametro di un insieme}), for all \(n \in \omega\)
\subsubsection{Controesempio}
\label{sec:orge8d6eb5}

Sia \(X \coloneqq \R\times [0,1]\), dotato della \href{20250301193511-spazio_metrico.org}{distanza}:
\begin{equation*}
d\left((x,t),(y,s)\right) \coloneqq \begin{cases}
2 & x\neq y\\
|t-s| & x=y
\end{cases}
\end{equation*}

La funzione \(d\) è realmente una distanza: per ogni \((x,t),(y,s), (z,k) \in X\)
\begin{enumerate}
\item \(d\left((x,t),(y,s)\right)\ge 0\);
\item \(d\left((x,t),(y,s)\right) = 0\) se e solo se \(x=y\) e \(|t-s|=0\) se e solo se \((x,t) = (y,s)\);
\item \(d\left((x,t),(y,s)\right) = d\left((y,s), (x,t)\right)\);
\item la \href{20250306115949-disuguaglianza_triangolare.org}{disuguaglianza triangolare}:
\begin{equation*}
 	d\left((x,t),(y,s)\right)\le d\left((x,t),(z,k)\right)+d\left((z,k),(y,s)\right)
\end{equation*}
per casi:
\begin{itemize}
\item se \(x\neq y \neq z\) allora
\begin{align*}
   d\left((x,t),(y,s)\right) &= 2\\
   d\left((x,t),(z,k)\right) &= 2\\
   d\left((z,k),(y,s)\right) &= 2
\end{align*}
e quindi
\begin{equation*}
   2=d\left((x,t),(y,s)\right) \le d\left((x,t),(z,k)\right)+d\left((z,k),(y,s)\right)=4
\end{equation*}
\item se \(x=y\neq z\) allora
\begin{align*}
   d\left((x,t),(y,s)\right)&\le 1\\
   d\left((x,t),(z,k)\right) = 2 &= d\left((z,k),(y,s)\right)
\end{align*}
e quindi si ha la tesi;
\item se \(x=z\neq y\) oppure \(y=z \neq x\) (solo il primo per simmetria):
\begin{align*}
   d\left((x,t),(y,s)\right) &= 2\\
   d\left((x,t),(z,k)\right) &= \ell \le 1\\
   d\left((z,k),(y,s)\right) &= 2
\end{align*}
e quindi
\begin{equation*}
   2 = d\left((x,t),(y,s)\right)\le d\left((x,t),(z,k)\right)+d\left((z,k),(y,s)\right) = \ell + 2
\end{equation*}
\item se \(x=y=z\) allora
\begin{equation*}
   |t-s|\le |t-k| + |k-s|.
\end{equation*}
\end{itemize}
\end{enumerate}

Dunque \(\left(\R\times[0,1], d\right)\) è uno \href{20250301193511-spazio_metrico.org}{spazio metrico}.

L'aperto \(\R\times [0,1]\) non ammette alcun \href{20250103164252-ricoprimento.org}{ricoprimento} \((U_{n})_{n \in \omega}\) tale che, per ogni \(n \in \omega\): \(\operatorname{diam}(U_{n})< 1/2\). Si supponga per assurdo che esista.

Fissato \(U_{n}\): per ogni \((x,t), (y,s) \in U_{n}\): \(d\left((x,t),(y,s)\right)<1/2\), e quindi \(x=y\). Pertanto esiste \(x_{n} \in \R\) tale che \(U_{n} \subseteq \set{x_{n}}\times [0,1]\).

Si ha quindi che
\begin{equation*}
\bigcup_{n \in \omega} U_{n} \subseteq \set{x_{n}\mid n \in \omega}\times [0,1] \subsetneqq \R\times[0,1]
\end{equation*}
e pertanto \((U_{n})_{n \in \omega}\) non è un ricoprimento. Assurdo.
\end{document}
