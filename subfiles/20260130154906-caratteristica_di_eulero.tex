% Intended LaTeX compiler: pdflatex
\documentclass[../main]{subfiles}


\begin{document}

\section{Caratteristica di Eulero per una superficie topologica}
\label{sec:org721174d}
\href{20250515141706-da_finire.org}{DA FINIRE}: Caratteristica di Eulero per una \href{20251230172241-superficie_topologica.org}{Superficie Topologica}

\begin{prop}
La caratteristica eulero di una \href{20251230172241-superficie_topologica.org}{superficie topologica} \href{20250103163701-spazio_topologico_compatto.org}{compatta} è una delle seguenti, per il \href{20260130155105-teorema_di_classificazione_delle_superfici_topologiche.org}{teorema di classificazione}:\footnote{\(T_{g}\) è il \href{20260130155514-sfera_con_g_buchi.org}{Toro con \(g\)-buchi}, \(P_{n}\) è la \href{20260130155555-somma_connessa_di_superfici_topologiche.org}{somma connessa} di \(n\) copie di \href{20260130160217-piano_proiettivo_reale_e_varieta_topologica.org}{\(\mathds{P}^{1}_{\R}\)} e \(\mathds{S}^{2}\) è la \href{20250115150754-sfera_n_dimensionale.org}{sfera \(2\)-dimensionale}.}
\begin{itemize}
\item \(\chi(T_{g}) = 2-2g\);
\item \(\chi(P_{n}) = 2-n\);
\item \(\chi(\mathds{S}^{2}) = 2\).
\end{itemize}
\end{prop}
\end{document}
