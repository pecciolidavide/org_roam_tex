% Intended LaTeX compiler: pdflatex
\documentclass[../main]{subfiles}


\begin{document}

\section{Spazio affine è irriducibile}
\label{sec:org346abd3}
Sia \(\K\) un \href{20241231112713-campo_algebricamente_chiuso.org}{campo algebricamente chiuso}.
\subsection{Proposizione}
\label{sec:org9229bf0}
\(\A^{n}\) (vedi \href{20241231114009-spazio_affine.org}{Spazio Affine}) con la \href{20250103101459-topologia_di_zariski_affine.org}{topologia di Zariski} è \href{20250103164917-sottospazio_irriducibile.org}{irriducibile}
\subsubsection{Dimostrazione}
\label{sec:orgb8b39ad}
Per la \href{20250103170329-caratterizzazione_di_sottospazi_affini_irriducibili_tramite_ideali.org}{caratterizzazione degli irriducibili con la topologia di Zariski affine}, \(\A^{n}\) è irriducibile se e solo se il suo \href{20250103144124-ideale_di_un_sottoinsieme.org}{ideale} \(I(\A^{n})\) è \href{20250103171055-ideale_primo.org}{primo}.
\paragraph{Claim: \(I(\A^{n})=(0)\).}
\label{sec:org272e39d}
(vedi \href{20241219113154-ideale_generato.org}{Ideale-Generato} e \href{20250103144124-ideale_di_un_sottoinsieme.org}{Ideale di un sottoinsieme})

Dal momento che \((0)=\set{0}\), è sufficiente dimostrare che se \(f \in \K[x_{1},\dots,x_{n}]\) e \(f \neq 0\), allora esiste \(p \in \A^{n}\) tale che \(f(p)\neq 0\). Questo implica che \(I(\A^{n})=\{0\}\). Si dimostra per induzione su \(n\).
\begin{enumerate}
\item Caso base: \(n=1\)
\label{sec:orgeeef3e5}
Dal momento che \(\K\) è algebricamente chiuso, \href{20250102103618-infinitezza_campi_alg_chiusi.org}{\(\K\) è infinito}, mentre per ogni  \(f \in \K[x_{1}]\) di grado \(d\), \(f\) ha al più \(d\) radici distinte (per il \href{20250102154204-teorema_fondamentale_dell_algebra.org}{Teorema Fondamentale dell'Algebra}).
\item Passo induttivo
\label{sec:org2fabff3}
Supponiamo l'ipotesi vera per un numero di variabili minori o uguali ad \(n-1\), e sia \(f \in \K[x_{1},\dots,x_{n}]\). Se \(x_{n}\) non compare in \(f\) si ha la tesi per ipotesi induttiva.

Altrimenti, scriviamo
\begin{equation*}
f = \sum_{i=0}^{d} f_{i}x_{n}^{d},\qquad f_{i} \in \K[x_{1},\dots,x_{n}]
\end{equation*}
dove \(d\) è il \href{20241231124742-grado_polinomi.org}{grado} di \(f\) riferito alla variabile \(x_{n}\).

Per ipotesi induttiva, esiste \((a_{1},\dots,a_{n-1})\in\A^{n-1}\) tale che \(f_{d}(a_{1},\dots,a_{n-1})\neq 0\). Pertanto, il polinomio
\begin{equation*}
g(x_{n}) = f(a_{1},\dots,a_{n-1}, x_{n})
\end{equation*}
non è il polinomio nullo (infatti almeno il coefficiente di grado massimo è non nullo). Applicando nuovamente l'ipotesi induttiva, si ha che esiste \(a_{n} \in \A\) tale che \(g(a_{n})\neq 0\), e pertanto si è trovato \((a_{1},\dots,a_{n})\) tale che
\begin{equation*}
f(a_{1},\dots,a_{n-1},a_{n}) \neq  0.
\end{equation*}
\end{enumerate}
\paragraph{Claim: \((0)\) è un ideale primo}
\label{sec:org575f509}
\href{20250104101516-caratterizzazione_domini_d_integrita_tramite_ideali_primi.org}{perché} l'\href{20241219113434-anello_dei_polinomi.org}{anello dei polinomi} \(\K[x_{1},\dots,x_{n}]\) è un \href{20250103143950-dominio_di_integrita.org}{dominio di integrità} (poichè l'\href{20250102115740-anello_dei_polinomi_ad_ideali_principali.org}{Anello dei polinomi è ad ideali principali} e dunque è un dominio di integrità).
\end{document}
