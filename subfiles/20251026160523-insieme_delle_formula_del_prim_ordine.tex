% Intended LaTeX compiler: pdflatex
\documentclass[../main]{subfiles}


\begin{document}

\def\L{\mathcal{L}} % Il linguaggio
\def\U{\mathcal{U}} % Modello mostro

\section{Insieme delle formule del prim'ordine}
\label{sec:org7414ef1}
Sia \(\L\) un \href{20250130162057-linguaggio_del_prim_ordine.org}{linguaggio} del prim'ordine.

\begin{definizione}
L'insieme delle \href{20250131103317-formula_del_prim_ordine.org}{formule del prim'ordine} del linguaggio \(\L\) e variabile libera \(z=\langle z_{i}:i<\alpha\rangle\) per qualche \href{20250203111003-ordinali.org}{ordinale} \(\alpha\) si indica con \(\L_{z}\).

Se \(M\) è una \(\L\)-struttura e \(A \subseteq M\), allora l'insieme delle \href{20250131103317-formula_del_prim_ordine.org}{\(\L\)-formule} di variabile \(z\) e \href{20250212102927-enunciato_con_parametri.org}{parametri in \(A\)} si indica con \(\L_{z}(A)\).
\end{definizione}
\section{Cardinalità dell'insieme delle formule di un linguaggio del prim'ordine}
\label{sec:org7f8fa48}
Sia \(\mathcal{L}\) un linguaggio del prim'ordine.

\begin{prop}
La \href{20241213101756-cardinalita.org}{cardinalità} dell'insieme delle \(\mathcal{L}\)-\href{20250131103317-formula_del_prim_ordine.org}{formule} è
\begin{equation*}
\card{\operatorname{Form}_{\mathcal{L}}} = \max (\card{\mathcal{L}}, \aleph_{0}) = \card{\mathcal{L}} + \omega.
\end{equation*}
\end{prop}
\end{document}
