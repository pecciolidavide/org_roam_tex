% Intended LaTeX compiler: pdflatex
\documentclass[../main]{subfiles}


\begin{document}

\begin{thm}
Si consideri la \href{20250624155858-neurone_artificiale.org}{funzione di attivazione} \(\operatorname{ReLU}(x) = x\,H(x)\)\footnote{Dove \(H(x)\) è la \href{20250624161413-funzione_di_heaviside.org}{Funzione di Heaviside}}. Allora per ogni \(g \in C[0,1]\)\footnote{Vedi ``\href{20250113125602-classe_c_di_una_funzione.org}{Classe C di una funzione}''} esiste \(N \in \N\) ed esistono, per \(i=0,\dots,N-1\) degli \(\alpha_{i}, \theta_{i}, \beta \in \R\) tali che, detta
\begin{equation*}
G(x) \coloneqq \beta+\sum_{i=0}^{N-1} \alpha_{i}\operatorname{ReLU}(x+\theta_{i})
\end{equation*}
si ha che
\begin{equation*}
\forall x \in[0,1]\qquad |g(x)-G(x)|<\varepsilon.
\end{equation*}
\end{thm}
\end{document}
