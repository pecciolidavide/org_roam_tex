% Intended LaTeX compiler: pdflatex
\documentclass[../main]{subfiles}


\begin{document}

\section{Compatibilità tra atlanti astratti e varietà topologiche}
\label{sec:orgcd67682}
\subsection{Proposizione 1}
\label{sec:org29462c8}
Sia \(M\) un insieme e sia \(\mathcal{U}\coloneqq\set{(U_{\alpha},\varphi_{\alpha})}\) un \href{20250113104517-atlante_astratto.org}{atlante astratto}. Se per ogni \(p,q \in M\) vale una tra le seguenti
\begin{itemize}
\item \(p,q \in U_{\alpha}\) per qualche \(\alpha\);
\item esistono \(\alpha,\beta\) tali che \(U_{\alpha}\cap U_{\beta}=\emptyset\) e \(p \in U_{\alpha}, q \in U_{\beta}\);
\end{itemize}
allora la \href{20250113104517-atlante_astratto.org}{topologia indotta} da \(\mathcal{U}\) è \href{20250109155715-spazio_topologico_di_hausdorff.org}{T2}.
\subsection{Proposizione 2}
\label{sec:orgab470d5}
Sia \(M\) un insieme e sia \(\mathcal{U}\coloneqq\set{(U_{\alpha},\varphi_{\alpha})}\) un \href{20250113104517-atlante_astratto.org}{atlante astratto}. Se \(\mathcal{U}\) ha una \href{20250111143651-insieme_numerabile.org}{quantità numerabile} di carte, allora la topologia indotta è numerabile.
\subsection{Proposizione}
\label{sec:orge0c492f}
Se \(M\) è una \href{20250111092123-varieta_topologica.org}{varietà topologica} e \(\mathcal{U}\coloneqq\set{(U_{\alpha},\varphi_{\alpha})}\) è un \href{20250113095039-atlante_topologico.org}{atlante topologico}, allora \(\mathcal{U}\) \href{20250113104517-atlante_astratto.org}{induce} su \(M\) la stessa \href{20250103145124-topologia.org}{topologia}.
\end{document}
