% Intended LaTeX compiler: pdflatex
\documentclass[../main]{subfiles}

\usepackage[hyperref]{biblatex}
\date{}
\title{}
\begin{document}

\section{Estensione di una funzione continua ad un dominio Gdelta}
\label{sec:orga9ae0af}
\subsection{Teorema}
\label{sec:orgceb6a09}
Siano \(X,Y\) due \href{20250103145124-topologia.org}{spazi topologici}, \(X\) \href{20250301193401-spazio_topologico_metrizzabile.org}{metrizzabile} e \(Y\) \href{20250301193401-spazio_topologico_metrizzabile.org}{completamente metrizzabile}.
Sia \(A \subseteq X\) e sia \(f:A\to Y\) una \href{20250103103252-funzione_continua.org}{funzione continua}.
$\backslash$[
\begin{tikzcd}[ampersand replacement=\&,cramped]
	{X\supseteq A} \& Y
	\arrow["f", from=1-1, to=1-2]
\end{tikzcd}
$\backslash$]

Allora esiste \(G \subseteq X\) un \href{20250130104331-insieme_mk.org}{insieme} \(\bm{G}_{\delta}\) ed esiste \(g:G\to Y\) tali che
\begin{enumerate}
\item \(A \subseteq G \subseteq \operatorname{Cl}(A)\) (vedi \href{20250103144944-chiusura_topologica.org}{Chiusura Topologica});
\item \(g\) \href{20250103103252-funzione_continua.org}{continua};
\item \(g \upharpoonright A = f\) (vedi \href{20250205170515-restrizione_di_una_classe.org}{Restrizione di una classe MK})
\end{enumerate}
\subsubsection{Dimostrazione}
\label{sec:org88d96bc}
Sia \(d'\) una \href{20250301194153-spazio_metrico_completo.org}{metrica completa} \href{20250301193530-topologia_indotta_da_una_distanza.org}{compatibile} su \(Y\).
\paragraph{Definizione insieme \(G\)}
\label{sec:org4714019}
Si ponga
\begin{equation*}
G \coloneqq \operatorname{Cl}(A)\cap \set{z \in X\ |\ \operatorname{osc}_{f}(z) = 0}
\end{equation*}
(vedi \href{20250103144944-chiusura_topologica.org}{Chiusura Topologica} e \href{20250306135302-oscillazione_di_una_funzione_in_uno_spazio_metrico.org}{Oscillazione di una funzione in uno spazio metrico} e \href{20250306135302-oscillazione_di_una_funzione_in_uno_spazio_metrico.org}{Insieme dei punti in cui l'oscillazione di una funzione si annulla})

Questo è un insieme \(\bm{G}_{\delta}\), in quanto \href{20250131155822-operazioni_insiemistiche_tra_classi_mk.org}{intersezione} di due \(\bm{G}_{\delta}\):
\begin{itemize}
\item \(\operatorname{Cl}(A)\) è un \href{20250306142322-chiuso_in_uno_spazio_metrizzabile_e_gdelta.org}{chiuso in uno spazio metrizzabile};
\item per \(\set{z \in X\ |\ \operatorname{osc}_{f}(z) =0}\) si veda ``\href{20250306135302-oscillazione_di_una_funzione_in_uno_spazio_metrico.org}{Insieme dei punti in cui l'oscillazione di una funzione si annulla}''
\end{itemize}

Inoltre, ovviamente \(G \subseteq \operatorname{Cl}(A)\), mentre, siccome \(f\) è continua su \(A\), \(A \subseteq \set{z \in X\ |\ \operatorname{osc}_{f}(z) = 0}\) (vedi \href{20250306135302-oscillazione_di_una_funzione_in_uno_spazio_metrico.org}{Insieme dei punti in cui l'oscillazione di una funzione si annulla}), e quindi
\begin{equation*}
A \subseteq G \subseteq \operatorname{Cl}(A)
\end{equation*}
\paragraph{Definizione funzione \(g\)}
\label{sec:org6e200b8}
Sia \(z \in G \subseteq \operatorname{Cl}(A)\). Sia \((z_{n})_{n \in \N} \subseteq A\) una \href{20250115100904-successione.org}{successione} \href{20250115100930-convergenza_per_una_successione.org}{convergente} a \(z\) (vedi \href{20250303121451-caratterizzazione_della_chiusura_in_termini_di_successioni.org}{Caratterizzazione della chiusura in termini di successioni})

Si consideri la successione \(\left(f(z_{n})\right)_{n \in \N} \subseteq Z_{2}\). Si dimostra ora che questa sia una successione di \(d'\)-\href{20250303134529-successione_di_cauchy.org}{Cauchy}.

Si fissi \(\varepsilon>0\).
Siccome \(z \in G\) allora \(\operatorname{osc}_{f}(z) = 0\), e dunque esiste \(U\) \href{20250111142313-intorno.org}{intorno} \href{20250103145124-topologia.org}{aperto} di \(z\) tale che\footnote{Dove se \(B\neq \emptyset\):
\begin{equation*}
\operatorname{diam}(B)\coloneqq\set{ \sup{d'(x,y)\ |\ x,y \in B} \in \R}
\end{equation*}
mentre \(\operatorname{diam}(\emptyset) = 0\).}
\begin{equation*}
\operatorname{diam}\left(f(U\cap A)\right)<\varepsilon
\end{equation*}
In particolare, siccome \(z_{n}\to z\), allora \href{20250304141512-proprieta_vere_definitivamente.org}{definitivamente} \(z_{n} \in U\) (ovvero esiste \(N \in \N\) tale che, \(\forall\, n > N\) si ha \(z_{n} \in U\) e in particolare \(z_{n} \in U\cap A\)).

Allora \(\forall\, n>N\) si ha che \(f(z_{n}) \in f(U\cap A)\) e, per definizione di diametro, si ha che \(\forall\, n,m > N\)
\begin{equation*}
d'\left(f(z_{n}), f(z_{m})\right) < \varepsilon
\end{equation*}

Quindi \(\left(f(z_{n})\right)_{n \in \N}\) è di Cauchy e quindi, siccome \(d'\) è una \href{20250301194153-spazio_metrico_completo.org}{metrica completa}, allora \href{20250115100930-convergenza_per_una_successione.org}{converge}.

Si pone quindi
\begin{equation*}
g(z) \coloneqq \lim_{n \to \infty} f(z_{n})
\end{equation*}
\begin{enumerate}
\item Buona definizione
\label{sec:org3ca32c7}

Sia \(z \in G\) e siano \((z_{n})_{n \in \N} \subseteq A\) e \((w_{n})_{n \in \N} \subseteq A\) due successioni convergenti a \(z\). Dobbiamo dimostrare che
\begin{equation*}
\lim_{n\to \infty} f(z_{n}) = \lim_{n\to \infty} f(w_{n})
 \end{equation*}

Dimostriamo che \(\lim_{n\to\infty} d'\left(f(z_{n}), f(w_{n})\right) = 0\). Questo dimostra la tesi\footnote{\href{20250317125810-successioni_in_spazio_metrico_hanno_lo_stesso_limite_sse_limite_delle_distanze_e_nullo.org}{Successione delle distanze tende a zero implica successioni hanno lo stesso limite}}.

Sia \(\varepsilon>0\). Siccome \(f(z_{n})\) e \(f(w_{n})\) sono successioni convergenti, allora esiste \(N \in \N\) tale che, per ogni \(n>N\)
\begin{equation*}
d'\left(g(z), f(z_{n})\right)< \frac{\varepsilon}{2},\qquad d'\left(g(z), f(w_{n})\right) < \frac{\varepsilon}{2}
\end{equation*}
e dunque, per la \href{20250306115949-disuguaglianza_triangolare.org}{disuguaglianza triangolare}
\begin{align*}
d'\left(f(z_{n}), f(w_{n})\right)&\le d'\left(f(z_{n}),g(z)\right)+ d'\left(f(w_{n}), g(z)\right)\\
&< \frac{\varepsilon}{2}+\frac{\varepsilon}{2} = \varepsilon.
\end{align*}
\item Estende \(f\)
\label{sec:org87a14ae}


Sia \(z \in A\), e si consideri la \href{20250115100904-successione.org}{successione} \((z_{n})_{n \in \N} \subseteq A\) tale che, \(\forall\, n \in \N\), \(z_{n}=z\). Questa ovviamente converge a \(z\), e pertanto, per definizione,
\begin{equation*}
g(z) = \lim_{n\to \infty} f(z_{n}) = \lim_{n\to \infty} f(z) = f(z)
\end{equation*}
Quindi \(g\upharpoonright A = f\).
\item Continuità
\label{sec:org6b4804f}
Per dimostrare che \(g\) sia continua in \(G\), \href{20250306135302-oscillazione_di_una_funzione_in_uno_spazio_metrico.org}{dimostriamo} che \(\forall\, z \in G\) si ha \(\operatorname{osc}_{g}(z) = 0\).
Per ogni aperto \(U \subseteq X\), per definizione di \(g\) si ha che
\begin{equation*}
g(G\cap U) \subseteq \operatorname{Cl}\left(f(A\cap U)\right)
\end{equation*}
e \href{20250306161159-disugliaglianze_per_il_diametro_di_un_insieme.org}{pertanto} \(\operatorname{diam}\left(g(G\cap U)\right) \le \operatorname{diam}\left(f(A\cap U)\right)\)

In particolare, per \href{20250306135302-oscillazione_di_una_funzione_in_uno_spazio_metrico.org}{definizione},
\begin{align*}
\operatorname{osc}_{g}(z) &= \inf\set{\operatorname{diam}\left(g(U\cap G)\right)\ |\ U \subseteq X\text{ aperto, } z \in U}\\
&\le \inf\set{\operatorname{diam}\left(f(A\cap U)\right)\ |\ U \subseteq X\text{ aperto, }z \in U}\\
&= \operatorname{osc}_{f}(z)
\end{align*}
Ma \(\operatorname{osc}_{f}(z) = 0\) poiché \(z \in G\), dunque \(\operatorname{osc}_{g}(z)= 0\) e quindi \(g\) continua in \(z\).
\end{enumerate}
\end{document}
