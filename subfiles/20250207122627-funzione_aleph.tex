% Intended LaTeX compiler: pdflatex
\documentclass[../main]{subfiles}


\begin{document}

\section{Funzione Aleph}
\label{sec:org6caa852}
Contesto: \href{20250130104245-morse_kelly_set_theory.org}{Morse Kelly Set Theory}

La \href{20250203161341-cardinali.org}{classe dei cardinali} privata di \href{20250203161110-numeri_naturali_sono_ordinali.org}{\(\omega\)} è una \href{20250203104134-buon_ordine_mk.org}{classe ben ordinata} con l'inclusione ed inoltre è una \href{20250130104320-classe_mk.org}{classe propria}.

\href{20250203133527-insiemi_ben_ordinati_sono_isomorfi_ad_un_ordinale_unico.org}{Dunque} \(\operatorname{Ord}\cong \operatorname{Card}\setminus\omega\) e l'\href{20250203110432-isomorfismo_tra_ordini.org}{isomorfismo} è dato dalla \href{20250202170607-classe_relazione_binaria.org}{classe-funzione} \(\aleph\).
\begin{equation*}
\aleph:\operatorname{Ord}\to\operatorname{Card}\setminus\omega
\end{equation*}
In quanto isomorfismo è \href{20250203132953-funzione_monotona.org}{crescente} e \href{20250203161326-topologia_sugli_ordinali.org}{continua}.


\begin{align*}
\aleph_{0} &=\omega\\
\aleph_{\operatorname{S}(\alpha)} &= (\aleph_{\alpha})^{+}\\
\aleph_{\lambda} &= \sup_{\alpha<\lambda}\aleph_{\alpha}
\end{align*}

Spesso si scrive \(\omega_{\alpha}\) in luogo di \(\aleph_{\alpha}\).
\begin{prop}
Poiché \(\aleph:\operatorname{Ord}\to\operatorname{Ord}\) è crescente e continua, \href{20250207122856-punto_fisso_di_funzioni_continue_e_crescenti_sugli_ordinali.org}{allora} ammette punto fisso, ovvero esistono degli cardinali \(\kappa =\aleph_{\kappa}\).

Il più piccolo di questi cardinali è il supremum di
\begin{equation*}
\aleph_{0},\quad \aleph_{\aleph_{0}},\quad\aleph_{\aleph_{\aleph_{0}}}, \quad \aleph_{\aleph_{\aleph_{\aleph_{0}}}},\quad\dots
\end{equation*}
\end{prop}
\end{document}
