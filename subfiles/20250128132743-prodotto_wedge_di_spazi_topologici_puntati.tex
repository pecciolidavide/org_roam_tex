% Intended LaTeX compiler: pdflatex
\documentclass[../main]{subfiles}


\begin{document}

\section{Prodotto wedge di spazi topologici puntati}
\label{sec:orgd4fcd27}
Siano \((X,x),(Y,y)\) \href{20250103145124-topologia.org}{spazi topologici} \href{20241205115614-categoria_top.org}{puntati} tali che
\begin{itemize}
\item esiste \(U_{x} \subseteq X\) \href{20250111142313-intorno.org}{intorno} di \(x\) tale che \(\set{x}\hookrightarrow U_{x}\) sia un \href{20250129112659-retratto_di_deformazione_forte_di_uno_spazio_topologico.org}{retratto di deformazione forte}
\item esiste \(U_{y} \subseteq Y\) \href{20250111142313-intorno.org}{intorno} di \(y\) tale che \(\set{y}\hookrightarrow U_{y}\) sia un \href{20250129112659-retratto_di_deformazione_forte_di_uno_spazio_topologico.org}{retratto di deformazione forte}
\end{itemize}

\begin{definizione}
Si definisce la \href{20250113110148-relazione_di_equivalenza.org}{relazione di equivalenza} \(\sim\) sull'\href{20250113175700-unione_disgiunta.org}{unione disgiunta} \(X\sqcup Y\):
\begin{equation*}
r\sim s\qquad \iff\qquad \begin{cases}
r=s\\
r=x \,\land\, s = y\\
r=y \,\land\, s=x
\end{cases}
\end{equation*}
e si definsice quindi il prodotto wedge di \(X\) e \(Y\) come il \href{20250114100810-quoziente_rispetto_a_relazione_di_equivalenza.org}{quoziente}:
\begin{equation*}
X\vee Y \coloneqq X\sqcup Y/\sim
\end{equation*}
con la \href{20250129155316-spazio_topologico_quoziente.org}{topologia quoziente indotta dalla proiezione} \(p:X\sqcup Y \to X\vee Y\).
\end{definizione}
\subsection{Omologia singolare del prodotto wedge di spazi topologici puntati}
\label{sec:org173340c}
\begin{prop}
Sia \(R\) un \href{20241219112842-pid.org}{PID}. Per ogni \(q>0\), l'\href{20250122133631-omologia_singolare.org}{omologia singolare} dello spazio \(X\vee Y\) è uguale alla \href{20241213095808-somma_diretta.org}{somma diretta} dell'omologia singolare di \(X\) e \(Y\):
\begin{equation*}
H_{q}(X\vee Y) = H_{q}(X)\oplus H_{q}(Y)
\end{equation*}
\end{prop}

\begin{proof}
Sia \(p=[x]=[y]\), e si definiscano (visti come immagini nel quoziente):
\begin{align*}
Y_{1} &= X\cup U_{y} & X_{1} &= X\\
Y_{2} &= Y\cup U_{x} & X_{2} &= Y\\
Y_{1}\cap Y_{2} &= U_{x}\cup U_{y} & X_{1}\cap X_{2} &= \set{p}
\end{align*}
Allora la quadrupla \((X\vee Y, Y_{1}, Y_{2}, Y_{1}\cap Y_{2})\) soddisfa \href{20250128132648-teorema_di_mayer_vietoris.org}{Mayer-Vietoris}, e
\begin{equation*}
\Id: (X\vee Y, X_{1}, X_{2}, X_{1}\cap X_{2})\longrightarrow (X\vee Y, Y_{1}, Y_{2}, Y_{1}\cap Y_{2})
\end{equation*}
si restringe a tutte \href{20250122155727-retratto_di_deformazione_di_uno_spazio_topologico.org}{retrazioni di deformazione}. Quindi
\begin{equation*}
(X\vee Y, X_{1}, X_{2}, X_{1}\cap X_{2}) = (X\vee Y, X, Y, p)
\end{equation*}
soddisfa Mayer-Vietoris.
\begin{itemize}
\item Per \(q>1\), si ha
\begin{equation*}
  H_{q}(p) \longrightarrow H_{q}(X) \oplus H_{q}(Y) \longrightarrow H_{q}(X\vee Y) \longrightarrow H_{q-1}(p)
\end{equation*}
e, vista l'\href{20250122154153-calcolo_dell_omologia_del_punto.org}{omologia del punto}, \(H_{q}(p) =H_{q-1}(p) = 0\), e \href{20250120130155-caratterizzazione_di_alcune_successioni_esatte_di_r_moduli.org}{quindi}
\begin{equation*}
  H_{q}(X) \oplus H_{q}(Y) \cong H_{q}(X\vee Y).
\end{equation*}
\item Per \(q=1\):
\begin{equation*}
 \scalebox{0.8}{%
 \begin{tikzcd}[ampersand replacement=\&,row sep=small]
         {H_1(p)} \& {H_1(X)\oplus H_1(Y)} \& {H_1(X\vee Y)} \& {H_0(p)} \& {H_0(X)\oplus H_0(Y)} \& {H_0(X\vee Y)} \& 0
         \arrow[from=1-1, to=1-2]
         \arrow[from=1-2, to=1-3]
         \arrow[from=1-3, to=1-4]
         \arrow[from=1-4, to=1-5]
         \arrow[from=1-5, to=1-6]
         \arrow[from=1-6, to=1-7]
 \end{tikzcd}}
\end{equation*}
e ricordando l'omologia del punto, si ottiene:
\begin{equation*}
\scalebox{0.8}{%
\begin{tikzcd}[ampersand replacement=\&,row sep=small]
        \&\&\& \textcolor{rgb,255:red,214;green,92;blue,92}{R} \\
        \textcolor{rgb,255:red,214;green,92;blue,92}{0} \& {H_1(X)\oplus H_1(Y)} \& {H_1(X\vee Y)} \& {H_0(p)} \& {H_0(X)\oplus H_0(Y)} \& {H_0(X\vee Y)} \& 0 \\
        \&\&\& {[p]} \& {\big([p],[p]\big) = \big([x],[y]\big)}
        \arrow[from=2-1, to=2-2]
        \arrow[from=2-2, to=2-3]
        \arrow["\alpha", from=2-3, to=2-4]
        \arrow["\cong"{marking, allow upside down}, color={rgb,255:red,214;green,92;blue,92}, draw=none, from=2-4, to=1-4]
        \arrow["\beta",from=2-4, to=2-5]
        \arrow[from=2-5, to=2-6]
        \arrow[from=2-6, to=2-7]
        \arrow[maps to, from=3-4, to=3-5]
\end{tikzcd}}
\end{equation*}
In particolare, \(\beta\) è iniettiva, e quindi \(\alpha\) è il morfismo nullo. \href{20250120130155-caratterizzazione_di_alcune_successioni_esatte_di_r_moduli.org}{Segue}:
\begin{equation*}
  H_{1}(X)\oplus H_{1}(Y) \cong H_{1}(X\vee Y).%
  \qedhere
\end{equation*}
\end{itemize}
\end{proof}
\end{document}
