% Intended LaTeX compiler: pdflatex
\documentclass[../main]{subfiles}


\begin{document}

\section{Colorazione}
\label{sec:org935cfb4}
\uline{Notazione}:
\begin{itemize}
\item per \(k \in \N\), \((k] \coloneqq \set{1,\dots,k}\)
\item per \(I\) insieme e \(\lambda\) \href{20250203161341-cardinali.org}{cardinale}, si indica con
\begin{equation*}
  \binom{I}{\lambda} = I^{(\lambda)} =  \set{J \subseteq I \mid \card{J} = \lambda}.
\end{equation*}
l'insieme dei \href{20250131155822-operazioni_insiemistiche_tra_classi_mk.org}{sottoinsiemi} di \(I\) di \href{20241213101756-cardinalita.org}{cardinalità} fissata.
\end{itemize}
\subsection{Colorazione di un insieme}
\label{sec:org4b4a615}

\begin{definizione}
Una \uline{colorazione} per un \href{20250130104331-insieme_mk.org}{insieme} \(M\) è una \href{20250202170607-classe_relazione_binaria.org}{funzione}
\begin{equation*}
f: M \to (k]
\end{equation*}
per qualche \href{20250203161110-numeri_naturali_sono_ordinali.org}{\(n,k \in \omega\)} fissato.
\end{definizione}
\subsubsection{Sottoinsieme monocromatico}
\label{sec:orgc17b3b0}
\begin{definizione}
Dato un insieme \(M\) ed una colorazione \(f:M\to (k]\), \(H \subseteq M\) è \uline{monocromatico} se la \href{20250205170515-restrizione_di_una_classe.org}{restrizione} \(f \upharpoonright H\) è costante.
\end{definizione}
\subsection{Colorazione di un grafo}
\label{sec:org7a3bdfa}
\end{document}
