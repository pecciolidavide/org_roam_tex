% Intended LaTeX compiler: pdflatex
\documentclass[../main]{subfiles}


\begin{document}

Sia \(\K\) un \href{20241231112713-campo_algebricamente_chiuso.org}{campo algebricamente chiuso}.
\section{Proposizione}
\label{sec:org5e43357}
Sia \(X \subseteq \mathds{P}^{m}\) una \href{20241231123223-varieta_algebrica_proiettiva.org}{varietà algebrica proiettiva} (vedi \href{20241231115051-spazio_proiettivo.org}{spazio proiettivo}), e sia \(F:X \longrightarrow \mathds{P}^{m}\) un \href{20250104120600-morfismo_tra_varieta_algebriche_proiettive.org}{morfismo}. Indicando con \(\Gamma_{F}\) il suo grafico,
\begin{equation*}
\Gamma_{F}\coloneqq\set{\left(x,F(x)\right) \in X \times \mathds{P}^{m}} \subseteq \mathds{P}^{n}\times \mathds{P}^{m}
\end{equation*}
si ha che
\begin{enumerate}
\item \(\Gamma_{F}\) è una sottovarietà di \(X\times \mathds{P}^{n}\) (vedi \href{20250104190620-mappa_di_segre.org}{Prodotto di spazi proiettivi come varietà proiettiva}, \href{20250108102640-prodotti_di_varieta_qp.org}{Prodotti di varietà QP});
\item \(\Gamma_{F}\cong X\) \href{20250104120600-morfismo_tra_varieta_algebriche_proiettive.org}{isomorfismo}.
\end{enumerate}
\subsection{Dimostrazione}
\label{sec:org7d6bc23}

\subsubsection{Sottovarietà}
\label{sec:org464acf7}
Sia \(\mathcal{U}=\set{U_{\alpha}}_{\alpha}\) un \href{20250103164252-ricoprimento.org}{ricoprimento} \href{20250103145124-topologia.org}{aperto} di \(X\) tale che \(\restriction{F}{U_{\alpha}}\) è polinomiale per ogni \(\alpha\).

Sia \(W_{\alpha}\coloneqq U_{\alpha}\times \mathds{P}^{m}\). \(W_{\alpha}\) è aperto.\footnote{Infatti, se \(U_{\alpha}\) è aperto, allora esistono \(G_{1},\dots,G_{k} \in \K[x_{0},\dots,x_{n}]\) tali che
\begin{equation*}
U_{\alpha}^{C}=V(G_{1},\dots,G_{k})
\end{equation*}
Ma considerando \(G_{1},\dots,G_{k} \in \K[x_{0},\dots,x_{n}; y_{0},\dots,y_{n}]\) (vedi \href{20241219113434-anello_dei_polinomi.org}{Anello-dei-polinomi}) si ha che
\begin{equation*}
V(G_{1},\dots,G_{k}) = U_{\alpha}^{C}\times \mathds{P}^{m} = (U_{\alpha}\times \mathds{P}^{m})^{C}
\end{equation*}
(vedi \href{20250108102640-prodotti_di_varieta_qp.org}{Prodotti di varietà QP}) e dunque
\begin{equation*}
W_{\alpha}^{C} = V(G_{1},\dots,G_{k})
\end{equation*}}
Quindi \(\mathcal{W}\coloneqq\set{W_{\alpha}}_{\alpha}\) è un ricoprimento aperto di \(X\times \mathds{P}^{m}\).

Sia ora \(x \in U_{\alpha}\), e sia \(F=[F_{0}:\dots:F_{m}]\) la scrittura polinomiale di \(F\) su \(U_{\alpha}\).
\((x,P) \in \Gamma_{F}\cap W_{\alpha}\) se e solo se \(P=F(x)\), ovvero se e solo se, posto \(P=[P_{0}:\dots:P_{m}]\)
\begin{equation*}
\rank\begin{bmatrix}
P_{0} & P_{1} & \dots & P_{m}\\
F_{0}(x) & F_{1}(x) & \dots & F_{m}(x)
\end{bmatrix}=1
\end{equation*}
(vedi \href{20250104170945-rango_di_una_matrice.org}{Rango di una matrice}). Questo equivalente a richiedere che tutti i \href{20250104171415-minori_di_una_matrice.org}{minori} \(2\times 2\) abbiano \href{20250104111751-determinante_di_una_matrice.org}{determinante} nullo. Dunque
\begin{equation*}
\Gamma_{F}\cap W_{\alpha} = V\left(y_{i}F_{j}(x_{0},\dots,x_{n})-y_{j}F_{i}(x_{0},\dots,x_{n})\right)
\end{equation*}
Questi sono tutti polinomi \href{20250107172231-polinomi_biomogenei.org}{biomogenei} di \href{20250107172249-bigrado_di_un_polinomio.org}{bigrado} \((\deg F, 2)\) (vedi \href{20241231124742-grado_polinomi.org}{grado di un polinomio}), siccome tuttele \(F_{i}\) hanno lo stesso grado per definizione di \href{20250104120600-morfismo_tra_varieta_algebriche_proiettive.org}{morfismo proiettivo}.

Siccome per ogni \(\alpha\) \(\Gamma_{F}\cap W_{\alpha}\) è chiuso, \href{20250108160159-intersezione_con_un_ricoprimento_aperto_e_chiusa_allora_chiuso.org}{allora} \(\Gamma_{F}\) è chiuso, ovvero è una varietà proiettiva.
\subsubsection{Isomorfismo}
\label{sec:org6bb0952}
Dal momento che è varietà, basta considerare i morfismi:
$\backslash$[
\begin{tikzcd}[ampersand replacement=\&,column sep=large,row sep=tiny]
	X \& {\Gamma_F} \& X \\
	x \& {\left(x,F(x)\right)} \& x
	\arrow["{\tilde{F}}", from=1-1, to=1-2]
	\arrow["{\pi_1}", from=1-2, to=1-3]
	\arrow[maps to, from=2-1, to=2-2]
	\arrow[maps to, from=2-2, to=2-3]
\end{tikzcd}
$\backslash$]
\end{document}
