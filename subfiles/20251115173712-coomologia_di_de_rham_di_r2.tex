% Intended LaTeX compiler: pdflatex
\documentclass[../main]{subfiles}


\begin{document}

\section{Coomologia di De Rham di \(\R^{n}\).}
\label{sec:orgfcd169d}
\def\d{\operatorname{d}}
\def\H{\operatorname{H}_{\text{dR}}}

\begin{thm}
(Lemma di Poincaré)
Sia \(\H^{k}(\R^{n})\) la \href{20251115172442-gruppo_di_coomologia_di_de_rham.org}{coomologia di De Rham} di \(\R^{n}\)\footnote{Ovviamente \(\R^{n}\) ha una struttura di \href{20250113115909-struttura_differenziabile.org}{varietà differenziabile}.}, per \(n>1\). Allora
\begin{equation*}
\H^{k}(\R^{n}) = %
\begin{cases}
\R & k=0\\
O & k\neq 0
\end{cases}
\end{equation*}
dove \(O\) è lo \href{20241205142027-spazio_vettoriale.org}{spazio vettoriale banale}.
\end{thm}
\end{document}
