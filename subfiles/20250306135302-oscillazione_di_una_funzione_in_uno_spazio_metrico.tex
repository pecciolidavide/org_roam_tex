% Intended LaTeX compiler: pdflatex
\documentclass[../main]{subfiles}

\usepackage[hyperref]{biblatex}
\date{}
\title{}
\begin{document}

\section{Oscillazione di una funzione in uno spazio metrico}
\label{sec:org3aca91c}
Sia \(Z_{1}\) uno \href{20250103145124-topologia.org}{spazio topologico}, e sia \(Z_{2} = (Z_{2},d')\) uno \href{20250301193511-spazio_metrico.org}{spazio metrico}.
Sia \(A \subseteq Z_{1}\) e sia \(f:A\to Z_{2}\) una \href{20250103103252-funzione_continua.org}{funzione continua}.
\subsection{Definizione}
\label{sec:org82e7d62}
Per ogni \(z \in Z_{1}\) si definisce l'\texttt{oscillazione} di \(f\) in \(z\) come
\begin{equation*}
\operatorname{osc}_{f}(z) \coloneqq \inf\set{\operatorname{diam}\left(f(U\cap A)\right)\ |\ U \subseteq Z_{1}\text{ aperto, }z \in  U}
\end{equation*}
dove, se \(B\neq \emptyset\):
\begin{equation*}
\operatorname{diam}(B)\coloneqq \sup{d'(x,y)\ |\ x,y \in B} \in \R
\end{equation*}
mentre \(\operatorname{diam}(\emptyset) = 0\).
\subsection{Insieme dei punti in cui l'oscillazione di una funzione si annulla}
\label{sec:org1d7f1f9}
\subsubsection{Continuità}
\label{sec:org2c59538}
Se \(z \in Z_{1}\setminus \operatorname{Cl}(A)\) (vedi \href{20250103144944-chiusura_topologica.org}{Chiusura Topologica}), allora \(\operatorname{osc}_{f}(z) = 0\).

Se \(z \in A\) allora
\begin{equation*}
\operatorname{osc}_{f}(z) = 0\qquad \iff\qquad f\text{ è continua in }z
\end{equation*}
(vedi \href{20250103103252-funzione_continua.org}{Funzione Continua} e \href{20250306140014-funzione_continua_in_un_punto.org}{Funzione Continua in un punto})
\subsubsection{Insieme \(G_{\delta}\)}
\label{sec:org532c695}
È possibile dimostrare che per ogni \(\varepsilon>0\), gli insiemi
\begin{equation*}
A_{\varepsilon}\coloneqq \set{z \in Z_{1}\ |\ \operatorname{osc}_{f}(z)<\varepsilon}
\end{equation*}
sono aperti, e pertanto
\begin{equation*}
\set{z \in Z_{1}\ |\ \operatorname{osc}_{f}(z) =0} = \bigcap_{n \in \N} A_{2^{-n}}
\end{equation*}
è un \href{20250304152026-sottoinsiemi_gdelta_e_fsigma.org}{insieme \(\bm{G}_{\delta}\)}.
\subsubsection{L'insieme dei punti di continuità di una funzione continua in uno spazio metrizzabile è Gdelta}
\label{sec:orge3f49b2}
Siano \(X,Y\) \href{20250103145124-topologia.org}{spazi topologici}, \(Y\) \href{20250301193401-spazio_topologico_metrizzabile.org}{metrizzabile}, e sia \(f:X\to Y\) una \href{20250202170607-classe_relazione_binaria.org}{funzione}.
L'insieme dei punti in \(X\) su cui \(f\) \href{20250306140014-funzione_continua_in_un_punto.org}{è continua} è un \href{20250304152026-sottoinsiemi_gdelta_e_fsigma.org}{insieme \(\bm{G}_{\delta}\)}.
\end{document}
