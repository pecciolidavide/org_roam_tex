% Intended LaTeX compiler: pdflatex
\documentclass[../main]{subfiles}


\begin{document}

\section{Morfismo tra complessi di cocatene induce morfismo in coomologia}
\label{sec:orgf154edc}
Siano
\begin{equation*}
C^{\bullet} \coloneqq \left\langle (C^{k}, \mathrm{d}_{C}^{k})\right\rangle_{k \in \Z}, \qquad %
D^{\bullet} \coloneqq \left\langle (D^{k}, \mathrm{d}_{D}^{k})\right\rangle_{k \in \Z}
\end{equation*}
due \href{20251115182320-complesso_di_cocatene.org}{complessi di cocatene} e sia \(F^{\bullet} : C^{\bullet} \to D^{\bullet}\) un \href{20251115182606-morfismo_tra_complessi_di_cocatene.org}{morfismo}, \(F^{\bullet} = \langle F^{k}\rangle_{k \in \Z}\).

Allora, per ogni \(k \in \Z\), \(F^{\bullet}\) induce un morfismo tra i \href{20251115182537-coomologia_di_un_complesso_di_cocatene.org}{gruppi di coomologia}:
\begin{align*}
F^{*} : \operatorname{H}^{k}(C^{\bullet}) &\longrightarrow \operatorname{H}^{k}(D^{\bullet})\\
[c] &\longmapsto \big[F^{k}c\big]
\end{align*}
\end{document}
