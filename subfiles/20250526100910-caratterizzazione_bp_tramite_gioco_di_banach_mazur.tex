% Intended LaTeX compiler: pdflatex
\documentclass[../main]{subfiles}


\begin{document}

\section{Lemma}
\label{sec:org3ad28b6}

Sia \(X\) uno \href{20250103145124-topologia.org}{spazio topologico} \href{20250514174255-gioco_di_choquet.org}{di Choquet} non \href{20250131161811-insieme_vuoto_mk.org}{vuoto} tale che esista una \href{20250301193511-spazio_metrico.org}{distanza} \(d\) su \(X\) le cui \href{20250301193511-spazio_metrico.org}{palle aperte} sono aperti di \(X\). Sia \(A \subseteq X\).

Se per ogni aperto \(U \subseteq X\) il \href{20250513111844-gioco_di_banach_mazur.org}{gioco} \(G^{**}\left((X\setminus A)\cup U\right)\) è \href{20250513155732-logic_game.org}{determinato} allora \(A \subseteq X\) ha \href{20250514154039-proprieta_di_baire.org}{BP}.
\subsection{Dimostrazione}
\label{sec:org3bdb568}

Sia \(A \subseteq X\). Si definisce l'aperto
\begin{equation*}
U(A) \coloneqq \bigcup \set{
U\subseteq X\text{ aperto}\mid U\setminus A\text{ è magro}
}.
\end{equation*}
Allora \(U(A)\setminus A\) è magro e inoltre, se \(A\) ha la BP, allora \(A\mathrel{=^{*}} U(A)\). Questo segue direttamente dal Teorema 8.29 del Kechris.

In particolare quindi il gioco è determinato  per
\begin{equation*}
G^{**}\left((X\setminus A)\cup U(A)\right).
\end{equation*}

Necessariamente è il giocatore II a vincere questo gioco. Infatti, si supponga per assurdo che I abbia una strategia vincente. Allora, per il \href{20250514174717-teorema_di_caratterizzazione_dei_comagri_tramite_il_gioco_di_banach_mazur.org}{Teorema II} \((X\setminus A)\cup U(A)\) è magro in un aperto non vuoto \(B\). In particolare, quindi \(U(A)\) è magro in \(B\), ovvero \(U(A)\cap B\) è magro in \(B\).
\begin{itemize}
\item Se \(U(A)\cap B\neq\emptyset\), siccome \(B \subseteq X\) è un aperto di uno spazio di Baire, allora è uno spazio di Baire; inoltre \(U(A)\cap B\) è un aperto non vuoto di \(B\), quindi è \uline{non magro}. Assurdo.
\item Se invece \(U(A)\cap B = \emptyset\), si consideri il seguente insieme, magro per definizione:
\begin{equation*}
  \left((X\setminus A)\cup U(A)\right) \cap B = \left((X\setminus A)\cap B\right) \cup \left(U(A)\cap B\right) = (X\setminus A)\cap B = B\setminus A
\end{equation*}
Allora, per definizione di \(U(A)\), \(B \subseteq U(A)\). Assurdo.
\end{itemize}

Pertanto, per il \href{20250514174717-teorema_di_caratterizzazione_dei_comagri_tramite_il_gioco_di_banach_mazur.org}{Teorema I}, \((X\setminus A)\cup U(A)\) è comagro. Ma
\begin{equation*}
(X\setminus A)\cup U(A) = X \setminus\left(A\setminus U(A)\right)
\end{equation*}
e pertanto \(A\setminus U(A)\) è magro. Per il risultato precedente \(U(A)\setminus A\) è magro, e dunque
\begin{equation*}
A\mathrel{\triangle}U(A)
\end{equation*}
è magro, ovvero \(A\) ha la BP.\qed
\end{document}
