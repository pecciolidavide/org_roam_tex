% Intended LaTeX compiler: pdflatex
\documentclass[../main]{subfiles}


\begin{document}

\section{Proposizione}
\label{sec:org3e9f679}
Siano \(M, N\) due \href{20250113115909-struttura_differenziabile.org}{varietà differenziabili}, e sia \(F: M\longrightarrow N\) una \href{20250113144722-funzioni_cinfinito_tra_varieta_differenziabili.org}{funzione \(C^{\infty}\).}
Sia \((U,\varphi)\) una \href{20250111092123-varieta_topologica.org}{carta} di \(M\), \((V,\psi)\) una \href{20250111092123-varieta_topologica.org}{carta} di \(N\) tali che \(F(U)\cap V\neq \emptyset\).

Allora \(\psi\circ F\circ \varphi^{-1}\) è \href{20250113125602-classe_c_di_una_funzione.org}{una funzione \(C^{\infty}\) (tra aperti di \(\R^{n}/\R^{m}\))}.
$\backslash$[
\begin{tikzcd}[ampersand replacement=\&]
	\&\& {U\subseteq M} \& {N\supseteq V} \\
	{\R^m\supseteq\varphi(U)} \&\&\&\&\& {\psi(V)\subseteq R^n}
	\arrow["F", from=1-3, to=1-4]
	\arrow["\varphi"', from=1-3, to=2-1]
	\arrow["\psi", from=1-4, to=2-6]
	\arrow["{\psi\circ F\circ\varphi^{-1}}"', from=2-1, to=2-6]
\end{tikzcd}
$\backslash$]
\subsection{{\bfseries\sffamily TODO} Dimostrazione\hfill{}\textsc{matematica\_lm:geo\_diff}}
\label{sec:org8a3fddf}
\end{document}
