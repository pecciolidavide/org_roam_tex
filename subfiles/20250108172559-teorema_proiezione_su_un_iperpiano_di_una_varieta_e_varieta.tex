% Intended LaTeX compiler: pdflatex
\documentclass[../main]{subfiles}


\begin{document}

Sia \(\K\) un \href{20241231112713-campo_algebricamente_chiuso.org}{campo algebricamente chiuso}, sia \(p \in \mathds{P}^{n}\) fissato (vedi \href{20241231115051-spazio_proiettivo.org}{Spazio Proiettivo}), \(H \subseteq \mathds{P}^{n}\) un \href{20250102144306-iperpiano_proeittivo.org}{iperpiano}, e sia
\begin{equation*}
\pi_{p}:\mathds{P}^{n}\setminus\set{p}\longrightarrow H
\end{equation*}
la \href{20250108162146-proiezione_su_un_iperpiano_dentro_allo_spazio_proiettivo.org}{proiezione} da \(p\) su \(H\).
\section{Teorema}
\label{sec:orga1df582}
Sia \(X \subseteq \mathds{P}^{n}\) una \href{20241231123223-varieta_algebrica_proiettiva.org}{sottovarietà algebrica proiettiva}. Allora \(\pi_{p}(X)\) è una sottovarietà di \(H\).
\subsection{Lemma}
\label{sec:org2d77d6c}
Sia \(X \subseteq \mathds{P}^{n}\) una \href{20241231123223-varieta_algebrica_proiettiva.org}{sottovarietà algebrica proiettiva}, \(p \notin X\) e \(\ell\) una qualsiasi \href{20250109101841-retta_proiettiva.org}{retta} per \(p\).

\(\ell\cap X \neq \emptyset\) se e solo se per ogni \(F,G \in I(X)^{h}\) (vedi \href{20250103144124-ideale_di_un_sottoinsieme.org}{Ideale di un sottoinsieme} e \href{20241231121125-polinomi_omogenei.org}{Polinomi Omogenei}), \(F\) e \(G\) hanno uno zero in comune su \(\ell\setminus\set{p}\).
\subsubsection{Dimostrazione del lemma}
\label{sec:org8ff3982}

\paragraph{Implicazione ->}
\label{sec:org86cfdbb}
Se \(r \in \ell \cap X\) allora \(r \in \ell\) e \(r \neq p\), siccome \(p\notin X\). Siccome \(r \in X\), allora \(\forall\, F \in I(X)^{h}\) si ha che \(F(r)=0\). Pertanto \(r \in \ell\setminus\set{p}\) è radice comune di tutti i polinomi di \(I(X)^{h}\).
\paragraph{Implicazione <-}
\label{sec:orgde19c85}
Per assurdo, supponiamo ogni coppia di polinomi \(F,G \in I(X)^{h}\) abbia uno zero in comune su \(\ell\setminus\set{p}\) e che \(\ell \cap X = \emptyset\).

Sia \(F \in I(X)\) tale che \(F(p) \neq 0\). Tale polinomio esiste poiché \(p\notin X\). Infatti, \href{20250109115213-zeri_di_un_ideale_di_un_sottoinsieme_proiettivo.org}{si ha che} \(X=V\left(I(X)^{h}\right)\).

Siano \(p_{1},\dots,p_{k}\) i punti di \(\ell\) nei quali si annulla \(F\), \(p_{1},\dots,p_{k}\notin X\) (esistono poiché per ipotesi ogni coppia di polinomi ha uno zero in comune su \(\ell\setminus\set{p}\), e pertanto ogni polinomio deve avere almeno uno zero in \(\ell\setminus\set{p}\)).
Per ogni \(i=1,\dots,k\) sia \(G_{i} \in I(X)^{h}\) un polinomio tale che \(G_{i}(p_{i})\neq 0\). Tali polinomi esistono perché \(p_{i}\notin X\).

A meno di moltiplicare per coordinate non nulle in \(p_{i}\), possiamo supporre che i \(G_{i}\) siano tutti dello stesso grado. Pertanto
\begin{equation*}
G\coloneqq \sum_{i=1}^{k} \lambda_{i} G_{i},\qquad \lambda_{i} \in \K\setminus\set{0}
\end{equation*}
è un polinomio omogeneo che non si annulla in nessuno degli \(p_{i}\). Pertanto \(F\) e \(G\) non hanno zeri comuni su \(\ell\setminus\set{p}\). Assurdo.
\subsection{Dimostrazione del teorema}
\label{sec:orge8f3a42}
Scegliendo opportunamente le coordinate, sia
\begin{equation*}
p=[0:\dots:0:1], \qquad H=V(x_{n})
\end{equation*}
(vedi \href{20241231112823-radici_polinomiali.org}{Luogo di zeri}).

Sia \(q \in H\), \(q=[q_{0}:\dots:q_{n-1}:0]\).

\(q \in \pi_{p}(X)\) se e solo se la \href{20250109101841-retta_proiettiva.org}{retta} \(\ell_{pq}\) che passa per \(p\) e per \(q\) interseca \(X\), ovvero se e solo se
\begin{equation*}
\ell_{pq}\cap X \neq \emptyset
\end{equation*}
Per il lemma, questo è vero se e solo se per ogni \(F,G \in I(X)^{h}\) esiste \(r \in \ell_{pq}\setminus\set{p}\) tali che \(F(r)=G(r)=0\).

È possibile parametrizzare la retta \(\ell_{pq}\setminus\set{p}\) con un unico parametro,
\begin{equation*}
\ell: q+t\ p
\end{equation*}
ovvero \(\ell(t): [q_{0}:\dots:q_{n-1}: t]\). Dunque \(F\) e \(G\) hanno uno zero in comune su \(\ell_{pq}\), per definizione, sse esiste \(t_{0}\) tale che
\begin{equation*}
F\left(\ell(t_{0})\right)=G\left(\ell(t_{0})\right)
\end{equation*}
\(F\circ \ell\) e \(G\circ \ell\) sono polinomi di \(\K[t]\). Questi polinomi hanno uno zero comune se e solo se hanno un \href{20250108173845-fattore_comune.org}{fattore} non costante in comune (poiché \(\K\) algebricamente chiuso), se e solo se (per il \href{20250108173056-fattori_non_costanti_comuni_tra_polinomi.org}{lemma}, vedi \href{20250108173056-fattori_non_costanti_comuni_tra_polinomi.org}{Risultante})
\begin{equation*}
\operatorname{Res}(F\circ \ell, G\circ \ell) = 0
\end{equation*}

Ma per costruzione \(\operatorname{Res}(F\circ\ell, G\circ \ell)= \operatorname{Res}_{x_{n}}(F,G) (q)\) (vedi \href{20250109095734-risultante_per_polinomi_in_piu_variabili.org}{Risultante per polinomi in più variabili}), e pertanto
\begin{equation*}
q \in \pi_{p}(X)\quad\iff\quad \operatorname{Res}_{x_{n}}(F,G)(q) = 0
\end{equation*}
Siccome \(\operatorname{Res}_{x_{n}}(F,G)\) è un \href{20241231121125-polinomi_omogenei.org}{polinomio omogeneo}, allora
\begin{equation*}
\pi_{p}(X) = V\left(\set{\operatorname{Res}_{x_{n}}(F,G): F,G \in I(X)^{h}}\right)
\end{equation*}
\end{document}
