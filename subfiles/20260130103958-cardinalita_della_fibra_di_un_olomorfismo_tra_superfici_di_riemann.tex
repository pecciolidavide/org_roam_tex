% Intended LaTeX compiler: pdflatex
\documentclass[../main]{subfiles}

\def\mult#1{\mathrm{mult}_{#1}\,}


\begin{document}

\section{Cardinalità della fibra di un olomorfismo tra superfici di Riemann}
\label{sec:org81b8fc0}
Siano \(X,Y\) \href{20260127112828-superficie_di_riemann.org}{superfici di Riemann} \href{20250103163701-spazio_topologico_compatto.org}{compatte}, e sia \(F:X\to Y\) \href{20260128143822-funzione_olomorfa_su_una_superficie_di_riemann.org}{olomorfismo} non costante. Sia \(d\coloneqq \deg F\) il \href{20260129171444-teorema_del_grado_per_olomorfismi_tra_superfici_di_riemannn.org}{grado di \(F\)}.

\begin{prop}
Per ogni \(y \in Y\), la \href{20241213101756-cardinalita.org}{cardinalità} della \href{20260128184014-fibra_di_una_funzione.org}{fibra} di \(y\) è:
\begin{itemize}
\item se \(y\) \textbf{non è} di \href{20260129104340-punto_di_ramificazione_per_una_funzione_tra_superfici_di_riemann.org}{diramazione}, \(\card{F^{-1}(y)} = d\);
\item se \(y\) è di \href{20260129104340-punto_di_ramificazione_per_una_funzione_tra_superfici_di_riemann.org}{diramazione}, \(\card{F^{-1}(y)} < d\).
\end{itemize}
\end{prop}
\begin{proof}
Poiché le due superfici sono compatte, allora \(F\) ha un numero finito di punti di diramazione.

\begin{itemize}
\item Se \(y\) non è di diramazione, significa che per ogni \(p \in F^{-1}(y)\), \(\mult{p} F = 1\), e pertanto la tesi.
\item Se \(y\) è di diramazione, significa che esiste \(q \in F^{-1}(y)\) tale che \(\mult{q} F > 1\), e pertanto la tesi.
\qedhere
\end{itemize}
\end{proof}
\end{document}
