% Intended LaTeX compiler: pdflatex
\documentclass[../main]{subfiles}


\begin{document}

\section{Copie dello spazio proiettivo dentro la varietà di Segre sono lineari}
\label{sec:orgb95d07d}
Sia \(\K\) un \href{20241231112713-campo_algebricamente_chiuso.org}{campo algebricamente chiuso}.
\subsection{Proposizione}
\label{sec:orgf534a3e}
Sia \(\sigma:\mathds{P}^{n}\times \mathds{P}^{m} \longrightarrow \Sigma_{n,m}\) la \href{20250104190620-mappa_di_segre.org}{mappa di Segre}. (vedi \href{20241231115051-spazio_proiettivo.org}{Spazio Proiettivo}).
Allora, per \(p \in \mathds{P}^{n}\) e \(q \in \mathds{P}^{m}\), gli spazi
\begin{equation*}
\sigma\left(\set{p}\times \mathds{P}^{m}\right),\qquad \sigma\left(\mathds{P}^{n}\times\set{q}\right)
\end{equation*}
sono descritti da equazioni di primo grado.
\subsubsection{Dimostrazione}
\label{sec:org07d8dae}
Fissiamo \(p \in \mathds{P}^{n}\), \(p=[p_{0}:\dots:p_{n}]\).

Si ha che
\begin{equation*}
z=[\dots:z_{ij}:\dots] \in \sigma\left(\set{p}\times \mathds{P}^{m}\right)
\end{equation*}
se e solo se le colonne sono proporzionali a \(p\). Pertanto, se e solo se
\begin{equation*}
\forall\, j,\quad \rank\begin{bmatrix}
z_{0j} & p_{0}\\
z_{1j} & p_{1}\\
\vdots\\
z_{nj} & p_{n}
\end{bmatrix}=1
\end{equation*}
(vedi \href{20250104170945-rango_di_una_matrice.org}{Rango di una matrice}). Come già osservato, questo succede se e solo se i \href{20250104111751-determinante_di_una_matrice.org}{determinanti} dei \href{20250104171415-minori_di_una_matrice.org}{minori} della \href{20250104111539-spazio_delle_matrici.org}{matrice} sono tutti nulli. Queste condizioni sono tutte polinomiali di primo grado.
\end{document}
