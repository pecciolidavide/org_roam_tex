% Intended LaTeX compiler: pdflatex
\documentclass[../main]{subfiles}


\begin{document}

\begin{thm}
Sia \(K \subseteq \R^{n}\) \href{20250103163701-spazio_topologico_compatto.org}{compatto}, e sia \(\mathcal{A} \subseteq C(K)\)\footnote{Vedi ``\href{20250113125602-classe_c_di_una_funzione.org}{Classe C di una funzione}''\label{org6caba49}} una \href{20250629165520-algebra_di_funzioni_reali.org}{\(\R\)-algebra}. Se
\begin{enumerate}
\item \(\mathcal{A}\) separa i punti di \(K\);
\item \(\mathcal{A}\) contiene le funzioni costanti;
\end{enumerate}
allora \(\mathcal{A}\) è un \href{20250301193045-sottoinsieme_denso.org}{sottoinsieme denso} di \(C(K)\) munito della \href{20250103145124-topologia.org}{topologia} \href{20250301193530-topologia_indotta_da_una_distanza.org}{indotta dalla metrica}\footnote{Tale massimo esiste per il \href{20250627153319-teorema_di_weierstrass.org}{Teorema di Weierstrass}\label{org536e03e}}:
\begin{equation*}
\forall f,g \in C(K):\quad d(f,g) \coloneqq \max_{x \in K} |f(x)-g(x)|.
\end{equation*}
\end{thm}
\section{Corollari del teorema}
\label{sec:orgd5b90a6}

\begin{prop}
Sia \([a,b] \subseteq \R\), e si consideri \(C([a,b])\)\textsuperscript{\ref{org6caba49}} munito della \href{20250103145124-topologia.org}{topologia} \href{20250301193530-topologia_indotta_da_una_distanza.org}{indotta dalla metrica}\textsuperscript{\ref{org536e03e}}:
\begin{equation*}
\forall f,g \in C(K):\quad d(f,g) \coloneqq \max_{x \in [a,b]} |f(x)-g(x)|.
\end{equation*}
Sia:
\begin{equation*}
\mathcal{A} \coloneqq \set{\restriction{f}{[a,b]}\mid f(x) = \sum_{k=0}^{n} a_{k}x^{k}; a_{k} \in \R, n \in \N}.
\end{equation*}

Allora \(\mathcal{A}\) è \href{20250301193045-sottoinsieme_denso.org}{denso} in \(C([a,b])\), ovvero: per ogni \(f:[a,b]\to \R\) \href{20250103103252-funzione_continua.org}{continua} e per ogni \(\varepsilon>0\) esiste \(g \in \mathcal{A}\) tale che
\begin{equation*}
\max_{x \in [a,b]} |f(x)-g(x)|<\varepsilon
\end{equation*}
\end{prop}

\begin{proof}
\(\mathcal{A}\) è una \href{20250629165520-algebra_di_funzioni_reali.org}{\(\R\)-algebra} che \href{20250629151420-algebra_di_funzioni_separa_i_punti.org}{separa i punti} e contiene le funzioni costanti.
\end{proof}
\begin{prop}
Sia \(f:[a,b]\times[c,d]\to \R\) una funzione continua. Allora, per ogni \(\varepsilon>0\) esiste \(N\ge 1\) ed esistono, per ogni \(i=1,\dots,N\), delle \(g_{i} \in C([a,b])\)\textsuperscript{\ref{org6caba49}} e \(h_{i} \in C([c,d])\) tali che
\begin{equation*}
\max_{\substack{
x \in [a,b]\\
y \in [c,d]}}
\bigg\lvert
f(x,y)-\sum_{i=1}^{N}g_{i}(x)h_{i}(y)
\bigg\rvert<\varepsilon
\end{equation*}
\end{prop}
\begin{proof}
Si consideri
\begin{equation*}
\mathcal{A}\coloneqq \set{
G(x,y) = \sum_{i=1}^{N} g_{i}(x)h_{i}(y)\mid g_{i} \in C([a,b]), h_{i} \in C([c,d]), N = 1,2,\dots
}
\end{equation*}
Si osserva facilmente che \(\mathcal{A}\) è chiuso per somme, prodotti e prodotti per scalari, ovvero è una \href{20250629165520-algebra_di_funzioni_reali.org}{\(\R\)-algebra}. Inoltre contiene le funzioni costanti.

Se \((x_{0},y_{0})\neq (x_{1},y_{1})\):
\begin{itemize}
\item se \(x_{0}\neq x_{1}\) allora \(G(x,y) = x\cdot 1 \in \mathcal{A}\) \href{20250629151420-algebra_di_funzioni_separa_i_punti.org}{separa i due punti};
\item se \((y_{0}\neq y_{1})\) allora \(G(x,y) = 1\cdot y \in \mathcal{A}\) \href{20250629151420-algebra_di_funzioni_separa_i_punti.org}{separa i due punti}.\qedhere
\end{itemize}
\end{proof}
\end{document}
