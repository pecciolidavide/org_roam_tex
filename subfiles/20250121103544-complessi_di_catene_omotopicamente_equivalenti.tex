% Intended LaTeX compiler: pdflatex
\documentclass[../main]{subfiles}

\usepackage[hyperref]{biblatex}
\date{}
\title{}
\begin{document}

\section{Complessi di catene omotopicamente equivalenti}
\label{sec:orgb2089b5}
Sia \(R\) un \href{20241205141119-anello.org}{anello} commutativo con unità.

\begin{definizione}
Siano \(\mathcal{C}_{\bullet},\mathcal{D}_{\bullet}\) due \href{20250120163114-complesso_di_catene.org}{complessi di catene di \(R\)-moduli}. Questi si dicono \textbf{omotopicamente equivalenti}, e si scrive \(\mathcal{C}_{\bullet} \sim \mathcal{D}_{\bullet}\) se esistono \href{20250120163759-categoria_complessi_di_catene.org}{morfismi}
\begin{equation*}
f_{\bullet}: \mathcal{C}_{\bullet}\to \mathcal{D}_{\bullet},\qquad g_{\bullet}: \mathcal{D}_{\bullet}\to \mathcal{C}_{\bullet}
\end{equation*}
tali che la loro composizione sia \href{20250121094935-omotopia_tra_morfismi_di_complessi_di_catene.org}{omotopa} all'identità, ovvero
\begin{equation*}
f_{\bullet}\circ g_{\bullet} \sim \id_{\mathcal{D}_{\bullet}},\qquad g_{\bullet}\circ f_{\bullet}\sim \id_{\mathcal{C}_{\bullet}}
\end{equation*}
\end{definizione}
\begin{prop}
Se \(\mathcal{C}_{\bullet}\sim \mathcal{D}_{\bullet}\) allora i loro \href{20250120164857-modulo_di_omologia_dei_complessi_di_catene.org}{moduli di omologia} sono \href{20241206115416-morfismi_r_moduli.org}{isomorfi}:
\begin{equation*}
\forall n:\qquad
H_{n}(\mathcal{C}_{\bullet})\cong H_{n}(\mathcal{D}_{\bullet})
\end{equation*}
\end{prop}
\begin{proof}
Per definizione esistono:
\begin{tikzcd}
	{\mathcal{C}_\bullet} & {\mathcal{D}_\bullet}
	\arrow["{f_\bullet}", shift left, from=1-1, to=1-2]
	\arrow["{g_\bullet}", shift left, from=1-2, to=1-1]
\end{tikzcd}
e, applicando il \href{20250120165029-funtore_tra_chr_e_rmod.org}{funtore di omologia}, si ottiene
\begin{equation*}
H_{n}(f_{\bullet}\circ g_{\bullet})  = H_{n}(\Id_{\mathcal{D}_{\bullet}}) = \Id_{H_{n}(\mathcal{D}_{\bullet})}
\end{equation*}
ovvero \(H_{n}(f_{\bullet})\) invertibile e quindi \href{20241206115416-morfismi_r_moduli.org}{isomorfismo}.
\end{proof}
\end{document}
