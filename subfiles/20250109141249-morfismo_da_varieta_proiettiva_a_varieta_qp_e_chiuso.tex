% Intended LaTeX compiler: pdflatex
\documentclass[../main]{subfiles}


\begin{document}

\section{Morfismo da varietà proiettiva a varietà qp è chiuso}
\label{sec:orgc787e0d}
Sia \(\K\) un \href{20241231112713-campo_algebricamente_chiuso.org}{campo algebricamente chiuso}. e sia \(\mathds{P}^{n}\) \href{20241231115051-spazio_proiettivo.org}{spazio proiettivo}.
\subsection{Teorema 1}
\label{sec:org4719a9e}
Sia \(X\) una \href{20241231123223-varieta_algebrica_proiettiva.org}{varietà proiettiva}, \(Y\) una \href{20250107112123-varieta_algebrica_quasi_proiettiva_qp.org}{varietà quasi proiettiva}. Se \(f:X \longrightarrow Y\) è un \href{20250107112412-morfismo_tra_varieta_algebriche_qp.org}{morfismo}, allora \(f\) è \href{20250104114559-funzione_chiusa.org}{chiusa}.
\subsection{Teorema 2}
\label{sec:org642a578}
Sia \(X\) una \href{20241231123223-varieta_algebrica_proiettiva.org}{varietà proiettiva}, \(Y\) una \href{20250107112123-varieta_algebrica_quasi_proiettiva_qp.org}{varietà quasi proiettiva}. Allora la proiezione dal \href{20250108102640-prodotti_di_varieta_qp.org}{prodotto} su \(Y\)
\begin{align*}
\pi_{Y}: X\times Y &\longrightarrow Y\\
 (x,y) &\longmapsto y
\end{align*}
è \href{20250104114559-funzione_chiusa.org}{chiusa}.
\subsubsection{Relazione tra compattezza e varietà proiettive}
\label{sec:orgb06edb9}
Il teorema 2 afferma che se \(X\) è una varietà proiettiva, allora per ogni \(Y\) varietà affine la proiezione \(\pi_{Y}: X\times Y \longrightarrow Y\) è chiusa.
Il \href{20250109154613-teorema_di_kuratowski_mrowka.org}{Teorema di Kuratowski-Mrówka} afferma che se \(A\) è un generico \href{20250103145124-topologia.org}{spazio topologico}, \(A\) è \href{20250103163701-spazio_topologico_compatto.org}{compatto} se e solo se per ogni spazio topologico \(B\) si ha che
\begin{equation*}
\pi_{B}: A\times B\longrightarrow B
\end{equation*}
è \href{20250104114559-funzione_chiusa.org}{chiusa}.

Pertanto, in geometria algebrica, la nozione di compattezza è equivalente a quella di proiettiva.
\subsubsection{Relazione tra Hausdorff e varietà qp.}
\label{sec:orgebfb937}
Se \(Y\) è una \href{20250107112123-varieta_algebrica_quasi_proiettiva_qp.org}{varietà qp}, allora \(\Delta_{Y}\) (la \href{20250109155520-diagonale_di_uno_spazio_topologico.org}{diagonale}) è chiuso dentro \(Y\times Y\) (vedi \href{20250109154723-topologia_prodotto.org}{Topologia prodotto}).
Questo \href{20250109155704-caratterizzazione_spazi_t2_con_la_diagonale.org}{\textbf{non} implica} che \(Y\) sia \href{20250109155715-spazio_topologico_di_hausdorff.org}{T2}, poiché la topologia di \(Y\times Y\) non è la topologia prodotto, \textbf{ma} significa che tutti i teoremi che riguardano la compattezza che nella dimostrazione sfruttano l'argomento della diagonale sono ancora validi.

Questo definisce la proprietà che le varietà qp sono \textbf{\textbf{separate}}.
\subsubsection{Dimostrazione che 2->1}
\label{sec:orgc38477f}
Basta dimostrare che \(f(X)\) è \href{20250103145124-topologia.org}{chiuso} in \(Y\). Infatti, se \(Z \subseteq X\) è chiuso allora è \href{20241231123223-varieta_algebrica_proiettiva.org}{varietà proiettiva}, e quindi
\begin{equation*}
\restriction{f}{Z}:Z \longrightarrow Y
\end{equation*}
è un \href{20250107112412-morfismo_tra_varieta_algebriche_qp.org}{morfismo}, e quindi \(\restriction{f}{Z}(Z)=f(Z) \subseteq Y\) è chiuso.

Si ha che il \href{20250104112443-grafico_di_una_funzione.org}{grafico} di \(f\), \(\Gamma_{f} \subseteq X\times Y\) è una varietà (vedi \href{20250108124906-grafico_di_un_morfismo_proiettivo_e_varieta.org}{Grafico di un morfismo proiettivo è varietà}), e quindi è un chiuso.
Pertanto \(\pi_{Y}(\Gamma_{f}) = f(X)\) è chiuso, poiché per il teorema 2 \(\pi_{Y}\) è chiusa.
\subsubsection{{\bfseries\sffamily TODO} Dimostrazione di 2\hfill{}\textsc{matematica\_lm:geo\_alg}}
\label{sec:orgda73e39}
È sufficiente dimostrare il teorema per \(X= \mathds{P}^{n}\).\footnote{Infatti, se \(X \subseteq \mathds{P}^{n}\) chiuso, sia \(Z \subseteq X\times Y\) chiuso
Allora \(Z \subseteq X\times Y\) è chiuso e \(X\times Y \subseteq \mathds{P}^{n}\times Y\) è chiuso, dunque \(Z \subseteq \mathds{P}^{m}\times Y\) è chiuso.}

(Lezione 9, appunti di Piero)
\end{document}
