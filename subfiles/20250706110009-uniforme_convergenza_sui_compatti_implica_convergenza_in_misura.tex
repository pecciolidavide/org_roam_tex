% Intended LaTeX compiler: pdflatex
\documentclass[../main]{subfiles}


\begin{document}

Si ricorda che lo \href{20250704104947-funzione_misurabile.org}{spazio delle funzioni misurabili} \(\mathcal{M}^{n}\) è dotato di una \href{20250301193511-spazio_metrico.org}{distanza} \(d_{\mu}\) per ogni \href{20250704163455-misura.org}{misura} \(\mu\) su \(\R^{n}\)\footnote{Vedi ``\href{20250704104537-spazio_delle_funzioni_misurabili_come_spazio_metrico.org}{Spazio delle funzioni misurabili come spazio metrico}''}.
\begin{prop}
Sia \((f_{j})_{j} \subseteq \mathcal{M}^{n}\) una \href{20250629105815-successione_di_funzioni.org}{successione di funzioni} in \(\mathcal{M}^{n}\) che \href{20250706121659-convergenza_uniforme_sui_compatti.org}{converge uniformemente sui compatti} a \(f\), ovvero tale che, per ogni \(K \subseteq \R^{n}\) compatto
\begin{equation*}
\sup_{x \in K} |f_{j}(x)-f(x)|\to 0,\quad j\to\infty.
\end{equation*}

Allora \(d_{\mu}(f_{j},f)\to 0\) per \(j\to \infty\).
\label{prop9.3.21}
\end{prop}
\end{document}
