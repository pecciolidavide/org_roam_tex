% Intended LaTeX compiler: pdflatex
\documentclass[../main]{subfiles}


\begin{document}

\section{Pic0 di una superficie di Riemann}
\label{sec:org8c990d8}
Sia \(X\) una \href{20260127112828-superficie_di_riemann.org}{superficie di Riemann}, e si indichi con \(\operatorname{Pic}(X)\) il \href{20260202100511-gruppo_di_picard_di_una_superficie_di_riemann.org}{gruppo di Picard di \(X\)}.

\begin{oss}
\href{20260202100601-caratterizzazione_sfera_di_riemann_tramite_gruppo_di_picard.org}{In generale}, se \(X\) è \href{20250103163701-spazio_topologico_compatto.org}{compatto} di \href{20260127112828-superficie_di_riemann.org}{genere} \(g\), allora la \href{20260202100511-gruppo_di_picard_di_una_superficie_di_riemann.org}{mappa di grado} è \href{20241213105600-funzione_suriettiva.org}{suriettiva}
\begin{equation*}
\deg: \operatorname{Pic}(X) \twoheadrightarrow \Z
\end{equation*}
e si indica con \(\operatorname{Pic}^{0}(X) \coloneqq \ker \deg\) il \href{20241213105201-kernel.org}{kernel}: si ha che \(\operatorname{Pic}^{0}(X)\) è un \href{20260127113001-toro_complesso.org}{toro complesso di dimensione \(g\)}.
\end{oss}
\end{document}
