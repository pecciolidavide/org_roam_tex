% Intended LaTeX compiler: pdflatex
\documentclass[../main]{subfiles}

\usepackage[hyperref]{biblatex}
\date{}
\title{}
\begin{document}

\section{Unione di funzioni}
\label{sec:orgb0111cb}
Contesto: \href{20250130104245-morse_kelly_set_theory.org}{Morse Kelly Set Theory}
\subsection{Proposizione}
\label{sec:orgd8e7b20}
Sia \(\mathcal{F}\) una \href{20250130104320-classe_mk.org}{classe} di \href{20250202170607-classe_relazione_binaria.org}{funzioni} \href{20250202184517-ordine_superiormente_diretto.org}{superiormente diretta} rispetto a \(\subseteq\) (vedi \href{20250131155822-operazioni_insiemistiche_tra_classi_mk.org}{Sottoclasse MK}). Allora l'\href{20250131155822-operazioni_insiemistiche_tra_classi_mk.org}{unione}
\begin{equation*}
\bigcup \mathcal{F}
\end{equation*}
è una \href{20250202170607-classe_relazione_binaria.org}{relazione funzionale}.
\subsubsection{Dimostrazione}
\label{sec:orgd214c72}
L'unione \(\bigcup\mathcal{F}\) è una classe di \href{20250131162451-coppia_ordinata_mk.org}{coppie ordinate}, e pertanto è una \href{20250202170607-classe_relazione_binaria.org}{relazione}.

Se \((x,y), (x,y') \in \bigcup\mathcal{F}\), allora \((x,y) \in f\) e \((x,y') \in g\) per qualche \(f,g \in \mathcal{F}\).
Pertanto (per definizione di ordine superiormente diretta) esiste \(h \in \mathcal{F}\) tale che \(f \subseteq h\) e \(g \subseteq h\), e dunque
\begin{equation*}
(x,y), (x,y') \in h
\end{equation*}
ma \(h\) è una funzione, e pertanto \(y=y'\).
\end{document}
