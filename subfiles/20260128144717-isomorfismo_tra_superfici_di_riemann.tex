% Intended LaTeX compiler: pdflatex
\documentclass[../main]{subfiles}


\begin{document}

\section{Isomorfismo tra superfici di Riemann}
\label{sec:org1de6188}
Siano \(X,Y\) due \href{20260127112828-superficie_di_riemann.org}{superfici di Riemann}, \(F:X\to Y\) una \href{20250202170607-classe_relazione_binaria.org}{funzione}.

\begin{definizione}
\(F\) si dice un \textbf{isomorfismo} (o biolomorfismo) tra \(X\) e \(Y\) (e si dice che \(X\cong Y\) sono isomorfe) se
\begin{itemize}
\item \(F\) è \href{20260128143822-funzione_olomorfa_su_una_superficie_di_riemann.org}{olomorfa}
\item \(F\) è \href{20250104111707-funzione_biunivoca.org}{biiettiva}
\item \(F^{-1}:Y\to X\) è \href{20260126110551-funzione_olomorfa.org}{olomorfa}.
\end{itemize}
\end{definizione}

\begin{prop}
Se \(X,Y\) sono \href{20260127112828-superficie_di_riemann.org}{superfici di Riemann} \hyperref[sec:org1de6188]{biolomorfe}, allora sono
\begin{itemize}
\item \href{20250103145124-topologia.org}{spazi topologici} \href{20250111142332-omeomorfismo.org}{omeomorfi};
\item \href{20250113115909-struttura_differenziabile.org}{varietà differenziabili} \href{20250113172924-diffeomorfismo_tra_varieta_differenziabili.org}{diffeomorfe}.
\end{itemize}
\end{prop}
\end{document}
