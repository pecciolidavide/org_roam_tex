% Intended LaTeX compiler: pdflatex
\documentclass[../main]{subfiles}


\begin{document}

\section{Caratterizzazione Teoria delta-model-completa tramite eliminazione positiva dei quantificatori}
\label{sec:orge09e4b3}
Si utilizza la \href{20250612143636-notazione_teoria_dei_modelli.org}{Notazione della TEORIA DEI MODELLI}
\begin{itemize}
\item Sia \(\mathcal{L}\) un \href{20250130162057-linguaggio_del_prim_ordine.org}{linguaggio del prim'ordine}
\item Sia \(\Delta\) un insieme di formule chiuso per sostituzione di variaibli e che contenga la formula ``\(x=y\)''.
\item Se \(C \subseteq \set{\forall , \exists , \land,\lor,\lnot}\) è un insieme di connettivi, \(C\Delta\) denota la chiusura di \(\Delta\) per i connettivi in \(C\). \(\Delta^{\pm} \coloneqq \set{\lnot}\Delta\).
\end{itemize}

\begin{prop}
Sia \(T\) una \href{20250130114950-teoria_del_prim_ordine.org}{\(\mathcal{L}\)-teoria}. Sono fatti equivalenti:
\begin{enumerate}
\item \(T\) è \href{20251020171034-teoria_delta_model_completa.org}{\(\Delta\)-\emph{model}-completa};
\item \(T\) ha \href{20251020161056-teoria_con_delta_eliminazione_positiva_dei_quantificatori.org}{\(\set{\exists, \land  }\!\Delta\)-eliminazione positiva dei quantificatori}.
\end{enumerate}
\end{prop}
\begin{proof}

\end{proof}
\end{document}
