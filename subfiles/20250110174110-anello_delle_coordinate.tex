% Intended LaTeX compiler: pdflatex
\documentclass[../main]{subfiles}


\begin{document}

\section{Anello delle coordinate}
\label{sec:org981b1d2}
Sia \(\K\) un \href{20241231112713-campo_algebricamente_chiuso.org}{campo algebricamente chiuso}. Sia \(Y \subseteq \A^{n}\) una \href{20241231114256-varieta_algebrica_affine.org}{varietà affine} (vedi \href{20241231114009-spazio_affine.org}{Spazio Affine}), e sia \(I(Y)\) il suo \href{20250103144124-ideale_di_un_sottoinsieme.org}{ideale}.
\subsection{Definizione}
\label{sec:orgc3193f4}
L'anello delle coordinate di \(Y\) è
\begin{equation*}
\K[Y]\coloneqq\frac{\K[x_{1},\dots,x_{n}]}{I(Y)}
\end{equation*}

Questo è l'anello delle funzioni polinomiali su \(Y\), ovvero i \href{20250104110524-morfismo_tra_varieta_algebriche_affini.org}{morfismi}
\begin{equation*}
Y \longrightarrow \A^{1}
\end{equation*}
\end{document}
