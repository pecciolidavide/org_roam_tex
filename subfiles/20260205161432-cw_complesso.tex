% Intended LaTeX compiler: pdflatex
\documentclass[../main]{subfiles}


\begin{document}

\section{CW-complesso}
\label{sec:org7c18d9d}
\begin{definizione}
Un \textbf{CW-complesso} è uno \href{20250103145124-topologia.org}{spazio topologico} \(X\) dato dall'unione:
\begin{equation*}
X=\bigcup_{n \in \N} X_{n}
\end{equation*}
tale che:
\begin{itemize}
\item \(X_{0}\) è \href{20260128123515-sottoinsieme_discreto.org}{discreto} e finito;
\item per ogni \(k\): esiste una famiglia \(\set{e_{\alpha}^{k+1}}_{\alpha \in A^{k+1}}\) di \href{20260205151208-attaccamento_di_una_k_cella_a_uno_spazio_topologico.org}{\((k+1)\)-celle} ed una famiglia di funzioni \href{20250103103252-funzione_continua.org}{continue} dal \href{20250129161026-bordo.org}{bordo} \(\varphi_{\alpha}^{k+1}: \partial e_{\alpha}^{k+1} \to X_{k}\); tali che \(X_{k+1}\) sia dato dal \href{20260205151208-attaccamento_di_una_k_cella_a_uno_spazio_topologico.org}{seguente attaccamento}:
\begin{equation*}
X_{k+1} \coloneqq X_{k} \mathrel{\cup_{\set{\varphi_{\alpha}^{k+1}}}} \left(
  \coprod_{\alpha \in A^{k+1}} e_{\alpha}^{k+1}
\right)
\end{equation*}
\end{itemize}
Lo spazio \(X_{n}\) si dice \(n\)-scheletro.
\end{definizione}

\begin{oss}
Se \(\Phi_{\alpha}^{n}\) sono le mappe caratteristiche per \(e_{\alpha}^{n}\), allora il seguente diagramma commuta
\begin{equation*}
\begin{tikzcd}[ampersand replacement=\&,column sep=scriptsize]
	{\coprod_{\alpha} e_\alpha^n} \&\& {X_n} \\
	\\
	{\coprod_{\alpha} \partial e_\alpha^n} \&\& {X_{n-1}}
	\arrow["{\coprod_\alpha \Phi_\alpha^n}", from=1-1, to=1-3]
	\arrow[hook, from=3-1, to=1-1]
	\arrow["{\coprod_\alpha \varphi_\alpha^n}"', from=3-1, to=3-3]
	\arrow[hook, from=3-3, to=1-3]
\end{tikzcd}
\end{equation*}
dove \(\amalg\) indica l'\href{20250113175700-unione_disgiunta.org}{unione disgiunta}.
\end{oss}

\begin{oss}
Per ogni \(n\), \((X_{n+1}, X_{n})\) è una \href{20250129183256-coppia_topologica_buona.org}{coppia topologica buona}.
\end{oss}

\begin{esempio}
La \href{20250115150754-sfera_n_dimensionale.org}{sfera \(\mathds{S}^{n}\)} ha una naturale struttura di CW-complesso. Infatti, fissato \(p \in \mathds{S}^{n}\), siano
\begin{equation*}
X_{0} = X_{1} =\dots = X_{n-1} \coloneqq \set{p}
\end{equation*}
e si definisca \(X_{n} = \set{p} \mathrel{\cup_{\varphi}} e^n_{1}\), dove
\begin{align*}
\varphi: \partial e^{n}_{1} &\longrightarrow X_{n-1}\\
x &\longmapsto p
\end{align*}
\end{esempio}
\subsection{Topologia}
\label{sec:org74b80d1}

\begin{oss}
La topologia su un CW-complesso è la topologia debole, ovvero
\begin{equation*}
C \subseteq X\ \text{chiuso}%
\IFF%
\forall i\ C\cap X_{i}\ \text{chiuso}.
\end{equation*}
\end{oss}

\begin{prop}
Un \(CW\)-complesso è \href{20260209153646-sistema_diretto.org}{limite diretto} dei suoi scheletri con le mappe di immersione.
\end{prop}
\end{document}
