% Intended LaTeX compiler: pdflatex
\documentclass[../main]{subfiles}


\begin{document}

\section{Genere topologico di una Superficie di Riemann non cresce per olomorfismo}
\label{sec:orgcc6055d}
Siano \(X,Y\) \href{20260127112828-superficie_di_riemann.org}{superfici di Riemann} \href{20250103163701-spazio_topologico_compatto.org}{compatte}, e sia \(F:X\to Y\) \href{20260128143822-funzione_olomorfa_su_una_superficie_di_riemann.org}{olomorfismo} non costante.

\begin{prop}
Per il \href{20260127112828-superficie_di_riemann.org}{genere topologico} vale:
\begin{equation*}
g(X) \ge g(Y).
\end{equation*}
\end{prop}

\begin{proof}
È sufficente utilizzare la \href{20260129171832-formula_di_hurewicz_superfici_di_riemann.org}{Formula di Hurwitz}, ottenendo che:\footnote{\(\deg F\) è il \href{20260129171444-teorema_del_grado_per_olomorfismi_tra_superfici_di_riemannn.org}{grado di \(F\)}.}
\begin{equation*}
g(X) = 1 + \parentesi{\ge 1}{(\deg F)} \cdot \big(g(Y)-1\big) +\parentesi{\ge 0}{ \frac{1}{2}\, \mathrm{K} } \ge g(Y).\qedhere
\end{equation*}
\end{proof}

\begin{cor}
Inoltre, se \(F\) non è \href{20260128144717-isomorfismo_tra_superfici_di_riemann.org}{isomorfismo}, allora si ha una tra le seguenti:
\begin{itemize}
\item \(g(X)> g(Y)\);
\item \(g(X)=g(Y) \le 1\).
\end{itemize}
\end{cor}
\begin{proof}
Se \(F\) non è isomorfismo, \href{20260129171557-caratterizzazione_isomorfismo_tra_superfici_di_riemann_tramite_grado.org}{allora} \(\deg F \ge 2\).
\begin{equation*}
g(X) \ge 1 + 2 \cdot \big(g(Y)-1\big) = 2 g(Y) - 1
\end{equation*}
\begin{itemize}
\item se \(g(Y) > 1\), allora \(2g(Y)-1 > g(Y)\), e pertanto \(g(X) > g(Y)\);
\item se \(g(Y) = 1\), allora \(g(X) = 1 + K\)
\item se \(g(Y) = 0\),
\end{itemize}
\end{proof}
\end{document}
