% Intended LaTeX compiler: pdflatex
\documentclass[../main]{subfiles}


\begin{document}

\section{Esercizio 3}
\label{sec:orge6f7293}

Let \(E\) be an equivalence relation on a Polish space \(X\). A \textbf{transversal} for \(E\) is a set \(T \subseteq X\) meeting every \(E\)-equivalence class in exactly one point. A \textbf{selector} for \(E\) is a map \(s: X \to X\) selecting one element from each \(E\)-equivalence class, that is, \(s(x) \in [x]_E\) and \(s(x) = s(y)\) if \(x \mathrel{E} y\). Show that if \(E\) is analytic, then the following are equivalent:

\begin{enumerate}
\item \(E\) admits an analytic transversal;
\item \(E\) admits a Borel transversal;
\item \(E\) admits a Borel selector.
\end{enumerate}
\subsection{Soluzione}
\label{sec:orgf3c861b}

\subsubsection{c. implica b.}
\label{sec:org4df0486}

Sia \(s:X\to X\) un selettore boreliano per \(E\) e sia \(T\coloneqq s(X)\).

Allora \uline{\(T\) è trasversale}. Infatti incontra ogni classe di \(E\)-equivalenza esattamente una volta.
\begin{itemize}
\item \uline{Almeno una volta}: Per ogni \(x \in X\) esiste \(t \in T\) tale che \(x\mathrel{R}t\): \(t=s(x)\).
\item \uline{Al più una volta}: Siano \(x\neq y \in T\) e siano \(x_{0},y_{0} \in X\) tali che
\begin{equation*}
  s(x_{0})=x,\quad s(y_{0})=y.
\end{equation*}
Per definizione \(x\mathrel{E}x_{0}\) e \(y\mathrel{E}y_{0}\). Se per assurdo \(x\mathrel{E}y\) allora \(x_{0}\mathrel{E}y_{0}\) per transitività di \(E\). Per definizione, allora
\begin{equation*}
  s(x_{0}) = s(y_{0})
\end{equation*}
ovvero \(x=y\). Assurdo.
\end{itemize}

Inoltre, sia
\begin{align*}
f: X &\longrightarrow X\times X\\
x &\longmapsto \left(x,s(x)\right)
\end{align*}
Questa è una funzione boreliana, poiché \(s\) è boreliana: \(f=\operatorname{Id}_{X}\times s\) e per le proprietà di pag. 54, \(f\) è boreliana.

Allora, detta \(D \subseteq X\times X\) la diagonale,
\begin{equation*}
D\coloneqq\set{(x,x)\mid x \in X}
\end{equation*}
si ha che \(D\) è chiuso, poiché \(X\) è metrizzabile e quindi Haussdorf. Inoltre \(T=f^{-1}(D)\)
\begin{itemize}
\item (\(\subseteq\)): Se \(t \in T\), allora \(s(t)=t\), poiché altrimenti \(s(t) \in T\) sarebbe un elemento distinto da \(t\) della classe \([t]_{E}\). Pertanto \(f(t) = \left(t,s(t)\right) = (t,t) \in D\).
\item (\(\supseteq\)): Se \(t \in f^{-1}(D)\) allora \(s(t)=t\) e quindi \(t \in s(X) = T\).
\end{itemize}

Dunque, siccome \(f\) è boreliana e \(D\) è chiuso, \uline{\(T\) è un boreliano}.
\subsubsection{b. implica a.}
\label{sec:org8450c5a}

Questo è ovvio, poiché \(\bm{{\operatorname{Bor}}}(X) \subseteq \bm{\Sigma}_{1}^{1}(X)\) per il Corollario 3.1.4.
\subsubsection{a. implica c.}
\label{sec:orgd086295}

Sia \(T \subseteq X\) un insieme analitico trasversale per \(E\).

Siccome \(T\) è trasversale per \(E\), allora è ben definita la funzione
\begin{align*}
\varphi: X/E &\longrightarrow T\\
[x]_{E} &\longmapsto t \in [x]_{E}.
\end{align*}
poiché per ogni classe di \(E\)-equivalenza esiste un unico elemento \(t \in T\) tale che \(t \in [x]_{E}\).

Si definisce dunque la funzione \(s: X\to T: x\mapsto \varphi\left([x]_{E}\right)\). Questa è un \uline{selettore}, poiché:
\begin{itemize}
\item per ogni \(x \in X\): \(s(x) = \varphi\left([x]_{E}\right) = t \in [x]_{E}\);
\item se \(x\mathrel{E}y\) allora \([x]_{E}= [y]_{E}\) e pertanto
\begin{equation*}
  s(x) = \varphi\left([x]_{E}\right) = \varphi\left([y]_{E}\right) = s(y).
\end{equation*}
\end{itemize}

Resta da dimostrare che \(s\) sia Boreliana. Sfruttando il Teorema 3.2.4 è sufficiente dimostrare che \(\operatorname{graph}(s) \subseteq X\times X\) sia analitico. Si ha che
\begin{equation*}
\operatorname{graph}(s) = E\cap (X\times T)
\end{equation*}
infatti:
\begin{itemize}
\item se \((x,y) \in \operatorname{graph}(s)\) allora \(y=s(x)\), e poiché \(s\) è un selettore: \(x\mathrel{E} s(x)\) e quindi \((x,y) \in E\); inoltre \(x \in X\) e \(y=s(x) \in T\);
\item viceversa, se \((x,y) \in E\cap (X\times T)\) allora \(y \in T\) e \(x\mathrel{E} y\); inoltre \(y\) è l'unico elemento di \(T\) tale che \(x\mathrel{E}y\), e pertanto, per definizione \(y=s(x)\).
\end{itemize}

Sia \(T\) che \(E\) sono analitici per ipotesi. Inoltre \(X\times T = \pi_{2}^{-1}(T)\) è analitico, in quanto retroimmagine continua di un analitico (per la Proposizione 3.1.5), e dunque \(\operatorname{graph}(s)\) è analitico.
\end{document}
