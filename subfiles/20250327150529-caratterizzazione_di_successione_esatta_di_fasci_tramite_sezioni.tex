% Intended LaTeX compiler: pdflatex
\documentclass[../main]{subfiles}


\begin{document}

\section{Caratterizzazione di successione esatta di fasci tramite sezioni}
\label{sec:orgc476d30}
Sia \(X\) uno \href{20250103145124-topologia.org}{spazio topologico}, e sia
\begin{equation*}
\begin{tikzcd}[ampersand replacement=\&]
	{\mathcal{F}} \& {\mathcal{G}} \& {\mathcal{H}}
	\arrow["\varphi", from=1-1, to=1-2]
	\arrow["\psi", from=1-2, to=1-3]
\end{tikzcd}
\end{equation*}
una \href{20250327122142-successione_di_una_categoria.org}{successione} di \href{20250325180613-morfismo_di_prefasci.org}{morfismi} di \href{20250325180613-morfismo_di_prefasci.org}{fasci} di \href{20241205141146-gruppo_abeliano.org}{gruppi}.

\begin{prop}
La successione è \href{20250327150404-successione_di_fasci_esatta.org}{esatta in \(\mathcal{G}\)} se e solo se, per ogni \(U\subseteq X\) aperto, data
\begin{equation*}
\begin{tikzcd}[ampersand replacement=\&]
	{\mathcal{F}(U)} \& {\mathcal{G}(U)} \& {\mathcal{H}(U)}
	\arrow["{\varphi_U}", from=1-1, to=1-2]
	\arrow["{\psi_U}", from=1-2, to=1-3]
\end{tikzcd}
\end{equation*}
si ha
\begin{enumerate}
\item \(\operatorname{Im}\varphi_{U} \subseteq \ker\psi_{U}\);
\item per ogni \(g \in \ker\psi_{U} \subseteq \mathcal{G}(U)\), per ogni \(p \in U\)
\begin{itemize}
\item esiste \(W \subseteq {U}\) intorno aperto di \(p\);
\item esiste \(f \in \mathcal{F}(W)\)
\end{itemize}
tali che \(\restriction{g}{W} = \varphi_{W}(f)\).
\end{enumerate}
\end{prop}

\begin{oss}
La condizione 1. ci dice che \(\operatorname{Im}\varphi \subseteq\ker\psi\). La condizione 2. è l'altra inclusione.
\end{oss}
\end{document}
