% Intended LaTeX compiler: pdflatex
\documentclass[../main]{subfiles}


\begin{document}

\section{Successione di Mayer-Vietoris per coomologia a supporto compatto}
\label{sec:orga9da3a9}
Sia \(M\) una \href{20250113115909-struttura_differenziabile.org}{varietà differenziabile}. Si indichi con \(A^{\bullet}_{\text{c}}\) l'insieme delle \href{20251229110021-forma_differenziale_a_supporto_compatto.org}{forme differenziali a supporto compatto} e con \(j_{\star}\) l'\href{20251229110947-estensione_a_zero_di_una_forma_differenziale_a_supporto_compatto.org}{estensione a zero} di tali forme.

Sia \(\set{U_{0},U_{1}}\) un \href{20250103164252-ricoprimento.org}{ricoprimento aperto} di \(M\). Si definiscono\footnote{Vedi ``\href{20241213095808-somma_diretta.org}{Somma Diretta}''}
\begin{align*}
s: A^{\bullet}_{\text{c}}(U_{0}) \oplus A^{\bullet}_{\text{c}}(U_{1})  &\longrightarrow A^{\bullet}_{\text{c}}(M)\\
(\omega_{0},\omega_{1}) &\longmapsto j_{*}\omega_{0} + j_{*}\omega_{1}\\[1.5em]
\delta: A^{\bullet}_{\text{c}}(U_{0}\cap U_{1}) &\longrightarrow A^{\bullet}_{\text{c}}(U_{0}) \oplus A^{\bullet}_{\text{c}}(U_{1})\\
\eta &\longmapsto (-j_{*}\eta, j_{*}\eta).
\end{align*}
Queste sono \href{20250114101949-funzione_lineare.org}{funzioni lineari} e \href{20251115182606-morfismo_tra_complessi_di_cocatene.org}{morfismi tra complessi di cocatene}.

\begin{prop}
(Proposizione~5.5.8 di ).

La successione
\begin{equation*}
\begin{tikzcd}
	0 & {A^{\bullet}_{\text{c}}(U_0 \cap U_1)} & {A^{\bullet}_{\text{c}} (U_0)\oplus A^{\bullet}_{\text{c}}(U_1)} & {A^{\bullet}_{\text{c}}(M)} & 0
	\arrow[from=1-1, to=1-2]
	\arrow["\delta", from=1-2, to=1-3]
	\arrow["s", from=1-3, to=1-4]
	\arrow[from=1-4, to=1-5]
\end{tikzcd}
\end{equation*}
è una \href{20251115182133-successione_di_spazi_vettoriali_esatta.org}{SEC di spazi vettoriali} (e di \href{20251115182320-complesso_di_cocatene.org}{complessi di cocatene}\footnote{Vedi anche ``\href{20251115182707-sec_di_complessi_di_cocatene.org}{Successione Esatta di Complessi di Cocatene}''}), e pertanto \href{20251115182954-zig_zag_lemma_per_complessi_di_cocatene.org}{induce} una \href{20251115182133-successione_di_spazi_vettoriali_esatta.org}{successione esatta} in \href{20251115182537-coomologia_di_un_complesso_di_cocatene.org}{coomologia}:
\begin{equation*}
\begin{tikzcd}
	\cdots & {H_{\text{c}}^{k-1}(M)} & {H_{\text{c}}^k(U_0 \cap U_1)} & {H_{\text{c}}^k(U_0)\oplus H_{\text{c}}^k (U_1)} & \cdots
	\arrow[from=1-1, to=1-2]
	\arrow["{\partial_*}", from=1-2, to=1-3]
	\arrow[from=1-3, to=1-4]
	\arrow[from=1-4, to=1-5]
\end{tikzcd}
\end{equation*}
Inoltre, anche la \href{20251229101448-successione_di_spazi_vettoriali_esatta_induce_successione_esatta_sui_duali.org}{successione duale} lo è:
\end{prop}
\begin{oss}
La scrittura di cui sopra si ha in virtù del fatto che la \href{20251115192241-coomologia_a_supporto_compatto.org}{coomologia a supporto compatto} è uguale alla \href{20251115182537-coomologia_di_un_complesso_di_cocatene.org}{coomologia di complesso di cocatene} di \href{20251229110021-forma_differenziale_a_supporto_compatto.org}{\(A^{\bullet}_{\text{c}}\)}.
\end{oss}
\begin{proof}
Si dimostra unicamente la \uline{suriettività di \(s\)} (ovvero l'esattezza della catena \(A^{\bullet}_{\text{c}}(U_{0}) \oplus A^{\bullet}_{\text{c}}(U_{1}) \xrightarrow{s} A^{\bullet}_{\text{c}}(M) \to 0\)).

Sia \(\omega \in A^{\bullet}_{\text{c}}(M)\). Sia \(\set{\rho_{0},\rho_{1}}\) una \href{20251223153807-partizione_dell_unita.org}{partizione dell'unità} relativa al ricoprimento \(\set{U_{0},U_{1}}\).

Si ha che \(\rho_{0}\omega \in A^{\bullet}_{\text{c}}(U_{0})\) e \(\rho_{1}\omega \in A^{\bullet}_{\text{c}}(U_{1})\):
\begin{equation*}
s (\rho_{0}{\omega},\rho_{1}\omega) = j_{*}\rho_{0}\omega + j_{*}\rho_{1}\omega = \rho_{0}\omega + \rho_{1}\omega = (\rho_{0}+\rho_{1})\omega = \omega.%
\qedhere
\end{equation*}
\end{proof}
\end{document}
