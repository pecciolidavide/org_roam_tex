% Intended LaTeX compiler: pdflatex
\documentclass[../main]{subfiles}


\begin{document}

\section{Faccia di un simplesso}
\label{sec:orgf304f42}
\begin{definizione}
Siano \(p_{0},\dots,p_{q} \in \R^{n}\). Se \(\set{p_{i_{0}},\dots,p_{i_{k}}} \subseteq \set{p_{0},\dots,p_{q}}\), allora i \href{20250121121923-simplessi.org}{simplessi}
\begin{equation*}
\Delta'=[p_{i_{0}},\dots,p_{i_{k}}] \subseteq [p_{0},\dots,p_{q}] = \Delta
\end{equation*}

Tali simplessi sono detti faccia di \(\Delta\), e si scrive \(\Delta'\preceq \Delta\)
Se \(k=q-1\) allora sono \textbf{facciata}.
\end{definizione}
\end{document}
