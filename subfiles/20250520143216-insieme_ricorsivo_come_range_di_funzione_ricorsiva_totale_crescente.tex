% Intended LaTeX compiler: pdflatex
\documentclass[../main]{subfiles}


\begin{document}

In questa sezione si identificano i \href{20250131155822-operazioni_insiemistiche_tra_classi_mk.org}{sottinsiemi} di \(\N^{k}\) (vedi \href{20250202130045-insieme_dei_numeri_naturali_mk.org}{Insieme dei numeri naturali MK}) con i \href{20250131103317-formula_del_prim_ordine.org}{predicati} \(k\)-ari (ovvero con \(k\) \href{20250131103429-variabile_libera_di_una_formula.org}{variabili libere}), per mezzo degli \href{20250131122913-soddisfazione_di_una_formula.org}{insiemi di verità}.

Scriveremo indifferentemente \((x_{1},\dots,x_{k}) \in P\) oppure \(P(x_{1},\dots,x_{k})\) per dire che \(\N\vDash P(x_{1},\dots,x_{k})\).
\section{Proposizione}
\label{sec:org6048033}

Un insieme infinito \(P \subseteq \N\) è \href{20250216173925-insieme_ricorsivo.org}{ricorsivo} se e solo se \(P= \operatorname{rng}(f)\) \footnote{Vedi ``\href{20250202173528-dominio_range_e_campo_di_una_classe_relazione.org}{Dominio, Range e Campo di una Classe Relazione}''} per qualche \href{20250202170607-classe_relazione_binaria.org}{funzione}
\begin{equation*}
f: \N\to \N
\end{equation*}
\href{20250207104855-funzione_ricorsiva.org}{ricorsiva} \uline{\href{20250213105339-funzione_parziale.org}{totale}} e \href{20250203132953-funzione_monotona.org}{strettamente crescente}.
\subsection{Dimostrazione}
\label{sec:orgc1c135d}

(\(\Rightarrow\)): Si costruisce la funzione \(f\) \uline{per ricorsione}:
\begin{equation*}
\begin{cases}
f(0) = \minim{x}{P(x)}\\
f(y+1) = \minim{x}{P(x) \,\land\, f(y)<x}
\end{cases}
\end{equation*}
(vedi  ``\href{20250520101418-funzioni_ricorsive_per_minimizzazione_su_un_predicato.org}{Funzioni ricorsive per minimizzazione su un predicato}'' e ``\href{20250215151458-funzioni_ricorsive.org}{Schema di Ricorsione di funzioni ricorsive}'').

Questa è ricorsiva, totale e tale che \(P=\operatorname{rng}(f)\).

(\(\Leftarrow\)): Si ha che
\begin{equation*}
P(y)\quad \iff \quad \exists\,x\le y\ \left(\operatorname{graph}(f)(x,y)\right)
\end{equation*}
dove il \href{20250104112443-grafico_di_una_funzione.org}{grafico} di \(f\) è \href{20250216173925-insieme_ricorsivo.org}{ricorsivo} \href{20250601162421-funzioni_ricorsive_e_loro_grafico.org}{poiché} \(f\) è \href{20250213105339-funzione_parziale.org}{totale} e \href{20250207104855-funzione_ricorsiva.org}{ricorsiva}. \href{20250519112917-proprieta_di_chiusura_degli_insiemi_ricorsivi.org}{Quindi} \(P\) è ricorsivo.
\section{Corollario}
\label{sec:org6ab5a1a}

Se \(P \subseteq \N\) è ricorsivo e infinito, allora esiste una \href{20250104111707-funzione_biunivoca.org}{biiezione} ricorsiva \(g:P\to \N\).

Sia infatti \(f: \N\to \N\) la funzione strettamente crescente totale e ricorsiva tale che \(P=f(\N)\).

Poiché questa è strettamente crescente allora è \href{20241219101956-funzione_iniettiva.org}{iniettiva}, e restringendo il codominio si ottiene
\begin{equation*}
f: \N\to P = f(\N)
\end{equation*}
biiettiva.

Pertanto l'inversa \(g: P\to \N\) è una biiezione tra \(P\) e \(\N\).

Inoltre, questa è ricorsiva, \href{20250601162421-funzioni_ricorsive_e_loro_grafico.org}{poiché} il suo \href{20250104112443-grafico_di_una_funzione.org}{grafico} è \href{20250216173925-insieme_ricorsivo.org}{ricorsivo}
\begin{equation*}
\operatorname{graph}(g)(x,y) \quad\iff \quad \operatorname{graph}(f)\left(F(x,y)\right)
\end{equation*}
dove \(F: \N^{2}\to \N^{2}: (m,n)\mapsto (n,m)\).

Il grafico è ricorsivo perché:
\begin{itemize}
\item \(\operatorname{graph}(f)\) è ricorsivo \href{20250601162421-funzioni_ricorsive_e_loro_grafico.org}{poiché \(f\) è ricorsiva totale} ed \(F\) è una funzione ricorsiva;
\item \href{20250519112917-proprieta_di_chiusura_degli_insiemi_ricorsivi.org}{gli insiemi ricorsivi sono chiusi per sostituzioni totali ricorsive.}
\end{itemize}
\end{document}
