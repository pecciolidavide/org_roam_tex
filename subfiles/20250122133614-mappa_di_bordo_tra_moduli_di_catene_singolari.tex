% Intended LaTeX compiler: pdflatex
\documentclass[../main]{subfiles}

\usepackage[hyperref]{biblatex}
\date{}
\title{}
\begin{document}

\section{Mappa di bordo tra moduli di catene singolari}
\label{sec:org1c89046}
Sia \(X\) uno spazio topologico, e siano, per ogni \(q\), \(S_{q}(X)\) il \href{20250122133435-simplesso_singolare.org}{modulo delle \(q\)-catene singolari su \(X\)}\footnote{Questo è definito come la \href{20241213095808-somma_diretta.org}{somma diretta} \(R^{(\Sigma_{q}(X))}\), dove \(\Sigma_{q}(X)\) è l'\href{20250122133435-simplesso_singolare.org}{insieme dei simplessi singolari}.}
\begin{definizione}
Si definisce la mappa \(\partial_{q}: S_{q}(X)\longrightarrow S_{q-1}(X)\) come \href{20241206115416-morfismi_r_moduli.org}{morfismo} che, per ogni \href{20250122133435-simplesso_singolare.org}{simplesso singolare} \(\sigma\)
\begin{equation*}
\partial_{q}(\sigma) \coloneqq \sum_{i=0}^{q} (-1)^{i} \sigma\circ \varepsilon_{i}^{q}
\end{equation*}
dove le \(\varepsilon_{i}^{q}\) sono gli \href{20250122133535-operatori_di_facciata_del_simplesso_standard.org}{operatori di facciata}.
\end{definizione}
\begin{prop}
Si ha che \(\partial_{q-1}\circ\partial_{q} = 0\).
\end{prop}
\begin{proof}
Si inizia con il seguente:
\begin{itemize}
\item \uline{Claim}: se \(i\le j\), gli operatori di facciata rispettano:
\begin{equation*}
\varepsilon_{i}^{q}\circ \varepsilon_{j}^{q-1} = \varepsilon_{j+1}^{q} \circ \varepsilon_{i}^{q-1}
\end{equation*}

\item \emph{Dimostrazione del claim}: per dimostrarlo, è sufficiente considerare \(e_{k} \in \Delta_{q-1}\):
\begin{center}
\begin{tabular}{c|c|c|c|}
\(\varepsilon_{i}^{q}\circ \varepsilon_{j}^{q-1}(e_{k})\) & \(k < i\) & \(k=i\)                   < & \(k>i\)\\
\hline
\(k<j\) & \(\varepsilon_{i}^{q}(e_{k}) = e_{k}\) & \(\varepsilon_{i}^{q}(e_{k}) = e_{k+1}\) & \(\varepsilon_{i}^{q}(e_{k}) = e_{k+1}\)\\
\hline
\(k=j\) & X & \(\varepsilon_{i}^{q}(e_{k+1}) = e_{k+2}\) & \(\varepsilon_{i}^{q}(e_{k+1}) = e_{k+2}\)\\
\hline
\(k>j\) & X & X & \(\varepsilon_{i}^{q}(e_{k+1}) = e_{k+2}\)\\
\hline
\end{tabular}
\end{center}

\begin{center}
\begin{tabular}{c|c|c|c|}
\(\varepsilon_{j+1}^{q} \circ \varepsilon_{i}^{q-1}(e_{k})\) & \(k < i\) & \(k=i\) & \(k>i\)\\
\hline
\(k<j\) & \(\varepsilon_{j+1}^{q}(e_{k}) = e_{k}\) & \(\varepsilon_{j+1}^{q}(e_{k+1}) = e_{k+1}\) & \(\varepsilon_{j+1}^{q}(e_{k+1}) = e_{k+1}\)\\
\hline
\(k=j\) & X & \(\varepsilon_{j+1}^{q}(e_{k+1}) = e_{k+2}\) & \(\varepsilon_{j+1}^{q}(e_{k+1}) = e_{k+2}\)\\
\hline
\(k>j\) & X & X & \(\varepsilon_{j+1}^{q}(e_{k+1}) = e_{k+2}\)\\
\hline
\end{tabular}
\end{center}

E pertanto si ottiene che
\begin{equation*}
  \varepsilon_{i}^{q}\circ \varepsilon_{j}^{q-1}(e_{k}) = \varepsilon_{j+1}^{q} \circ \varepsilon_{i}^{q-1}(e_{k}) = \begin{cases}
  	e_{k} & k<i\\
  	e_{k+1} & i \le k < j\\
  	e_{k+2} & k \ge j
  \end{cases}
\end{equation*}
\end{itemize}

Sia quindi ora \(\sigma \in \Sigma_{q}(X)\), \href{20241213094625-modulo_libero.org}{base} di \(S_{q}(X)\).
\begin{align*}
\partial_{q-1}\circ \partial_{q} \, \sigma &=
	\partial_{q-1} \big(\sum_{i=0}^{q} (-1)^{i} \sigma \circ \varepsilon_{i}^{q}\big) =
	\sum_{i=0}^{q} (-1)^{i} \partial_{q-1} (\sigma \circ \varepsilon_{i}^{q}) = \sum_{i=0}^{q} \sum_{j=0}^{q-1} (-1)^{i+j} \sigma \circ \varepsilon_{i}^{q} \circ \varepsilon_{j}^{q-1} = \\
	&= \sum_{\substack{0\le i\le q\\ 0 \le j < i}} (-1)^{i+j} \sigma \circ \varepsilon_{i}^{q} \circ \varepsilon_{j}^{q-1}+
		\sum_{\substack{0\le i \le q\\ i\le j \le q-1}} (-1)^{i+j} \sigma \circ \varepsilon_{i}^{q} \circ \varepsilon_{j}^{q-1} = \\
	&\underset{\dagger}{=} \sum_{\substack{0\le i\le q\\ 0 \le j < i}} (-1)^{i+j} \sigma \circ \varepsilon_{i}^{q} \circ \varepsilon_{j}^{q-1}+
		\sum_{\substack{0\le i \le q\\ i\le j \le q-1}} (-1)^{i+j} \sigma \circ \varepsilon_{j+1}^{q} \circ \varepsilon_{i}^{q-1}\\
	&= \sum_{\substack{0\le i\le q\\ 0 \le j < i}} (-1)^{i+j} \sigma \circ \varepsilon_{i}^{q} \circ \varepsilon_{j}^{q-1} -
		\sum_{\substack{0\le i \le q\\ i\le j \le q-1}} (-1)^{i+j+1} \sigma \circ \varepsilon_{j+1}^{q} \circ \varepsilon_{i}^{q-1}\\
	&\underset{\ddagger}{=} \sum_{\substack{0\le i\le q\\ 0 \le j < i}} (-1)^{i+j} \sigma \circ \varepsilon_{i}^{q} \circ \varepsilon_{j}^{q-1} -
		\sum_{\substack{0\le j' \le q\\ j'+1\le i' \le q}} (-1)^{i'+j'} \sigma \circ \varepsilon_{i'}^{q} \circ \varepsilon_{j'}^{q-1} = 0\\
\end{align*}
dove:
\begin{itemize}
\item (\(\dagger\)): è per il claim;
\item (\(\ddagger\)): si è fatta la sostituzione, nella seconda parte:  \(\displaystyle\begin{aligned}
	i & \mapsto j'\\
	j+1 & \mapsto i'
  \end{aligned}\)\qedhere
\end{itemize}
\end{proof}
\subsection{Complesso di catene singolare}
\label{sec:orga6271a8}
Si definisce il \textbf{\href{20250120163114-complesso_di_catene.org}{complesso di catene} singolare}:
\begin{equation*}
\mathcal{S}_{\bullet}(X) \coloneqq \set{\left(S_{q}(X),\partial_{q}\right)}_{q}
\end{equation*}
\end{document}
