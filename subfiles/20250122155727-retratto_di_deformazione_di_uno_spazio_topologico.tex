% Intended LaTeX compiler: pdflatex
\documentclass[../main]{subfiles}

\usepackage[hyperref]{biblatex}
\date{}
\title{}
\begin{document}

\section{Retratto di deformazione di uno spazio topologico}
\label{sec:org8cb866a}
Sia \(X\) uno \href{20250103145124-topologia.org}{spazio topologico}.
\begin{definizione}
Un \href{20250103163814-sottospazio_topologico.org}{sottospazio} \(A \subseteq X\), con l'inclusione
\begin{equation*}
i_{A}: A\hookrightarrow X
\end{equation*}
si dice \textbf{retratto di deformazione di \(X\)} se esiste una \href{20250122155714-retratto_di_uno_spazio_topologico.org}{retrazione}
\begin{equation*}
r: X\longrightarrow A
\end{equation*}
tale che \(i_{A}\circ r \sim \operatorname{Id}_{X}\) sono \href{20250121094654-omotopia_tra_funzioni_continue.org}{omotope}.
\end{definizione}
\begin{oss}
Si hanno
\begin{equation*}
\begin{tikzcd}[ampersand replacement=\&,cramped,column sep=3.15em]
	A \& X
	\arrow["i_{A}", shift left, hook, from=1-1, to=1-2]
	\arrow["r", shift left, from=1-2, to=1-1]
\end{tikzcd}
\end{equation*}
con \(r\circ i_{A} = \operatorname{Id}_{A}\) e \(i_{A}\circ r\sim \operatorname{Id}_{X}\) \href{20250103103252-funzione_continua.org}{funzioni continue}.

Pertanto \(A\) e \(X\) sono \href{20250124155008-spazi_topologici_omotopicamente_equivalenti.org}{omotopicamente equivalenti}.
\end{oss}
\end{document}
