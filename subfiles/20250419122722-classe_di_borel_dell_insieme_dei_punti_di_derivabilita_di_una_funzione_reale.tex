% Intended LaTeX compiler: pdflatex
\documentclass[../main]{subfiles}

\usepackage[hyperref]{biblatex}
\date{}
\title{}
\begin{document}

\section{Classe di Borel dell'insieme dei punti di derivabilità di una funzione reale}
\label{sec:org249f564}
\subsection{Proprietà}
\label{sec:orga8906ef}

Given a continuous function \(f : [0, 1] \to \mathbb{R}\), let
\[
D_f = \set{ x \in [0, 1] \mid f' \text{ exists}}.
\]
(At endpoints we take one-sided derivatives.) Prove that \(D_f \in \bm{\Pi}^{0}_{3}\left([0,1]\right)\).
\subsubsection{Dimostrazione}
\label{sec:org7c78cd1}

Si ha che \(x \in D_{f}\) se e solo se \(x \in [0,1]\) e per ogni \(\varepsilon \in \R^{+}\) esiste \(\delta \in \R^{ +}\) tale che per ogni \(p, q \in [0,1]\):
\begin{equation*}
0<|p-x|,|q-x|< \delta
\quad\implies\quad
\left|\frac{f(p)-f(x)}{p-x}-\frac{f(q)-f(x)}{q-x}\right|\le\varepsilon
\end{equation*}
se e solo se \(x \in [0,1]\) e per ogni \(\varepsilon \in \Q^{+}\) esiste \(\delta \in \Q^{ +}\) tale che per ogni \(p, q \in [0,1]\cap \Q\):
\begin{equation*}
0<|p-x|,|q-x|< \delta
\quad\implies\quad
\left|\frac{f(p)-f(x)}{p-x}-\frac{f(q)-f(x)}{q-x}\right|\le\varepsilon
\end{equation*}
ovvero
\begin{equation*}
\left(
\begin{aligned}
0<|p-x| \,&\mathord{\wedge}\,|p-x|< \delta\\
&\mathord{\wedge}\\
0<|q-x| \,&\mathord{\wedge}\,|q-x|<\delta
\end{aligned}
\right)
\quad\implies\quad
\left|\frac{f(p)-f(x)}{p-x}-\frac{f(q)-f(x)}{q-x}\right|\le\varepsilon
\end{equation*}
ovvero
\begin{equation*}
\lnot\,\left(
\begin{aligned}
0<|p-x| \,&\mathord{\wedge}\,|p-x|< \delta\\
&\mathord{\wedge}\\
0<|q-x| \,&\mathord{\wedge}\,|q-x|<\delta
\end{aligned}
\right)
\,\lor\,
\left|\frac{f(p)-f(x)}{p-x}-\frac{f(q)-f(x)}{q-x}\right|\le\varepsilon
\end{equation*}
ovvero
\begin{multline*}
0\ge |p-x| \,\lor\, |p-x|\ge \delta \,\lor\, 0\ge|q-x| \,\lor\,\\  \,\lor\, |q-x|\ge \delta
\,\lor\, \left|\frac{f(p)-f(x)}{p-x}-\frac{f(q)-f(x)}{q-x}\right|\le\varepsilon.
\end{multline*}

Siano quindi
\begin{align*}
A_{p} &\coloneqq \set{x \in [0,1]\,\tc\, |p-x|=0} = \set{p}\\
B_{p,\delta} &\coloneqq \set{x \in [0,1]\,\tc\, |p-x|\ge \delta}\\
C_{q} &\coloneqq \set{x \in [0,1]\,\tc\, |q-x|=0} = \set{q}\\
D_{q,\delta} &\coloneqq \set{x \in [0,1]\,\tc\, |q-x|\ge \delta}\\
E_{p,q}^{\varepsilon} &\coloneqq \set{x \in [0,1]\,\tc\,\left|\frac{f(p)-f(x)}{p-x}-\frac{f(q)-f(x)}{q-x}\right|\le\varepsilon}
\end{align*}

Vale dunque l'uguaglianza
\begin{equation*}
D_{f} = \bigcap_{\varepsilon \in \Q^{+}} \bigcup_{\delta \in \Q^{ +}} \bigcap_{p,q \in [0,1]\cap \Q } A_{p}\cup B_{p,\delta}\cup C_{q}\cup D_{q,\delta}\cup E_{p,q}^{\varepsilon},
\end{equation*}
e pertanto:
\begin{itemize}
\item l'insieme \(V_{p,q}^{\varepsilon,\delta} \coloneqq A_{p}\cup B_{p,\delta}\cup C_{q}\cup D_{q,\delta}\cup E_{p,q}^{\varepsilon}\) è chiuso:
\begin{itemize}
\item i \(B_{p,\delta}\) e \(D_{q,\delta}\) sono chiusi;
\item si consideri ora la funzione continua:
\begin{align*}
	F: [0,1]\setminus\set{p,q} &\longrightarrow \R\\
	x&\longmapsto \frac{f(p)-f(x)}{p-x}-\frac{f(q)-f(x)}{q-x}
\end{align*}
pertanto \(E_{p,q}^{\varepsilon} = F^{-1}\left([-\varepsilon,\varepsilon]\right)\) è un chiuso di \([0,1]\setminus\set{p,q}\); esiste pertanto un \textbf{\textbf{chiuso}} \(W\) di \([0,1]\) tale che
\begin{equation*}
	E_{p,q}^{\varepsilon} = \left([0,1]\setminus\set{p,q}\right) \cap W = W\setminus\set{p,q}
\end{equation*}
e pertanto
\begin{equation*}
	W= E_{p,q}^{\varepsilon} \cup\set{p,q} = E_{p,q}^{\varepsilon} \cup A_{p}\cup C_{q};
\end{equation*}
\end{itemize}
\item l'insieme \(\bigcap_{p,q \in [0,1]\cap \Q} V_{p,q}^{\varepsilon,\delta}\) è chiuso, poiché intersezione di chiusi;
\item l'insieme \(\bigcup_{\delta \in \Q^{ +}} \bigcap_{p,q \in [0,1]\cap \Q } V_{p,q}^{\varepsilon,\delta}\) è un \(\bm{\Sigma}^0_2(X)\), poiché unione numerabile di chiusi;
\item l'insieme \(D_{f}\) è un \(\bm{\Pi}^0_3(X)\) poiché è intersezione numerabile di \(\bm{\Sigma}^0_2(X)\).\qed
\end{itemize}
\end{document}
