% Intended LaTeX compiler: pdflatex
\documentclass[../main]{subfiles}


\begin{document}

\begin{definizione}
Siano \(\mathcal{C},\mathcal{C'}\) due \href{20241126100904-categoria.org}{categorie}. Un \textbf{funtore covariante}
\begin{equation*}
F:\mathcal{C}\longrightarrow \mathcal{C'}
\end{equation*}
è una funzione tale che
\begin{enumerate}
\item per ogni \(X \in \operatorname{Ob}(\mathcal{C})\) si ha che \(F(X) \in \operatorname{Ob}(\mathcal{C'})\);
\item per ogni \(f \in \operatorname{Hom}_{\mathcal{C}} (X,Y)\) si ha che
\begin{equation*}
F(f)\in \operatorname{Hom}_{\mathcal{C'}}\left(F(X),F(Y)\right)
\end{equation*}
\end{enumerate}
tali che
\begin{enumerate}
\item \(F(f\circ g) = F(f)\circ F(g)\)
\item \(F(\1_X)=\1_{F(X)}\)
\end{enumerate}
\end{definizione}
\begin{esempio}
Un funtore covariante è il \href{20241204223712-gruppo_fondamentale.org}{gruppo fondamentale}\footnote{Vedi:
\begin{itemize}
\item \href{20241205115614-categoria_top.org}{Categoria-Top*}
\item \href{20241205115631-categoria_grp.org}{Categoria-Grp}
\end{itemize}}:
\begin{align*}
\Pi_1: \cat{Top*} &\longrightarrow \cat{Grp}\\
(X,p)&\longmapsto \Pi_1(X,p)
\end{align*}
dove se \(f:(X,p) \longrightarrow (Y,q)\) il gruppo fondamentale induce
\begin{align*}
\Pi_1(f): \Pi_1(X,p) &\longrightarrow \Pi_1(Y,q)\\
[\gamma]&\longmapsto [f\circ \gamma]
\end{align*}
e inoltre
\begin{align*}
    \Pi_1(f\circ g ) &= \Pi_1(f)\circ\Pi_1(g)\\
    \Pi_1\left(\id_{(X,p)}\right) &= \1_{\Pi_1(X,p)}.
\end{align*}
\end{esempio}
\uline{Notazione}
Se \(F\)  è un funtore \textbf{covariante} e \(X \dfreccia{f} Y\) allora
\begin{equation*}
F(X)\dfreccia{F(f)}F(Y)
\end{equation*}
e si indica con \(f_* \coloneqq F(f)\). \(f_*\) si dice il \textbf{push forward} di \(f\) tramite \(F\).
\end{document}
