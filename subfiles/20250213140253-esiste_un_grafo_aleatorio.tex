% Intended LaTeX compiler: pdflatex
\documentclass[../main]{subfiles}

\usepackage[hyperref]{biblatex}
\date{}
\title{}
\begin{document}

\section{Esiste un grafo aleatorio}
\label{sec:org9d836db}
\subsection{Proposizione}
\label{sec:org1205740}
Esiste un \href{20250213123032-teoria_dei_grafi.org}{grafo aleatorio} (ovvero la \href{20250130114950-teoria_del_prim_ordine.org}{teoria} \href{20250213123032-teoria_dei_grafi.org}{dei grafi aleatori} è \href{20250131123128-teoria_soddisfacibile.org}{soddisfacibile}).
\subsubsection{Dimostrazione}
\label{sec:orgdfaf1bf}
Sia \(p : \N\to \N\) la funzione che enumera i numeri primi.

Si consideri \(\langle\N, r\rangle\) con \(r\) \href{20250202170607-classe_relazione_binaria.org}{relazione binaria} tale che, per ogni \(n,m \in \N\),
\begin{equation*}
r(n,m)\qquad \iff\qquad \left[p(n)\ |\ m\quad\lor\quad p(m)\ |\ n\right].
\end{equation*}
(vedi \href{20250120122938-divisore.org}{Divisore}).

\(\langle N, r\rangle \vDash T_{\text{rg}}\) (vedi \href{20250131122913-soddisfazione_di_una_formula.org}{Soddisfazione di una formula}).
\end{document}
