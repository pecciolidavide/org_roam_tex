% Intended LaTeX compiler: pdflatex
\documentclass[../main]{subfiles}

\usepackage[hyperref]{biblatex}
\date{}
\title{}
\begin{document}

\section{K-Algebre fin generate e ridotte come anelli delle coordinate di varietà affini}
\label{sec:orgd31a4b6}
Sia \(\K\) un \href{20241231112713-campo_algebricamente_chiuso.org}{campo algebricamente chiuso}.
\subsection{Teorema}
\label{sec:orgd380d89}
Un \href{20241205141119-anello.org}{anello} \(R\) è una \href{20250110175552-algebra_su_un_campo.org}{\(\K\)-algebra} \href{20250110180014-algebra_su_un_campo_finitamente_generata.org}{finitamente generata} e \href{20250110181430-anello_ridotto.org}{ridotta} se e solo se esiste \(Y \subseteq \A^{n}\) \href{20241231114256-varieta_algebrica_affine.org}{varietà} \href{20241231114009-spazio_affine.org}{affine} tale che
\begin{equation*}
R\cong \K[Y]
\end{equation*}
(vedi \href{20250110174110-anello_delle_coordinate.org}{Anello delle coordinate}).
\subsubsection{Dimostrazione}
\label{sec:org318744f}

\paragraph{{\bfseries\sffamily TODO} Implicazione <-\hfill{}\textsc{matematica\_lm:geo\_alg}}
\label{sec:org3a1431f}
Ricordando la \href{20250110150011-corrispondenza_ideali_radicali_e_chiusi_algebrici_dello_spazio_affine.org}{Corrispondenza ideali radicali e chiusi algebrici dello spazio affine}, si ha che \(I(Y)\) è un \href{20250110142357-ideale_radicale.org}{ideale radicale} di \(\K[x_{1},\dots,x_{n}]\) (vedi \href{20241219113434-anello_dei_polinomi.org}{Anello-dei-polinomi}) e \href{20250110182255-ideale_radicale_e_anello_quoziente.org}{pertanto} \(\K[Y]\) anello quoziente è un \href{20250110181430-anello_ridotto.org}{Anello ridotto}.

Inoltre \(\K[Y]\) è \href{20250110180014-algebra_su_un_campo_finitamente_generata.org}{finitamente generato} perché lo è \(\K[x_{1},\dots,x_{n}]\).

Che sia una \(\K\) -algebra
\paragraph{{\bfseries\sffamily TODO} Implicazione ->\hfill{}\textsc{matematica\_lm:geo\_alg}}
\label{sec:org86815ac}
\end{document}
