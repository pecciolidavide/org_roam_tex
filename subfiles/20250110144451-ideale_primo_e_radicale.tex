% Intended LaTeX compiler: pdflatex
\documentclass[../main]{subfiles}

\usepackage[hyperref]{biblatex}
\date{}
\title{}
\begin{document}

\section{Ideale primo è radicale}
\label{sec:orgd5de16b}
Sia \(S\) un \href{20241205141119-anello.org}{anello commutativo}. Se \(I\) è un \href{20241219112955-ideale.org}{ideale} \href{20250103171055-ideale_primo.org}{primo}, allora \(I\) è \href{20250110142357-ideale_radicale.org}{radicale}.

Infatti, se \(f^{r} \in I\), allora \(f \cdots f^{r-1} \in I\).

Siccome l'ideale è primo, allora \(f \in I\) oppure \(f^{r-1} \in I\). Iterando si ottiene la tesi.
\end{document}
