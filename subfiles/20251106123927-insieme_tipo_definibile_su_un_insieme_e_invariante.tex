% Intended LaTeX compiler: pdflatex
\documentclass[../main]{subfiles}


\begin{document}

\def\D{\mathcal{D}}
\def\DD{\bm{\D}}
\def\U{\mathcal{U}}
\def\eq{{\rm eq}}
\def\Ueq{\U^\eq}
\def\L{\mathcal{L}}
\def\orbita{\mathcal{O}}
\def\Aut{\operatorname{Aut}}
\def\tc{\mid}
\def\tp{\operatorname{tp}}
\def\<{\langle}
\def\>{\rangle}
\def\b{\bm{b}}

\def\restricted#1{\,\mathord{\upharpoonright}{{\scriptstyle #1}}}
\def\equivalentover#1{\mathrel{\equiv_{ #1 }}}

%% NON FORKING
\def\nonforkSymbol{%
\mathbin{\raise1.8ex%
\rlap{\kern0.6ex\rule{0.6ex}{0.1ex}}%
\rlap{\kern1.1ex\rule{0.1ex}{1.9ex}}\raise-0.3ex\hbox{$\smile$}}}
\def\defaultnonforkmodel{M}
\def\nonfork{\nonforkSymbol}
\renewcommand{\nonfork}[1][\defaultnonforkmodel]{%
\mathrel{\nonforkSymbol_{#1}}}
\section{Insieme tipo-definibile su un insieme è invariante}
\label{sec:orgd7354a7}
Si utilizza la \href{20250612143636-notazione_teoria_dei_modelli.org}{notazione della Teoria dei Modelli}.

Sia \(\L\) un \href{20250130162057-linguaggio_del_prim_ordine.org}{linguaggio}, \(T\) una \href{20250130114950-teoria_del_prim_ordine.org}{teoria} \href{20250131123151-teoria_completa.org}{completa} senza \href{20250131122945-modello_di_un_insieme_di_formule.org}{modelli} finiti e \(\U\) un \href{20250617095548-modello_lambda_saturo.org}{modello saturo} di \href{20241213101756-cardinalita.org}{cardinalità} \href{20250211123155-cardinale_limite_forte.org}{inaccessibile} \(\kappa>\card{\L}+ \omega\). \(\U\) è un \href{20250617102733-modello_mostro.org}{modello mostro}.
\begin{prop}
Se \(B \subseteq \U\) è \href{20250212164424-tipo_teoria_dei_modelli.org}{tipo-definibile su \(A\)}, ovvero esiste \(p(x) \subseteq \L(A)\)\footnote{Vedi ``\href{20250212102927-enunciato_con_parametri.org}{Formula con parametri}'' e ``\href{20250212164424-tipo_teoria_dei_modelli.org}{Tipo - Teoria dei Modelli}''} tale che \(B= p(\U^{x})\)\footnote{Vedi ``\href{20250212164424-tipo_teoria_dei_modelli.org}{Soddisfazione di un tipo}''}, allora \(B\) è \href{20251020150308-insieme_invariante_su_un_insieme_in_un_modello_mostro.org}{invariante su \(A\)}.
\end{prop}
\begin{proof}
Sia \(b \in B\) e sia \(f \in \Aut(\U/A)\).
\begin{equation*}
b \in p(\U^{x}) = \bigcap_{\varphi \in p} \varphi(\U^{x})
\end{equation*}
e quindi
\begin{equation*}
f(b) \in f[p(\U^{x})] = \bigcap_{\varphi \in p} f[\varphi(\U^{x})] = \bigcap_{\varphi \in p} \varphi(\U^{x}) = p(\U^{x})
\end{equation*}
come dimostrato in ``\href{20250617103021-insieme_definibil_e_automorfismi_in_un_modello_mostro.org}{Insieme definibile su un insieme è invariante}'' e ``\href{20251106125420-scambio_di_unioni_intersezioni_e_immagini_retroimmagini_di_funzione.org}{Scambio di unioni/intersezioni e immagini/retroimmagini di funzione}''.
\end{proof}
\end{document}
