% Intended LaTeX compiler: pdflatex
\documentclass[../main]{subfiles}


\begin{document}

\section{Omologia per attaccamento di k-cella}
\label{sec:org0c5ecc3}
Sia \(R\) un \href{20241219112842-pid.org}{PID}.

Sia \(X\) uno \href{20250103145124-topologia.org}{spazio topologico} di \href{20250109155715-spazio_topologico_di_hausdorff.org}{Haussrdof}, \(e^{k}\) una \(k\)-cella, \(\varphi: \partial e^{k}\to X\) \href{20250103103252-funzione_continua.org}{continua} e
\begin{equation*}
Y \coloneqq X \mathrel{\cup_{\varphi}} e^k
\end{equation*}
l'\href{20250215122759-export.org_archive}{attaccamento di \(e^{k}\) ad \(X\)}.

\begin{prop}
La mappa \(\Phi: (e^{k}, \partial e^{k})\longrightarrow \big(Y, j(X)\big)\) \href{20250126191208-funtore_da_topp_a_rmod_di_omologia.org}{induce} un \href{20241206115416-morfismi_r_moduli.org}{isomorfismo} in \href{20250122154903-omologia_singolare_relativa.org}{omologia relativa}:
\begin{equation*}
\forall q:\qquad H_{q}(e^{k},\partial e^{k}) \cong H_{q}\big(Y,j(X)\big).
\end{equation*}
\end{prop}
\begin{proof}
Sia \(C\) un \href{20250215122759-export.org_archive}{collare} di \(\partial e^{k}\).
\begin{enumerate}
\item Abbiamo il seguente \href{20250111142332-omeomorfismo.org}{omeomorfismo}:
\begin{equation*}
\restriction{\Phi}{e^{k}\setminus \partial e^{k}}:(e^{k}\setminus \partial e^{k}, C\setminus \partial e^{k})\longrightarrow \big(
 Y \setminus j(X),\ (j(X) \cup \Phi(C))\setminus j(X)
\big)
\end{equation*}
che \href{20250126191208-funtore_da_topp_a_rmod_di_omologia.org}{induce} un \href{20241206115416-morfismi_r_moduli.org}{isomorfismo}
\begin{equation*}
\forall  n:\qquad H_{n}(e^{k}\setminus \partial e^{k}, C\setminus \partial e^{k})\cong H_{n}\big(
 Y \setminus j(X),\ (j(X) \cup \Phi(C))\setminus j(X)
\big)
\end{equation*}

\item Poiché \(\partial e^{k} \subseteq C\) e \(j(X) \subseteq j(X) \cup \Phi(C)\) sono \href{20250122155727-retratto_di_deformazione_di_uno_spazio_topologico.org}{retratti di deformazione}, \href{20250126190440-equivalenze_omotopiche_tra_coppie_topologiche_induce_isomorfismo_tra_omologia_singolare_relativa.org}{allora}
\begin{align*}
 \forall n:\qquad H_{n}\big(Y,j(X)\big) & \cong \big(Y, j(X) \cup \Phi(C)\big)\\
 \forall n:\qquad H_{n}(e^{k},\partial e^{k}) & \cong H_{n}(e^{k}, C)
\end{align*}
\item Per il \href{20250126223310-teorema_di_escissione.org}{Teorema di Escissione}:
\begin{align*}
  H_{n}(e^{k}\setminus \partial e^{k}, C\setminus \partial e^{k}) & \cong H_{n}(e^{k}, C)\\
  H_{n}\big(
         Y \setminus j(X),\ (j(X) \cup \Phi(C))\setminus j(X)
  \big) & \cong H_{n}\big(
         Y,\ (j(X) \cup \Phi(C))
  \big)
\end{align*}
\end{enumerate}
Tutto questo dà l'isomorfismo richiesto:
\begin{equation*}
\begin{tikzcd}[ampersand replacement=\&]
	{H_{n}(e^{k},\partial e^{k})} \&\& {H_{n}(e^{k}, C)} \&\& {H_{n}(e^{k}\setminus \partial e^{k}, C\setminus \partial e^{k})} \\
	\\
	{H_{n}\big(Y,j(X)\big)} \&\& {H_n\big(Y, j(X) \cup \Phi(C)\big)} \&\& {H_{n}\big(Y \setminus j(X),\ (j(X) \cup \Phi(C))\setminus j(X)\big)}
	\arrow["{{\cong }}", from=1-1, to=1-3]
	\arrow["{{2.}}"', draw=none, from=1-1, to=1-3]
	\arrow["\cong"', from=1-5, to=1-3]
	\arrow["{{3.}}", draw=none, from=1-5, to=1-3]
	\arrow["\cong", from=1-5, to=3-5]
	\arrow["{{1.}}"', draw=none, from=1-5, to=3-5]
	\arrow["\cong"', from=3-1, to=3-3]
	\arrow["{{2.}}", draw=none, from=3-1, to=3-3]
	\arrow["\cong", from=3-5, to=3-3]
	\arrow["{{3.}}"', draw=none, from=3-5, to=3-3]
\end{tikzcd}
\qedhere
\end{equation*}
\end{proof}

\begin{prop}
Per l'\href{20250122133631-omologia_singolare.org}{omologia singolare}:
\begin{enumerate}
\item Per ogni \(q \neq k,k-1\):
\begin{equation*}
 H_{q}(Y) \cong H_{q}(X);
\end{equation*}
\item Vale una e una sola delle seguenti:
\begin{itemize}
\item si ha un \href{20241206115416-morfismi_r_moduli.org}{isomorfismo} con la \href{20241213095808-somma_diretta.org}{somma diretta}
\begin{equation*}
   H_{k}(Y) \cong H_{k}(X) \oplus R
\end{equation*}
e, se \(H_{k-1}(X)\) è \href{20241213100845-modulo_finitamente_generato.org}{finitamente generato}, allora lo è anche \(H_{k-1}(Y)\) e vale la seguente formula per il \href{20260109173733-rango_di_un_modulo.org}{rango}:
\begin{equation*}
   \operatorname{rk}\big(H_{k-1}(Y)\big) = \operatorname{rk}\big(H_{k-1}(X)\big);
\end{equation*}
\item si ha un \href{20241206115416-morfismi_r_moduli.org}{isomorfismo}:
\begin{equation*}
   H_{k}(Y) \cong H_{k}(X)
\end{equation*}
e, se \(H_{k-1}(X)\) è \href{20241213100845-modulo_finitamente_generato.org}{finitamente generato}, allora lo è anche \(H_{k-1}(Y)\) e vale la seguente formula per il \href{20260109173733-rango_di_un_modulo.org}{rango}:
\begin{equation*}
   \operatorname{rk}\big(H_{k-1}(Y)\big) = \operatorname{rk}\big(H_{k-1}(X)\big) - 1.
\end{equation*}
\end{itemize}
\end{enumerate}
\end{prop}
\begin{proof}
Si noti che per la proposizione precedente:
\begin{equation*}
     H_{q}\big(Y,j(X)\big) \cong H_{q}(e^{k}, \partial e^{k}) = \begin{cases}
             R & q = k\\
             0 & q \neq k
     \end{cases}
\end{equation*}
per l'\href{20250127162702-calcolo_dell_omologia_singolare_della_sfera_e_dell_omologia_singolare_relativa_del_disco_rispetto_alla_sfera.org}{omologia relativa già calcolata}.

\begin{enumerate}
\item Si scriva la \href{20250122154927-successione_esatta_di_una_coppia_topologica.org}{SEL per l'omologia relativa}:
\begin{equation*}
\begin{tikzcd}[ampersand replacement=\&]
        {H_{q+1}\big(Y,j(X)\big)} \& {H_{q}\big(j(X)\big)} \& {H_{q}(Y)} \& {H_{q}\big(Y,j(X)\big)} \\
        \& {H_q(X)}
        \arrow[from=1-1, to=1-2]
        \arrow[from=1-2, to=1-3]
        \arrow["\cong"{marking, allow upside down}, draw=none, from=1-2, to=2-2]
        \arrow[from=1-3, to=1-4]
\end{tikzcd}
\end{equation*}
Per \(q \neq k\), \(q\neq k-1\), allora \(q \neq k\) e \(q+1 \neq k\), e quindi
\begin{equation*}
 H_{q+1}\big(Y,j(X)\big) = 0,\qquad H_{q}\big(Y,j(X)\big) = 0.
\end{equation*}
\href{20250120130155-caratterizzazione_di_alcune_successioni_esatte_di_r_moduli.org}{Pertanto}, \(H_{q}(X) \cong H_{q}(j(X)) \cong H_{q}(Y)\)
\item Si consideri sempre la SEL per l'omologia relativa, vicino a \(q=k\).
\begin{equation*}
\scalebox{0.8}{%
\begin{tikzcd}[ampersand replacement=\&,column sep=small]
        {H_{k+1}\big(Y,j(X)\big)} \& {H_{k}\big(j(X)\big)} \& {H_{k}(Y)} \& {H_{k}\big(Y,j(X)\big)} \& {H_{k-1}\big(j(X)\big)} \& {H_{k-1}(Y)} \& {H_{k-1}\big(Y,j(X)\big)} \\
        \& {H_k(X)} \&\&\& {H_{k-1}(X)}
        \arrow[from=1-1, to=1-2]
        \arrow[from=1-2, to=1-3]
        \arrow["\cong"{marking, allow upside down}, draw=none, from=1-2, to=2-2]
        \arrow[from=1-3, to=1-4]
        \arrow[from=1-4, to=1-5]
        \arrow[from=1-5, to=1-6]
        \arrow["\cong"{marking, allow upside down}, draw=none, from=1-5, to=2-5]
        \arrow[from=1-6, to=1-7]
\end{tikzcd} %
}
\end{equation*}
e si facciano le dovute sostituzioni
\begin{equation*}
\begin{tikzcd}[ampersand replacement=\&]
        0 \& {H_k(X)} \& {H_{k}(Y)} \& R \& {H_{k-1}(X)} \& {H_{k-1}(Y)} \& 0
        \arrow[from=1-1, to=1-2]
        \arrow[from=1-2, to=1-3]
        \arrow[from=1-3, to=1-4]
        \arrow["\alpha", from=1-4, to=1-5]
        \arrow[from=1-5, to=1-6]
        \arrow[from=1-6, to=1-7]
\end{tikzcd} %
\end{equation*}
Siccome \(R\) è un \(R\)-modulo libero, ed \(R\) è un PID, \href{20241219192830-pid_sottomoduli_sono_liberi.org}{allora ogni suo sottomodulo è libero}. In particolare lo è il \href{20241213105201-kernel.org}{kernel} \(\ker\alpha\).
Ci sono solo due casi possibili:

\begin{enumerate}
\item \(\boxed{\ker \alpha \cong R}\): allora sicuramente la seguente è esatta:
\begin{equation*}
\begin{tikzcd}[ampersand replacement=\&]
        0 \& {H_k(X)} \& {H_{k}(Y)} \& {\ker\alpha} \& 0
        \arrow[from=1-1, to=1-2]
        \arrow[from=1-2, to=1-3]
        \arrow[from=1-3, to=1-4]
        \arrow[from=1-4, to=1-5]
\end{tikzcd}
\end{equation*}
e siccome \(\ker\alpha \cong R\) è \href{20241213094625-modulo_libero.org}{libero}, \href{20250120131729-teorema_di_spezzamento_sec.org}{si ha che}
\begin{equation*}
  H_{k}(Y) \cong H_{k}(X) \oplus R.
\end{equation*}

Per la seconda parte, per il \href{20250120155457-morfismo_iniettivo_di_r_moduli_induce_isomorfismo.org}{primo teorema di isomorfismo} questa è una \href{20250120125004-successione_di_r_moduli_esatta.org}{successione esatta corta}:
\begin{equation*}
\begin{tikzcd}[ampersand replacement=\&]
        0 \& {R/\ker\alpha} \&\& {H_{k-1}(X)} \&\& {H_{k-1}(Y)} \& 0
        \arrow[from=1-1, to=1-2]
        \arrow["{\overline{\alpha}}"', from=1-2, to=1-4]
        \arrow[from=1-4, to=1-6]
        \arrow[from=1-6, to=1-7]
\end{tikzcd}
\end{equation*}
\begin{itemize}
\item \href{20250120121333-quoziente_di_modulo_fg_e_fg.org}{\(R/\ker \alpha\) è finitamente generato}. Inoltre è \href{20250120103129-modulo_di_torsione.org}{di torsione}.

Infatti, siccome \(\ker\alpha \subseteq R\) è sottomodulo, allora è ideale di un PID, e pertanto esiste \(r \in R\), \(r \neq 0\) tale che\footnote{Vedi l'\href{20241219113154-ideale_generato.org}{ideale generato da un elemento}}
\begin{equation*}
\ker\alpha = (r)
\end{equation*}
Allora per ogni \(p + \ker \alpha \in R/\alpha\), \(r\cdot [p] = 0\).

Pertanto ha \href{20260109173733-rango_di_un_modulo.org}{rango nullo}.

\item Se \(H_{k-1}(X)\) è finitamente generato, allora lo è anche \(H_{k-1}(Y)\).\footnote{Questo segue dal fatto che \(H_{k-1}(Y)\) è l'immagine di \(H_{k-1}(X)\) tramite morfismo.}

Pertanto \href{20260206182305-rango_in_una_successione_di_r_moduli_finitamente_generati.org}{vale la formula}
\begin{equation*}
\operatorname{rk}H_{k-1}(X) = \operatorname{rk}H_{k-1}(Y) + \parentesi{=0}{\operatorname{rk}(R/\ker\alpha)}.
\end{equation*}
\end{itemize}
\end{enumerate}
\end{enumerate}


\begin{enumerate}
\item \(\boxed{\ker \alpha = 0}\): allora la seguente è esatta:
\begin{equation*}
\scalebox{0.93}{%
\begin{tikzcd}[ampersand replacement=\&]
        0 \& {H_k(X)} \& {H_{k}(Y)} \&\& R \& {H_{k-1}(X)} \& {H_{k-1}(Y)} \& 0 \\
        \&\&\& {\ker\alpha}
        \arrow[from=1-1, to=1-2]
        \arrow[from=1-2, to=1-3]
        \arrow["0"', from=1-3, to=2-4]
        \arrow["\alpha", from=1-5, to=1-6]
        \arrow[from=1-6, to=1-7]
        \arrow[from=1-7, to=1-8]
        \arrow[hook, from=2-4, to=1-5]
\end{tikzcd}%
}
\end{equation*}
e pertanto lo sono entrambe le seguenti:
\begin{equation*}
\begin{tikzcd}[ampersand replacement=\&]
        \& 0 \& {H_k(X)} \& {H_{k}(Y)} \& 0 \\
        0 \& R \& {H_{k-1}(X)} \& {H_{k-1}(Y)} \& 0
        \arrow[from=1-2, to=1-3]
        \arrow[from=1-3, to=1-4]
        \arrow[from=1-4, to=1-5]
        \arrow[from=2-1, to=2-2]
        \arrow["\alpha", from=2-2, to=2-3]
        \arrow[from=2-3, to=2-4]
        \arrow[from=2-4, to=2-5]
\end{tikzcd}
\end{equation*}
\href{20250120130155-caratterizzazione_di_alcune_successioni_esatte_di_r_moduli.org}{Segue} che \(H_{k}(X) \cong H_{k}(Y)\), e \(H_{k-1}(Y) \cong H_{k-1}(X) / R\).

Quindi, se \(H_{k-1}(X)\) è finitamente generato, \href{20250120121333-quoziente_di_modulo_fg_e_fg.org}{allora} \(H_{k-1}(Y)\) è finitamente generato, e inoltre per il \href{20260109173733-rango_di_un_modulo.org}{rango} \href{20260206182305-rango_in_una_successione_di_r_moduli_finitamente_generati.org}{vale}:
\begin{equation*}
  \operatorname{rk} H_{k-1}(X) = \operatorname{rk} H_{k-1}(Y) + \parentesi{=1}{\operatorname{rk} R}.
\end{equation*}
\end{enumerate}
\end{proof}
\end{document}
