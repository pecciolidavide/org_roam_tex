% Intended LaTeX compiler: pdflatex
\documentclass[../main]{subfiles}


\begin{document}

\section{Sottomoduli}
\label{sec:org98ea0cb}
Sia \(R\) un \href{20241205141119-anello.org}{anello commutativo con unità}.
\begin{definizione}
Sia \(M\) un \href{20241205141053-r_moduli.org}{\(R\)-modulo}. \(N\)  si dice un \textbf{sottomodulo} di \(M\), e si scrive
\begin{equation*}
N \mathrel{\subseteq_R} M
\end{equation*}
un \href{20241206143051-sottogruppo.org}{\textbf{sottogruppo}} di \(M\) tale che
\begin{equation*}
\forall\,a \in R,\ \forall\,n \in N:\quad an \in N.
\end{equation*}
\end{definizione}
\begin{prop}
Sia \(f: M_1 \longrightarrow M_2\) un \href{20241206115416-morfismi_r_moduli.org}{morfismo tra \(R\)-moduli}, e sia (vedi \href{20241213105201-kernel.org}{Kernel})
\begin{equation*}
\operatorname{Ker}(f)\coloneqq \set{m \in M_1: f(m)=0}
\end{equation*}

Allora \(\operatorname{Ker}(f) \mathrel{\subseteq_R} M_1\). Inoltre \(\operatorname{Im}(f)= f(M_1) \mathrel{\subseteq_R} M_2\)
\end{prop}
\subsection{Sottomoduli dell'anello sono ideali}
\label{sec:org3231b19}
\begin{oss}
\href{20260112124828-ogni_anello_ha_struttura_di_modulo_su_se_stesso.org}{Se si considera \(R\) come un \(R\)-modulo}, allora ogni sottomodulo di \(R\) è un suo ideale.
\end{oss}
\section{Quoziente di moduli}
\label{sec:orgf6cd0d3}
Se \(M,N\) sono due \href{20241205141053-r_moduli.org}{\(R\)-moduli} e \(N  \mathrel{\subseteq_R} M\) è \hyperref[sec:org98ea0cb]{sottomodulo}, allora il \href{20250127093245-gruppo_abeliano.org}{gruppo abeliano} del \hyperref[sec:org3938071]{quoziente} \(M/N\) eredita una struttura di \(R\)-modulo data da
\begin{align*}
R\times M/N &\longrightarrow M/N\\
\left(r,m+N\right)&\longmapsto rm+N
\end{align*}

Anche citato come:
\subsection{Quoziente di Moduli}
\label{sec:org3938071}
\subsection{Quoziente di moduli è invariante per iniezione}
\label{sec:org7a1fade}
Se \(M,N\) sono due \href{20241205141053-r_moduli.org}{\(R\)-moduli} e \(N  \mathrel{\subseteq_R} M\) è \hyperref[sec:org98ea0cb]{sottomodulo}. Sia \(\varphi: M\to P\) un \href{20241206115416-morfismi_r_moduli.org}{morfismo di moduli}.

\begin{lem}
Se il morfismo \(\varphi: M\to P\) è \textbf{\href{20241219101956-funzione_iniettiva.org}{iniettivo}}, allora induce un \href{20241206115416-morfismi_r_moduli.org}{isomorfismo} con il \hyperref[sec:org3938071]{quoziente} tra le \href{20250202190147-immagine_punto_a_punto_di_due_classi.org}{immagini}:
\begin{equation*}
\frac{M}{N} \cong \frac{\varphi[M]}{\varphi[N]}.
\end{equation*}
\end{lem}
\section{Somma di sottomoduli}
\label{sec:org0381cb5}
Se \(M\) è un \href{20241205141053-r_moduli.org}{\(R\)-modulo} e \(A_{1},A_{2}\mathrel{\subseteq_{R}} M\) sono \hyperref[sec:org98ea0cb]{sottomoduli}, allora si definisce il \hyperref[sec:org98ea0cb]{sottomodulo} \(A_{1}+A_{2}\mathrel{\subseteq_{R}} M\):
\begin{equation*}
A_{1}+A_{2} \coloneqq \set{a_{1}+a_{2} \in M: a_{1} \in A_{1},\ a_{2} \in A_{2}}
\end{equation*}

\begin{prop}
Si ha che\footnote{Vedi:
\begin{itemize}
\item \hyperref[sec:org3938071]{Quoziente di Moduli}
\item \href{20241213095808-somma_diretta.org}{Somma Diretta di moduli}
\item \hyperref[sec:org9d3fe9f]{Intersezione di sottomoduli}
\end{itemize}}
\begin{equation*}
A_{1}+A_{2}\cong \frac{A_{1}\oplus A_{2}}{A_{1}\cap A_{2}}
\end{equation*}
sono \href{20241206115416-morfismi_r_moduli.org}{isomorfi}.
\end{prop}
\begin{proof}
Tale \href{20241206115416-morfismi_r_moduli.org}{isomorfismo} è \href{20250120130155-caratterizzazione_di_alcune_successioni_esatte_di_r_moduli.org}{indotto} dalla seguente \href{20250120131527-sec.org}{SEC}:
\begin{equation*}
0 \longrightarrow A_{1} \cap A_{2} \xrightarrow{\,\alpha\,} A_{1} \oplus A_{2} \xrightarrow{\,\beta\,} A_{1} + A_{2} \longrightarrow 0
\end{equation*}

I \href{20241206115416-morfismi_r_moduli.org}{morfismi} coinvolti sono definiti esplicitamente come segue:

\begin{enumerate}
\item Il morfismo \(\alpha\):
\begin{equation*}
\alpha: A_{1} \cap A_{2} \longrightarrow A_{1} \oplus A_{2}
\end{equation*}
\begin{equation*}
\alpha(x) = (x, x)
\end{equation*}

\item Il morfismo \(\beta\):
\begin{equation*}
\beta: A_{1} \oplus A_{2} \longrightarrow A_{1} + A_{2}
\end{equation*}
\begin{equation*}
\beta(a_{1}, a_{2}) = a_{1} - a_{2}
\end{equation*}
\end{enumerate}

È sufficiente dimostrare l'esattezza.
\begin{itemize}
\item Ovviamente \(\alpha\) è iniettivo e \(\beta\) è suriettivo. \href{20250120130155-caratterizzazione_di_alcune_successioni_esatte_di_r_moduli.org}{Questo garantisce} l'esattezza nei due estremi.
\item Resta quindi l'esattezza centrale, ovvero da dimostrare che \(\ker(\beta) = \operatorname{Im}(\alpha)\)\footnote{Vedi \href{20241213105201-kernel.org}{kernel} e \href{20250202190147-immagine_punto_a_punto_di_due_classi.org}{immagine}.}.

\((a_{1},a_{2}) \in \ker\beta\) se e solo se \(a_{1}-a_{2}=0\) se e solo se \(a_{1}=a_{2}\): se e solo se
\begin{itemize}
\item \(a_{1}= a_{2} = a\in A_{1}\cap A_{2}\)

\item \((a_{1},a_{2}) = \alpha(a)\).\qedhere
\end{itemize}
\end{itemize}
\end{proof}
\section{Intersezione di sottomoduli}
\label{sec:org9d3fe9f}
\lipsum[1]
\href{20250515141706-da_finire.org}{DA FINIRE}
\end{document}
