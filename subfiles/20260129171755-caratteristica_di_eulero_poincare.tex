% Intended LaTeX compiler: pdflatex
\documentclass[../main]{subfiles}


\begin{document}

\section{Caratteristica di Eulero-Poincaré per una superficie di Riemann}
\label{sec:org7d1e20b}
\begin{definizione}
La \textbf{caratteristica di Eulero} di una \href{20260127112828-superficie_di_riemann.org}{superficie di Riemann} \href{20250103163701-spazio_topologico_compatto.org}{compatta} \(X\), indicata con \(\chi(X)\), è la sua \href{20260130154906-caratteristica_di_eulero.org}{caratteristica di Eulero} come \href{20251230172241-superficie_topologica.org}{superficie topologica} (\href{20260130155129-superficie_topologica_orientabile.org}{orientabile}):\footnote{\(g(X)\) è il \href{20260127112828-superficie_di_riemann.org}{genere topologico} di \(X\).}
\begin{equation*}
\chi(X) = 2-2g(X) = V-L+ F
\end{equation*}
dove, fissata una qualsiasi \href{20260130154758-triangolazione_di_una_superficie_topologica.org}{triangolazione}:
\begin{itemize}
\item \(V\) è il numero di vertici;
\item \(L\) è il numero di lati;
\item \(F\) è il numero di facce.
\end{itemize}
\end{definizione}
\end{document}
