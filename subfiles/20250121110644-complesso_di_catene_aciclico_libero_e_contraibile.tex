% Intended LaTeX compiler: pdflatex
\documentclass[../main]{subfiles}

\usepackage[hyperref]{biblatex}
\date{}
\title{}
\begin{document}

\section{Complesso di catene aciclico libero è contraibile}
\label{sec:orgc8c4986}
\begin{prop}
Sia \(R\) un \href{20241219112842-pid.org}{PID}, sia \(\mathcal{C}_{\bullet}\) un \href{20250120163114-complesso_di_catene.org}{complesso di catene} \href{20250121110816-complesso_di_catene_libero.org}{libero}.
Se \(\mathcal{C}_{\bullet}\) è \href{20250121104306-complesso_di_catene_aciclico.org}{aciclico}, allora \(\mathcal{C}_{\bullet}\) è \href{20250121104453-complesso_di_catene_contraibile.org}{contraibile}.
\end{prop}
\begin{proof}
Sia
\begin{equation*}
\mathcal{C}_{\bullet} = \set{(C_{n},\partial_{n})}_{n \in \Z}
\end{equation*}

Per ogni \(n \in \Z\), siccome \(R\) è un PID e \(C_{n}\) è un \href{20241213094625-modulo_libero.org}{modulo libero}, \href{20241219192830-pid_sottomoduli_sono_liberi.org}{allora} i \href{20241206142802-sottomoduli.org}{sottomoduli} \(B_{n},Z_{n} \subseteq_{R} C_{n}\)\footnote{\(B_{n}\) e \(Z_{n}\) sono definiti in ``\href{20250120164857-modulo_di_omologia_dei_complessi_di_catene.org}{Modulo di omologia dei complessi di catene}''} sono liberi.

Sia \(E_{n}\) una \href{20241213094625-modulo_libero.org}{base} di \(Z_{n} = \operatorname{ker}\partial_{n}\)\footnote{Vedi ``\href{20241213105201-kernel.org}{Kernel}''}. Per ogni \(e_{i} \in E_{n}\) si fissi \(\tilde{e}_{i} \in \partial_{n+1}^{-1}(e_{i}) \subseteq C_{n+1}\).

Si definisce il \href{20241206115416-morfismi_r_moduli.org}{morfismo}
\begin{align*}
\sigma_{n}:Z_{n} &\longrightarrow C_{n+1}\\
\sum a_{i}\ e_{i} &\longmapsto \sum a_{i}\ \tilde{e}_{i}
\end{align*}
Osserviamo che \(\partial_{n+1}\circ \sigma_{n} = \Id\)\footnote{Infatti si ha che
\begin{equation*}
\partial_{n+1}\circ \sigma_{n}\big(\sum a_{i} e_{i}\big) = \partial_{n+1} \big(\sum a_{i} \tilde{e}_{i}\big) = \sum a_{i}\, \partial_{n+1}(\tilde{e}_{i}) =\sum a_{i} e_{i}.
\end{equation*}}, (\(\star\)).

Si definisce inoltre il morfismo \(\tau_{n}: C_{n}\to C_{n}\):
\(\tau_{n} = \1_{C_{n}} - \sigma_{n-1}\circ \partial_{n}\)

Osserviamo che, per ogni \(c_{n} \in C_{n}\):
\begin{align*}
\partial_{n}\circ \tau_{n}(c_{n})  &= \partial_{n}\left(c_{n} - \sigma_{n-1}(\partial_{n} c_{n})\right)\\
&= \partial_{n}c_{n} - \parentesi{= \Id}{\partial_{n}\circ\sigma_{n-1}}(\partial_{n}c_{n})\\
&= \partial_{n}c_{n} - \partial_{n}c_{n} = 0.
\end{align*}
Pertanto \(\operatorname{Im}\tau_{n} \subseteq \operatorname{ker}\partial_{n} = Z_{n}\) e dunque \(\tau_{n}: C_{n}\to Z_{n}\)

Definisco \(s_{n}: C_{n}\to C_{n+1}\) \href{20250121094935-omotopia_tra_morfismi_di_complessi_di_catene.org}{mappa di omotopia}
\begin{equation*}
s_{n} = \sigma_{n}\circ \tau_{n}
\end{equation*}

Verifichiamo in particolar modo che valga
\begin{equation*}
s_{n-1}\circ \partial_{n} + \partial_{n+1}\circ \partial_{n} = \1_{n} - \mathds{O}_{n} = \1_{n}
\end{equation*}
Dunque
\begin{align*}
s_{n-1}\circ \partial_{n} (c_{n}) + \partial_{n+1}\circ s_{n}(c_{n}) &= \sigma_{n-1}\circ\tau_{n-1}\circ\partial_{n} c_{n} + \parentesi{= \1_{n}}{\partial_{n+1}\circ\sigma_{n}}\circ\tau_{n}(c_{n})\\
&= \sigma_{n-1}\left(\partial_{n}c_{n} - \sigma_{n-2}\circ\partial_{n-1}\circ\partial_{n}c_{n}\right) + \tau_{n}(c_{n})\\
&= \sigma_{n-1}\circ\partial_{n}c_{n} + \tau_{n}(c_{n})\\
&= \sigma_{n-1}\circ\partial_{n}c_{n} +\left( c_{n} - \sigma_{n-1}\circ \partial_{n}c_{n}\right) = c_{n} = \1_{n}(c_{n}).
\qedhere
\end{align*}
\end{proof}
\end{document}
