% Intended LaTeX compiler: pdflatex
\documentclass[../main]{subfiles}


\begin{document}

\section{Mappe di bordo tra moduli di catene simpliciali}
\label{sec:org754da7f}
Sia \(R\) un \href{20241219112842-pid.org}{PID}.

Sia \(K\) un \href{20250121124147-complesso_simpliciale.org}{complesso simpliciale} \href{20250121131005-complesso_simpliciale_totalmente_ordinato.org}{totalmente ordinato}, e sia per ogni \(q\) l'\href{20241205141053-r_moduli.org}{\(R\)-modulo} \href{20250121124410-modulo_delle_catene_simpliciali.org}{delle catene simpliciali \(C_{q}(K)\)}. Si definisce il \href{20241206115416-morfismi_r_moduli.org}{morfismo}:
\begin{align*}
\partial_{q}: C_{q}(K) &\longrightarrow C_{q-1}(K)\\
[p_{0},\dots,p_{q}] &\longmapsto \sum_{i=0}^{q} (-1)^{i} [p_{0},\dots,\hat{p}_{i},\dots,p_{q}]
\end{align*}
\begin{prop}
Per ogni \(q\) si ha che \(\partial_{q-1}\partial_{q} = 0\).
\end{prop}
\begin{proof}
\href{20241213095808-somma_diretta.org}{Siccome} \(C_{q}(K)\) è \href{20241213094625-modulo_libero.org}{libero}, è sufficiente mostrarlo su una \href{20241213094625-modulo_libero.org}{base}.

Sia \([p_{0},\dots,p_{q}] \in K^{q}\) un \(q\)-\href{20250121121923-simplessi.org}{simplesso}. Per definizione:
\begin{equation*}
\partial_{q}[p_{0},\dots,p_{q}] = \sum_{i=0}^{q}(-1)^{i} [p_{0},\dots,\hat{p}_{i},\dots,p_{q}].
\end{equation*}
Dunque
\begin{align*}
\partial_{q-1}&\left(\partial_{q}[p_{0},\dots,p_{q}]\right) %
	= \sum_{i=0}^{q} (-1)^{i} \partial_{q-1}[p_{0},\dots,\hat{p}_{i},\dots,p_{q}]\\
&= \sum_{i=0}^{q} (-1)^{i} \left[ %
	\sum_{j=0}^{i-1}(-1)^{j} [p_{0},\dots,\hat{p}_{j},\dots,\hat{p}_{i},\dots,p_{q}] + %
	\sum_{j=i}^{q-1}(-1)^{j} [p_{0},\dots,\hat{p}_{i},\dots,\hat{p}_{j+1},\dots,p_{q}] %
\right]\\
&=
	\sum_{\substack{j<i\\ i,j=0}}^{q} (-1)^{i+j} [p_{0},\dots,\hat{p}_{j},\dots,\hat{p}_{i}, \dots, p_{q}] +
	\sum_{\substack{j+1>i\\ i,j=0}}^{q} - (-1)^{i+j+1} [p_{0},\dots,\hat{p}_{i},\dots,\hat{p}_{j+1},\dots,p_{q}]\\
&\underset{\dagger}{=}
	\sum_{\substack{j<i\\ i,j=0}}^{q} (-1)^{i+j} [p_{0},\dots,\hat{p}_{j},\dots,\hat{p}_{i}, \dots, p_{q}] +
	\sum_{\substack{i'>j'\\ i',j'=0}}^{q} - (-1)^{i'+j'} [p_{0},\dots,\hat{p}_{j'},\dots,\hat{p}_{i'},\dots,p_{q}]\\
& =
	\sum_{\substack{j<i\\ i,j=0}}^{q} (-1)^{i+j} [p_{0},\dots,\hat{p}_{j},\dots,\hat{p}_{i}, \dots, p_{q}]
	- \sum_{\substack{i'>j'\\ i',j'=0}}^{q} (-1)^{i'+j'} [p_{0},\dots,\hat{p}_{j'},\dots,\hat{p}_{i'},\dots,p_{q}] = 0
\end{align*}
dove l'uguaglianza (\(\dagger\)) si è ottenuta ponento \(j'=i\) e \(i'=j+1\).
\end{proof}
\end{document}
