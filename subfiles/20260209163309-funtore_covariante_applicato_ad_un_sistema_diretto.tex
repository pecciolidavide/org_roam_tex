% Intended LaTeX compiler: pdflatex
\documentclass[../main]{subfiles}


\begin{document}

\section{Funtore covariante applicato ad un sistema diretto}
\label{sec:org317a2ba}
Siano \(\mathcal{C}, \mathcal{D}\) due \href{20241126100904-categoria.org}{categorie} e \(I\) un \href{20250202184517-ordine_superiormente_diretto.org}{insieme superiormente diretto}.

\begin{definizione}
Sia \(\{(X_i, \varphi_{ij})\}_{i \in I}\) un \href{20260209153646-sistema_diretto.org}{sistema diretto} in \(\mathcal{C}\) e sia \(F: \mathcal{C} \longrightarrow \mathcal{D}\) un \href{20241204222455-funtore_covariante.org}{funtore covariante.}
L'applicazione del funtore \(F\) al sistema definisce un nuovo \textbf{\textbf{sistema diretto}} \(\{(F(X_i), F(\varphi_{ij}))\}_{i \in I}\) nella categoria \(\mathcal{D}\), dove:
\begin{enumerate}
\item Gli oggetti sono \(\{F(X_i)\}_{i\in I}\);
\item Per ogni \(i \le j\), i morfismi strutturali sono \(F(\varphi_{ij}): F(X_i) \longrightarrow F(X_j)\).
\end{enumerate}
\end{definizione}

\begin{oss}
La coerenza del nuovo sistema è garantita dalle proprietà di composizione del funtore. Se \(i \le j \le k\), nel sistema originale vale \(\varphi_{ik} = \varphi_{jk} \circ \varphi_{ij}\). Applicando \(F\):
\begin{equation*}
F(\varphi_{ik}) = F(\varphi_{jk} \circ \varphi_{ij}) = F(\varphi_{jk}) \circ F(\varphi_{ij})
\end{equation*}
Visualizzandolo con un diagramma commutativo:
\begin{equation*}
\scalebox{0.9}{%
\begin{tikzcd}[ampersand replacement=\&]
	{F(X_i)} \&\& {F(X_j)} \\
	\& {F(X_k)}
	\arrow["{F(\varphi_{ij})}", from=1-1, to=1-3]
	\arrow["{F(\varphi_{jk})}", from=1-3, to=2-2]
	\arrow["{F(\varphi_{ik})}"', from=1-1, to=2-2]
\end{tikzcd}%
}
\end{equation*}
Inoltre, \(F\) mappa l'identità in identità: \(F(\varphi_{ii}) = F(\operatorname{id}_{X_i}) = \operatorname{id}_{F(X_i)}\).
\end{oss}
\end{document}
