% Intended LaTeX compiler: pdflatex
\documentclass[../main]{subfiles}


\begin{document}

\section{Funzione meromorfa su una superficie di Riemann a valori complessi}
\label{sec:org4e35825}
Sia \(X\) una \href{20260127112828-superficie_di_riemann.org}{superficie di Riemann}.

\begin{definizione}
Sia \(A \subseteq X\) aperto. Una \textbf{funzione meromorfa} su \(A\) è una \href{20260128143822-funzione_olomorfa_su_una_superficie_di_riemann.org}{funzione olomorfa}
\begin{equation*}
f : A \setminus S \to \C
\end{equation*}
tale che:
\begin{itemize}
\item \(S \subseteq A\) è un sottoinsieme \href{20250103145124-topologia.org}{chiuso} e \href{20260128123515-sottoinsieme_discreto.org}{discreto}
\item nei punti di \(S\), \(f\) ha \href{20260128154601-singolarita_isolata_analisi_complessa.org}{singolarità eliminabili} o \href{20260128154601-singolarita_isolata_analisi_complessa.org}{poli}.
\end{itemize}
\end{definizione}

\begin{prop}
Sia \(U \subseteq X\) \href{20250103145124-topologia.org}{aperto} \href{20250103165325-spazio_topologico_connesso.org}{connesso}, \(f\) meromorfa su \(U\), \(f \not\equiv 0\). Allora
\begin{equation*}
\bm{T} \coloneqq \set{z_{0} \in U \mid f(z_{0}) = 0\text{ oppure \(z_{0}\) è polo per \(f\)}}
\end{equation*}
è \href{20250103145124-topologia.org}{chiuso} e \href{20260128123515-sottoinsieme_discreto.org}{discreto} in \(U\).
\end{prop}

\begin{prop}
Sia \(U \subseteq X\) è un \href{20250103145124-topologia.org}{aperto} \href{20250103165325-spazio_topologico_connesso.org}{connesso} e \(f,g\) sono funzioni meromorfe su \(U\). Se \(f\) e \(g\) coincidono su un sottoinsieme non \href{20260128123515-sottoinsieme_discreto.org}{discreto} di \(U\), allora \(f=g\).
\end{prop}
\end{document}
