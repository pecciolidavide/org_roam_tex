% Intended LaTeX compiler: pdflatex
\documentclass[../main]{subfiles}


\begin{document}

\section{Chiarimenti sui Passaggi Algebrici}
\label{sec:org07e288f}
I passaggi segnati con ``WHY'' nella dimostrazione dell'isomorfismo tra omologia singolare e cellulare si basano su due fatti generali di algebra omologica riguardanti i moduli e le mappe iniettive.
\subsection{1. Isomorfismo di quozienti tramite iniezione (Primo WHY)}
\label{sec:org6d8108c}

Il passaggio:
\begin{equation*}
\frac{H_{q}(X_{q})}{\operatorname{Im}\partial_{q+1}^{(q+1,q)}} \cong \frac{\operatorname{Im} \pi_{q}^{(q,q-1)}}{\operatorname{Im} \bigg(\pi_{q}^{(q,q-1)}\circ \partial_{q+1}^{q+1,q}\bigg)}
\end{equation*}
è giustificato dal seguente lemma generale:

\begin{lem}
Siano \(M, N\) due \(R\)-moduli e \(\varphi: M \to N\) un omomorfismo \textbf{iniettivo}. Sia \(K \subseteq M\) un sottomodulo. Allora \(\varphi\) induce un isomorfismo tra i quozienti:
\begin{equation*}
\frac{M}{K} \cong \frac{\varphi(M)}{\varphi(K)}
\end{equation*}
\end{lem}

\begin{proof}
Consideriamo la restrizione di \(\varphi\) alla sua immagine, \(\tilde{\varphi}: M \to \operatorname{Im}(\varphi)\). Poiché \(\varphi\) è iniettiva, \(\tilde{\varphi}\) è un isomorfismo di moduli.
Un isomorfismo mappa sottomoduli in sottomoduli e preserva i quozienti. Specificamente:
\begin{enumerate}
\item L'immagine di \(M\) tramite \(\tilde{\varphi}\) è \(\varphi(M)\) (ovvio).
\item L'immagine del sottomodulo \(K\) tramite \(\tilde{\varphi}\) è \(\varphi(K)\).
\end{enumerate}
Pertanto, l'isomorfismo \(\tilde{\varphi}\) scende al quoziente:
\begin{equation*}
\frac{M}{K} \xrightarrow{\cong} \frac{\varphi(M)}{\varphi(K)}, \quad [m] \mapsto [\varphi(m)].
\end{equation*}
Nel nostro caso specifico:
\begin{itemize}
\item \(M = H_q(X_q)\)
\item \(N = H_q(X_q, X_{q-1})\)
\item \(\varphi = \pi_q^{(q,q-1)}\) (che sappiamo essere iniettiva)
\item \(K = \operatorname{Im}\partial_{q+1}^{(q+1,q)}\)
\end{itemize}
Quindi il quoziente viene preservato mappando tutto tramite \(\pi_q\).
\end{proof}
\subsection{2. Nucleo di una composizione con iniezione (Secondo WHY)}
\label{sec:org75ca3dc}

Il passaggio:
\begin{equation*}
\ker \partial_{q}^{(q,q-1)} \cong \ker \left( \pi_{q-1}^{(q-1,q-2)}\circ \partial_{q}^{(q,q-1)} \right)
\end{equation*}
è giustificato dal fatto che comporre con una funzione iniettiva non cambia il nucleo della funzione precedente.

\begin{lem}
Siano \(f: A \to B\) e \(g: B \to C\) due omomorfismi di moduli. Se \(g\) è \textbf{iniettiva}, allora:
\begin{equation*}
\ker(f) = \ker(g \circ f)
\end{equation*}
\end{lem}

\begin{proof}
Dobbiamo mostrare la doppia inclusione (o l'uguaglianza diretta).
Ricordiamo che \(x \in \ker(g \circ f) \iff g(f(x)) = 0\).

Poiché \(g\) è iniettiva, il suo nucleo è banale, ovvero:
\begin{equation*}
g(y) = 0 \iff y = 0.
\end{equation*}
Sostituendo \(y\) con \(f(x)\), otteniamo:
\begin{equation*}
g(f(x)) = 0 \iff f(x) = 0 \iff x \in \ker(f).
\end{equation*}
Quindi i due insiemi coincidono esattamente.

Nel nostro caso specifico:
\begin{itemize}
\item \(f = \partial_q^{(q,q-1)}\)
\item \(g = \pi_{q-1}^{(q-1,q-2)}\) (che sappiamo essere iniettiva)
\item \(g \circ f = \operatorname{d}_q\)
\end{itemize}
Pertanto \(\ker \partial_q^{(q,q-1)} = \ker \operatorname{d}_q\).
\end{proof}
\end{document}
