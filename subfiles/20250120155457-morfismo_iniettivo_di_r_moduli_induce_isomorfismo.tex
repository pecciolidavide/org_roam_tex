% Intended LaTeX compiler: pdflatex
\documentclass[../main]{subfiles}


\begin{document}

\section{Teoremi di isomorfismo}
\label{sec:orgff7abcf}
\subsection{Primo Teorema di Isomorfismo}
\label{sec:org60343b0}

\subsubsection{Primo Teorema di Isomorfismo (Teoria dei Gruppi)}
\label{sec:org0a3b8ea}
\begin{thm}
Sia \(\phi: G \to H\) un \href{20241206115531-morfismo_di_gruppi.org}{omomorfismo} di \href{20241205141146-gruppo_abeliano.org}{gruppi}. Allora il \href{20241213105201-kernel.org}{nucleo} \(\ker(\phi)\) è un \href{20241206143051-sottogruppo.org}{sottogruppo} \href{20241206143051-sottogruppo.org}{normale} di \(G\) e vi è un \href{20241206115531-morfismo_di_gruppi.org}{isomorfismo}:\footnote{Vedi ``\href{20250202190147-immagine_punto_a_punto_di_due_classi.org}{Immagine e retroimmagine tramite una funzione}''}
\begin{equation*}
G / \ker(\phi) \cong \mathrm{im}(\phi)
\end{equation*}
dato da
\begin{align*}
G/\ker(\phi) &\longrightarrow \mathrm{im}(\phi)\\
g + \ker(\phi) &\longmapsto \phi(g).
\end{align*}
\end{thm}

\begin{oss}
Pertanto, se \(\phi:G\to H\) è \href{20241219101956-funzione_iniettiva.org}{iniettivo}, allora \(G \cong \phi[G]\).
\end{oss}
\subsubsection{Primo Teorema di Isomorfismo (Moduli)}
\label{sec:org90f6c7d}
\begin{thm}
Sia \(\psi: M \to N\) un \href{20241206115416-morfismi_r_moduli.org}{omomorfismo} di \href{20241205141053-r_moduli.org}{\(R\)-moduli}. Allora:
\begin{equation*}
M / \ker(\psi) \cong \text{im}(\psi)
\end{equation*}
\end{thm}
\subsubsection{Primo Teorema di Isomorfismo (Anelli)}
\label{sec:org0cf4fb9}
\begin{thm}
Sia \(f: R \to S\) un \href{20241205141119-anello.org}{omomorfismo} di \href{20241205141119-anello.org}{anelli}. Allora \(\ker(f)\) è un \href{20241219112955-ideale.org}{ideale} di \(R\) e:
\begin{equation*}
R / \ker(f) \cong \text{im}(f)
\end{equation*}
\end{thm}
\subsection{Secondo Teorema di Isomorfismo}
\label{sec:orgb122e2e}
\subsubsection{Secondo Teorema di Isomorfismo (Teoria dei Gruppi)}
\label{sec:org9f5f5df}
\subsubsection{Secondo Teorema di Isomorfismo (Moduli)}
\label{sec:orgf0f2bba}
\subsubsection{Secondo Teorema di Isomorfismo (Anelli)}
\label{sec:orgb2e402d}
\subsection{Terzo Teorema di Isomorfismo}
\label{sec:orge83227a}
\subsubsection{Terzo Teorema di Isomorfsimo (Teoria dei Gruppi)}
\label{sec:org4bcd778}
\subsubsection{Terzo Teorema di Isomorfismo (Moduli)}
\label{sec:orgd40261e}
\subsubsection{Terzo Teorema di Isomorfismo (Anelli)}
\label{sec:org5eacb7c}
\subsection{Teoremi di Isomorfismo per complessi di catene}
\label{sec:orge5a7d28}

\begin{thm}
Siano \(\mathcal{A}_{\bullet}, \mathcal{B}_{\bullet}, \mathcal{C}_{\bullet}\) complessi di catene di \(R\)-moduli. Valgono i seguenti teoremi di isomorfismo:

\begin{enumerate}
\item \textbf{\textbf{Primo Teorema di Isomorfismo}}:
Sia \(f_{\bullet}: \mathcal{A}_{\bullet} \longrightarrow \mathcal{B}_{\bullet}\) un morfismo di complessi. Allora esiste un isomorfismo di complessi di catene:
\begin{equation*}
\mathcal{A}_{\bullet} / \ker(f_{\bullet}) \cong \operatorname{Im}(f_{\bullet})
\end{equation*}

\item \textbf{\textbf{Secondo Teorema di Isomorfismo}}:
Siano \(\mathcal{S}_{\bullet}\) e \(\mathcal{T}_{\bullet}\) sottocomplessi di \(\mathcal{C}_{\bullet}\). Allora esiste un isomorfismo:
\begin{equation*}
(\mathcal{S}_{\bullet} + \mathcal{T}_{\bullet}) / \mathcal{T}_{\bullet} \cong \mathcal{S}_{\bullet} / (\mathcal{S}_{\bullet} \cap \mathcal{T}_{\bullet})
\end{equation*}

\item \textbf{\textbf{Terzo Teorema di Isomorfismo}}:
Siano \(\mathcal{S}_{\bullet} \subseteq \mathcal{T}_{\bullet} \subseteq \mathcal{C}_{\bullet}\) sottocomplessi. Allora esiste un isomorfismo:
\begin{equation*}
(\mathcal{C}_{\bullet} / \mathcal{S}_{\bullet}) / (\mathcal{T}_{\bullet} / \mathcal{S}_{\bullet}) \cong \mathcal{C}_{\bullet} / \mathcal{T}_{\bullet}
\end{equation*}
\end{enumerate}
\end{thm}
\end{document}
