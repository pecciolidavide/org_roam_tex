% Intended LaTeX compiler: pdflatex
\documentclass[../main]{subfiles}


\begin{document}

\section{Fascio immagine}
\label{sec:org390aa59}
Siano \(\mathcal{F},\mathcal{G}\) due \href{20250324174728-fascio.org}{fasci} sullo \href{20250103145124-topologia.org}{spazio topologico} \(X\), e sia \(\varphi: \mathcal{F}\to \mathcal{G}\) un \href{20250325180613-morfismo_di_prefasci.org}{morfismo di fasci}.

Per ogni \(U \subseteq X\) aperto, \(\operatorname{Im}\varphi_{U} \subseteq \mathcal{G}(U)\)\footnote{Vedi ``\href{20250202173528-dominio_range_e_campo_di_una_classe_relazione.org}{Range di una funzione}''}, e inoltre, se \(W \subseteq U\)
\begin{equation*}
\rho_{V}^{U} (\operatorname{Im} \varphi_{U}) \subseteq \operatorname{Im} \varphi_{V}.
\end{equation*}

Quindi \(\set{\operatorname{Im}\varphi_{U}}\) è un sottoprefascio di \(\mathcal{G}\), ma in generale non è un fascio.

\begin{definizione}
Si definisce il \textbf{fascio immagine} di \(\varphi\), denotato con \(\operatorname{Im} \varphi\), come il \href{20250327114851-fascio_associato_ad_un_prefascio.org}{fascificato}:\footnote{Vedi anche ``\href{20250327114913-fascio_associato_ad_un_sottoprefascio.org}{Fascio associato ad un sottoprefascio}''}
\begin{equation*}
\set{\operatorname{Im}\varphi_{U}}^{+} \subseteq G
\end{equation*}
sottofascio.
\end{definizione}

\begin{oss}
Per ogni \(U \subseteq X\) aperto, si ha che
\begin{equation*}
(\operatorname{Im} \varphi)(U) =  \set{%
g \in \mathcal{G}(U) \mid%
	\forall p \in U\ %
	\exists W \subseteq U \text{ intorno aperto di \(p\)}, \ %
	\exists f \in \mathcal{F}(W):\hspace{1em} %
	\restriction{g}{W} = \varphi_{W}(f) %
}.
\end{equation*}
\end{oss}
\end{document}
