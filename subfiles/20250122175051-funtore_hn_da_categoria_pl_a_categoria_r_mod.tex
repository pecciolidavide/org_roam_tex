% Intended LaTeX compiler: pdflatex
\documentclass[../main]{subfiles}


\begin{document}

\section{Funtore da Pl a R-Mod - di omologia}
\label{sec:org842fb27}
Sia \(R\) un \href{20241219112842-pid.org}{PID}.

Si considerino le \href{20241126100904-categoria.org}{categorie}
\begin{itemize}
\item \(\cat{Pl}\) dei \href{20250122133147-categoria_di_complessi_e_mappe_simpliciali.org}{complessi simpliciali};
\item \(\cat{Ch}_{R}\) dei \href{20250120163759-categoria_complessi_di_catene.org}{complessi di catene};
\item \(R\text{-}\cat{Mod}\) degli \href{20241206115740-categoria_degli_r_moduli.org}{\(R\)-moduli}.
\end{itemize}

e si considerino i \href{20241205131958-funtore.org}{funtori} \href{20250122133308-funtore_diesis_da_complessi_simpliciali_a_complessi_di_catene.org}{diesis} e di \href{20250120165029-funtore_tra_chr_e_rmod.org}{omologia}:
\begin{equation*}
    \diesis: \cat{Pl}\longrightarrow \cat{Ch}_{R},\qquad H_{n}:\cat{Ch}_{R}\longrightarrow R\text{-}\cat{Mod}
\end{equation*}

Questo dà origine ad un funtore
\begin{equation*}
  (H_{n},\star): \cat{Pl}\longrightarrow R\text{-}\cat{Mod}
\end{equation*}
come segue:\footnote{Vedi:
\begin{itemize}
\item \(\mathcal{C}_{\bullet}\): \href{20250121160600-complesso_di_catene_simpliciali.org}{Complesso di catene simpliciali}
\item \(H_{n}(K)\): \href{20250121160827-omologia_simpliciale.org}{Omologia Simpliciale}
\end{itemize}}
\begin{equation*}
\begin{tikzcd}[ampersand replacement=\&,column sep=large,row sep=tiny]
	K \& {\mathcal{C}_{\bullet}(K)} \& {H_n(K)} \\
	{K\xrightarrow{f}L} \& {\mathcal{C}_{\bullet}(K) \xrightarrow{f_{\diesis}} \mathcal{C}_{\bullet}(L)} \& {H_n(K)\xrightarrow{(f_{\diesis})_{\star}} H_n(L)}
	\arrow[color={rgb,255:red,214;green,92;blue,92}, maps to, from=1-1, to=1-2]
	\arrow[color={rgb,255:red,214;green,92;blue,92}, maps to, from=1-2, to=1-3]
	\arrow[color={rgb,255:red,214;green,92;blue,92}, maps to, from=2-1, to=2-2]
	\arrow[color={rgb,255:red,214;green,92;blue,92}, maps to, from=2-2, to=2-3]
\end{tikzcd}
  \end{equation*}
e si indicherà \(f_{\star}\coloneqq (f_{\diesis})_{\star}\).
\end{document}
