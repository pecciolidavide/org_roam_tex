% Intended LaTeX compiler: pdflatex
\documentclass[../main]{subfiles}


\begin{document}

\section{Scomposizione di un modulo con quoziente libero}
\label{sec:orge543f4c}
Sia \(R\) un \href{20241205141119-anello.org}{anello abeliano con unità}, e sia \(M\) un \href{20241205141053-r_moduli.org}{\(R\)-modulo}.
\begin{prop}
Se \(N \subseteq_{R} M\)\footnote{Vedi ``\href{20241206142802-sottomoduli.org}{Sottomodulo}''} è tale che \(M/N\) è \href{20241213094625-modulo_libero.org}{libero}, allora esiste \(P \subseteq_R M\) tale che
\begin{equation*}
P\cong M/N,\qquad M \cong P \oplus N.
\end{equation*}
\end{prop}
\begin{proof}
Sia
\begin{align*}
\pi: M &\longrightarrow M/N\\
m&\longmapsto m+N
\end{align*}
la proiezione sul \href{20241206142802-sottomoduli.org}{quoziente}, e sia \(E=\set{e_1,\dots,e_n}\) una \href{20241213094625-modulo_libero.org}{base} di \(M/N\).

Per ogni \(e_i\), fissiamo \(\tilde{e_i} \in \pi^{-1}(e_i)\), \(\tilde{e_{i}} \in M\). L'insieme
\begin{equation*}
\tilde{E}\coloneqq \set{\tilde{e}_1,\dots,\tilde{e}_n}
\end{equation*}
è \href{20241212142019-insiemi_linearmente_indipendenti.org}{linearmente indipendente}, poiché \(\pi\) è un \href{20241206115416-morfismi_r_moduli.org}{morfismo}. Sia \(P\) il \href{20241212141101-generatori_modulo.org}{modulo generato da \(\tilde{E}\)}, ovvero
\begin{equation*}
P = \set{\sum_{i=1}^n a_i \tilde{e}_i : a_i \in R
}
\end{equation*}
\begin{itemize}
\item \textbf{Claim}: \(P\cap N = \set{0}\).

Se \(m \in P \cap N\) allora \(m \in P\), dunque
\begin{equation*}
m = \sum_{i=1}^k a_i\tilde{e}_i
\end{equation*}
ma \(m \in N\), e quindi \(\pi(m)=0\), dunque
\begin{equation*}
0 = \pi(m) = \sum_{i=1}^k a_i e_i
\end{equation*}
dove l'ultima uguaglianza vale poiché \(\pi\) è un \href{20241206115416-morfismi_r_moduli.org}{morfismo}.

Siccome \(E\) è \href{20241212142019-insiemi_linearmente_indipendenti.org}{linearmente indipendente}, allora \(\forall\,i,\ a_i=0\), e dunque \(m = 0\).

\item \textbf{Claim}: \(M \cong P\oplus N\).

Sia \(\psi : P\oplus N \longrightarrow M\), definita come
\begin{equation*}
\psi(p,n) \coloneqq  p+n
\end{equation*}
Si verifica facilmente che \(\psi\) sia un \href{20241206115416-morfismi_r_moduli.org}{morfismo}. Inoltre
\begin{itemize}
\item \(\psi\) è \href{20241213105600-funzione_suriettiva.org}{suriettiva}: infatti \(\forall\, m \in M\) scriviamo
\begin{equation*}
\pi(m) = \sum_{i = 1}^n a_i e_i
\end{equation*}
Sia dunque \(p \coloneqq  \sum_{i=1}^n a_i \tilde{e_i} \in P\) e sia \(n \coloneqq  m-p\).

Allora \(n \in N\), siccome
\begin{equation*}
\pi(n) = \pi(m-p) = \pi(m)-\pi(p) = \bigg(\sum_{i=1}^n a_i e_i\bigg) - \bigg(\sum_{i=1}^n a_i \parentesi{=e_{i}}{\pi(\tilde{e_i})}\bigg) = 0.
\end{equation*}
Inoltre \(\psi (p,n) = p + n = p + (m - p) = m\).
\item \(\psi\) è \href{20241219101956-funzione_iniettiva.org}{iniettiva}: se \(\psi(p,n) = 0\) allora \(n=-p\) e siccome sia \(N\) che \(P\) sono \href{20241206142802-sottomoduli.org}{sottomoduli} allora
\begin{equation*}
n,p \in P\cap N =\set{0}
\end{equation*}
e pertanto \(n = p = 0\) e \((n,p)=(0,0)\).
\end{itemize}
\item \textbf{Claim}: \(P\cong M/N\):

Consideriamo la restrizione \(\restriction{\pi}{P} : P \longrightarrow M/N\), che è morfismo.

Inoltre, la funzione
\begin{align*}
\sigma: M/N &\longrightarrow P\\
\sum_{i=1}^n a_i \ e_i &\longmapsto \sum_{i=1}^n a_i\ \tilde{e}_i
\end{align*}
è il morfismo inverso.
\qedhere
\end{itemize}
\end{proof}
\end{document}
