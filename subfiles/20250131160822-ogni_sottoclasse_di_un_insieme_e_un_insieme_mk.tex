% Intended LaTeX compiler: pdflatex
\documentclass[../main]{subfiles}


\begin{document}

\section{Ogni sottoclasse di un insieme è un insieme MK}
\label{sec:org49f767c}
Contesto: \href{20250130104245-morse_kelly_set_theory.org}{Morse Kelly Set Theory}

Siano \(A,B\) due \href{20250130104320-classe_mk.org}{classi}.
\subsection{Corollario}
\label{sec:org009a325}
Se \(B\) è un \href{20250130104331-insieme_mk.org}{insieme} e \(A \subseteq B\) (vedi \href{20250131155822-operazioni_insiemistiche_tra_classi_mk.org}{Sottoclasse MK}), allora \(A\) è un \href{20250130104331-insieme_mk.org}{insieme}
\subsubsection{Dimostrazione}
\label{sec:orgc5e36d5}
Se \(B\) è un insieme, allora per l'assioma of Power-set esiste \(P\) tale che
\begin{equation*}
\forall\, A (A \subseteq B\,\implies A \in P)
\end{equation*}
e pertanto \(A\) è un insieme (per definizione).
\end{document}
