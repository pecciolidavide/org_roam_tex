% Intended LaTeX compiler: pdflatex
\documentclass[../main]{subfiles}


\begin{document}

\section{Chiuso in uno spazio metrizzabile è Gdelta}
\label{sec:org031aa6b}
\subsection{Proposizione}
\label{sec:org4770e9c}
Sia \(X\) uno \href{20250103145124-topologia.org}{spazio topologico} \href{20250301193401-spazio_topologico_metrizzabile.org}{metrizzabile}, e sia \(F \subseteq X\) un \href{20250131155822-operazioni_insiemistiche_tra_classi_mk.org}{sottoinsieme} \href{20250103145124-topologia.org}{chiuso}.
Allora \(F\) è un \href{20250304152026-sottoinsiemi_gdelta_e_fsigma.org}{insieme \(\bm{G}_{\delta}\)}.
\subsubsection{Dimostrazione}
\label{sec:org7bed116}
Sia \(d\) una \href{20250301193511-spazio_metrico.org}{metrica} \href{20250301193530-topologia_indotta_da_una_distanza.org}{compatibile} su \(X\).

Si denoti con
\begin{equation*}
B_{d}(x,\varepsilon) \coloneqq \set{y \in X\ |\ d(x,y)<\varepsilon}
\end{equation*}

Per ogni \(n \in \N\) si definisce
\begin{equation*}
U_{n} \coloneqq \bigcup_{x \in F} B_{d}(x, 2^{-n})
\end{equation*}
(vedi \href{20250131155822-operazioni_insiemistiche_tra_classi_mk.org}{Classe Unione Generalizzata})

Allora \(F= \bigcap_{n \in \N} U_{n}\) (vedi \href{20250131155822-operazioni_insiemistiche_tra_classi_mk.org}{Classe Intersezione Generalizzata}).
\end{document}
