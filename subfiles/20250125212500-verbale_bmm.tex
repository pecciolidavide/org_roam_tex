% Intended LaTeX compiler: pdflatex
\documentclass[../main]{subfiles}


\begin{document}

Tutti presenti
Inizio ore 21:25
Sede della BMM
\section{Gestione dei parimerito}
\label{sec:orgb61a0fb}
\begin{itemize}
\item Gianrenza PIANA ha deciso di rinunciare alla propria posizione; al suo posto Graziella ZITO prenderà il posto di Tesoriere
\item Giorgia GHIOSSO ha deciso di rinunciare alla propria posizione
\end{itemize}
\section{Designare le cariche del Direttivo}
\label{sec:org200781d}

\begin{center}
\begin{tabular}{ll}
Carica & Incaricato\\
\hline
Presidente & Renato ARMELLIN\\
Vicepresidente & Alessandro DANZERO\\
Tesoriere & Graziella ZITO\\
Segretario & Davide PECCIOLI\\
Capobanda & Alessio SCULLINO\\
Consiglieri & Emanuela BORGHESIO\\
 & Fabrizio MONTAGNER\\
 & Marco RUZZANTE\\
\hline
\end{tabular}
\end{center}
\section{Selezione pezzi del concertp}
\label{sec:orga204e27}
Fabrizio MONTAGNER si impegna a presentare al direttivo una pseudo-scaletta del concerto prima dell'inizio delle prove, così che eventuali pezzi possano essere scartati. Si pongono dei vincoli in merito ai pezzi scartati: nessun pezzo originale di MONTAGNER potrà venir scartato dal direttivo.
\section{Lezioni presso la scuola media}
\label{sec:org71513e1}

Il professor Barone chiamerà MONTAGNER per discutere degli sviluppi.

Potrebbe Barone stesso fare le lezioni private agli studenti che scelgono di proseguire. Si delinea la necessità di trovare insegnanti.
\section{Gestione delle sfilate}
\label{sec:orgcf564ef}
Si nominano le figure del vice-capobanda:
\begin{itemize}
\item Alessandro DANZERO
\item Marco RUZZANTE
\end{itemize}

È necessario organizzare una prova, prima dell'inizio di marzo, per rinfrescare il libretto e aggiungere nuovi pezzi.
\end{document}
