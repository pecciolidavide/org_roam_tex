% Intended LaTeX compiler: pdflatex
\documentclass[../main]{subfiles}


\begin{document}

\section{Teorema Cinese dei resti}
\label{sec:org48f0bd3}
\begin{thm}
Siano \(c_{0},\dots,c_{k}\) \uline{\href{20250202130045-insieme_dei_numeri_naturali_mk.org}{numeri naturali} maggiori di 1}, a due a due \href{20250531114007-numeri_naturali_coprimi.org}{coprimi}. Allora per ogni \(a_{0},\dots,a_{k} \in \N\) esiste un unico
\begin{equation*}
0 \le x< \prod_{i=0}^{k} c_{i}
\end{equation*}
tale che, per ogni \(i=0,\dots,k\) si abbia la seguente equivalenza di \href{20250531114133-classe_di_resto.org}{classi di resto}:
\begin{equation*}
x\equiv a_{i}\mod{c_{i}}
\end{equation*}
\end{thm}
\end{document}
