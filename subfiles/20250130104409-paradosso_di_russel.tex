% Intended LaTeX compiler: pdflatex
\documentclass[../main]{subfiles}


\begin{document}

\section{Paradosso di Russel}
\label{sec:org8e04e41}
\subsection{Paradosso di Russel}
\label{sec:org2347b2f}

Si consideri l'insieme
\begin{equation*}
R=\set{x : x \notin x}
\end{equation*}
\begin{itemize}
\item Se \(R \in R\), allora \(R\notin R\);
\item Se \(R\notin R\), allora \(R \in R\).
\end{itemize}

Questo è assurdo.
\subsubsection{Soluzione nell'ambito di MK}
\label{sec:orgbfe6574}
Contesto: \href{20250130104245-morse_kelly_set_theory.org}{Morse Kelly Set Theory}

La classe
\begin{equation*}
R = \set{x\,|\, x \notin x}
\end{equation*}
esiste per l'assioma di Comprehension, e contiene tutti gli \href{20250130104331-insieme_mk.org}{insiemi} \(x\) tali che \(x\notin x\).

Se \(R \in R\), allora \(R\notin R\), assurdo.
Se invece \(R\notin R\), allora
\begin{itemize}
\item se \(R\) è un \href{20250130104331-insieme_mk.org}{insieme}, si ha la contraddizione che \(R \in R\).
\item se \(R\) è una \href{20250130104320-classe_mk.org}{classe propria}, allora non vi è alcuna contraddizione.
\end{itemize}

Quindi \(R\) è una classe propria, tale che \(R\notin R\).
\end{document}
