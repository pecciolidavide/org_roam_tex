% Intended LaTeX compiler: pdflatex
\documentclass[../main]{subfiles}


\begin{document}

\section{Caratterizzazione insiemi invarianti su un insieme in un modello mostro}
\label{sec:orgd861afb}
Si utilizza la \href{20250612143636-notazione_teoria_dei_modelli.org}{notazione della Teoria dei Modelli}.

Sia \(\mathcal{L}\) un \href{20250130162057-linguaggio_del_prim_ordine.org}{linguaggio del prim'ordine} fissato, \(T\) una \(\mathcal{L}\)-\href{20250130114950-teoria_del_prim_ordine.org}{teoria} \href{20250131123151-teoria_completa.org}{completa} senza \href{20250131122945-modello_di_un_insieme_di_formule.org}{modelli} finiti. Sia \(\mathcal{U}\vDash T\) un \href{20250617102733-modello_mostro.org}{modello mostro} di \href{20241213101756-cardinalita.org}{cardinalità} \(\kappa>\card{\mathcal{L}}+\omega\).

\begin{thm}
Sia \(\mathcal{D} \subseteq \mathcal{U}^{x}\) insieme \href{20250131122913-soddisfazione_di_una_formula.org}{definibile}. Sono fatti equivalenti:
\begin{enumerate}
\item \(\mathcal{D}\) è \uline{invariante su \(A\)};
\item \(\mathcal{D}=\varphi(\mathcal{U}^{x})\) per qualche \(\varphi(x) \in \mathcal{L}(A)\).
\end{enumerate}
\end{thm}
\begin{thm}
Sia \(\mathcal{D} \subseteq \mathcal{U}^{x}\) insieme \href{20250212164424-tipo_teoria_dei_modelli.org}{tipo-definibile}. Sono fatti equivalenti:
\begin{enumerate}
\item \(\mathcal{D}\) è \uline{invariante su \(A\)};
\item \(\mathcal{D}=p(\mathcal{U}^{x})\) per qualche \(p(x) \subseteq \mathcal{L}(A)\).
\end{enumerate}
\end{thm}

Si noti che le definizioni di ``insieme definibile'' e ``insieme tipo-definibile'' sono differenti nell'ambito di un modello mostro.
\end{document}
