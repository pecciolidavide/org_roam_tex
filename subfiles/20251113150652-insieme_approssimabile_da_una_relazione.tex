% Intended LaTeX compiler: pdflatex
\documentclass[../main]{subfiles}


\begin{document}

\def\U{\mathcal{U}}
\def\L{\mathcal{L}}
\def\X{\mathcal{X}}
\def\Z{\mathcal{Z}}
\def\D{\mathcal{D}}
%
\def\eq{{\rm eq}}
\def\Ueq{\U^\eq}
%
\def\orbita{\mathcal{O}}
\def\Aut{\operatorname{Aut}}
%
\def\tc{\mid}
\def\tp{\operatorname{tp}}
\def\EMtp{\operatorname{EM}\text{-}\operatorname{tp}}
\def\<{\langle}
\def\>{\rangle}
%
\def\restricted#1{\,\mathord{\upharpoonright}{{\scriptstyle #1}}}
\def\equivalentover#1{\mathrel{\equiv_{ #1 }}}

%% NON FORKING
\def\nonforkSymbol{%
\mathbin{\raise1.8ex%
\rlap{\kern0.6ex\rule{0.6ex}{0.1ex}}%
\rlap{\kern1.1ex\rule{0.1ex}{1.9ex}}\raise-0.3ex\hbox{$\smile$}}}
\def\defaultnonforkmodel{M}
\def\nonfork{\nonforkSymbol}
\renewcommand{\nonfork}[1][\defaultnonforkmodel]{%
\mathrel{\nonforkSymbol_{#1}}}
\section{Insieme approssimabile da una relazione}
\label{sec:org10b595c}
Siano \(\X, \Z\) due \href{20250130104331-insieme_mk.org}{insiemi} arbitrari, e sia \(\pi \subseteq \X\times \Z\) una \href{20250202170607-classe_relazione_binaria.org}{relazione} binaria. Si scriverà indifferentemente
\begin{equation*}
(x,z) \in \pi,\qquad \pi(x,z).
\end{equation*}

\begin{definizione}
\(\D \subseteq \Z\) è \uline{approssimabile da \(\pi\)} se per ogni
\(B \subseteq \Z\) finito esiste \(a \in \X\) tale che
\begin{equation*}
\pi(a;B) =\set{b \in B\mid \pi(a,b)} = \D \cap B.
\end{equation*}
\end{definizione}
\begin{oss}
Considero \href{20250130104245-morse_kelly_set_theory.org}{\(\parti{\Z}\)} come \href{20250103145124-topologia.org}{spazio topologico}, \href{20251113174453-cardinalita_insieme_delle_parti.org}{identificandolo} \href{20250202192030-classe_delle_classi_funzioni.org}{con \(2^{\Z}\)} con la \href{20250109154723-topologia_prodotto.org}{topologia prodotto}.\footnote{Vedi in particolare \href{20251113174136-topologia_di_2_x.org}{Topologia di 2\textsuperscript{X}}}
Quindi gli intorni di base di \(\parti{\Z}\) hanno la forma, per \(B\) finito e \(C \subseteq B\)
\begin{equation*}
\set{A \in \parti{\Z} \mid A\cap B = C}.
\end{equation*}

\(\D\) è approssimabile da \(\pi\) sse \(\D\) è nella chiusura topologica di
\begin{equation*}
\set{\pi(a; \Z) \mid a \in X}
\end{equation*}
dove come al solito \(\pi(a; \Z) = \set{b \in \Z \mid \pi(a;b)}\).
\end{oss}
\end{document}
