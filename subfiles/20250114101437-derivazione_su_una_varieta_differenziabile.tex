% Intended LaTeX compiler: pdflatex
\documentclass[../main]{subfiles}


\begin{document}

\section{Derivazione su una varietà differenziabile}
\label{sec:orgb5bf3b5}
\begin{definizione}
Sia \(M\) una \href{20250113115909-struttura_differenziabile.org}{varietà differenziabile}, e sia \(p \in M\). Una \textbf{derivazione} su \(M\) in \(p\) è una funzione
\begin{equation*}
\bm{v}:C^{\infty}_{p} \longrightarrow \R
\end{equation*}
(vedi \href{20250114100254-germi_di_funzioni.org}{Germi di funzioni}), \(\R\)-\href{20250114101949-funzione_lineare.org}{lineare} e che verifichi la regola di Leibniz, ovvero
\begin{enumerate}
\item \(\bm{v}(\lambda f_{p}) = \lambda \bm{v}(f_{p})\) per ogni \(f_{p} \in C^{\infty}_{p}\);
\item \(\bm{v}(f_{p} g_{p}) = \bm{v}(f_{p}) g(p)+ f(p) \bm{v}(f_{p})\) per ogni \(f_{p}, g_{p} \in C^{\infty}_{p}\).
\end{enumerate}
\end{definizione}
\subsection{{\bfseries\sffamily TODO} Esempio: Derivazioni su \(\R\).\hfill{}\textsc{matematica\_lm:geo\_diff}}
\label{sec:orgb712fbc}
\end{document}
