% Intended LaTeX compiler: pdflatex
\documentclass[../main]{subfiles}


\begin{document}

\section{Proprietà di prodotto e somma generalizzata di cardinali}
\label{sec:orga16fe73}
Contesto: \href{20250130104245-morse_kelly_set_theory.org}{Morse Kelly Set Theory} + \href{20250206171508-axiom_of_choiche.org}{AC}

Si considerino le operazioni di \href{20250612151505-aritmetica_dei_cardinali.org}{somma} e \href{20250612151505-aritmetica_dei_cardinali.org}{prodotto} tra \href{20250203161341-cardinali.org}{cardinali} generalizzata. Allora per ogni cardinale \(\kappa\) si ha
\begin{align*}
\kappa &= \sum_{i \in \kappa} 1\\
2^{\kappa} &= \prod_{i \in \kappa} 2
\end{align*}
(vedi \href{20250612151505-aritmetica_dei_cardinali.org}{Elevamento a potenza di Cardinali})

Inoltre, queste operazioni sono monotone, nel senso che se \(\langle \kappa_{i}\ |\ i \in I\rangle\) e \(\langle \lambda_{i}\ |\ i \in I\rangle\) sono \href{20250206170922-sequenze_e_stringhe.org}{sequenze} di \href{20250203161341-cardinali.org}{cardinali} t.c. per ogni \(i \in I\) si ha \(\kappa_{i}<\lambda_{i}\) (vedi \href{20250203111003-ordinali.org}{Relazione d'ordine sugli ordinali}) allora
\begin{align*}
\sum_{i \in I} \kappa_{i} &\le \sum_{i \in I}\lambda_{i}\\
\prod_{i \in I} \kappa_{i} &\le \prod_{i \in I}\lambda_{i}
\end{align*}
\end{document}
