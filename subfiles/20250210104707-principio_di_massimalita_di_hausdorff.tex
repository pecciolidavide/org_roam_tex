% Intended LaTeX compiler: pdflatex
\documentclass[../main]{subfiles}

\usepackage[hyperref]{biblatex}
\date{}
\title{}
\begin{document}

\section{Principio di Massimalità di Hausdorff}
\label{sec:org7bfbc36}
Contesto: \href{20250130104245-morse_kelly_set_theory.org}{Morse Kelly Set Theory}
\begin{definizione}
Il Principio di massimalità di Haussdorf dice che \(\forall\, X\ \textsc{MaxHaus}(X)\), dove
\begin{description}
\item[{\(\textsc{MaxHaus}(X)\):}] se \(\le\) è un \href{20250203101604-ordine.org}{ordine} su \(X\), allora esiste \(C \subseteq X\) tale che \(C\) sia una \href{20250102120836-catena.org}{catena massimale}.
\end{description}
\end{definizione}
\end{document}
