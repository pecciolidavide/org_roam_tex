% Intended LaTeX compiler: pdflatex
\documentclass[../main]{subfiles}


\begin{document}

\section{Classi ambigue di un sottospazio polacco nella gerarchia di Borel}
\label{sec:org1fb1595}
\subsection{Proposizione}
\label{sec:org76036a4}

Let \(Y \subseteq X\) be Polish spaces. Show that for every \(\alpha \geq 3\),
\[
\bm{\Delta}^0_\alpha(Y) = \bm{\Delta}^0_\alpha(X) \upharpoonright Y,
\]
where as usual \(\bm{\Delta}^0_\alpha(X) \upharpoonright Y = \{ A \cap Y \mid A \in \Delta^0_\alpha(X) \}\).
\subsubsection{Dimostrazione}
\label{sec:orgd4c8020}

Si richiama il Lemma 2.1.5(vi):
\begin{align*}
\bm{\Sigma}^{0}_{\alpha}(Y) &= \bm{\Sigma}^{0}_{\alpha}(X)\upharpoonright Y;\\
\bm{\Pi}^{0}_{\alpha}(Y) &= \bm{\Pi}^{0}_{\alpha}(X)\upharpoonright Y.
\end{align*}
\paragraph{Inclusione ``\(\subseteq\)''}
\label{sec:org1ef95df}

Sia \(A \in \bm{\Delta}^{0}_{\alpha}(Y) = \bm{\Sigma}^{0}_{\alpha}(Y) \cap \bm{\Pi}^{0}_{\alpha}(Y)\). Allora esistono \(B \in \bm{\Sigma}^{0}_{\alpha}(X)\) e \(C \in \bm{\Pi}^{0}_{\alpha}(X)\) tali che
\begin{equation*}
A = B\cap Y,\quad A= C\cap Y
\end{equation*}

Siccome \(Y \subseteq X\) è polacco, \href{20250306134632-caratterizzazione_dei_sottoinsiemi_polacchi_di_uno_spazio_polacco.org}{allora} \(Y\) è un \href{20250304152026-sottoinsiemi_gdelta_e_fsigma.org}{sottoinsieme \(\bm{G}_{\delta}\)} di \(X\), ovvero \(Y \in \bm{\Pi}^{0}_{2}(X)\). Poiché \(\alpha\ge 3\), \(\bm{\Pi}^{0}_{2}(X) \subseteq \bm{\Pi}^{0}_{\alpha}(X), \bm{\Sigma}^{0}_{\alpha}(X)\): \(Y \in \bm{\Sigma}^{0}_{\alpha}(X)\) e \(Y \in \bm{\Pi}^{0}_{\alpha}(X)\), e quindi, poiché entrambe le classi \(\bm{\Sigma}_{\alpha}^{0}(X), \bm{\Pi}^{0}_{\alpha}(X)\) sono chiuse per intersezioni finite:
\begin{equation*}
A=B\cap Y \in \bm{\Sigma}^{0}_{\alpha}(X),\qquad A=C\cap Y \in \bm{\Pi}^{0}_{\alpha}(X)
\end{equation*}
ovvero \(A \in \bm{\Delta}^{0}_{\alpha}(X)\). Inoltre \(A \subseteq Y\), e pertanto
\begin{equation*}
A = A\cap Y \in \bm{\Delta}^{0}_{\alpha}(X)\upharpoonright Y = \set{V\cap Y\mid V \in \bm{\Delta}^{0}_{\alpha}(X)}.
\end{equation*}
\paragraph{Inclusione ``\(\supseteq\)''}
\label{sec:org7add0f6}

Sia \(A \in \bm{\Delta}^{0}_{\alpha}(X)\), ovvero \(A\cap Y \in \bm{\Delta}^{0}_{\alpha}(X)\upharpoonright Y\).

Allora
\begin{itemize}
\item \(A \in \bm{\Sigma}^{0}_{\alpha}(X)\), e quindi \(A\cap Y \in \bm{\Sigma}^{0}_{\alpha}(Y)\);
\item \(A \in \bm{\Pi}^{0}_{\alpha}(X)\), e quindi \(A\cap Y \in \bm{\Pi}^{0}_{\alpha}(Y)\).
\end{itemize}

Pertanto
\begin{equation*}
(A\cap Y) \in \bm{\Sigma}^{0}_{\alpha}(Y)\cap \bm{\Pi}^{0}_{\alpha}(Y) = \bm{\Delta}^{0}_{\alpha}(Y).\qedd
\end{equation*}
\end{document}
