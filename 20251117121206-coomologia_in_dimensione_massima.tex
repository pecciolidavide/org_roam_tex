% Created 2026-02-07 Sat 19:32
% Intended LaTeX compiler: pdflatex
\documentclass[10pt]{article}
%% CREATO CON ORG - EMACS
\newcommand{\use}[2][]{\usepackage[#1]{#2}}
% PACCHETTI FONDAMENTLAI
\use[utf8]{inputenc}
\use[T1]{fontenc}
\use{graphicx}
\use{longtable}
\use{wrapfig}
\use{rotating}
\use[normalem]{ulem}
\use{amsmath}
\use{amsthm}
\use{amssymb}

\use{eucal} % Cambia mathcal{...}

\use{capt-of}
\use[italian]{babel}
\use[babel]{csquotes}
% bib la TEX lo carica in automatico org-cite
\use{microtype}
\use{lmodern}
\use{subfig} % sottofigure
\use{multicol} % due colonne
\use{lipsum} % lorem ipsum
\use{color} % colori in latex
\use{parskip} % rimuove l'indentazione dei nuovi paragrafi %% Add parbox=false to all new tcolorbox
\use{centernot}
\use[outline]{contour}\contourlength{3pt}
\use{fancyhdr}
\use{layout}
\use[most]{tcolorbox} % Riquadri colorati
\use{ifthen} % IFTHEN
\use{geometry}

% pacchetti matematica
\use{yhmath}
\use{dsfont}
\use{mathrsfs}
\use{cancel} % semplificare
\use{polynom} %divisione tra polinomi
\use{forest} % grafi ad albero
\use{booktabs} % tabelle
\use{commath} %simboli e differenziali
\use{bm} %bold
\use[fulladjust]{marginnote} %to use marginnote for date notes
\use{arrayjobx}%array
\use[intlimits]{empheq} % Riquadri colorati attorno alle equazioni
\use{mathtools}
\use{circuitikz} % Disegnare i circuiti
\use{mathtools}
\use{stmaryrd} % [[ \llbracket ]] \rrbracket
\use{bussproofs} % dimostrazioni

%%%%%%%%%%%%%


%%%% QUIVER
\newcommand{\duepunti}{\,\mathchar\numexpr"6000+`:\relax\,}
% A TikZ style for curved arrows of a fixed height, due to AndréC.
\tikzset{curve/.style={settings={#1},to path={(\tikztostart)
    .. controls ($(\tikztostart)!\pv{pos}!(\tikztotarget)!\pv{height}!270:(\tikztotarget)$)
    and ($(\tikztostart)!1-\pv{pos}!(\tikztotarget)!\pv{height}!270:(\tikztotarget)$)
    .. (\tikztotarget)\tikztonodes}},
    settings/.code={\tikzset{quiver/.cd,#1}
        \def\pv##1{\pgfkeysvalueof{/tikz/quiver/##1}}},
    quiver/.cd,pos/.initial=0.35,height/.initial=0}

% TikZ arrowhead/tail styles.
\tikzset{tail reversed/.code={\pgfsetarrowsstart{tikzcd to}}}
\tikzset{2tail/.code={\pgfsetarrowsstart{Implies[reversed]}}}
\tikzset{2tail reversed/.code={\pgfsetarrowsstart{Implies}}}
% TikZ arrow styles.
\tikzset{no body/.style={/tikz/dash pattern=on 0 off 1mm}}
%%%%%%%%%%


%% DEFINIZIONI COMANDI MATEMATICI
\let\sin\relax %TOGLIE LA DEFINIZIONE SU "\sin"

% cambia la definizione di empty set
% ---
\let\oldemptyset\emptyset
% ---
% \let\emptyset\varnothing
% ---
% \let\emptyset\relax
% \newcommand{\emptyset}{\text{\textnormal{\O}}}
% ---

\DeclareMathOperator{\bounded}{bd}
\DeclareMathOperator{\sin}{sen}
\DeclareMathOperator{\epi}{Epi}
\DeclareMathOperator{\cl}{cl}
\DeclareMathOperator{\graph}{graph}
\DeclareMathOperator{\arcsec}{arcsec}
\DeclareMathOperator{\arccot}{arccot}
\DeclareMathOperator{\arccsc}{arccsc}
\DeclareMathOperator{\spettro}{Spettro}
\DeclareMathOperator{\nulls}{nullspace}
\DeclareMathOperator{\dom}{dom}
\DeclareMathOperator{\ar}{ar}
\DeclareMathOperator{\const}{Const}
\DeclareMathOperator{\fun}{Fun}
\DeclareMathOperator{\rel}{Rel}
\DeclareMathOperator{\altezza}{ht}
\let\det\relax %TOGLIE LA DEFINIZIONE SU "\det"
\DeclareMathOperator{\det}{det}
\DeclareMathOperator{\End}{End}
\DeclareMathOperator{\gl}{GL}
\def\Id{\mathrm{Id}}
\def\id{\mathrm{id}}
\DeclareMathOperator{\I}{\mathds{1}}
\DeclareMathOperator{\II}{II}
\DeclareMathOperator{\rank}{rank}
\DeclareMathOperator{\tr}{tr}
\DeclareMathOperator{\tc}{t.c.}
\DeclareMathOperator{\T}{T}
\DeclareMathOperator{\var}{Var}
\DeclareMathOperator{\cov}{Cov}
\DeclareMathOperator{\st}{st}
\DeclareMathOperator{\mon}{Mon}
\newcommand{\card}[1]{\left\vert #1 \right\vert}
\newcommand{\trasposta}[1]{\prescript{\text{T}}{}{#1}}
\newcommand{\1}{\mathds{1}}
\newcommand{\R}{\mathds{R}}
\newcommand{\diesis}{\#}
\newcommand{\bemolle}{\flat}
\newcommand{\nonstandard}[1]{\prescript{*}{}{#1}}
\newcommand{\starR}{\nonstandard{\R}}
\newcommand{\borel}{\mathscr{B}}
\newcommand{\lebesgue}[1]{\mathscr{L}\left(#1\right)}
\newcommand{\media}{\mathds{E}}
\newcommand{\K}{\mathds{K}}
\newcommand{\A}{\mathds{A}}
\newcommand{\Q}{\mathds{Q}}
\newcommand{\N}{\mathds{N}}
\newcommand{\C}{\mathds{C}}
\newcommand{\Z}{\mathds{Z}}
\newcommand{\qo}{\hspace{1em}\text{q.o.}\,}
\renewcommand{\tilde}[1]{\widetilde{#1}}
\renewcommand{\parallel}{\mathrel{/\mkern-5mu/}}
\newcommand{\parti}[2][]{\wp_{#1}(#2)}
\newcommand{\diff}[1]{\operatorname{d}_{#1}}
\let\oldvec\vec
\renewcommand{\vec}[1]{\overrightarrow{\vphantom{i}#1}}
\newcommand{\floor}[1]{\left\lfloor #1 \right\rfloor}
\newcommand{\cat}[1]{\mathbf{#1}}
\newcommand{\dfreccia}[1]{\xrightarrow{\ #1 \ }}
\newcommand{\sfreccia}[1]{\xleftarrow{\ #1 \ }}
\newcommand{\formalsum}[2]{{\sum_{#1}^{#2}}{\vphantom{\sum}}'}
\newcommand{\minim}[2]{\mu_{#1}\, \left(#2\right)}
\newcommand{\concat}{\null^{\frown}} % concatenazione di stringe
\newcommand{\godelcode}[1]{\langle\!\langle #1 \rangle\!\rangle}
\newcommand{\godeldec}[1]{(\!(#1)\!)}
\newcommand{\termcode}[1]{\ulcorner #1\urcorner}
\newcommand{\partialto}{\dashrightarrow}
\newcommand{\restricted}{\upharpoonright}
\newcommand{\embeds}{\precsim}
\newcommand{\surjects}{\twoheadrightarrow}
\newcommand{\equipotenti}{\asymp}
%% \newcommand{\dotplus}{\mathbin{\dot{+}}} %% A quanto pare esiste già
\newcommand{\bigdot}{\mathbin{\boldsymbol{\cdot}}}
\newcommand{\dotexp}[1]{^{.#1}}
\newcommand{\conv}{\mathbin{*}}
\newcommand{\convolution}[2]{(#1\conv #2)}
\newcommand{\nil}{\mathfrak{N}}
\newcommand{\divisore}{\mathrel{|}}
\newcommand{\simplesso}[1]{\mathrm{e}_{#1}}

\renewcommand{\iff}{\mathrel{\longleftrightarrow}} %% Notazione Logica.
\newcommand{\oldiff}{\mathrel{\Longleftrightarrow}}
\renewcommand{\implies}{\mathrel{\rightarrow}} %% Notazione Logica
\newcommand{\oldimplies}{\mathrel{\Longrightarrow}}
\renewcommand{\impliedby}{\mathrel{\leftarrow}} %% Notazione Logica
\newcommand{\oldimpliedby}{\mathrel{\Longleftarrow}}

\newcommand{\IFF}{\quad\Longleftrightarrow\quad}
\newcommand{\IMPLICA}{\quad\Longrightarrow\quad}


\renewcommand{\descriptionlabel}[1]{\hspace{\labelsep}\normalfont #1} % remove bold from description


%% Definizione di Divergenza di K-L

\DeclarePairedDelimiterX{\infdivx}[2]{(}{)}{%
  #1\;\delimsize\|\;#2%
}
\newcommand{\kldiv}{D_{KL}\infdivx}

%% Definizione di \dotminus

\makeatletter
\newcommand{\dotminus}{\mathbin{\text{\@dotminus}}}

\newcommand{\@dotminus}{%
  \ooalign{\hidewidth\raise1ex\hbox{.}\hidewidth\cr$\m@th-$\cr}%
}
\makeatother

%tramite i prossimi due comandi posso decidere come scrivere i logaritmi naturali in tutti i documenti: ho infatti eliminato qualsiasi differenza tra "ln" e "log": se si vuole qualcosa di diverso bisogna inserire manualmente il tutto
\let\ln\relax
\DeclareMathOperator{\ln}{ln}
\let\log\relax
\DeclareMathOperator{\log}{log}
%%%%%%

%% NUOVI COMANDI
\newcommand{\straniero}[1]{\textit{#1}} %parole straniere
\newcommand{\titolo}[1]{\textsc{#1}} %titoli
\newcommand{\qedd}{\tag*{$\blacksquare$}} %qed per ambienti matemastici
\renewcommand{\qedsymbol}{$\blacksquare$} %modifica colore qed
\newcommand{\ooverline}[1]{\overline{\overline{#1}}}
\newcommand{\circoletto}[1]{\left(#1\right)^{\text{o}}}
%
\newcommand{\qmatrice}[1]{\begin{pmatrix}
#1_{11} & \cdots & #1_{1n}\\
\vdots & \ddots & \vdots \\
#1_{m1} & \cdots & #1_{mn}
\end{pmatrix}}
%
\newcommand{\parentesi}[2]{%
\underset{#1}{\underbrace{#2}}%
}
%
\newcommand{\norma}[1]{% Norma
\left\lVert#1\right\rVert%
}
\newcommand{\scalare}[2]{% Scalare
\left\langle #1, #2\right\rangle
}
%%%%%

%% RESTRIZIONI
\newcommand{\referenze}[2]{
        \phantomsection{}#2\textsuperscript{\textcolor{blue}{\textbf{#1}}}
}

\let\restriction\relax

\def\restriction#1#2{\mathchoice
              {\setbox1\hbox{${\displaystyle #1}_{\scriptstyle #2}$}
              \restrictionaux{#1}{#2}}
              {\setbox1\hbox{${\textstyle #1}_{\scriptstyle #2}$}
              \restrictionaux{#1}{#2}}
              {\setbox1\hbox{${\scriptstyle #1}_{\scriptscriptstyle #2}$}
              \restrictionaux{#1}{#2}}
              {\setbox1\hbox{${\scriptscriptstyle #1}_{\scriptscriptstyle #2}$}
              \restrictionaux{#1}{#2}}}
\def\restrictionaux#1#2{{#1\,\smash{\vrule height .8\ht1 depth .85\dp1}}_{\,#2}}
%%%%%%%%%%%

%%% FORMATTAZIONE FOOTNOTEMARK

\def\footnotemarkformatting#1{[#1]}
\renewcommand{\thefootnote}{\footnotemarkformatting{\arabic{footnote}}}

%% SEZIONE GRAFICA
\use{tikz}
\usetikzlibrary{matrix, patterns, calc, decorations.pathreplacing, hobby, decorations.markings, decorations.pathmorphing, babel}
\use{tikz-3dplot}
\use{mathrsfs} %per geogebra
\use{tikz-cd}
\tikzset
{
  %surface/.style={fill=black!10, shading=ball,fill opacity=0.4},
  plane/.style={black,pattern=north east lines},
  curve/.style={black,line width=0.5mm},
  dritto/.style={decoration={markings,mark=at position 0.5 with {\arrow{Stealth}}}, postaction=decorate},
  rovescio/.style={decoration={markings,mark=at position 0.5 with {\arrow{Stealth[reversed]}}}, postaction=decorate}
}
\use{pgfplots} % stampare le funzioni
        \pgfplotsset{/pgf/number format/use comma,compat=1.15}
        %\pgfplotsset{compat=1.15} %per geogebra
        \usepgfplotslibrary{fillbetween, polar}
%%%%%%

%% CITAZIONI
\use{lineno}

\newcommand{\citazione}[1]{%
  \begin{quotation}
  \begin{linenumbers}
  \modulolinenumbers[5]
  \begingroup
  \setlength{\parindent}{0cm}
  \noindent #1
  \endgroup
  \end{linenumbers}
  \end{quotation}\setcounter{linenumber}{1}
  }
%%%%%%

%%%%%%%%%%%%%%%%%%%%%%%%%%%%%%%%%%%%%%%%%%%%
%%%%%%%%%%%%%%%%%%%%%%%%%%%%%%%%%%%%%%%%%%%%

%% AMS THM

\theoremstyle{definition}% default
\newtheorem{thm}{Teorema}[section]
\newtheorem{lem}[thm]{Lemma}
\newtheorem{prop}[thm]{Proposizione}
\newtheorem{cor}[thm]{Corollario}
\newtheorem{esempio}[thm]{Esempio}
\theoremstyle{plain}
\newtheorem{definizione}[thm]{Definizione}
\theoremstyle{remark}
\newtheorem*{oss}{Osservazione}


%%%%%%%%%%%%%%%%%%%%%%%%%%%%%%%%%%%%%%%%%%%%
%%%%%%%%%%%%%%%%%%%%%%%%%%%%%%%%%%%%%%%%%%%%

\use{hyperref}
\hypersetup{%
        pdfauthor={Davide Peccioli},
        pdfsubject={},
        allcolors=black,
        citecolor=black,
%	colorlinks=true,
        bookmarksopen=true}
\setcounter{secnumdepth}{0} % rimuove i numeri di sezione senza rimuovere le ref
\renewcommand{\href}[2]{\textcolor{blue}{#2}} % disabilita il comando href
\use{enotez} %
\setenotez{%
 mark-format = \footnotemarkformatting % Mette i numeri tra parentesi quadre%
}\let\footnote=\endnote % rende tutte le note a pié pagina come delle note a fine file 


\let\olddocument\document % modifico l'ambiende documenti per non dover stampare \printendnote
\let\oldenddocument\enddocument
\renewenvironment{document}%
{%
  \olddocument
}{%
  \printendnotes\oldenddocument
}
\renewcommand{\thethm}{\arabic{thm}}

\usepackage[hyperref]{biblatex}
\addbibresource{~/Documents/org/roam/bib/master.bib}
\author{Davide Peccioli}
\date{\today}
\title{}
\begin{document}

\section{Coomologia in dimensione massima}
\label{sec:org46d5ebc}
Si indica con \(H^{k}\) la \href{20251115172442-gruppo_di_coomologia_di_de_rham.org}{Coomologia di De Rham}, con \(H^{k}_{\text{c}}\) \href{20251115192241-coomologia_a_supporto_compatto.org}{coomologia a supporto compatto}, e con \(\cong\) gli \href{20250113125833-isomorfismo_tra_spazi_vettoriali.org}{isomorfismi}
\begin{thm}
Sia \(M\) una \href{20250113115909-struttura_differenziabile.org}{varietà differenziabile} \href{20250103165325-spazio_topologico_connesso.org}{connessa} di dimensione \(n\).
\begin{enumerate}
\item Se \(M\) è \uline{\href{20251223152054-varieta_differenziabile_orientabile.org}{orientabile} e \href{20250103163701-spazio_topologico_compatto.org}{compatta}}, allora
\begin{equation*}
 H^{n}(M) \cong \R.
\end{equation*}
\item Se \(M\) è \uline{\href{20251223152054-varieta_differenziabile_orientabile.org}{orientabile} e \href{20250103163701-spazio_topologico_compatto.org}{non compatta}}, allora
\begin{equation*}
 H^{n}(M) = 0.
\end{equation*}
\item Se \(M\) è \uline{\href{20251223152054-varieta_differenziabile_orientabile.org}{non orientabile}}, allora
\begin{equation*}
 H^{n}(M) = 0.
\end{equation*}
\item Se \(M\) è \uline{\href{20251223152054-varieta_differenziabile_orientabile.org}{orientabile}}, allora
\begin{equation*}
 H^{n}_{\text{c}}(M) \cong \R.
\end{equation*}
\item Se \(M\) è \uline{\href{20251223152054-varieta_differenziabile_orientabile.org}{non orientabile}}, allora
\begin{equation*}
 H^{n}_{\text{c}}(M) = 0.
\end{equation*}
\end{enumerate}
\end{thm}
\begin{proof}
\begin{enumerate}
\item Per \href{20251229114152-dualita_di_poincare.org}{DP} e \href{20251229114152-dualita_di_poincare.org}{DP per varietà compatte}
\begin{equation*}
 H^{n}(M) \cong H^{0}(M)
\end{equation*}
e \href{20251115174538-0_gruppo_di_coomologia_di_de_rham_di_una_varieta_connessa.org}{siccome} \(M\) connessa
\begin{equation*}
  H^{0}(M) \cong \R.
\end{equation*}
\item Siccome \(M\) è connessa, allora \(H^{0}_{\text{c}}(M) = 0\) (\href{20251115174538-0_gruppo_di_coomologia_di_de_rham_di_una_varieta_connessa.org}{come per \(H^{0}\)}). Quindi per \href{20251229114152-dualita_di_poincare.org}{DP}
\begin{equation*}
 H^{n}(M) \cong H^{0}_{\text{c}}(M) = 0.
\end{equation*}

\item Come nel calcolo della \href{20251115190905-coomologia_dello_spazio_proiettivo_reale.org}{coomologia di \(\mathds{P}^{n}\R\)}, per \(M\) non orientabile esiste \(\tilde{M}\) \href{20251223152054-varieta_differenziabile_orientabile.org}{orientabile} \(n\)-dimensionale e
\begin{equation*}
 \pi:\tilde{M} \to M
\end{equation*}
\href{20250103103252-funzione_continua.org}{continua} e \href{20241213105600-funzione_suriettiva.org}{suriettiva} tale che il \href{20251121174615-pullback_di_una_funzione_tra_varieta_differenziabili_in_coomologia.org}{pullback}
\begin{equation*}
 \bm{\pi}^{*}: H^{n}(M) \to H^{n}(\tilde{M})
\end{equation*}
è \href{20241219101956-funzione_iniettiva.org}{iniettivo}

\begin{itemize}
\item Se \(M\) è non compatta, allora \(\tilde{M}\) è non compatta (poiché l'\href{20250202190147-immagine_punto_a_punto_di_due_classi.org}{immagine} \(\pi[\tilde{M}] = M\) e \href{20251229125103-immagine_continua_di_spazio_compatto_e_compatto.org}{immagine continua di spazio compatto è compatto}).

Per 2. allora \(H^{n}(\tilde{M}) = 0\) e siccome \(\bm{\pi}^{*}\) è iniettiva allora \(H^{n}(M) = 0\).

\item Se \(M\) è compatta, allora \(\tilde{M}\) è compatta. Per dimostrare che \(H^{n}(M) = 0\) è sufficiente mostrare che per ogni \(\omega\), \([\omega] = 0\). Per iniettività di \(\bm{\pi}^{*}\), è sufficiente mostrare che \(\bm{\pi}^{*}[\omega] = 0\) per ogni \(\omega\).

Si considerino quindi:
\begin{itemize}
\item il \href{20251115174001-pullback_di_una_funzione_tra_varieta_differenziabili.org}{pullback}:
\begin{equation*}
 \pi^{*}: A^{n}(M) \to A^{n}(\tilde{M})
\end{equation*}

\item la mappa \(A:\tilde{M} \to \tilde{M}\) tale che \(A^{2} = \Id\) e che scambia l'orientazione di \(M\).
\end{itemize}

Quindi, per ogni \(\omega\), siccome \(A\) inverte l'orientazione:
\begin{equation*}
   \int_{\tilde{M}} \pi^{*} \omega = - \int_{\tilde{M}}A^{*} \pi^{*}\omega.
\end{equation*}
Ma \(A^{*}\pi^{*}\omega = (\pi\circ A)^{*} \omega = \pi^{*}\omega\), e quindi
 \begin{equation*}
- \int_{\tilde{M}}A^{*} \pi^{*}\omega = -\int_{\tilde{M}} \pi^{*} \omega.
 \end{equation*}
Segue che \(\int_{\tilde{M}} \pi^{*} \omega = 0\) e, per la \href{20251229114152-dualita_di_poincare.org}{DP}, \(\pi^{*}\omega = 0\).

Pertanto, \(\bm{\pi}^{*}[\omega] = [\pi^{*} \omega] = 0\).
\end{itemize}

\item Per \href{20251229114152-dualita_di_poincare.org}{DP} \(H^{n}_{\text{c}}(M)^{*} \cong H^{0}(M) \cong \R\), quindi \(\dim H^{n}_{\text{c}}(M)^{*}<\infty\) e dunque \(H^{n}_{\text{c}}(M)^{*} \cong H^{n}_{\text{c}}(M)\)
\begin{equation*}
 H^{n}_{\text{c}}(M) \cong \R.
\end{equation*}

\item Lasciato per esercizio, simile a 3.\qedhere
\end{enumerate}
\end{proof}
\begin{oss}
Mettendo insieme l'isomorfismo dato da DP e quello tra spazio vettoriale e il suo duale, si ottiene che, se \(M\) è una \href{20250113115909-struttura_differenziabile.org}{varietà differenziabile} \href{20250103165325-spazio_topologico_connesso.org}{connessa}, \href{20251223152054-varieta_differenziabile_orientabile.org}{orientabile} e \href{20250103163701-spazio_topologico_compatto.org}{compatta}, e \(\dim M = n\), allora
\begin{align*}
\int_{M}: H^{n}(M) &\longrightarrow \R\\
[\omega] &\longmapsto \int_{M}\omega
\end{align*}
è un \href{20250113125833-isomorfismo_tra_spazi_vettoriali.org}{isomorfismo}.

Questo segue anche dal fatto che:
\begin{enumerate}
\item \(\int_{M}\) è una mappa lineare;
\item è suriettiva, poiché \(\int_{M} \nu \neq 0\)\footnote{Infatti \href{20251115184544-forma_volume_su_una_varieta_differenziabile.org}{l'integrale della forma volume è sempre positivo.}}, per \(\nu\) \href{20251115184544-forma_volume_su_una_varieta_differenziabile.org}{forma volume} (che esiste per la \href{20251115185324-caratterizzazione_varieta_differenziabile_orientabile_tramite_forma_voluma.org}{caratterizzazione} delle \href{20251223152054-varieta_differenziabile_orientabile.org}{v.d. orientabili})
\item è iniettiva, in quanto \(\ker \int_{M} \neq H^{n}(M)\) (poiché \(\int_{M} \nu \neq 0\), per il \href{20251230144150-teorema_nullita_rango.org}{Teorema Nullità + Rango})
\end{enumerate}
\end{oss}
\begin{oss}
Se \(M\) è una \href{20250113115909-struttura_differenziabile.org}{varietà differenziabile} qualsiasi, allora la \href{20241205142027-spazio_vettoriale.org}{dimensione} di \href{20251115192241-coomologia_a_supporto_compatto.org}{\(H^{n}_{\text{c}}(M)\)} è uguale al numero di \href{20250325160128-componente_connessa_di_uno_spazio_topologico.org}{componenti connesse} \href{20251223152054-varieta_differenziabile_orientabile.org}{orientabili} di \(M\).
\end{oss}
\begin{oss}
Quindi, se \(M\) è una \href{20250113115909-struttura_differenziabile.org}{varietà differenziabile} \href{20250103165325-spazio_topologico_connesso.org}{connessa} e \href{20251223152054-varieta_differenziabile_orientabile.org}{orientabile} di dimensione \(n\), esiste \href{20251229110021-forma_differenziale_a_supporto_compatto.org}{\(\omega_{M} \in A^{n}_{\text{c}}(M)\)} tale che
\begin{enumerate}
\item La \href{20251115192241-coomologia_a_supporto_compatto.org}{coomologia a supporto compatto} \(H^{n}_{\text{c}}(M) = \langle [\omega_{M}] \rangle\)\footnote{Vedi ``\href{20250102163502-base_di_uno_spazio_vettoriale.org}{Base di uno spazio vettoriale}''};
\item \(\int_{M} \omega_{M} = 1\).
\end{enumerate}

Infatti, per la \href{20251229114152-dualita_di_poincare.org}{Dualità di Poincaré} si ha il seguente \href{20250113125833-isomorfismo_tra_spazi_vettoriali.org}{isomorfismo} con lo \href{20250105124008-spazio_vettoriale_duale.org}{spazio duale}\footnote{Vedi anche ``\href{20251115172442-gruppo_di_coomologia_di_de_rham.org}{Gruppo di Coomologia di De Rham}''}:
\begin{equation*}
\begin{tikzcd}
	{H^0(M)} & {H^n_{\text{c}}(M)^*} \\
	{[\omega]} & {\displaystyle\bigg([\tau]\mapsto\int_M\omega\wedge \tau\bigg)}
	\arrow["\cong", from=1-1, to=1-2]
	\arrow[maps to, from=2-1, to=2-2]
\end{tikzcd}
\end{equation*}
\begin{itemize}
\item Siccome \(M\) è connessa, \href{20251115174538-0_gruppo_di_coomologia_di_de_rham_di_una_varieta_connessa.org}{allora} \(\dim H^{0}(M) = 1 < +\infty\).
\item Pertanto \(\dim H^{n}_{\text{c}}(M)^{*} = 1\) (per la dualità di Poincaré).
\item Quindi esiste
\begin{equation*}
  F: H^{n}_{\text{c}}(M) \to \R
\end{equation*}
non nulla. \href{20251230144150-teorema_nullita_rango.org}{Quindi} \(F\) \href{20250113125833-isomorfismo_tra_spazi_vettoriali.org}{isomorfismo}, ed esiste \([\tau_{M}] \in H^{n}_{\text{c}}(M)\) non nullo tale che \(F[\tau_{M}] = 1\).
\item Per l'isomorfismo dalla dualità di Poincaré, esiste \([\omega_{0}] \in H^{0}(M)\) tale che
\begin{align*}
F: H^{n}_{\text{c}}(M) &\longrightarrow \R\\
\displaystyle [\tau] &\longmapsto \int_{M}\omega_{0}\wedge\tau
\end{align*}
\item In particolare, quindi
\begin{equation*}
  1 = F[\tau_{M}] = \int_{M} \omega_{0} \wedge \tau_{M}.
\end{equation*}
\end{itemize}

Si pone \(\omega_{M} \coloneqq \omega_{0}\wedge \tau_{M}\).
\begin{itemize}
\item Sicuramente \([\omega_{M}] \in H^{n}_{\text{c}}(M)\), per definizione del \href{20251115175943-prodotto_wedge_in_coomologia_di_de_rham.org}{prodotto wedge}.
\item \([\omega_{M}] \neq 0\) (e quindi genera tutto lo spazio): infatti, se per assurdo \([\omega_{M}] = 0\) \href{20251115172517-forma_differenziale_chiusa.org}{allora \(\omega_{M}\) è esatta}, e quindi, per il \href{20251115190058-teorema_di_stokes.org}{Teorema di Stokes},
\begin{equation*}
  \int_{M} \omega_{M} = 0.
\end{equation*}
Assurdo.
\end{itemize}
\end{oss}
\end{document}
