% Created 2026-02-07 Sat 19:31
% Intended LaTeX compiler: pdflatex
\documentclass[10pt]{article}
%% CREATO CON ORG - EMACS
\newcommand{\use}[2][]{\usepackage[#1]{#2}}
% PACCHETTI FONDAMENTLAI
\use[utf8]{inputenc}
\use[T1]{fontenc}
\use{graphicx}
\use{longtable}
\use{wrapfig}
\use{rotating}
\use[normalem]{ulem}
\use{amsmath}
\use{amsthm}
\use{amssymb}

\use{eucal} % Cambia mathcal{...}

\use{capt-of}
\use[italian]{babel}
\use[babel]{csquotes}
% bib la TEX lo carica in automatico org-cite
\use{microtype}
\use{lmodern}
\use{subfig} % sottofigure
\use{multicol} % due colonne
\use{lipsum} % lorem ipsum
\use{color} % colori in latex
\use{parskip} % rimuove l'indentazione dei nuovi paragrafi %% Add parbox=false to all new tcolorbox
\use{centernot}
\use[outline]{contour}\contourlength{3pt}
\use{fancyhdr}
\use{layout}
\use[most]{tcolorbox} % Riquadri colorati
\use{ifthen} % IFTHEN
\use{geometry}

% pacchetti matematica
\use{yhmath}
\use{dsfont}
\use{mathrsfs}
\use{cancel} % semplificare
\use{polynom} %divisione tra polinomi
\use{forest} % grafi ad albero
\use{booktabs} % tabelle
\use{commath} %simboli e differenziali
\use{bm} %bold
\use[fulladjust]{marginnote} %to use marginnote for date notes
\use{arrayjobx}%array
\use[intlimits]{empheq} % Riquadri colorati attorno alle equazioni
\use{mathtools}
\use{circuitikz} % Disegnare i circuiti
\use{mathtools}
\use{stmaryrd} % [[ \llbracket ]] \rrbracket
\use{bussproofs} % dimostrazioni

%%%%%%%%%%%%%


%%%% QUIVER
\newcommand{\duepunti}{\,\mathchar\numexpr"6000+`:\relax\,}
% A TikZ style for curved arrows of a fixed height, due to AndréC.
\tikzset{curve/.style={settings={#1},to path={(\tikztostart)
    .. controls ($(\tikztostart)!\pv{pos}!(\tikztotarget)!\pv{height}!270:(\tikztotarget)$)
    and ($(\tikztostart)!1-\pv{pos}!(\tikztotarget)!\pv{height}!270:(\tikztotarget)$)
    .. (\tikztotarget)\tikztonodes}},
    settings/.code={\tikzset{quiver/.cd,#1}
        \def\pv##1{\pgfkeysvalueof{/tikz/quiver/##1}}},
    quiver/.cd,pos/.initial=0.35,height/.initial=0}

% TikZ arrowhead/tail styles.
\tikzset{tail reversed/.code={\pgfsetarrowsstart{tikzcd to}}}
\tikzset{2tail/.code={\pgfsetarrowsstart{Implies[reversed]}}}
\tikzset{2tail reversed/.code={\pgfsetarrowsstart{Implies}}}
% TikZ arrow styles.
\tikzset{no body/.style={/tikz/dash pattern=on 0 off 1mm}}
%%%%%%%%%%


%% DEFINIZIONI COMANDI MATEMATICI
\let\sin\relax %TOGLIE LA DEFINIZIONE SU "\sin"

% cambia la definizione di empty set
% ---
\let\oldemptyset\emptyset
% ---
% \let\emptyset\varnothing
% ---
% \let\emptyset\relax
% \newcommand{\emptyset}{\text{\textnormal{\O}}}
% ---

\DeclareMathOperator{\bounded}{bd}
\DeclareMathOperator{\sin}{sen}
\DeclareMathOperator{\epi}{Epi}
\DeclareMathOperator{\cl}{cl}
\DeclareMathOperator{\graph}{graph}
\DeclareMathOperator{\arcsec}{arcsec}
\DeclareMathOperator{\arccot}{arccot}
\DeclareMathOperator{\arccsc}{arccsc}
\DeclareMathOperator{\spettro}{Spettro}
\DeclareMathOperator{\nulls}{nullspace}
\DeclareMathOperator{\dom}{dom}
\DeclareMathOperator{\ar}{ar}
\DeclareMathOperator{\const}{Const}
\DeclareMathOperator{\fun}{Fun}
\DeclareMathOperator{\rel}{Rel}
\DeclareMathOperator{\altezza}{ht}
\let\det\relax %TOGLIE LA DEFINIZIONE SU "\det"
\DeclareMathOperator{\det}{det}
\DeclareMathOperator{\End}{End}
\DeclareMathOperator{\gl}{GL}
\def\Id{\mathrm{Id}}
\def\id{\mathrm{id}}
\DeclareMathOperator{\I}{\mathds{1}}
\DeclareMathOperator{\II}{II}
\DeclareMathOperator{\rank}{rank}
\DeclareMathOperator{\tr}{tr}
\DeclareMathOperator{\tc}{t.c.}
\DeclareMathOperator{\T}{T}
\DeclareMathOperator{\var}{Var}
\DeclareMathOperator{\cov}{Cov}
\DeclareMathOperator{\st}{st}
\DeclareMathOperator{\mon}{Mon}
\newcommand{\card}[1]{\left\vert #1 \right\vert}
\newcommand{\trasposta}[1]{\prescript{\text{T}}{}{#1}}
\newcommand{\1}{\mathds{1}}
\newcommand{\R}{\mathds{R}}
\newcommand{\diesis}{\#}
\newcommand{\bemolle}{\flat}
\newcommand{\nonstandard}[1]{\prescript{*}{}{#1}}
\newcommand{\starR}{\nonstandard{\R}}
\newcommand{\borel}{\mathscr{B}}
\newcommand{\lebesgue}[1]{\mathscr{L}\left(#1\right)}
\newcommand{\media}{\mathds{E}}
\newcommand{\K}{\mathds{K}}
\newcommand{\A}{\mathds{A}}
\newcommand{\Q}{\mathds{Q}}
\newcommand{\N}{\mathds{N}}
\newcommand{\C}{\mathds{C}}
\newcommand{\Z}{\mathds{Z}}
\newcommand{\qo}{\hspace{1em}\text{q.o.}\,}
\renewcommand{\tilde}[1]{\widetilde{#1}}
\renewcommand{\parallel}{\mathrel{/\mkern-5mu/}}
\newcommand{\parti}[2][]{\wp_{#1}(#2)}
\newcommand{\diff}[1]{\operatorname{d}_{#1}}
\let\oldvec\vec
\renewcommand{\vec}[1]{\overrightarrow{\vphantom{i}#1}}
\newcommand{\floor}[1]{\left\lfloor #1 \right\rfloor}
\newcommand{\cat}[1]{\mathbf{#1}}
\newcommand{\dfreccia}[1]{\xrightarrow{\ #1 \ }}
\newcommand{\sfreccia}[1]{\xleftarrow{\ #1 \ }}
\newcommand{\formalsum}[2]{{\sum_{#1}^{#2}}{\vphantom{\sum}}'}
\newcommand{\minim}[2]{\mu_{#1}\, \left(#2\right)}
\newcommand{\concat}{\null^{\frown}} % concatenazione di stringe
\newcommand{\godelcode}[1]{\langle\!\langle #1 \rangle\!\rangle}
\newcommand{\godeldec}[1]{(\!(#1)\!)}
\newcommand{\termcode}[1]{\ulcorner #1\urcorner}
\newcommand{\partialto}{\dashrightarrow}
\newcommand{\restricted}{\upharpoonright}
\newcommand{\embeds}{\precsim}
\newcommand{\surjects}{\twoheadrightarrow}
\newcommand{\equipotenti}{\asymp}
%% \newcommand{\dotplus}{\mathbin{\dot{+}}} %% A quanto pare esiste già
\newcommand{\bigdot}{\mathbin{\boldsymbol{\cdot}}}
\newcommand{\dotexp}[1]{^{.#1}}
\newcommand{\conv}{\mathbin{*}}
\newcommand{\convolution}[2]{(#1\conv #2)}
\newcommand{\nil}{\mathfrak{N}}
\newcommand{\divisore}{\mathrel{|}}
\newcommand{\simplesso}[1]{\mathrm{e}_{#1}}

\renewcommand{\iff}{\mathrel{\longleftrightarrow}} %% Notazione Logica.
\newcommand{\oldiff}{\mathrel{\Longleftrightarrow}}
\renewcommand{\implies}{\mathrel{\rightarrow}} %% Notazione Logica
\newcommand{\oldimplies}{\mathrel{\Longrightarrow}}
\renewcommand{\impliedby}{\mathrel{\leftarrow}} %% Notazione Logica
\newcommand{\oldimpliedby}{\mathrel{\Longleftarrow}}

\newcommand{\IFF}{\quad\Longleftrightarrow\quad}
\newcommand{\IMPLICA}{\quad\Longrightarrow\quad}


\renewcommand{\descriptionlabel}[1]{\hspace{\labelsep}\normalfont #1} % remove bold from description


%% Definizione di Divergenza di K-L

\DeclarePairedDelimiterX{\infdivx}[2]{(}{)}{%
  #1\;\delimsize\|\;#2%
}
\newcommand{\kldiv}{D_{KL}\infdivx}

%% Definizione di \dotminus

\makeatletter
\newcommand{\dotminus}{\mathbin{\text{\@dotminus}}}

\newcommand{\@dotminus}{%
  \ooalign{\hidewidth\raise1ex\hbox{.}\hidewidth\cr$\m@th-$\cr}%
}
\makeatother

%tramite i prossimi due comandi posso decidere come scrivere i logaritmi naturali in tutti i documenti: ho infatti eliminato qualsiasi differenza tra "ln" e "log": se si vuole qualcosa di diverso bisogna inserire manualmente il tutto
\let\ln\relax
\DeclareMathOperator{\ln}{ln}
\let\log\relax
\DeclareMathOperator{\log}{log}
%%%%%%

%% NUOVI COMANDI
\newcommand{\straniero}[1]{\textit{#1}} %parole straniere
\newcommand{\titolo}[1]{\textsc{#1}} %titoli
\newcommand{\qedd}{\tag*{$\blacksquare$}} %qed per ambienti matemastici
\renewcommand{\qedsymbol}{$\blacksquare$} %modifica colore qed
\newcommand{\ooverline}[1]{\overline{\overline{#1}}}
\newcommand{\circoletto}[1]{\left(#1\right)^{\text{o}}}
%
\newcommand{\qmatrice}[1]{\begin{pmatrix}
#1_{11} & \cdots & #1_{1n}\\
\vdots & \ddots & \vdots \\
#1_{m1} & \cdots & #1_{mn}
\end{pmatrix}}
%
\newcommand{\parentesi}[2]{%
\underset{#1}{\underbrace{#2}}%
}
%
\newcommand{\norma}[1]{% Norma
\left\lVert#1\right\rVert%
}
\newcommand{\scalare}[2]{% Scalare
\left\langle #1, #2\right\rangle
}
%%%%%

%% RESTRIZIONI
\newcommand{\referenze}[2]{
        \phantomsection{}#2\textsuperscript{\textcolor{blue}{\textbf{#1}}}
}

\let\restriction\relax

\def\restriction#1#2{\mathchoice
              {\setbox1\hbox{${\displaystyle #1}_{\scriptstyle #2}$}
              \restrictionaux{#1}{#2}}
              {\setbox1\hbox{${\textstyle #1}_{\scriptstyle #2}$}
              \restrictionaux{#1}{#2}}
              {\setbox1\hbox{${\scriptstyle #1}_{\scriptscriptstyle #2}$}
              \restrictionaux{#1}{#2}}
              {\setbox1\hbox{${\scriptscriptstyle #1}_{\scriptscriptstyle #2}$}
              \restrictionaux{#1}{#2}}}
\def\restrictionaux#1#2{{#1\,\smash{\vrule height .8\ht1 depth .85\dp1}}_{\,#2}}
%%%%%%%%%%%

%%% FORMATTAZIONE FOOTNOTEMARK

\def\footnotemarkformatting#1{[#1]}
\renewcommand{\thefootnote}{\footnotemarkformatting{\arabic{footnote}}}

%% SEZIONE GRAFICA
\use{tikz}
\usetikzlibrary{matrix, patterns, calc, decorations.pathreplacing, hobby, decorations.markings, decorations.pathmorphing, babel}
\use{tikz-3dplot}
\use{mathrsfs} %per geogebra
\use{tikz-cd}
\tikzset
{
  %surface/.style={fill=black!10, shading=ball,fill opacity=0.4},
  plane/.style={black,pattern=north east lines},
  curve/.style={black,line width=0.5mm},
  dritto/.style={decoration={markings,mark=at position 0.5 with {\arrow{Stealth}}}, postaction=decorate},
  rovescio/.style={decoration={markings,mark=at position 0.5 with {\arrow{Stealth[reversed]}}}, postaction=decorate}
}
\use{pgfplots} % stampare le funzioni
        \pgfplotsset{/pgf/number format/use comma,compat=1.15}
        %\pgfplotsset{compat=1.15} %per geogebra
        \usepgfplotslibrary{fillbetween, polar}
%%%%%%

%% CITAZIONI
\use{lineno}

\newcommand{\citazione}[1]{%
  \begin{quotation}
  \begin{linenumbers}
  \modulolinenumbers[5]
  \begingroup
  \setlength{\parindent}{0cm}
  \noindent #1
  \endgroup
  \end{linenumbers}
  \end{quotation}\setcounter{linenumber}{1}
  }
%%%%%%

%%%%%%%%%%%%%%%%%%%%%%%%%%%%%%%%%%%%%%%%%%%%
%%%%%%%%%%%%%%%%%%%%%%%%%%%%%%%%%%%%%%%%%%%%

%% AMS THM

\theoremstyle{definition}% default
\newtheorem{thm}{Teorema}[section]
\newtheorem{lem}[thm]{Lemma}
\newtheorem{prop}[thm]{Proposizione}
\newtheorem{cor}[thm]{Corollario}
\newtheorem{esempio}[thm]{Esempio}
\theoremstyle{plain}
\newtheorem{definizione}[thm]{Definizione}
\theoremstyle{remark}
\newtheorem*{oss}{Osservazione}


%%%%%%%%%%%%%%%%%%%%%%%%%%%%%%%%%%%%%%%%%%%%
%%%%%%%%%%%%%%%%%%%%%%%%%%%%%%%%%%%%%%%%%%%%

\use{hyperref}
\hypersetup{%
        pdfauthor={Davide Peccioli},
        pdfsubject={},
        allcolors=black,
        citecolor=black,
%	colorlinks=true,
        bookmarksopen=true}
\setcounter{secnumdepth}{0} % rimuove i numeri di sezione senza rimuovere le ref
\renewcommand{\href}[2]{\textcolor{blue}{#2}} % disabilita il comando href
\use{enotez} %
\setenotez{%
 mark-format = \footnotemarkformatting % Mette i numeri tra parentesi quadre%
}\let\footnote=\endnote % rende tutte le note a pié pagina come delle note a fine file 


\let\olddocument\document % modifico l'ambiende documenti per non dover stampare \printendnote
\let\oldenddocument\enddocument
\renewenvironment{document}%
{%
  \olddocument
}{%
  \printendnotes\oldenddocument
}
\renewcommand{\thethm}{\arabic{thm}}

\usepackage[hyperref]{biblatex}
\addbibresource{~/Documents/org/roam/bib/master.bib}
\author{Davide Peccioli}
\date{\today}
\title{Esempi di Fasci e Prefasci}
\begin{document}

\section{Esempi di Fasci e Prefasci}
\label{sec:orgd3bbbb3}
\begin{esempio}
Si consideri \(X\) \href{20250103145124-topologia.org}{spazio topologico} qualsiasi. Allora i seguenti, con la \href{20250205170515-restrizione_di_una_classe.org}{restrizione ovvia}, sono dei \href{20250324165349-prefascio.org}{prefasci} di \href{20250110175552-algebra_su_un_campo.org}{\(\R\)-algebre}: per ogni \(U \subseteq X\) \href{20250103145124-topologia.org}{aperto}
\begin{itemize}
\item \(\mathcal{F}_{X}(U)\coloneqq\set{f:U\to \R}\);
\item \(\mathcal{C}_{X}(U) \coloneqq \set{f:U \to \R\text{ continua}}\), fascio delle funzioni continue.
\end{itemize}
Entrambi sono anche dei \href{20250324174728-fascio.org}{fasci}: per il secondo esempio è necessario ricorrere al \href{20250127144302-lemma_di_incollamento.org}{Lemma di Incollamento}.
\end{esempio}
\subsection{Forme differenziali come fascio}
\label{sec:orge23db47}
\begin{esempio}
Sia \(X\) varietà differenziabile. Il seguente è un \href{20250324165349-prefascio.org}{prefascio} di \href{20241205142027-spazio_vettoriale.org}{spazi vettoriali reali}:
\begin{itemize}
\item per ogni \(U \subseteq X\) aperto: \(\Omega^{p}(U)\), lo spazio delle \href{20251115155511-forma_differenziale_in_un_punto.org}{\(p\)-forme differenziali \(\mathcal{C}^{\infty}\)} su \(U\);
\item come \href{20251201155413-restrizione_di_una_forma_ad_una_sottovarieta.org}{restrizione quella ovvia}.
\end{itemize}
Questo è anche un \href{20250324174728-fascio.org}{fascio}. Si indica con \(\mathcal{A}_{X}^{p} = \Omega_{X}^{p}\).
\end{esempio}
\subsection{Funzioni Cinfinito da una varietà ai reali come fascio}
\label{sec:org4252aab}
\begin{esempio}
Sia \(X\) varietà differenziabile. Il seguente è un \href{20250324165349-prefascio.org}{prefascio} di \href{20250110175552-algebra_su_un_campo.org}{\(\R\)-algebre}:
\begin{itemize}
\item per ogni \(U \subseteq X\) aperto: \(\mathcal{C}^{\infty}(U)\coloneqq \set{f:U\to \R\text{ funzioni }\mathcal{C}^{\infty}}\)\footnote{Vedi ``\href{20250113144722-funzioni_cinfinito_tra_varieta_differenziabili.org}{Anello delle funzioni Cinfinito da una varietà ai reali}''};
\item come \href{20250205170515-restrizione_di_una_classe.org}{restrizione quella ovvia}.
\end{itemize}
Questo è anche un \href{20250324174728-fascio.org}{fascio}. Si indica con \(\mathcal{C}_{X}^{\infty}\).
\end{esempio}
\subsection{Prefascio delle funzioni costanti a valori in un gruppo}
\label{sec:org8ae5511}
Sia \(X\) uno \href{20250103145124-topologia.org}{spazio topologico}, e sia \(G\) un gruppo. Il \href{20250324165349-prefascio.org}{prefascio} delle \href{20250325160105-funzione_costante.org}{funzioni costanti} a valori in \(G\) è, rispetto alle \href{20250205170515-restrizione_di_una_classe.org}{restrizioni} di funzioni, per ogni \(U \subseteq X\) \href{20250103145124-topologia.org}{aperto}
\begin{equation*}
\mathcal{F}(U) \coloneqq\set{f:U\to G\text{ costante}}
\end{equation*}
Questo \uline{non è}, in generale, un \href{20250324174728-fascio.org}{fascio}. Infatti, siano \(A,B \subseteq X\) aperti, non vuoti e disgiunti, e sia \(U\coloneqq A\cup B\). Allora \(\set{A,B}\) è un \href{20250103164252-ricoprimento.org}{ricoprimento} \href{20250103145124-topologia.org}{aperto} di \(U\), e
\begin{equation*}
f_{A}: A\to G: a\mapsto g_{A},\qquad f_{B}:B\to G: b\mapsto g_{B}
\end{equation*}
con \(g_{A}\neq g_{B}\) sono tali che \(f_{A} \in \mathcal{F}(A)\) e \(f_{B} \in \mathcal{F}(B)\), ma non esiste una funzione costante \(f:X\to G\) tale che
\begin{equation*}
\restriction{f}{A} = f_{A},\qquad \restriction{f}{B} = f_{B}.
\end{equation*}
\subsection{Sezioni Cinfinito di un fibrato vettoriale come fascio}
\label{sec:orgcdad47c}
\begin{esempio}
Sia \(X\) una \href{20250113115909-struttura_differenziabile.org}{varietà differenziabile}, \(E \xrightarrow{\pi} X\) un fibrato vettoriale \(C^{\infty}\). Si definisce un \href{20250324174728-fascio.org}{fascio} di \href{20241205142027-spazio_vettoriale.org}{spazi vettoriali reali} come segue:
\begin{itemize}
\item Si pone:\footnote{Vedi:
\begin{itemize}
\item \href{20250202190147-immagine_punto_a_punto_di_due_classi.org}{Immagine e retroimmagine tramite una funzione}
\end{itemize}}
\begin{align*}
  \mathcal{F}(U) &\coloneqq \set{\text{sezioni \(C^{\infty}\) di \(R\) ristrette a \(U\)}}\\
  &= \set{s:U\to \pi^{-1}[U]\text{ di classe \(C^{\infty}\)} \mid \pi \circ s = \Id_{U}}
\end{align*}
\item come \href{20250205170515-restrizione_di_una_classe.org}{restrizione quella ovvia}.
\end{itemize}
\end{esempio}
\subsection{Fascio delle funzioni olomorfe}
\label{sec:orgb3cae47}
\begin{definizione}
Sia \(X\) un aperto di \(\C\) o di \(\C^{n}\). Si definisce il \textbf{\href{20250324174728-fascio.org}{fascio} delle \href{20260126110551-funzione_olomorfa.org}{funzioni olomorfe}} \(\mathcal{O}_{X}\) come segue:
\begin{itemize}
\item si pone
\begin{equation*}
  \mathcal{O}_{X}(U) = \set{f:U\to \C \mid f\text{ olomorfa}}.
\end{equation*}
\item come \href{20250205170515-restrizione_di_una_classe.org}{restrizione quella ovvia}.
\end{itemize}
Questo è un fascio di \href{20250110175552-algebra_su_un_campo.org}{\(\C\)-algebre}.
\end{definizione}
\subsection{Fascio delle funzioni olomorfe mai nulle}
\label{sec:org9da810a}
\begin{definizione}
Sia \(X\) un aperto di \(\C\) o di \(\C^{n}\). Si definisce il \textbf{\href{20250324174728-fascio.org}{fascio} delle \href{20260126110551-funzione_olomorfa.org}{funzioni olomorfe}} \(\mathcal{O}_{X}^{*}\) come segue:
\begin{itemize}
\item si pone
\begin{equation*}
  \mathcal{O}_{X}^{*}(U) = \set{f:U\to \C \mid f\text{ olomorfa} \land 0 \notin f[U]}.
\end{equation*}
\item come \href{20250205170515-restrizione_di_una_classe.org}{restrizione quella ovvia}.
\end{itemize}
Questo è un fascio di \href{20250127093245-gruppo_abeliano.org}{gruppi abeliani}.
\end{definizione}
\subsection{Fascio delle funzioni regolari su una varietà algebrica qp}
\label{sec:org19be40a}
\begin{definizione}
Sia \(X\) una varietà algebrica q.p. su \(\K\) campo algebricamente chiuso.\footnote{\(X\) è dotato della \href{20250107112123-varieta_algebrica_quasi_proiettiva_qp.org}{topologia di Zariski}.} Si definisce il \textbf{\href{20250324174728-fascio.org}{fascio} delle funzioni regolari} \(\mathcal{O}_{X}\) come segue:
\begin{itemize}
\item si pone
\begin{equation*}
  \mathcal{O}_{X}(U) = \set{f:U\to \K \mid f\text{ regolare}}.
\end{equation*}
\item come \href{20250205170515-restrizione_di_una_classe.org}{restrizione quella ovvia}.
\end{itemize}
Questo è un fascio di \href{20250110175552-algebra_su_un_campo.org}{\(\K\)-algebre}.
\end{definizione}
\end{document}
