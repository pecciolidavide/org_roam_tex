\documentclass[10pt]{article}

\usepackage{subfiles}
\input{preamble.tex}

\begin{document}

\subfile{subfiles/20241126100904-categoria}
\subfile{subfiles/20241126171926-topologiaalgebrica}
\subfile{subfiles/20241128141258-inverso_categoriale}
\subfile{subfiles/20241128162125-isomorfismo}
\subfile{subfiles/20241128162518-unicita_inverso}
\subfile{subfiles/20241201115908-istituzioni_di_geometria}
\subfile{subfiles/20241201115941-istituzioni_di_analisi}
\subfile{subfiles/20241201120006-istituzioni_di_logica}
\subfile{subfiles/20241201120053-istituzioni_di_fisica}
\subfile{subfiles/20241204222455-funtore_covariante}
\subfile{subfiles/20241204223502-funtore_controvariante}
\subfile{subfiles/20241204223712-gruppo_fondamentale}
\subfile{subfiles/20241205113007-categorie_esempi}
\subfile{subfiles/20241205115542-categoria_set}
\subfile{subfiles/20241205115600-categoria_top}
\subfile{subfiles/20241205115614-categoria_top}
\subfile{subfiles/20241205115631-categoria_grp}
\subfile{subfiles/20241205115727-categoria_vectr}
\subfile{subfiles/20241205131908-funtori_e_isomorfismi}
\subfile{subfiles/20241205131958-funtore}
\subfile{subfiles/20241205132705-trasformazioni_naturali}
\subfile{subfiles/20241205135638-categoria_top2}
\subfile{subfiles/20241205141053-r_moduli}
\subfile{subfiles/20241205141119-anello}
\subfile{subfiles/20241205141146-gruppo_abeliano}
\subfile{subfiles/20241205142027-spazio_vettoriale}
\subfile{subfiles/20241205142049-campo}
\subfile{subfiles/20241206115416-morfismi_r_moduli}
\subfile{subfiles/20241206115531-morfismo_di_gruppi}
\subfile{subfiles/20241206115740-categoria_degli_r_moduli}
\subfile{subfiles/20241206142802-sottomoduli}
\subfile{subfiles/20241206143051-sottogruppo}
\subfile{subfiles/20241206145529-quoziente}
\subfile{subfiles/20241212141101-generatori_modulo}
\subfile{subfiles/20241212142019-insiemi_linearmente_indipendenti}
\subfile{subfiles/20241213093511-gruppi_zp}
\subfile{subfiles/20241213094625-modulo_libero}
\subfile{subfiles/20241213095808-somma_diretta}
\subfile{subfiles/20241213100845-modulo_finitamente_generato}
\subfile{subfiles/20241213101621-teorema_della_base}
\subfile{subfiles/20241213101756-cardinalita}
\subfile{subfiles/20241213102346-moduli_quoziente_mod_liberi}
\subfile{subfiles/20241213105036-primo_teorema_di_isomorfismo}
\subfile{subfiles/20241213105201-kernel}
\subfile{subfiles/20241213105600-funzione_suriettiva}
\subfile{subfiles/20241213110005-caratterizzazione_moduli_fg}
\subfile{subfiles/20241217101345-scomposizione_modulo_quozienti_liberi}
\subfile{subfiles/20241219101956-funzione_iniettiva}
\subfile{subfiles/20241219112842-pid}
\subfile{subfiles/20241219112955-ideale}
\subfile{subfiles/20241219113154-ideale_generato}
\subfile{subfiles/20241219113434-anello_dei_polinomi}
\subfile{subfiles/20241219192830-pid_sottomoduli_sono_liberi}
\subfile{subfiles/20241220000402-torsione_moduli}
\subfile{subfiles/20241231112713-campo_algebricamente_chiuso}
\subfile{subfiles/20241231112750-polinomio}
\subfile{subfiles/20241231112823-radici_polinomiali}
\subfile{subfiles/20241231114009-spazio_affine}
\subfile{subfiles/20241231114256-varieta_algebrica_affine}
\subfile{subfiles/20241231115051-spazio_proiettivo}
\subfile{subfiles/20241231121125-polinomi_omogenei}
\subfile{subfiles/20241231123223-varieta_algebrica_proiettiva}
\subfile{subfiles/20241231124230-omogenizzazione}
\subfile{subfiles/20241231124742-grado_polinomi}
\subfile{subfiles/20241231124828-deomogenizzazione}
\subfile{subfiles/20250102103618-infinitezza_campi_alg_chiusi}
\subfile{subfiles/20250102104043-cubica_gobba}
\subfile{subfiles/20250102104749-curva_piana}
\subfile{subfiles/20250102115740-anello_dei_polinomi_ad_ideali_principali}
\subfile{subfiles/20250102115942-anello_noetheriano}
\subfile{subfiles/20250102120132-emmy_noether}
\subfile{subfiles/20250102120655-catena_stazionaria}
\subfile{subfiles/20250102120722-acc}
\subfile{subfiles/20250102120836-catena}
\subfile{subfiles/20250102141936-insiemi_finiti_di_punti_dello_spazio_proiettivo}
\subfile{subfiles/20250102142629-posizione_generale}
\subfile{subfiles/20250102143739-insiemi_finiti_di_punti_dello_spazio_affine}
\subfile{subfiles/20250102144306-iperpiano_proeittivo}
\subfile{subfiles/20250102154204-teorema_fondamentale_dell_algebra}
\subfile{subfiles/20250102163502-base_di_uno_spazio_vettoriale}
\subfile{subfiles/20250102163844-polinomio_di_lagrange}
\subfile{subfiles/20250102165143-ideali_generati_di_un_anello_noetheriano}
\subfile{subfiles/20250102165420-teorema_della_base_di_hilbert}
\subfile{subfiles/20250102175222-campi_sono_anelli_noetheriani}
\subfile{subfiles/20250102175446-ideali_di_un_campo}
\subfile{subfiles/20250102180050-zeri_di_un_ideale_generato_in_uno_spazio_affine}
\subfile{subfiles/20250102182726-ideale_di_polinomi_omogeneo}
\subfile{subfiles/20250102183523-luogo_di_zeri_di_un_ideale_omogeneo}
\subfile{subfiles/20250103100601-varieta_algebriche_luogo_di_zeri_di_finiti_polinomi}
\subfile{subfiles/20250103101459-topologia_di_zariski_affine}
\subfile{subfiles/20250103103232-topologia_euclidea}
\subfile{subfiles/20250103103252-funzione_continua}
\subfile{subfiles/20250103143950-dominio_di_integrita}
\subfile{subfiles/20250103144124-ideale_di_un_sottoinsieme}
\subfile{subfiles/20250103144944-chiusura_topologica}
\subfile{subfiles/20250103145124-topologia}
\subfile{subfiles/20250103145915-zeri_di_un_ideale_di_un_sottoinsieme_affine}
\subfile{subfiles/20250103152646-spazio_topologico_noetheriano}
\subfile{subfiles/20250103153610-topologia_di_zariski_affine_e_noetheriana}
\subfile{subfiles/20250103155209-caratterizzazione_topologia_noetheriana}
\subfile{subfiles/20250103163621-spazio_topologico_noetheriano_e_compatto}
\subfile{subfiles/20250103163701-spazio_topologico_compatto}
\subfile{subfiles/20250103163814-sottospazio_topologico}
\subfile{subfiles/20250103164252-ricoprimento}
\subfile{subfiles/20250103164917-sottospazio_irriducibile}
\subfile{subfiles/20250103165325-spazio_topologico_connesso}
\subfile{subfiles/20250103170236-scomposizione_di_sp_top_noetheriani_in_componenti_irriducibili}
\subfile{subfiles/20250103170329-caratterizzazione_di_sottospazi_affini_irriducibili_tramite_ideali}
\subfile{subfiles/20250103171055-ideale_primo}
\subfile{subfiles/20250103180214-topologia_di_zariski_proiettiva}
\subfile{subfiles/20250104095230-proprieta_operazioni_tra_insimi}
\subfile{subfiles/20250104100907-spazio_affine_e_irriducibile}
\subfile{subfiles/20250104101516-caratterizzazione_domini_d_integrita_tramite_ideali_primi}
\subfile{subfiles/20250104110524-morfismo_tra_varieta_algebriche_affini}
\subfile{subfiles/20250104111539-spazio_delle_matrici}
\subfile{subfiles/20250104111707-funzione_biunivoca}
\subfile{subfiles/20250104111735-matrice_invertibile}
\subfile{subfiles/20250104111751-determinante_di_una_matrice}
\subfile{subfiles/20250104112443-grafico_di_una_funzione}
\subfile{subfiles/20250104112713-morfismo_tra_varieta_algebriche_affini_e_continuo}
\subfile{subfiles/20250104114505-morfismo_tra_varieta_algebriche_affini_non_e_chiuso}
\subfile{subfiles/20250104114559-funzione_chiusa}
\subfile{subfiles/20250104115050-spazio_affine_unidimensionale_senza_un_punto_non_e_chiuso}
\subfile{subfiles/20250104120600-morfismo_tra_varieta_algebriche_proiettive}
\subfile{subfiles/20250104121634-curva_razionale_normale}
\subfile{subfiles/20250104121837-mappa_di_veronese}
\subfile{subfiles/20250104170945-rango_di_una_matrice}
\subfile{subfiles/20250104171415-minori_di_una_matrice}
\subfile{subfiles/20250104190620-mappa_di_segre}
\subfile{subfiles/20250105122522-multi_indice}
\subfile{subfiles/20250105124008-spazio_vettoriale_duale}
\subfile{subfiles/20250107112123-varieta_algebrica_quasi_proiettiva_qp}
\subfile{subfiles/20250107112308-insieme_localmente_chiuso}
\subfile{subfiles/20250107112412-morfismo_tra_varieta_algebriche_qp}
\subfile{subfiles/20250107123816-proiezioni_da_prodotti_di_varieta_proiettive_sono_morfisi}
\subfile{subfiles/20250107143702-copie_dello_spazio_proiettivo_dentro_la_varieta_di_segre_sono_lineari}
\subfile{subfiles/20250107170928-varieta_proiettiva_dentro_prodotti_di_spazi_proiettivi}
\subfile{subfiles/20250107172231-polinomi_biomogenei}
\subfile{subfiles/20250107172249-bigrado_di_un_polinomio}
\subfile{subfiles/20250108102640-prodotti_di_varieta_qp}
\subfile{subfiles/20250108103928-proprieta_universale_del_prodotto}
\subfile{subfiles/20250108124906-grafico_di_un_morfismo_proiettivo_e_varieta}
\subfile{subfiles/20250108160159-intersezione_con_un_ricoprimento_aperto_e_chiusa_allora_chiuso}
\subfile{subfiles/20250108162146-proiezione_su_un_iperpiano_dentro_allo_spazio_proiettivo}
\subfile{subfiles/20250108172559-teorema_proiezione_su_un_iperpiano_di_una_varieta_e_varieta}
\subfile{subfiles/20250108173056-fattori_non_costanti_comuni_tra_polinomi}
\subfile{subfiles/20250108173116-ufd}
\subfile{subfiles/20250108173845-fattore_comune}
\subfile{subfiles/20250108174027-divisione}
\subfile{subfiles/20250108181703-combinazione_lineare}
\subfile{subfiles/20250109095734-risultante_per_polinomi_in_piu_variabili}
\subfile{subfiles/20250109101841-retta_proiettiva}
\subfile{subfiles/20250109115213-zeri_di_un_ideale_di_un_sottoinsieme_proiettivo}
\subfile{subfiles/20250109120343-cono_in_uno_spazio_proiettivo}
\subfile{subfiles/20250109141249-morfismo_da_varieta_proiettiva_a_varieta_qp_e_chiuso}
\subfile{subfiles/20250109154613-teorema_di_kuratowski_mrowka}
\subfile{subfiles/20250109154723-topologia_prodotto}
\subfile{subfiles/20250109155520-diagonale_di_uno_spazio_topologico}
\subfile{subfiles/20250109155704-caratterizzazione_spazi_t2_con_la_diagonale}
\subfile{subfiles/20250109155715-spazio_topologico_di_hausdorff}
\subfile{subfiles/20250109164824-morfismo_da_varieta_proiettiva_connessa_allo_spazio_affine_unidimensionale_e_costante}
\subfile{subfiles/20250109164935-ipersuperfici_dello_spazio_proiettivo_si_intersecano_con_varieta_connesse}
\subfile{subfiles/20250109165800-immagine_continua_di_spazio_connesso_e_connessa}
\subfile{subfiles/20250110103632-varieta_algebrica_ipersuperficie}
\subfile{subfiles/20250110125225-radicale_di_un_ideale}
\subfile{subfiles/20250110125621-radicale_di_un_ideale_e_ideale}
\subfile{subfiles/20250110142129-teorema_del_coefficiente_binomiale}
\subfile{subfiles/20250110142357-ideale_radicale}
\subfile{subfiles/20250110143120-nullstellensatz}
\subfile{subfiles/20250110143644-nullstellensatz_debole}
\subfile{subfiles/20250110144451-ideale_primo_e_radicale}
\subfile{subfiles/20250110145609-nullstellensatz_debole_implica_nullstellensatz}
\subfile{subfiles/20250110150011-corrispondenza_ideali_radicali_e_chiusi_algebrici_dello_spazio_affine}
\subfile{subfiles/20250110174110-anello_delle_coordinate}
\subfile{subfiles/20250110175521-k_algebre_fin_generate_e_ridotte_come_anelli_delle_coordinate_di_varieta_affini}
\subfile{subfiles/20250110175552-algebra_su_un_campo}
\subfile{subfiles/20250110175843-sottoanello}
\subfile{subfiles/20250110180014-algebra_su_un_campo_finitamente_generata}
\subfile{subfiles/20250110181430-anello_ridotto}
\subfile{subfiles/20250110181511-nilpotenza}
\subfile{subfiles/20250110182255-ideale_radicale_e_anello_quoziente}
\subfile{subfiles/20250110182342-quoziente_di_anello_e_ideale}
\subfile{subfiles/20250111092123-varieta_topologica}
\subfile{subfiles/20250111142303-spazio_topologico_a_base_numerabile}
\subfile{subfiles/20250111142313-intorno}
\subfile{subfiles/20250111142332-omeomorfismo}
\subfile{subfiles/20250111142446-funzione_inversa}
\subfile{subfiles/20250111142837-base_di_una_topologia}
\subfile{subfiles/20250111143651-insieme_numerabile}
\subfile{subfiles/20250111161343-esempi_fondamentali_di_varieta_topologiche}
\subfile{subfiles/20250111161808-immagini_tramiti_omeomorfismi_di_varieta_topologiche_sono_varieta_topologiche}
\subfile{subfiles/20250113095039-atlante_topologico}
\subfile{subfiles/20250113095300-la_sfera_n_dimensionale_e_varieta_topologica}
\subfile{subfiles/20250113095404-spazio_proiettivo_e_varieta_topologica}
\subfile{subfiles/20250113095724-spazio_delle_matrici_invertibili_e_varieta_topologica}
\subfile{subfiles/20250113095822-spazio_delle_matrici_invertibili}
\subfile{subfiles/20250113100307-gruppo_lineare_complesso_e_varieta_topologica}
\subfile{subfiles/20250113100451-spazio_topologico_semplicemente_connesso}
\subfile{subfiles/20250113102456-connessione_per_archi_di_varieta_topologiche}
\subfile{subfiles/20250113102837-spazio_topologico_localmente_connesso_per_archi}
\subfile{subfiles/20250113103025-spazio_topologico_connesso_per_archi}
\subfile{subfiles/20250113103136-atlante_topologico_differenziabile}
\subfile{subfiles/20250113103405-cambio_di_carte_per_un_atlante_topologico}
\subfile{subfiles/20250113103415-diffeomorfismo}
\subfile{subfiles/20250113104517-atlante_astratto}
\subfile{subfiles/20250113105334-compatibilita_tra_atlanti_astratti_e_varieta_topologiche}
\subfile{subfiles/20250113110035-atlanti_differenziabili_compatibili}
\subfile{subfiles/20250113110148-relazione_di_equivalenza}
\subfile{subfiles/20250113115909-struttura_differenziabile}
\subfile{subfiles/20250113120151-esistenza_e_unicita_di_una_struttura_differenziabile_per_ogni_atlante}
\subfile{subfiles/20250113120835-esempi_fondamentali_di_varieta_differenziabili}
\subfile{subfiles/20250113121849-varieta_topologica_omeomorfa_a_varieta_differenziabile_ne_eredita_la_struttura}
\subfile{subfiles/20250113122627-varieta_topologiche_in_dimensione_minore_uguale_a_tre_hanno_struttura_differenziabile}
\subfile{subfiles/20250113122808-strutture_differenziabili_sulla_sfera_7_dimensionale}
\subfile{subfiles/20250113124917-prodotto_di_varieta_differenziabili_e_varieta_differenziabile}
\subfile{subfiles/20250113125032-toro_ha_struttura_di_varieta_differenziabile}
\subfile{subfiles/20250113125113-toro}
\subfile{subfiles/20250113125429-teorema_dell_inversa_locale}
\subfile{subfiles/20250113125602-classe_c_di_una_funzione}
\subfile{subfiles/20250113125641-differenziale_di_una_funzione_reale}
\subfile{subfiles/20250113125833-isomorfismo_tra_spazi_vettoriali}
\subfile{subfiles/20250113125903-teorema_della_funzione_implicita}
\subfile{subfiles/20250113130125-valore_regolare_di_una_funzione}
\subfile{subfiles/20250113141843-differenziale_del_determinante_di_una_matrice_reale}
\subfile{subfiles/20250113142038-traccia_di_una_matrice}
\subfile{subfiles/20250113142135-gruppo_lineare_speciale_e_varieta_differenziabile}
\subfile{subfiles/20250113142317-gruppo_lineare_speciale}
\subfile{subfiles/20250113144147-gruppo_ortogonale_e_una_varieta_differenziabile}
\subfile{subfiles/20250113144228-gruppo_ortogonale}
\subfile{subfiles/20250113144338-matrice_trasposta}
\subfile{subfiles/20250113144514-gruppo_lineare_e_sconnesso_e_compatto}
\subfile{subfiles/20250113144722-funzioni_cinfinito_tra_varieta_differenziabili}
\subfile{subfiles/20250113151718-restrizione_di_funzioni_reali_cinfinito_su_varieta_differenziabili_definite_in_forma_implicita_e_ancora_cinfinito}
\subfile{subfiles/20250113152036-lemma_di_incollamento_tra_funzioni_cinfinito_tra_varieta_differenziabili}
\subfile{subfiles/20250113152503-aperti_di_una_varieta_differenziabile_sono_varieta_differenziabili}
\subfile{subfiles/20250113154629-funzione_sulle_carte_indotta_da_funzione_cinfinito_tra_due_varieta_e_cinfinito}
\subfile{subfiles/20250113172736-composizione_di_funzioni_cinfinito_e_cinfinito}
\subfile{subfiles/20250113172924-diffeomorfismo_tra_varieta_differenziabili}
\subfile{subfiles/20250113173218-diffeomorfismo_tra_le_strutture_differenziali_sui_numeri_reali}
\subfile{subfiles/20250113173255-gruppo_dei_diffeomorfismi_di_una_varieta_differenziabile}
\subfile{subfiles/20250113175231-rivestimento}
\subfile{subfiles/20250113175700-unione_disgiunta}
\subfile{subfiles/20250114094653-rivestimenti_e_gruppo_fondamentale}
\subfile{subfiles/20250114095222-funtore_del_gruppo_fondamentale}
\subfile{subfiles/20250114095243-teorema_del_rivestimento_universale}
\subfile{subfiles/20250114095744-rivestimento_n_1}
\subfile{subfiles/20250114095901-teorema_sui_rivestimenti_di_varieta_differenziabili}
\subfile{subfiles/20250114100254-germi_di_funzioni}
\subfile{subfiles/20250114100810-quoziente_rispetto_a_relazione_di_equivalenza}
\subfile{subfiles/20250114101157-spazio_dei_germi_di_funzioni_e_algebra_reale}
\subfile{subfiles/20250114101437-derivazione_su_una_varieta_differenziabile}
\subfile{subfiles/20250114101917-derivazioni_canoniche_su_una_varieta_differenziabile}
\subfile{subfiles/20250114101949-funzione_lineare}
\subfile{subfiles/20250114102823-spazio_tangente_ad_un_punto_di_una_varieta_differenziabile}
\subfile{subfiles/20250114103118-sottospazio_vettoriale}
\subfile{subfiles/20250114103136-spazio_delle_funzioni_lineare}
\subfile{subfiles/20250114103236-derivata_parziale}
\subfile{subfiles/20250114103339-teorema_sulla_base_dello_spazio_tangente_ad_un_punto_di_una_varieta_differenziabile}
\subfile{subfiles/20250114104914-cambio_di_base_sullo_spazio_tangente_ad_un_punto_di_una_varieta_differenziabile}
\subfile{subfiles/20250114105929-matrice_del_cambiamento_di_base}
\subfile{subfiles/20250114105957-notazione_di_einstein}
\subfile{subfiles/20250114110247-curve_su_varieta_e_loro_derivata}
\subfile{subfiles/20250114110703-derivata}
\subfile{subfiles/20250114111331-differenziale_di_una_funzione_tra_varieta_differenziabili}
\subfile{subfiles/20250114112126-differenziale_di_composizione_di_funzioni_tra_varieta}
\subfile{subfiles/20250114112633-differenziale_di_diffeomorfismo_tra_varieta_e_isomorfismo}
\subfile{subfiles/20250114112834-differenziale_in_coordinate_locali_su_una_varieta_differenziabile}
\subfile{subfiles/20250114123754-matrice_jacobiana}
\subfile{subfiles/20250114124407-cambiamento_del_differenziale_in_coordinate_locali_tramite_un_cambio_di_coordinate_su_varieta_differenziali}
\subfile{subfiles/20250114124504-immersione_di_varieta_differenziabili}
\subfile{subfiles/20250114124533-embedding_di_varieta_differenziabili}
\subfile{subfiles/20250114124541-sottovarieta_differenziabile}
\subfile{subfiles/20250114130106-esempio_di_sottovarieta_tramite_il_teorema_della_funzione_implicita}
\subfile{subfiles/20250114130223-teorema_delle_slice_delle_sottovarieta}
\subfile{subfiles/20250114131231-teorema_del_rango_per_funzioni_reali}
\subfile{subfiles/20250114132151-teorema_della_funzione_implicita_tra_varieta_differenziabili}
\subfile{subfiles/20250114151913-immagine_dello_spazio_tangente_ad_un_punto_di_una_sottovarieta_tramite_il_differenziale_dell_immersione}
\subfile{subfiles/20250114151933-immersione_da_compatti_e_embedding}
\subfile{subfiles/20250115095737-restrizione_di_funzioni_a_sottovarieta_e_loro_differenziale}
\subfile{subfiles/20250115100507-funzione_propria}
\subfile{subfiles/20250115100745-compattezza_e_compattezza_per_successioni}
\subfile{subfiles/20250115100856-spazio_topologico_compatto_per_successioni}
\subfile{subfiles/20250115100904-successione}
\subfile{subfiles/20250115100916-sottosuccessione}
\subfile{subfiles/20250115100930-convergenza_per_una_successione}
\subfile{subfiles/20250115100954-funzioni_cinfintio_tra_varieta_e_proprie_sono_chiuse}
\subfile{subfiles/20250115102238-immersione_iniettiva_chiusa_e_embedding}
\subfile{subfiles/20250115102602-teoremi_di_whitney}
\subfile{subfiles/20250115103245-fibrato_tangente}
\subfile{subfiles/20250115104113-atlante_differenziabile_induce_una_topologia}
\subfile{subfiles/20250115105535-sommersione_di_varieta_differenziabili}
\subfile{subfiles/20250115110259-campo_vettoriale_su_una_varieta_differenziabile}
\subfile{subfiles/20250115113256-componenti_locali_di_un_campo_vettoriale}
\subfile{subfiles/20250115113948-componenti_locali_di_un_campo_vettoriale_sono_cinfinito}
\subfile{subfiles/20250115115535-azione_di_un_campo_vettoriale_su_una_funzione}
\subfile{subfiles/20250115120103-azione_di_un_campo_vettoriale_su_una_funzione_e_cinfinito}
\subfile{subfiles/20250115121514-insieme_dei_derivatori_du_una_varieta}
\subfile{subfiles/20250115122306-insieme_dei_campi_vettoriali_come_spazio_vettoriale_reale_e_come_modulo}
\subfile{subfiles/20250115124556-push_forward_di_campi_vettoriali}
\subfile{subfiles/20250115125404-campi_vettoriali_f_riferiti}
\subfile{subfiles/20250115144427-differenziale_dell_inclusione_del_fibrato_tangente_ad_una_sottovarieta}
\subfile{subfiles/20250115144839-restrizione_di_un_campo_vettoriale_ad_una_sottovarieta}
\subfile{subfiles/20250115145035-varieta_differenziabile_pettinabile}
\subfile{subfiles/20250115145200-pettinabilita_delle_sfere}
\subfile{subfiles/20250115150428-varieta_differenziabile_parallelizzabile}
\subfile{subfiles/20250115150659-parallelizzabilita_delle_sfere}
\subfile{subfiles/20250115150754-sfera_n_dimensionale}
\subfile{subfiles/20250115151027-bracket_di_campi_vettoriali}
\subfile{subfiles/20250115151742-algebra_di_lie}
\subfile{subfiles/20250115151900-scrittura_locale_del_bracket_di_campi_vettoriali}
\subfile{subfiles/20250115162927-bracket_di_campi_vettoriali_f_riferiti}
\subfile{subfiles/20250115163338-bracket_di_campi_vettoriali_ristretti_a_sottovarieta}
\subfile{subfiles/20250120102645-torsione_di_moduli_su_un_dominio_di_integrita}
\subfile{subfiles/20250120103005-modulo_libero_da_torsione}
\subfile{subfiles/20250120103129-modulo_di_torsione}
\subfile{subfiles/20250120103205-modulo_di_torsione_finitamente_generato_e_libero}
\subfile{subfiles/20250120121333-quoziente_di_modulo_fg_e_fg}
\subfile{subfiles/20250120122601-teorema_fondamentale_dei_moduli_finitamente_generati_su_un_pid}
\subfile{subfiles/20250120122938-divisore}
\subfile{subfiles/20250120123610-permutazione}
\subfile{subfiles/20250120124105-modulo_della_somma_diretta_di_m_anelli}
\subfile{subfiles/20250120125004-successione_di_r_moduli_esatta}
\subfile{subfiles/20250120125644-successione_di_r_moduli}
\subfile{subfiles/20250120130155-caratterizzazione_di_alcune_successioni_esatte_di_r_moduli}
\subfile{subfiles/20250120131527-sec}
\subfile{subfiles/20250120131729-teorema_di_spezzamento_sec}
\subfile{subfiles/20250120155457-morfismo_iniettivo_di_r_moduli_induce_isomorfismo}
\subfile{subfiles/20250120155801-lemma_del_cinque}
\subfile{subfiles/20250120163114-complesso_di_catene}
\subfile{subfiles/20250120163759-categoria_complessi_di_catene}
\subfile{subfiles/20250120164857-modulo_di_omologia_dei_complessi_di_catene}
\subfile{subfiles/20250120164930-morfismo_tra_complessi_di_catene_induce_morfismo_tra_moduli_di_omologia}
\subfile{subfiles/20250120164938-zig_zag_lemma}
\subfile{subfiles/20250120165029-funtore_tra_chr_e_rmod}
\subfile{subfiles/20250120183640-sec_di_complessi_di_catene}
\subfile{subfiles/20250121094654-omotopia_tra_funzioni_continue}
\subfile{subfiles/20250121094935-omotopia_tra_morfismi_di_complessi_di_catene}
\subfile{subfiles/20250121100726-funtore_di_omologia_di_funzioni_omotope}
\subfile{subfiles/20250121102813-categoria_omotopica_dei_complessi_di_catene}
\subfile{subfiles/20250121103544-complessi_di_catene_omotopicamente_equivalenti}
\subfile{subfiles/20250121104306-complesso_di_catene_aciclico}
\subfile{subfiles/20250121104453-complesso_di_catene_contraibile}
\subfile{subfiles/20250121105137-complesso_di_catene_contraibile_e_aciclico}
\subfile{subfiles/20250121110644-complesso_di_catene_aciclico_libero_e_contraibile}
\subfile{subfiles/20250121110816-complesso_di_catene_libero}
\subfile{subfiles/20250121121613-indipendenza_affine}
\subfile{subfiles/20250121121923-simplessi}
\subfile{subfiles/20250121122315-simplesso_standard}
\subfile{subfiles/20250121122324-simplesso_standard}
\subfile{subfiles/20250121122621-faccia_di_un_simplesso}
\subfile{subfiles/20250121124147-complesso_simpliciale}
\subfile{subfiles/20250121124213-sottocomplesso_simpliciale}
\subfile{subfiles/20250121124230-supporto_di_un_complesso_simpliciale}
\subfile{subfiles/20250121124249-topologia_debole_di_un_complesso_simpliciale}
\subfile{subfiles/20250121124310-scheletro_di_un_complesso_simpliciale}
\subfile{subfiles/20250121124410-modulo_delle_catene_simpliciali}
\subfile{subfiles/20250121125446-complesso_simpliciale_generato_da_un_simplesso}
\subfile{subfiles/20250121130442-palla_n_dimensionale}
\subfile{subfiles/20250121131005-complesso_simpliciale_totalmente_ordinato}
\subfile{subfiles/20250121131920-notazione_moduli_liberi}
\subfile{subfiles/20250121132856-mappe_di_bordo_tra_moduli_di_catene_simpliciali}
\subfile{subfiles/20250121160600-complesso_di_catene_simpliciali}
\subfile{subfiles/20250121160827-omologia_simpliciale}
\subfile{subfiles/20250121162333-calcolo_dell_omologia_simpliciale_per_il_complesso_di_catene_generato_da_un_2_simplesso}
\subfile{subfiles/20250122100507-omologia_simpliciale_per_il_complesso_di_catene_simpliciali_di_un_complesso_simpliciale_generato_da_un_generico_simplesso}
\subfile{subfiles/20250122100541-notazione_per_i_simplessi}
\subfile{subfiles/20250122101105-complesso_augmentato_di_un_complesso_simpliciale}
\subfile{subfiles/20250122103014-omologia_ridotta_di_un_complesso_simpliciale}
\subfile{subfiles/20250122110747-terzo_teorema_di_isomorfismo}
\subfile{subfiles/20250122111953-complesso_di_catene_relative}
\subfile{subfiles/20250122112115-omologia_simpliciale_relativa}
\subfile{subfiles/20250122112244-successione_esatta_di_una_coppia_di_complesso_e_sottocomplesso_in_omologia}
\subfile{subfiles/20250122122650-quoziente_di_somma_diretta_di_moduli}
\subfile{subfiles/20250122133125-mappa_simpliciale}
\subfile{subfiles/20250122133147-categoria_di_complessi_e_mappe_simpliciali}
\subfile{subfiles/20250122133308-funtore_diesis_da_complessi_simpliciali_a_complessi_di_catene}
\subfile{subfiles/20250122133348-funtore_da_complessi_simpliciali_a_spazi_topologici}
\subfile{subfiles/20250122133435-simplesso_singolare}
\subfile{subfiles/20250122133535-operatori_di_facciata_del_simplesso_standard}
\subfile{subfiles/20250122133614-mappa_di_bordo_tra_moduli_di_catene_singolari}
\subfile{subfiles/20250122133631-omologia_singolare}
\subfile{subfiles/20250122154136-funtorialita_dell_omologia_singolare}
\subfile{subfiles/20250122154153-calcolo_dell_omologia_del_punto}
\subfile{subfiles/20250122154349-complesso_di_catene_singolare_augmentato}
\subfile{subfiles/20250122154406-omologia_singolare_ridotta}
\subfile{subfiles/20250122154451-spazio_topologico_aciclico}
\subfile{subfiles/20250122154457-coprodotto}
\subfile{subfiles/20250122154543-significato_geometrico_del_modulo_di_omologia_singolare_0}
\subfile{subfiles/20250122154613-insieme_stellato}
\subfile{subfiles/20250122154637-insieme_stellato_e_spazio_topologico_aciclico}
\subfile{subfiles/20250122154711-estensione_di_un_simplesso_singolare_in_uno_spazio_stellato}
\subfile{subfiles/20250122154728-coppia_topologica}
\subfile{subfiles/20250122154809-complesso_di_catene_singolari_relative}
\subfile{subfiles/20250122154903-omologia_singolare_relativa}
\subfile{subfiles/20250122154927-successione_esatta_di_una_coppia_topologica}
\subfile{subfiles/20250122155528-teorema_dell_invarianza_per_omotopia}
\subfile{subfiles/20250122155640-spazio_topologico_contraibile}
\subfile{subfiles/20250122155700-spazio_topologico_contraibile_e_aciclico}
\subfile{subfiles/20250122155714-retratto_di_uno_spazio_topologico}
\subfile{subfiles/20250122155727-retratto_di_deformazione_di_uno_spazio_topologico}
\subfile{subfiles/20250122160057-inclusione_di_un_retratto_induce_iniezione_in_omologia_singolare}
\subfile{subfiles/20250122160145-categoria_topp}
\subfile{subfiles/20250122175051-funtore_hn_da_categoria_pl_a_categoria_r_mod}
\subfile{subfiles/20250122181431-parte_interna}
\subfile{subfiles/20250123115927-funtore_di_omologia_singolare}
\subfile{subfiles/20250123124404-riassunto_funtori_di_omologia}
\subfile{subfiles/20250124100209-immagine_continua_di_spazio_cpa_e_cpa}
\subfile{subfiles/20250124103105-segmento_tra_due_punti}
\subfile{subfiles/20250124145633-moduli_di_catene_singolari_per_coppie_topologiche}
\subfile{subfiles/20250124155008-spazi_topologici_omotopicamente_equivalenti}
\subfile{subfiles/20250124162217-composizione_di_funzioni_initiettiva_o_suriettiva}
\subfile{subfiles/20250124163705-funtore_da_topp_a_chr_diesis}
\subfile{subfiles/20250124163710-diagramma_commutativo_dell_omologia_di_due_coppie_topologiche}
\subfile{subfiles/20250125212500-verbale_bmm}
\subfile{subfiles/20250126190440-equivalenze_omotopiche_tra_coppie_topologiche_induce_isomorfismo_tra_omologia_singolare_relativa}
\subfile{subfiles/20250126191208-funtore_da_topp_a_rmod_di_omologia}
\subfile{subfiles/20250126215428-funtore_da_grp_a_ab}
\subfile{subfiles/20250126215522-funtore_da_ab_a_grp_dimenticante}
\subfile{subfiles/20250126215534-teorema_di_hurewicz}
\subfile{subfiles/20250126223310-teorema_di_escissione}
\subfile{subfiles/20250127093245-gruppo_abeliano}
\subfile{subfiles/20250127093602-ogni_funzione_da_un_gruppo_ad_un_gruppo_abeliano_fattorizza_tramite_il_gruppo_abelianizzato}
\subfile{subfiles/20250127093652-gruppo_abelianizzato}
\subfile{subfiles/20250127093729-commutatore_di_un_gruppo}
\subfile{subfiles/20250127093819-quoziente_di_gruppo_e_sottogruppo}
\subfile{subfiles/20250127095321-categoria_ab}
\subfile{subfiles/20250127115631-giustapposizione_di_cammini}
\subfile{subfiles/20250127125010-cammino_inverso}
\subfile{subfiles/20250127144302-lemma_di_incollamento}
\subfile{subfiles/20250127162702-calcolo_dell_omologia_singolare_della_sfera_e_dell_omologia_singolare_relativa_del_disco_rispetto_alla_sfera}
\subfile{subfiles/20250127170831-disco_n_dimensionale}
\subfile{subfiles/20250128111617-immagine_continua_di_spazio_semplicemente_connesso_e_semplicemente_connesso}
\subfile{subfiles/20250128131221-complesso_di_catene_singolare_somma}
\subfile{subfiles/20250128131232-complesso_di_catene_singolare_intersezione}
\subfile{subfiles/20250128131908-suddivisione_baricentrica}
\subfile{subfiles/20250128132040-mappa_di_suddivisione_tra_complessi_di_catene_singolari}
\subfile{subfiles/20250128132104-mappa_di_suddivisione_e_omotopa_a_identita}
\subfile{subfiles/20250128132200-mesh_di_un_elemento_del_modulo_delle_catene_singolari}
\subfile{subfiles/20250128132613-applicazioni_successive_della_mappa_di_suddivisione_finisce_nel_sottomodulo_del_complesso_singolare_somma}
\subfile{subfiles/20250128132648-teorema_di_mayer_vietoris}
\subfile{subfiles/20250128132703-formulazione_differente_del_teorema_di_mayer_vietoris}
\subfile{subfiles/20250128132743-prodotto_wedge_di_spazi_topologici_puntati}
\subfile{subfiles/20250128132815-calcolo_dell_omologia_singolare_del_punto}
\subfile{subfiles/20250128132830-teorema_del_punto_fisso_di_brower}
\subfile{subfiles/20250128132917-omologia_locale}
\subfile{subfiles/20250128132928-omologia_locale_di_una_varieta_topologica}
\subfile{subfiles/20250128133005-teorema_di_invarianza_della_dimensione}
\subfile{subfiles/20250128151459-sottocomplesso_di_catene}
\subfile{subfiles/20250128151511-quoziente_di_complesso_di_catene_e_sottocomplesso}
\subfile{subfiles/20250128161448-secondo_teorema_di_isomorfismo}
\subfile{subfiles/20250129094132-trasformazione_affine}
\subfile{subfiles/20250129100530-lemma_del_numero_di_lebesgue}
\subfile{subfiles/20250129112659-retratto_di_deformazione_forte_di_uno_spazio_topologico}
\subfile{subfiles/20250129155316-spazio_topologico_quoziente}
\subfile{subfiles/20250129160423-punto_fisso}
\subfile{subfiles/20250129161026-bordo}
\subfile{subfiles/20250129170910-grado_di_un_endomorfismo_della_sfera}
\subfile{subfiles/20250129174802-grado_di_una_matrice_ortogonale_come_endomorfismo_della_sfera}
\subfile{subfiles/20250129175123-gruppo_ortogonale_speciale_e_connesso_per_archi}
\subfile{subfiles/20250129175208-gruppo_ortogonale_speciale}
\subfile{subfiles/20250129180509-endomorfismo_di_una_sfera_senza_punti_fissi_e_omotopa_alla_mappa_antipodale}
\subfile{subfiles/20250129183256-coppia_topologica_buona}
\subfile{subfiles/20250129183516-quoziente_di_una_coppia_topologica_buona}
\subfile{subfiles/20250130104245-morse_kelly_set_theory}
\subfile{subfiles/20250130104320-classe_mk}
\subfile{subfiles/20250130104331-insieme_mk}
\subfile{subfiles/20250130104409-paradosso_di_russel}
\subfile{subfiles/20250130114950-teoria_del_prim_ordine}
\subfile{subfiles/20250130162057-linguaggio_del_prim_ordine}
\subfile{subfiles/20250130162316-termine_del_prim_ordine}
\subfile{subfiles/20250131103035-struttura_del_prim_ordine}
\subfile{subfiles/20250131103053-morfismo_tra_strutture_del_prim_ordine}
\subfile{subfiles/20250131103212-sottostruttura_del_prim_ordine}
\subfile{subfiles/20250131103317-formula_del_prim_ordine}
\subfile{subfiles/20250131103429-variabile_libera_di_una_formula}
\subfile{subfiles/20250131103446-enunciato_del_prim_ordine}
\subfile{subfiles/20250131103457-chiusura_universale_di_una_formula}
\subfile{subfiles/20250131122913-soddisfazione_di_una_formula}
\subfile{subfiles/20250131122945-modello_di_un_insieme_di_formule}
\subfile{subfiles/20250131123011-conseguenza_logica}
\subfile{subfiles/20250131123033-equivalenza_logica_tra_due_enunciati}
\subfile{subfiles/20250131123109-insieme_di_assiomi_per_una_teoria}
\subfile{subfiles/20250131123128-teoria_soddisfacibile}
\subfile{subfiles/20250131123151-teoria_completa}
\subfile{subfiles/20250131123208-teorie_elementarmente_equivalente}
\subfile{subfiles/20250131123228-teoria_di_una_struttura}
\subfile{subfiles/20250131123530-formula_valida}
\subfile{subfiles/20250131123540-formula_soddisfacibile}
\subfile{subfiles/20250131123704-sostituzione_di_termini_in_una_formula}
\subfile{subfiles/20250131155822-operazioni_insiemistiche_tra_classi_mk}
\subfile{subfiles/20250131160822-ogni_sottoclasse_di_un_insieme_e_un_insieme_mk}
\subfile{subfiles/20250131160843-ogni_sottoclasse_di_una_classe_propria_e_una_classe_propria_mk}
\subfile{subfiles/20250131161811-insieme_vuoto_mk}
\subfile{subfiles/20250131162451-coppia_ordinata_mk}
\subfile{subfiles/20250131180704-nessun_insieme_appartiene_a_se_stesso}
\subfile{subfiles/20250131183735-prodotto_cartesiano_di_classi_mk}
\subfile{subfiles/20250202124648-successore_di_un_insieme_mk}
\subfile{subfiles/20250202124855-esistono_infiniti_insiemi_mk}
\subfile{subfiles/20250202125627-insieme_induttivo_mk}
\subfile{subfiles/20250202130045-insieme_dei_numeri_naturali_mk}
\subfile{subfiles/20250202170607-classe_relazione_binaria}
\subfile{subfiles/20250202173528-dominio_range_e_campo_di_una_classe_relazione}
\subfile{subfiles/20250202180416-unione_di_relazioni_funzionali_mk}
\subfile{subfiles/20250202184517-ordine_superiormente_diretto}
\subfile{subfiles/20250202190147-immagine_punto_a_punto_di_due_classi}
\subfile{subfiles/20250202192030-classe_delle_classi_funzioni}
\subfile{subfiles/20250203095749-relazione_left_narrow_mk}
\subfile{subfiles/20250203100901-relazione_well_founded_mk}
\subfile{subfiles/20250203101604-ordine}
\subfile{subfiles/20250203102516-massimo_e_minimo}
\subfile{subfiles/20250203104134-buon_ordine_mk}
\subfile{subfiles/20250203104513-classe_totale}
\subfile{subfiles/20250203105434-funzione_di_scelta}
\subfile{subfiles/20250203110432-isomorfismo_tra_ordini}
\subfile{subfiles/20250203110714-classe_transitiva}
\subfile{subfiles/20250203111003-ordinali}
\subfile{subfiles/20250203132953-funzione_monotona}
\subfile{subfiles/20250203133527-insiemi_ben_ordinati_sono_isomorfi_ad_un_ordinale_unico}
\subfile{subfiles/20250203154624-due_classi_proprie_ben_ordinate_sono_isomorfe}
\subfile{subfiles/20250203161043-intersezione_di_una_sottoclasse_degli_ordinali}
\subfile{subfiles/20250203161110-numeri_naturali_sono_ordinali}
\subfile{subfiles/20250203161132-ordinale_limite}
\subfile{subfiles/20250203161243-caratterizzazione_di_ordinali_limite}
\subfile{subfiles/20250203161326-topologia_sugli_ordinali}
\subfile{subfiles/20250203161341-cardinali}
\subfile{subfiles/20250203161431-classe_ben_ordinabile_mk}
\subfile{subfiles/20250204124929-non_esistono_classi_funzioni_a_dominio_naturale_che_formano_una_catena_discendente}
\subfile{subfiles/20250205120448-classe_finita_e_infinita_mk}
\subfile{subfiles/20250205150457-teorema_di_cantor_bernstein_schroder}
\subfile{subfiles/20250205152531-numeri_di_hartogs}
\subfile{subfiles/20250205170515-restrizione_di_una_classe}
\subfile{subfiles/20250205180824-numero_di_hartogs_di_un_ordinale}
\subfile{subfiles/20250205180911-buon_ordine_di_godel_per_ordxord}
\subfile{subfiles/20250205181254-order_type_del_prodotto_cartesiano_di_un_cardinale_e_il_cardinale_stesso}
\subfile{subfiles/20250205182017-insieme_dei_sottoinsiemi_con_order_type_fissato}
\subfile{subfiles/20250205182056-equipotenza_di_insiemi_di_funzioni}
\subfile{subfiles/20250206100719-ordine_prodotto_per_classi}
\subfile{subfiles/20250206100734-ordine_lessicografico_per_classi}
\subfile{subfiles/20250206114139-soluzione_equazione_simone}
\subfile{subfiles/20250206120526-segmento_iniziale_per_un_ordine}
\subfile{subfiles/20250206170922-sequenze_e_stringhe}
\subfile{subfiles/20250206171120-operazione_su_una_classe_mk}
\subfile{subfiles/20250206171508-axiom_of_choiche}
\subfile{subfiles/20250206171704-axiom_of_global_choice}
\subfile{subfiles/20250207104855-funzione_ricorsiva}
\subfile{subfiles/20250207121712-teorema_di_ricorsione_caso_speciale}
\subfile{subfiles/20250207121728-chiusura_transitiva_di_una_classe_mk}
\subfile{subfiles/20250207121738-chiusura_transitiva_di_una_relazione_mk}
\subfile{subfiles/20250207121906-teorema_di_ricorsione}
\subfile{subfiles/20250207122234-rango_di_una_relazione_well_founded}
\subfile{subfiles/20250207122435-relazione_estensionale}
\subfile{subfiles/20250207122453-lemma_del_collasso_di_mostowski}
\subfile{subfiles/20250207122542-collasso_di_mostowski}
\subfile{subfiles/20250207122627-funzione_aleph}
\subfile{subfiles/20250207122856-punto_fisso_di_funzioni_continue_e_crescenti_sugli_ordinali}
\subfile{subfiles/20250207123015-aritmentica_per_gli_ordinali}
\subfile{subfiles/20250207123105-rango_di_un_insieme}
\subfile{subfiles/20250207123246-gerarchia_di_von_neumann}
\subfile{subfiles/20250207123526-modelli_di_zfc_nella_gerarchia_di_von_neumann}
\subfile{subfiles/20250207123940-formula_assoluta}
\subfile{subfiles/20250207124220-zermelo_franklin_set_theory}
\subfile{subfiles/20250207162513-relazione_irriflessiva}
\subfile{subfiles/20250208144924-sottoclasse_di_ord_e_classe_propria_allora_e_ord}
\subfile{subfiles/20250208172824-induzione_transfinita_per_le_relazioni_well_founded}
\subfile{subfiles/20250210095802-elevamento_a_potenza_di_cardinali}
\subfile{subfiles/20250210101346-collezione_dei_sottoinsiemi_ben_ordinabili_di_cardinalita_limitata}
\subfile{subfiles/20250210103126-classe_funzione_beth}
\subfile{subfiles/20250210103648-ipotesi_del_continuo}
\subfile{subfiles/20250210103702-ipotesi_del_continuo_generalizzata}
\subfile{subfiles/20250210104221-ch_e_gch_sono_indipendenti_dall_assiomatizzazione_della_teoria_degli_insiemi}
\subfile{subfiles/20250210104302-forme_deboli_di_ac}
\subfile{subfiles/20250210104427-assiomi_equivalenti_ad_ac}
\subfile{subfiles/20250210104534-ac_e_classi_ben_ordinabili}
\subfile{subfiles/20250210104633-lemma_di_zorn}
\subfile{subfiles/20250210104707-principio_di_massimalita_di_hausdorff}
\subfile{subfiles/20250210104809-insiemi_ben_ordinabile_implica_maxhaus_che_implica_zorn_che_implica_weak_zorn}
\subfile{subfiles/20250210105502-ac_e_comparabilita_delle_cardinalita}
\subfile{subfiles/20250210105804-ipotesi_per_cui_omega1_non_e_unione_numerabile_di_insiemi_numerabili}
\subfile{subfiles/20250210115830-insieme_di_formule_indipendenti}
\subfile{subfiles/20250211104106-somma_di_cardinali_e_minore_del_prodotto_di_cardinali}
\subfile{subfiles/20250211104227-somma_generalizzata_di_cardinali_e_minore_del_supremum_dei_cardinali}
\subfile{subfiles/20250211104310-cardinalita_dell_unione_di_insiemi_e_minore_del_supremum_della_cardinalita_degli_insiemi_per_la_cardinalita_dell_insieme_degli_indici}
\subfile{subfiles/20250211104356-teorema_di_cantor}
\subfile{subfiles/20250211104500-funzione_cofinale}
\subfile{subfiles/20250211104628-proprieta_di_funzioni_cofinali_e_cofinalita_di_un_ordinale}
\subfile{subfiles/20250211104639-ordinale_regolare}
\subfile{subfiles/20250211104941-cardinali_infiniti_hanno_numero_di_hartogs_regolare}
\subfile{subfiles/20250211105023-ogni_cardinale_singolare_e_estremo_superiore_di_una_sequenza_crescente_di_cardinali_regolari}
\subfile{subfiles/20250211105119-ordinale_elevato_alla_sua_cofinalita_e_maggiore_a_se_stesso}
\subfile{subfiles/20250211105133-formula_di_hausdorff}
\subfile{subfiles/20250211105153-teorema_di_bukovsky_hechler}
\subfile{subfiles/20250211105332-chiusura_rispetto_ad_una_collezione_di_operatori_di_una_sottoclasse}
\subfile{subfiles/20250211111630-maggiorazioni_della_cardinalita_della_chiusura_rispetto_ad_una_collezione_di_operatori}
\subfile{subfiles/20250211111723-ordinale_e_compatto_sse_zero_o_successore}
\subfile{subfiles/20250211111812-spazio_topologico_totalmente_disconnesso}
\subfile{subfiles/20250211112135-spazio_topologico_regolare}
\subfile{subfiles/20250211112210-spazio_topologico_completamente_regolare}
\subfile{subfiles/20250211112246-spazio_topologico_completamente_regolare_che_non_surietta_su_r_e_totalmente_disconnesso}
\subfile{subfiles/20250211113036-caratterizzazione_funzioni_continue_e_monotone_da_sottoinsieme_degli_ordinali_agli_ordinali}
\subfile{subfiles/20250211120015-caratterizzazione_di_sottoinsiemi_chiusi_e_illimitati_in_un_cardinale_regolare_o_ord}
\subfile{subfiles/20250211120127-club_set}
\subfile{subfiles/20250211120146-club_set_di_un_cardinale_e_un_filtro_proprio_del_cardinale}
\subfile{subfiles/20250211120754-club_set_di_un_cardinale_e_un_filtro_k_completo}
\subfile{subfiles/20250211121223-filtro_kappa_completo}
\subfile{subfiles/20250211121245-ordinale_chiuso_rispetto_ad_una_operazione}
\subfile{subfiles/20250211121743-insieme_degli_ordinali_chiusi_rispetto_ad_una_operazione_e_sottoinsiemi_di_un_cardinale_chiusi_e_illimitati}
\subfile{subfiles/20250211121805-intersezione_diagonale_di_una_sequenza}
\subfile{subfiles/20250211121853-intersezione_diagonale_di_una_sequenza_di_chiusi_e_illimitati_di_un_cardinale_e_un_chiuso_e_illimitato}
\subfile{subfiles/20250211121910-sottoinsieme_stazionario_di_un_cardinale}
\subfile{subfiles/20250211121932-lemma_di_fodor}
\subfile{subfiles/20250211123144-classe_funzione_esponenziale_sui_cardinali}
\subfile{subfiles/20250211123155-cardinale_limite_forte}
\subfile{subfiles/20250211123243-universo}
\subfile{subfiles/20250211123850-universo_se_e_solo_se_nella_gerarchia_di_von_neumann_di_un_cardinale_fortemente_inaccessibile}
\subfile{subfiles/20250211145622-proprieta_di_prodotto_e_somma_generalizzata_di_cardinali}
\subfile{subfiles/20250211173143-ac_implica_a_si_inietta_in_b_sse_b_si_surietta_su_a}
\subfile{subfiles/20250212100302-interpretazione_di_un_termine}
\subfile{subfiles/20250212100332-sottostruttura_generata_da_un_insieme}
\subfile{subfiles/20250212101432-modello_assiomatizzabile}
\subfile{subfiles/20250212102253-sottostruttura_elementare}
\subfile{subfiles/20250212102412-teoria_di_una_struttura_con_parametri}
\subfile{subfiles/20250212102927-enunciato_con_parametri}
\subfile{subfiles/20250212103145-sottostrutture_sono_elementarmente_equivalenti_su_un_insieme_sse_lo_sono_su_ogni_sottoinsieme_finito}
\subfile{subfiles/20250212112324-estensione_di_un_linguaggio_del_prim_ordine}
\subfile{subfiles/20250212112537-caratterizzazione_di_teoria_completa}
\subfile{subfiles/20250212113245-criterio_di_tarski_vaught}
\subfile{subfiles/20250212115524-teorema_di_lowenheim_skolem_all_ingiu}
\subfile{subfiles/20250212122610-caratterizzazione_sottostruttura_elementare}
\subfile{subfiles/20250212144403-formula_consistente}
\subfile{subfiles/20250212144849-cardinalita_dell_insieme_delle_formule_di_un_linguaggio_del_prim_ordine}
\subfile{subfiles/20250212164424-tipo_teoria_dei_modelli}
\subfile{subfiles/20250212165544-tipo_finitamente_consistente_e_consistente}
\subfile{subfiles/20250212165712-teorema_di_compattezza}
\subfile{subfiles/20250212171043-teorema_di_compattezza_per_tipi}
\subfile{subfiles/20250212172708-teorema_di_lowenheim_skolem_all_insu}
\subfile{subfiles/20250212185030-immersione_elementare_induce_isomorfismo}
\subfile{subfiles/20250213104706-lemma_di_estensione_di_un_isomorfismo_parziale_tra_ordini}
\subfile{subfiles/20250213105339-funzione_parziale}
\subfile{subfiles/20250213111504-teoria_lambda_categorica}
\subfile{subfiles/20250213111842-teoria_degli_ordini_lineari_densi_senza_punto_finale_e_omega_categorica}
\subfile{subfiles/20250213123032-teoria_dei_grafi}
\subfile{subfiles/20250213140253-esiste_un_grafo_aleatorio}
\subfile{subfiles/20250213140655-teoria_dei_grafi_aleatori_e_omega_categorica}
\subfile{subfiles/20250213142026-categorie_di_modelli_e_morfismi_parziali}
\subfile{subfiles/20250213151058-morfismi_che_preservano_la_verita}
\subfile{subfiles/20250213151902-modello_lambda_ricco}
\subfile{subfiles/20250213151951-modello_lambda_universale}
\subfile{subfiles/20250213152011-modello_lambda_omogeneo}
\subfile{subfiles/20250213152036-modello_lambda_ultraomogeneo}
\subfile{subfiles/20250213152410-modello_e_ricco_sse_omogeneo_e_universale}
\subfile{subfiles/20250213152449-morfismi_tra_modelli_lambda_ricchi_sono_elementari}
\subfile{subfiles/20250213153736-componente_connessa_di_una_categoria_di_modelli_e_morfismi_parziali}
\subfile{subfiles/20250213162407-caratterizzazione_di_modello_lambda_ricco}
\subfile{subfiles/20250214101844-lemma_di_estensione_di_morfismi_tra_modelli_ricchi}
\subfile{subfiles/20250214102108-unione_di_morfismi_della_categorie_categorie_dei_modelli_e_dei_morfismi_parziali}
\subfile{subfiles/20250214120959-mappe_tra_strutture_del_prim_ordine}
\subfile{subfiles/20250214165749-teoria_con_eliminazione_dei_quantificatori}
\subfile{subfiles/20250215122759-export}
\subfile{subfiles/20250215141024-funzioni_primitive_ricordive}
\subfile{subfiles/20250215151413-biiezione_canonica_tra_n_e_n2}
\subfile{subfiles/20250215151440-operatore_di_minimizzazione_non_limitato}
\subfile{subfiles/20250215151458-funzioni_ricorsive}
\subfile{subfiles/20250215151703-inversa_di_una_funzione_totale_iniettiva_e_ricorsiva_e_ricorsiva}
\subfile{subfiles/20250215151720-tesi_di_church}
\subfile{subfiles/20250215160218-funzione_caratteristica}
\subfile{subfiles/20250215171731-quoziente_e_resto_sono_funzioni_ricorsive_primitive}
\subfile{subfiles/20250215174227-mcd}
\subfile{subfiles/20250215174234-mcm}
\subfile{subfiles/20250216162850-funzioni_ricorsive_in_piu_dimensioni}
\subfile{subfiles/20250216162903-funzioni_primitive_ricorsive_in_piu_dimensioni}
\subfile{subfiles/20250216173925-insieme_ricorsivo}
\subfile{subfiles/20250216174510-insieme_ricorsivo_primitivo}
\subfile{subfiles/20250224143326-machine_learning}
\subfile{subfiles/20250224143434-geometria_superiore}
\subfile{subfiles/20250226123416-sistemi_dinamici_e_teoria_del_caos}
\subfile{subfiles/20250226123520-teoria_descrittiva_degli_insiemi}
\subfile{subfiles/20250227105439-teoria_delle_categorie}
\subfile{subfiles/20250301192908-spazio_topologico_separabile}
\subfile{subfiles/20250301193045-sottoinsieme_denso}
\subfile{subfiles/20250301193254-spazio_topologico_primo_numerabile}
\subfile{subfiles/20250301193341-sistema_fondamentale_di_intorni}
\subfile{subfiles/20250301193401-spazio_topologico_metrizzabile}
\subfile{subfiles/20250301193511-spazio_metrico}
\subfile{subfiles/20250301193530-topologia_indotta_da_una_distanza}
\subfile{subfiles/20250301193925-distanze_equivalenti}
\subfile{subfiles/20250301193939-distanze_equivalenti_inducono_la_stessa_topologia}
\subfile{subfiles/20250301194013-spazio_polacco}
\subfile{subfiles/20250301194153-spazio_metrico_completo}
\subfile{subfiles/20250303114357-teorema_di_completamento_di_spazi_metrici}
\subfile{subfiles/20250303115144-isomorfismo_tra_spazi_metrici}
\subfile{subfiles/20250303115158-isometria}
\subfile{subfiles/20250303115241-proprieta_di_chiusura_degli_spazi_polacchi}
\subfile{subfiles/20250303120747-caratterizzazione_dei_chiusi_in_termini_di_successioni}
\subfile{subfiles/20250303121451-caratterizzazione_della_chiusura_in_termini_di_successioni}
\subfile{subfiles/20250303121635-punto_di_accumulazione_di_una_successione}
\subfile{subfiles/20250303134529-successione_di_cauchy}
\subfile{subfiles/20250304141512-proprieta_vere_definitivamente}
\subfile{subfiles/20250304142114-funzione_continua_e_continua_per_successioni}
\subfile{subfiles/20250304151924-gruppo_polacco}
\subfile{subfiles/20250304152026-sottoinsiemi_gdelta_e_fsigma}
\subfile{subfiles/20250304161634-teorema_di_permanenza_del_segno}
\subfile{subfiles/20250304162602-unicita_del_limite}
\subfile{subfiles/20250306115949-disuguaglianza_triangolare}
\subfile{subfiles/20250306134632-caratterizzazione_dei_sottoinsiemi_polacchi_di_uno_spazio_polacco}
\subfile{subfiles/20250306135124-intersezione_di_gdelta_densi_e_densa_in_un_polacco}
\subfile{subfiles/20250306135302-oscillazione_di_una_funzione_in_uno_spazio_metrico}
\subfile{subfiles/20250306140014-funzione_continua_in_un_punto}
\subfile{subfiles/20250306141326-estensione_di_una_funzione_ad_un_dominio_gdelta}
\subfile{subfiles/20250306142322-chiuso_in_uno_spazio_metrizzabile_e_gdelta}
\subfile{subfiles/20250306161159-disugliaglianze_per_il_diametro_di_un_insieme}
\subfile{subfiles/20250310110857-estensione_di_una_funzione_continua_da_un_sottoinsieme_denso}
\subfile{subfiles/20250310111151-funzione_identita}
\subfile{subfiles/20250310112816-funzione_continua_per_successioni}
\subfile{subfiles/20250310113111-spazio_topologico_sequenziale}
\subfile{subfiles/20250310121944-esempi_di_spazi_polacchi}
\subfile{subfiles/20250310122026-cubo_di_hilbert}
\subfile{subfiles/20250313193059-proprieta_di_base_di_un_ultrametrica}
\subfile{subfiles/20250313194245-sottospazi_di_spazi_polacchi_fsigma_densi_e_codensi_non_sono_gdelta}
\subfile{subfiles/20250313194836-spazio_di_baire}
\subfile{subfiles/20250315151142-forza_elastica}
\subfile{subfiles/20250315151443-forza}
\subfile{subfiles/20250315151458-proporzionalita_diretta}
\subfile{subfiles/20250315153915-equilibrio_cinematico}
\subfile{subfiles/20250317093153-insieme_aperto_sse_intorno_di_ogni_suo_punto}
\subfile{subfiles/20250317100425-complementare_di_un_insieme}
\subfile{subfiles/20250317125810-successioni_in_spazio_metrico_hanno_lo_stesso_limite_sse_limite_delle_distanze_e_nullo}
\subfile{subfiles/20250317130124-spazi_metrici_sono_t2}
\subfile{subfiles/20250317165247-topologia_discreta}
\subfile{subfiles/20250318161815-ultrametrica}
\subfile{subfiles/20250318170914-insieme_clopen}
\subfile{subfiles/20250320150051-gruppo_archimedeo}
\subfile{subfiles/20250320184931-gruppo_ordinato}
\subfile{subfiles/20250324165349-prefascio}
\subfile{subfiles/20250324165859-gruppo_banale}
\subfile{subfiles/20250324170250-categoria_degli_aperti_di_uno_spazio_topologico}
\subfile{subfiles/20250324174714-oggetto_terminale_di_una_categoria}
\subfile{subfiles/20250324174728-fascio}
\subfile{subfiles/20250324175845-esempi_di_prefasci}
\subfile{subfiles/20250324175851-esempi_di_fasci}
\subfile{subfiles/20250325150647-sottoprefascio}
\subfile{subfiles/20250325153824-funzione_localmente_costante}
\subfile{subfiles/20250325154046-funzione_localmente_costante_sse_costante_sulle_componenti_connesse}
\subfile{subfiles/20250325160105-funzione_costante}
\subfile{subfiles/20250325160128-componente_connessa_di_uno_spazio_topologico}
\subfile{subfiles/20250325171002-fascio_di_gruppo_localmente_costante}
\subfile{subfiles/20250325171249-fascio_grattacielo}
\subfile{subfiles/20250325180613-morfismo_di_prefasci}
\subfile{subfiles/20250325183418-esempi_di_morfismi_di_prefasci}
\subfile{subfiles/20250325183434-spiga_di_un_prefascio}
\subfile{subfiles/20250325192239-fascio_come_funtore}
\subfile{subfiles/20250325195205-caratterizzazione_morfismo_di_fasci}
\subfile{subfiles/20250327104701-immersione_topologica_dello_spazio_di_baire_nello_spazio_di_cantor}
\subfile{subfiles/20250327104743-caratterizzazione_dei_compatti_dello_spazio_di_baire}
\subfile{subfiles/20250327104804-insieme_limitato_dello_spazio_di_baire}
\subfile{subfiles/20250327105051-caratterizzazione_dei_sigma_compatti_nello_spazio_di_baire}
\subfile{subfiles/20250327105207-spazio_di_baire_si_surietta_in_ogni_spazio_polacco_non_vuoto_tramite_una_mappa_aperta_e_continua}
\subfile{subfiles/20250327114727-esempi_di_spighe}
\subfile{subfiles/20250327114817-morfismo_di_fasci_induce_omomorfismo_tra_spighe}
\subfile{subfiles/20250327114851-fascio_associato_ad_un_prefascio}
\subfile{subfiles/20250327114913-fascio_associato_ad_un_sottoprefascio}
\subfile{subfiles/20250327114922-fascio_nucleo}
\subfile{subfiles/20250327114937-fascio_immagine}
\subfile{subfiles/20250327115206-morfismo_di_fasci_iniettivo}
\subfile{subfiles/20250327115214-morfismo_di_fasci_suriettivo}
\subfile{subfiles/20250327122142-successione_di_una_categoria}
\subfile{subfiles/20250327131547-diametro_di_un_insieme}
\subfile{subfiles/20250327132216-ogni_aperto_di_uno_spazio_metrico_ammette_un_ricoprimento_numerabile_di_diametro_arbitrariamente_piccolo}
\subfile{subfiles/20250327150404-successione_di_fasci_esatta}
\subfile{subfiles/20250327150529-caratterizzazione_di_successione_esatta_di_fasci_tramite_sezioni}
\subfile{subfiles/20250328151135-spazio_di_lindelof}
\subfile{subfiles/20250328151247-sottoricoprimento}
\subfile{subfiles/20250328151316-spazio_topologico_secondo_numerabile_implica_lindelof}
\subfile{subfiles/20250328154313-ogni_base_di_uno_spazio_topologico_secondo_numerabile_ammette_una_sottobase_numerabile}
\subfile{subfiles/20250328162603-in_spazio_metrico_ogni_aperto_e_fsigma}
\subfile{subfiles/20250331095811-spazio_topologico_secondo_numerabile_implica_separabile}
\subfile{subfiles/20250331100937-sistema_fondamentale_di_intorni_in_uno_spazio_metrico}
\subfile{subfiles/20250331122739-immersione_topologica}
\subfile{subfiles/20250331174140-compatto_in_un_haussdorf_e_chiuso}
\subfile{subfiles/20250401124050-teorema_di_tichonov}
\subfile{subfiles/20250401125136-chiuso_in_un_compatto_e_compatto}
\subfile{subfiles/20250402170018-spazio_di_baire_non_e_compatto}
\subfile{subfiles/20250403093420-rango_di_cantor_bendixson}
\subfile{subfiles/20250403131856-punto_isolato}
\subfile{subfiles/20250417125850-caratterizzazione_insieme_mai_denso}
\subfile{subfiles/20250417180515-insieme_mai_denso}
\subfile{subfiles/20250419121342-proprieta_della_gerarchia_di_borel}
\subfile{subfiles/20250419121450-gerarchia_di_borel}
\subfile{subfiles/20250419122215-proprieta_insiemi_magri_comagri_non_magri}
\subfile{subfiles/20250419122543-classi_ambigue_di_un_sottospazio_polacco_nella_gerarchia_di_borel}
\subfile{subfiles/20250419122722-classe_di_borel_dell_insieme_dei_punti_di_derivabilita_di_una_funzione_reale}
\subfile{subfiles/20250419122752-insieme_magro}
\subfile{subfiles/20250505103058-caratterizzazione_dei_punti_non_isolati_di_uno_spazio_polacco}
\subfile{subfiles/20250505103212-caratterizzazione_dei_sottoinsiemi_chiusi_ma_non_aperti_di_un_polacco}
\subfile{subfiles/20250505103416-esempi_di_sottoinsimi_pi03_completi}
\subfile{subfiles/20250505103631-esempi_di_sottoinsiemi_sigma02_completi}
\subfile{subfiles/20250505103829-esempi_di_sottoinsiemi_analitici}
\subfile{subfiles/20250513111844-gioco_di_banach_mazur}
\subfile{subfiles/20250513155732-logic_game}
\subfile{subfiles/20250513171520-giochi_di_gale_stewart}
\subfile{subfiles/20250514142154-albero_teoria_descrittiva_degli_insiemi}
\subfile{subfiles/20250514142208-albero_potato}
\subfile{subfiles/20250514142251-corpo_di_un_albero}
\subfile{subfiles/20250514142938-posizioni_ammissibili_in_un_gioco_logico}
\subfile{subfiles/20250514143441-giochi_logici_equivalenti}
\subfile{subfiles/20250514144736-teorema_di_gale_stewart}
\subfile{subfiles/20250514154039-proprieta_di_baire}
\subfile{subfiles/20250514154101-spazio_topologico_di_baire}
\subfile{subfiles/20250514174255-gioco_di_choquet}
\subfile{subfiles/20250514174717-teorema_di_caratterizzazione_dei_comagri_tramite_il_gioco_di_banach_mazur}
\subfile{subfiles/20250514174859-insiemi_analitici_di_un_polacco_hanno_bp}
\subfile{subfiles/20250514175252-magrezza_dentro_ad_un_polacco_tramite_gioco_di_banach_mazur}
\subfile{subfiles/20250515141706-da_finire}
\subfile{subfiles/20250519112500-proprieta_di_chiusura_delle_funzioni_primitive_ricorsive}
\subfile{subfiles/20250519112917-proprieta_di_chiusura_degli_insiemi_ricorsivi}
\subfile{subfiles/20250520101337-funzioni_ricorsive_definite_per_casi}
\subfile{subfiles/20250520101418-funzioni_ricorsive_per_minimizzazione_su_un_predicato}
\subfile{subfiles/20250520105413-algoritmo_di_tarski_kuratowski}
\subfile{subfiles/20250520110008-insiemi_disgiunti}
\subfile{subfiles/20250520113238-insieme_semiricorsivo}
\subfile{subfiles/20250520113316-proprieta_di_chiusura_degli_insiemi_semiricorsivi}
\subfile{subfiles/20250520113349-teorema_di_post}
\subfile{subfiles/20250520113608-insieme_semiricorsivo_come_range_di_funzioni_ricorsive}
\subfile{subfiles/20250520143216-insieme_ricorsivo_come_range_di_funzione_ricorsiva_totale_crescente}
\subfile{subfiles/20250521105415-separazione_tramite_boreliani_di_insiemi_invarianti_per_una_relazione_di_equivalenza}
\subfile{subfiles/20250521105548-insieme_parzialmente_trasversale_per_una_relazione_di_equivalenza}
\subfile{subfiles/20250521105603-insieme_trasversale_per_una_relazione_di_equivalenza}
\subfile{subfiles/20250521105653-selettore_per_una_relazione_di_equivalenza}
\subfile{subfiles/20250521110737-proprieta_insieme_parzialmente_trasversale_per_una_relazione_di_equivalenza_in_uno_spazio_polacco}
\subfile{subfiles/20250521110811-proprieta_insieme_trasversale_e_selettore_per_una_relazione_di_equivalenza_in_uno_spazio_polacco}
\subfile{subfiles/20250521110928-coanalitici_sono_unione_di_omega1_boreliani}
\subfile{subfiles/20250522113216-ogni_insieme_analitico_non_numerabile_ammette_un_sottoinsieme_boreliano_non_numerabile}
\subfile{subfiles/20250525113346-base_debole_di_uno_spazio_topologico}
\subfile{subfiles/20250525220742-insieme_analitico}
\subfile{subfiles/20250526100313-sigma_algebra}
\subfile{subfiles/20250526100910-caratterizzazione_bp_tramite_gioco_di_banach_mazur}
\subfile{subfiles/20250531104048-codifica_di_un_insieme_numerabile}
\subfile{subfiles/20250531105333-insiemi_ricorsivi_tramite_codifica}
\subfile{subfiles/20250531110714-teorema_cinese_dei_resti}
\subfile{subfiles/20250531110725-funzione_beta_di_godel}
\subfile{subfiles/20250531110737-codifica_delle_sequenze_finite_tramite_beta_di_godel}
\subfile{subfiles/20250531114007-numeri_naturali_coprimi}
\subfile{subfiles/20250531114133-classe_di_resto}
\subfile{subfiles/20250601160026-funzione_memoria}
\subfile{subfiles/20250601161456-generalizzazione_schema_di_ricorsione}
\subfile{subfiles/20250601162102-teorema_di_forma_normale_di_kleene}
\subfile{subfiles/20250601162421-funzioni_ricorsive_e_loro_grafico}
\subfile{subfiles/20250601165443-funzioni_ricorsive_e_loro_dominio}
\subfile{subfiles/20250601171055-insieme_semiricorsivo_come_monio_di_funzione_ricorsiva_parziale}
\subfile{subfiles/20250601171113-funzione_ricorsiva_parziale_con_dominio_ricorsivo_e_restrizione_di_funzione_ricorsiva_totale}
\subfile{subfiles/20250603170350-gerarchia_aritmetica}
\subfile{subfiles/20250603170559-complessita_di_una_formula_del_modello_standard}
\subfile{subfiles/20250603170634-insieme_definibile_nel_modello_standard}
\subfile{subfiles/20250603171922-funzioni_ricorsive_vs_definibili}
\subfile{subfiles/20250603173727-algebra_di_boole}
\subfile{subfiles/20250605140804-relazione_tra_mcd_e_mcm}
\subfile{subfiles/20250606095019-modello_standard_dell_artimetica}
\subfile{subfiles/20250608093535-aritmetica_di_peano_del_second_ordine}
\subfile{subfiles/20250608093604-aritmetica}
\subfile{subfiles/20250608094213-insieme_rappresentato_da_una_formula}
\subfile{subfiles/20250608094553-aritmetica_di_robinson_rappresenta_funzioni_ricorsive_totali_e_predicati_ricorsivi}
\subfile{subfiles/20250608105738-logica_del_second_ordine}
\subfile{subfiles/20250608165649-descrizione_modelli_dell_aritmetica_di_robinson}
\subfile{subfiles/20250609104647-buona_codifica_di_un_linguaggio}
\subfile{subfiles/20250609111619-sottotermine}
\subfile{subfiles/20250609113022-sottoformula_del_prim_ordine}
\subfile{subfiles/20250609125154-sostituzione_di_termini_in_un_termine}
\subfile{subfiles/20250609135250-teoria_ricorsivamente_assiomatizzabile}
\subfile{subfiles/20250609135524-codifica_delle_dimostrazioni_a_partire_dagli_assiomi}
\subfile{subfiles/20250609162617-primo_teorema_di_incompletezza_di_godel}
\subfile{subfiles/20250609162657-teoria_coerente}
\subfile{subfiles/20250609162711-teoria_omega_coerente}
\subfile{subfiles/20250610134232-teoria_decidibile}
\subfile{subfiles/20250610135033-teoria_completa_e_ricorsivamente_assiomatizzabile_e_decidibile}
\subfile{subfiles/20250610135104-estensione_finita_di_una_teoria_decidibile_e_decidibile}
\subfile{subfiles/20250610135350-aritmetica_di_robinson_e_essenzialmente_indecidibile}
\subfile{subfiles/20250610135416-teoria_decidibile_e_coerente_ha_estesione_decidibile_coerente_e_completa}
\subfile{subfiles/20250610145135-teorema_dell_indefinibilita_della_verita}
\subfile{subfiles/20250610145208-teorema_di_church}
\subfile{subfiles/20250611111403-analisi_non_standard}
\subfile{subfiles/20250611115646-campo_ordinato}
\subfile{subfiles/20250611135127-funzione_uniformemente_continua}
\subfile{subfiles/20250612110627-chiusura_logica_di_una_teoria}
\subfile{subfiles/20250612143636-notazione_teoria_dei_modelli}
\subfile{subfiles/20250612151505-aritmetica_dei_cardinali}
\subfile{subfiles/20250613093648-formule_risolutive_equazioni_polinomiali}
\subfile{subfiles/20250613121118-mappe_parziali_tra_strutture_del_prim_ordine}
\subfile{subfiles/20250616135710-teoria_dei_gruppi_abeliani}
\subfile{subfiles/20250616140010-lemma_di_estensione_di_un_isomorfismo_parziale_tra_gruppi_abeliani}
\subfile{subfiles/20250616140201-teoria_dei_gruppi_abeliani_privi_di_torsione}
\subfile{subfiles/20250616140755-teoria_dei_gruppi_abeliani_divisibili}
\subfile{subfiles/20250616140840-lemma_di_estensione_di_un_isomorfismo_parziale_tra_gruppi_abeliani_divisibili}
\subfile{subfiles/20250616141052-teoria_dei_gruppi_abeliani_privi_di_torsione_e_categorica}
\subfile{subfiles/20250616141222-modelli_lambda_ricchi_nella_categoria_dei_modelli_della_teoria_tfag}
\subfile{subfiles/20250616141738-teoria_dei_domini_di_integrita}
\subfile{subfiles/20250616152011-teoria_dei_campi}
\subfile{subfiles/20250617093912-soddisfazione_di_un_tipo_e_mappa_elementare}
\subfile{subfiles/20250617095548-modello_lambda_saturo}
\subfile{subfiles/20250617102642-esistenza_di_modelli_saturi_di_cardinalita_fissata}
\subfile{subfiles/20250617102704-modello_lambda_saturo_sse_lambda_ricco}
\subfile{subfiles/20250617102733-modello_mostro}
\subfile{subfiles/20250617103021-insieme_definibil_e_automorfismi_in_un_modello_mostro}
\subfile{subfiles/20250617104602-catena_elementare_di_modelli}
\subfile{subfiles/20250617113538-topologia_zero_dimensionale}
\subfile{subfiles/20250618095344-elementi_algebrici_e_definibili_teoria_dei_modelli}
\subfile{subfiles/20250618101423-caratterizzazione_chiusura_algebrica_in_un_modello_mostro}
\subfile{subfiles/20250618102057-caratterizzazione_chiusura_definibile_in_un_modello_mostro}
\subfile{subfiles/20250618103257-gruppo_degli_automorfismi_che_fissano_un_sottoinsieme_e_tipo_di_un_elemento}
\subfile{subfiles/20250618110012-automorfismo_e_chiusura_algebrica_in_un_modello_mostro}
\subfile{subfiles/20250618153446-struttura_minimale}
\subfile{subfiles/20250618153755-teoria_fortemente_minimale}
\subfile{subfiles/20250618155509-principio_dello_scambio_per_chiusura_algebrica_in_un_modello_mostro}
\subfile{subfiles/20250618155810-base_di_un_insieme_dentro_un_modello_mostro}
\subfile{subfiles/20250618160252-teorema_della_base_dentro_un_modello_mostro}
\subfile{subfiles/20250618160625-teoria_fortemente_minimale_e_lambda_categorica}
\subfile{subfiles/20250619101109-classi_equipotenti}
\subfile{subfiles/20250619161501-caratteristiche_delle_relazioni_binarie}
\subfile{subfiles/20250619163724-preordine}
\subfile{subfiles/20250620092953-order_topology}
\subfile{subfiles/20250620163542-complessita_di_una_formula_nel_linguaggio_della_teoria_degli_insiemi}
\subfile{subfiles/20250620163603-complessita_di_una_formula}
\subfile{subfiles/20250620172440-assolutezza_delle_formule_tra_un_insieme_transitivo_e_un_modello_di_mk}
\subfile{subfiles/20250620173236-insieme_transitivo_e_assiomi_di_zf}
\subfile{subfiles/20250621113243-cardinalita_dei_reali}
\subfile{subfiles/20250621133056-teichmuller_tukey_lemma}
\subfile{subfiles/20250621133123-axiom_of_multiple_choices}
\subfile{subfiles/20250621133254-kurepa_s_maximality_principle}
\subfile{subfiles/20250622095126-classe_di_equivalenza_di_scott}
\subfile{subfiles/20250624154311-metodi_matematici_per_il_machine_learning}
\subfile{subfiles/20250624155858-neurone_artificiale}
\subfile{subfiles/20250624161413-funzione_di_heaviside}
\subfile{subfiles/20250624162220-spazi_lp}
\subfile{subfiles/20250624171244-gradiente_di_una_funzione}
\subfile{subfiles/20250625095723-prodotto_scalare}
\subfile{subfiles/20250625100117-distribuzione_analisi_matematica}
\subfile{subfiles/20250625100133-delta_di_dirac}
\subfile{subfiles/20250625104200-misura_di_baire}
\subfile{subfiles/20250625104334-funzione_discriminatoria_per_una_misur}
\subfile{subfiles/20250625105528-funzione_discriminatoria_per_una_misura_di_baire_sul_cubo_unitario}
\subfile{subfiles/20250625110016-misura_finita}
\subfile{subfiles/20250625110024-misura_con_segno}
\subfile{subfiles/20250625110032-misura_regolare}
\subfile{subfiles/20250625110110-funzione_sigmoidale}
\subfile{subfiles/20250625110412-limite_analisi_matematica}
\subfile{subfiles/20250625110457-funzioni_sigmoidali_sono_discriminatorie_per_le_misure_di_baire_sul_cubo_unitario}
\subfile{subfiles/20250625123506-spazio_normato}
\subfile{subfiles/20250627103519-overfitting}
\subfile{subfiles/20250627104011-moltiplicatore_di_lagrange}
\subfile{subfiles/20250627104832-p_norma_in_rn}
\subfile{subfiles/20250627110009-training_error_and_test_error}
\subfile{subfiles/20250627130736-gradiente_e_perpendicolare_alle_curve_di_livello}
\subfile{subfiles/20250627130923-esistenza_di_una_curva_perpendicolare_a_tutte_le_curve_di_livello}
\subfile{subfiles/20250627131207-curva_di_livello}
\subfile{subfiles/20250627153141-esistenza_di_un_minimo}
\subfile{subfiles/20250627153319-teorema_di_weierstrass}
\subfile{subfiles/20250627153543-massimo_e_minimo_di_una_funzione_reale}
\subfile{subfiles/20250627153729-condizioni_necessarie_per_l_esistenza_di_un_minimo_di_una_funzione_reale}
\subfile{subfiles/20250627155347-teorema_di_fermat_sui_punti_stazionari}
\subfile{subfiles/20250627155431-funzione_derivabile}
\subfile{subfiles/20250627161722-hessiana_e_punti_stazionari_di_una_funzione_reale}
\subfile{subfiles/20250627161731-matrice_hessiana}
\subfile{subfiles/20250627184228-funzioni_iperboliche}
\subfile{subfiles/20250627184319-funzioni_trigonometriche}
\subfile{subfiles/20250629105513-teoremi_di_dini_per_la_convergenza_uniforme}
\subfile{subfiles/20250629105745-convergenza_uniforme}
\subfile{subfiles/20250629105815-successione_di_funzioni}
\subfile{subfiles/20250629110306-funzioni_uniformemente_limitate}
\subfile{subfiles/20250629112810-disuguaglianza_di_cauchy_schwarz}
\subfile{subfiles/20250629113211-famiglia_di_funzioni_equicontinua}
\subfile{subfiles/20250629120441-teorema_di_ascoli_arzela}
\subfile{subfiles/20250629143200-teorema_di_lagrange}
\subfile{subfiles/20250629151233-algebra_di_funzioni_reali}
\subfile{subfiles/20250629151420-algebra_di_funzioni_separa_i_punti}
\subfile{subfiles/20250629165421-teorema_di_stone_weierstrass}
\subfile{subfiles/20250629165520-algebra_di_funzioni_reali}
\subfile{subfiles/20250630100737-polinomi_sono_densi_nelle_funzioni_continue}
\subfile{subfiles/20250630103653-rete_neurale_che_approssima_funzioni_continue_periodiche}
\subfile{subfiles/20250630103725-funzioni_approssimate_da_una_rete_neurale}
\subfile{subfiles/20250630103918-funzione_periodica}
\subfile{subfiles/20250630121612-wiener_s_tauberian_theorems}
\subfile{subfiles/20250630121906-trasformata_di_fourier}
\subfile{subfiles/20250630122400-span}
\subfile{subfiles/20250630122745-proprieta_vera_quasi_ovunque}
\subfile{subfiles/20250630122824-misura_di_lebesgue}
\subfile{subfiles/20250630154418-one_hidden_layer_perceptron_network_impara_funzioni_continue}
\subfile{subfiles/20250630155015-ogni_funzione_semplice_e_combinazione_lineare_di_heaviside}
\subfile{subfiles/20250630155113-derivata_di_funzione_semplice_con_delta_di_dirac}
\subfile{subfiles/20250630155139-misura_di_dirac_approssimata_da_misure_a_campana}
\subfile{subfiles/20250630155208-funzione_reale_continua_su_un_compatto_e_uniformemente_continua}
\subfile{subfiles/20250630171950-funzione_semplice}
\subfile{subfiles/20250701080039-derivata_distribuzioni}
\subfile{subfiles/20250701115005-supporto_di_una_funzione}
\subfile{subfiles/20250701121640-teorema_di_heine_borel}
\subfile{subfiles/20250701121756-insieme_limitato}
\subfile{subfiles/20250701140551-funzione_lineare_a_tratti}
\subfile{subfiles/20250701140621-funzione_costante_a_tratti}
\subfile{subfiles/20250702101346-punto_critico_di_una_funzione_reale}
\subfile{subfiles/20250702101532-funzione_reale_differenziabile}
\subfile{subfiles/20250702102107-punto_di_sella}
\subfile{subfiles/20250702102213-matrice_definita_positiva}
\subfile{subfiles/20250702104312-metodo_del_gradient_descent}
\subfile{subfiles/20250702114407-chain_rule}
\subfile{subfiles/20250702114642-derivata_direzionale}
\subfile{subfiles/20250703105424-prodotto_di_convoluzione}
\subfile{subfiles/20250703105508-funzioni_integrabili_secondo_lebesgue}
\subfile{subfiles/20250703142106-densita_di_probabilita}
\subfile{subfiles/20250704104537-spazio_delle_funzioni_misurabili_come_spazio_metrico}
\subfile{subfiles/20250704104938-algebra_di_borel}
\subfile{subfiles/20250704104947-funzione_misurabile}
\subfile{subfiles/20250704105515-misura_di_probabilita}
\subfile{subfiles/20250704145211-teorema_di_rappresentazione_di_riesz}
\subfile{subfiles/20250704145425-misura_di_radon}
\subfile{subfiles/20250704145518-funzione_limitata}
\subfile{subfiles/20250704152055-teorema_di_hahn_banach}
\subfile{subfiles/20250704154416-funzione_l2_nulla_se_integrale_nullo_su_tutti_i_semispazi}
\subfile{subfiles/20250704163455-misura}
\subfile{subfiles/20250704170145-esistenza_di_una_funzione_continua_approssimabile_da_una_rete_neurale}
\subfile{subfiles/20250704170145-one_hidden_layer_relu_network_impara_funzioni_continue_unidimensionali}
\subfile{subfiles/20250704170145-one_hidden_layer_softplus_network_impara_funzioni_continue_unidimensionali}
\subfile{subfiles/20250704170145-rete_neurale_che_approssima_funzioni_continue_periodiche}
\subfile{subfiles/20250706105700-caratterizzazione_convergenza_in_misura}
\subfile{subfiles/20250706110009-uniforme_convergenza_sui_compatti_implica_convergenza_in_misura}
\subfile{subfiles/20250706110508-numero_di_neuroni_di_one_hidden_layer_network_e_accuratezza_dell_approssimazione}
\subfile{subfiles/20250706110652-funzioni_continue_a_supporto_compatto_su_rn_sono_dense_in_lq}
\subfile{subfiles/20250706110741-funzioni_continue_e_lineari_a_tratti_sono_dense_nelle_funzioni_continue_a_supporto_compatto_su_rn}
\subfile{subfiles/20250706121659-convergenza_uniforme_sui_compatti}
\subfile{subfiles/20250706121659-funzioni_continue_approssimate_lineari_a_tratti}
\subfile{subfiles/20250706121659-ohdspn_fun_cont}
\subfile{subfiles/20250706121659-ohldn_f_continue}
\subfile{subfiles/20250706121659-ohldnl1}
\subfile{subfiles/20250706121659-ohldnl2}
\subfile{subfiles/20250706121659-ohlsnfmc}
\subfile{subfiles/20250706121659-one_hidden_layer_sigmoidal_network}
\subfile{subfiles/20250706121659-rnaqi}
\subfile{subfiles/20250708121822-one_hidden_layer_relu_network_impara_esattamente_funzioni_a_supporto_finito}
\subfile{subfiles/20250708122516-rete_neurale_relu_feedforward_impara_esattamente_la_funzione_massimo}
\subfile{subfiles/20250708122736-exact_learning_machine_learning}
\subfile{subfiles/20250708165332-archivio_tikz}
\subfile{subfiles/20250709105733-toc_laurea_magistrale_in_matematica}
\subfile{subfiles/20250709134236-piu_profondita_in_favore_di_meno_larghezza_rete_neurale_relu_feedforward}
\subfile{subfiles/20250709134434-rete_neurale_relu_feedforward_ha_funzione_di_input_output_continua_e_lineare_a_tratti}
\subfile{subfiles/20250709134700-rete_neurale_relu_feedforward_rappresenta_esattamente_funzioni_continue_lineari_a_tratti}
\subfile{subfiles/20250709135034-teorema_di_rappresentazione_di_kolmogorov_arnold}
\subfile{subfiles/20250709150055-funzione_analitica}
\subfile{subfiles/20250709231208-concentrazione}
\subfile{subfiles/20250710102223-misura_di_baire_sul_cubo_nulla_se_nulla_su_tutti_i_semispazi}
\subfile{subfiles/20250710102223-percettrone}
\subfile{subfiles/20250710111143-regressione_lineare}
\subfile{subfiles/20250710111302-neurone_sigmoidale}
\subfile{subfiles/20250710120853-and_logico}
\subfile{subfiles/20250710120858-or_logico}
\subfile{subfiles/20250710120916-xor_logico}
\subfile{subfiles/20250710140734-valore_atteso}
\subfile{subfiles/20250710140836-legge_dei_grandi_numeri}
\subfile{subfiles/20250710141709-mean_square_error}
\subfile{subfiles/20250710150636-bias_variance_tradeoff}
\subfile{subfiles/20250710152408-bias}
\subfile{subfiles/20250710152417-varianza}
\subfile{subfiles/20250711122025-subderivata}
\subfile{subfiles/20250711122823-algoritmo_di_gradient_descent}
\subfile{subfiles/20250711122823-metodo_del_subgradiente}
\subfile{subfiles/20250711125821-teoria_dell_informazione_shannon}
\subfile{subfiles/20250711141953-insieme_convesso}
\subfile{subfiles/20250711142403-funzione_convessa}
\subfile{subfiles/20250711174144-attesa_condizionata}
\subfile{subfiles/20250711175559-spazio_di_probabilita}
\subfile{subfiles/20250711175937-variabile_aleatoria}
\subfile{subfiles/20250712102732-eventi_indipendenti}
\subfile{subfiles/20250712103837-distrubuzione_di_una_variabile_aleatoria}
\subfile{subfiles/20250713214832-neurone_ad_input_continuo}
\subfile{subfiles/20250713215224-backpropagation_per_una_rete_neurale}
\subfile{subfiles/20250713221021-pooling}
\subfile{subfiles/20250713221345-rete_di_convoluzione}
\subfile{subfiles/20250714105222-rete_neurale_che_implementa_lo_xor}
\subfile{subfiles/20250714154153-sigma_algebra_come_campo_di_informazione}
\subfile{subfiles/20250714154501-sigma_algebra_generata_da_una_variabile_aleatoria}
\subfile{subfiles/20250714161943-leggi_di_de_morgan}
\subfile{subfiles/20250714162717-approssimazione_per_regressione_lineare_machine_learning}
\subfile{subfiles/20250714162717-sigma_algebra_generata_dal_massimo_di_variabili_aleatorie}
\subfile{subfiles/20250716183001-curva}
\subfile{subfiles/20250716183127-vettori_perpendicolari}
\subfile{subfiles/20250716183256-metodo_dell_hessiana}
\subfile{subfiles/20250716185823-metodo_di_newton}
\subfile{subfiles/20250717132708-serie_di_taylor}
\subfile{subfiles/20250721232924-studiare}
\subfile{subfiles/20250821170414-relazione_tra_reali_e_omega1}
\subfile{subfiles/20250901104138-teoria_dei_modelli_corso_sel_2025}
\subfile{subfiles/20250901104229-scuola_estiva_di_logica_corso}
\subfile{subfiles/20250901104256-teoria_della_ricorsivita_corso_sel2025}
\subfile{subfiles/20250901104317-reverse_mathematics_corso_sel2025}
\subfile{subfiles/20250901153512-iperoperazioni}
\subfile{subfiles/20250902132244-lectio_magistralis_corso_sel2025}
\subfile{subfiles/20250904101844-in_lode_della_quasi_verita_logica_di_lukasiewicz_e_algebre_di_boole_perturbate_corso_sel2025}
% \subfile{subfiles/20250922095937-istituzioni_di_algebra_corso}
\subfile{subfiles/20250922171928-teoria_dei_modelli_corso}
\subfile{subfiles/20250924152557-temp}
\subfile{subfiles/20250924174557-modello_lambda_saturo_sse_debolmente_lambda_saturo_e_debolmente_lambda_omogeneo}
\subfile{subfiles/20251001144404-condizione_sufficiente_per_teoria_piccola}
\subfile{subfiles/20251010110031-subacquea_meta}
\subfile{subfiles/20251015172832-bib_a_creche_course_in_model_theory_2024}
\subfile{subfiles/20251020142629-modello_debolmente_lambda_saturo}
\subfile{subfiles/20251020143206-modello_saturo_sse_omogeneo_e_debolmente_saturo}
\subfile{subfiles/20251020145412-modello_debolmente_lambda_omogeneo}
\subfile{subfiles/20251020150308-insieme_invariante_su_un_insieme_in_un_modello_mostro}
\subfile{subfiles/20251020150315-caratterizzazione_insiemi_invarianti_su_un_insieme_in_un_modello_mostro}
\subfile{subfiles/20251020151126-insieme_degli_automorfismi_di_una_struttura_del_prim_ordine}
\subfile{subfiles/20251020153459-lyndon_robinson_lemma}
\subfile{subfiles/20251020160238-caratterizzazione_delle_conseguenze_di_un_delta_tipo_tramite_la_preservazione_per_delta_morfismi}
\subfile{subfiles/20251020161056-teoria_con_delta_eliminazione_positiva_dei_quantificatori}
\subfile{subfiles/20251020164754-estendere_dominio_e_codominio_di_un_delta_morfismo}
\subfile{subfiles/20251020170439-caratterizzazione_delta_morfismi_estendibili_a_delta_morfismi_totali}
\subfile{subfiles/20251020170634-caratterizzazione_delta_morfismi_estendibili_a_delta_morfismi_suriettivi}
\subfile{subfiles/20251020170822-caratterizzazione_della_delta_eliminazione_positiva_dei_quantificatori_tramite_delta_morfismi}
\subfile{subfiles/20251020171034-teoria_delta_model_completa}
\subfile{subfiles/20251020171129-caratterizzazione_teoria_delta_model_completa_tramite_eliminazione_positiva_dei_quantificatori}
\subfile{subfiles/20251020171928-caratterizzazione_delle_formule_preservate_da_delta_morfismi}
\subfile{subfiles/20251021120818-chiusura_di_un_insieme_di_formule_rispetto_a_connettivi_logici}
\subfile{subfiles/20251022093422-relazione_tra_acl_automorfismi_e_shelah_equivalenza_nella_eq_espansione}
\subfile{subfiles/20251024102757-topological_games_lectio}
\subfile{subfiles/20251026160431-cardinalita_degli_insiemi_invarianti}
\subfile{subfiles/20251026160523-insieme_delle_formula_del_prim_ordine}
\subfile{subfiles/20251026161128-formula_invariante_su_un_insieme}
\subfile{subfiles/20251026161720-phi_formula_su_un_insieme}
\subfile{subfiles/20251026161851-phi_tipo_invariante_su_un_insieme}
\subfile{subfiles/20251026195543-caratterizzazione_phi_tipi_invarianti_su_un_insieme}
\subfile{subfiles/20251026195636-tipo_finitamente_soddisfacibile_in_un_insieme}
\subfile{subfiles/20251026195723-caratterizzazione_tipi_finitamente_soddisfacibili_in_un_insieme}
\subfile{subfiles/20251026195746-tipo_finitamente_soddisfacibile_e_invariante}
\subfile{subfiles/20251026195832-estensione_tipo_finitamente_soddisfacibile_in_un_insieme_a_tipo_massimale_e_completo}
\subfile{subfiles/20251026210840-sequenza_di_morley}
\subfile{subfiles/20251026210908-sequenza_di_indiscernibili}
\subfile{subfiles/20251026211005-sequenza_di_morely_inviariante_e_indiscernibile}
\subfile{subfiles/20251026211032-coerede_di_un_tipo}
\subfile{subfiles/20251026211057-sequenza_di_coeredi}
\subfile{subfiles/20251028160440-sequenza_di_indiscernibili_eq_expansion}
\subfile{subfiles/20251028160605-eq_espansione_di_un_modello_mostro}
\subfile{subfiles/20251029151843-caratterizzazione_sequenza_di_coeredi}
\subfile{subfiles/20251029152405-tipo_completo}
\subfile{subfiles/20251029153442-insieme_dei_tipi_completi_e_finitamente_soddisfacibili}
\subfile{subfiles/20251029154447-tuple_indipendenti_rispetto_ad_un_modello_monster_model}
\subfile{subfiles/20251029160457-restrizione_di_un_tipo_ad_un_insieme_di_parametri}
\subfile{subfiles/20251029171516-teorema_di_ramsey}
\subfile{subfiles/20251104131959-lemma_del_diagramma_elementare}
\subfile{subfiles/20251104154216-relazione_erede_coerede_stazionaria}
\subfile{subfiles/20251104154427-estensione_di_un_semigruppo_ad_un_modello_mostro}
\subfile{subfiles/20251104154531-teorema_di_hindman}
\subfile{subfiles/20251104163738-semigruppo}
\subfile{subfiles/20251104164315-teoria_dei_gruppi}
\subfile{subfiles/20251105165655-tuple_elementarmente_equivalenti_su_un_insieme_di_parametri_inducono_mappa_elementare}
\subfile{subfiles/20251106123927-insieme_tipo_definibile_su_un_insieme_e_invariante}
\subfile{subfiles/20251106125420-scambio_di_unioni_intersezioni_e_immagini_retroimmagini_di_funzione}
\subfile{subfiles/20251107125412-tipo_di_ehrenfeucht_mostowski}
\subfile{subfiles/20251107125444-teorema_di_ehrenfeucht_mostowski}
\subfile{subfiles/20251107125536-estensione_di_sequenza_di_indiscernibili_su_un_insieme_ad_un_modello}
\subfile{subfiles/20251107232133-cefis_m1_crediti_formativi}
\subfile{subfiles/20251111114323-latex_creare_flashcard}
\subfile{subfiles/20251111145841-teorema_di_hales_jewett}
\subfile{subfiles/20251111150142-colorazione}
\subfile{subfiles/20251111151425-teorema_di_van_der_waerden}
\subfile{subfiles/20251111151506-progressione_aritmetica}
\subfile{subfiles/20251111151537-teorema_di_szemeredi}
\subfile{subfiles/20251111151641-densita_di_banach}
\subfile{subfiles/20251111165459-stile_scala_di_formalita}
\subfile{subfiles/20251112104601-coordinate_locali_su_una_varieta_sono_cinfinito}
\subfile{subfiles/20251112211429-idee_tesi_laurea_magistrale_bibliografia}
\subfile{subfiles/20251112215318-collegamenti_note_in_uno_zettelkasten}
\subfile{subfiles/20251112215403-zettelkasten}
\subfile{subfiles/20251113150354-relazione_stabile}
\subfile{subfiles/20251113150436-relazione_stabile_in_un_modello_mostro}
\subfile{subfiles/20251113150604-combinazione_booleana_di_relazioni_stabili_e_stabile}
\subfile{subfiles/20251113150652-insieme_approssimabile_da_una_relazione}
\subfile{subfiles/20251113150745-caratterizzazione_relazione_stabile_tramite_insiemi_approssimabili}
\subfile{subfiles/20251113150815-insieme_approssimabile_da_una_relazione_in_un_modello_mostro}
\subfile{subfiles/20251113174136-topologia_di_2_x}
\subfile{subfiles/20251113174453-cardinalita_insieme_delle_parti}
\subfile{subfiles/20251113174803-matrice}
\subfile{subfiles/20251113175327-insieme_esternamente_definibile_in_un_modello_mostro}
\subfile{subfiles/20251115155511-forma_differenziale_in_un_punto}
\subfile{subfiles/20251115160537-differenziale_di_una_forma}
\subfile{subfiles/20251115172442-gruppo_di_coomologia_di_de_rham}
\subfile{subfiles/20251115172517-forma_differenziale_chiusa}
\subfile{subfiles/20251115172652-campo_vettoriale_conservativo}
\subfile{subfiles/20251115173611-coomologia_di_de_rham_di_r}
\subfile{subfiles/20251115173712-coomologia_di_de_rham_di_r2}
\subfile{subfiles/20251115173810-varieta_differenziabili_diffeomorfe_hanno_stessa_coomologia_di_de_rham}
\subfile{subfiles/20251115174001-pullback_di_una_funzione_tra_varieta_differenziabili}
\subfile{subfiles/20251115174538-0_gruppo_di_coomologia_di_de_rham_di_una_varieta_connessa}
\subfile{subfiles/20251115175943-prodotto_wedge_in_coomologia_di_de_rham}
\subfile{subfiles/20251115180438-anello_di_coomologia_di_de_rham}
\subfile{subfiles/20251115180518-diffeomorfismo_tra_varieta_differenziabili_induce_isomorfismo_tra_anelli_di_coomologia_di_de_rham}
\subfile{subfiles/20251115180827-anello_graduato}
\subfile{subfiles/20251115180902-anello_graduato_anticommutativo}
\subfile{subfiles/20251115181519-coomologia_di_de_rham_come_funtore}
\subfile{subfiles/20251115182133-successione_di_spazi_vettoriali_esatta}
\subfile{subfiles/20251115182320-complesso_di_cocatene}
\subfile{subfiles/20251115182537-coomologia_di_un_complesso_di_cocatene}
\subfile{subfiles/20251115182606-morfismo_tra_complessi_di_cocatene}
\subfile{subfiles/20251115182707-sec_di_complessi_di_cocatene}
\subfile{subfiles/20251115182904-morfismo_tra_complessi_di_cocatene_induce_morfismo_in_coomologia}
\subfile{subfiles/20251115182954-zig_zag_lemma_per_complessi_di_cocatene}
\subfile{subfiles/20251115183635-teorema_di_mayer_vietoris_in_coomologia}
\subfile{subfiles/20251115184223-coomologia_della_circonferenza}
\subfile{subfiles/20251115184248-coomologia_delle_sfere}
\subfile{subfiles/20251115184544-forma_volume_su_una_varieta_differenziabile}
\subfile{subfiles/20251115185324-caratterizzazione_varieta_differenziabile_orientabile_tramite_forma_voluma}
\subfile{subfiles/20251115185654-integrazione_di_forme_su_varieta_differenziabile_orientata}
\subfile{subfiles/20251115190058-teorema_di_stokes}
\subfile{subfiles/20251115190905-coomologia_dello_spazio_proiettivo_reale}
\subfile{subfiles/20251115190941-coomologia_del_toro}
\subfile{subfiles/20251115191224-ricoprimento_aciclico_di_una_varieta_differenziabile}
\subfile{subfiles/20251115191303-varieta_differenziabile_di_tipo_finito}
\subfile{subfiles/20251115191355-varieta_differenziabile_di_tipo_finito_ha_coomologia_di_dimensione_finita}
\subfile{subfiles/20251115192241-coomologia_a_supporto_compatto}
\subfile{subfiles/20251115192308-coomologia_a_supporto_compatto_di_r}
\subfile{subfiles/20251115192344-coomologia_a_supporto_compatto_di_rn}
\subfile{subfiles/20251117121206-coomologia_in_dimensione_massima}
\subfile{subfiles/20251117145729-doppia_applicazione_della_dualita_di_poincare}
\subfile{subfiles/20251117150647-caratterizzazione_forme_esatte_di_grado_massimo_su_varieta_orientabile_connessa_e_compatta}
\subfile{subfiles/20251117151400-boolean_valued_semantics_for_infinitary_logics_corso}
\subfile{subfiles/20251118175203-associativita}
\subfile{subfiles/20251118175221-proprieta_distributiva}
\subfile{subfiles/20251120111212-teorema_di_milliken_taylor}
\subfile{subfiles/20251121140243-teorema_di_schwarz}
\subfile{subfiles/20251121143525-kernel_di_una_funzione_tra_spazi_vettoriali}
\subfile{subfiles/20251121143644-quoziente_di_spazi_vettoriali}
\subfile{subfiles/20251121174615-pullback_di_una_funzione_tra_varieta_differenziabili_in_coomologia}
\subfile{subfiles/20251123155658-insieme_approssimabile_dal_basso_da_una_relazione}
\subfile{subfiles/20251124090935-rango_binario_di_shelah_di_una_relazione}
\subfile{subfiles/20251124094518-gestione_documenti_cartacei}
\subfile{subfiles/20251125114225-bib_fipsas_p1_lezione_t5_apparato_respiratorio_e_circolatorio}
\subfile{subfiles/20251125114515-bib}
\subfile{subfiles/20251127142345-bib_baro}
\subfile{subfiles/20251127172859-esercizio_nuoto_in_superficie}
\subfile{subfiles/20251127172906-esercizio_apnea_dinamica_senza_attrezzi}
\subfile{subfiles/20251127172912-esercizio_salvamento_di_un_apneista_incosciente_sul_fondo}
\subfile{subfiles/20251127172918-esercizio_capovolte_in_raccolta}
\subfile{subfiles/20251127172923-esercizio_equipaggiamento_sul_fondo}
\subfile{subfiles/20251127172928-esercizio_capovolte_con_attrezzatura}
\subfile{subfiles/20251201092513-sicurezza_in_immersione}
\subfile{subfiles/20251201092909-attrezzatura_subacquea}
\subfile{subfiles/20251201093227-istruttore_subacqueo}
\subfile{subfiles/20251201093908-subacquea_lancio_del_pallone_sparabile}
\subfile{subfiles/20251201094746-decompressione_gas_in_immersione}
\subfile{subfiles/20251201155413-restrizione_di_una_forma_ad_una_sottovarieta}
\subfile{subfiles/20251201160758-isomorfismo_tra_algebre_graduate}
\subfile{subfiles/20251201163733-gruppo_abeliano_graduato}
\subfile{subfiles/20251210093225-inbox}
\subfile{subfiles/20251222142956-teorema_della_derivata_nulla}
\subfile{subfiles/20251222162256-funzione_bilineare}
\subfile{subfiles/20251223102452-pullback_di_una_inclusione_tra_varieta_differenziabili}
\subfile{subfiles/20251223104832-forme_differenziali_come_fascio}
\subfile{subfiles/20251223131440-coomologia_delle_unioni_disgiunte}
\subfile{subfiles/20251223145108-riduzione_della_coomologia_di_un_prodotto_con_la_retta_reale}
\subfile{subfiles/20251223145901-teorema_di_invarianza_omotopica_per_la_coomologia_di_de_rham}
\subfile{subfiles/20251223152054-varieta_differenziabile_orientabile}
\subfile{subfiles/20251223153341-supporto_di_una_forma_differenziale}
\subfile{subfiles/20251223153742-atlante_orientato}
\subfile{subfiles/20251223153807-partizione_dell_unita}
\subfile{subfiles/20251223161934-orientabilita_delle_sfere}
\subfile{subfiles/20251229094742-formula_di_kunneth}
\subfile{subfiles/20251229100321-rn_e_contraibile}
\subfile{subfiles/20251229101317-varieta_differenziabili_diffeomorfe_hanno_stessa_coomologia_a_supporto_compatto}
\subfile{subfiles/20251229101448-successione_di_spazi_vettoriali_esatta_induce_successione_esatta_sui_duali}
\subfile{subfiles/20251229102111-teorema_fondamentale_del_calcolo_integrale}
\subfile{subfiles/20251229103408-funzione_duale}
\subfile{subfiles/20251229105636-successione_di_mayer_vietoris_per_coomologia_a_supporto_compatto}
\subfile{subfiles/20251229110021-forma_differenziale_a_supporto_compatto}
\subfile{subfiles/20251229110947-estensione_a_zero_di_una_forma_differenziale_a_supporto_compatto}
\subfile{subfiles/20251229114152-dualita_di_poincare}
\subfile{subfiles/20251229120415-prodotto_tensoriale}
\subfile{subfiles/20251229125103-immagine_continua_di_spazio_compatto_e_compatto}
\subfile{subfiles/20251229151511-forma_volume_sulla_sfera}
\subfile{subfiles/20251229151929-coomologia_del_nastro_di_moebius}
\subfile{subfiles/20251229160152-caratterizzazione_forma_esatta_sulla_sfera}
\subfile{subfiles/20251229160239-caratterizzazione_forma_esatta_nello_spazio_proiettivo_reale_di_dimensione_dispari}
\subfile{subfiles/20251230110059-integrale_di_forme_su_sottovarieta_differenziabile}
\subfile{subfiles/20251230111259-coomologia_delle_superfici_topologiche_compatti_orientabili}
\subfile{subfiles/20251230113510-grado_di_una_funzione_propria_tra_varieta_differenziabili}
\subfile{subfiles/20251230113815-teorema_di_sard}
\subfile{subfiles/20251230113846-teorema_del_grado}
\subfile{subfiles/20251230144150-teorema_nullita_rango}
\subfile{subfiles/20251230172241-superficie_topologica}
\subfile{subfiles/20260108192915-funtore_da_rmod_a_rmod_torsione}
\subfile{subfiles/20260109154336-delta_di_kronecher}
\subfile{subfiles/20260109173733-rango_di_un_modulo}
\subfile{subfiles/20260109174823-prodotto_cartesiano_di_moduli}
\subfile{subfiles/20260112124147-sottomoduli_in_somma_diretta}
\subfile{subfiles/20260112124828-ogni_anello_ha_struttura_di_modulo_su_se_stesso}
\subfile{subfiles/20260115121346-logical_pluralism}
\subfile{subfiles/20260115123905-sottogruppo_generato}
\subfile{subfiles/20260117172721-condizione_sufficiente_modello_omega_satudo}
\subfile{subfiles/20260126110551-funzione_olomorfa}
\subfile{subfiles/20260126184314-funzione_olomorfa_su_un_aperto_semplicemente_connesso_ammette_primitiva}
\subfile{subfiles/20260126184347-logaritmo_complesso_e_olomorfo_su_un_aperto_semplicemente_connesso}
\subfile{subfiles/20260126184450-funzione_olomorfa_su_un_aperto_semplicemente_connesso_ammette_radice_m_esima}
\subfile{subfiles/20260127095157-kkk}
\subfile{subfiles/20260127100741-spazio_topologico_semplicemente_connesso_e_connesso}
\subfile{subfiles/20260127101434-caratterizzazione_di_successione_esatta_di_fasci_tramite_spighe}
\subfile{subfiles/20260127101935-successione_di_morfismi_esatta}
\subfile{subfiles/20260127104824-caratterizzazione_morfismo_di_fasci_iniettivo_tramite_spighe}
\subfile{subfiles/20260127105736-esattezza_di_successione_di_fasci_rispetto_alle_sezioni_globali}
\subfile{subfiles/20260127110049-complesso}
\subfile{subfiles/20260127112715-atlante_complesso}
\subfile{subfiles/20260127112733-struttura_complessa}
\subfile{subfiles/20260127112752-varieta_complessa}
\subfile{subfiles/20260127112828-superficie_di_riemann}
\subfile{subfiles/20260127112849-genere_topologico}
\subfile{subfiles/20260127112905-sfera_di_riemann}
\subfile{subfiles/20260127112924-esempi_fondamentali_di_varieta_complesse}
\subfile{subfiles/20260127112948-struttura_complessa_per_le_sfere}
\subfile{subfiles/20260127113001-toro_complesso}
\subfile{subfiles/20260127131350-sottoinsieme_discreto}
\subfile{subfiles/20260127132618-funzione_biolomorfa}
\subfile{subfiles/20260127134302-equazioni_di_cauchy_riemann}
\subfile{subfiles/20260127142419-proiezione_stereografica}
\subfile{subfiles/20260128102226-topologia_piu_fine}
\subfile{subfiles/20260128105613-azione_di_gruppo_su_uno_spazio_topologico}
\subfile{subfiles/20260128120739-punto_di_accumulazione}
\subfile{subfiles/20260128123515-sottoinsieme_discreto}
\subfile{subfiles/20260128124105-ordine_di_una_funzione_olomorfa}
\subfile{subfiles/20260128124510-teorema_di_analicita_delle_funzioni_olomorfe}
\subfile{subfiles/20260128124917-funzione_olomorfa_e_localmente_potenza_di_biolomorfismo}
\subfile{subfiles/20260128130443-comportamento_locale_di_una_funzione_analitica}
\subfile{subfiles/20260128131354-principio_di_identita_per_funzioni_olomorfe}
\subfile{subfiles/20260128143021-teorema_di_inversione_locale}
\subfile{subfiles/20260128143427-funzione_olomorfa_iniettiva_e_biolomorfismo_locale}
\subfile{subfiles/20260128143822-funzione_olomorfa_su_una_superficie_di_riemann}
\subfile{subfiles/20260128143847-fascio_delle_funzioni_olomorfe_su_una_superficie_di_riemann}
\subfile{subfiles/20260128143914-teorema_del_massimo_modulo}
\subfile{subfiles/20260128144016-principio_di_identita_per_funzioni_olomorfe_su_superfici_di_riemann}
\subfile{subfiles/20260128144055-funzione_meromorfa}
\subfile{subfiles/20260128144105-funzione_meromorfa_su_una_superficie_di_riemann}
\subfile{subfiles/20260128144151-fascio_delle_funzioni_meromorfe_su_una_superficie_di_riemann}
\subfile{subfiles/20260128144305-funzioni_meromorfe_sulla_sfera_di_riemann}
\subfile{subfiles/20260128144433-ordine_di_una_funzione_meromorfa}
\subfile{subfiles/20260128144450-ordine_di_una_funzione_meromorfa_su_una_superficie_di_riemann}
\subfile{subfiles/20260128144717-isomorfismo_tra_superfici_di_riemann}
\subfile{subfiles/20260128144827-teorema_della_mappa_aperta_superfici_di_riemann}
\subfile{subfiles/20260128152203-modulo_di_un_numero_complesso}
\subfile{subfiles/20260128152954-funzione_olomorfa_su_una_superficie_di_riemann_compatta_a_valori_complessi}
\subfile{subfiles/20260128154601-singolarita_isolata_analisi_complessa}
\subfile{subfiles/20260128163831-serie_di_laurent}
\subfile{subfiles/20260128172415-sottoinsieme_discreto_in_un_compatto}
\subfile{subfiles/20260128175800-torrent}
\subfile{subfiles/20260128181135-sfera_di_riemann_biolomorfa_al_piano_proiettivo_complesso}
\subfile{subfiles/20260128182932-teorema_della_mappa_aperta_analisi_complessa}
\subfile{subfiles/20260128183731-corrispondenza_funzioni_meromorfe_su_una_superficie_di_riemann_e_funzioni_olomorfe_sulla_sfera_di_riemann}
\subfile{subfiles/20260128183856-funzione_olomorfa_tra_superfici_di_riemann_condominio_compatto}
\subfile{subfiles/20260128184014-fibra_di_una_funzione}
\subfile{subfiles/20260129095428-latex_testo_multiriga_in_ambiente_matematico}
\subfile{subfiles/20260129103610-grafico_di_funzioni_olomorfe_e_superficie_di_riemann}
\subfile{subfiles/20260129103755-teorema_della_funzione_implicita_complesso}
\subfile{subfiles/20260129103901-superficie_di_riemann_definita_tramite_teorema_della_funzione_implicita}
\subfile{subfiles/20260129103934-curva_piana_affine_liscia_complessa}
\subfile{subfiles/20260129103947-curva_piana_proiettiva_liscia_complessa}
\subfile{subfiles/20260129104037-sottovarieta_complessa_dello_spazio_proiettivo}
\subfile{subfiles/20260129104052-superficie_di_riemann_proiettiva}
\subfile{subfiles/20260129104215-forma_normale_locale_per_superfici_di_riemann}
\subfile{subfiles/20260129104340-punto_di_ramificazione_per_una_funzione_tra_superfici_di_riemann}
\subfile{subfiles/20260129120125-polinomio_irriducibile}
\subfile{subfiles/20260129141700-esempio_di_calcolo_delle_moltiplicita_di_una_funzione_tra_superfici_di_riemann}
\subfile{subfiles/20260129171444-teorema_del_grado_per_olomorfismi_tra_superfici_di_riemannn}
\subfile{subfiles/20260129171557-caratterizzazione_isomorfismo_tra_superfici_di_riemann_tramite_grado}
\subfile{subfiles/20260129171638-somma_ordini_di_una_funzione_meromorfa_su_una_superficie_di_riemann_e_nulla}
\subfile{subfiles/20260129171718-caratterizzazione_sfera_di_riemann_tramite_meromorfismo}
\subfile{subfiles/20260129171755-caratteristica_di_eulero_poincare}
\subfile{subfiles/20260129171832-formula_di_hurewicz_superfici_di_riemann}
\subfile{subfiles/20260130094844-funzione_z_m}
\subfile{subfiles/20260130100707-differenza_di_potenze}
\subfile{subfiles/20260130101500-polinomi_sono_continui}
\subfile{subfiles/20260130103009-successione_in_uno_spazio_topologico}
\subfile{subfiles/20260130103958-cardinalita_della_fibra_di_un_olomorfismo_tra_superfici_di_riemann}
\subfile{subfiles/20260130154758-triangolazione_di_una_superficie_topologica}
\subfile{subfiles/20260130154906-caratteristica_di_eulero}
\subfile{subfiles/20260130155105-teorema_di_classificazione_delle_superfici_topologiche}
\subfile{subfiles/20260130155129-superficie_topologica_orientabile}
\subfile{subfiles/20260130155514-sfera_con_g_buchi}
\subfile{subfiles/20260130155555-somma_connessa_di_superfici_topologiche}
\subfile{subfiles/20260130155743-superficie_topologica_non_orientabile_di_genere_n}
\subfile{subfiles/20260130155800-genere_di_una_superficie_topologica_compatta}
\subfile{subfiles/20260130160217-piano_proiettivo_reale_e_varieta_topologica}
\subfile{subfiles/20260130162420-genere_topologico_di_una_superficie_di_riemann_non_cresce_per_olomorfismi}
\subfile{subfiles/20260130162657-genere_topologico_di_un_curva_piana_proiettiva_liscia_complessa}
\subfile{subfiles/20260130172938-funzione_olomorfa_tra_superfici_di_riemann_compatte_induce_rivestimento_topologico}
\subfile{subfiles/20260131171203-topic_geometria_differenziale_complessa}
\subfile{subfiles/20260131194939-funzione_olomorfa_tra_superfici_di_riemann_di_genere_uno_e_rivestimento_topologico}
\subfile{subfiles/20260131201301-funzione_olomorfa_a_valori_nella_sfera_di_riemann_ramifica}
\subfile{subfiles/20260201170208-prova}
\subfile{subfiles/20260201172613-mappe_tra_tori_complessi}
\subfile{subfiles/20260201182313-isomorfismo_tra_tori_complessi}
\subfile{subfiles/20260201182337-classificazione_dei_tori_complessi}
\subfile{subfiles/20260201185059-azione_di_gruppo}
\subfile{subfiles/20260201200827-teorema_di_liouville}
\subfile{subfiles/20260201232630-complesso_coniugato}
\subfile{subfiles/20260201235333-gruppo_delle_funzioni_in_z}
\subfile{subfiles/20260201235551-divisore_di_una_superficie_di_riemann}
\subfile{subfiles/20260202095650-divisore_di_una_funzione_meromorfa}
\subfile{subfiles/20260202100443-divisori_linearmente_equivalenti_di_una_superficie_di_riemann}
\subfile{subfiles/20260202100511-gruppo_di_picard_di_una_superficie_di_riemann}
\subfile{subfiles/20260202100601-caratterizzazione_sfera_di_riemann_tramite_gruppo_di_picard}
\subfile{subfiles/20260202100903-divisore_effettivo_di_una_superficie_di_riemann}
\subfile{subfiles/20260202101030-pullback_di_un_divisore_tra_superfici_di_riemann}
\subfile{subfiles/20260202101057-divisore_di_ramificazione_di_una_funzione_olomorfa}
\subfile{subfiles/20260202101128-fascio_associato_ad_un_divisore_su_una_superficie_di_riemann}
\subfile{subfiles/20260202101334-sezioni_globali_del_fascio_associato_ad_un_divisore_su_una_superficie_di_riemann}
\subfile{subfiles/20260202101604-divisori_linearmente_equivalenti_inducono_fasci_isomorfi}
\subfile{subfiles/20260202103934-pic0_di_una_superficie_di_riemann}
\subfile{subfiles/20260202132204-latex_enumerate_skippabile}
\subfile{subfiles/20260202141057-latex_verificare_se_argomento_condizionale_e_pieno}
\subfile{subfiles/20260202173818-successione_esatta_corta_di_fasci_data_da_un_divisore_e_un_punto_di_una_superficie_di_riemann}
\subfile{subfiles/20260202174051-fibrato_vettoriale_complesso_di_una_varieta_differenziabile}
\subfile{subfiles/20260202174103-fibrato_vettoriale_olomorfo}
\subfile{subfiles/20260202174123-cociclo_di_un_fibrato_vettoriale_olomorfo}
\subfile{subfiles/20260202174221-spazio_tangente_complessificato}
\subfile{subfiles/20260202174737-funzione_antiolomorfa_su_una_superficie_di_riemann}
\subfile{subfiles/20260202175200-spazio_tangente_olomorfo}
\subfile{subfiles/20260202221918-sezioni_globali_del_fascio_associato_ad_un_divisore_sulla_sfera_di_riemann}
\subfile{subfiles/20260202222451-sezioni_globali_del_fascio_associato_ad_un_punto_per_superfici_di_riemann_compatte}
\subfile{subfiles/20260202230147-formula_di_grassman}
\subfile{subfiles/20260203110110-complesso_di_catene_immagine}
\subfile{subfiles/20260203110118-complesso_di_catene_nucleo}
\subfile{subfiles/20260203110150-complesso_di_catene_somma}
\subfile{subfiles/20260203110214-intersezione_di_complessi_di_catene}
\subfile{subfiles/20260203120429-caratterizzazione_funzioni_tra_complessi_di_catene_tramite_successioni_esatte}
\subfile{subfiles/20260203164710-latex_dimensione_font_mathmode}
\subfile{subfiles/20260204100611-somma_diretta_di_complessi_di_catene}
\subfile{subfiles/20260204110609-latex_mostrare_margini_di_un_documento}
\subfile{subfiles/20260204120902-omologia_della_somma_diretta_di_complessi_di_catene}
\subfile{subfiles/20260205123011-omologia_singolare_relativa_di_una_coppia_topologica_buona}
\subfile{subfiles/20260205151208-attaccamento_di_una_k_cella_a_uno_spazio_topologico}
\subfile{subfiles/20260205160413-omologia_per_attaccamento_di_k_cella}
\subfile{subfiles/20260205161432-cw_complesso}
\subfile{subfiles/20260205161626-struttura_di_cw_complesso_per_spazio_proiettivo_reale}
\subfile{subfiles/20260205161836-omologia_singolare_di_cw_complessi}
\subfile{subfiles/20260205161912-complesso_di_catene_cellulari}
\subfile{subfiles/20260206124233-latex_gestione_todos}
\subfile{subfiles/20260206182305-rango_in_una_successione_di_r_moduli_finitamente_generati}
\subfile{subfiles/20260207173708-unione_finita_di_compatti_e_compatta}
\subfile{subfiles/20260209110448-latex_ambiene_commento}
\subfile{subfiles/20260209135755-latex_gestione_multi_file}
\subfile{subfiles/20260209153646-sistema_diretto}
\subfile{subfiles/20260209161309-omologia_singolare_di_cw_complessi_come_limite_diretto}
\subfile{subfiles/20260209161631-calcolo_dell_omologia_singolare_dello_spazio_proiettivo_reale}
\subfile{subfiles/20260209161721-teorema_di_confronto_tra_omologia_singolare_e_simpliciale}
\subfile{subfiles/20260209161917-funtore_esatto}
\subfile{subfiles/20260209161944-prodotto_tensoriale_di_moduli}
\subfile{subfiles/20260209162059-funtore_prodotto_tensoriale}
\subfile{subfiles/20260209162209-complesso_di_catene_come_modulo_graduato}
\subfile{subfiles/20260209162236-prodotto_tensoriale_di_complessi_di_catene}
\subfile{subfiles/20260209162445-teorema_di_kunneth}
\subfile{subfiles/20260209162551-teorema_dei_coefficienti_universali}
\subfile{subfiles/20260209163200-topologia_debole}
\subfile{subfiles/20260209163309-funtore_covariante_applicato_ad_un_sistema_diretto}
\subfile{subfiles/20260209174827-mappe_di_bordo_tra_moduli_di_catene_cellulari_dello_spazio_proiettivo_reale}
%% \subfile{subfiles/index}
%% \subfile{subfiles/prova}

\end{document}
