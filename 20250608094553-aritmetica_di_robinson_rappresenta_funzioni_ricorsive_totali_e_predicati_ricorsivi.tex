% Created 2026-02-07 Sat 19:33
% Intended LaTeX compiler: pdflatex
\documentclass[10pt]{article}
%% CREATO CON ORG - EMACS
\newcommand{\use}[2][]{\usepackage[#1]{#2}}
% PACCHETTI FONDAMENTLAI
\use[utf8]{inputenc}
\use[T1]{fontenc}
\use{graphicx}
\use{longtable}
\use{wrapfig}
\use{rotating}
\use[normalem]{ulem}
\use{amsmath}
\use{amsthm}
\use{amssymb}

\use{eucal} % Cambia mathcal{...}

\use{capt-of}
\use[italian]{babel}
\use[babel]{csquotes}
% bib la TEX lo carica in automatico org-cite
\use{microtype}
\use{lmodern}
\use{subfig} % sottofigure
\use{multicol} % due colonne
\use{lipsum} % lorem ipsum
\use{color} % colori in latex
\use{parskip} % rimuove l'indentazione dei nuovi paragrafi %% Add parbox=false to all new tcolorbox
\use{centernot}
\use[outline]{contour}\contourlength{3pt}
\use{fancyhdr}
\use{layout}
\use[most]{tcolorbox} % Riquadri colorati
\use{ifthen} % IFTHEN
\use{geometry}

% pacchetti matematica
\use{yhmath}
\use{dsfont}
\use{mathrsfs}
\use{cancel} % semplificare
\use{polynom} %divisione tra polinomi
\use{forest} % grafi ad albero
\use{booktabs} % tabelle
\use{commath} %simboli e differenziali
\use{bm} %bold
\use[fulladjust]{marginnote} %to use marginnote for date notes
\use{arrayjobx}%array
\use[intlimits]{empheq} % Riquadri colorati attorno alle equazioni
\use{mathtools}
\use{circuitikz} % Disegnare i circuiti
\use{mathtools}
\use{stmaryrd} % [[ \llbracket ]] \rrbracket
\use{bussproofs} % dimostrazioni

%%%%%%%%%%%%%


%%%% QUIVER
\newcommand{\duepunti}{\,\mathchar\numexpr"6000+`:\relax\,}
% A TikZ style for curved arrows of a fixed height, due to AndréC.
\tikzset{curve/.style={settings={#1},to path={(\tikztostart)
    .. controls ($(\tikztostart)!\pv{pos}!(\tikztotarget)!\pv{height}!270:(\tikztotarget)$)
    and ($(\tikztostart)!1-\pv{pos}!(\tikztotarget)!\pv{height}!270:(\tikztotarget)$)
    .. (\tikztotarget)\tikztonodes}},
    settings/.code={\tikzset{quiver/.cd,#1}
        \def\pv##1{\pgfkeysvalueof{/tikz/quiver/##1}}},
    quiver/.cd,pos/.initial=0.35,height/.initial=0}

% TikZ arrowhead/tail styles.
\tikzset{tail reversed/.code={\pgfsetarrowsstart{tikzcd to}}}
\tikzset{2tail/.code={\pgfsetarrowsstart{Implies[reversed]}}}
\tikzset{2tail reversed/.code={\pgfsetarrowsstart{Implies}}}
% TikZ arrow styles.
\tikzset{no body/.style={/tikz/dash pattern=on 0 off 1mm}}
%%%%%%%%%%


%% DEFINIZIONI COMANDI MATEMATICI
\let\sin\relax %TOGLIE LA DEFINIZIONE SU "\sin"

% cambia la definizione di empty set
% ---
\let\oldemptyset\emptyset
% ---
% \let\emptyset\varnothing
% ---
% \let\emptyset\relax
% \newcommand{\emptyset}{\text{\textnormal{\O}}}
% ---

\DeclareMathOperator{\bounded}{bd}
\DeclareMathOperator{\sin}{sen}
\DeclareMathOperator{\epi}{Epi}
\DeclareMathOperator{\cl}{cl}
\DeclareMathOperator{\graph}{graph}
\DeclareMathOperator{\arcsec}{arcsec}
\DeclareMathOperator{\arccot}{arccot}
\DeclareMathOperator{\arccsc}{arccsc}
\DeclareMathOperator{\spettro}{Spettro}
\DeclareMathOperator{\nulls}{nullspace}
\DeclareMathOperator{\dom}{dom}
\DeclareMathOperator{\ar}{ar}
\DeclareMathOperator{\const}{Const}
\DeclareMathOperator{\fun}{Fun}
\DeclareMathOperator{\rel}{Rel}
\DeclareMathOperator{\altezza}{ht}
\let\det\relax %TOGLIE LA DEFINIZIONE SU "\det"
\DeclareMathOperator{\det}{det}
\DeclareMathOperator{\End}{End}
\DeclareMathOperator{\gl}{GL}
\def\Id{\mathrm{Id}}
\def\id{\mathrm{id}}
\DeclareMathOperator{\I}{\mathds{1}}
\DeclareMathOperator{\II}{II}
\DeclareMathOperator{\rank}{rank}
\DeclareMathOperator{\tr}{tr}
\DeclareMathOperator{\tc}{t.c.}
\DeclareMathOperator{\T}{T}
\DeclareMathOperator{\var}{Var}
\DeclareMathOperator{\cov}{Cov}
\DeclareMathOperator{\st}{st}
\DeclareMathOperator{\mon}{Mon}
\newcommand{\card}[1]{\left\vert #1 \right\vert}
\newcommand{\trasposta}[1]{\prescript{\text{T}}{}{#1}}
\newcommand{\1}{\mathds{1}}
\newcommand{\R}{\mathds{R}}
\newcommand{\diesis}{\#}
\newcommand{\bemolle}{\flat}
\newcommand{\nonstandard}[1]{\prescript{*}{}{#1}}
\newcommand{\starR}{\nonstandard{\R}}
\newcommand{\borel}{\mathscr{B}}
\newcommand{\lebesgue}[1]{\mathscr{L}\left(#1\right)}
\newcommand{\media}{\mathds{E}}
\newcommand{\K}{\mathds{K}}
\newcommand{\A}{\mathds{A}}
\newcommand{\Q}{\mathds{Q}}
\newcommand{\N}{\mathds{N}}
\newcommand{\C}{\mathds{C}}
\newcommand{\Z}{\mathds{Z}}
\newcommand{\qo}{\hspace{1em}\text{q.o.}\,}
\renewcommand{\tilde}[1]{\widetilde{#1}}
\renewcommand{\parallel}{\mathrel{/\mkern-5mu/}}
\newcommand{\parti}[2][]{\wp_{#1}(#2)}
\newcommand{\diff}[1]{\operatorname{d}_{#1}}
\let\oldvec\vec
\renewcommand{\vec}[1]{\overrightarrow{\vphantom{i}#1}}
\newcommand{\floor}[1]{\left\lfloor #1 \right\rfloor}
\newcommand{\cat}[1]{\mathbf{#1}}
\newcommand{\dfreccia}[1]{\xrightarrow{\ #1 \ }}
\newcommand{\sfreccia}[1]{\xleftarrow{\ #1 \ }}
\newcommand{\formalsum}[2]{{\sum_{#1}^{#2}}{\vphantom{\sum}}'}
\newcommand{\minim}[2]{\mu_{#1}\, \left(#2\right)}
\newcommand{\concat}{\null^{\frown}} % concatenazione di stringe
\newcommand{\godelcode}[1]{\langle\!\langle #1 \rangle\!\rangle}
\newcommand{\godeldec}[1]{(\!(#1)\!)}
\newcommand{\termcode}[1]{\ulcorner #1\urcorner}
\newcommand{\partialto}{\dashrightarrow}
\newcommand{\restricted}{\upharpoonright}
\newcommand{\embeds}{\precsim}
\newcommand{\surjects}{\twoheadrightarrow}
\newcommand{\equipotenti}{\asymp}
%% \newcommand{\dotplus}{\mathbin{\dot{+}}} %% A quanto pare esiste già
\newcommand{\bigdot}{\mathbin{\boldsymbol{\cdot}}}
\newcommand{\dotexp}[1]{^{.#1}}
\newcommand{\conv}{\mathbin{*}}
\newcommand{\convolution}[2]{(#1\conv #2)}
\newcommand{\nil}{\mathfrak{N}}
\newcommand{\divisore}{\mathrel{|}}
\newcommand{\simplesso}[1]{\mathrm{e}_{#1}}

\renewcommand{\iff}{\mathrel{\longleftrightarrow}} %% Notazione Logica.
\newcommand{\oldiff}{\mathrel{\Longleftrightarrow}}
\renewcommand{\implies}{\mathrel{\rightarrow}} %% Notazione Logica
\newcommand{\oldimplies}{\mathrel{\Longrightarrow}}
\renewcommand{\impliedby}{\mathrel{\leftarrow}} %% Notazione Logica
\newcommand{\oldimpliedby}{\mathrel{\Longleftarrow}}

\newcommand{\IFF}{\quad\Longleftrightarrow\quad}
\newcommand{\IMPLICA}{\quad\Longrightarrow\quad}


\renewcommand{\descriptionlabel}[1]{\hspace{\labelsep}\normalfont #1} % remove bold from description


%% Definizione di Divergenza di K-L

\DeclarePairedDelimiterX{\infdivx}[2]{(}{)}{%
  #1\;\delimsize\|\;#2%
}
\newcommand{\kldiv}{D_{KL}\infdivx}

%% Definizione di \dotminus

\makeatletter
\newcommand{\dotminus}{\mathbin{\text{\@dotminus}}}

\newcommand{\@dotminus}{%
  \ooalign{\hidewidth\raise1ex\hbox{.}\hidewidth\cr$\m@th-$\cr}%
}
\makeatother

%tramite i prossimi due comandi posso decidere come scrivere i logaritmi naturali in tutti i documenti: ho infatti eliminato qualsiasi differenza tra "ln" e "log": se si vuole qualcosa di diverso bisogna inserire manualmente il tutto
\let\ln\relax
\DeclareMathOperator{\ln}{ln}
\let\log\relax
\DeclareMathOperator{\log}{log}
%%%%%%

%% NUOVI COMANDI
\newcommand{\straniero}[1]{\textit{#1}} %parole straniere
\newcommand{\titolo}[1]{\textsc{#1}} %titoli
\newcommand{\qedd}{\tag*{$\blacksquare$}} %qed per ambienti matemastici
\renewcommand{\qedsymbol}{$\blacksquare$} %modifica colore qed
\newcommand{\ooverline}[1]{\overline{\overline{#1}}}
\newcommand{\circoletto}[1]{\left(#1\right)^{\text{o}}}
%
\newcommand{\qmatrice}[1]{\begin{pmatrix}
#1_{11} & \cdots & #1_{1n}\\
\vdots & \ddots & \vdots \\
#1_{m1} & \cdots & #1_{mn}
\end{pmatrix}}
%
\newcommand{\parentesi}[2]{%
\underset{#1}{\underbrace{#2}}%
}
%
\newcommand{\norma}[1]{% Norma
\left\lVert#1\right\rVert%
}
\newcommand{\scalare}[2]{% Scalare
\left\langle #1, #2\right\rangle
}
%%%%%

%% RESTRIZIONI
\newcommand{\referenze}[2]{
        \phantomsection{}#2\textsuperscript{\textcolor{blue}{\textbf{#1}}}
}

\let\restriction\relax

\def\restriction#1#2{\mathchoice
              {\setbox1\hbox{${\displaystyle #1}_{\scriptstyle #2}$}
              \restrictionaux{#1}{#2}}
              {\setbox1\hbox{${\textstyle #1}_{\scriptstyle #2}$}
              \restrictionaux{#1}{#2}}
              {\setbox1\hbox{${\scriptstyle #1}_{\scriptscriptstyle #2}$}
              \restrictionaux{#1}{#2}}
              {\setbox1\hbox{${\scriptscriptstyle #1}_{\scriptscriptstyle #2}$}
              \restrictionaux{#1}{#2}}}
\def\restrictionaux#1#2{{#1\,\smash{\vrule height .8\ht1 depth .85\dp1}}_{\,#2}}
%%%%%%%%%%%

%%% FORMATTAZIONE FOOTNOTEMARK

\def\footnotemarkformatting#1{[#1]}
\renewcommand{\thefootnote}{\footnotemarkformatting{\arabic{footnote}}}

%% SEZIONE GRAFICA
\use{tikz}
\usetikzlibrary{matrix, patterns, calc, decorations.pathreplacing, hobby, decorations.markings, decorations.pathmorphing, babel}
\use{tikz-3dplot}
\use{mathrsfs} %per geogebra
\use{tikz-cd}
\tikzset
{
  %surface/.style={fill=black!10, shading=ball,fill opacity=0.4},
  plane/.style={black,pattern=north east lines},
  curve/.style={black,line width=0.5mm},
  dritto/.style={decoration={markings,mark=at position 0.5 with {\arrow{Stealth}}}, postaction=decorate},
  rovescio/.style={decoration={markings,mark=at position 0.5 with {\arrow{Stealth[reversed]}}}, postaction=decorate}
}
\use{pgfplots} % stampare le funzioni
        \pgfplotsset{/pgf/number format/use comma,compat=1.15}
        %\pgfplotsset{compat=1.15} %per geogebra
        \usepgfplotslibrary{fillbetween, polar}
%%%%%%

%% CITAZIONI
\use{lineno}

\newcommand{\citazione}[1]{%
  \begin{quotation}
  \begin{linenumbers}
  \modulolinenumbers[5]
  \begingroup
  \setlength{\parindent}{0cm}
  \noindent #1
  \endgroup
  \end{linenumbers}
  \end{quotation}\setcounter{linenumber}{1}
  }
%%%%%%

%%%%%%%%%%%%%%%%%%%%%%%%%%%%%%%%%%%%%%%%%%%%
%%%%%%%%%%%%%%%%%%%%%%%%%%%%%%%%%%%%%%%%%%%%

%% AMS THM

\theoremstyle{definition}% default
\newtheorem{thm}{Teorema}[section]
\newtheorem{lem}[thm]{Lemma}
\newtheorem{prop}[thm]{Proposizione}
\newtheorem{cor}[thm]{Corollario}
\newtheorem{esempio}[thm]{Esempio}
\theoremstyle{plain}
\newtheorem{definizione}[thm]{Definizione}
\theoremstyle{remark}
\newtheorem*{oss}{Osservazione}


%%%%%%%%%%%%%%%%%%%%%%%%%%%%%%%%%%%%%%%%%%%%
%%%%%%%%%%%%%%%%%%%%%%%%%%%%%%%%%%%%%%%%%%%%

\use{hyperref}
\hypersetup{%
        pdfauthor={Davide Peccioli},
        pdfsubject={},
        allcolors=black,
        citecolor=black,
%	colorlinks=true,
        bookmarksopen=true}
\setcounter{secnumdepth}{0} % rimuove i numeri di sezione senza rimuovere le ref
\renewcommand{\href}[2]{\textcolor{blue}{#2}} % disabilita il comando href
\use{enotez} %
\setenotez{%
 mark-format = \footnotemarkformatting % Mette i numeri tra parentesi quadre%
}\let\footnote=\endnote % rende tutte le note a pié pagina come delle note a fine file 


\let\olddocument\document % modifico l'ambiende documenti per non dover stampare \printendnote
\let\oldenddocument\enddocument
\renewenvironment{document}%
{%
  \olddocument
}{%
  \printendnotes\oldenddocument
}
\renewcommand{\thethm}{\arabic{thm}}

\usepackage[hyperref]{biblatex}
\addbibresource{~/Documents/org/roam/bib/master.bib}
\author{Davide Peccioli}
\date{\today}
\title{Aritmetica di Robinson rappresenta funzioni ricorsive totali e predicati ricorsivi}
\begin{document}

Sia \(L_{Q}=\set{+,\cdot,S,0}\) il \href{20250130162057-linguaggio_del_prim_ordine.org}{linguaggio} dell'aritmetica di Robinson, e sia \(Q\) l'\href{20250608093604-aritmetica.org}{aritmetica di Robinson}.
\begin{lem}
Se \(P,R \subseteq \N^{k}\) sono \href{20250608094213-insieme_rappresentato_da_una_formula.org}{rappresentabili} in \(Q\) dalle \href{20250131103317-formula_del_prim_ordine.org}{formule} \(\varphi\) e \(\psi\) rispettivamnte, allora anche
\begin{equation*}
P\cap R,\quad P\cup R, \sim P
\end{equation*}
sono rappresentati dalle formule \(\varphi \,\land\, \psi\), \(\varphi \,\lor\,\psi\) e \(\lnot\psi\), rispettivamente.

In particolare gli insiemi rappresentabili in \(Q\) formano un'algebra di Boole.
\end{lem}
\begin{lem}
Un sottoinsieme \(P \subseteq \N^{k}\) è rappresentabile in \(Q\) (da \href{20250603170559-complessita_di_una_formula_del_modello_standard.org}{una formula \(\Delta_{0}\)}) se e solo se la sua \href{20250215160218-funzione_caratteristica.org}{funzione caratteristica} \(\chi_{P}: \N^{k}\to \N\) è rappresentabile in \(Q\) (da \href{20250603170559-complessita_di_una_formula_del_modello_standard.org}{una formula \(\Delta_{0}\)}).
\end{lem}
\begin{proof}
(\(\Rightarrow\)): Se \(\varphi(x_{1},\dots,x_{k})\) rappresenta \(P \subseteq \N^{k}\) allora \(\psi\) rappresenta \(\chi_{P}\):
\begin{equation*}
\psi(x_{1},\dots,x_{k}, y):\quad \left[\varphi(x_{1},\dots,x_{k}) \,\land\, y=\overline{1}\right]\,\lor\,\left[\lnot\varphi(x_{1},\dots,x_{k}) \,\land\, y=\overline{0}\right].
\end{equation*}

(\(\Leftarrow\)): Se \(\psi(x_{1},\dots,x_{k},y)\) rappresenta \(\chi_{P}\), allora \(\varphi\) rappresenta \(P\):
\begin{equation*}
\varphi(x_{1},\dots,x_{k}):\quad \psi(x_{1},\dots,x_{k}, \overline{1}).
\end{equation*}
dove si è effettuata una \href{20250131123704-sostituzione_di_termini_in_una_formula.org}{sostituzione} di \(y\) con il \href{20250130162316-termine_del_prim_ordine.org}{termine} \(\overline{1}\).
\end{proof}
\begin{lem}
Per dimostrare che una funzione \(F:\N^{k}\to \N\) è rappresentata in una \(L_{Q}\subseteq L\)-\href{20250130114950-teoria_del_prim_ordine.org}{teoria} \(T\) da una formula del tipo
\begin{equation*}
f(x_{1},\dots,x_{k})=y
\end{equation*}
con \(f \in L\), allora è sufficiente mostrare che per ogni \(a_{1},\dots,a_{k} \in \N\)
\begin{equation*}
T\vdash f(\overline{a_{1}},\dots,\overline{a_{k}}) = \overline{F(a_{1},\dots,a_{k})}.
\end{equation*}

In particolare, quindi, la funzione successore su \(\N\): \(n\mapsto n+1\) è rappresentata in \(Q\) dalla formula
\begin{equation*}
S(x)=y.
\end{equation*}
\end{lem}
\begin{lem}
La formula \(x+y=z\) rappresenta in \(Q\) la funzione \(+:\N^{2}\to \N\).

La formula \(x\cdot y = z\) rappresenta in \(Q\) la funzione \(\cdot:\N^{2}\to \N\).
\end{lem}
\begin{cor}
Per ogni \(L_{Q}\)-termine chiuso \(t\) esiste \(n \in \N\) tale che \(Q\vdash t=\overline{n}\). \emph{(dimostrazione per induzione sull'\href{20250130162316-termine_del_prim_ordine.org}{altezza del termine})}
\end{cor}
\begin{lem}
La formula \(x_{1}=x_{2}\) rappresenta la relazione di uguaglianza
\begin{equation*}
P = \set{(n,m) \in \N^{2}\mid n=m}.
\end{equation*}
\end{lem}
\begin{cor}
Se \(t_{1},t_{2}\) sono \(L_{Q}\)-termini chiusi, allora
\begin{equation*}
Q\vdash t_{1}=t_{2}\quad\text{oppure}\quad Q\vdash \lnot(t_{1}=t_{2}).
\end{equation*}
\end{cor}
\begin{lem}
Per ogni \(n \in \N\)
\begin{equation*}
Q\vdash \forall\,x\ \left[\varphi_{\le}(x,\overline{n})\iff x=\overline{0} \,\lor\,\dots \,\lor\, x=\overline{n}\right]
\end{equation*}
dove \(\varphi_{\le}(x,y)\) è la formula \(\exists\,z\ (z+x=y)\).
\end{lem}
\begin{cor}
Questo lemma implica che in qualunque modello \(M\) di \(Q\), se un elemento \(q \in M\) è minore o uguale (nel senso della relazione binaria definita in \(M\) da \(\varphi_{\le}\)) ad un \href{20250608093604-aritmetica.org}{numero standard} \(\overline{n}^{M}\), allora \(q=\overline{m}^{M}\) per qualche \(m\le n\). Dunque l'\href{20250608093604-aritmetica.org}{isomorfismo canonico tra \(\N\) e i numeri standard di \(M\)} preserva anche l'ordine \(\le\).
\end{cor}
\begin{cor}
Se \(n \in \N\) e \(\varphi(x)\) è una \(L_{Q}\)-formula, allora sono equivalenti
\begin{enumerate}
\item \(Q\vdash \varphi(a)\) per ogni \(a \le n\);
\item \(Q\vdash \forall\,x\le\overline{n}\ \varphi(x)\).
\end{enumerate}
\end{cor}
\begin{lem}
La formula \(\varphi_{\le}(x,y): \exists\,z\ (z+x=y)\) \href{20250608094213-insieme_rappresentato_da_una_formula.org}{rappresenta} la relazione \(\le \subseteq \N^{2}\) in \(Q\).
\end{lem}
\begin{lem}
Per ogni \(b \in \N\):
\begin{equation*}
Q\vdash \forall\,x\ (\varphi_{\le}(x,\overline{b}) \,\lor\, \varphi_{\le}(\overline{b},x)).
\end{equation*}
\end{lem}
\begin{thm}
Ogni \href{20250215151458-funzioni_ricorsive.org}{funzione ricorsiva} \href{20250213105339-funzione_parziale.org}{totale} è \href{20250608094213-insieme_rappresentato_da_una_formula.org}{rappresentata in \(Q\)} da una \href{20250131103317-formula_del_prim_ordine.org}{formula} di \href{20250603170559-complessita_di_una_formula_del_modello_standard.org}{complessità} \(\Sigma_{1}\).
\end{thm}
\#+BEGIN\textsubscript{cor}
Sia \(L\supseteq L_{Q}\) un linguaggio del prim'ordine e \(T\) una \(L\)-teoria tale che \(T\supseteq Q\). Allora:
\begin{itemize}
\item Ogni \href{20250215151458-funzioni_ricorsive.org}{funzione ricorsiva} \href{20250213105339-funzione_parziale.org}{totale} è \href{20250608094213-insieme_rappresentato_da_una_formula.org}{rappresentata in \(T\)} da una \href{20250131103317-formula_del_prim_ordine.org}{formula} di \href{20250603170559-complessita_di_una_formula_del_modello_standard.org}{complessità} \(\Sigma_{1}\).
\item Ogni insieme ricorsivo è \href{20250608094213-insieme_rappresentato_da_una_formula.org}{rappresentata in \(T\)} sia da una \href{20250131103317-formula_del_prim_ordine.org}{formula} di \href{20250603170559-complessita_di_una_formula_del_modello_standard.org}{complessità} \(\Sigma_{1}\) che da una formula di complessità \(\Pi_{1}\).
\end{itemize}
\#+END\textsubscript{lem}
\end{document}
