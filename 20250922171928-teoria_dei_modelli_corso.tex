% Created 2026-02-07 Sat 19:30
% Intended LaTeX compiler: pdflatex
\documentclass[10pt]{article}
%% CREATO CON ORG - EMACS
\newcommand{\use}[2][]{\usepackage[#1]{#2}}
% PACCHETTI FONDAMENTLAI
\use[utf8]{inputenc}
\use[T1]{fontenc}
\use{graphicx}
\use{longtable}
\use{wrapfig}
\use{rotating}
\use[normalem]{ulem}
\use{amsmath}
\use{amsthm}
\use{amssymb}

\use{eucal} % Cambia mathcal{...}

\use{capt-of}
\use[italian]{babel}
\use[babel]{csquotes}
% bib la TEX lo carica in automatico org-cite
\use{microtype}
\use{lmodern}
\use{subfig} % sottofigure
\use{multicol} % due colonne
\use{lipsum} % lorem ipsum
\use{color} % colori in latex
\use{parskip} % rimuove l'indentazione dei nuovi paragrafi %% Add parbox=false to all new tcolorbox
\use{centernot}
\use[outline]{contour}\contourlength{3pt}
\use{fancyhdr}
\use{layout}
\use[most]{tcolorbox} % Riquadri colorati
\use{ifthen} % IFTHEN
\use{geometry}

% pacchetti matematica
\use{yhmath}
\use{dsfont}
\use{mathrsfs}
\use{cancel} % semplificare
\use{polynom} %divisione tra polinomi
\use{forest} % grafi ad albero
\use{booktabs} % tabelle
\use{commath} %simboli e differenziali
\use{bm} %bold
\use[fulladjust]{marginnote} %to use marginnote for date notes
\use{arrayjobx}%array
\use[intlimits]{empheq} % Riquadri colorati attorno alle equazioni
\use{mathtools}
\use{circuitikz} % Disegnare i circuiti
\use{mathtools}
\use{stmaryrd} % [[ \llbracket ]] \rrbracket
\use{bussproofs} % dimostrazioni

%%%%%%%%%%%%%


%%%% QUIVER
\newcommand{\duepunti}{\,\mathchar\numexpr"6000+`:\relax\,}
% A TikZ style for curved arrows of a fixed height, due to AndréC.
\tikzset{curve/.style={settings={#1},to path={(\tikztostart)
    .. controls ($(\tikztostart)!\pv{pos}!(\tikztotarget)!\pv{height}!270:(\tikztotarget)$)
    and ($(\tikztostart)!1-\pv{pos}!(\tikztotarget)!\pv{height}!270:(\tikztotarget)$)
    .. (\tikztotarget)\tikztonodes}},
    settings/.code={\tikzset{quiver/.cd,#1}
        \def\pv##1{\pgfkeysvalueof{/tikz/quiver/##1}}},
    quiver/.cd,pos/.initial=0.35,height/.initial=0}

% TikZ arrowhead/tail styles.
\tikzset{tail reversed/.code={\pgfsetarrowsstart{tikzcd to}}}
\tikzset{2tail/.code={\pgfsetarrowsstart{Implies[reversed]}}}
\tikzset{2tail reversed/.code={\pgfsetarrowsstart{Implies}}}
% TikZ arrow styles.
\tikzset{no body/.style={/tikz/dash pattern=on 0 off 1mm}}
%%%%%%%%%%


%% DEFINIZIONI COMANDI MATEMATICI
\let\sin\relax %TOGLIE LA DEFINIZIONE SU "\sin"

% cambia la definizione di empty set
% ---
\let\oldemptyset\emptyset
% ---
% \let\emptyset\varnothing
% ---
% \let\emptyset\relax
% \newcommand{\emptyset}{\text{\textnormal{\O}}}
% ---

\DeclareMathOperator{\bounded}{bd}
\DeclareMathOperator{\sin}{sen}
\DeclareMathOperator{\epi}{Epi}
\DeclareMathOperator{\cl}{cl}
\DeclareMathOperator{\graph}{graph}
\DeclareMathOperator{\arcsec}{arcsec}
\DeclareMathOperator{\arccot}{arccot}
\DeclareMathOperator{\arccsc}{arccsc}
\DeclareMathOperator{\spettro}{Spettro}
\DeclareMathOperator{\nulls}{nullspace}
\DeclareMathOperator{\dom}{dom}
\DeclareMathOperator{\ar}{ar}
\DeclareMathOperator{\const}{Const}
\DeclareMathOperator{\fun}{Fun}
\DeclareMathOperator{\rel}{Rel}
\DeclareMathOperator{\altezza}{ht}
\let\det\relax %TOGLIE LA DEFINIZIONE SU "\det"
\DeclareMathOperator{\det}{det}
\DeclareMathOperator{\End}{End}
\DeclareMathOperator{\gl}{GL}
\def\Id{\mathrm{Id}}
\def\id{\mathrm{id}}
\DeclareMathOperator{\I}{\mathds{1}}
\DeclareMathOperator{\II}{II}
\DeclareMathOperator{\rank}{rank}
\DeclareMathOperator{\tr}{tr}
\DeclareMathOperator{\tc}{t.c.}
\DeclareMathOperator{\T}{T}
\DeclareMathOperator{\var}{Var}
\DeclareMathOperator{\cov}{Cov}
\DeclareMathOperator{\st}{st}
\DeclareMathOperator{\mon}{Mon}
\newcommand{\card}[1]{\left\vert #1 \right\vert}
\newcommand{\trasposta}[1]{\prescript{\text{T}}{}{#1}}
\newcommand{\1}{\mathds{1}}
\newcommand{\R}{\mathds{R}}
\newcommand{\diesis}{\#}
\newcommand{\bemolle}{\flat}
\newcommand{\nonstandard}[1]{\prescript{*}{}{#1}}
\newcommand{\starR}{\nonstandard{\R}}
\newcommand{\borel}{\mathscr{B}}
\newcommand{\lebesgue}[1]{\mathscr{L}\left(#1\right)}
\newcommand{\media}{\mathds{E}}
\newcommand{\K}{\mathds{K}}
\newcommand{\A}{\mathds{A}}
\newcommand{\Q}{\mathds{Q}}
\newcommand{\N}{\mathds{N}}
\newcommand{\C}{\mathds{C}}
\newcommand{\Z}{\mathds{Z}}
\newcommand{\qo}{\hspace{1em}\text{q.o.}\,}
\renewcommand{\tilde}[1]{\widetilde{#1}}
\renewcommand{\parallel}{\mathrel{/\mkern-5mu/}}
\newcommand{\parti}[2][]{\wp_{#1}(#2)}
\newcommand{\diff}[1]{\operatorname{d}_{#1}}
\let\oldvec\vec
\renewcommand{\vec}[1]{\overrightarrow{\vphantom{i}#1}}
\newcommand{\floor}[1]{\left\lfloor #1 \right\rfloor}
\newcommand{\cat}[1]{\mathbf{#1}}
\newcommand{\dfreccia}[1]{\xrightarrow{\ #1 \ }}
\newcommand{\sfreccia}[1]{\xleftarrow{\ #1 \ }}
\newcommand{\formalsum}[2]{{\sum_{#1}^{#2}}{\vphantom{\sum}}'}
\newcommand{\minim}[2]{\mu_{#1}\, \left(#2\right)}
\newcommand{\concat}{\null^{\frown}} % concatenazione di stringe
\newcommand{\godelcode}[1]{\langle\!\langle #1 \rangle\!\rangle}
\newcommand{\godeldec}[1]{(\!(#1)\!)}
\newcommand{\termcode}[1]{\ulcorner #1\urcorner}
\newcommand{\partialto}{\dashrightarrow}
\newcommand{\restricted}{\upharpoonright}
\newcommand{\embeds}{\precsim}
\newcommand{\surjects}{\twoheadrightarrow}
\newcommand{\equipotenti}{\asymp}
%% \newcommand{\dotplus}{\mathbin{\dot{+}}} %% A quanto pare esiste già
\newcommand{\bigdot}{\mathbin{\boldsymbol{\cdot}}}
\newcommand{\dotexp}[1]{^{.#1}}
\newcommand{\conv}{\mathbin{*}}
\newcommand{\convolution}[2]{(#1\conv #2)}
\newcommand{\nil}{\mathfrak{N}}
\newcommand{\divisore}{\mathrel{|}}
\newcommand{\simplesso}[1]{\mathrm{e}_{#1}}

\renewcommand{\iff}{\mathrel{\longleftrightarrow}} %% Notazione Logica.
\newcommand{\oldiff}{\mathrel{\Longleftrightarrow}}
\renewcommand{\implies}{\mathrel{\rightarrow}} %% Notazione Logica
\newcommand{\oldimplies}{\mathrel{\Longrightarrow}}
\renewcommand{\impliedby}{\mathrel{\leftarrow}} %% Notazione Logica
\newcommand{\oldimpliedby}{\mathrel{\Longleftarrow}}

\newcommand{\IFF}{\quad\Longleftrightarrow\quad}
\newcommand{\IMPLICA}{\quad\Longrightarrow\quad}


\renewcommand{\descriptionlabel}[1]{\hspace{\labelsep}\normalfont #1} % remove bold from description


%% Definizione di Divergenza di K-L

\DeclarePairedDelimiterX{\infdivx}[2]{(}{)}{%
  #1\;\delimsize\|\;#2%
}
\newcommand{\kldiv}{D_{KL}\infdivx}

%% Definizione di \dotminus

\makeatletter
\newcommand{\dotminus}{\mathbin{\text{\@dotminus}}}

\newcommand{\@dotminus}{%
  \ooalign{\hidewidth\raise1ex\hbox{.}\hidewidth\cr$\m@th-$\cr}%
}
\makeatother

%tramite i prossimi due comandi posso decidere come scrivere i logaritmi naturali in tutti i documenti: ho infatti eliminato qualsiasi differenza tra "ln" e "log": se si vuole qualcosa di diverso bisogna inserire manualmente il tutto
\let\ln\relax
\DeclareMathOperator{\ln}{ln}
\let\log\relax
\DeclareMathOperator{\log}{log}
%%%%%%

%% NUOVI COMANDI
\newcommand{\straniero}[1]{\textit{#1}} %parole straniere
\newcommand{\titolo}[1]{\textsc{#1}} %titoli
\newcommand{\qedd}{\tag*{$\blacksquare$}} %qed per ambienti matemastici
\renewcommand{\qedsymbol}{$\blacksquare$} %modifica colore qed
\newcommand{\ooverline}[1]{\overline{\overline{#1}}}
\newcommand{\circoletto}[1]{\left(#1\right)^{\text{o}}}
%
\newcommand{\qmatrice}[1]{\begin{pmatrix}
#1_{11} & \cdots & #1_{1n}\\
\vdots & \ddots & \vdots \\
#1_{m1} & \cdots & #1_{mn}
\end{pmatrix}}
%
\newcommand{\parentesi}[2]{%
\underset{#1}{\underbrace{#2}}%
}
%
\newcommand{\norma}[1]{% Norma
\left\lVert#1\right\rVert%
}
\newcommand{\scalare}[2]{% Scalare
\left\langle #1, #2\right\rangle
}
%%%%%

%% RESTRIZIONI
\newcommand{\referenze}[2]{
        \phantomsection{}#2\textsuperscript{\textcolor{blue}{\textbf{#1}}}
}

\let\restriction\relax

\def\restriction#1#2{\mathchoice
              {\setbox1\hbox{${\displaystyle #1}_{\scriptstyle #2}$}
              \restrictionaux{#1}{#2}}
              {\setbox1\hbox{${\textstyle #1}_{\scriptstyle #2}$}
              \restrictionaux{#1}{#2}}
              {\setbox1\hbox{${\scriptstyle #1}_{\scriptscriptstyle #2}$}
              \restrictionaux{#1}{#2}}
              {\setbox1\hbox{${\scriptscriptstyle #1}_{\scriptscriptstyle #2}$}
              \restrictionaux{#1}{#2}}}
\def\restrictionaux#1#2{{#1\,\smash{\vrule height .8\ht1 depth .85\dp1}}_{\,#2}}
%%%%%%%%%%%

%%% FORMATTAZIONE FOOTNOTEMARK

\def\footnotemarkformatting#1{[#1]}
\renewcommand{\thefootnote}{\footnotemarkformatting{\arabic{footnote}}}

%% SEZIONE GRAFICA
\use{tikz}
\usetikzlibrary{matrix, patterns, calc, decorations.pathreplacing, hobby, decorations.markings, decorations.pathmorphing, babel}
\use{tikz-3dplot}
\use{mathrsfs} %per geogebra
\use{tikz-cd}
\tikzset
{
  %surface/.style={fill=black!10, shading=ball,fill opacity=0.4},
  plane/.style={black,pattern=north east lines},
  curve/.style={black,line width=0.5mm},
  dritto/.style={decoration={markings,mark=at position 0.5 with {\arrow{Stealth}}}, postaction=decorate},
  rovescio/.style={decoration={markings,mark=at position 0.5 with {\arrow{Stealth[reversed]}}}, postaction=decorate}
}
\use{pgfplots} % stampare le funzioni
        \pgfplotsset{/pgf/number format/use comma,compat=1.15}
        %\pgfplotsset{compat=1.15} %per geogebra
        \usepgfplotslibrary{fillbetween, polar}
%%%%%%

%% CITAZIONI
\use{lineno}

\newcommand{\citazione}[1]{%
  \begin{quotation}
  \begin{linenumbers}
  \modulolinenumbers[5]
  \begingroup
  \setlength{\parindent}{0cm}
  \noindent #1
  \endgroup
  \end{linenumbers}
  \end{quotation}\setcounter{linenumber}{1}
  }
%%%%%%

%%%%%%%%%%%%%%%%%%%%%%%%%%%%%%%%%%%%%%%%%%%%
%%%%%%%%%%%%%%%%%%%%%%%%%%%%%%%%%%%%%%%%%%%%

%% AMS THM

\theoremstyle{definition}% default
\newtheorem{thm}{Teorema}[section]
\newtheorem{lem}[thm]{Lemma}
\newtheorem{prop}[thm]{Proposizione}
\newtheorem{cor}[thm]{Corollario}
\newtheorem{esempio}[thm]{Esempio}
\theoremstyle{plain}
\newtheorem{definizione}[thm]{Definizione}
\theoremstyle{remark}
\newtheorem*{oss}{Osservazione}


%%%%%%%%%%%%%%%%%%%%%%%%%%%%%%%%%%%%%%%%%%%%
%%%%%%%%%%%%%%%%%%%%%%%%%%%%%%%%%%%%%%%%%%%%

\use{hyperref}
\hypersetup{%
        pdfauthor={Davide Peccioli},
        pdfsubject={},
        allcolors=black,
        citecolor=black,
%	colorlinks=true,
        bookmarksopen=true}
\setcounter{secnumdepth}{0} % rimuove i numeri di sezione senza rimuovere le ref
\renewcommand{\href}[2]{\textcolor{blue}{#2}} % disabilita il comando href
\use{enotez} %
\setenotez{%
 mark-format = \footnotemarkformatting % Mette i numeri tra parentesi quadre%
}\let\footnote=\endnote % rende tutte le note a pié pagina come delle note a fine file 


\let\olddocument\document % modifico l'ambiende documenti per non dover stampare \printendnote
\let\oldenddocument\enddocument
\renewenvironment{document}%
{%
  \olddocument
}{%
  \printendnotes\oldenddocument
}
\renewcommand{\thethm}{\arabic{thm}}

\usepackage[hyperref]{biblatex}
\addbibresource{~/Documents/org/roam/bib/master.bib}
\def\U{\mathcal{U}}
\def\L{\mathcal{L}}
%
\def\acl{\operatorname{acl}}
\def\eq{{\rm eq}}
\def\Ueq{\U^\eq}
%
\def\orbita{\mathcal{O}}
\def\Aut{\operatorname{Aut}}
%
\def\extD{\mathscr{D}}
%
\def\tc{\mid}
\def\tp{\operatorname{tp}}
\def\EMtp{\operatorname{EM}\text{-}\operatorname{tp}}
\def\<{\langle}
\def\>{\rangle}
%
\def\restricted#1{\,\mathord{\upharpoonright}{{\scriptstyle #1}}}
\def\equivalentover#1{\mathrel{\equiv_{ #1 }}}
%% NON FORKING
\def\nonforkSymbol{\mathbin{\raise1.8ex\rlap{\kern0.6ex\rule{0.6ex}{0.1ex}}\rlap{\kern1.1ex\rule{0.1ex}{1.9ex}}\raise-0.3ex\hbox{$\smile$}}}
\def\defaultnonforkmodel{M}
\def\nonfork{\nonforkSymbol}
\renewcommand{\nonfork}[1][\defaultnonforkmodel]{%
\mathrel{\nonforkSymbol_{#1}}}
\def\X{\mathcal{X}}
\def\Z{\mathcal{Z}}
\def\DD{\mathscr{D}}
\def\CC{\mathscr{C}}
\def\strongSigma{\vphantom{\Sigma}^\text{s}\Sigma}
\def\X{\mathcal{X}}
\def\Z{\mathcal{Z}}
\def\V{\mathcal{V}}
\author{Davide Peccioli}
\date{\today}
\title{Teoria dei modelli [CORSO]}
\begin{document}

\maketitle
\textbf{\textbf{NOTAZIONE IMPORTANTE}}: Tutte le mappe sono parziali, se non diversamente indicato.

Le dispense di questo corso sono \autocite{zambellaCrecheCourseModel2024}
\section{Istruzioni Esame}
\label{sec:org3b14dda}

\begin{itemize}
\item Esercizi settimanali da consegnare entro un paio di settimane.
\item Parte da dove abbiamo smesso in Istituzioni
\item Prima di scrivere bene l'esercizio chiediamo a lui che lo sketch di dimostrazione sia giusto. Inoltre, se non riusciamo lui ci dà anche i suggerimenti. (Questo garantisce un minimo di 28)
\item Gli esercizi POTREBBERO essere sbagliati.
\item Alla fine vuole fare un esame in cui si fanno gli esercizi fatti dagli altri durante il corso.
\end{itemize}
\section{Lezione 1 - \textit{<2025-09-23 Tue>}}
\label{sec:org5bac8a3}

\subsection{Ripasso}
\label{sec:org433ec68}

\subsubsection{Saturazione e omogeneità}
\label{sec:org179ed99}

Fissiamo un modello saturo \(\mathcal{U}\) di cardinalità \(\kappa>\card{\mathcal{L}}+\omega\).

\begin{itemize}
\item \uline{Saturo}: realizza tutti i tipi \(p(x)\) tali che
\begin{enumerate}
\item \(p(x) \subseteq \mathcal{L}(A)\) per qualche \(A \subseteq \mathcal{U}\) di cardinalità piccola \(\card{A} <\kappa\).
\item \(p(x)\) è finitamente consistente (in \(\mathcal{U}\)).
\end{enumerate}
\item \(\mathcal{U}\) è detto \uline{modello mostro}.
\item Se \(\kappa\) è inaccessibile ogni teoria \(T\) con un modello infinito ha un modello mostro di cardinalità \(\kappa\); \uline{nota}: due modelli mostro della stessa cardinalità sono isomorfi.
\item Nota: se non si vogliono usare gli inaccessibili possiamo richiedere che \(\mathcal{U}\) sia \(\kappa\)-saturo e \(\kappa\)-omogeneo e di cardinalità arbitraria.
\end{itemize}

Alcuni fatti importanti:
\begin{enumerate}
\item Sono fatti equivalenti:
\begin{itemize}
\item \(\mathcal{U}\) è saturo;
\item \(\mathcal{U}\) realizza tutti i tipi finitamente consistenti con \(<\kappa\) parametri e \(\le\kappa\) variabili.
\end{itemize}
\item Se \(\mathcal{U}\) è saturo allora \(\mathcal{U}\) è omogeneo: se \(f:\mathcal{U}\to\mathcal{U}\) elementare e \(\card{f}<\kappa\) allora esiste \(h \in \operatorname{Aut}(\mathcal{U})\) tale che \(f \subseteq h\).
\end{enumerate}

\begin{definizione}
\(M\) è \uline{debolmente \(\lambda\)-saturo} se realizza \(p(x) \subseteq \mathcal{L}(\emptyset)\), finitamente consistente in \(M\), con \(\card{x}\le \lambda\).

\(M\) si dice \uline{debolmente saturo} se è debolmente \(\card{M}\)-saturo.
\end{definizione}

\begin{prop}
Sono fatti equivalenti:
\begin{enumerate}
\item \(\mathcal{U}\) è saturo;
\item \(\mathcal{U}\) è omogeneo e debolmente saturo\footnote{Essere \uline{debolmente saturo} significa essere universale nella categoria di modelli e morfismi ricchi.}.
\end{enumerate}
\end{prop}

\begin{oss}
Se \(N \equiv \mathcal{U}\) è un modello e \(c\) enumerazione di \(N\), il tipo \(\operatorname{tp}(c) =p(x)\) è finitamente consistente in \(\mathcal{U}\); ogni realizzazione \(b\vDash p(x)\) enumera un modello isomorfo a \(N\). (Lemma del diagramma elementare)
\end{oss}

\begin{definizione}
\(M\) è \uline{debolmente \(\lambda\)-omogeneo} se per ogni \(f:M\to M\) elementare, \(\card{f}<\lambda\), per ogni \(b \in M\) esiste \(c \in M\) tale che \(f\cup\set{\langle b,c\rangle }: M\to M\) elementare. \(M\) si dice anche \uline{back and forth \(\lambda\)-omogeneo}.
\end{definizione}

\begin{prop}
Sono fatti equivalenti:
\begin{enumerate}
\item \(\mathcal{U}\) è \(\lambda\)-saturo;
\item \(\mathcal{U}\) è debolmente \(\lambda\)-omogeneo e debolmente \(\lambda\)-saturo.
\end{enumerate}
\end{prop}
\subsubsection{\(\mathcal{U}\) come spazio topologico}
\label{sec:org104eb4b}

Sia \(A \subseteq \mathcal{U}\) piccolo.

\begin{definizione}
La \uline{\(A\)-topologia} su \(\mathcal{U}^{x}\) è la topologia generata dalla seguente base di clopen:
\begin{equation*}
\set{\varphi(\mathcal{U}^{x})\mid \varphi(x)\in \mathcal{L}(A)}.
\end{equation*}
\end{definizione}

\begin{oss}
La \(A\)-topologia è compatta per saturazione di \(\mathcal{U}\).
\end{oss}

\begin{prop}
Per ogni \(p(x) \subseteq \mathcal{L}(A)\), \(q(z) \subseteq \mathcal{L}(B)\) tali che \(p(\mathcal{U}^{x})\cap q(\mathcal{U}^{x}) = \emptyset\).

Allora esistono:
\begin{itemize}
\item \(\varphi(x)\) congiunzione di formule di \(p(x)\);
\item \(\psi(x)\) congiunzione di formule di \(q(x)\);
\end{itemize}
tali che \(\varphi(\mathcal{U}^{x})\cap \psi(\mathcal{U}^{x}) = \emptyset\).
\label{prop:Fatto5}
\end{prop}

\begin{oss}
Si noti che \(p(\mathcal{U}^{x}) \subseteq \varphi(\mathcal{U}^{x})\) e \(q(\mathcal{U}^{x}) \subseteq \psi(\mathcal{U}^{x})\).

Nel caso \(A=B\), la Proposizione~\ref{prop:Fatto5} afferma che la \(A\)-topologia è normale.
\end{oss}
\subsubsection{Definibilità e Automorfismi}
\label{sec:org1f528df}

\begin{prop}
Sia \(\varphi(x;z) \in \mathcal{L}\), \(b \in \mathcal{U}^{z}\) e \(f \in \operatorname{Aut}(\mathcal{U})\):
\begin{align*}
f\big[\varphi(\mathcal{U}^{x},b)\big] &= \set{fa\mid a \in \varphi(\mathcal{U}^{x},b)} = \set{fa\mid a \in \mathcal{U}^{x}, \varphi(a,b)}\\
&= \set{fa\mid a \in \mathcal{U}^{x}, \varphi(fa,fb)} = \set{fa\mid fa \in \mathcal{U}^{x}, \varphi(fa,fb)}\\
&= \set{a'\mid a' \in \mathcal{U}^{x}, \varphi(a',fb)} = \varphi(\mathcal{U}^{x},fb)
\end{align*}
\end{prop}

Ovvero gli automorfismi mappano insiemi definibili in insiemi definibili, con stessa formula e parametro diverso.
\subsubsection{Definibilità e invarianza}
\label{sec:org5796991}

\begin{thm}
Sia \(\mathcal{D} \subseteq \mathcal{U}^{x}\) insieme definibile. Sono fatti equivalenti:
\begin{enumerate}
\item \(\mathcal{D}\) è \uline{invariante su \(A\)}\footnote{Ovvero per ogni \(f \in \operatorname{Aut}(\mathcal{U}/A)\), \(f[\mathcal{D}]=\mathcal{D}\).};
\item \(\mathcal{D}=\varphi(\mathcal{U}^{x})\) per qualche \(\varphi(x) \in \mathcal{L}(A)\).
\end{enumerate}
\end{thm}

\begin{thm}
Sia \(\mathcal{D} \subseteq \mathcal{U}^{x}\) insieme tipo-definibile\footnote{Ovvero definibile da \(p(x) \subseteq \mathcal{L}(\mathcal{U})\)  con insieme di parametri piccolo: \(\mathcal{D}=p(\mathcal{U}^{x})\).}. Sono fatti equivalenti:
\begin{enumerate}
\item \(\mathcal{D}\) è \uline{invariante su \(A\)};
\item \(\mathcal{D}=p(\mathcal{U}^{x})\) per qualche \(p(x) \subseteq \mathcal{L}(A)\).
\end{enumerate}
\end{thm}
\subsection{Preservation Theorem}
\label{sec:org551e94c}

Sia \(\mathcal{L}\) un \href{20250130162057-linguaggio_del_prim_ordine.org}{linguaggio del prim'ordine}, \(T\) una \(\mathcal{L}\)-\href{20250130114950-teoria_del_prim_ordine.org}{teoria} senza \href{20250131122945-modello_di_un_insieme_di_formule.org}{modelli} finiti.

Sia \(\Delta\) un insieme di formule chiuso per sostituzione di variaibli e che contenga la formula ``\(x=y\)''.

Se \(C \subseteq \set{\forall , \exists , \land,\lor,\lnot}\) è un insieme di connettivi, \(C\Delta\) denota la chiusura di \(\Delta\) per i connettivi in \(C\). \(\Delta^{\pm} \coloneqq \set{\lnot}\Delta\).

\begin{prop}
Sia \(M\vDash T\), \(b \in M^{x}\) e
\begin{equation*}
q(x) \coloneqq \Delta\text{-}\operatorname{tp}_{M}(b) \coloneqq \set{\psi(x) \in \Delta\mid M\vDash \psi(b)}.
\end{equation*}
Per ogni \(\varphi(x) \in \mathcal{L}\) sono fatti equivalenti:
\begin{enumerate}
\item per ogni \(\Delta\)-morfismo\footnote{\(k:M \xrightarrow{\Delta} N\) mappa parziale si dice \(\Delta\)-morfismo se per ogni \(a \in \operatorname{dom}(k)^{x}\) e per ogni \(\varphi(x) \in \Delta\)
\begin{equation*}
M\vDash\varphi(a)\,\implies\, N\vDash\varphi(ka)
\end{equation*}} \(k:M \xrightarrow{\Delta} N\), \(N\vDash T\) e \(b \in \operatorname{dom}(k)\)
\begin{equation*}
 N\vDash\varphi(kb).
\end{equation*}
\item \(T\vdash q(x)\implies\varphi(x)\)\footnote{Questa scrittura significa che per ogni \(N\vDash T\) si ha \(q(N^{x}) \subseteq \varphi(N^{x})\)}
\end{enumerate}
\end{prop}
\begin{proof}
(\(2.\Rightarrow 1.\)): Poiché \(M,b\vDash q(x)\), allora \(N,kb\vDash q(x)\), e quindi \(N\vDash\varphi(kb)\).

(\(1.\Rightarrow 2.\)): Sia per assurdo \(N\vDash T\) tale che \(N,c\vDash q(x) \land \lnot\varphi(x)\).

Allora, detto \(k = \set{\langle b,c\rangle}\)\footnote{Se \(b=\langle b_{i}\mid i<\card{x}\rangle\), \(c=\langle c_{i}\mid i<\card{x}\rangle\)
\begin{equation*}
\set{\langle b,c\rangle}\coloneqq\set{\langle b_{i},c_{i}\rangle\mid i<\card{x}}.
\end{equation*}}, si ha che \(k:M\to N\) è \(\Delta\)-morfismo, poiché \(N,kb=c\vDash q(x)\); ma \(N\vDash \lnot\varphi(kb)\). Assurdo.
\end{proof}

\begin{thm}
(Lyndon-Robinson Lemma). Sia \(\varphi(x) \in \mathcal{L}\). Sono fatti equivalenti:
\begin{enumerate}
\item \(\varphi(x)\) è equivalente ad una \(\set{\land ,\lor }\Delta\)-formula;
\item \(\varphi(x)\) è preservata da \(\Delta\)-morfismi.
\end{enumerate}
\end{thm}
\begin{proof}
(\(1.\Rightarrow 2.\)): Ovvia.

(\(2.\Rightarrow 1.\)): Mostriamo che
\begin{equation*}
T\vdash \varphi(x) \iff \bigvee \set{p(x) \subseteq \Delta: T\vdash p(x)\implies \varphi(x)}
\end{equation*}

L'implicazione \(\impliedby\) è ovvia, mentre per \(\implies\), se \(M,b\vDash T \land \varphi(x)\), sia \(q(x)=\Delta\text{-}\operatorname{tp}_{M}(b) \subseteq\Delta\). Per la proposizione precedente,
\begin{equation*}
T\vdash q(x)\implies \varphi(x).
\end{equation*}

Quindi
\begin{align*}
T\vdash \varphi(x) &\iff \bigvee \set{p(x) \subseteq \Delta: T\vdash p(x)\implies \varphi(x)}\\
&\iff \bigvee \set{\psi(x) \subseteq \set{\land}\Delta: T\vdash \psi(x)\implies \varphi(x)}\\
&\iff \psi_{1}(x) \lor \dots \lor \psi_{n}(x).\qedhere
\end{align*}
\end{proof}
\section{Lezione 2 - \textit{<2025-09-24 Wed>}}
\label{sec:orge2cd651}

\subsection{Preservation Theorem}
\label{sec:org0e329bb}

\begin{cor}
(Fatto 1). Sono fatti equivalenti:
\begin{enumerate}
\item \(T\) ha l'eliminazione dei quantificatori\footnote{Per ogni \(\varphi(x) \in \mathcal{L}\) esiste \(\psi(x) \in \mathcal{L}\) \uline{senza quantificatori} tale che
\begin{equation*}
T\vdash \forall x\ \varphi(x)\iff\psi(x)
\end{equation*}};
\item ogni \href{20250214120959-mappe_tra_strutture_del_prim_ordine.org}{isomorfismo parziale} tra \href{20250131122945-modello_di_un_insieme_di_formule.org}{modelli} di \(T\) è una \href{20250214120959-mappe_tra_strutture_del_prim_ordine.org}{morfismo elementare}
\end{enumerate}
\end{cor}
\begin{proof}
(\(1.\Rightarrow 2.\)): Ovvio.

(\(2.\Rightarrow 1.\)): Segue banalmente dal Teorema precedente. (\href{20251020153459-lyndon_robinson_lemma.org}{Lyndon-Robinson Lemma})
\end{proof}

\begin{cor}
(Fatto 1.5). Sono fatti equivalenti:
\begin{enumerate}
\item \(T\) ha \(\Delta\)-eliminazione positiva dei quantificatori\footnote{Per ogni \(\varphi(x) \in \set{\exists ,\forall ,\land , \lor}\!\Delta\) esiste \(\psi(x) \in \set{\land ,\lor}\Delta\) tale che
\begin{equation*}
T\vdash \forall x\ \varphi(x)\iff\psi(x)
\end{equation*}};
\item ogni \(\Delta\)-morfismo tra modelli di \(T\) è un \(\set{\land ,\lor}\Delta\)-morfismo.
\end{enumerate}
\end{cor}

\begin{prop}
(Fatto 2). Siano \(M,N\) due \(\mathcal{L}\)-strutture, e sia \(N\) \(\omega\)-saturo, e sia \(k:M\to N\) un \(\Delta\)-morfismo con \(\card{k}<\omega\). Sono fatti equivalenti:
\begin{enumerate}
\item \(k\) è un \(\set{\exists, \land }\!\Delta\)-morfismo;
\item per ogni \(b \in M^{<\omega}\) esiste \(c \in N^{<\omega}\) tale che \(k\cup\set{\langle b,c\rangle}\) è un \(\Delta\)-morfismo.
\end{enumerate}
\end{prop}
\begin{proof}
(\(1.\Rightarrow 2.\)): Sia \(a\) una enumerazione di \(\dom k\), e sia \(p(x,y) = \Delta\text{-}\operatorname{tp}(a,b)\). Per 1., \(p(ka,y)\) è finitamente consistente in \(N\) (in quanto considero la congiunzione di formule di \(p(x,y)\) e ne prendo un esiste).

Siccome \(k\) è finita e \(N\) è \(\omega\)-saturo, allora esiste \(c \in N\) tale che \(N,c\vDash p(ka,y)\). Tale \(c\) è quello cercato.

(\(2.\Rightarrow 1.\)): Ogni \(\varphi(x) \in \set{\exists ,\land}\!\Delta\) è della forma \(\exists \overline{y}\,\psi(x,\overline{y})\), con \(\psi(x,\overline{y}) \in \set{\land}\!\Delta\) e \(\card{\overline{y}} < \omega\).

Sia \(a \in \dom(k)^{x}\) tale che \(M\vDash\varphi(a)\). Voglio mostrare \(N\vDash\varphi(ka)\).

Sia \(\overline{b} \in N^{\overline{y}}\) tale che \(M\vDash \psi(a,\overline{b})\); \textbf{FINIRE}
\end{proof}

\begin{prop}
(Fatto 3). Siano \(M,N\) due \(\mathcal{L}\)-strutture, e sia \(N\) \(\omega\)-saturo, e sia \(k:M\to N\) un \(\Delta\)-morfismo con \(\card{k}<\omega\). Sono fatti equivalenti:
\begin{enumerate}
\item \(k\) è un \(\set{\forall , \lor }\!\Delta\)-morfismo;
\item per ogni \(c \in N^{<\omega}\) esiste \(b \in M^{<\omega}\) tale che \(k\cup\set{\langle b,c\rangle}\) è un \(\Delta\)-morfismo.
\end{enumerate}
\end{prop}

\begin{cor}
Siano \(M,N\) due \(\mathcal{L}\)-strutture, con \(N\) \(\omega\)-saturo,  \(\card{M}=\omega\) e \(k:M\to N\) un \(\Delta\)-morfismo, \(\card{k}<\omega\). Sono fatti equivalenti:
\begin{enumerate}
\item \(k\) è un \(\set{\exists ,\land}\!\Delta\)-morfismo.
\item \(k\) si estende ad un \(\Delta\)-morfismo totale.
\end{enumerate}
\end{cor}

\begin{cor}
Siano \(M,N\) due \(\mathcal{L}\)-strutture, con \(N\) \(\omega\)-saturo,  \(\card{M}=\omega\) e \(k:N\to M\) un \(\Delta\)-morfismo, \(\card{k}<\omega\). Sono fatti equivalenti:
\begin{enumerate}
\item \(k\) è un \(\set{\forall ,\lor}\!\Delta\)-morfismo.
\item \(k\) si estende ad un \(\Delta\)-morfismo suriettivo (parziale).
\end{enumerate}
\end{cor}

\begin{thm}
(Teorema 1). Sia \(\varphi(x) \in \mathcal{L}\). Sono fatti equivalenti:
\begin{enumerate}
\item esiste \(\psi(x) \in \set{\exists ,\land }\!\Delta\) tale che \(T\vdash \forall x\ (\varphi(x)\iff\psi(x))\);
\item \(\varphi(x)\) è preservata da \(\Delta\)-morfismi totali tra modelli di \(T\).
\end{enumerate}
\end{thm}

\begin{cor}
Sono fatti equivalenti:
\begin{enumerate}
\item \(T\) ha \(\Delta\)-eliminazione positiva dei quantificatori;
\item ogni \(\Delta\)-morfismo tra modelli di \(T\) è sia un \(\set{\exists ,\land }\!\Delta\)-morfismo che un \(\set{\forall , \lor}\!\Delta\)-morfismo.
\end{enumerate}
\end{cor}
\begin{proof}
(\(1.\Rightarrow 2.\)): Ovvia.

(\(2.\Rightarrow 1.\)): Per induzione sulla sintassi. Assumiamo 2. e che \(\varphi(x,y)\) sia preservata da \(\Delta\)-morfismi. Mostriamo che la verità di \(\exists y\ \varphi(x;y)\) è conservata.

Per ipotesi induttiva \(\varphi(x,y)\) è equivalente a \(\psi(x,y) \in \set{\land ,\lor}\!\Delta\), ed inoltre \(\exists y\ \varphi(x;y)\) è equivalente a \(\exists y\ \psi(x,y) \in \set{\exists , \lor,\land }\Delta = \set{\exists , \land }\Delta\). ???????

Per 2. \(\exists y\ \psi(x;y)\) è preservata dai \(\Delta\)-morfismi. Idem per \(\forall y\ \psi(x;y)\).

Induzione ovvia per \(\land ,\lor\).
\end{proof}
\subsection{Back-and-forth conditions for QE}
\label{sec:org526cfa5}

\begin{prop}
(Fatto 4). Sono fatti equivalenti
\begin{enumerate}
\item \(T\) ha \(\Delta\)-eliminazione positiva dei quantificatori;
\item per ogni \(k:M\to N\) \(\Delta\)-morfismo finito tra modelli \(\omega\)-saturi di \(T\):
\begin{itemize}
\item per ogni \(b \in M\) esiste \(c \in N\) tale che \(k\cup\set{\langle b,c\rangle}\) è un \(\Delta\)-morfismo;
\item per ogni \(c \in N\) esiste \(b \in N\) tale che \(k\cup\set{\langle b,c\rangle}\) è un \(\Delta\)-morfismo.
\end{itemize}
\end{enumerate}
\end{prop}
\subsection{Model-completeness}
\label{sec:org254760b}

\begin{definizione}
\(T\) è \(\Delta\)-model-completo se ogni \(\Delta\)-morfismo totale \(k:M\to N\) tra modelli di \(T\) è un \(\set{\exists ,\forall ,\land ,\lor}\!\Delta\)-morfismo
\end{definizione}

\begin{prop}
(Fatto 5). Sono fatti equivalenti:
\begin{enumerate}
\item \(T\) è \(\Delta\)-model-completo;
\item \(T\) ha \(\set{\exists, \land  }\!\Delta\)-eliminazione positiva dei quantificatori.
\end{enumerate}
\end{prop}
\begin{proof}
???
\end{proof}

\begin{prop}
(Fatto 6). Si
\end{prop}

\uline{Esercizio}: \(T\) è model completa sse per ogni \(M\vDash T\) la teoria \(T\cup\Delta\text{-}\operatorname{Diag}(M)\)\footnote{Si definisce
\begin{equation*}
\Delta\text{-}\operatorname{Diag}(M) \coloneqq \set{\varphi(x)\mid \varphi(x) \in \Delta, a \in M^{x}}
\end{equation*}} è una \(\mathcal{L}(M)\)-teoria completa.
\section{Lezione 3 - \textit{<2025-09-30 Tue>}}
\label{sec:orgf370d5f}

Sia \(T\) una teoria completa con un modello infinito (ovvero senza modelli finiti). Sia \(\mathcal{U}\vDash T\) modello mostro di cardinalità \(\kappa>\card{\mathcal{L}}+\omega\).

\uline{Convenzione per questa lezione}: tutti i tipi \(p(x)\) sono:
\begin{itemize}
\item \(p(x) \subseteq \mathcal{L}(\mathcal{U})\);
\item \(\card{p(x)}<\kappa\) (ovvero i tipi sono piccoli).
\end{itemize}

\begin{definizione}
Un tipo \(p(x)\) si dice \uline{isolato} (da \(\varphi\)) se esiste \(\varphi(x) \in \mathcal{L}(\mathcal{U})\) consistente tale che \(\varphi(x) \implies p(x)\) ovvero \(\varphi(x)\implies \psi(x)\) per ogni \(\psi(x) \in p(x)\) ovvero \(\varphi(\mathcal{U}^{x}) \subseteq p(\mathcal{U}^{x})\).
\end{definizione}
\begin{definizione}
Un tipo \(p(x)\) si dice \uline{isolato da \(\Delta\)} (con \(\Delta\) insieme di formule) se esiste \(\varphi(x) \in \Delta\) che isola \(p(x)\).

Un tipo \(p(x)\) si dice \uline{isolato su \(A \subseteq\mathcal{U}\)} se è isolato da \(\mathcal{L}(A)\).
\end{definizione}

\begin{oss}
Per ogni \(p(x) \subseteq \mathcal{L}(A)\) consistente, esiste un modello \(M\supseteq A\) che realizza \(p(x)\)\footnote{Ovvero esiste \(c \in M\) tale che \(M,c\vDash p(x)\)}.
\end{oss}

\uline{Domanda 1}: dato \(p(x) \subseteq \mathcal{L}(A)\) consistente, esiste un modello \(M\supseteq A\) che \uline{non} realizza \(p(x)\)?

\uline{Risposta banale}: se \(p(x)\) è isolato da \(A\) allora NO. Infatti, se \(\varphi(x) \implies p(x)\) per qualche \(\varphi(x) \in \mathcal{L}(A)\), siccome ogni \(M\supseteq A\) ha una soluzione di \(\varphi(x)\), questa realizza \(p(x)\) in \(M\).

\begin{oss}
Se \(a\vDash p(x)\) allora la formula \(x=a\) isola \(p(x)\).
\end{oss}
\begin{prop}
Sia \(p(x) \subseteq \mathcal{L}(M)\). Sono fatti equivalenti:
\begin{enumerate}
\item \(M\) realizza \(p(x)\);
\item \(p(x)\) è isolato su \(M\).
\end{enumerate}
\end{prop}

\uline{Domanda 2}: dato \(p(x) \subseteq \mathcal{L}(A)\) consistente non isolato da \(A\), esiste \(M\supseteq A\) che non realizza \(p(x)\)? La riposta è NO.

\begin{thm}
(Teorema di omissione dei tipi). Se \(\card{\mathcal{L}(A)}\ge \omega\) allora per ogni \(p(x) \subseteq \mathcal{L}(A)\) non isolato\footnote{Quando è chiaro dal contesto, non si dice ``non isolato da \(A\)'', ma semplicemente ``non isolato''.} esiste \(M\supseteq A\) che omette \(p(x)\).
\label{thm:OTT}
\end{thm}
In inglese questo è l'\emph{Omitting Type Theorem} (OTT).

\begin{thm}
Sia \(\mathcal{L}(A)\) numerabile, \(p(x) \subseteq \mathcal{L}(A)\). Sono fatti equivalenti.
\begin{enumerate}
\item \(p(x)\) è realizzato in ogni modello \(M\supseteq A\);
\item \(p(x)\) è isolato su \(A\).
\end{enumerate}
\end{thm}
\begin{proof}
(\(2.\Rightarrow 1.\)): questo è ovvio, vedi la Domanda~1.

(\(1.\Rightarrow 2.\)): Questo è il Teorema~\ref{thm:OTT}. Si dimostra \(\lnot 2.\Rightarrow \lnot 1.\): sia \(p(x) \subseteq \mathcal{L}(A)\) non isolato e costruiamo un modello \(M\supseteq A\) che omette \(p(x)\). Si vuole applicare una costruzione simile a quella di \href{20250212115524-teorema_di_lowenheim_skolem_all_ingiu.org}{LS all'ingiù}.

Si costruisce \(\langle A_{i}:i<\omega\rangle\) per ricorsione.
\begin{itemize}
\item Sia \(A_{0}\coloneqq A\).
\item Data \(\varphi(y) \in \mathcal{L}(A_{0})\) consistente, \(\card{y}=1\), prendiamo \(b\vDash \varphi(y)\) tale che \(A_{0}\cup\set{b}\) non isola \(p\).

Tale \(b\) esiste per il lemma successivo.

Definiamo \(A_{1}\coloneqq A_{0}\cup\set{b}\)
\end{itemize}
\end{proof}

\begin{lem}
Sia \(\mathcal{L}(A)\) numerabile, \(p(x) \subseteq \mathcal{L}(A)\) non isolato, \(\varphi(y) \in \mathcal{L}(A)\) consistente con \(\card{y} =1\). Allora esiste \(b\vDash \varphi(y)\) e \(A\cup\set{b}\) non isola \(p(x)\).
\label{lem:kurulma}
\end{lem}
\begin{proof}
Costruiamo in \(\omega\) passi un tipo \(q(y)\) consistente che ``descrive'' le proprietà richieste a \(b\).

\uline{Passo 0}: \(q_{0}(y)\coloneqq \set{\varphi(y)}\).

\uline{Passo 1}: Sia \(\xi(x,y) \in \mathcal{L}(A)\). Voglio che ogni \(b\vDash q(y)\) sia tale che \(\lnot\forall x\ \big(\xi(x,b) \implies p(x)\big)\).

Vorrei poter porre \(q_{1}(y) \coloneqq q_{0}(y)\cup\set{\exists x\, \big[\xi(x,y) \land \lnot p(x)\big]}\). Purtroppo quella non è una formula. Allora cerco un \(\psi(x) \in p(x)\) tale che
\begin{equation*}
q_{1}(y) \coloneqq q_{0}(y)\cup \set{\exists x\, \big[\xi(x,y) \land \lnot \psi(x)\big]}
\end{equation*}
sia consistente.

Provo con tutte le \(\psi(x) \in p(x)\) fino a che non ne trovo uno. Se per assurdo non esistesse, allora per
\begin{equation*}
q_{0}(y)\cup \set{\exists x\, \big[\xi(x,y) \land \lnot \psi(x)\big]}
\end{equation*}
è inconsistente per ogni \(\psi(x) \in p(x)\), ovvero \(q_{0}(y) \land \exists x\, \big[\xi(x,y) \land \lnot \psi(x)\big]\) è inconsistente\footnote{Ad ogni passo \(q_{n}(y)\) è un insieme finito di formule, quindi posso trattarlo come una formula (la congiunzione di tutti i suoi elementi).} per ogni \(\psi(x) \in p(x)\), ovvero
\begin{align*}
&\forall y\, \lnot\bigg[q_{0}(y) \land \exists x\, \big[\xi(x,y) \land \lnot \psi(x)\big]\bigg]\\
&\forall y\,\bigg[\lnot q_{0}(y) \lor \forall x\, \big[\lnot\xi(x,y) \lor \psi(x)\big]\bigg]\\
&\forall y\,\bigg[\lnot q_{0}(y) \lor \forall x\, \big[\xi(x,y) \implies \psi(x)\big]\bigg]\\
&\forall y\, \forall x\ \bigg[q_{0}(y) \land \big(\xi(x,y) \implies \psi(x)\big)\bigg]\\
&\forall x\ \bigg[\exists y\ \big(q_{0}(y) \land \xi(x,y)\big) \implies \psi(x)\bigg]
\end{align*}
Siccome quello vale per ogni \(\psi(x) \in p(x)\), allora \(\exists y\ \big(q_{0}(y) \land \xi(x,y)\big)\) isola \(p(x)\).
\end{proof}

\uline{Esercizio di meditazione}: Leggere la dimostrazione del Teorema di Kuratowski-Ulam\footnote{Vedi \autocite{mottorosNotesDescriptiveSet2024} per una dimostrazione.} e convincersi che in fondo è il Lemma~\ref{lem:kurulma}

\uline{Esercizio} (?) Sia \(\mathcal{L}(A)\) numerabile, e sia \(P \subseteq S_{x}(A)\). Sono fatti equivalenti:
\begin{enumerate}
\item esiste un modello \(M\supseteq A\) che omette tutti i tipi in \(P\);
\item \(P\) è magro nella \(A\)-topologia;
\item esiste un modello \(M\supseteq A\) tale che \(\set{p \in S_{x}(M)\mid p\upharpoonright A \in P}\) è magro.
\end{enumerate}

Riprendiamo ora la Domanda~2:
\begin{quote}
\uline{Domanda 2}: dato \(p(x) \subseteq \mathcal{L}(A)\) consistente non isolato da \(A\), esiste \(M\supseteq A\) che non realizza \(p(x)\)?
\end{quote}
Siano \(X,Y\) insiemi qualsiasi con \(\card{X}=\card{Y}>\omega\), sia \(F\) l'insieme delle biiezioni tra \(X\) e \(Y\). Sia \(M=X\sqcup Y \sqcup F\).

Il linguaggio \(\mathcal{L}\) contiene simboli per gli insiemi \(X,Y,F\) e il simbolo \(r(f,x,y)\) che vale se \(f(x)=y\).

Fisso \(\mathcal{U}\succeq M\) modello mostro, e siano
\begin{align*}
\mathcal{U}_{X}&\coloneqq X(\mathcal{U})\\
\mathcal{U}_{Y}&\coloneqq Y(\mathcal{U})\\
\mathcal{U}_{F}&\coloneqq F(\mathcal{U}).
\end{align*}
Allora \(\mathcal{U}=\mathcal{U}_{X}\sqcup\mathcal{U}_{Y}\sqcup\mathcal{U}_{F}\).

Per ogni \(f \in \mathcal{U}_{F}\), \(r(f,x,y)\) definisce una biiezione tra \(\mathcal{U}_{X}\) e \(\mathcal{U}_{Y}\).

Sia \(Y_{0} \subseteq Y\) numerabile e \(c \in \mathcal{U}_{Y}\setminus Y_{0}\). Detto \(A\coloneqq X\cup Y_{0}\) sia \(p(y) = \operatorname{tp}(c/A)\). Notiamo che \(\card{A}>\omega\).

Voglio dimostrare che:
\begin{enumerate}
\item \(p(y)\) è realizzato in ogni modello \(N\supseteq A\);
\item \(p(y)\) non è isolato.
\end{enumerate}

Osserviamo che per ogni \(c' \in \mathcal{U}_{Y}\setminus Y_{0}\) esiste \(\hat{g} \in \operatorname{Aut}(\mathcal{U}/A)\) e \(\hat{g}c=c'\) Infatti, sia
\begin{equation*}
g\coloneqq \Id_{\mathcal{U}_{Y}\setminus\set{c,c'}}\cup \set{\langle c,c'\rangle, \langle c',c\rangle}
\end{equation*}
Allora la \(\hat{g}\) diventa
\begin{equation*}
\hat{g} = g\cup\Id_{\mathcal{U}_{X}}\cup \set{\langle f,g\circ f\rangle\mid f \in \mathcal{U}_{F}}.
\end{equation*}

Utilizzando questa osservazione:
\begin{enumerate}
\item Se \(N\supseteq A\) allora \(\card{Y(N)}>\omega\) e quindi \(Y(N)\) contiene un elemento \(c' \notin Y_{0}\), e per l'osservazione \(c'\mathrel{\equiv_{A}}c\).
\item Supponiamo che \(\varphi(y) \in \mathcal{L}(X,Y_{0})\) tale che \(\varphi(y)\implies p(y)\). Sia \(Y_{1} \subseteq Y_{0}\) finito tale che \(\varphi(y) \in \mathcal{L}(X,Y_{1})\).

Sia \(b\) tale che \(b\vDash\varphi(y)\). Sicuramente \(b\notin Y_{0}\). Prendo \(b' \in Y_{0}\setminus Y_{1}\). Per l'osservazione \(c'\mathrel{\equiv_{A}}c\) e quindi \(c'\vDash \varphi(x)\). Assurdo.
\end{enumerate}
\section{Lezione 4 - \textit{<2025-10-01 Wed>}}
\label{sec:org362d335}

Sia \(T\) una teoria completa con un modello infinito (ovvero senza modelli finiti). Sia \(\mathcal{U}\vDash T\) modello mostro di cardinalità \(\kappa>\card{\mathcal{L}}+\omega\).

\uline{Convenzione per questa lezione}: tutti i tipi \(p(x)\) sono:
\begin{itemize}
\item \(p(x) \subseteq \mathcal{L}(\mathcal{U})\);
\item \(\card{p(x)}<\kappa\) (ovvero i tipi sono piccoli).
\end{itemize}
\subsection{Modelli primi e modelli atomici}
\label{sec:org45b7517}

\begin{definizione}
Un modello \(M\) è \uline{primo su \(A \subseteq \mathcal{U}\)} se per ogni modello \(N \supseteq A\) esiste \(f \in \operatorname{Aut}(\mathcal{U}/A)\) tale che \(f[M] \preceq N\)\footnote{Questo è equivalente a richiedere che \(f[M] \subseteq N\).}.
\end{definizione}

\begin{definizione}
Un modello \(M\) è \uline{atomico su \(A \subseteq \mathcal{U}\)} se per ogni \(a \in M^{<\omega}\) il \(\operatorname{tp}(a/A)\) è isolato su \(A\).
\end{definizione}

\begin{thm}
Se \(\mathcal{L}(A)\le \omega\) allora le seguenti affermazioni sono equivalenti.
\begin{enumerate}
\item \(M\) è primo su \(A\);
\item \(M\) è numerabile e atomico su \(A\).
\end{enumerate}
\label{thm:primosseatomico}
\end{thm}

\uline{Domanda 1}: Esistono dei modelli così? Prossima settimana.

\begin{proof}
(del Teorema~\ref{thm:primosseatomico})

(\(1.\Rightarrow 2.\)): Corollario del OTT. Infatti se 1. allora \(M\) è numerabile; inoltre, se non fosse atomico, prendo \(b \in M^{<\omega}\) tale che \(p(x)\coloneqq\operatorname{tp}(b/A)\) non è isolato su \(A\); prendo \(N\) che omette \(p(x)\). Non posso immergere \(M\) in \(N\) (altrimenti l'immagine di \(b\) realizzerebbe \(p(x)\) in \(N\)).
\end{proof}

\uline{Notazione}: diremo che \(b\) è isolato da \(A\) se \(\operatorname{tp}(b/A)\) è isolato su \(A\).

\begin{prop}
Siano \(a,b \in \mathcal{U}^{<\omega}\). Sono fatti equivalenti:
\begin{enumerate}
\item \(A\) isola \(a,b\);
\item \(A,a\)\footnote{Con questo si intende \(A\cup\operatorname{rng}(a)\).} isola \(b\) e \(A\) isola \(a\).
\end{enumerate}
\label{prop:isolchainrule}
\end{prop}

\begin{oss}
Se \(p(x;y) \coloneqq \operatorname{tp}\big((a,b)/A\big)\) allora
\begin{enumerate}
\item \(\operatorname{tp}\big(b/(A,a)\big) = p(a;y)\);
\item \(\operatorname{tp}(a/A)\) è equivalente a \(\exists y\ p(x;y)\).
\end{enumerate}
Notiamo inoltre che \(\exists y\ p(x;y)\) è equivalente al tipo \(\set{\exists y\ \varphi(x;y)\mid \varphi \in p}\).
\end{oss}

\begin{proof}
(della Proposizione~\ref{prop:isolchainrule}). Sia \(p(x;y) \coloneqq \operatorname{tp}\big((a,b)/A\big)\)

(\(1.\Rightarrow 2.\)): Fissiamo \(\varphi \in p\) tale che
\begin{equation*}
\forall x,y\ \big[\varphi(x,y) \implies p(x,y)\big].
\end{equation*}
Allora ?????
\begin{itemize}
\item per il punto 1. dell'Osservazione precedente si ha che \(\forall y \ \big[\varphi(a,y)\implies p(a,y)\big]\), e quindi \(A,a\) isola \(b\).
\item per il punto 2. dell'Osservazione precedente si ha che \(\forall x\ \big[\exists y\ \varphi(x,y)\implies \exists y\ p(x,y)\big]\), e quindi \(A\) isola \(a\).
\end{itemize}

(\(2.\Rightarrow 1.\)): Siccome \(A,a\) isola \(b\), allora
\begin{equation*}
\varphi(a;y)\implies p(a;y)
\end{equation*}
per qualche \(\varphi \in p\). Siccome \(A\) isola \(a\), allora
\begin{equation*}
\psi(x) \implies \exists y\ p(x,y)
\end{equation*}
per qualche \(\psi \in \exists y\ p(x,y)\), ovvero \(\psi(x) \implies \exists y\ \xi(x,y)\) per ogni \(\xi \in p\).

Fissiamo una \(\xi \in p\). Osserviamo che \(\varphi(a,y)\implies \xi(a,y)\), quindi
\begin{equation*}
\forall y\ \big[\varphi(x,y)\implies \xi (x,y)\big] \in \operatorname{tp}(a/A) = \exists y\ p(x;y)
\end{equation*}
e quindi
\begin{align*}
&\forall x\ \bigg[\psi(x)\implies \forall y\ \big[\varphi(x,y)\implies \xi (x,y)\big]\bigg]\\
&\forall x\,\forall y\ \bigg[\psi(x)\implies\big[\varphi(x,y)\implies \xi (x,y)\big]\bigg]\\
&\forall x\,\forall y\ \bigg[\big[\psi(x) \land \varphi(x,y)\big]\implies\xi (x,y)\bigg]
\end{align*}
e siccome \(\xi\) è arbitrarira allora
\begin{equation*}
\forall x\,\forall y\ \bigg[\big[\psi(x) \land \varphi(x,y)\big]\implies p (x,y)\bigg]\qedhere
\end{equation*}
\end{proof}

\begin{cor}
Se \(M\) è atomico su \(A\), allora \(M\) è atomico su \(A,a\) per ogni \(a \in M\).
\end{cor}

\begin{proof}
Sia \(b \in M^{<\omega}\). \(A\) isola \(a,b\), quindi \(A,a\) isola \(b\).
\end{proof}

\begin{prop}
Sia \(k:M\to N\) una mappa elementare e \(M\) atomico su \(\dom k\). Allora per ogni \(b \in M\) esiste \(c \in N\) tale che \(k\cup\set{\langle b,c\rangle}:M\to N\) è elementare.
\label{prop:stellinaoijoijoijoijoijoijjhbhdjkcas}
\end{prop}
\begin{proof}
Sia \(a\) una enumerazione di \(\dom k\). Sia \(p(x;y) = \operatorname{tp}(a,b)\).
\begin{itemize}
\item \uline{Oss. 1}: \(p(a,y)\) è isolato su \(\dom k\).
\item \uline{Oss. 2}: \(p(ka,y)\) è isolato su \(\operatorname{rng} k\)\footnote{Passa per mappe elementari, vedi i suoi appunti}.
\end{itemize}

Sia quindi \(\varphi(x,y)\) tale che \(\varphi(ka,y)\implies p(ka,y)\). Il \(c \in N\) cercato p testimone di
\begin{equation*}
N\vDash \exists y\ \varphi(ka,y)\qedhere
\end{equation*}
\end{proof}

Possiamo quindi terminare la dimostrazione del Teorema~\ref{thm:primosseatomico}.
\begin{proof}
(del Teorema~\ref{thm:primosseatomico})

(\(2.\Rightarrow 1.\)): Dato \(N\supseteq A\), si consideri \(\Id_{A}:M\to N\) mappa elementare. Estendendo \(\omega\) volte usando la Proposizione~\ref{prop:stellinaoijoijoijoijoijoijjhbhdjkcas}.
\end{proof}

\begin{cor}
Sia \(\mathcal{L}(A)\) numerabile. Due modelli primi su \(A\) sono isomorfi.
\label{cor:asteriscojknkjnkjndavljhbdnlahkjdacnskjdnc}
\end{cor}
\subsection{Categoricità numerabile}
\label{sec:org2591530}

\begin{thm}
Se \(\mathcal{L}(A)\) è numerabile le seguenti affermazioni sono equivalenti.
\begin{enumerate}
\item \(T\) è \(\omega\)-categorica (su \(A\))\footnote{Questa nozione è non standard: signfica che due modelli di \(T\) che contengono \(A\) sono isomorfi tramite un isomorfismo che fissa \(A\).}
\item Ogni \(p(x) \subseteq \mathcal{L}(A)\) è isolato su \(A\).
\end{enumerate}
\end{thm}

\begin{proof}
(\(1.\Rightarrow 2.\)): Per OTT.

(\(2.\Rightarrow 1.\)): Se ogni tipo è isolato allora ogni modello numerabile è primo su \(A\), quindi per il Corollario~\ref{cor:asteriscojknkjnkjndavljhbdnlahkjdacnskjdnc}.
\end{proof}

\begin{prop}
(Forse \(\mathcal{L}\) deve essere numerabile?). Fissato \(A \subseteq \mathcal{U}\) arbitrario e \(x\) con \(\card{x}<\omega\) le seguenti affermazioni sono equivalenti:
\begin{enumerate}
\item Ogni \(p(x) \subseteq \mathcal{L}(A)\) è isolato su \(A\);
\item \(S_{x}(A)\) è finito;
\item \(\mathcal{L}_{x}(A)\)\footnote{\(\mathcal{L}_{x}(A) \subseteq \mathcal{L}(A)\) sono le formule con variabile \(x\).} è finito;
\item L'azione di \(\operatorname{Aut}(\mathcal{U}/A)\) su \(\mathcal{U}^{x}\) ha un numero finito di orbite.
\end{enumerate}
\end{prop}

\begin{proof}
(\(1.\Rightarrow 2.\)): Si ha che
\begin{equation*}
\mathcal{U}^{x} = \coprod_{p \in S_{x}(A)}p(\mathcal{U}^{x}).
\end{equation*}

Se ogni \(p(x) \in S_{x}(A)\) è equivalente ad una formula in \(\mathcal{L}_{x}(A)\) (ovvero \(p(x)\) è isolato su \(A\)) allora per compattezza \(S_{x}(A)\) deve essere finito.

(\(2.\Rightarrow 1.\)): Se \(p(x) \in S_{x}(A)\) finito, allora \(\lnot p(x)\) è disgiunzione finita di tipi, e quindi un tipo, quindi una formula, e quindi \(p(x)\) è isolato.

Ogni \(q(x) \subseteq \mathcal{L}(A)\) è disgiunzione di tipi completi (che sono tutti isolati) e quindi anche \(q(x)\) è isolato.

(\(2.\Leftrightarrow 3.\)): Ovvia.

(\(2.\Leftrightarrow 4.\)): I tipi completi definiscono orbite e viceversa.
\end{proof}

\begin{oss}
Ovviamente se \(A\) è infinito allora tutte le 4 opzioni sono false, in quanto
\begin{equation*}
\mathcal{L}_{x}(A)\supseteq \set{x=a\mid a \in A}
\end{equation*}
non può essere finito.
\end{oss}

\uline{Esercizio}: Sia \(\mathcal{L}\) numerabile. Se \(T\) è \href{20250618153446-struttura_minimale.org}{fortemente minimale} e \(\omega\)-categorica allora \(T\) non è finitamente assiomatizzabile.

\begin{thm}
Sia \(\mathcal{L}\) numerabile. Se \(T\) è totalmente categorica allora non è finitamente assiomatizzabile.
\end{thm}
\begin{thm}
Sia \(\mathcal{L}\) numerabile. Se \(T\) è totalmente categorica allora è finitamente assiomatizzabile modulo \(\set{\exists^{\ge n} x\,(x=x)\mid n<\omega }\).
\end{thm}

---

\begin{definizione}
\(T\) ha la \uline{proprietà del modello finito} (pmf) se per ogni enunciato \(\varphi \in \mathcal{L}\) esiste \(A \subseteq\mathcal{U}\) finito, \(A\) sottostruttura di \(\mathcal{U}\) e
\begin{equation*}
A\vDash \varphi\iff\mathcal{U}\vDash\varphi.
\end{equation*}
\end{definizione}

\begin{prop}
Se \(T\) ha la pmf allora non è finitamente assiomatizzabile.
\end{prop}
\begin{proof}
Supponiamo che \(T\vdash\varphi\vdash T\) per \(\varphi \in \mathcal{L}\). Se \(A\) sottostruttura tale che \(A\vDash\varphi\) allora \(A\vDash \exists^{\ge n}x \,(x=x)\) per ogni \(n\), e quindi \(\card{A}\ge \omega\).
\end{proof}

\begin{definizione}
Un insieme \(C \subseteq \mathcal{U}\) è \uline{omogeneo} se per ogni mappa elementare (su \(\mathcal{U}\)) \(k:C\to C\) e per ogni \(b \in C\) esiste \(c \in C\) tale che \(k\cup\set{\langle b,c\rangle}:C\to C\) è elementare.
\end{definizione}

\begin{prop}
Se \(T\) è fortemente minimale allora ogni \(C=\operatorname{acl}(C)\) è omogeneo.
\label{prop:dadimostrare_lezione5}
\end{prop}


(dimostrato la prossima lezione)
\section{Lezione 5 - \textit{<2025-10-07 Tue>}}
\label{sec:orgbb96fad}

Sia \(T\) una teoria completa con un modello infinito (ovvero senza modelli finiti). Sia \(\mathcal{U}\vDash T\) modello mostro di cardinalità \(\kappa>\card{\mathcal{L}}+\omega\).

Dalla lezione precedente si hanno questi due teoremi:
\begin{thm}
Sia \(\mathcal{L}\) numerabile. Se \(T\) è \href{20250618153446-struttura_minimale.org}{fortemente minimale} e \(\omega\)-categorica allora \(T\) non è finitamente assiomatizzabile.
\end{thm}
\begin{thm}
Sia \(\mathcal{L}\) numerabile. Se \(T\) è totalmente categorica allora non è finitamente assiomatizzabile.
\end{thm}

\begin{oss}
Questi due risultati sono collegati, in quanto una teoria fortemente minimale è \(\lambda\)-categorica per ogni \(\lambda>\omega\). Il collegamente è chiaro in luce del seguente teorema:
\end{oss}

\begin{thm}
(Teorema di Lachlon-Baldwin). Ogni modello \(M\), \(\card{M}>\omega\) contiene \(A \subseteq M\) definibile, fortemente minimale\footnote{Ovvero per ogni \(\varphi(x) \in \mathcal{L}(M)\), \(\card{x}=1\), \(\varphi(A)\) finito o \(\lnot\varphi(A)\) finito.} e \(M=\operatorname{acl}(A)\)
\end{thm}

\begin{quote}
Se \(T\) è fortemente minimale allora ogni \(C=\operatorname{acl}(C)\) è omogeneo.
\end{quote}

\begin{proof}
(della Proposizione~\ref{prop:dadimostrare_lezione5}). Sia \(k:C\to C\) elementare e finita, \(a\) una enumerazione di \(\dom k\), \(b \in C\).
\begin{itemize}
\item Caso 1: \(b \in \operatorname{acl}(\dom k)\). Per ogni \(h \in \operatorname{Aut}(\mathcal{U})\), \(h\supseteq k\)
\begin{equation*}
  h[\operatorname{acl}\dom k] = \operatorname{acl}(\operatorname{rng} k)
\end{equation*}
Tale \(h\supseteq k\) esiste per omogeneità e quindi prendo \(c =h(b) \in C\) poiché \(C =\operatorname{acl}(C)\). Ottengo \(k\cup\set{\langle b,c\rangle}\) elementare.
\item Caso 2: \(b\notin \operatorname{acl}(\dom k)\). Allora \(\dim(\operatorname{acl}(\dom k))<\dim C\) e quindi
\begin{equation*}
  \dim(\operatorname{acl}(\operatorname{rng} k))<\dim C
\end{equation*}
e quindi esiste \(c \in C\setminus \operatorname{acl}(\operatorname{rng} k)\). Quindi \(k\cup\set{\langle b,c\rangle}\) è elementare.
\end{itemize}
\end{proof}

\begin{lem}
Sia \(\mathcal{L}\) numerabile. Se \(T\) è fortemente minimale e \(\omega\)-categorica allora \(T\) ha p.m.f.
\end{lem}
\begin{proof}
Mostriamo per induzione che per ogni \(\varphi \in \mathcal{L}(\mathcal{U})\) esiste \(A \subseteq \mathcal{U}\) sottostruttura finita di \(\mathcal{U}\) che contiene i parametri di \(\varphi\) e \(A\vDash \varphi\).

Fissiamo \(n \in \N\). Allora esiste \(A_{n} \subseteq \mathcal{U}\) sottostruttura tale che
\begin{equation*}
A\vDash\varphi \quad \iff\quad \mathcal{U}\vDash\varphi
\end{equation*}
per ogni \(\varphi \in \mathcal{L}(A)\) tale che il numero di parametri di \(\varphi\) + il numero dei quantificatori in \(\varphi\) è \(\le n\).

Dimostro per induzione sulla complessità di \(\varphi\). Sia \(A_{n}=\operatorname{acl}(A_{n})\) finito tale che ogni \(p(z) \in S(A_{n})\) con \(\card{z}=n\) ha soluzione in \(A\). (\uline{Esercizio}: dimostrare che esiste).

Passo atomica e induzione per \(\land, \lor, \lnot\) banale. Mostriamo l'induzione per \(\exists\).

Consideriamo \(\exists x\ \varphi(x,a)\) con \(\card{x}=1\), \(\card{a}<n\). Per ipotesi induttiva si ha che, per ogni \(c \in A\):
\begin{equation*}
A\vDash \varphi(c,a)\iff \mathcal{U}\vDash\varphi(c,a).
\end{equation*}
Voglio dimostrare
\begin{equation*}
A\vDash\exists x\ \varphi(x,a)\iff \mathcal{U}\vDash\exists x\ \varphi(x,a)
\end{equation*}
(\(\Rightarrow\): segue immediatamente da ipotesi induttiva.

(\(\Leftarrow\)): Sia \(b \in \mathcal{U}\) tale che \(b,a\vDash\varphi(x,z)\), e sia \(p(x,z)=\operatorname{tp}(b,a)\). \(\card{x,z}\le n\), quindi esiste \(c,a' \in A^{\card{x,z}}\) tale che \(c,a'\vDash\varphi(x,z)\). Sia \(k=\set{\langle a',a\rangle}:A\to A\) mappa elementare. Per omogeneità la estendo a \(k\cup\set{\langle c,c'\rangle}:A\to A\) elementare, quindi \(c',a\vDash\varphi(x;z)\). Pertanto \(c'\) è testimone di \(A\vDash\exists x\ \varphi(x;z)\).
\end{proof}
\subsection{Teorie sottili (\emph{small})}
\label{sec:org54cf610}

Sia \(\mathcal{L}\) numerabile. Sia \(T\) una teoria completa con un modello infinito (ovvero senza modelli finiti). Sia \(\mathcal{U}\vDash T\) modello mostro di cardinalità \(\kappa>\card{\mathcal{L}}+\omega\).

\begin{definizione}
\(T\) è \uline{sottile su \(A\)} se per ogni \(x\), \(\card{x}<\omega\), \(S_{x}(A)\) è numerabile.
\end{definizione}

\begin{definizione}
\(T\) è \uline{sottile} se è sottile su \(\emptyset\).
\end{definizione}

\begin{prop}
Se \(T\) è sottile su \(A\) e \(B\) è finito allora \(T\) è sottile su \(A\cup B\).
\end{prop}
\begin{proof}
Sia \(b\) una enumerazione di \(B\). Ogni \(p(x) \in S_{x}(A\cup B)\) allora \(p(x)=q(x,b)\) con \(q(x,z) \in S_{x,z}(A)\). Ma se \(\card{x}<\omega\) allora \(S_{x,z}(A)\) è numerabile, poiché \(z\) finito in quanto \(B\) è finito.
\end{proof}
\begin{definizione}
Sia \(\Delta\) un insieme di formule con \(x\) libera. Un \uline{\(\Delta\)-albero binario} è una sequenza
\begin{gather*}
2^{<\lambda}\to \Delta\\
\langle \varphi_{s}(x): s \in 2^{<\lambda}\rangle
\end{gather*}
tale che
\begin{enumerate}
\item per ogni \(s \in 2^{\lambda}\) il tipo
\begin{equation*}
 p_{s}(x)\coloneqq\set{\varphi_{s\upharpoonright i}(x): i<\lambda}
\end{equation*}
è consistente.
\item per ogni \(r,s \in 2^{\lambda}\), se \(r\neq s\) allora \(p_{r}(x)\cup p_{s}(x)\) è inconsistente.
\end{enumerate}

\(\lambda\) si dice \uline{altezza dell'albero}.
\end{definizione}
In particolare:
\begin{itemize}
\item per oggi \(\Delta=L_{x}(A)\);
\item più avanti si potrà avere \(\Delta = \set{\varphi(x,b),\lnot\varphi(x,b)\mid b \in B}\).
\end{itemize}

Spesso la condizione 2. si ottiene richiedendo che
\begin{equation*}
\varphi_{s\concat 0}(x) \iff\lnot \varphi_{s\concat 1}(x)
\end{equation*}


\uline{Notazione}: si denota con \(S(\Delta)\) l'insieme dei \(\Delta\)-tipi massimali consistenti\footnote{\(p(x) \in S(\Delta)\) sse \(p(x) \subseteq \Delta\) consistente e per ogni \(\varphi(x) \in \Delta\), se \(p(x)\cup\set{\varphi(x)}\) consistente allora \(\varphi(x) \in p(x)\).}

\begin{lem}
Sia \(\Delta\) numerabile e chiuso per negazioni. LSASE
\begin{enumerate}
\item esiste un \(\Delta\)-albero binario di altezza \(\omega\);
\item \(\card{S(\Delta)} = 2^{\omega}\);
\item \(\card{S(\Delta)} >\omega\)
\end{enumerate}
\end{lem}
\begin{proof}
(\(1.\Rightarrow 2.\Rightarrow 3.\)): ovvio.

(\(3.\Rightarrow 1.\)): Costruiamo un albero per induzione su \(n<\omega\). Assumiamo come ipotesi induttiva che per ogni \(s \in 2^{n}\) il tipo
\begin{equation*}
p_{s}(x) = \set{\varphi_{s\upharpoonright i}(x)\mid i \le n}
\end{equation*}
ha \(>\omega\) estensioni in \(S(\Delta)\).

\uline{Claim}: esiste \(\psi^{s}(x) \in \Delta\) tale che \(p_{s}(x)\cup\set{\psi(x)}\) e \(p_{s}(x)\cup\set{\lnot\psi(x)}\) abbiano \(>\omega\) estensioni in \(S(\Delta)\)..

Posso quindi porre \(\varphi_{s\concat 0 }(x) \coloneqq\psi^{s}(x)\), \(\varphi_{s\concat 1}(x) \coloneqq \lnot\psi^{s}(x)\).

\uline{Dimostrazione del claim}: per assurdo, supponiamo che per ogni \(\psi(x) \in \Delta\) si ha che \(p_{s}(x)\cup\set{\psi(x)}\) o \(p_{s}(x)\cup\set{\lnot\psi(x)}\) hanno \(\le \omega\) estensioni. Notiamo che \uline{esattamente} uno tra i due tipi può avere \(\le \omega\) estensioni per ipotesi induttiva. Definiamo
\begin{align*}
q(x)&=p_{s}(x)\cup\set{\psi(x) \in\Delta\mid p_{s}(x)\cup\set{\lnot\psi(x)}\text{ ha }\le\omega\text{ estensioni}} \\
&=p_{s}(x)\cup\set{\psi(x) \in\Delta\mid p_{s}(x)\cup\set{\psi(x)}\text{ ha }>\omega\text{ estensioni}}
\end{align*}
Allora \(q(x)\) è consistente\footnote{Se per assurdo \(q(x)\) non fosse consistente, allora \(p_{s}(x)\implies \bigvee_{i=1}^{n}\lnot \psi_{i}(x)\) ma ogni \(\psi_{i}(x)\) ha \(\le\omega\) estensioni, quindi assurdo.}, e inoltre \(q(x) \in S(\Delta)\).

Contiamo i tipi in \(S(\Delta)\setminus\set{q(x)}\) che estendono \(p_{s}(x)\). Se \(p(x)\) soddisfa queste ipotesi allora è inconsistente con \(q(x)\), quindi contiene \(\psi(x)\) tale che \(p_{s}(x)\cup\set{\psi(x)}\) ha \(\le\omega\) estensioni. Siccome \(\Delta\) è numerabile allora si ottiene che la cardinalità dell'insieme dei tipi in \(S(\Delta)\setminus\set{q(x)}\) che estendono \(p_{s}(x)\) è numerabile. Assurdo.
\end{proof}

\begin{thm}
Sia \(\mathcal{L}\) numerabile. LSASE:
\begin{enumerate}
\item \(T\) è sottile.
\item esiste un modello saturo numerabile;
\item non esiste un \(L_{x}\)-albero binario di altezza \(\omega\) per nessun \(x\) con \(\card{x}<\omega\).
\end{enumerate}
\end{thm}
\begin{proof}
(\(1.\Leftrightarrow 3.\)): per il lemma precedente.

(\(2.\Rightarrow 1.\)): lasciato per esercizio.
\end{proof}

\begin{thm}
Ogni teoria sottile \(T\) ha modelli atomici su ogni \(A\) numerabile.
\label{thm:thmino}
\end{thm}
\begin{lem}
Sia \(\mathcal{L}(A)\) numerabile. Sono fatti equivalenti:
\begin{enumerate}
\item esiste un modello atomico su \(A\);
\item ogni formula \(\varphi(y) \in \mathcal{L}(A)\) consistnete, con \(\card{y}<\omega\), ha una soluzione isolata su \(A\)\footnote{Ovvero (una delle due seguenti condizioni equivalenti):
\begin{itemize}
\item esiste \(a\vDash\varphi(y)\) e \(\operatorname{tp}(a/A)\) è isolato
\item è conseguenza di una \(\psi(y) \in \mathcal{L}(A)\) completa su \(A\).
\end{itemize}}.
\end{enumerate}
\end{lem}
\begin{proof}
(\(1.\Rightarrow 2.\)): ovvia.

(\(1.\Rightarrow 2.\)): assumiamo 2. e costruiamo \(M\supseteq A\) atomico: \(M=\set{a_{i}\mid i <\omega}\) costruito per induzione.

Sia \({a} \coloneqq \langle a_{i}\mid i<\omega\rangle\). L'ipotesi induttiva è che \(a\upharpoonright i\) è isolato su \(A\).
\begin{itemize}
\item \uline{Passo \(i\)}: considero \(\psi(x,a\upharpoonright i)\) consistente. Per ipotesi induttiva esiste \(\varphi(z)\) che isola \(\operatorname{tp}(a\upharpoonright i /A)\). Quindi \(\psi(x;z) \land  \varphi(z)\) È consistente. Sia \(c,a'\vDash \psi(x;z) \land  \varphi(z)\) isolato su \(A\). Quindi \(a' \equiv_{A} a\upharpoonright i\) e quindi esiste \(h \in \operatorname{Aut}(\mathcal{U}/A)\) con \(h(a') = a\upharpoonright i\). quindi \(h(c),a\upharpoonright i\vDash \psi(x;z) \land  \varphi(z)\) isolato su \(A\).
\end{itemize}
\end{proof}
\begin{proof}
(del Teorema~\ref{thm:thmino}).
Per assurdo, neghiamo l'ipotesi 2. del lemma precedente.
Sia \(\varphi(x)\) formula che non segue da nessuna formula completa.

Costruisco albero binario: \(p_{0}(x)=\set{\varphi(x)}\).
Dato \(p_{s}(x)\), cerco \(\psi(x) \in \mathcal{L}(A)\) tale che \(p_{s}(x)\cup\set{\psi(x)}, p_{s}(x)\cup\set{\lnot\psi(x)}\) ed estendo nel modo ovvio.

Tale \(\psi\) esiste, altrimenti \(p_{s}(x)\) sarebbe una formula completa su \(A\).
\end{proof}
\section{Lezione 6 - \textit{<2025-10-08 Wed>}}
\label{sec:org6316f53}

\uline{Esercizio} (Vaught): Nessuna teoria completa \(T\) ha esattamente due modelli numerabili. (\(\mathcal{L}\) numerabile)

\emph{Soluzione}:
Assumiamo per assurdo che \(T\) abbia esattamente due modelli numerabili.
Allora \(T\) non ha modelli finiti e non è \(\omega\)-categorica, e \(T\) è sottile,
in quanto un modello numerabile realizza al più \(\omega\) tipi su \(\omega\).

Quindi \(T\) ha un modello numerabile saturo \({N}\), ed un modello atomico \({M}\).
WLOG \({M}\preceq {N}\).
Sia quindi \(a \in N^{<\omega}\) tale che \(\operatorname{tp}(a)\) non è isolato.
Necessariamente \(a\notin {M}\), e quindi \(M\not\cong N\).

\(T\) è piccola su \(a\).
Sia \({M}'\) modello atomico su \(a\),
\({M}\cup\set{a} \subseteq {M}'\preceq {N}\), \({M}\preceq {M}'\)

Se ci sono solo due modelli, allora \({M}'\cong {M}\) oppure \({M}'\cong {N}\).
\begin{itemize}
\item Sicuramente \({M}'\not\cong {M}\) (come per \({N}\not\cong{N}\))
\item Se fosse \(M'\cong N\) allora \(M'\) è atomico e saturo su \(a\), e quindi \(T\) è \(\omega\)-categorica su \(a\) (Esercizio del foglio 2). Assurdo.
\end{itemize}

\begin{esempio}
Esiste una teoria con esattamente 3 modelli. Vedi esercizio di IdL.
\end{esempio}
\subsection{Linguaggio a più sorte}
\label{sec:org8338a4f}

Un linguaggio a più sorte \(\mathcal{L}\) è un insieme
\begin{equation*}
\mathcal{L} = \mathcal{L}_{\text{fun}} \sqcup \mathcal{L}_{\text{rel}}\sqcup \mathcal{L}_{\text{sort}}.
\end{equation*}
dotato di una funzione \uline{arietà} che assegna:
\begin{itemize}
\item a \(r \in \mathcal{L}_{\text{rel}}\) una tupla \(\langle s_{0},\dots,s_{n}\rangle \in \mathcal{L}_{\text{sort}}^{<\omega}\); si dirà che \(r\) è di sorta \(\langle s_{0},\dots,s_{n}\rangle\)
\item a \(f \in \mathcal{L}_{\text{fun}}\) una tupla \(\langle s_{0},\dots,s_{n}\rangle \in \mathcal{L}_{\text{sort}}^{<\omega}\); si dirà che \(f\) è di sorta \(\langle s_{0},\dots,s_{n}\rangle\)
\end{itemize}

Un \(\mathcal{L}\)-modello \(\mathcal{M}\) è composto da
\begin{itemize}
\item \(M_{s}\) insieme per ogni \(s \in \mathcal{L}_{\text{sort}}\)
\item per ogni \(r \in \mathcal{L}_{\text{rel}}\) di sorta \(\langle s_{0},\dots,s_{n}\rangle\) un insieme
\begin{equation*}
  r^{\mathcal{M}} \subseteq M_{s_{0}}\times \dots\times M_{s_{n}}
\end{equation*}
\item per ogni \(f \in \mathcal{L}_{\text{fun}}\) di sorta \(\langle s_{0},\dots,s_{n}\rangle\) una funzione totale
\begin{equation*}
  f^{\mathcal{M}} : M_{s_{1}}\times \dots\times M_{s_{n}}\to M_{s_{0}}
\end{equation*}
\end{itemize}

Fisso per ogni \(s \in \mathcal{L}_{\text{sort}}\) un insieme di variabili \(V_{s}\),
tale che se \(s\neq s'\) allora \(V_{s}\cap V_{s'}=\emptyset\).
\begin{itemize}
\item I termini sono definiti in ``modo ragionevole''. N.B. i termini hanno anche una arietà.
\item Una formula atomica è \(r(t_{1},\dots,t_{n})\) con \(r \in \mathcal{L}_{\text{rel}}\) con \(t_{i}\) che rispettano la sorta di \(r\).
\item Per costruire le formule:
\begin{itemize}
\item per \(\land ,\lor,\lnot\) tutto uguale:
\item i quantificatori sono \(\exists_{s}\) e \(\forall_{s}\) per ogni \(s \in \mathcal{L}_{\text{sort}}\).
\end{itemize}
\end{itemize}

\begin{oss}
In generale, per un linguaggio a più sorti, posso considerare un linguaggio ad un'unica sorta che contenga un predicato per ogni sorta. Il problema insorge per linguaggi ad infinite sorti, poiché i modelli saturi costruiti nel linguaggio ``ad una sorta'' avrebbero elementi che non sono di nessuna sorte.
\end{oss}
\subsection{Linguaggio del secondo ordine}
\label{sec:org45185ac}

???
\subsection{Immaginari}
\label{sec:orgc9f97e1}

Sia \(T\) una \(\mathcal{L}\)-teoria completa con un modello infinito (ovvero senza modelli finiti). Sia \(\mathcal{U}\vDash T\) modello mostro di cardinalità \(\kappa>\card{\mathcal{L}}+\omega\).

Si definisce \(\mathcal{U}^{\text{eq}}\) struttura a più sorti nel linguaggio \(\mathcal{L}^{\text{eq}}\): si considera \(\mathcal{L}_{\text{sort}} = \set{0}\cup \mathcal{L}_{x,z}\), con \(\card{x}=\card{z}=\omega\)\footnote{Si noti che \(\mathcal{L}_{x,z}\) è l'insieme delle \(\mathcal{L}\)-formule nelle variabili \(x,z\)}
\begin{enumerate}
\item \(\mathcal{U}\) è il dominio della sorta 0 (anche detta \emph{home} o reale);
\item per ogni \(\sigma(x,z) \in \mathcal{L}_{x,z}\), il dominio della sorta \(\sigma\) è
\begin{equation*}
 \mathcal{U}_{\sigma} \coloneqq \set{\sigma(\mathcal{U}^{x}; b)\mid b \in \mathcal{U}^{z}}
\end{equation*}
\end{enumerate}

\(\mathcal{L}^{\text{eq}}\) contiene i simboli di \(\mathcal{L}\) che ``vivono nella sorta \(0\)'', e \(\in_{\sigma}\) per ogni \(\sigma \in\mathcal{L}_{x,z}\), di arietà \(\langle 0^{n},\sigma\rangle\) dove \(n\) è il più piccolo tale che solo le variabili in \(x\upharpoonright n\) occorrono in \(\sigma\).

\begin{lem}
Sia \(\mathcal{X}\) tupla di variabili di sorta \(\sigma(x;z)\) e \(\mathcal{A}\) tupla di parametri di sorta \(\sigma(x;z)\), \(u\) variabile di sorta 0. Se \(\varphi(u,\mathcal{X}) \in \mathcal{L}^{\text{eq}}\). Allora esiste \(\varphi'(u;z) \in \mathcal{L}\) e \(b \in \mathcal{U}^{z}\) tale che
\begin{equation*}
\mathcal{U}^{\text{eq}}\vDash \forall  u\ \big[\varphi(u,\mathcal{A})\iff \varphi'(u,b)\big]
\end{equation*}
\end{lem}

\uline{Morale del lemma}: \sout{questa espansione non serve a nulla}; i sottoinsiemi di \(\mathcal{U}\) definibili (con parametri) in \(\mathcal{L}^{\text{eq}}\) sono gli stessi che quelli definibili in \(\mathcal{L}\).

\begin{proof}
???? FINIRE
\end{proof}

\begin{oss}
Per ogni \(A \subseteq \mathcal{U}^{\text{eq}}\) esiste un \(B \subseteq \mathcal{U}\) tale che i sottoinsiemi \(\mathcal{L}^{\text{eq}}(A)\)-definibili di \(\mathcal{U}\) sono contenuti nei \(\mathcal{L}(B)\)-definibili di \(\mathcal{U}\).

In generale, però, non esiste \(B\) tale che questa inclusione sia una uguaglianza.
\end{oss}

\begin{esempio}
Sia \(\mathcal{L}=\set{E}\), \(E^{\mathcal{U}}\) relazione di equivalenza con due classi infinite.

Se \(\mathcal{A} \subseteq\mathcal{U}\) è una delle due classi, allora
\(x \in \mathcal{A}\) è equivalente a \(E[b,x]\) per qualche \(b \in \mathcal{U}\).
\end{esempio}
\section{Lezione 7 - \textit{<2025-10-14 Tue>}}
\label{sec:orga3f1f25}

\begin{oss}
\(\mathcal{U}^{\text{eq}}\) è saturo, ovvero realizza ogni tipo \(p(x,\mathcal{X})\) finitamente consistente con ``pochi'' parametri.
\end{oss}
\begin{oss}
\(\mathcal{U}^{\text{eq}}\) è un modello omogeneo:
se \(a,\mathcal{A}\mathrel{\equiv_{\mathcal{L}^{\text{eq}}}}a',\mathcal{A}'\),
con \(\card{a,\mathcal{A}}<\kappa\) allora esiste \(f \in \operatorname{Aut}(\mathcal{U}^{\text{eq}})\) tale che
\begin{equation*}
fa=a',\qquad f\mathcal{A}=\mathcal{A}'.
\end{equation*}
\end{oss}
\begin{oss}
Si ha che \(\operatorname{Aut}(\mathcal{U}^{\text{eq}})\cong \operatorname{Aut}(\mathcal{U})\) canonicamente.
\end{oss}
\subsection{Sottostrutture elementari}
\label{sec:org24859ac}

Sia \(\mathcal{V}^{\dag}\preceq \mathcal{U}^{\text{eq}}\).
\begin{itemize}
\item Il dominio della \emph{home sort} di \(\mathcal{V}^{\dag}\) è \(\mathcal{V}\preceq\mathcal{U}\).
\item Il dominio della sorta \(\sigma(x;z)\) è della forma \(\sigma(\mathcal{U}^{x}; b)\) per qualche \(b \in \mathcal{U}^{z}\).
\end{itemize}

Detto \(\mathcal{A}\coloneqq \sigma(\mathcal{U}^{x};b)\) allora
\begin{equation*}
\mathcal{U}^{\text{eq}}\vDash \exists z \ \forall x\ [x \in \mathcal{A}\iff \sigma(x;z)]
\end{equation*}
e per elementarietà
\begin{equation*}
\mathcal{V}^{\dag}\vDash \exists z \ \forall x\ [x \in \mathcal{A}\iff \sigma(x;z)]
\end{equation*}
e quindi esiste \(b' \in \mathcal{V}\) tale che
\(\mathcal{A}=\sigma(\mathcal{U}^{x};v')\).

Viceversa, preso \(\mathcal{V}\preceq \mathcal{U}\) e definisco \(\mathcal{V}^{\text{eq}}\) ``aggiungendo'' gli insiemi \(\sigma(\mathcal{U}^{x};b)\) con \(b \in \mathcal{V}^{z}\) ottengo \(\mathcal{V}^{\text{eq}}\preceq \mathcal{U}^{\text{eq}}\).

---

\begin{definizione}
Un elemento \(a \in \mathcal{U}^{\text{eq}}\) (di qualsiasi sorta) si dice \uline{definibile su \(A \subseteq \mathcal{U}^{\text{eq}}\)} se esiste \(\varphi (u) \in \mathcal{L}^{\text{eq}}(A)\) tale che
\begin{equation*}
\varphi(a) \land \exists^{=1} u\ \varphi(u).
\end{equation*}
\end{definizione}

\begin{prop}
LSASE:
\begin{enumerate}
\item \(\mathcal{A} \in \mathcal{U}^{\text{eq}}\) è definibile su \(A \subseteq \mathcal{U}^{\text{eq}}\);
\item esiste \(\psi(x) \in \mathcal{L}(A)\) tale che \(x \in \mathcal{A} \iff \psi(x)\).
\end{enumerate}
\end{prop}
\begin{proof}
(\(1.\Rightarrow 2.\)): Sia \(\varphi(\mathcal{X}) \in \mathcal{L}^{\text{eq}}(A)\) tale che \(\mathcal{A}\) sia l'unica soluzione di \(\varphi(\mathcal{X})\). Quindi
\begin{equation*}
x \in A\iff \exists \mathcal{X}\ [x \in \mathcal{X} \land \varphi(\mathcal{X})]
\end{equation*}
e poi posso eliminare la parte immaginaria.
\end{proof}

\begin{definizione}
Se \(A \subseteq \mathcal{U}^{\text{eq}}\) si definisce \(\operatorname{dcl}^{\text{eq}}(A)\) l'insieme degli elementi di \(\mathcal{U}^{\text{eq}}\) definibili su \(A\).

Se \(A \subseteq \mathcal{U}^{\text{eq}}\) si definisce \(\operatorname{dcl}(A)\) l'insieme degli elementi di \(\mathcal{U}\) definibili su \(A\).
\end{definizione}
\begin{definizione}
Se \(A \subseteq \mathcal{U}^{\text{eq}}\) si definisce \(\operatorname{acl}^{\text{eq}}(A)\) l'insieme degli elementi di \(\mathcal{U}^{\text{eq}}\) algebrici su \(A\).

Se \(A \subseteq \mathcal{U}^{\text{eq}}\) si definisce \(\operatorname{acl}(A)\) l'insieme degli elementi di \(\mathcal{U}\) algebrici su \(A\).
\end{definizione}

Però come possiamo definire gli elementi algebrici?

\begin{definizione}
\emph{(Tentativo 1).}
Un elemento \(a \in \mathcal{U}^{\text{eq}}\) si dice \uline{algebrico su \(A \subseteq \mathcal{U}^{\text{eq}}\)} se esiste \(\varphi (u) \in \mathcal{L}^{\text{eq}}(A)\) tale che
\begin{equation*}
\varphi(a) \land \exists^{=n} u\ \varphi(u)
\end{equation*}
per qualche \(n \in \N\).
\end{definizione}

\begin{definizione}
\emph{(Tentativo 2).}
Un elemento \(a \in \mathcal{U}^{\text{eq}}\) si dice \uline{algebrico su \(A \subseteq \mathcal{U}^{\text{eq}}\)} se l'orbita
\begin{equation*}
O(a/A) = \set{fa \mid f \in \operatorname{Aut}(\mathcal{U}^{\text{eq}}/A)}
\end{equation*}
è finita.
\end{definizione}

\begin{definizione}
\emph{(Tentativo 3).}
Un elemento \(a \in \mathcal{U}^{\text{eq}}\) si dice \uline{algebrico su \(A \subseteq \mathcal{U}^{\text{eq}}\)} se
\begin{equation*}
a \in \bigcap_{A \subseteq M^{eq}} M^{eq}.
\end{equation*}
\end{definizione}

Ci si deve chiedere quale di queste definizioni è ben chiara quando \(a\) è un insieme definibile.

Notiamo che, per il tentativo 3, un insieme \(\mathcal{A} \in M^{eq}\) sse \(\mathcal{A} \subseteq \mathcal{U}\) è definibile su \(M\) in \(\mathcal{U}\).

\begin{prop}
Le tre diverse definizioni di elementi algebrici sono equivalenti.
\end{prop}

---

\begin{definizione}
Una \uline{equivalenza finita definibile in \(A \subseteq \mathcal{U}^{\text{eq}}\)} è una formula \(\varepsilon(x;y) \in \mathcal{L}(A)\) che definisce una relazione di equivalenza su \(\mathcal{U}^{x}\) con un numero finito di classi.
\end{definizione}

\begin{thm}
LSASE
\begin{enumerate}
\item \(\mathcal{A} \in \operatorname{acl}^{\text{eq}}(A)\);
\item \(\mathcal{A}\) è unione di classi di una equivalenza finita definibile in \(A\).
\end{enumerate}
\end{thm}
\begin{proof}
(\(2.\Rightarrow 1.\)):
Supponiamo \(\mathcal{A}=\bigcup_{i=1}^{n} \varepsilon(\mathcal{U}^{x},c_{i})\) per \(c_{i} \in \mathcal{U}^{x}\), con \(\varepsilon(x;y) \in \mathcal{L}(A)\).

Utilizzando la definizione del Tentativo 2, l'orbita di \(\mathcal{A}/A\) ha cardinalità al più \(\binom{m}{n}\) dove \(m\) è il numero di classi di \(\varepsilon\).

Utilizzando invece la definizione del Tentativo 3, sia \(M^{\text{eq}}\supseteq A\). Allora
\begin{equation*}
\mathcal{U}^{\text{eq}}\vDash \text{``esistono }m\text{ classi di }\varepsilon\text{''}
\end{equation*}
e per elementarietà
\begin{equation*}
M^{\text{eq}}\vDash \text{``esistono }m\text{ classi di }\varepsilon\text{''}
\end{equation*}
quindi per ogni \(i=1,\dots,n\) esiste \(c_{i}' \in M^{x}\) tale che \(\varepsilon(c_{i}',c_{i})\).

(\(1.\Rightarrow 2.\)):
Sia \(\mathcal{A} \in \operatorname{acl}^{\text{eq}}(A)\) (utilizzando la definizione del Tentativo 1) e sia \(\varphi(\mathcal{X}) \in \mathcal{L}^{\text{eq}}(A)\) tale che
\begin{equation*}
\varphi(\mathcal{A}) \land \exists^{=n} \mathcal{X}\ \varphi(\mathcal{X}).
\end{equation*}

Si definisce
\begin{equation*}
\varepsilon(x;y) = \forall \mathcal{X}\ \bigg[\varphi(\mathcal{X})\implies \big[x \in \mathcal{X}\iff y \in \mathcal{X}\big]\bigg]
\end{equation*}
\uline{Claim}: \(\varepsilon\) è una relazione di equivalenza finita.

Siano \(\mathcal{A}_{1},\dots,\mathcal{A}_{n}\) tutte le soluzioni di \(\varphi(\mathcal{X})\). Allora
\begin{equation*}
\varepsilon(x;y) = \bigwedge_{i=1}^{n}\big[x \in \mathcal{A}_{i}\iff y \in \mathcal{A}_{i}\big]
\end{equation*}
e quindi \(\varepsilon\) ha al più \(2^{n}\) classi.
\end{proof}
\begin{definizione}
\(a,b \in \mathcal{U}^{x}\) hanno lo stesso tipo forte di Shelah su \(A \subseteq \mathcal{U}\) (e si scrive \(a\mathrel{\overset{\text{Sh}}{\equiv}_{A}}b\)) se per ogni \(\varepsilon(x;y) \in \mathcal{L}(A)\) relazione di equivalenza finita
\begin{equation*}
\mathcal{U}\vDash\varepsilon(a,b).
\end{equation*}
\end{definizione}
\begin{oss}
Se \(a\mathrel{\overset{\text{Sh}}{\equiv}_{A}}b\) allora \(a,b\) hanno lo stesso tipo su \(A\): infatti, presa \(\varphi(x) \in \mathcal{L}(A)\) prendo
\begin{equation*}
\varepsilon(x,y) \coloneqq \big[\varphi(x)\iff\varphi(y)\big].
\end{equation*}
\end{oss}

\uline{Esercizio}: LSASE:
\begin{enumerate}
\item \(a\mathrel{\overset{\text{Sh}}{\equiv}_{A}}b\);
\item \(a \mathrel{\equiv_{\operatorname{acl}^{\text{eq}}A}} b\).
\end{enumerate}

\emph{Soluzione}.
(\(\lnot 1.\Rightarrow \lnot 2.\)):
Esiste \(\varepsilon(x,y) \in \mathcal{L}(A)\) tale che \(\lnot\varepsilon(a,b)\).

Sia \(\mathcal{A}=\varepsilon(\mathcal{U}^{x},b) \in\operatorname{acl}^{\text{eq}}A\).
Poiché \(b \in \mathcal{A}\) e \(a\notin \mathcal{A}\) ottengo \(\lnot 2.\)

(\(\lnot 2.\Rightarrow \lnot 1.\)):
Esiste \(\varphi(x) \in \mathcal{L}(\operatorname{acl}^{\text{eq}}A)\) tale che
\begin{equation*}
\varphi(a) \land \lnot\varphi(b)
\end{equation*}
e pertanto
\begin{equation*}
\mathscr{B}\coloneqq
\varphi(\mathcal{U}^{x}) \in \operatorname{dcl}^{\text{eq}}\big(\operatorname{acl}^{\text{eq}}A\big)
= \operatorname{acl}^{\text{eq}}A.
\end{equation*}
Siccome \(\mathscr{B} \in \operatorname{acl}^{\text{eq}}A\)
\begin{equation*}
\mathscr{B}=\bigcup_{i=1}^{n}\varepsilon(\mathcal{U}^{x};c_{i})
\end{equation*}
dove \(\varepsilon(x;y) \in \mathcal{L}(A)\) relazione di equivalenza finita.

\(a \in \mathscr{B}\), quindi per qualche \(c_{i}\) si ha \(\varepsilon(a;c_{i})\); siccome \(b\notin \mathscr{B}\) si ha \(\lnot \varepsilon(b;c_{i})\) e quindi \(\lnot \varepsilon(a;b)\).\qed
\section{Lezione 8 - \textit{<2025-10-15 Wed>}}
\label{sec:orgf9069b8}

\subsection{Eliminazione degli immaginari}
\label{sec:orgc1d4d6b}

\begin{definizione}
Una teoria \(T\) ha \uline{eliminazione degli immaginari} se per ogni \(\mathscr{D} \subseteq \mathcal{U}^{x}\) definibile, esiste \(\varphi(x;z) \in \mathcal{L}\) tale che
\begin{equation*}
\mathcal{U}\vDash \exists^{=1}z\ \forall x\ [x \in \mathscr{D}\iff \varphi(x;z)] .
\end{equation*}
\end{definizione}
Bisogna verificare che la definizione dipende solo da \(T\) (e non dipende dal modello mostro scelto).

\begin{definizione}
Una teoria \(T\) ha \uline{eliminazione debole degli immaginari} se per ogni \(\mathscr{D} \subseteq \mathcal{U}^{x}\) definibile, esiste \(\varphi(x;z) \in \mathcal{L}\) ed esiste \(k \in \N^{+}\) tale che
\begin{equation*}
\mathcal{U}\vDash \exists^{=k}z\ \forall x\ [x \in \mathscr{D}\iff \varphi(x;z)].
\end{equation*}
\end{definizione}
\uline{Nota}: quando \(\card{z}=\emptyset\) si considera \(\exists^{=1}z\) come sempre vera.

\begin{thm}
LSASE:
\begin{enumerate}
\item \(T\) ha eliminazione degli immaginari;
\item ogni insieme definibile è \uline{interdefinibile} con una tupla reale;
\item ogni \(\mathscr{D} \subseteq \mathcal{U}^{x}\) definibile è definibile su \(\operatorname{dcl}\set{\mathscr{D}}\), ovvero \(\mathscr{D} \in \operatorname{dcl}^{\text{eq}}\big(\operatorname{dcl}\set{\mathscr{D}}\big)\).
\end{enumerate}
\end{thm}
\begin{definizione}
\(a,b \in \mathcal{U}^{\text{eq}}\) di sorte arbitraria sono \uline{interdefinibili} se vale uno dei seguenti fatti equivalenti:
\begin{enumerate}
\item \(\operatorname{dcl}^{\text{eq}}(a)=\operatorname{dcl}^{\text{eq}}(b)\);
\item \(a \in \operatorname{dcl}^{\text{eq}}(b)\) e \(b \in \operatorname{dcl}^{\text{eq}}(a)\);
\item \(\operatorname{Aut}(\mathcal{U}/a)=\operatorname{Aut}(\mathcal{U}/b)\).
\end{enumerate}
\end{definizione}

\begin{thm}
LSASE:
\begin{enumerate}
\item \(T\) ha eliminazione debole degli immaginari;
\item ogni insieme definibile è \uline{interdefinibile} con un insieme finito;
\item ogni \(\mathscr{D} \subseteq \mathcal{U}^{x}\) definibile è definibile su \(\operatorname{acl}\set{\mathscr{D}}\), ovvero \(\mathscr{D} \in \operatorname{dcl}^{\text{eq}}\big(\operatorname{acl}\set{\mathscr{D}}\big)\).
\end{enumerate}
\label{thm:ikajncdahbckhdasgjbmnjkhkgjvmbkgyunvmbhygkufjhcvm}
\end{thm}
\begin{proof}
(\(1.\Rightarrow 2.\)): Sia \(\mathscr{D}\) un insieme definibile: per ipotesi
\begin{equation*}
\mathcal{U}\vDash \exists^{=k}z\ \forall x\ [x \in \mathscr{D}\iff \varphi(x;z)]
\end{equation*}
e dunque sia \(\mathscr{B}\coloneqq \set{z\mid \forall x\ [x \in \mathscr{D}\iff \varphi(x;z)]} = \set{z_{1},\dots,z_{k}}\).

Quindi \(\mathscr{B} \in \operatorname{dcl}^{\text{eq}}\set{\mathscr{D}}\), e inoltre \(\mathscr{D} \in \operatorname{dcl}^{\text{eq}}\set{\mathscr{B}}\) poiché
\begin{equation*}
\mathscr{D}=\set{x \mid \exists z \in \mathscr{B}\ \varphi(x;z)}.
\end{equation*}

(\(2.\Rightarrow 3.\)): Se \(\mathscr{B}=\set{b_{1},\dots,b_{n}}\) è tale che
\begin{equation*}
\operatorname{dcl}^{\text{eq}}\set{\mathscr{D}}=
\operatorname{dcl}^{\text{eq}}\set{\mathscr{B}}
\end{equation*}
allora \(b_{i} \in \operatorname{acl}\set{\mathscr{B}}\) testimoniato dalla formula \(z \in \mathscr{B}\) e quindi \(b \in \operatorname{acl}\set{\mathscr{D}}\):
\begin{equation*}
\operatorname{acl}\set{\mathscr{D}} = \operatorname{acl}\operatorname{acl}\set{\mathscr{D}} =
\operatorname{acl}\operatorname{dcl}\set{\mathscr{D}}=
\operatorname{acl}\operatorname{dcl}\set{\mathscr{B}}=
\operatorname{acl}\set{\mathscr{B}}
\end{equation*}
Sappiamo che \(\mathscr{D} \in \operatorname{dcl}^{\text{eq}}\set{\mathscr{B}}\) e quindi esiste \(\psi(x;\mathscr{B})\) tale che
\begin{equation*}
\mathscr{D}=\set{x\mid \psi(x;\mathscr{B})}
\end{equation*}
e quindi esiste \(\psi'(x;b_{1},\dots,b_{n})\) tale che
\begin{equation*}
\mathscr{D} = \set{x\mid \psi'(x;b_{1},\dots,b_{n})}
\end{equation*}
e quindi \(\mathscr{D} \in \operatorname{dcl}^{\text{eq}}(\operatorname{acl}\set{\mathscr{D}})\).

(\(3.\Rightarrow 1.\)): Fissiamo \(\mathscr{D}\neq\emptyset\) definibile e fissiamo \(\sigma(x;z)\) e \(b \in (\operatorname{acl}\set{\mathscr{D}})^{z}\) tale che
\begin{equation*}
\forall x\ [x \in \mathscr{D}\iff\sigma(x;z)].
\end{equation*}
Fisso \(\delta(z,\mathcal{X})\) tale che \(\delta(b;\mathscr{D}) \land \exists^{\le k}z \ \delta(z;\mathscr{D})\) e pongo
\begin{equation*}
\psi(z;\mathcal{X}) =  \forall x\ [x \in \mathcal{X}\iff\sigma(x;z)] \land \delta(z;\mathcal{X}).
\end{equation*}
Affermo che
\begin{equation*}
\exists^{=k}z\ \forall x\ [x \in \mathscr{D}\iff \sigma(x;z) \land \exists \mathcal{X}\ \psi(z;\mathcal{X})].
\end{equation*}
\(\mathscr{B}\) è testimone di \(\exists z\). Vogliamo mostrare che esistono \(k\) testimoni di \(\exists z\). Sia \(c\) tale che
\begin{equation*}
\forall x\ [x \in \mathscr{D}\iff \sigma(x;c) \land \exists \mathcal{X}\ \psi(c;\mathcal{X})]
\end{equation*}
FINIRE
\end{proof}

\begin{thm}
La teoria \(T_{\text{acf}}^{p}\)\footnote{Questa è la teoria dei campi algebricamente chiusi di caratteristica \(p\).} ha eliminazione degli immaginari.
\label{thm:tacfpeliminazoijaijoijfortedididididi}
\end{thm}

\begin{lem}
La teoria \(T_{\text{acf}}^{p}\) ha eliminazione debole degli immaginari.
\end{lem}

\begin{thm}
Se \(T\) è una teoria fortemente minimale e
\(\operatorname{acl}(\emptyset)\) è infinita allora \(T\) ha l'eliminazione debole degli immaginari.
\end{thm}

\begin{lem}
Se ogni formula \(\varphi(z) \in \mathcal{L}^{\text{eq}}(A)\) consistente ha una soluzione in \(\operatorname{acl}(A)\) allora \(T\) ha l'eliminazione debole degli immaginari.
\label{lem:oijsoidisojnckskjndkjncskdjn}
\end{lem}
\begin{proof}
(del Lemma~\ref{lem:oijsoidisojnckskjndkjncskdjn}).
Se \(\mathscr{D}\) è di sorta \(\sigma(x;z)\) prendo come
\begin{equation*}
\varphi(z) = \forall x\ [x \in \mathscr{D}\iff\sigma(x;z)]
\end{equation*}
con \(\varphi(z) \in \mathcal{L}^{\text{eq}}(\set{\mathscr{D}})\).

Se \(\varphi(z)\) ha soluzione in \(\operatorname{acl}\set{\mathscr{D}}\) allora \(\mathscr{D}\) è definibile su \(\operatorname{acl}\set{\mathscr{D}}\). La tesi segue dal
Teorema~\ref{thm:ikajncdahbckhdasgjbmnjkhkgjvmbkgyunvmbhygkufjhcvm}
\end{proof}

\begin{lem}
\(T_{\text{acf}}^{p}\) elimina gli immaginari finiti.
\label{lem:kjnkjnkadsjnclakjdnaclkdjn}
\end{lem}
\begin{proof}
(del Lemma~\ref{lem:kjnkjnkadsjnclakjdnaclkdjn}).
Sia \(\mathcal{A}=\set{a_{1},\dots,a_{n}} \subseteq \mathcal{U}^{m}\), e sia
\begin{equation*}
a_{i} = a_{i,1},\dots,a_{i,m}.
\end{equation*}
Siano \(x,y\) variabili con \(\card{x}=1\) e \(y=y_{1},\dots,y_{m}\): si definisce
\begin{equation*}
t(x,y) = \prod_{i=1}^{n}\left(x-\sum_{k=1}^{m}a_{i,k}y_{k}\right).
\end{equation*}
L'interpretazione di \(t(x,y)\) dipende solo da \(\mathcal{A}\) e non dall'enumerazione.

Sia \(c\) la tupla dei coefficienti dello sviluppo in monomi di \(t(x,y)\).

Dato \(\mathcal{A}\) definisco \(c\) e viceversa.
\end{proof}

\begin{proof}
(del Teorema~\ref{thm:tacfpeliminazoijaijoijfortedididididi}).
Segue banalmente dal Lemma~\ref{lem:kjnkjnkadsjnclakjdnaclkdjn}.
\end{proof}

\uline{Esercizio}: I grafi aleatori non hanno l'eliminazione degli immaginari.

\begin{thm}
I grafi aleatori hanno eliminazione debole degli immaginari.
\end{thm}

\begin{lem}
Se \(\mathscr{D}\) è definibile sia su \(A\) che su \(B\), allora è definibile su \(A\cap B\) allora \(T\) ha eliminazione debole degli immaginari.
\label{lem:kjnasodkchjgvjbvcgjmvcxhfcjgvkhbjnbhvgcjfvbj}
\end{lem}
\begin{proof}
(del Lemma~\ref{lem:kjnasodkchjgvjbvcgjmvcxhfcjgvkhbjnbhvgcjfvbj})
Sia \(A\) \(\subseteq\)-minimale tale che \(\mathscr{D}\) sia definibile su \(A\) finito.

Sia \(f \in \operatorname{Aut}(\mathcal{U}/\set{\mathscr{D}})\). Affermo che \(f[A]=A\). Infatti \(\mathscr{D}\) è definibile su \(f[A]\) e pertanto, per ipotesi, su \(f[A]\cap A\). Per minimalità \(f[A]=A\).

Sia \(a\) tupla che enumera \(A\). L'orbita di \(a/\set{\mathscr{D}}\) è finita, e quindi \(a \in \operatorname{acl}\set{\mathscr{D}}\).
\end{proof}
\section{{\bfseries\sffamily TODO} Lezione 9 - \textit{<2025-10-21 Tue>}}
\label{sec:orge717d26}

\subsection{Formule e tipi invarianti}
\label{sec:orga0bbde6}

In this chapter, \(\mathcal{L}\) è un linguaggio, \(T\) is a complete theory without finite models, and \(\mathcal{U}\) is a saturated model of inaccessible cardinality \(\kappa\) strictly larger than \(\card{\mathcal{L}}+\omega\).
\begin{definizione}
\(\mathscr{D} \subseteq \mathcal{U}^{z}\) è \uline{invariante su \(A \subseteq \mathcal{U}\)} piccolo, se per ogni \(f \in \operatorname{Aut}(\mathcal{U}/A)\)
\begin{equation*}
f[\mathscr{D}] =\mathscr{D}
\end{equation*}
\end{definizione}

\begin{oss}
\(\mathscr{D}\) è inviariante se
\begin{equation*}
a \in \mathscr{D}\iff b \in \mathscr{D}
\end{equation*}
per ogni \(a\mathrel{\equiv_{A}}b\), \(a,b \in \mathcal{U}^{z}\).
\end{oss}
\begin{prop}
Detto \(\lambda=\card{\mathcal{L}_{z}(A)}\), gli insiemi invarianti su \(A\) sono al più \(2^{2^{\lambda}}\).
\end{prop}
\begin{proof}
Vedi Proposizione~14.1 di \autocite{zambellaCrecheCourseModel2025}.
\end{proof}
\begin{definizione}
Una formula \(\varphi(x) \in \mathcal{L}(\mathcal{U})\) è \(A\)-invariante se \(\varphi(\mathcal{U}^{x})\) lo è.
\end{definizione}
\begin{definizione}
Sia \(\varphi(x;z) \in \mathcal{L}\). Una \(\varphi\)-formula su \(\mathcal{U}\) è (alcune possibilità):
\begin{enumerate}
\item \(\varphi(x;b)\) e \(\lnot\varphi(x;b)\) per \(b \in \mathcal{U}^{z}\);
\item congiunzioni di formule della forma \(\varphi(x;b)\) e \(\lnot\varphi(x;b)\) per \(b \in \mathcal{U}^{z}\).
\item combinazioni booleane\footnote{Ovvero chiusura rispetto a tutti i connettivi tranne i quantificatori.} di formule della forma \(\varphi(x;b)\) per \(b \in \mathcal{U}^{z}\).
\item formula in \(\mathcal{L}_{x}'(\mathcal{U})\) costruite a partire da \(\varphi(x';z')\) dove \(x'\) è una copia di \(x\) e \(z'\) è una copia di \(z\).
\end{enumerate}
\end{definizione}

\begin{definizione}
Sia \(\varphi(x;z) \in \mathcal{L}\). Una \(\varphi\)-formula su \(A\) è:
\begin{itemize}
\item[{3'.}] combinazioni booleane di formule della forma \(\varphi(x;b)\) per \(b \in \mathcal{U}^{z}\) \uline{che siano inviarianti su \(A\)}.
\end{itemize}

Con \(\mathcal{L}_{\varphi}(A)\) si intende l'insieme delle \(\varphi\)-formule su \(A\).
\end{definizione}

Quest'ultima è la definizione utilizzata.
\begin{definizione}
Sia \(\varphi(x;z) \in \mathcal{L}\).
Un \(\varphi\)-tipo è un insieme di \(\varphi\)-formule.

Un \(\varphi\)-tipo \uline{globale} è un insieme massimale finitamente consistente di \(\varphi\)-formule su \(\mathcal{U}\).
\(S_{\varphi}(\mathcal{U})\) è l'insieme di tutti i \(\varphi\)-tipi globali.
\end{definizione}
\begin{itemize}
\item[{$\square$}] \href{20251113175327-insieme_esternamente_definibile_in_un_modello_mostro.org}{Insieme esternamente definibile} vedi p.118 di \autocite{zambellaCrecheCourseModel2025a}
\end{itemize}

\begin{definizione}
Un \(\varphi\)-tipo \(p(x) \subseteq \mathcal{L}_{\varphi}(\mathcal{U})\) è \(A\)-invariante se per ogni \(\psi(x;\overline{b}) \in p\) e per ogni \(f \in \operatorname{Aut}(\mathcal{U}/A)\)
\begin{equation*}
p(x)\vdash \psi(x;f\overline{b})
\end{equation*}
ovvero esiste una congiunzione finita di formule in \(p(x)\) che implica \(\psi(x;f\overline{b})\). ?????
\end{definizione}

\uline{Notazione}: scriveremo \(p(x)\vdash fp(x)\) dove
\begin{equation*}
fp(x) \coloneqq \set{\psi(x;f\overline{b})\mid \psi(x;\overline{b}) \in p(x)}.
\end{equation*}

Se \(p(x) \in S_{\varphi}(\mathcal{U})\) allora a meno di equivalenza
\begin{equation*}
p(x) \subseteq \set{\varphi(x;b),\lnot\varphi(x;b)\mid b \in \mathcal{U}^{z}}.
\end{equation*}
\uline{Notazione}: definisco
\begin{equation*}
\mathscr{D}_{p,\varphi} \coloneqq \set{b \in \mathcal{U}^{z}\mid \varphi(x;b) \in p(x)}.
\end{equation*}

\(\mathscr{D}_{p,\varphi}\) è un insieme \textbf{esternamente definibile} (da \(\varphi(x;z)\) o da \(p\)).

Sia \(\nonstandard{\mathcal{U}}\succeq \mathcal{U}\) saturo di cardinalità maggiore, tale che
\begin{equation*}
\nonstandard{\mathcal{U}},\nonstandard{a}\vDash p(x)
\end{equation*}
allora
\begin{equation*}
\mathscr{D}_{p,\varphi} = \mathcal{U}^{z}\cap \varphi(\nonstandard{\!a},\nonstandard{\mathcal{U}}^{z}).
\end{equation*}

\begin{prop}
\(p(x)\) è \(A\)-invariante sse \(\mathscr{D}_{p,\varphi}\) è \(A\)-invariante.
\end{prop}

\uline{Fatto.}
Sia \(p(x) \subseteq \mathcal{L}_{\varphi}(\mathcal{U})\). \(p(x)\) è \(A\)-invariante sse per ogni \(\psi(x,\overline{b}) \in \mathcal{L}_{\varphi}(\mathcal{U})\) e per ogni \(a\mathrel{\equiv_{A}}b\) si ha:
\begin{equation*}
	p(x)\vdash \psi(x;a) \oldiff p(x)\vdash \psi(x;b)
\end{equation*}

\uline{Fatto.}
Sia \(p(x) \in S_{\varphi}(\mathcal{U})\). \(p(x)\) è \(A\)-invariante sse per ogni \(a\mathrel{\equiv_{A}}b\) si ha:
\begin{equation*}
p(x)\vdash \varphi(x;a)\iff\varphi(x;b).
\end{equation*}

\uline{Fatto.}
Sia \(p(x) \in S_{\varphi}(\mathcal{U})\). \(p(x)\) è \(A\)-invariante sse per ogni \(a\mathrel{\equiv_{A}}b\) e per ogni \(c\vdash p(x)\upharpoonright A\cup\set{a,b}\)\footnote{Ovvero tenendo soltanto le formule con parametri in \(A\cup\set{a,b}\).} si ha:
\begin{equation*}
\varphi(c;a)\iff\varphi(c;b).
\end{equation*}

\uline{Fatto.}
Sia \(p(x) \in S_{\varphi}(\mathcal{U})\). \(p(x)\) è \(A\)-invariante sse per ogni \(a\mathrel{\equiv_{A}}b\) e per ogni \(c\vdash p(x)\upharpoonright A\) si ha:
\begin{equation*}
a\mathrel{\equiv_{A\cup\set{c}}}b.
\end{equation*}
\subsubsection{Eredi e coeredi}
\label{sec:orgf32995d}

\begin{definizione}
Un tipo \(p(x) \subseteq \mathcal{L}(\mathcal{U})\) si dice \uline{finitamente soddisfacibile in \(A\)} se per ogni \(\psi(x) \in \set{\land}\!p\):
\begin{equation*}
\psi(A^{x})\neq \emptyset.
\end{equation*}
\end{definizione}
\begin{prop}
Se \(p(x) \in S_{\varphi}(\mathcal{U})\) è finitamente soddisfacibile in \(A\), allora \(p(x)\) è \(A\)-invariante.
\end{prop}
\begin{proof}
Vogliamo mostrare che \(p(x)\vdash \varphi(x,a)\iff\varphi(x,b)\) per ogni \(a\mathrel{\equiv_{A}}b\).

Per assurdo se \(p(x)\not\vdash \varphi(x,a)\iff\varphi(x,b)\) allora
\begin{equation*}
p(x)\vdash \varphi(x,a)\not\iff\varphi(x,b).
\end{equation*}
Poiché è finitamente soddisfacibile allora esiste \(c \in A^{x}\) tale che
\begin{equation*}
\varphi(c,a)\not\iff\varphi(c,b)
\end{equation*}
e quindi \(a\not\mathrel{\equiv_{A}}b\). Assurdo.
\end{proof}
\begin{oss}
Sia \(a \in \mathcal{U}^{x}\) e \(p(x)= \operatorname{tp}(a/B)\). LSASE:
\begin{enumerate}
\item \(p(x)\) è finitamente soddisfacibile in \(A\);
\item \(a \in\) chiusura di \(A^{x}\) nella \(B\)-topologia di \(\mathcal{U}\).
\end{enumerate}
\end{oss}
\begin{prop}
Se \(q(x) \subseteq \mathcal{L}(\mathcal{U})\) finitamente soddisfacibile in \(A\), allora esiste \(q(x) \subseteq p(x) \in S(\mathcal{U})\), \(p(x)\) finitamente soddisfacibile in \(A\).
\end{prop}
\begin{proof}
Sia \(p(x)\supseteq q(x)\) un tipo massimale finitamente soddisfacibile in \(A\) (che esiste per il \href{20250210104633-lemma_di_zorn.org}{Lemma di Zorn}).

Affermo che \(p(x)\) è completo. Per assurdo, supponiamo che \(\psi(x), \lnot\psi(x)\notin p(x)\). Per massimalità esiste \(\varphi(x) \in p(x)\) tale che né \(\varphi(x) \land \psi(x)\) né \(\varphi(x) \land\lnot\psi(x)\) sono realizzate in \(A\). Ma esiste \(c \in A^{x}\) tale che \(c\vdash \varphi(x)\). Quindi
\begin{equation*}
C\vdash \psi(x) \lor\lnot\psi(x).\qedhere
\end{equation*}
\end{proof}
\begin{cor}
Tutti i tipi su un modello \(M\) si estendono a tipi globali invarianti su \(M\).
\end{cor}
\section{{\bfseries\sffamily DONE} Lezione 10 - \textit{<2025-10-22 Wed>}}
\label{sec:orgc3fa062}

\subsection{{\bfseries\sffamily DONE} \href{20251026210840-sequenza_di_morley.org}{Sequenze di Morley}}
\label{sec:orga67fa01}

Sia \(p(x) \in S(\mathcal{U})\) invariante su \(A\).

\begin{definizione}
Una \uline{sequenza di Morley} di \(p(x)\) su \(A\) è \(\overline{c} \in (\mathcal{U}^{x})^{\alpha}\), \(\overline{c} = \langle c_{i}: i<\alpha\rangle\) tale che
\begin{equation*}
\forall  i<\alpha: \quad c_{i}\vDash p(x)\upharpoonright A\cup{\overline{c}\upharpoonright i}.
\end{equation*}
\end{definizione}
\subsection{Sequenze di indiscernibili}
\label{sec:orgf5e50e3}

Sia \((I,<)\) un ordine lineare. Una \(I\)-sequenza è
\begin{equation*}
\overline{c} =\langle c_{i}: i \in I\rangle,\quad c_{i} \in \mathcal{U}.
\end{equation*}

\uline{Notazione}: se \(I_{0} \subseteq I\) finito, \(c\upharpoonright I_{0}\) è letto come una tupla finita.
\begin{definizione}
\(\overline{c}\) è una \uline{sequenza di indiscernibili su \(A\)} se per ogni \(I_{0}, I_{1} \subseteq I\), \(\card{I_{0}}=\card{I_{1}} = n<\omega\)
\begin{equation*}
c\upharpoonright I_{0} \equivalentover{A} c\upharpoonright I_{1}.
\end{equation*}
\end{definizione}

\uline{Nota}: in questo caso scriviamo \(I_{0},I_{1} \in I^{(n)}\) o
\begin{equation*}
I_{0},I_{1} \in \binom{I}{n}.
\end{equation*}
Se \(I\) è insieme qualsiasi e \(\overline{c}=\set{c_{i}\mid i \in I}\), si ha la seguente:
\begin{definizione}
\(\overline{c}\) è una \uline{sequenza di totalmente indiscernibili su \(A\)} se per ogni \(n\), per ogni \(\set{i_{1},\dots,i_{n}}, \set{j_{1},\dots,j_{n}} \subseteq I\) insiemi di elementi distinti
\begin{equation*}
(c_{i_{1}},\dots,c_{i_{n}})\equivalentover{A} (c_{j_{1}},\dots,c_{j_{n}}).
\end{equation*}
\end{definizione}
\begin{thm}
Ogni \(\overline{c}=\langle c_{i}:i<\omega\rangle\) sequenza di Morley di \(p(x) \in S(\mathcal{U})\) invariante su \(A\) è indiscernibile su \(A\).
\end{thm}
\begin{proof}
Mostriamo per induzione su \(n\) che per ogni \(I \in \omega^{(n)}\)
\begin{description}
\item[{Ipotesi induttiva}] \(c\upharpoonright n \equivalentover{A} c\upharpoonright I\);
\item[{Tesi}] \(c\upharpoonright n, c_{n} \equivalentover{A} c\upharpoonright I, c_{i}\), per \(i>I\).
\end{description}

Vedi Proposizione~14.20 di \autocite{zambellaCrecheCourseModel2025a}
\end{proof}
\subsection{Eredi e coeredi}
\label{sec:org9961c78}

\begin{definizione}
Si dice \(p(x) \subseteq \mathcal{L}(\mathcal{U})\) finitamente soddisfacibile in \(M\) se
\begin{equation*}
\varphi(M^{x}) \neq \emptyset \quad\text{ per ogni }\varphi(x) \in \set{\land}p.
\end{equation*}
In questo caso si dice che \(p(x)\) è un \uline{coerede} di \(p(x)\upharpoonright M\).
\end{definizione}
\begin{definizione}
Se \(p(x)\) è finitamente soddisfacibile in \(M\), una sequenza di Morley di \(p(x)\) su \(M\) si dice \uline{sequenza di coeredi}.
\end{definizione}
\uline{Notazione}: Siano \(a \in \mathcal{U}^{x}\), \(b \in \mathcal{U}^{z}\). Si scriverà
\begin{equation*}
a \nonfork b \qquad \oldiff\qquad \operatorname{tp}(a/M,b)\text{ è finitamente soddisfacibile in }M
\end{equation*}
ovvero per ogni \(\varphi(x;z) \in \mathcal{L}(M)\)
\begin{equation*}
\varphi(a;b)\quad\oldimplies\quad \varphi(m;b)\text{ per qualche }m \in M
\end{equation*}
ovvero \(a\) appartiene alla chiusura (topologica) di \(M^{x}\) nella \((M,b)\)-topologia.

Si definisce il tipo
\begin{equation*}
x\nonfork  b = \set{\lnot \varphi(x,b): \varphi(x;z) \in \mathcal{L}(M), \varphi(M^{x},b)= \emptyset}
\end{equation*}
soddisfatto da tutti i seguenti elementi:
\begin{equation*}
a\vDash x\nonfork  b \IFF a\nonfork  b.
\end{equation*}

\begin{lem}
Valgono le seguenti proprietà.
\begin{enumerate}
\item Se \(a\nonfork  b\) allora \(fa\nonfork  fb\) per ogni \(f \in \operatorname{Aut}(\mathcal{U}/M)\). \hfill(invarianza)
\item \(a\mathrel{\nonfork_{M}} b\) sse \(a_{0}\nonfork b_{0}\) per ogni \(a_{0} \subseteq a\), \(b_{0} \subseteq b\) sottotuple finite.
\hfill(carattere finito)
\item Se \(a\nonfork  b, c\) e \(c\nonfork  b\) allora \(a,c\nonfork  b\).
\hfill(transitività)
\item Se \(a\nonfork b\), per ogni \(c\) esiste \(a'\equivalentover{M,b}a\) e \(a'\nonfork b,c\).
\hfill(\emph{coheir extension})
\item Se \(a\nonfork b_{1},b_{2}\) e \(b_{1} \equivalentover{M} b_{2}\) allora \(b_{1}\equivalentover{M,a}b_{2}\).
\hfill(\emph{non-splitting})
\end{enumerate}
\label{lem:propfork}
\end{lem}
\begin{proof}
Vedi hw4 - TdM
\end{proof}

\uline{Domanda}: \(a \nonfork z\) è un tipo?
\begin{equation*}
a \nonfork z = \set{\lnot \varphi(a;z)\mid \varphi(M^{x}, z) = \emptyset\ \varphi(x;z) \in \L(M)}.
\end{equation*}

La risposta è NO!! See Remark~14.7 e Esercizio~14.13 di \autocite{zambellaCrecheCourseModel2025a}.
\section{{\bfseries\sffamily DONE} Lezione 11 - \textit{<2025-10-28 Tue>}}
\label{sec:orgb93674b}

\begin{lem}
LSASE:
\begin{enumerate}
\item \(\overline{c} = \langle c_{i} : i<\omega \rangle\) è sequenza di coeredi su \(M\) (rispetto a qualche \(p(x) \in S(\U)\));
\item \(c_{i} \nonfork (c \restricted i)\) e \(c_{i+1} \equivalentover{M, c \restricted i} c_{i}\);
\item \(c_{i} \nonfork (c \restricted i)\) e \(c_{j} \equivalentover{M, c \restricted i} c_{i}\) per ogni \(j>i\).
\end{enumerate}
\end{lem}

Le condizioni 2. e 3. sono banalmente equivalenti, e pertanto saranno utilizzate in maniera interscambiabile.

\begin{proof}
(\(1.\Rightarrow 2.\)): Sia \(p(x) \in S(\U)\) finitamente soddisfacibile in \(M\) tale che \(\overline{c}\) sia sequenza di coeredi rispetto a \(p\).

\uline{Si dimostra \(c_{i} \nonfork (c \restricted i)\)}.

Sia \(\varphi(x, z_{0},\dots,z_{i-1}) \in \L(M)\) tale che \(\varphi(c_{i}, c \restricted i)\).

Poiché \(p(x)\) è completo e \(c_{i}\vDash p(x) \restricted {\scriptstyle M, (c\restricted i)}\) allora:
\begin{equation*}
\varphi(x; c \restricted i) \in p(x) \restricted {\scriptstyle M, (c\restricted i)}.
\end{equation*}
dove con \(p(x)\upharpoonright A\) si intende la restrizione di un tipo ad un insieme di parametri.

Siccome \(p(x)\) finitamente soddisfacibile in \(M\) allora \(\varphi(M^{x}; c \restricted i) \neq \emptyset\). Per definizione questo significa che
\begin{equation*}
c_{i} \nonfork (c \restricted i).
\end{equation*}

\uline{Si dimostra l'equivalenza elementare}.

Per definizione si ha che
\begin{equation*}
c_{i}\vDash p(x) \restricted{\scriptstyle M, (c\restricted{i})},\qquad c_{i+1}\vDash p(x) \restricted{\scriptstyle M, (c\restricted{i+1})}
\end{equation*}
In particolare, dalla seconda relazione, segue che
\begin{equation*}
c_{i+1}\vDash p(x) \restricted{\scriptstyle M, (c\restricted i)} \subseteq p(x) \restricted{\scriptstyle M, (c\restricted{i+1})}
\end{equation*}

Per la completezza di \(p(x)\) segue \(c_{i} \equivalentover{M, c\restricted i} c_{i+1}\).

(\(2.\Rightarrow 1.\)):
Sia \(q(x) = \set{\varphi(x) \in \L(M, \overline{c}) \mid
\null\vDash\varphi(c_{i})\text{ per infiniti }i<\omega}\).
È evidente che in realtà
\begin{equation*}
q(x)
= \set{\varphi(x) \in \L(M, \overline{c}) \mid
\null \vDash\varphi(c_{i})\text{ per cofiniti }i<\omega}.
\end{equation*}
Infatti ``\(\supseteq\)'' è ovvio, mentre per il viceversa, se per infiniti \(c_{i}\) vale \(\varphi(c_{i})\), allora in particolare esisterà \(I \in \N\) tale che
\begin{equation*}
\varphi(x) \in \L(M, c\restricted I),\qquad \varphi(c_{I})
\end{equation*}
e allora vale anche \(\varphi(c_{j})\) per ogni \(j>I\), in quanto
\begin{equation*}
c_{j} \equivalentover{M, c \restricted I} c_{I}
\end{equation*}
per ogni \(j>I\).

In particolare \(q(x)\) è chiuso per congiunzioni e finitamente soddisfacibile (ne sono testimoni i \(c_{i}\)).

Quindi \(q(x)\) è finitamente soddisfacibile in \(M\) (poiché \(M\) è sottostruttura elementare di \(\U\)).

Sia quindi \(p(x)\supseteq q(x)\), \(p(x) \in S(\U)\). Questo è finitamente soddisfacibile in \(M\) in quanto lo è in \(\U\).

\uline{\(\overline{c} = \langle c_{i} : i<\omega \rangle\) è sequenza di Morley di \(p(x)\) su \(M\)}

Si deve dimostrare che \(c_{i}\vDash p(x) \restricted {\scriptstyle M, (c \restricted i)}\) per ogni \(i \in \omega\).

Se per qualche \(c_{i}\) e per qualche \(\varphi(x; c \restricted i) \in  p(x) \restricted {\scriptstyle M, (c \restricted i)}\) si avesse \(\lnot \varphi(c_{i}; c \restricted i)\), allora per ogni \(j>i\) si avrebbe (siccome \(c_{j} \equivalentover{M, (c\restricted i)} c_{i}\))
\begin{equation*}
\lnot\varphi(c_{j}; c \restricted i)
\end{equation*}
e pertanto \(\lnot\varphi(x; c\restricted i) \in q(x)\). Assurdo perché \(q(x) \subseteq p(x)\) e \(p(x)\) finitamente soddisfacibile.
\end{proof}
\subsection{Teorema di Ramsey}
\label{sec:org4761d84}

Vedi Sezione 15.1 di \autocite{zambellaCrecheCourseModel2025a}.

\href{20251029171516-teorema_di_ramsey.org}{Teorema di Ramsey}
\section{{\bfseries\sffamily DONE} Lezione 12 - \textit{<2025-10-29 Wed>}}
\label{sec:org64daed1}

\subsection{Teorema di Ehrenfeucht-Mostowski}
\label{sec:orgbbd5ff8}

\begin{thm}
(Teorema di Ramsey).
Sia \(M\) insieme infinito.
Per ogni \(k\)-colorazione di \(M^{(n)}\) esiste \(H \subseteq M\) infinito tale che gli \(H^{(n)}\) sono monocromatici.
\end{thm}
\begin{definizione}
Sia \((I,<)\) un insieme parzialmente ordinato senza elementi massimi, e sia \(\overline{a} \coloneqq \langle a_{i}: i \in I \rangle\), con \(a_{i} \in \U\). Sia \(\overline{x} = \langle x_{i} : i<\omega \rangle\). Si definisce il \uline{tipo di Ehrenfeucht-Mostowski} come
\begin{equation*}
\operatorname{EM-tp}(\overline{a}/A) \coloneqq \set{\varphi(\overline{x}) \in \L(A) \mid \varphi(a \upharpoonright I_{0})\text{ per ogni }I_{0} \in \binom{I}{\omega}}.
\end{equation*}
\end{definizione}

\begin{oss}
Se \(\overline{a}\) è una sequenza di indiscernibili su \(A\), allora \(\operatorname{EM-tp}(\overline{a}/A)\) è completo.
\end{oss}
\begin{thm}
(Teorema di Ehrenfeucht-Mostowski).
Data \(\overline{a} \coloneqq \langle a_{i}: i \in I \rangle\) con \((I,<)\) un insieme parzialmente ordinato senza elementi massimi, esiste \(\overline{c} = \langle c_{i} : i<\omega \rangle\) di indiscernibili su \(A\) tale che \begin{equation*}
\operatorname{EM-tp}(\overline{c}/A) \supseteq \operatorname{EM-tp}(\overline{a}/A).
\end{equation*}
\end{thm}
\begin{proof}
(idea). Sia \(q(\overline{x}) \coloneqq \operatorname{EM-tp}(\overline{a}/A)\).
Vogliamo realizzare
\begin{equation*}
q(x) \cup \set{\varphi(\overline{x} \restricted I_{0}) \iff \varphi(\overline{x} \restricted I_{1}) \mid I_{0},I_{1} \in I^{(\omega)}, \varphi(x) \in \L(A)}
\end{equation*}
Devo dimostrare che il tipo di cui sopra sia finitamente consistente. Basta mostrare che per ogni \(\varphi_{1},\dots,\varphi_{n} \in \L(A)\) soddisfo
\begin{equation*}
\set{\varphi(\overline{x} \restricted I_{0}) \iff \varphi(\overline{x} \restricted I_{1}) \mid I_{0},I_{1} \in I^{(\omega)}, i=1,\dots,n}
\end{equation*}
con qualche \(\overline{a} \restricted H\), \(H \in I^{(\omega)}\).

Sia \(m\) il massimo tale che \(x_{m}\) occorre in \(\varphi_{1},\dots,\varphi_{n}\). Soddisfo
\begin{equation*}
\set{\varphi(\overline{x} \restricted I_{0}) \iff \varphi(\overline{x} \restricted I_{1}) \mid I_{0},I_{1} \in I^{(m)}, i=1,\dots,n}
\end{equation*}
con qualche \(\overline{a} \restricted H\), \(H \in I^{(\omega)}\).

Coloro \(I^{(m)}\) con \(2^{n}\) colori: un colore per ogni valore di verità di
\begin{equation*}
\varphi_{1}(a_{i_{1}},\dots,a_{i_{m}}), \dots , \varphi_{n}(a_{i_{1}},\dots, a_{i_{m}}).
\end{equation*}

Per Ramsey esiste \(H \subseteq I^{(\omega)}\) monocromatico.
\end{proof}
\begin{prop}
Sia \(\overline{c} = \langle c_{i} : i <\omega \rangle\) sequenza di indiscernibili su \(A\). Esiste \(M\supseteq A\) tale che \(\overline{c}\) è una sequenza di indiscernibili su \(M\).
\end{prop}
\begin{proof}
Sia \(M\) un modello arbitrario.
Consideriamo \(\operatorname{EM-tp}(\overline{c}/A)\) completo, e definiamo
\begin{equation*}
p(\overline{x})\coloneqq \operatorname{EM-tp}(\overline{c}/M)\supseteq \operatorname{EM-tp}(\overline{c}/A)
\end{equation*}
Per il Teorema di Ehrenfeucht-Mostowski esiste \(\overline{b}\vDash p(\overline{x})\) sequenza di indiscernibili su \(M\).
Siccome \(\overline{b}\vDash \operatorname{EM-tp}(\overline{c}/A)\) allora
\begin{equation*}
\overline{b} \equivalentover{A} \overline{c}
\end{equation*}
e pertanto esiste \(f \in \operatorname{Aut}(\U/A)\) tale che \(f(\overline{b}) = c\), ovvero \(\overline{c}\) è indiscernibile su \(f[M] \supseteq A\).
\end{proof}
\subsection{Teorema di Hindman}
\label{sec:orgb86d36f}

Sia \(S\) un semigruppo, \(\L\) un linguaggio che estende il linguaggio dei gruppi moltiplicativi. Sia \(\U \succeq S\) modello mostro.

\begin{thm}
(Teorema di Hindman).
Per ogni colorazione finita di \(\N\) esiste un insieme infinito \(H \subseteq \N\) tale che
\begin{equation*}
\set{\sum_{a \in A} a \mid A \subseteq H\text{ finito}}
\end{equation*}
è monocromatico.
\end{thm}
\begin{oss}
Questo teorema dice ``quasi'' che esiste un insieme monocromatico chiuso per somma. Infatti, mancano le somme di lunghezza arbitraria dello stesso elemento.
\end{oss}

\begin{thm}
(Teorema di Hindman).
Per ogni colorazione finita di \(S\) esiste un insieme infinito ordinato \((H,<) \subseteq S\) tale che
\begin{equation*}
\set{\prod_{a \in A}^{\text{ordinato}} a \mid A \subseteq H\text{ finito}}
\end{equation*}
è monocromatico.
\label{thm:hindman}
\end{thm}

\begin{thm}
(Teorema di van der Waerden).
Per ogni colorazione finita di \(\N\) esistono \uline{progressioni aritmetiche}\footnote{Un progressione aritmetica è una sequenza finita della forma
\begin{equation*}
a, a+b, a+2b, \dots, a+nb.
\end{equation*}} di lunghezza arbitraria monocromatiche.
\end{thm}
Per \(\mathcal{A}, \mathcal{B} \subseteq \U\) si definisce
\begin{equation*}
\mathcal{A} * \mathcal{B} \coloneqq \set{a\cdot b\mid a \in \mathcal{A}, b \in \mathcal{B}, a \nonfork[S] b}.
\end{equation*}

\uline{Notazione}: abbrevio
\begin{align*}
\mathcal{A}*b &\coloneqq \mathcal{A}*\orbita(b/S)\\
a*\mathcal{B} &\coloneqq \orbita(a/S)*\mathcal{B}\\
a*b &\coloneqq \orbita(a/S)* \orbita(b/S)
\end{align*}

\begin{oss}
È possibile definire i sequenti tipi
\begin{align*}
x \nonfork[S] b &= \set{\lnot\varphi(x;b) \mid \varphi(S^{x}; b) = \emptyset, \varphi(x;z) \in \L(S)}\\
a \nonfork[S] z \equivalentover{S} b &= \set{\lnot\varphi(a;z) \mid \varphi(S^{x}; b) = \emptyset, \varphi(x;z) \in \L(S)}\\
x \nonfork[S] z \equivalentover{S} b &= \set{\lnot\varphi(x;z) \mid \varphi(S^{x}; b) = \emptyset, \varphi(x;z) \in \L(S)}
\end{align*}
\end{oss}

\begin{prop}
Se \(\mathcal{A}\) è tipo-definibile su \(S\) allora anche \(\mathcal{A}*b\) è tipo-definibile su \(S\).
\end{prop}
\begin{proof}
Voglio mostrare che
\begin{equation*}
\set{a,b'\mid a \in \mathcal{A}, b' \equivalentover{S} b, a \nonfork[S] b'}
\end{equation*}
è tipo-definibile.

Voglio quindi
\begin{equation*}
p(x;y) = x \in \mathcal{A} \land x \nonfork[S] y \equivalentover{S} b.
\end{equation*}
e pertanto \(\mathcal{A}*b\) è definito da
\begin{equation*}
q(z) = \exists x,y\ \big(z = x\cdot y \land p(x,y)\big).\qedhere
\end{equation*}
\end{proof}
\begin{oss}
È possibile pensare a \(*\) come una operazione:
\begin{align*}
*: \U/\equivalentover{S} \times \U/\equivalentover{S} &\longrightarrow \U/\equivalentover{S}\\
(a/\equivalentover{S}), (b/\equivalentover{S}) &\longmapsto ab/\equivalentover{S}
\end{align*}
continua a sinistra e in generale non a destra (nella \(S\)-topologia).
\end{oss}
\section{{\bfseries\sffamily TODO} Lezione 13 - \textit{<2025-11-04 Tue>}}
\label{sec:org5dc7650}

\subsection{Topological dynamics}
\label{sec:orgf7035cb}

Sia \(S\) semigruppo (operazione definibile in \(\L\)), \(\U \succeq S\).

Vogliamo definire una operazione \(*\) tra sottoinsiemi di \(\U\). [l'obiettivo finale è mostrare che \(*\) è operazione di semigruppo su \(\U/\equivalentover{S}\)]

Per \(\mathcal{A}, \mathcal{B} \subseteq \U\) si definisce
\begin{equation*}
\mathcal{A} * \mathcal{B} \coloneqq \set{a\cdot b\mid a \in \mathcal{A}, b \in \mathcal{B}, a \nonfork[S]b}.
\end{equation*}

\uline{Notazione}: abbrevio
\begin{align*}
\mathcal{A}*b &\coloneqq \mathcal{A}*\orbita(b/S)\\
a*\mathcal{B} &\coloneqq \orbita(a/S)*\mathcal{B}\\
a*b &\coloneqq \orbita(a/S)* \orbita(b/S)
\end{align*}

\begin{oss}
Se \(\mathcal{A},\mathcal{B}\) sono invarianti su \(S\), allora \(\mathcal{A}*\mathcal{B}\) è invariante su \(S\).
\end{oss}

\begin{thm}
Se \(\mathcal{A}\) è tipo-definibile (su \(S\)) allora \(\mathcal{A}*b\) è tipo-definibile per ogni \(b\).
\end{thm}

\begin{lem}
Se \(\mathcal{A},\mathcal{B},\mathcal{C}\) sono invarianti su \(S\) allora
\begin{equation*}
\mathcal{A}*(\mathcal{B}*\mathcal{C}) \subseteq (\mathcal{A}*\mathcal{B})*\mathcal{C}.
\end{equation*}
\end{lem}
\begin{proof}
Sia \(a\cdot b\cdot c \in \mathcal{A}*(\mathcal{B}*\mathcal{C})\). Allora
\begin{equation*}
b \nonfork[S]c, \qquad
a \nonfork[S]b\cdot c
\end{equation*}
Voglio mostrare che esistono \(a' \in \mathcal{A}\), \(b' \in \mathcal{B}\), \(c' \in \mathcal{C}\) tale che
\begin{equation*}
a' \nonfork[S]b', \qquad
a'\cdot b' \nonfork[S]c'
\end{equation*}
e che \(a\cdot b\cdot c = a'\cdot b' \cdot c'\).

È sufficiente che esiste \(a' \in \mathcal{A}\) tale che
\begin{equation*}
a' \nonfork[S]b, \qquad
a', b \nonfork[S]c
\end{equation*}
e che
\(a\cdot b\cdot c \equivalentover{S} a'\cdot b \cdot c\).

Per il Lemma~\ref{lem:propfork}(4) trovo \(a'\) tale che
\begin{equation*}
a \equivalentover{S,b\cdot c} a' \nonfork[S]b\cdot c, b,c.
\end{equation*}
(allora immediatamente \(a'\nonfork[S]b,c\) e \(a' \nonfork[S]b\)).

In particolare, siccome \(a' \nonfork[S]b,c\) e \(b \nonfork[S]c\) per il Lemma~\ref{lem:propfork}(3) si ottiene che \(a',b \nonfork[S]c\).

Resta da dimostrare l'elementare equivalenza. \(a' \in \mathcal{A}\) per invarianza, e siccome \(a \equivalentover{S,b\cdot c} a'\)
\begin{equation*}
a, b\cdot c \equivalentover{S} a', b \cdot c
\end{equation*}
e pertanto \(a\cdot b\cdot c \equivalentover{S} a'\cdot b \cdot c\).
\end{proof}
\begin{definizione}
La relazione \(\nonfork\) è \uline{stazionaria} se
\begin{equation*}
a \equivalentover{S} x \nonfork[S]b
\end{equation*}
è un tipo completo su \(S,b\).
\end{definizione}
\textbf{D'ora in avanti \(\nonfork\) è assunta stazionaria}.

\uline{Conseguenza}: se \(a\nonfork[S]b\) e \(a' \equivalentover{S} a\) allora \(a' \nonfork[S]b\).

\begin{lem}
Se \(a \nonfork[S]b\) e \(a' \equivalentover{S} a\) allora
\(a\cdot b \equivalentover{S} a' \cdot b\).
\end{lem}
\begin{proof}
Infatti \(a,b \equivalentover{S} a',b\).
\end{proof}
\begin{cor}
Se \(a \nonfork[S]b\), \(a' \nonfork[S]b'\),
\(a' \equivalentover{S} a\), \(b' \equivalentover{S} b\)
allora
\(a\cdot b \equivalentover{S} a'\cdot b'\).
\end{cor}
\begin{proof}
Se per assurdo \(a\cdot b \not\equivalentover{S} a'\cdot b'\) sia \(fb'=b\) per \(f \in \Aut(\U/S)\), e quindi
\(a\cdot b \not\equivalentover{S} f(a'\cdot b')\)
e quindi
\(a\cdot b \not\equivalentover{S} f(a') \cdot f(b') = a''\cdot b\), dove
\(a'' \equivalentover{S} a\).
\end{proof}
\begin{cor}
\(a*b\) è un'orbita.
\end{cor}

\begin{thm}
Se \(\mathcal{A},\mathcal{B},\mathcal{C}\) sono invarianti su \(S\) allora
\begin{equation*}
\mathcal{A}*(\mathcal{B}*\mathcal{C}) = (\mathcal{A}*\mathcal{B})*\mathcal{C}.
\end{equation*}
\end{thm}
\begin{proof}
Considerare il caso \(\mathcal{A},\mathcal{B}, \mathcal{C}\) orbite.
\end{proof}
\def\defaultnonforkmodel{M}

Questo teorema vale in generale, non solo per \(S\) sottogruppo.
\begin{prop}
Se per ogni \(\varphi(x) \in \L(\U)\) esiste \(\psi(x) \in\L(M)\) tale che
\begin{equation*}
\varphi(M^{x}) = \psi(M^{x}),
\end{equation*}
dove \(\varphi(M^{x}) = \varphi(\U^{x})\cap M^{x}\), \uline{allora} \(\nonfork[M]\) è stazionaria:
\begin{equation*}
a \equivalentover{M} x \nonfork[M] b.
\end{equation*}
è un tipo completo su \(M,b\).
\end{prop}
\begin{proof}
Sia \(\varphi(x) = \varphi(x;b)\) per \(\varphi(x;z) \in \L\) e \(b \in \U^{z}\).

Siano \(a_{1},a_{2} \in \U^{x}\) tali che
\(a_{1} \equivalentover{M} a_{2}\) e \(a_{1} \nonfork[S]b\), \(a_{2} \nonfork[S]b\).
Voglio mostrare che \(\varphi(a_{1},b) \iff \varphi (a_{2},b)\).

Sappiamo che \(\varphi(a_{i},b) \iff \psi(a_{i})\) perché \(\varphi(M^{x}) = \psi(M^{x})\).

Siccome \(a_{1}\equivalentover{M} a_{2}\) allora \(\psi(a_{1}) \iff \psi(a_{2})\).
\end{proof}

\begin{oss}
Per ottenere che \(\nonfork\) è 1-stazionaria\footnote{Ovvero il tipo è completo per \(\card{x}=1\).} basta aggiungere al linguaggio un predicato per ogni sottoinsieme di \(S\).
\end{oss}
\begin{definizione}
\(\mathcal{A}\neq\emptyset\) è idempotente se \(\mathcal{A}*\mathcal{A} \subseteq \mathcal{A}\).
\end{definizione}
\begin{oss}
Se \(\mathcal{A}\) è idempotente e \(\mathcal{B} \subseteq \mathcal{A}\) allora \(\mathcal{A}*\mathcal{B}\) è idempotente.
\end{oss}
\def\defaultnonforkmodel{S}
\begin{thm}
(Ellis).
Se \(\mathcal{A}\) è tipo-definibile e idempotente, allora esiste \(b \in \mathcal{A}\) tale che \(\orbita(b/S)\) è idempotente.
\end{thm}
\begin{proof}
Per compattezza possiamo assumere che \(\mathcal{A}\) sia \(\subseteq\)-minimale tra gli idempotenti tipo-definibili.

Prendiamo \(b \in \mathcal{A}\). Allora \(\mathcal{A}*b = \mathcal{A}\) per minimalità.
\begin{equation*}
\mathcal{A}'\coloneqq
\set{a \in \mathcal{A} \mid a \nonfork[S]b, a\cdot b \equivalentover{S} b}.
\end{equation*}
Questo insieme è tipo-definibile su \(S,b\), ed è invariante su \(S\). (quindi è tipo-definibile su \(S\)).

Si ha che \(\mathcal{A}' \neq \emptyset\) poiché \(\mathcal{A}*b=\mathcal{A}\). Inoltre \(\mathcal{A}'\) è idempotente.

Quindi \(\mathcal{A}'=\mathcal{A}\), e quindi
\(\orbita(b/S) \subseteq \mathcal{A}'\).

Quindi \(\orbita(b/S)\) è idempotente.
\end{proof}
\def\fp{\operatorname{fp}}
Sia \(S=\big(\N, \cdot, \parti(\N)\big)\), \(\U\succeq S\), \(\overline{a}\) tupla in \(\U\) di lunghezza \(\le \omega\).
\begin{equation*}
\fp(\overline{a}) \coloneqq \set{a_{i_{1}}\cdot \dots \cdot a_{i_{n}}\mid i_{0}<\dots<i_{n} <\card{\overline{a}}}.
\end{equation*}


\begin{thm}
(Teorema di Hindman).
Per ogni colorazione finita di \(\N\) esiste \(\overline{a} \in \N^{\omega}\) tale che \(\fp(\overline{a})\) è monocromatico.
\end{thm}
\def\defaultnonforkmodel{\N}
\begin{proof}
Sia
\begin{equation*}
\mathcal{A} \coloneqq \set{a \in \U\mid a\neq 0}
\end{equation*}
che è definibile e idem-potente. Quindi contiene \(b_{0}\) tale che \(\orbita(b_{0}/\N)\) è idempotente.

Sia \(\overline{b} = \langle b_{i} : i<\omega \rangle\) sequenza di coeredi su \(N\).

Si noti che \(\fp(\overline{b})\) è monocromatica (ha colore 1). Infatti
\begin{equation*}
b_{0}\cdot b_{1} \equivalentover{\N} b_{0},\qquad
b_{0}\cdot b_{1} \equivalentover{\N} b_{i}
\end{equation*}
poiché \(b_{1}\nonfork[S]b_{0}\).

Definiamo per induzione una sequenza \(\overline{a} = \langle a_{i}: i<\omega\rangle\) tale che
\begin{equation*}
\fp(\overline{a} \restricted{n}, b_{1},b_{0})
\end{equation*}
ha colore \(1\). (definibile da \(\varphi(a\restricted{n}, b_{1},b_{0})\))

Lo assumo per ipotesi induttiva. Voglio trovare \(a_{n}\). Siccome \(b_{1}\nonfork[S]b_{0}\) allora esiste \(a_{n} \in \N\) tale che
\begin{equation*}
\varphi(a \restricted{n}, a_{n}, b_{0})
\end{equation*}
e quindi \(\fp(a \restricted{n}, a_{n}, b_{0})\) ha colore 1.
Voglio arrivare a mostrare che \(\fp(a \restricted{n}, a_{n}, b_{1}, b_{0})\) ha colore 1.

Siccome \(b_{1}\equivalentover{\N} b_{0}\), sicuramente
\begin{equation*}
a_{i_{1}}\cdot \dots a_{i_{n}} \cdot b_{0} \text{%
ha lo stesso colore di %
}a_{i_{1}}\cdot \dots a_{i_{n}} \cdot b_{1}
\end{equation*}
Se per assurdo \(a_{i_{1}}\cdot \dots a_{i_{n}} \cdot b_{1}\cdot b_{0}\) ha colore diverso, siccome \(b_{1}\nonfork[S] b_{0}\) ho
\begin{equation*}
b_{1}\cdot b_{0} \equivalentover{\N} b_{0}
\end{equation*}
per idempotenza quindi anche
\(a_{i_{1}}\cdot \dots a_{i_{n}} \cdot b_{0}\)
ha un colore diverso. Assurdo.
\end{proof}
\section{{\bfseries\sffamily TODO} Lezione 14 - \textit{<2025-11-05 Wed>}}
\label{sec:org597df12}

\def\D{\mathcal{D}}
\def\DD{\bm{\D}}
\def\U{\mathcal{U}}
\def\eq{{\rm eq}}
\def\Ueq{\U^\eq}
\def\L{\mathcal{L}}
\def\orbita{\mathcal{O}}
\def\Aut{\operatorname{Aut}}
\def\tc{\mid}
\def\tp{\operatorname{tp}}
\def\<{\langle}
\def\>{\rangle}
\def\b{\bm{b}}

\def\restricted#1{\,\mathord{\upharpoonright}{{\scriptstyle #1}}}
\def\equivalentover#1{\mathrel{\equiv_{ #1 }}}

%% NON FORKING
\def\nonforkSymbol{%
\mathbin{\raise1.8ex%
\rlap{\kern0.6ex\rule{0.6ex}{0.1ex}}%
\rlap{\kern1.1ex\rule{0.1ex}{1.9ex}}\raise-0.3ex\hbox{$\smile$}}}
\def\defaultnonforkmodel{M}
\def\nonfork{\nonforkSymbol}
\renewcommand{\nonfork}[1][\defaultnonforkmodel]{%
\mathrel{\nonforkSymbol_{#1}}}

\begin{thm}
(Teorema di van der Waerden).
Per ogni colorazione finita di \(\N\) e per ogni \(k<\omega\) esistono \(m,n < \omega\) tali che la sequenza\footnote{Questa è una progressione aritmetica}
\begin{equation*}
\langle n + m \cdot i \mid i<k \rangle
\end{equation*}
è monocromatica.
\label{thm:vdWaerden}
\end{thm}

\begin{thm}
(Teorema di Hales-Jewett).
Sia \(A\) insieme finito. Per ogni colorazione di \(A^{<\omega}\) esiste \(w(x) \in (A\cup \set{x})^{<\omega}\) tale che
\begin{equation*}
\set{w(d) \mid d \in A}
\end{equation*}
è monocromatico (dove con \(w(d)\) si intende sequenza in \(A^{<\omega}\) ottenuta sostituendo ogni occorenza di \(x\) in \(w(x)\) con \(d\)).
\label{thm:HalesJewett}
\end{thm}

\begin{thm}
(Teorema di Hales-Jewett - \emph{versione di Koppelberg}).
Sia \(S\) un semigruppo, \(C \subseteq S\) sottosemigruppo \emph{nice}. Per ogni colorazione di \(C\) e per ogni insieme finito di retrazioni di \(S\) in \(C\):
\begin{equation*}
\set{\sigma_{i} : S\to C \tc i<k}
\end{equation*}
esiste un \(w \in S\) tale che
\begin{equation*}
\set{\sigma_{i}(w) \tc i<k}
\end{equation*}
è monocromatico.
\label{thm:HalesJewettdiK}
\end{thm}

\begin{definizione}
Un sottosemigruppo \(C \subseteq S\) si dice \emph{nice} se per ogni \(a,b \in S\)
\begin{equation*}
a\cdot b \in C \IMPLICA a,b \in C.
\end{equation*}
\end{definizione}
\begin{esempio}
Si consideri come \(S \coloneqq (A\cup\set{x})^{<\omega}\) che ha come operazione la concatenazione. \(C\coloneqq A^{<\omega}\) è \emph{nice}.
\end{esempio}
\begin{definizione}
Un omomorfismo \(f:S\to C\) si dice \uline{retrazione}
\begin{equation*}
f \restricted{C} = \Id_{C}.
\end{equation*}
\end{definizione}
\begin{esempio}
Si consideri come \(S \coloneqq (A\cup\set{x})^{<\omega}\) che ha come operazione la concatenazione, e sia \(C\coloneqq A^{<\omega}\).

Dato \(d \in A\) considero
\begin{align*}
\sigma_{d}: S &\longrightarrow C\\
w(x) &\longmapsto w(d).
\end{align*}
\end{esempio}

\begin{proof}
(Teorema~\ref{thm:HalesJewettdiK}
\(\Rightarrow\)
Teorema~\ref{thm:vdWaerden}).

Sia \(S = \set{n+m\cdot x \mid n,m \in \N}\), e sia \(C = \langle \N,+\rangle\) un sottosemigruppo \emph{nice}.
Dato \(k<\omega\) sia
\begin{align*}
\sigma_{i}: S &\longrightarrow C\\
n+m\cdot x &\longmapsto n + m\cdot i.
\end{align*}

Per il Teorema~\ref{thm:HalesJewettdiK} esiste \(n'+m'\cdot x\) tale che
\begin{equation*}
\set{\sigma_{i}(n'+m'\cdot x) \tc i<k} = \set{n' + m'\cdot i \mid i < k}
\end{equation*}
è monocromatico.
\end{proof}

\def\defaultnonforkmodel{S}
Dato \(S\) semigruppo, sia \(\U \succeq S\) modello mostro nel linguaggio che contiene \(\cdot\) e un simbolo per ogni sottoinsieme di \(S\).
Allora la relazione \(\nonfork\) è \(1\)-stazionaria.

Allora l'operzione \(*\) tra elementi di \(\U/\equivalentover{S}\) è un'operazione di semigruppo.

\begin{oss}
\(\U/\equivalentover{S}\) è uno spazio topologico compatto.
\begin{quote}
\(*\) è continua a destra sse \(A* b\) è tipo-definibile per ogni \(A\) tipo-definibile.
\end{quote}
\end{oss}

\begin{definizione}
Un \(\emptyset \neq C \subseteq \U\) si dice:
\begin{itemize}
\item ideale sinistro se \(\U * C \subseteq C\);
\item ideale destro se \(C * \U \subseteq C\);
\item ideale bilatero se è sia ideale destro che sinistro.
\end{itemize}
\end{definizione}

\begin{prop}
Ogni ideale sinistro tipo-definibile contiene un ideale sinistro \(\subseteq\)-minimale e questo è tipo-definibile.
\end{prop}
\begin{proof}
Sia \(C\) ideale sinistro tipo-definibile. Sia \(b \in C\).
Allora \(C*b\) è un ideale sinistro tipo-definibile e \(C*b \subseteq C*C \subseteq U*C \subseteq C\).

Quindi esiste \(C' \subseteq C\) tale che \(C'* b = C'\) per ogni \(b \in C'\).
Affermo che \(C'\) è minimale. Questo è tipo-definibile poiché intersezione di chiusi.

Sia \(C'' \subseteq C'\) ideale sinistro: sia \(b \in C'' \subseteq C'\):
\begin{equation*}
C' = C' * b \subseteq \U*b \subseteq \U * C'' \subseteq C''.\qedhere
\end{equation*}
\end{proof}

\uline{Nota}: \(C*C \subseteq U*C \subseteq C\), quindi \(C\) è idempotente.

\begin{lem}
Sia \(M\) ideale sinistro minimale tipo-definibile. Siano \(u,v \in M\) idempotenti.
\begin{enumerate}
\item \(a*u \equivalentover{S} a\) per ogni \(a \in M\).
\item \(*\) è una operazione di gruppo su \(u*M\) con identità \(u\).
\item La mappa
\begin{align*}
 f_{v}: u*M &\longrightarrow v*M\\
 a&\longmapsto v*a
\end{align*}
è un isomorfismo di gruppi.
\item Se \(u\neq v\) allora \((u * M) \cap (v * M) = \emptyset\).
\item \(M = \bigcup_{\substack{u \in M\\ \text{idempotente}}} u*M\).
\end{enumerate}
\end{lem}
\begin{proof}
\begin{enumerate}
\item \(M*u=M\), allora esiste \(a' \in M\) tale che \(a'*u = a\). Quindi \begin{equation*}
     a * u = a' * u * u = a' * u = a.
\end{equation*}
\item Sia \(a \in u*M \subseteq M\), allora \(M*a = M\), quindi
\begin{equation*}
 u*M*a = u*M
\end{equation*}
e quindi esiste \(a' \in M\) tale che \(u*a'*a = u\) e quindi \(u*a'\) è l'inverso sinistro di \(a\).
\item Mostriamo che
\begin{align*}
 f_{u}: v*M &\longrightarrow u*M\\
 a&\longmapsto u*a
\end{align*}
è inversa:
\begin{equation*}
 f_{u}\,f_{v} (u*a) = \parentesi{u}{u * v} * u * a = u * u * a = u * a.
\end{equation*}
Inoltre si ha che
\begin{equation*}
 f_{v}(a*b) = v * a * b = v * a * v * b = f_{v}(a) * f_{v}(b).
\end{equation*}
\item Per assurdo sia \(a \in (u * M) \cap (v * M)\).
Sia \(a' \in u*M\) tale che \(a*a' = u\) (poiché \(u*M\) è un gruppo).

Siccome \(a \in v*M\) allora
\begin{equation*}
 u = a * a' \in v * \parentesi{=M}{M * a'} = v * M
\end{equation*}
e quindi \(u=v\).
\item Sia \(a \in M\) e sia \(\set{c \in M \mid c*a = a}\). Questo è un idempotente tipo-definibile.

Quindi esiste \(u \in \set{c \in M \mid c*a = a}\) elemento idempotente e quindi
\begin{equation*}
 u * a = a
\end{equation*}
e quindi \(a \in u*M\).
\end{enumerate}
\end{proof}

\begin{oss}
Partendo da \(S\) semigruppo abbiamo ottenuto \(\U/\equivalentover{S}\), definendo \(*\) operazione di semigruppo continua a dx.
Questo è il semigruppo di Ellis. (Stone-Check compactification)

Considerando ora \(u*M\) si è ottenuto un gruppo con l'operazione \(*\). Questo non dipende da \(M\) e \(u\).  Questo è il gruppo di Ellis.
\end{oss}
\section{{\bfseries\sffamily DONE} Lezione 15 - \textit{<2025-11-11 Tue>}}
\label{sec:org0959ebe}
\def\D{\mathcal{D}}
\def\DD{\bm{\D}}
\def\U{\mathcal{U}}
\def\eq{{\rm eq}}
\def\Ueq{\U^\eq}
\def\L{\mathcal{L}}
\def\orbita{\mathcal{O}}
\def\Aut{\operatorname{Aut}}
\def\tc{\mid}
\def\tp{\operatorname{tp}}
\def\<{\langle}
\def\>{\rangle}
\def\b{\bm{b}}

\def\restricted#1{\,\mathord{\upharpoonright}{{\scriptstyle #1}}}
\def\equivalentover#1{\mathrel{\equiv_{#1}}}

%% NON FORKING
\def\nonforkSymbol{%
\mathbin{\raise1.8ex%
\rlap{\kern0.6ex\rule{0.6ex}{0.1ex}}%
\rlap{\kern1.1ex\rule{0.1ex}{1.9ex}}\raise-0.3ex\hbox{$\smile$}}}
\def\defaultnonforkmodel{M}
\def\nonfork{\nonforkSymbol}
\renewcommand{\nonfork}[1][\defaultnonforkmodel]{%
\mathrel{\nonforkSymbol_{#1}}}

\def\defaultnonforkmodel{S}
Sia \(S\) un semigruppo,
\(\L= \set{\cdot} \cup \parti{S} \cup\text{altro}\) e sia \(\U\) modello saturo tale che \(S \preceq \U\).

Si ha \(*\) operazione di semigruppo su \(\U/\equivalentover{S}\) con
\begin{equation*}
a*b = \set{a' \cdot b' \mid %
a \equivalentover{S} a' \nonfork b' \equivalentover{S} b}.
\end{equation*}
L'operazione \(*\) è continua a sinistra, ovvero
\begin{quote}
\(x*b\) è un tipo (su \(S\)) per ogni \(b\).
\end{quote}

\begin{oss}
Esistono idempotenti, ovvero \(b\) tali che
\begin{equation*}
b*b = \orbita(b/S).
\end{equation*}
\end{oss}

Abbiamo introdotto il concetto di \textbf{ideale sinistro}:
\begin{equation*}
M \subseteq \U\text{ ideale sinistro} \IFF \U*M \subseteq M.
\end{equation*}

\begin{prop}
LSASE:
\begin{enumerate}
\item \(M\) ideale sinistro \uline{minimale};
\item per ogni \(b \in M\): \(M * b = M\).
\end{enumerate}
\end{prop}

\begin{prop}
Sia \(M\) ideale sinistro minimale, \(a \in \U\). Allora \(M*a\) è ideale sinistro minimale.
\end{prop}
\begin{proof}
Sia \(b*a \in M*a\) con \(b \in M\) arbitrario. Basta mostrare
\begin{equation*}
M*a*b*a = M*a
\end{equation*}
Ma \(a*b \in M\) e pertanto \(M*a*b = M\) per minimalità di \(M\).
\end{proof}
\begin{prop}
Sia \(M\) ideale sinistro minimale, \(a \in \U\) idempotente, allora
\(a*M*a\) contiene un idempotente.
\end{prop}
\begin{proof}
Sia \(u \in M*a\) idempotente. Affermo che \(a*u\) è idempotente.
\begin{equation*}
a*u*a*u = a*(u*a)*u = a * u * u = a*u. % Hello
\qedhere
\end{equation*}
\end{proof}
\begin{thm}
(Teorema di Hales-Jewett).
Sia \(S\) un semigruppo e sia \(C \subseteq S\) un \uline{sottosemigruppo \emph{nice}}\footnote{Ovvero se \(a\cdot b \in C\) allora \(a \in C\) e \(b \in C\).}. Dato \(k \in \omega\) sia
\begin{equation*}
\Sigma\coloneqq \set{%
\sigma_{i}: S\to C \text{ omomorfismo}%
\mid \sigma_{i} \restricted{C} = \Id_{C}, i<k}
\end{equation*}
Per ogni colorazione finita di \(C\) esiste \(a \in S\setminus C\) tale che
\begin{equation*}
\set{\sigma_{i}(a) \mid i < k}
\end{equation*}
è monocromatico.
\label{thm:halesjewett:bis}
\end{thm}
\begin{lem}
Per ogni \(\sigma \in \Sigma\), per ogni \(a,b \in \U\) e
\(A \subseteq \U\):
\begin{enumerate}
\item \(\sigma\big(\orbita(a/A)\big) = \orbita(\sigma a / A)\);
\item \(\sigma(a*b) = \sigma(a)*\sigma(b)\).
\end{enumerate}
\end{lem}
\def\C{\mathcal{C}}
\begin{proof}
(del Teorema~\ref{thm:halesjewett:bis}).
Sia \(\L=\set{\cdot}\cup\parti{S}\cup \Sigma\). Sia \(\U\succeq S\) modello mostro, e sia \(\C = \phi(\U^{x})\), dove
\begin{equation*}
\phi(x):\qquad x \in C.
\end{equation*}
\begin{itemize}
\item Dal momento che la proprietà \emph{nice} è esprimibile al prim'ordine, anche \(\mathcal{C}\) è \emph{nice}. Pertanto \(\U\setminus\C\) è ideale sinistro.
\item Sia \(M \subseteq \U\setminus\C\) ideale sinistro minimale tipo-definibile, e sia \(N \subseteq \C\) ideale sinistro di \(\C\)\footnote{Ovvero \(\C*N \subseteq N\).} minimale tipo-definibile.
\item Sia \(v \in N\) idempotente, e sia \(u \in v*M*v\) idempotente.
\item Per ogni \(\sigma \in \Sigma\)
\begin{equation}
\sigma(u) \in \sigma(v)*\sigma(M)*\sigma(v) = v*\sigma(M)*v %
\label{eq:vsigmaMv}
\end{equation}
con \(\sigma(u)\) idempotente.
\item \(\sigma(M)\) è ideale sinistro di \(\C\), in quanto \(\sigma\) è l'identità su \(\C\):
\begin{equation*}
\C*\sigma(M) = \sigma(\C*M) \subseteq \sigma(M).
\end{equation*}
\item Si ha che \(\sigma(M)*v \subseteq N\) e per minimalità di \(N\):
\begin{equation*}
\sigma(M)*v = N.
\end{equation*}
\item Applicando alla~\eqref{eq:vsigmaMv}: \(\sigma(u) \in v*N\), ma per un lemma già visto \(v*N\) è un gruppo con identità \(v\), e quindi \(\sigma(u) \equivalentover{S} v\) per ogni \(\sigma \in \Sigma\).
\end{itemize}

Dunque
\begin{equation*}
\U\vDash \exists x \in (\U\setminus\C)\ %
\bigg[ %
\bigwedge_{\sigma \in \Sigma}%
\text{``}\sigma(x)\text{ ha colore 1''}%
\bigg].
\end{equation*}
e pertanto vale anche per \(S\).
\end{proof}
\begin{thm}
(Teorema di van der Waerden).
Per ogni colorazione di \(\N\) e per ogni \(k<\omega\) esiste una progressione aritmetica di lunghezza \(k\) monocromatica.
\end{thm}
\begin{thm}
(Teorema di Szeméredi).
Per ogni \(A \subseteq \N\) ``abbastanza grande'' e per ogni \(k<\omega\) esiste una progressione aritmetica di lunghezza \(k\) contenuta in \(A\).
\end{thm}
\(A\) è ``abbastanza grande'' se ha \uline{densità di Banach positiva}, ovvero
\begin{equation*}
\limsup_{\substack{%
\card{I}\to \infty\\%
 I\text{ intervallo}}} %
\frac{\card{A\cap I}}{\card{I}} > 0.
\end{equation*}
\section{{\bfseries\sffamily DONE} Lezione 16 - \textit{<2025-11-12 Wed>}}
\label{sec:org9a49e11}

\def\U{\mathcal{U}}
\def\L{\mathcal{L}}
\def\X{\mathcal{X}}
\def\Z{\mathcal{Z}}
\def\D{\mathcal{D}}
%
\def\eq{{\rm eq}}
\def\Ueq{\U^\eq}
%
\def\orbita{\mathcal{O}}
\def\Aut{\operatorname{Aut}}
%
\def\tc{\mid}
\def\tp{\operatorname{tp}}
\def\EMtp{\operatorname{EM}\text{-}\operatorname{tp}}
\def\<{\langle}
\def\>{\rangle}
%
\def\restricted#1{\,\mathord{\upharpoonright}{{\scriptstyle #1}}}
\def\equivalentover#1{\mathrel{\equiv_{ #1 }}}

%% NON FORKING
\def\nonforkSymbol{%
\mathbin{\raise1.8ex%
\rlap{\kern0.6ex\rule{0.6ex}{0.1ex}}%
\rlap{\kern1.1ex\rule{0.1ex}{1.9ex}}\raise-0.3ex\hbox{$\smile$}}}
\def\defaultnonforkmodel{M}
\def\nonfork{\nonforkSymbol}
\renewcommand{\nonfork}[1][\defaultnonforkmodel]{%
\mathrel{\nonforkSymbol_{#1}}}
\subsection{Relazioni stabili}
\label{sec:org3692fb2}

Sia \(\pi \subseteq \X\times\Z\) una relazione. Si indica indifferentemente
\begin{equation*}
(x,z) \in \pi,\qquad \pi(x,z).
\end{equation*}

\begin{definizione}
\(\pi(x,z)\) è \uline{instabile} (o ha la \emph{order property}) se per ogni \(m<\omega\) esiste una sequenza
\(\< a_{i}, b_{i} : i<m \>\)
tale che
\begin{equation}
i<j<m \IMPLICA \pi(a_{i}, b_{j}) \land \lnot\pi(a_{j}, b_{i}). %
\label{eq:relinstabile}
\end{equation}
Questa sequenza prende il nome di \uline{scaletta di lunghezza \(m\) per \(\pi\)} (o \emph{ladder sequence}).
\end{definizione}

Si potrebbe sostituire la~\eqref{eq:relinstabile} con una delle seguenti:
\begin{enumerate}
\item \(i < j < m\) sse \(\pi(a_{i}, b_{j})\);
\item \(i \le j < m\) sse \(\pi(a_{i}, b_{j})\)
\end{enumerate}
Si ha che \(1.\oldiff 2.\), ma non sono equivalenti
alla~\eqref{eq:relinstabile}

\begin{prop}
LSASE:
\begin{enumerate}
\item \(\pi\) è instabile;
\item per ogni \(m<\omega\) esiste \(B \subseteq \Z\) e
\(\< a_{i} \mid i < m \>\) (con  \(a_{i} \in \X\))
tale che
\begin{equation*}
 \pi(a_{0}, B) \subsetneq \pi(a_{1}, B) \subsetneq \dots \subsetneq \pi(a_{m-1},B).
\end{equation*}
dove \(\pi(a, B) = \set{b \in B \mid \pi(a,b)}\).
\end{enumerate}
\end{prop}
\begin{prop}
Sia \(\psi(x;z) \in \L(A)\). LSASE:
\begin{enumerate}
\item \(\psi(x;z)\) è stabile;
\item non esistono scalette di lunghezza \(\omega\);
\item per ogni \(\< a_{i}, b_{i} \mid i <\omega \>\) sequenza di \(A\)-indiscernibili
\begin{equation*}
 \psi(a_{0},b_{1}) \iff \psi(a_{1}, b_{0}).
\end{equation*}
\end{enumerate}
\end{prop}
\begin{proof}
(\(1.\Leftrightarrow 2.\)): Per compattezza.

(\(2.\Leftrightarrow 3.\)): EM thm.
\end{proof}
\begin{prop}
\(\pi(x,z)\) è stabile sse \(\pi(x, x';z, z')\) stabile per ogni \(x',z'\).
\end{prop}
\begin{prop}
Combinazioni Booleane di relazioni stabili sono stabili.
\end{prop}
\begin{proof}
(\(\lnot\)): Se \(\< a_{i}, b_{i} \mid i<m \>\) è una scaletta per \(\pi\)
\begin{equation*}
i<j<m \IMPLICA \pi(a_{i}, b_{j}) \land \lnot\pi(a_{j}, b_{i}). %
\end{equation*}
allora \(\< a_{m-i+1}, b_{m-i+1} \mid i<m \>\) è una scaletta per \(\lnot \pi\).

(\(\land\)): \uline{Caso facile}: siano \(\pi(x;z), \sigma(x;z) \in \L(A)\). Si dimostra che
\begin{equation*}
\pi(x;z), \sigma(x;z) \text{ stabili} \IMPLICA
\pi(x;z)\land\sigma(x;z)\text{ stabile}
\end{equation*}
ovvero
\begin{equation*}
\pi(x;z)\land\sigma(x;z)\text{ instabile} \IMPLICA
\pi(x;z) \text{ instabile}\lor \sigma(x;z) \text{ instabile}.
\end{equation*}

Si supponga che \(\< a_{i}, b_{i} \mid i<\omega \>\) sequenza di \(A\) indiscernibili tali che
\begin{equation*}
\pi(a_{0},b_{1}) \land \sigma(a_{0},b_{1}) \land
\lnot\big[\pi(a_{1},b_{0}) \land \sigma(a_{1},b_{0})\big]
\end{equation*}
ovvero
\begin{equation*}
\pi(a_{0},b_{1}) \land \sigma(a_{0},b_{1}) \land
\big[\lnot\pi(a_{1},b_{0}) \lor \lnot\sigma(a_{1},b_{0})\big]
\end{equation*}
\begin{itemize}
\item Se \(\lnot\pi(a_{1},b_{0})\) è vera, allora \(\pi\) è instabile;
\item se \(\lnot\sigma(a_{1},b_{0})\) è vera, allora \(\sigma\) è instabile.
\end{itemize}

(\(\land\)): \uline{Caso generale}. Siano \(\pi,\sigma\) relazioni arbitrarie. Si vuole dimostrare
\begin{equation*}
\pi(x;z)\land\sigma(x;z)\text{ instabile} \IMPLICA
\pi(x;z) \text{ instabile}\lor \sigma(x;z) \text{ instabile}.
\end{equation*}
Sia \(\< a_{i}, b_{i} \mid i<k \>\) scaletta per
\(\pi(x;z)\land\sigma(x;z)\),
con \(k\) ``abbastanza grande'', ovvero se \(i<j<k\) si ha
\begin{equation*}
\pi(a_{i},b_{j}) \land \sigma(a_{i},b_{j}) \land
\big[\lnot\pi(a_{j},b_{i}) \lor \lnot\sigma(a_{j},b_{i})\big].
\end{equation*}
\begin{itemize}
\item Coloro \(i<j\) di verde se \(\lnot\pi(a_{j},b_{i})\);
\item Coloro \(i<j\) di blu se \(\lnot\sigma(a_{j},b_{i})\).
\end{itemize}

Per il \href{20251029171516-teorema_di_ramsey.org}{Teorema di Ramsey} per ogni \(m\) esiste \(k\) (ovvero quel \(k\) abbastanza grande) tale che per ogni colorazione di \(\set{0,\dots,k-1}^{(2)}\) esiste \(H \subseteq \set{0,\dots,k-1}\) tale che \(\card{H}\ge m\) e \(H^{(2)}\) è monocromatico.
\begin{itemize}
\item Se \(H^{(2)}\) è blu, allora \(\sigma\) ammette scaletta di lunghezza \(m\);
\item se \(H^{(2)}\) è verde, allora \(\pi\) ammette scaletta di lunghezza \(m\).
\end{itemize}
Segue che almeno una dei due è instabile.
Infatti, se \(\sigma\) non ammette scalette di lunghezza \(m'\), allora non ammette scalette di lunghezza \(m>m'\).
Pertanto, per ogni \(m>m'\) si ha che \(\pi\) ammette scaletta di lunghezza \(m\).
Segue che \(\pi\) ammette scalette di lunghezza \(m\) per ogni \(m<\omega\).
\end{proof}
\begin{definizione}
\(\D \subseteq \Z\) è \uline{approssimabile da \(\pi\)} se per ogni
\(B \subseteq \Z\) finito esiste \(a \in \X\) tale che
\begin{equation*}
\pi(a;B) =\set{b \in B\mid \pi(a,b)} = \D \cap B.
\end{equation*}
\end{definizione}
\begin{oss}
Considero \(\parti{\Z}\) come spazio topologico, identificando con \(2^{\Z}\) con la topologia prodotto.
Quindi gli intorni di base di \(\parti{\Z}\) hanno la forma, per \(B\) finito e \(C \subseteq B\)
\begin{equation*}
\set{A \in \parti{\Z} \mid A\cap B = C}.
\end{equation*}
\(\D\) è approssimabile da \(\pi\) sse \(\D\) è nella chiusura topologica di
\begin{equation*}
\set{\pi(a; \Z) \mid a \in X}.
\end{equation*}
\end{oss}
\begin{thm}
Se \(\pi \subseteq \X \times \Z\) è stabile e \(\D\) è approssimabile da \(\pi\), allora esistono \(\< a_{ij} \mid i,j < m\>\) in \(\X\) tali che
\begin{equation*}
\D = \bigcup_{i=1}^{m} \bigcap_{j=1}^{m} \pi(a_{i}, Z).
\end{equation*}
\end{thm}
\begin{prop}
Sia \(\psi(x;z) \in \L(A)\), \(\D \subseteq \U^{z}\). LSASE:
\begin{enumerate}
\item \(\D\) approssimabile da \(\psi\);
\item \(\D\) è esternamente definibile da \(\psi\)\footnote{\(\D\) è esternamente definibile da \(\psi\) se vale una selle seguenti affermazioni equivalenti:
\begin{enumerate}
\item esiste \(\nonstandard{\U} \succ \U\), \(\nonstandard{a} \in \nonstandard{\U}\) tale che
\begin{equation*}
\D = \psi(\nonstandard{a}, \nonstandard{\U}^{z}) \cap \U^{z}
\end{equation*}
\item esiste \(p(x) \in S(\U)\) tale che
\begin{equation*}
\D =\D_{p,\psi} = \set{b \in \U^{z} \mid \psi(x;b) \in p}
\end{equation*}
\end{enumerate}}.
\end{enumerate}
\end{prop}
\begin{proof}
(b. \(\Rightarrow 1.\)):
Sia \(B \subseteq \U^{z}\) finito.
Sia \(a\vDash p(x)\restricted{A,B}\).
Allora \(\pi(a, B) = \D\cap B\), ovvero che
\begin{equation*}
\forall b \in B\ %
\big[\pi(a,b) \implies b \in \D\big]%
\land %
\big[\lnot\pi(a,b) \implies b \notin \D\big]
\end{equation*}
Questo è ovvio poiché \(p\) è completo e \(a\vDash p(x) \restricted{A,B}\).

(\(1.\Rightarrow\) b.): Sia
\begin{equation*}
S(\U)\ni p(x) \supseteq \set{\pi(x;b) \mid b \in \D} \cup \set{\lnot\pi(x;b) \mid b \notin \D}.
\end{equation*}
Sia \(B \subseteq \U^{z}\) finito. Allora
\begin{equation*}
p(x) \restricted{B} \supseteq \set{\pi(x;b) \mid b \in \D\cap B} \cup \set{\lnot\pi(x;b) \mid b \in B\setminus\D}
\end{equation*}
Ma se \(a\) è tale che \(\pi(a,B) = \D\cap B\), allora \(a\vDash p(x) \restricted{B}\), e quindi \(p(x)\) è finitamente consistente.
\end{proof}
\section{{\bfseries\sffamily TODO} Lezione 17 - \textit{<2025-11-18 Tue>}}
\label{sec:org0b352c7}

\def\U{\mathcal{U}}
\def\L{\mathcal{L}}
%
\def\eq{{\rm eq}}
\def\Ueq{\U^\eq}
%
\def\orbita{\mathcal{O}}
\def\Aut{\operatorname{Aut}}
%
\def\tc{\mid}
\def\tp{\operatorname{tp}}
\def\EMtp{\operatorname{EM}\text{-}\operatorname{tp}}
\def\<{\langle}
\def\>{\rangle}
%
\def\restricted#1{\,\mathord{\upharpoonright}{{\scriptstyle #1}}}
\def\equivalentover#1{\mathrel{\equiv_{ #1 }}}

\def\D{\mathcal{D}}
\def\dlo{\text{dlo}}

%% NON FORKING
\def\nonforkSymbol{%
\mathbin{\raise1.8ex%
\rlap{\kern0.6ex\rule{0.6ex}{0.1ex}}%
\rlap{\kern1.1ex\rule{0.1ex}{1.9ex}}\raise-0.3ex\hbox{$\smile$}}}
\def\defaultnonforkmodel{M}
\def\nonfork{\nonforkSymbol}
\renewcommand{\nonfork}[1][\defaultnonforkmodel]{%
\mathrel{\nonforkSymbol_{#1}}}

Sia \(\varphi \subseteq \U^{x}\times \U^{z}\) una \uline{relazione qualsiasi}.
Forse questo può essere generalizzato al di fuori del Modello Mostro

\begin{definizione}
Un insieme \(\D\) è \uline{approssimabile da \(\varphi(x;z)\) dal basso} se vale una delle seguenti equivalenti:
\begin{itemize}
\item per ogni \(B \subseteq \U^{z}\) finito esiste \(a \in \U^{x}\) tale che
\begin{equation*}
\D \cap B = \varphi(a;B) \coloneqq \varphi(a;\U^{z})\cap B
\end{equation*}
e \(\varphi(a;\U^{z}) \subseteq \D\);
\item per ogni \(B \subseteq \D\) finito esiste \(a \in \U^{x}\) tale che
\begin{equation*}
B \subseteq \varphi(a;\U^{z}) \subseteq \D.
\end{equation*}
\end{itemize}
\end{definizione}
\begin{thm}
Se \(\varphi(x;z)\) è stabile e \(\D\) è approssimabile da \(\varphi(x;z)\) allora esiste \(\langle a_{ij} \mid i,j<m\rangle\) in \(\U^{x}\) tale che
\begin{equation*}
\D = \bigcup_{i=1}^{m}\bigcap_{j=1}^{m} \varphi(a_{ij}, \U^{z}).
\end{equation*}
\label{teorema_obiettivo_sjdancvlkjfbhsdlkvsj}
\end{thm}

\begin{lem}
Se \(\varphi(x;z)\) è stabile, \(\D\) approssimato da \(\varphi(x;z)\) dal basso, allora esiste
\(\< a_{i} \mid i<m \>\)
tale che
\begin{equation*}
\D = \bigcup_{i=1}^{m} \varphi(a_{i}, \U^{z}).
\end{equation*}
\end{lem}
\begin{proof}
Costruiamo una catena \(\< a_{i}, b_{i} \mid i<m \>\) come segue:
\begin{enumerate}
\item Sia \(b_{0} \in \D\) e \(a_{0}\) tale che
\(b_{0} \in \varphi(a_{0};\U^{z}) \subseteq \D\).
\item Sia \(b_{1} \in \D \setminus \varphi(a_{0};\U^{z})\) e \(a_{1}\) tale che
\(\set{b_{0},b_{1}} \subseteq \varphi(a_{1};\U^{z})\).
\item Sia \(b_{2} \in \D \setminus \big(\varphi(a_{0};\U^{z})\cup \varphi(a_{1};\U^{z})\big)\) e \(a_{2}\) tale che
\(\set{b_{0},b_{1},b_{2}} \subseteq \varphi(a_{2};\U^{z})\).
\end{enumerate}

Il \(b_{i}\) esiste poiché, se quell'insieme fosse vuoto, allora si avrebbe l'uguaglianza e quindi la tesi; gli \(a_{i}\) esistono per ipotesi di approssimazione dal basso. Per stabilità:
\begin{equation*}
i<j \IMPLICA \lnot\varphi(a_{i}, b_{j}) \land \varphi(a_{j}; b_{i}).
\end{equation*}
Quindi la costruzione si ferma in un passo finito.
\end{proof}
\begin{lem}
Se \(\varphi(x;z)\) è stabile e \(\D\) è approssimata da \(\varphi(x;z)\) allora \(\D\) è approssimata dal basso dalla formula
\begin{equation*}
\psi(x_{1},\dots,x_{m}; z) = \bigwedge_{i=1}^{m} \varphi(x_{i};z).
\end{equation*}
\end{lem}
\begin{proof}
Neghiamo il lemma e sia \(B \subseteq \D\) un controesempio.

Costruiamo una catena \(\< a_{i}, b_{i} \mid i<m \>\) come segue:
\begin{enumerate}
\item \(a_{0}\) e \(b_{0}\) sono tali che
\begin{equation*}
 B \subseteq \varphi(a_{0}; \U^{z}) \subseteq \D,\qquad b_{0} \in \varphi(a_{0};\U^{z})\setminus \D.
\end{equation*}
\item \(a_{1}\) e \(b_{1}\) sono tali che
\begin{equation*}
 B \subseteq \varphi(a_{1}; \U^{z}) \subseteq \D \setminus \set{b_{0}},\qquad %
 b_{1} \in %
 \big[\varphi(a_{0};\U^{z})\cap \varphi(a_{1};\U^{z})\big]\setminus \D.
\end{equation*}
\item \(a_{2}\) e \(b_{2}\) sono tali che
\begin{equation*}
 B \subseteq \varphi(a_{2}; \U^{z})\subseteq \D \setminus \set{b_{0},b_{1}},\qquad %
 b_{2} \in %
 \big[\varphi(a_{0};\U^{z})\cap \varphi(a_{1};\U^{z}) \cap \varphi(a_{2};\U^{z})\big]\setminus \D.
\end{equation*}
\end{enumerate}

Per la stabilità
\begin{equation*}
i<j \IMPLICA \varphi(a_{i}, b_{j}) \land \lnot \varphi(a_{j}, b_{i})
\end{equation*}
e quindi la scaletta deve terminare.
\end{proof}
\begin{thm}
LSASE:
\begin{enumerate}
\item \(\varphi(x;z)\) è stabile;
\item ogni \(\D \subseteq \U^{z}\) esternamente definibile è definibile
\item detta \(\kappa\) la cardinalità di \(\U\), ci sono \(\le\kappa\) insiemi esternamente definibile da \(\varphi(x;z)\), ovvero \(\card{S_{\varphi}(\U)}\le \kappa\).
\item ci sono \(<2^{k}\) insiemi esternamente definibile da \(\varphi(x;z)\).
\end{enumerate}
\end{thm}
\begin{proof}
(\(1.\Rightarrow 2.\)): versione debole del
Teorema~\ref{teorema_obiettivo_sjdancvlkjfbhsdlkvsj}.

(\(2.\Rightarrow 3.\)): ovvia.

(\(3.\Rightarrow 4.\)): ovvia.

(Cenno \(\lnot 1.\Rightarrow \lnot 4.\)):
Per ipotesi esiste una scaletta \(\< a_{i}, b_{i} \mid i \in I \>\) con \((I,<)\) modello saturo di \(T_{\dlo}\) di cardinalità \(\kappa\).
\begin{quote}
\uline{Esercizio}: esistono \(2^{\kappa}\) tagli di Dedekind in \(I\), ovvero insiemi \(C \subseteq I\) chiusi all'ingiù rispetto a \(<\).
\end{quote}

Per ogni taglio \(C\) considero il tipo
\begin{equation*}
p_{C}(x) = \set{\lnot\varphi(x;b_{i}) \mid i \in C} \cup%
\set{\varphi(x;b_{i}) \mid i \in I\setminus C}.
\end{equation*}

\begin{quote}
\uline{Esercizio}:  siccome \(\langle a_{i}, b_{i} \mid i \in I \rangle\) è una scaletta, per ogni \(B\) finito sia \(B \cap C < j < B \setminus C\).
\begin{equation*}
a_{j} \vDash p_{C} (x) \restricted{B}.
\end{equation*}
\end{quote}
\end{proof}
\begin{definizione}
Il \uline{rango binario} (di Shelah) di \(\varphi(x;z)\) ??? RECUPERARE
\end{definizione}
\begin{thm}
LSASE:
\begin{enumerate}
\item \(\varphi(x;z)\) stabile;
\item il rango binario di \(\varphi(x;z)\) è finito.
\end{enumerate}
\end{thm}
\begin{proof}
(Cenno \(\lnot 1.\Rightarrow\lnot 2.\)): Ci sono \(2^{\kappa}\) insiemi esternamente definibili.
\begin{quote}
\uline{Esercizio}: Esiste un \(b_{\emptyset}\) tale che \(2^{\kappa}\) insiemi esternamente definibili contengano \(b_{\emptyset}\) e \(2^{\kappa}\) insiemi esternamente definibili che non contengano \(b_{\emptyset}\).
\end{quote}
Iterando ottengo una sequenza \(\langle b_{s} \mid s \in 2^{<\omega} \rangle\)

(Cenno \(\lnot 2.\Rightarrow\lnot 1.\)): se rango \(\infty\) per compattezza esiste albero di altezza \(\kappa\) e quindi \(\card{S\varphi(\U)} = 2^{\kappa}\).
(Poiché \(2^{<\kappa }= \kappa\)).
\end{proof}
\section{{\bfseries\sffamily TODO} Lezione 18 - \textit{<2025-11-25 Tue>}}
\label{sec:orgb09e948}

\def\U{\mathcal{U}}
\def\L{\mathcal{L}}
%
\def\eq{{\rm eq}}
\def\Ueq{\U^\eq}
%
\def\orbita{\mathcal{O}}
\def\Aut{\operatorname{Aut}}
%
\def\tc{\mid}
\def\tp{\operatorname{tp}}
\def\EMtp{\operatorname{EM}\text{-}\operatorname{tp}}
\def\<{\langle}
\def\>{\rangle}
%
\def\restricted#1{\,\mathord{\upharpoonright}{{\scriptstyle #1}}}
\def\equivalentover#1{\mathrel{\equiv_{ #1 }}}

%% NON FORKING
\def\nonforkSymbol{%
\mathbin{\raise1.8ex%
\rlap{\kern0.6ex\rule{0.6ex}{0.1ex}}%
\rlap{\kern1.1ex\rule{0.1ex}{1.9ex}}\raise-0.3ex\hbox{$\smile$}}}
\def\defaultnonforkmodel{M}
\def\nonfork{\nonforkSymbol}
\renewcommand{\nonfork}[1][\defaultnonforkmodel]{%
\mathrel{\nonforkSymbol_{#1}}}

\def\D{\mathcal{D}}
\def\C{\mathcal{C}}
\def\acl{\operatorname{acl}}
\def\Lascar{\mathscr{L}}
\def\distanza#1{\operatorname{d}_{#1}}
\subsection{Lascar invariance}
\label{sec:orgf12ea67}

Sia \(\D \subseteq \U^{z}\) insieme arbitrario, denotiamo con
\begin{equation*}
f\D = f[\D] = \set{fa \mid a \in \D},\qquad \orbita(\D / A) = \set{f\D \mid f \in \Aut(\U/A)}.
\end{equation*}
\begin{definizione}
\(\D\) è \uline{Lascar invariante su \(A\)} se è invariante su \(M\) per ogni \(M\supseteq A\).
\end{definizione}

\begin{oss}
\begin{enumerate}
\item Se \(A\) è un modello, Lascar invariante su \(A\) significa invariante su \(A\).
\item Se \(\acl(A)\) è un modello\footnote{NOTA: la chiusura algebrica \uline{non è} necessariamente un modello.}, allora ``Lascar invariante su \(A\)'' = ``invariante su \(\acl(A)\)''
\end{enumerate}
\end{oss}

\uline{Domanda 1}: quanti sono gli insiemi Lascar invarianti su \(A\)?
\begin{prop}
Sia \(\lambda=\card{\L_{z}(A)}\), con \(\card{z}<\kappa\). Ci sono al più \(2^{2^{\lambda}}\) insiemi \(\D \subseteq \U^{z}\) Lascar invarianti su \(A\).
\label{quantisonolascarinvariante}
\end{prop}
\begin{proof}
Sia \(N \supseteq A\) modello, \(\card{N} = \lambda\). Allora ogni insieme Lascar invariante su \(A\) è invariante su \(N\).

Su \(N\) ci sono \(2^{2^{\card{N}}}\) insiemi invarianti su \(N\).
\end{proof}
\begin{thm}
Sia \(\D \subseteq \U^{z}\) e \(A \subseteq \U\) piccolo. Sia \(\lambda=\card{\L_{z}(A)}\), con \(\card{z}<\kappa\). LSASE:
\begin{enumerate}
\item \(\D\) è Lascar invariante su \(A\);
\item ogni insieme in \(\orbita(\D / A)\) è Lascar invariante su \(A\);
\item \(\orbita(\D / A)\) ha cardinalità \(\le 2^{2^{\lambda}}\);
\item \(\orbita(\D / A)\) ha cardinalità \(<\kappa\) (si dice che la cardinalità di \(\orbita(\D / A)\) è \uline{bounded}).
\item \(c_{0} \in \D \iff c_{1} \in \D\) per ogni sequenza
\(\langle c_{i} \mid i <\omega \rangle\)
di indiscernibili su \(A\).
\end{enumerate}
\end{thm}
\begin{proof}
(\(1.\Rightarrow 2.\)): Se \(f \in \Aut(\U / A)\) allora \(\D\) è invariante su \(M\supseteq A\) sse \(f\D\) è invariante su \(f[M] \supseteq A\), \(f[M]\) modello.

(\(2.\Rightarrow 3.\)): Ovvio per la Proposizione~\ref{quantisonolascarinvariante}.

(\(3.\Rightarrow 4.\)): Perché \(\kappa\) è inaccessibile.

(\(4.\Rightarrow 5.\)): Sia \(\langle c_{i} \mid i < \kappa \rangle\) di \(A\) indiscernibili (vedere esercizio).

Per assurdo, supponiamo che \(\lnot(c_{0} \in \D \iff c_{1} \in \D)\).
Definiamo una relazione di equivalenza su \(\U^{z}\)
\begin{equation*}
E(u,v) \IFF \forall \C \in \orbita(\D / A)\ (u \in \C \iff v \in \C)
\end{equation*}
Questa non è una relazione definibile. \(E\) è invariante su \(A\)\footnote{Come per le formule??}. Per ipotesi, \(\card{\orbita(\D / A)} = \mu < \kappa\).
Quindi \(E\) ha al più \(2^{\mu}\) classi.

Ma \(\lnot(c_{0} \in \D \iff c_{1} \in \D)\), quindi \(\lnot E(c_{0},c_{1})\), e quindi
\begin{equation*}
\forall i<j<\kappa\ \bigg(\lnot E(c_{i}, c_{j})\bigg).
\end{equation*}
Quindi \(E\) ha \(\kappa\) classi. Assurdo.

(\(5.\Rightarrow 1.\)): Fisso un modello \(M\supseteq A\) arbitrario. Devo mostrare che \(\D\) è invariante su \(M\).

Prendo \(a \equivalentover{M} b\) arbitrari, e mostro che \(a \in \D \iff b \in \D\). Prendo \(p(x) \in S(\U)\) tipo globale coerede di \(\tp(a/M) = \tp(b/M)\) (finitamente soddisfacibili poiché \(M\) è modello).

Definisco \(\langle c_{i} \mid i < \omega \rangle\) tale che
\begin{equation*}
c_{i} \vDash p(x) \restricted{M,\, c\restricted{i},\, a, b}
\end{equation*}
Quindi
\begin{align*}
&a, c_{0}, c_{1}, \dots\\
&b, c_{0}, c_{1}, \dots
\end{align*}
è una sequenza di Morley su \(M\) di \(p(x)\) e pertanto, per ipotesi
\begin{align*}
a \in \D &\iff c_{0} \in \D\\
b \in \D &\iff c_{0} \in \D
\end{align*}
e pertanto \(a \in \D \iff b \in \D\).
\end{proof}
\begin{definizione}
Il \uline{tipo di Lascar di \(a\) su \(A\)} è \(\Lascar(a / A)\),
\begin{equation*}
\Lascar(a / A) \coloneqq %
	\bigcap \set{%
		\D \mid a \in \D, \D \text{ Lascar invariante su \(A\)} %
	}
\end{equation*}
\end{definizione}
\begin{definizione}
Il \uline{grafo di Lascar su \(A\)} in \(\U^{x}\) è costruito come segue: per ogni \(a,b \in \U^{x}\), c'è un arco tra \(a\) e \(b\) se
\begin{equation*}
a \equivalentover{M} b \qquad \text{per qualche \(M\supseteq A\)}.
\end{equation*}
Si definisce \(\distanza{A}(a,b)\) come il minimo \(n\) tale che
\begin{equation*}
a = a_{0} \equivalentover{M_{0}} a_{1} \equivalentover{M_{1}} \dots \equivalentover{M_{n-2}} a_{n-1} \equivalentover{M_{n-1}} a_{n} = b
\end{equation*}
o \(\infty\) altrimenti.
\end{definizione}
\begin{prop}
Si ha che
\begin{equation*}
\Lascar(a / A) = \set{b \mid \distanza{A}(a,b)< \infty}.
\end{equation*}
\end{prop}
\begin{proof}
(\(\supseteq\)): Sia \(a \in \D\) Lascar invariante su \(A\) arbitrario. Devo mostrare che \(\D\) contiene ogni \(b\) tale che \(\distanza{A}(a,b) < \infty\). Sia quindi
\begin{equation*}
a = a_{0} \equivalentover{M_{0}} a_{1} \equivalentover{M_{1}} \dots \equivalentover{M_{n-2}} a_{n-1} \equivalentover{M_{n-1}} a_{n} = b
\end{equation*}
\begin{itemize}
\item Esiste \(f_{0} \in \Aut(\U / M_{0})\) tale che \(f_{0}(a_{0} = a_{1})\)
\item Esiste \(f_{1} \in \Aut(\U / M_{1})\) tale che \(f_{1}(a_{1} = a_{2})\)
\item \(\dots\)
\item Esiste \(f_{n-1} \in \Aut(\U / M_{n-1})\) tale che \(f_{n-1}(a_{n-1} = a_{n})\)
\end{itemize}

Poiché ciascun \(f_{i}\) fissa un modello che fissa \(A\), allora fissa anche \(\D\), e pertanto \(f_{n-1}\circ\dots\circ f_{0}\) fissa \(\D\):
\begin{equation*}
f_{n-1}\circ\dots\circ f_{0} \D = \D \IMPLICA %
b = f_{n-1}\circ\dots\circ f_{0} a \in \D
\end{equation*}

(\(\subseteq\)): basta mostrare che \(\set{b \mid \distanza{A}(a,b)< \infty}\) è Lascar invariante. Sia  \(b\) tale che \(\distanza{A}(a,b) < \infty\):
\begin{equation*}
a = a_{0} \equivalentover{M_{0}} a_{1} \equivalentover{M_{1}} \dots \equivalentover{M_{n-2}} a_{n-1} \equivalentover{M_{n-1}} a_{n} = b.
\end{equation*}
Sia \(f \in \Aut(\U / M)\), con \(M\supseteq A\). Voglio mostrare che \(\distanza{A}(a,fb)<\infty\)\_
\begin{equation*}
a = a_{0} \equivalentover{M_{0}} a_{1} \equivalentover{M_{1}} \dots \equivalentover{M_{n-2}} a_{n-1} \equivalentover{M_{n-1}} a_{n} = b \equivalentover{M} fb. %
\qedhere
\end{equation*}
\end{proof}
\begin{definizione}
Un tipo \(p(x) \in S_{\varphi}(\U)\) è Lascar invariante su \(A\) se \(\mathscr{D}_{p\varphi}\) è Lascar invariante su \(A\).
\end{definizione}
La definizione si intende per tipi \(q(x) = \bigcup_{\varphi \in \L(\U)} p_{\varphi}(x)\), con \(p_{\varphi} \in S_{\varphi}(\U)\).

\begin{oss}
Siano \(\varphi(x;z) \in \L(A)\) e \(p(x) \in S_{\varphi}(\U)\). Se \(p(x)\) è finitamente soddisfacibile in ogni \(M \supseteq A\) allora \(p(x)\) è Lascar invariante su \(A\).
\end{oss}

Diciamo che \(p(x) \in S_{\varphi}(\U)\) è un Lascar coerede su \(A\) se è finitamente soddisfacibile in ogni \(M \supseteq A\).

\begin{thm}
(falso).
Se \(q(x) \subseteq \L(\U)\) è finitamente soddisfacibile in ogni \(M\supseteq A\), allora esiste \(p(x) \in S(\U)\), \(p(x) \supseteq q(x)\) finitamente soddisfacibile in ogni \(M\supseteq A\).
\end{thm}

Affinché il teorema valga, mi serve la seguente proprietà: per ogni \(\theta(x), \rho(x) \in \L(\U)\): \(\theta(x) \lor \rho(x)\) soddisfacibile in ogni \(M\supseteq A\) allora o \(\theta(x)\) è soddisfacibile in ogni \(M\supseteq A\) o \(\rho(x)\) è soddisfacibile in ogni \(M\supseteq A\).

\begin{esempio}
Sia \(\U \succeq (\Q,<)\), e considero la formula \((x<a) \lor (x \ge a)\).
\end{esempio}
\section{{\bfseries\sffamily TODO} Lezione 19 - \textit{<2025-11-26 Wed>}}
\label{sec:org1e3001c}

\def\U{\mathcal{U}}
\def\L{\mathcal{L}}
%
\def\eq{{\rm eq}}
\def\Ueq{\U^\eq}
%
\def\orbita{\mathcal{O}}
\def\Aut{\operatorname{Aut}}
%
\def\tc{\mid}
\def\tp{\operatorname{tp}}
\def\EMtp{\operatorname{EM}\text{-}\operatorname{tp}}
\def\<{\langle}
\def\>{\rangle}
%
\def\restricted#1{\,\mathord{\upharpoonright}{{\scriptstyle #1}}}
\def\equivalentover#1{\mathrel{\equiv_{ #1 }}}

%% NON FORKING
\def\nonforkSymbol{%
\mathbin{\raise1.8ex%
\rlap{\kern0.6ex\rule{0.6ex}{0.1ex}}%
\rlap{\kern1.1ex\rule{0.1ex}{1.9ex}}\raise-0.3ex\hbox{$\smile$}}}
\def\defaultnonforkmodel{M}
\def\nonfork{\nonforkSymbol}
\renewcommand{\nonfork}[1][\defaultnonforkmodel]{%
\mathrel{\nonforkSymbol_{#1}}}

\def\V{\mathcal{V}}

\begin{thm}
Se \(\varphi(x;z), \psi(x;z) \in \L\) sono stabili, \(b \in \U^{z}\) e
\begin{equation*}
\varphi(x;b) \lor\psi(x;b)
\end{equation*}
è soddisfatta in ogni \(M\supseteq A\) allora:
\begin{itemize}
\item o \(\varphi(x;b)\) è soddisfatta in ogni \(M\supseteq A\);
\item o \(\psi(x;b)\) è soddisfatta in ogni \(M\supseteq A\).
\end{itemize}
\label{teorema_obiettivo:kjkjnkjnkjnkjnkjnkjnkjnkjn}
\end{thm}
\begin{oss}
Si dice ``quasi soddisfatta in \(A\)'' il luogo di ``è soddisfatta in ogni \(M\supseteq A\)''. La proprietà di ``quasi soddisfacibilità'' è \uline{partition-regular}.
\end{oss}

Questo teorema si trova in \autocite{harnikFundamentalsForking1984}.

\begin{prop}
Se \(\varphi(x;z) \in \L\) è stabile, \(b \in \U^{z}\), allora
\begin{itemize}
\item o \(\varphi(x;b)\) è soddisfatta in ogni modello \(M\);
\item o \(\lnot \varphi(x;b)\) è soddisfatta in ogni modello \(M\).
\end{itemize}
ovvero \(\varphi(x;b)\) è quasi soddisfatta in \(\emptyset\) oppure \(\lnot\varphi(x;b)\) è quasi soddisfatta in \(\emptyset\).
\label{prop:utilizzataunpounjkjoagitandolemani}
\end{prop}
\begin{proof}
Per disegno
\end{proof}
\begin{lem}
Sia \(\varphi(x) \in \L(\U)\) quasi soddisfacibile in \(A\). Allora esiste \(\theta(x_{1},\dots,x_{n}) \in \L(A)\) consistente, con \(\card{x_{i}} = \card{x}\) tale che
\begin{equation*}
\theta(x_{1},\dots,x_{n}) \implies \bigvee_{i=1}^{n} \varphi(x_{i}).
\end{equation*}
\label{lem:thetaamaroinbocca}
\end{lem}
\def\c{\overline{c}}
\def\x{\overline{x}}
\begin{proof}
Sia \(M\supseteq A\), sia \(\c\) una enumerazione di \(M^{x}\), e sia
\(q(\x) \coloneqq \tp(\c / A)\).

Ogni realizzazione di:
\begin{equation*}
q(\x) \cup \set{ \lnot \varphi(x_{i}) \mid i < \card{\c}}.
\end{equation*}
enumera \(N^{x}\), dove \(N\mathrel{\cong_{A}}M\) e \(N\) non soddisfa \(\varphi(x)\), quindi \(q(\x) \cup \set{ \lnot \varphi(x_{i}) \mid i < \card{\c}}\) è inconsistente.

Pertanto esiste \(\theta(x_{1},\dots,x_{n}) \in q(\x)\) tale che
\begin{equation*}
\theta(x_{1},\dots,x_{n}) \implies \bigvee_{i=1}^{n} \varphi(x_{i}).%
\qedhere
\end{equation*}
\end{proof}
\begin{oss}
Nel dimostrare il
Teorema~\ref{teorema_obiettivo:kjkjnkjnkjnkjnkjnkjnkjnkjn},
si ha come ipotesi che \(\varphi(x;b) \lor \psi(x;b)\) sia quasi soddisfacibile in \(A\). Per il
Lemma~\eqref{lem:thetaamaroinbocca}, questo è equivalente a
\begin{equation*}
\theta(x_{1},\dots,x_{n}) \implies \bigvee_{i=1}^{n} [\varphi(x_{i};b) \lor \psi(x_{i};b)]
\end{equation*}
per qualche \(\theta \in \L(A)\),
ancora equivalente a
\begin{equation*}
\theta(x_{1},\dots,x_{n}) \implies \bigvee_{i=1}^{n} \varphi(x_{i};b) \lor \bigvee_{i=1}^{n} \psi(x_{i};b)
\end{equation*}
per qualche \(\theta \in \L(A)\). ????
\end{oss}
Posso riscrivere il
Teorema~\ref{teorema_obiettivo:kjkjnkjnkjnkjnkjnkjnkjnkjn}
come segue:
\begin{thm}
Se \(\varphi(x;z), \psi(x;z) \in \L\) sono stabili, \(b \in \U^{z}\) e
\begin{equation*}
\theta(x) \implies \varphi(x;b) \lor\psi(x;b)
\end{equation*}
per qualche \(\theta(x) \in \L(A)\). Allora:
\begin{itemize}
\item o \(\varphi(x;b)\) è soddisfatta in ogni \(M\supseteq A\);
\item o \(\psi(x;b)\) è soddisfatta in ogni \(M\supseteq A\).
\end{itemize}
\end{thm}
\begin{proof}
Ripete quella della
Proposizione~\ref{prop:utilizzataunpounjkjoagitandolemani}
ma si scelgono gli \(a_{ij}\) che soddisfano \(\theta(x)\).
\end{proof}
\begin{oss}
La stabilità della \(\psi(x;z)\) non si usa.
\end{oss}

\begin{cor}
Sia \(\varphi(x;z) \in \L(A)\) stabile. Sia \(q(x) \subseteq \L(\U)\) finitamente soddisfacibile in ogni \(M\supseteq A\). Allora esiste \(p(x) \in S_{\varphi}(\U)\) tale che \(q(x) \cup p(x)\) è finitamente soddisfacibile in ogni \(M\supseteq A\).
\end{cor}
\begin{proof}
Sia \(p(x) \subseteq \L_{\varphi^{\pm}}(\U)\) massimale tale che \(q(x) \cup p(x)\) è finitamente soddisfacibile in ogni \(M\supseteq A\).
Voglio mostrare che \(p(x)\) è completo.

Se per assurdo esistesse \(\varphi(x;b)\) tale che \(\varphi(x;b),\lnot\varphi(x;b) \notin p(x)\), quindi esiste \(\alpha(x) \in q(x)\) e \(\theta(x;b_{1},\dots,b_{n}) \in \set{\land}p\) tale che
\begin{align*}
\alpha(x) \land \theta(x;b_{1},\dots,b_{n}) \land \varphi(x;b)\\
\alpha(x) \land \theta(x;b_{1},\dots,b_{n}) \land \lnot\varphi(x;b)\\
\end{align*}
non sono soddisfacibili in ogni \(M \supseteq A\). Ma la congiunzione delle due formule è soddisfacibile in ogni \(M\supseteq A\), in quanto elemento di \(p(x) \cup q(x)\).

Ma \(\alpha(x) \land \theta(x;z_{1},\dots,z_{n}) \land \varphi(x;z)\) è stabile, e pertanto per il teorema si ha la contraddizione.
\end{proof}

\begin{thm}
(\(T\) stabile). Se \(a \nonfork b\) allora \(b \nonfork a\).
\end{thm}

\begin{thm}
Sia \(\varphi(x;z) \in \L(M)\) stabile. Se \(a \nonfork b\) e \(\varphi(a,b)\), allora \(\varphi(a;M^{z}) \neq \emptyset\).
\label{thm:inversione_forking}
\end{thm}
\begin{lem}
Se \(a \nonfork b\), allora esiste \(\V \preceq \U \mathrel{\cong_{M, b}} \V\) tale che \(a \nonfork \V\).
\end{lem}
\def\w{\overline{w}}
\begin{proof}
Sia \(\c\) una enumerazione di \(\U\), e sia \(p(\w, z) = \tp(\c, b / M)\). Considero
\begin{equation*}
p(\w,b) \cup \set{%
	\lnot\varphi(a,\w) \mid %
	\varphi(x;\w) \in \L(M), \varphi(M^{x},\c)=\emptyset %
}
\end{equation*}
Un elemento che realizzi questo tipo enumera \(\V\).

Supponiamo che il tipo sia inconsistente. Allora
\begin{equation*}
\forall \w \big[%
	\theta(\w,b) \implies \bigvee_{i=1}^{n} \varphi_{i}(a, \w)%
\big]
\end{equation*}
ma siccome \(a\nonfork b\) allora esiste \(a' \in M\) tale che
\begin{equation*}
\forall \w \big[%
	\theta(\w,b) \implies \bigvee_{i=1}^{n} \varphi_{i}(a', \w)%
\big]
\end{equation*}
Ponendo \(\w = \c\) ottengo
\begin{equation*}
\theta(\c,b) \implies \bigvee_{i=1}^{n} \varphi_{i}(a', \c)
\end{equation*}
quindi \(\varphi_{i}(M^{x}, \c) \neq \emptyset\). Assurdo.
\end{proof}

\begin{proof}
(del Teorema~\ref{thm:inversione_forking}).
Sia \(\V\preceq \U\), \(\V \mathrel{\cong_{M,b}}\U\) tale che \(a \nonfork \V\).

Allora \(\varphi(a;\V^{z})\) è un insieme esternamente definibile (nella prospettiva di \(\V\)). Ma \(\varphi(x;z)\) è stabile, quindi \(\varphi(a;\V^{z}) = \psi(\V^{z})\) per qualche \(\psi(z) \in \L(\V)\).

Per \emph{non splitting} se \(b_{1},b_{2} \in \V^{z}\) e \(b_{1} \equivalentover{M} b_{2}\), siccome \(a \nonfork \V\) allora \(b_{1}\equivalentover{M,a} b_{2}\).

Quindi \(\varphi(a,b_{1}) \iff \varphi(a,b_{2})\) e pertanto \(\psi(\V^{z}) = \varphi(a, \V^{z})\) invariante per \(\Aut(\V / M)\), quindi esiste \(\theta(z) \in \L(M)\) tale che \(\psi(\V^{z}) = \theta(\V^{z})\).

Ma \(\theta(M^{z}) \neq \emptyset\), quindi anche \(\varphi(a,M^{x}) \neq \emptyset\).
\end{proof}
\section{{\bfseries\sffamily TODO} Lezione 20 - \textit{<2025-12-02 Tue>}}
\label{sec:orga1b96b8}

\def\U{\mathcal{U}}
\def\L{\mathcal{L}}
%
\def\acl{\operatorname{acl}}
\def\dcl{\operatorname{dcl}}
\def\eq{{\rm eq}}
\def\Ueq{\U^\eq}
%
\def\orbita{\mathcal{O}}
\def\Aut{\operatorname{Aut}}
%
\def\tc{\mid}
\def\tp{\operatorname{tp}}
\def\EMtp{\operatorname{EM}\text{-}\operatorname{tp}}
\def\<{\langle}
\def\>{\rangle}
%
\def\restricted#1{\,\mathord{\upharpoonright}{{\scriptstyle #1}}}
\def\equivalentover#1{\mathrel{\equiv_{ #1 }}}

%% NON FORKING
\def\nonforkSymbol{%
\mathbin{\raise1.8ex%
\rlap{\kern0.6ex\rule{0.6ex}{0.1ex}}%
\rlap{\kern1.1ex\rule{0.1ex}{1.9ex}}\raise-0.3ex\hbox{$\smile$}}}
\def\defaultnonforkmodel{M}
\def\nonfork{\nonforkSymbol}
\renewcommand{\nonfork}[1][\defaultnonforkmodel]{%
\mathrel{\nonforkSymbol_{#1}}}

\def\V{\mathcal{V}}
\def\D{\mathscr{D}}

Recap teorema dell'ultima volta:
\begin{thm}
(\(T\) stabile).
Per ogni \(M\vDash T\):
\begin{equation*}
a \nonfork b \IMPLICA b \nonfork a.
\end{equation*}
\end{thm}
\begin{thm}
Sia \(\varphi(x;z) \in \L\) stabile. Se \(a \nonfork b\) e \(\varphi(a,b)\), allora
\begin{equation*}
\varphi(a, M^{z}) \neq \emptyset.
\end{equation*}
\end{thm}

\begin{thm}
Sia \(\varphi(x;z) \in \L(\U)\) stabile, \(p(x) \in S_{\varphi}(\U)\), \(A \subseteq \U\). LSASE:
\begin{enumerate}
\item \(\D_{p,\varphi} \in \acl^{\eq} A\)\footnote{Ricordiamo che
\begin{equation*}
\D_{p,\varphi} = \set{b \in \U^{z} \mid \varphi(x,b) \in p}
\end{equation*}
è insieme definibile.};
\item \(p(x)\) è finitamente soddisfacibile in ogni \(M \supseteq A\);
\item \(p(x)\) è Lascar invariante su \(A\)\footnote{Ricordiamo che \(p(x)\) è Lascar invariante su \(A\) se \(\D_{p,\varphi}\) è invariante su ogni modello \(M \supseteq A\).};
\item \(p(x)\) è invariante su \(\acl^{\eq} A\).
\end{enumerate}
\end{thm}
\begin{proof}
(\(1. \Rightarrow 2.\)):
Fissiamo \(M \supseteq A\) e fissiamo \(\varphi(x;b) \in p(x)\), e mostriamo che \(\varphi(M^{x}, b) \neq \emptyset\).

\emph{\uline{Nota}:
in realtà dovremmo mostrare che ogni
\(\theta(x; b_{1},\dots,b_{n}) \in \set{\land}\!p\) è finitamente soddisfacibile.
Vedi sotto}

Sia \(a \vDash p(x) \restricted{M}\), allora
\begin{equation*}
\varphi(a; M^{z}) = \D_{p,\varphi} \cap M^{z}.
\end{equation*}
Possiamo assumere che \(b \nonfork a\) (se così non fosse potremmo trovare \(a'\vDash p(x) \restricted{M}\) che soddisfa questa condizione).

Quindi \(\varphi(a,b) \iff b \in \D_{p,\theta}\): se non lo fosse, esisterebbe \(b' \in M^{x}\) tale che
\(\varphi(a,b') \not\iff b' \in \D_{p,\theta}\)
contraddicendo \(\varphi(a; M^{z}) = \D_{p,\varphi} \cap M^{z}\).

Siccome \(b \in \D_{p,\varphi}\), ottengo \(\varphi(a,b)\), e per simmetria ottengo \(\varphi(M^{x}, b) \neq \emptyset\).
\end{proof}

Se invece di \(\varphi(x;b)\) prendiamo \(\theta(x;b_{1},\dots,b_{n}) \in \set{\land} p\)
osserviamo che \(\theta(x;z_{1},\dots,z_{n})\) è stabile e che \(\D_{p,\theta}\) è finita.
\subsection{Stazionarietà}
\label{sec:orgd3620d7}

\begin{definizione}
Sia \(q(x) \subseteq \L(\U)\). \(q(x)\) è \uline{stazionario} (su \(A\)) se esiste un \emph{unico} tipo \(p(x) \in S(\U)\) tale che:
\begin{itemize}
\item \(p(x) \supseteq q(x)\);
\item \(p(x)\) è Lascar invariante su \(A\).
\end{itemize}
\end{definizione}

\begin{thm}
(\(T\) stabile).
Ogni \(q(x) \in S(\acl^{\eq} A)\)\footnote{\(q(x)\) è un tipo completo a parametri in \(S(\acl^{\eq} A)\).\label{org17ac3d7}} è stazionario su \(A\).
\label{thm_stazionario_acleq}
\end{thm}
\begin{lem}
(Lemma misterioso di Harrington).
Sia \(\varphi(x;z) \in \L(\U)\) stabile,
\(p(x) \in S_{\varphi}(\U)\),
\(q(z) \in S_{\varphi^{\text{op}}}(\U)\)\footnote{Con \(\varphi^{\text{op}}\) intendiamo
\begin{equation*}
\varphi^{\text{op}}(z;x) = \varphi(x;z)
\end{equation*}} invarianti su \(M\).

Per ogni
\(a \vDash p(x) \restricted{M}\),
\(b \vDash q(z) \restricted{M}\)
abbiamo
\begin{equation*}
a \in D_{q, \varphi^{\text{op}}} \iff b \in \D_{p,\varphi}.
\end{equation*}
\end{lem}
\begin{proof}
Possiamo assumere che \(a \nonfork b\).
Affermo che
\begin{equation*}
\varphi(a;b) \iff  a \in \D_{q,\varphi^{\text{op}}}
\end{equation*}
altrimenti \(\varphi(a;b) \not\iff  a \in \D_{q,\varphi^{\text{op}}}\). Questa è una formula \(\psi(a,b)\) a parametri in \(M\).

Per \(a \nonfork b\) ottengo che esiste \(a' \in M^{x}\) tale che
\begin{equation*}
\varphi(a';b) \not\iff %
a' \in \D_{q,\varphi^{\text{op}}}
\end{equation*}
e pertanto \(\varphi(M^{x}, b) \neq \D_{q,\varphi^{\text{op}}} \cap M^{x}\).
Quindi \(b \not\vDash q(z) \restricted{M}\).

Per simmetria \(b \nonfork a\), quindi
\begin{equation*}
\varphi(a;b) \iff  b \in \D_{p,\varphi}.
\end{equation*}
Segue la tesi.
\end{proof}

\uline{Esercizio}:
(\(T\) stabile).
Ogni \(q(x) \in S(M)\)\textsuperscript{\ref{org17ac3d7}} è stazionario su \(M\).

\begin{proof}
(del Teorema~\ref{thm_stazionario_acleq}).
L'esistenza di \(p(x) \supseteq q(x)\) globale e Lascar invariante su \(A\) segue dall'esistenza di un coerede di Lascar globale.

Supponiamo che \(p_{1}(x), p_{2}(x) \in S(\U)\) estendono \(q(x)\). Fissiamo \(\varphi(x;z)\) e mostriamo che
\begin{equation*}
\D_{p_{1},\varphi} = \D_{p_{2},\varphi}.
\end{equation*}
Fissiamo \(b \in \U^{z}\) e mostriamo che
\begin{equation*}
b \in \D_{p_{1},\varphi} \iff b \in \D_{p_{2},\varphi}.
\end{equation*}
Sia \(t(z) \in S(\U)\) Lascar-coerede di \(\tp(b / \acl^{\eq} A)\). Fisso \(M \supseteq A\) arbitrario.

Siano \(a_{i} \vDash p_{i}(x) \restricted{M}\). Allora per il lemma:
\begin{equation*}
a_{i} \in \D_{t,\varphi^{\text{op}}} \iff b \in \D_{p_{i}, \varphi}.
\end{equation*}
Ma
\begin{equation*}
a_{1} \in \D_{t,\varphi^{\text{op}}} \iff a_{2} \in \D_{t,\varphi^{\text{op}}}
\end{equation*}
in quando \(\D_{t,\varphi^{\text{op}}} \in \acl^{\eq} A \subseteq M\).
\end{proof}
(idea un po' incasinata).
\section{{\bfseries\sffamily TODO} Lezione 21 - \textit{<2025-12-03 Wed>}}
\label{sec:org12d2d63}

\def\U{\mathcal{U}}
\def\L{\mathcal{L}}
%
\def\acl{\operatorname{acl}}
\def\eq{{\rm eq}}
\def\Ueq{\U^\eq}
%
\def\orbita{\mathcal{O}}
\def\Aut{\operatorname{Aut}}
%
\def\extD{\mathscr{D}}
%
\def\tc{\mid}
\def\tp{\operatorname{tp}}
\def\EMtp{\operatorname{EM}\text{-}\operatorname{tp}}
\def\<{\langle}
\def\>{\rangle}
%
\def\restricted#1{\,\mathord{\upharpoonright}{{\scriptstyle #1}}}
\def\equivalentover#1{\mathrel{\equiv_{ #1 }}}

%% NON FORKING
\def\nonforkSymbol{%
\mathbin{\raise1.8ex%
\rlap{\kern0.6ex\rule{0.6ex}{0.1ex}}%
\rlap{\kern1.1ex\rule{0.1ex}{1.9ex}}\raise-0.3ex\hbox{$\smile$}}}
\def\defaultnonforkmodel{M}
\def\nonfork{\nonforkSymbol}
\renewcommand{\nonfork}[1][\defaultnonforkmodel]{%
\mathrel{\nonforkSymbol_{#1}}}

\uline{ESERCIZIO BONUS}: correggere il capitolo 18.

Sia \(p(x) \in S_{\varphi}(\U)\) invariante su \(A\) se
\begin{equation*}
\extD_{p,\varphi} = \set{b \in \U^{z} \mid \varphi(x;b) \in p}
\end{equation*}
invariante su \(A\).

Sia \(X\) un insieme, \(G\) un gruppo che agisce su \(X\). Sia \(D \subseteq X\).

\begin{definizione}
Si dice che \(D\) è \uline{\(G\)-syndetico} se un numero finito di \(G\)-traslazioni di \(D\) ricopre \(X\), ovvero esiste \(C \subseteq G\) finito tale che
\begin{equation*}
X = \bigcup C \cdot D \coloneqq \bigcup \set{g D \mid g \in C}.
\end{equation*}
\end{definizione}
\begin{definizione}
Si dice che \(D\) è \uline{\(G\)-thick} se per ogni sottoinsieme \(C \subseteq G\) finito
\begin{equation*}
\bigcap C\cdot D \neq \emptyset.
\end{equation*}
\end{definizione}

\begin{prop}
LSASE:
\begin{enumerate}
\item \(D\) non è \(G\)-syndetico;
\item \(X \setminus D\)\footnote{D'ora in avanti si scriverà \(\lnot D \coloneqq X \setminus D\)} è \(G\)-thick.
\end{enumerate}
\label{prop:charsyndthi}
\end{prop}
\begin{proof}
Per ogni \(C \subseteq G\) finito si ha
\begin{equation*}
X \neq \bigcup C\cdot D
\end{equation*}
sse \(\emptyset \neq \bigcap C \cdot (X\setminus D)\).
\end{proof}
\begin{cor}
LSASE
\begin{enumerate}
\item \(D\) è \(G\)-thick;
\item per ogni \(C \subseteq X\) \(G\)-syndetici: \(D\cap C \neq \emptyset\).
\end{enumerate}
\end{cor}
\begin{proof}
(\(1.\Rightarrow 2.\)): Se \(D\cap C = \emptyset\) allora \(D \subseteq \lnot C\) quindi \(\lnot C\) è \(G\)-thick. Per la
Proposizione~\ref{prop:charsyndthi}
segue che \(C\) non è \(G\)-syndetic.

(\(\lnot 1.\Rightarrow \lnot 2.\)): \(\lnot D\) è \(G\)-syndetico, ma \(D\cap \lnot D = \emptyset\).
\end{proof}

D'ora in avanti, considereremo:
\begin{itemize}
\item \(X \subseteq \U^{x}\) insieme tipo-definibile su \(A\);
\item \(Z \subseteq \U^{z}\) insieme tipo-definibile su \(A\).
\item \(G \le \Aut(\U/A)\).
\end{itemize}

\textbf{Vedi ``Azione del gruppo degli automorfismi''}

D'ora in avanti, \(D=\varphi(X;b) \coloneqq \varphi(\U^{x}; b)\cap X\) per \(b \in Z\), \(\varphi(x;z) \in \L(A)\).

Quindi se \(g \in G\) allora \(gD = \varphi(X, gb)\).

Si scriverà
\begin{itemize}
\item \(L(Z)\) per indicare uno dei seguenti insiemi:
\begin{align*}
  &\set{\varphi(x;b) \mid b \in Z}\\
  &\set{\varphi(X;b) \mid b \in Z}
\end{align*}
\item \(\displaystyle \Sigma_{G}(x) \coloneqq \set{\theta \in L(Z) \mid \theta(x) \text{ è }G\text{ syndetico}}\).
\end{itemize}

\uline{Notazione}: in questa sezione, scriveremo \(S(Z)\) per indicare l'insieme dei tipi massimali finitamente soddisfacibili in \(X\) con parametri in \(Z\).

\begin{thm}
Sia \(p(x) \in S(Z)\). LSASE:
\begin{enumerate}
\item \(p(x)\) è \(G\)-invariante
\item \(p(x) \vdash \Sigma_{G}(x)\)
\item \(p(x)\) è \(G\)-thick (ogni formula in \(\set{\land}p\) è \(G\)-thick).
\end{enumerate}
\end{thm}
\begin{proof}
(\(1.\Rightarrow 2.\)): Se serve vedi sulla blackboard.

(\(2.\Rightarrow 3.\)): Se \(D\) è \(G\)-syndetico, se \(p(x) \vdash x \in \lnot D\) dove \(\lnot D\) non è \(G\)-thick, quindi \(\lnot 2.\)
\end{proof}

\begin{cor}
Assumiamo che esista \(p(x) \in S(Z)\) che sia \(G\)-thick. Allora
\begin{enumerate}
\item per ogni \(D \in L(Z)\): \(D\) e \(\lnot D\) non sono entrambi \(G\)-syndetici;
\item \(\Sigma_{G}(x)\) è finitamente consistente;
\item ogni \(G\)-syndetico è \(G\)-thick.
\end{enumerate}
\end{cor}
\begin{proof}
(solo di 3.).
Se \(D\) è \(G\)-syndetico, se non fosse \(G\)-thick allora \(\lnot D\) è \(G\)-syndetico. Questo contraddice 1.
\end{proof}

La seguente è una notazione \emph{non standard}.
\begin{definizione}
Un insieme \(D \in L(Z)\) è \uline{\(G\)-wide} se per ogni partizione finita di \(D\) in insiemi in \(L(Z)\), contiene un insieme \(G\)-thick.
\end{definizione}

\begin{thm}
Sia \(D \in L(Z)\). LSASE:
\begin{enumerate}
\item \(\Sigma_{G}(x) \cup \set{x \in D}\) è finitamente consistente in \(X\);
\item esiste \(p(x) \in S(Z)\) \(G\)-thick e \(p(x) \vdash x \in D\);
\item \(D\) è \(G\)-wide.
\end{enumerate}
\end{thm}
\begin{proof}
(\(1.\Rightarrow 2.\)): Per 1. esiste \(p(x) \in S(Z)\) tale che \(p(x) \vdash \Sigma_{G}(x) \cup \set{x \in D}\). Per il teorema, \(p(x)\) è \(G\)-thick.

(\(2.\Rightarrow 3.\)): Se \(C_{1},\dots,C_{n}\) sono un ricoprimento di \(D\) e \(p(x) \vdash x \in D\) è \(G\)-thick, allora \(p(x) \vdash x \in C_{i}\) per qualche \(i\). Quindi \(C_{i}\) è \(G\)-thick.

(\(2.\Rightarrow 1.\)): per il Teorema.

(\(3.\Rightarrow 2.\)): Sia \(p(x) \supseteq \set{x \in D}\) massimale tra i \(L(Z)\)-tipi che sono \(G\)-wide\footnote{Ovvero, come per \(G\)-thick, ogni formula in \(\set{\land} p\) è \(G\)-wide.}.

Basta mostrare che \(p(x)\) è completo.

Altrimenti sia \(\theta(x)\) tale che \(\theta(x), \lnot\theta(x) \notin p(x)\). Quindi esiste \(\psi_{i}(x) \in \set{\land}p\) tale che
\begin{equation*}
\psi_{1}(x) \land \theta(x), \qquad \psi_{2}(x)\land \lnot\theta(x)
\end{equation*}
non sono \(G\)-wide.

Sia quindi \(C_{1},\dots,C_{n}\) un ricoprimento di \(X\) che testimonia non \(G\)-wideness di entrambe. Ma \(C_{1},\dots,C_{n}\) ricopre
\begin{equation*}
\psi_{1}(x) \land \psi_{2}(x) \in p.%
\qedhere
\end{equation*}
\end{proof}
\section{Lezione 22 - \textit{<2025-12-09 Tue>}}
\label{sec:org9f5f133}

\(D\) è \(G\)-thick se \(\forall C \subseteq G\) finito
\begin{equation*}
\bigcap C \cdot D \neq \emptyset
\end{equation*}

Si ha che \(p \in S_{\varphi}(\U)\) è \(G\)-invariante se e solo se
\begin{equation*}
\forall \theta(x;b) \in \set{\land}p \qquad \theta(\U^x; b) \text{ è } G\text{-thick.}
\end{equation*}

\begin{definizione}
\(q(x) \subseteq \L(\U)\) è \(G\)-thick se
\begin{equation*}
\forall \theta(x;b) \in \set{\land}q \qquad \theta(\U^x; b) \text{ è } G\text{-thick.}
\end{equation*}
\end{definizione}

\begin{oss}
\(p(x) \in S_x(\U)\) è \(G\)-invariante sse è \(G\)-thick.
\end{oss}

\uline{Domanda:} se \(q(x) \subseteq \L(\U)\) è \(G\)-thick, esiste \(p(x) \in S(\U)\) tale che \(p(x) \supseteq q(x)\) è \(G\)-invariante? \textbf{NO}.

\uline{Nota:} D'ora in avanti gli insiemi sono definibili

\begin{definizione}
\(D\) è \(G\)-wide se per ogni \(C_1, \dots, C_n\) ricoprimento di \(D\) un \(C_i\) è \(G\)-thick.
\end{definizione}

\uline{Fatto}: \(q(x) \subseteq \L(\U)\) è \(G\)-wide sse
esiste \(q(x) \subseteq p(x) \in S(\U)\) : \(G\)-thick
\begin{quote}
\textbf{DIGRESSIONE SUL FORKING}
\end{quote}

\begin{definizione}
\(D\) è \uline{non-dividing} se per ogni \(B \subseteq G\) infinito,
e per ogni \(n<\omega\) esiste \(C \subseteq B\) tale che
\begin{equation*}
\card{C} > n, \qquad %
\bigcap C \cdot D \neq \emptyset.
\end{equation*}
\end{definizione}

\begin{definizione}
\(D\) è non-forking se per ogni \(C_1, \dots, C_n\) ricoprimento di \(D\), un \(C_i\) è non-dividing.
\end{definizione}

\begin{oss}
Se \(T\) è NIP (classe di teorie che include le teorie stabili)
\begin{equation*}
D \text{ non-divide } \IFF D \text{ è } H\text{-thick}
\end{equation*}
dove \(H = \Aut^f(\U/A)\).
\end{oss}

\emph{Nota}: \(\Aut^f(\U/A) \le \Aut(\U/A)\) sottogruppo generato dagli automorfismi che fissano un qualche \(M \supseteq A\).

\textbf{Nota:} se \(T\) è stabile allora
\begin{equation*}
\Aut^f(\U/A) = \Aut(\U/acl A).
\end{equation*}

\begin{quote}
\textbf{FINE DIGRESSIONE}
\end{quote}
Siano ora:
\begin{equation*}
\X \subseteq \U^x, \quad \X \subseteq \U^z, \quad \DD \subseteq \X
\end{equation*}

\begin{definizione}
\(\DD\) è \(G\)-syndetic se \(\exists C \subseteq G\) finito tale che
\begin{equation*}
\bigcup C \cdot \DD \supseteq \X
\end{equation*}
\end{definizione}

\begin{oss}
\(\DD\) è \(G\)-thick se \(\forall C \subseteq G\) finito
\begin{equation*}
\bigcap C \cdot \DD \neq \emptyset.
\end{equation*}
\end{oss}

\uline{Fatto}: \(\DD\) non è \(G\)-syndetic sset \(\neg \DD\) è \(G\)-thick.

Si ricorda che:
\begin{equation*}
\Sigma_G(x) = \{ x \in \DD \mid \DD \text{ è } G\text{-syndetic} \}
\end{equation*}

\textbf{Nota:} \(x \in \DD\) è una formula in quanto \(\DD\) è definibile

\uline{Fatto} \(\Sigma_G(x)\) finitamente consistente sse
esiste \(p(x) \in S_{\X}(\Z)\)\footnote{\(S_{\X}(\Z)\) è l'insieme dei tipi finitamente consistenti in \(\X\) a parametri in \(\Z\).} \(G\)-invariante.

\uline{Fatto}: \(\Sigma_{G}(x)\) è finitamente consistente con \(x \in \DD\) sse esiste \(p(x) \in S_{\X}(\Z)\) \(G\)-invariante/\(G\) thick, e \(p(x) \vdash x \in \DD\) sse \(\DD\) è \(G\)-wide.

\begin{definizione}
\(\DD\) è \uline{fortemente \(G\)-syndetic} se \(\forall  C \subseteq \DD\) finito
\(\bigcap C\cdot \DD\) è \(G\)-syndetic.
\end{definizione}
(``Ufficialmente'' questo si chiama insieme \emph{thickly syndetic}).

\begin{definizione}
\(\DD\) è \uline{debolmente \(G\)-thick} se \(\exists   C \subseteq \DD\) finito
\(\bigcup C\cdot \DD\) è \(G\)-thick.
\end{definizione}
(``Ufficialmente'' questo si chiama insieme \emph{weakly syndetic}).

\begin{thm}
Se \(\DD,\CC\) sono fortemente \(G\)-syndetici, allora \(\DD\cap \CC\) è fortemente \(G\)-syndetic.
\end{thm}
\begin{proof}
Sia \(C \subseteq G\) finito. Mostriamo che \(\bigcap C (\DD\cap\CC)\) è \(G\)-syndetico.
\begin{equation*}
\bigcap C (\DD\cap\CC) = %
	\parentesi{\DD'}{\bigg(\bigcap C \DD\bigg)} %
		\cap %
	\parentesi{\CC'}{\bigg(\bigcap C \CC\bigg)} = \DD' \cap \C'
\end{equation*}
con \(\DD'\) e \(\CC'\) che sono \(G\)-syndetici.

Fisso \(F \subseteq G\) finito tale che \(\bigcup F \CC' = \X\), e mostro che
\(\bigcup F (\DD' \cap \CC')\) è \(G\)-syndetic.
\begin{equation*}
\bigcup F (\DD' \cap \CC') \supseteq %
	\bigg(\bigcap F \DD'\bigg) \cap \bigg(\bigcup F \CC'\bigg) = \bigg(\bigcap F \DD'\bigg)
\end{equation*}
Infatti si ha che
\begin{align*}
\bigcup F (\DD' \cap \CC') &= %
	\bigcup_{f \in F} (f\DD' \cap f\CC') \supseteq %
	\bigcup_{f \in F} %
		\left(\bigg(\bigcap_{f' \in F} f' \DD'\bigg) \cap f\CC'\right)\\
	&= \bigg(\bigcap_{f' \in F} f' \DD'\bigg) \cap %
		\bigg(\bigcup_{f \in F} f \CC'\bigg) = %
		\bigg(\bigcap F \DD'\bigg) \cap \bigg(\bigcup F \CC'\bigg).
\end{align*}
Pertanto, siccome \(\bigcup F \CC' = \X\), si ha che
\begin{equation*}
\bigcup F (\DD' \cap \CC') \supseteq %
	\bigg(\bigcap F \DD'\bigg),
\end{equation*}
con \(\bigcap F \DD'\) \(G\)-syndetic.
\end{proof}

Questo Teorema implica che
\begin{equation*}
\strongSigma_{G}(x) = \set{x \in \DD \mid \DD\text{ è fortemente }G\text{-syndetic}}
\end{equation*}
è chiusa per congiunzione e quindi finitamente consistente.

\uline{Fatto} \(\strongSigma_{G}(x)\) finitamente consistente con \(x \in \DD\) sse \(x \in \DD\) è debolmente \(G\)-thick.

\begin{proof}
(\(\Rightarrow\)): se \(\strongSigma_{G}(x)\cup\set{x \in \DD}\) finitamente consistente allora \(x\notin \DD\) non fortmente \(G\)-syndetic sse \(x \in \DD\) è debolmente \(G\)-thick.

(\(\Leftarrow\)): Se \(\strongSigma_{G}(x) \vdash x \notin \DD\) allora esiste \(\CC\) fortemente \(G\)-syntetic tale che \(\CC \subseteq \lnot \DD\), quindi \(\lnot \DD\) è fortemente \(G\)-syndetic, quindi \(\D\) non è debolmente \(G\)-thick.
\end{proof}

\begin{thm}
Se \(\CC \cup \DD\) è debolmente \(G\)-thick allora almeno uno tra \(\CC\) e \(\DD\) è debolmente \(G\)-thick.
\label{thm:debthickpartitionregular}
\end{thm}

\begin{thm}
Per ogni definibile, LSASE:
\begin{enumerate}
\item \(G\)-thick sse \(G\)-wide;
\item \(G\)-syndetic sse fortemente \(G\)-syndetic;
\item \(G\)-thick sse debolmente \(G\)-thick.
\end{enumerate}
\end{thm}
Quando questo vale, ``ricorda'' forking=dividing. Mostreremo che questo succede per \(T\) stabile e \(G = \Aut(\U/\acl^{\eq} A)\).
\begin{proof}
(\(3.\Rightarrow 1.\)): debolmente \(G\)-thick è ``partition regular''
(Teorema~\ref{thm:debthickpartitionregular}) quindi anche \(G\)-thick. Segue 1.

(\(1.\Rightarrow 2.\)): Siano \(\DD, \CC\) \(G\)-syndetici. Mostriamo che \(\DD \cap \CC\) è \(G\)-syndetico.

Per assurdo, supponiamo sia falso. Allora \(\lnot \DD \cup \lnot \CC\) è \(G\)-thick, per 1. allora \(\lnot \DD\) è \(G\)-thick (oppure \(\lnot \CC\)).

Allora \(\DD\) è non \(G\)-syndetic. Assurdo.
\end{proof}

\begin{definizione}
Un insieme \(\X\) è \uline{stazionario} se esiste un unico \(p(x) \in S_{\X}(\U)\) che è \(G\)-invariante.
\end{definizione}

\uline{Fatto}: se \(\X\) è stazionario e \(G\)-thick=\(G\)-wide, allora
\begin{quote}
\(G\)-thick = \(G\)-syndetic.
\end{quote}
\begin{proof}
(\(\supseteq\)): siccome \(\Sigma_{G}(x)\) è finitamente consistente allora \(G\)-syndetico implica \(G\)-thick (non serve \(\X\) stazionario).

(\(\subseteq\)): per assurdo sia \(\DD\) un \(G\)-thick e non \(G\)-syndetic.
Quindi \(\lnot\DD\) è \(G\)-thick.

Poiché \(G\)-thick=\(G\)-wide, allora esiste \(p_{1}(x) \vdash x \in \DD\) e \(p_{2}(x) \vdash x \notin \DD\), con \(p_{1},p_{2} \in S_{\X}(\U)\). Questo contraddice \(\X\) stazionario.
\end{proof}
\section{Lezione 23 (Ultima) - \textit{<2025-12-16 Tue>}}
\label{sec:orgf5e77ca}

Sia \(\Z = \U^{z}\). Fissiamo
\begin{equation*}
G=\Aut(\U/A),\qquad H = \Aut(\U/\acl^{\eq} A), \qquad H \trianglelefteq G
\end{equation*}
Inoltre, per \(q(x) \in S(A)\) o \(q(x) \in S(\acl^{\eq} A)\), scegliamo
\begin{equation*}
\X = q(\U^{x}).
\end{equation*}

Tutti gli insiemi usati per syndetic, thick, wide, etc\ldots{} sono (solo) quelli definiti da una combinazione booleana di \(\varphi(x;b)\) con \(b \in \Z\) e \(\varphi(x;z) \in \L(A)\) fissata formula stabile. Questo insieme di formule è
\begin{equation*}
\Delta(\Z).
\end{equation*}

\begin{thm}
(strano).
Sia \(\varphi(x;z) \in \L(A)\) stabile. Dato \(M \supseteq A\), \(b \in \Z\), esistono \(b_{i} \equivalentover{M} b\) per \(i=1,\dots,n\) tali che una combinazione Booleana \uline{positiva}\footnote{Ovvero senza negazioni, solo \(\land\) e \(\lor\).} di \(\varphi(x;b_{i})\), \(\theta(x;b_{1},\dots,b_{n})\) è equivalente a una formula \(\psi(x)\) in \(\L(M)\), ovvero
\begin{equation*}
\psi(\U^{x}) = \theta(\U^{z},\overline{b}).
\end{equation*}
\label{thm_strano:2}
\end{thm}

Il Teorema ricorda questo: Sia \(\varphi(x;z) \in \L(A)\) stabile, \(M\supseteq A\), \(b \in \U^{z}\). Esistono \(c_{1},\dots,c_{n} \in M^{z}\) tali che
\begin{equation*}
\varphi(M^{x}, b) = \theta(M^{x}; c_{1},\dots,c_{n})
\end{equation*}
dove \(\theta(x;c_{1},\dots,c_{n})\) è combinazione booleana positiva di \(\varphi(x;c_{i})\).

Dal punto di vista di \(M\), \(\varphi(M^{x},b)\) è un insieme esternamente definibile.

Possiamo anche richiedere che \(c_{i} \equivalentover{A} b\), poiché se si definisce \(p(x) = \tp(b/A)\) allora \(p(x) \land \varphi(x;z)\) è relazione stabile.

\begin{proof}
(del Teorema~\ref{thm_strano:2}).
Sia \(\V \preceq \U\) isomorfo ad \(\U\) su \(M\). Allora esistono \(b_{i} \equivalentover{M} b\) per \(i=1,\dots,n\), \(b_{i} \in \V\) ed esiste \(\theta(x;\overline{b})\) combinazione booleana positiva delle \(\varphi(x;b_{i})\) tale che
\begin{equation*}
\varphi(\V^{x};b) = \theta(\V^{x}; \overline{b}).
\end{equation*}

WLOG posso trovare \(\V\) tale che \(b \nonfork \V\). Per \emph{non splitting}, se \(a_{1} \equivalentover{M} a_{2} \in \V\), allora \(a_{1} \equivalentover{M,b} a_{2}\).

Quindi \(\theta(\V^{x};\overline{b})\) è invariante per \(\Aut(\V/M)\), quindi esiste \(\psi(x) \in \L(M)\) tale che
\begin{equation*}
\theta(\V^{x}; \overline{b}) = \psi(\V^{x})
\end{equation*}
ovvero, in particolare
\begin{equation*}
\V \vDash \forall x\ [\theta(x;\overline{b}) \iff \psi(x)]
\end{equation*}
e, per elementarietà,
\begin{equation*}
\U \vDash \forall x\ [\theta(x;\overline{b}) \iff \psi(x)]
\end{equation*}
\end{proof}

\begin{thm}
(strano 2).
Sia \(\varphi(x;z) \in \L(A)\) stabile, \(b \in \Z\), esistono \(b_{i} \equivalentover{\acl^{\eq} A} b\) per \(i=1,\dots,n\) tali che una combinazione Booleana positiva di \(\varphi(x;b_{i})\), \(\theta(x;b_{1},\dots,b_{n})\) è equivalente a una formula \(\psi(x)\) in \(\L(A)\), ovvero
\begin{equation*}
\psi(\U^{x}) = \theta(\U^{z},\overline{b}).
\end{equation*}
\end{thm}

\begin{thm}
Sia \(\varphi(x;z) \in \L(A)\) stabile. Sia \(\X=q(\U^{x})\), \(q(x) \subseteq \L(\U)\) piccolo e finitamente soddisfacibile in ogni \(M\supseteq A\)\footnote{Basta \(q(x) \in S(A)\).}.
Si consideri il gruppo \(H=\Aut(\U / \acl^{\eq}A)\).

Per ogni \(\Delta(\Z)\)-definibile \(D\), LSASE:
\begin{enumerate}
\item \(D\) è \(H\)-thick;
\item \(D\) è \(H\)-wide.
\end{enumerate}
\end{thm}
\begin{proof}
(\(2.\Rightarrow 1.\)): ovvia.

(\(1.\Rightarrow 2.\)): Assumiamo WLOG che \(D=\varphi(\U^{x};b)\). Basta mostrare che \(q(x) \cup \set{x \in D}\) è finitamente soddisfacibile in ogni \(M\supseteq A\). In questo modo possiamo estenderla a \(p(x) \in S_{\varphi}(\U^{z})\) finitamente soddisfacibile in ogni \(M\supseteq A\). Quindi \(D\) è \(H\)-wide.

Fissiamo quindi \(M\supseteq A\) e fissiamo \(\sigma(x) \in \set{\land}q\). Mostriamo che \(\sigma(x)\land\varphi(x,b)\) è soddisfacibile in \(M\).

Per il Teorema (strano), esiste una combinazione booleana positiva di \(\sigma(x)\land \varphi(x;b_{i})\): \(\theta(x;b_{1},\dots,b_{n})\), con \(b_{i} \equivalentover{M} b\), tale che \(\theta(x;\overline{b}) \iff \psi(x)\) per qualche \(\psi(x) \in \L(M)\).

Affermo che \(\theta(x;\overline{b})\) è consistente. Da questo segue il teorema, poichhé allora \(\theta(x;\overline{b})\) è consistente in \(M\). Quindi \(\varphi(x) \land \theta(x;b_{i})\) è soddisfacibile in \(M\) per qualche \(i\), in particolare per \(m \in M^{x}\)
\begin{equation*}
\U \vDash \sigma(m) \land \varphi(m;b_{i})
\end{equation*}
e siccome \(b_{i} \equivalentover{M} b\) si ha che
\begin{equation*}
\U \vDash \sigma(m) \land \varphi(m;b).
\end{equation*}

Il fatto che \(\theta(x;\overline{b})\) è consistente segue dall'ipotesi che \(D\) sia \(H\)-thick. Infatti
\begin{equation*}
\theta(x:\overline{b}) = \bigvee_{i=1}^{n}\bigwedge_{j=1}^{n}
\big(\sigma(x) \land \varphi(x;b_{\langle i,j\rangle})\big)
\end{equation*}
dove \(b_{\langle i,j\rangle} \equivalentover{M} b\), e in particolare \(b_{\langle i,j\rangle} \equivalentover{\acl^{\eq} A} b\).

Quindi \(b_{\langle i ,j \rangle}\) sono \(H\)-coniugati. Siccome \(D\) è \(H\)-thickl, allora
\begin{equation*}
\bigwedge_{j=1}^{n}\varphi(x;b_{\langle i,j \rangle})
\end{equation*}
sono tutti consistenti in \(X\), quindi consistenti con \(q(x)\) e quindi consistenti con \(\sigma(x)\).
\end{proof}

Ricordiamo che
\begin{itemize}
\item \(D\) non syndetic sse \(\lnot D\) è thick.
\item \(D\) thick sse \(\lnot D\) non syndetic.
\item \(D\) non thick sse \(\lnot D\) è syndetic.
\item \(D\) syndetic sse \(\lnot D\) non thick.
\end{itemize}

\begin{cor}
LSASE:
\begin{enumerate}
\item syndetic implica thick
\item non esiste \(D\) tale che \(D,\lnot D\) entrambi syndetic.
\end{enumerate}
\end{cor}
\begin{proof}
(\(1.\Rightarrow 2.\)): Se 1. e \(D\) syndetic, allora \(D\) è thick. Allora \(\lnot D\) è non syndetic.

(\(2.\Rightarrow 1.\)): Se 2. e \(D\) syndetic, allora \(\lnot D\) non syndetic, quindi \(D\) thick.
\end{proof}

\begin{cor}
LSASE
\begin{enumerate}
\item thick implica syndetic;
\item non esiste \(D\) tale che \(D,\lnot D\) entrambi thick.
\end{enumerate}
\end{cor}

Ora assumiamo che thick=wide (sse syndetic è chiuso per congiunzioni), quindi vale ``syndetic implica thick''. Affermo che se \(H=\Aut(\U/\acl^{\eq} A)\) e \(\X=q(\U^{x})\), \(q(x) \in S(\acl^{\eq}A)\), allora vale anche ``thick implica syndetic''. Infatti abbiamo già visto che \(q(x)\) è stazionario (ovvero esiste un unico tipo \(p(x) \in S_{\varphi}(\U)\) \(H\)-thick/wide/invariante globale che lo estende).

Se per assurdo ``thick non implica syndetic'', allora esistono \(D\), \(\lnot D\) thick = wide, quindi posso estenderli a due diversi \(p(x) \in S_{\varphi}(\U)\) \(H\)-invarianti. Assurdo.

\begin{oss}
La stessa cosa vale se al posto di \(H\) si considera
\begin{equation*}
G=\Aut(\U/A).
\end{equation*}
\end{oss}
\section{INFO ESAME}
\label{sec:org5952367}

Possiamo tenere le dispense del corso durante l'esame. Ha detto che molto probabilmente non ci fornirà la lista degli esercizi papabili.
\newpage
\printbibliography
\end{document}
