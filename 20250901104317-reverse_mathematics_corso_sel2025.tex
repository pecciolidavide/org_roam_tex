% Created 2026-02-07 Sat 19:35
% Intended LaTeX compiler: pdflatex
\documentclass[10pt]{article}
%% CREATO CON ORG - EMACS
\newcommand{\use}[2][]{\usepackage[#1]{#2}}
% PACCHETTI FONDAMENTLAI
\use[utf8]{inputenc}
\use[T1]{fontenc}
\use{graphicx}
\use{longtable}
\use{wrapfig}
\use{rotating}
\use[normalem]{ulem}
\use{amsmath}
\use{amsthm}
\use{amssymb}

\use{eucal} % Cambia mathcal{...}

\use{capt-of}
\use[italian]{babel}
\use[babel]{csquotes}
% bib la TEX lo carica in automatico org-cite
\use{microtype}
\use{lmodern}
\use{subfig} % sottofigure
\use{multicol} % due colonne
\use{lipsum} % lorem ipsum
\use{color} % colori in latex
\use{parskip} % rimuove l'indentazione dei nuovi paragrafi %% Add parbox=false to all new tcolorbox
\use{centernot}
\use[outline]{contour}\contourlength{3pt}
\use{fancyhdr}
\use{layout}
\use[most]{tcolorbox} % Riquadri colorati
\use{ifthen} % IFTHEN
\use{geometry}

% pacchetti matematica
\use{yhmath}
\use{dsfont}
\use{mathrsfs}
\use{cancel} % semplificare
\use{polynom} %divisione tra polinomi
\use{forest} % grafi ad albero
\use{booktabs} % tabelle
\use{commath} %simboli e differenziali
\use{bm} %bold
\use[fulladjust]{marginnote} %to use marginnote for date notes
\use{arrayjobx}%array
\use[intlimits]{empheq} % Riquadri colorati attorno alle equazioni
\use{mathtools}
\use{circuitikz} % Disegnare i circuiti
\use{mathtools}
\use{stmaryrd} % [[ \llbracket ]] \rrbracket
\use{bussproofs} % dimostrazioni

%%%%%%%%%%%%%


%%%% QUIVER
\newcommand{\duepunti}{\,\mathchar\numexpr"6000+`:\relax\,}
% A TikZ style for curved arrows of a fixed height, due to AndréC.
\tikzset{curve/.style={settings={#1},to path={(\tikztostart)
    .. controls ($(\tikztostart)!\pv{pos}!(\tikztotarget)!\pv{height}!270:(\tikztotarget)$)
    and ($(\tikztostart)!1-\pv{pos}!(\tikztotarget)!\pv{height}!270:(\tikztotarget)$)
    .. (\tikztotarget)\tikztonodes}},
    settings/.code={\tikzset{quiver/.cd,#1}
        \def\pv##1{\pgfkeysvalueof{/tikz/quiver/##1}}},
    quiver/.cd,pos/.initial=0.35,height/.initial=0}

% TikZ arrowhead/tail styles.
\tikzset{tail reversed/.code={\pgfsetarrowsstart{tikzcd to}}}
\tikzset{2tail/.code={\pgfsetarrowsstart{Implies[reversed]}}}
\tikzset{2tail reversed/.code={\pgfsetarrowsstart{Implies}}}
% TikZ arrow styles.
\tikzset{no body/.style={/tikz/dash pattern=on 0 off 1mm}}
%%%%%%%%%%


%% DEFINIZIONI COMANDI MATEMATICI
\let\sin\relax %TOGLIE LA DEFINIZIONE SU "\sin"

% cambia la definizione di empty set
% ---
\let\oldemptyset\emptyset
% ---
% \let\emptyset\varnothing
% ---
% \let\emptyset\relax
% \newcommand{\emptyset}{\text{\textnormal{\O}}}
% ---

\DeclareMathOperator{\bounded}{bd}
\DeclareMathOperator{\sin}{sen}
\DeclareMathOperator{\epi}{Epi}
\DeclareMathOperator{\cl}{cl}
\DeclareMathOperator{\graph}{graph}
\DeclareMathOperator{\arcsec}{arcsec}
\DeclareMathOperator{\arccot}{arccot}
\DeclareMathOperator{\arccsc}{arccsc}
\DeclareMathOperator{\spettro}{Spettro}
\DeclareMathOperator{\nulls}{nullspace}
\DeclareMathOperator{\dom}{dom}
\DeclareMathOperator{\ar}{ar}
\DeclareMathOperator{\const}{Const}
\DeclareMathOperator{\fun}{Fun}
\DeclareMathOperator{\rel}{Rel}
\DeclareMathOperator{\altezza}{ht}
\let\det\relax %TOGLIE LA DEFINIZIONE SU "\det"
\DeclareMathOperator{\det}{det}
\DeclareMathOperator{\End}{End}
\DeclareMathOperator{\gl}{GL}
\def\Id{\mathrm{Id}}
\def\id{\mathrm{id}}
\DeclareMathOperator{\I}{\mathds{1}}
\DeclareMathOperator{\II}{II}
\DeclareMathOperator{\rank}{rank}
\DeclareMathOperator{\tr}{tr}
\DeclareMathOperator{\tc}{t.c.}
\DeclareMathOperator{\T}{T}
\DeclareMathOperator{\var}{Var}
\DeclareMathOperator{\cov}{Cov}
\DeclareMathOperator{\st}{st}
\DeclareMathOperator{\mon}{Mon}
\newcommand{\card}[1]{\left\vert #1 \right\vert}
\newcommand{\trasposta}[1]{\prescript{\text{T}}{}{#1}}
\newcommand{\1}{\mathds{1}}
\newcommand{\R}{\mathds{R}}
\newcommand{\diesis}{\#}
\newcommand{\bemolle}{\flat}
\newcommand{\nonstandard}[1]{\prescript{*}{}{#1}}
\newcommand{\starR}{\nonstandard{\R}}
\newcommand{\borel}{\mathscr{B}}
\newcommand{\lebesgue}[1]{\mathscr{L}\left(#1\right)}
\newcommand{\media}{\mathds{E}}
\newcommand{\K}{\mathds{K}}
\newcommand{\A}{\mathds{A}}
\newcommand{\Q}{\mathds{Q}}
\newcommand{\N}{\mathds{N}}
\newcommand{\C}{\mathds{C}}
\newcommand{\Z}{\mathds{Z}}
\newcommand{\qo}{\hspace{1em}\text{q.o.}\,}
\renewcommand{\tilde}[1]{\widetilde{#1}}
\renewcommand{\parallel}{\mathrel{/\mkern-5mu/}}
\newcommand{\parti}[2][]{\wp_{#1}(#2)}
\newcommand{\diff}[1]{\operatorname{d}_{#1}}
\let\oldvec\vec
\renewcommand{\vec}[1]{\overrightarrow{\vphantom{i}#1}}
\newcommand{\floor}[1]{\left\lfloor #1 \right\rfloor}
\newcommand{\cat}[1]{\mathbf{#1}}
\newcommand{\dfreccia}[1]{\xrightarrow{\ #1 \ }}
\newcommand{\sfreccia}[1]{\xleftarrow{\ #1 \ }}
\newcommand{\formalsum}[2]{{\sum_{#1}^{#2}}{\vphantom{\sum}}'}
\newcommand{\minim}[2]{\mu_{#1}\, \left(#2\right)}
\newcommand{\concat}{\null^{\frown}} % concatenazione di stringe
\newcommand{\godelcode}[1]{\langle\!\langle #1 \rangle\!\rangle}
\newcommand{\godeldec}[1]{(\!(#1)\!)}
\newcommand{\termcode}[1]{\ulcorner #1\urcorner}
\newcommand{\partialto}{\dashrightarrow}
\newcommand{\restricted}{\upharpoonright}
\newcommand{\embeds}{\precsim}
\newcommand{\surjects}{\twoheadrightarrow}
\newcommand{\equipotenti}{\asymp}
%% \newcommand{\dotplus}{\mathbin{\dot{+}}} %% A quanto pare esiste già
\newcommand{\bigdot}{\mathbin{\boldsymbol{\cdot}}}
\newcommand{\dotexp}[1]{^{.#1}}
\newcommand{\conv}{\mathbin{*}}
\newcommand{\convolution}[2]{(#1\conv #2)}
\newcommand{\nil}{\mathfrak{N}}
\newcommand{\divisore}{\mathrel{|}}
\newcommand{\simplesso}[1]{\mathrm{e}_{#1}}

\renewcommand{\iff}{\mathrel{\longleftrightarrow}} %% Notazione Logica.
\newcommand{\oldiff}{\mathrel{\Longleftrightarrow}}
\renewcommand{\implies}{\mathrel{\rightarrow}} %% Notazione Logica
\newcommand{\oldimplies}{\mathrel{\Longrightarrow}}
\renewcommand{\impliedby}{\mathrel{\leftarrow}} %% Notazione Logica
\newcommand{\oldimpliedby}{\mathrel{\Longleftarrow}}

\newcommand{\IFF}{\quad\Longleftrightarrow\quad}
\newcommand{\IMPLICA}{\quad\Longrightarrow\quad}


\renewcommand{\descriptionlabel}[1]{\hspace{\labelsep}\normalfont #1} % remove bold from description


%% Definizione di Divergenza di K-L

\DeclarePairedDelimiterX{\infdivx}[2]{(}{)}{%
  #1\;\delimsize\|\;#2%
}
\newcommand{\kldiv}{D_{KL}\infdivx}

%% Definizione di \dotminus

\makeatletter
\newcommand{\dotminus}{\mathbin{\text{\@dotminus}}}

\newcommand{\@dotminus}{%
  \ooalign{\hidewidth\raise1ex\hbox{.}\hidewidth\cr$\m@th-$\cr}%
}
\makeatother

%tramite i prossimi due comandi posso decidere come scrivere i logaritmi naturali in tutti i documenti: ho infatti eliminato qualsiasi differenza tra "ln" e "log": se si vuole qualcosa di diverso bisogna inserire manualmente il tutto
\let\ln\relax
\DeclareMathOperator{\ln}{ln}
\let\log\relax
\DeclareMathOperator{\log}{log}
%%%%%%

%% NUOVI COMANDI
\newcommand{\straniero}[1]{\textit{#1}} %parole straniere
\newcommand{\titolo}[1]{\textsc{#1}} %titoli
\newcommand{\qedd}{\tag*{$\blacksquare$}} %qed per ambienti matemastici
\renewcommand{\qedsymbol}{$\blacksquare$} %modifica colore qed
\newcommand{\ooverline}[1]{\overline{\overline{#1}}}
\newcommand{\circoletto}[1]{\left(#1\right)^{\text{o}}}
%
\newcommand{\qmatrice}[1]{\begin{pmatrix}
#1_{11} & \cdots & #1_{1n}\\
\vdots & \ddots & \vdots \\
#1_{m1} & \cdots & #1_{mn}
\end{pmatrix}}
%
\newcommand{\parentesi}[2]{%
\underset{#1}{\underbrace{#2}}%
}
%
\newcommand{\norma}[1]{% Norma
\left\lVert#1\right\rVert%
}
\newcommand{\scalare}[2]{% Scalare
\left\langle #1, #2\right\rangle
}
%%%%%

%% RESTRIZIONI
\newcommand{\referenze}[2]{
        \phantomsection{}#2\textsuperscript{\textcolor{blue}{\textbf{#1}}}
}

\let\restriction\relax

\def\restriction#1#2{\mathchoice
              {\setbox1\hbox{${\displaystyle #1}_{\scriptstyle #2}$}
              \restrictionaux{#1}{#2}}
              {\setbox1\hbox{${\textstyle #1}_{\scriptstyle #2}$}
              \restrictionaux{#1}{#2}}
              {\setbox1\hbox{${\scriptstyle #1}_{\scriptscriptstyle #2}$}
              \restrictionaux{#1}{#2}}
              {\setbox1\hbox{${\scriptscriptstyle #1}_{\scriptscriptstyle #2}$}
              \restrictionaux{#1}{#2}}}
\def\restrictionaux#1#2{{#1\,\smash{\vrule height .8\ht1 depth .85\dp1}}_{\,#2}}
%%%%%%%%%%%

%%% FORMATTAZIONE FOOTNOTEMARK

\def\footnotemarkformatting#1{[#1]}
\renewcommand{\thefootnote}{\footnotemarkformatting{\arabic{footnote}}}

%% SEZIONE GRAFICA
\use{tikz}
\usetikzlibrary{matrix, patterns, calc, decorations.pathreplacing, hobby, decorations.markings, decorations.pathmorphing, babel}
\use{tikz-3dplot}
\use{mathrsfs} %per geogebra
\use{tikz-cd}
\tikzset
{
  %surface/.style={fill=black!10, shading=ball,fill opacity=0.4},
  plane/.style={black,pattern=north east lines},
  curve/.style={black,line width=0.5mm},
  dritto/.style={decoration={markings,mark=at position 0.5 with {\arrow{Stealth}}}, postaction=decorate},
  rovescio/.style={decoration={markings,mark=at position 0.5 with {\arrow{Stealth[reversed]}}}, postaction=decorate}
}
\use{pgfplots} % stampare le funzioni
        \pgfplotsset{/pgf/number format/use comma,compat=1.15}
        %\pgfplotsset{compat=1.15} %per geogebra
        \usepgfplotslibrary{fillbetween, polar}
%%%%%%

%% CITAZIONI
\use{lineno}

\newcommand{\citazione}[1]{%
  \begin{quotation}
  \begin{linenumbers}
  \modulolinenumbers[5]
  \begingroup
  \setlength{\parindent}{0cm}
  \noindent #1
  \endgroup
  \end{linenumbers}
  \end{quotation}\setcounter{linenumber}{1}
  }
%%%%%%

%%%%%%%%%%%%%%%%%%%%%%%%%%%%%%%%%%%%%%%%%%%%
%%%%%%%%%%%%%%%%%%%%%%%%%%%%%%%%%%%%%%%%%%%%

%% AMS THM

\theoremstyle{definition}% default
\newtheorem{thm}{Teorema}[section]
\newtheorem{lem}[thm]{Lemma}
\newtheorem{prop}[thm]{Proposizione}
\newtheorem{cor}[thm]{Corollario}
\newtheorem{esempio}[thm]{Esempio}
\theoremstyle{plain}
\newtheorem{definizione}[thm]{Definizione}
\theoremstyle{remark}
\newtheorem*{oss}{Osservazione}


%%%%%%%%%%%%%%%%%%%%%%%%%%%%%%%%%%%%%%%%%%%%
%%%%%%%%%%%%%%%%%%%%%%%%%%%%%%%%%%%%%%%%%%%%

\use{hyperref}
\hypersetup{%
        pdfauthor={Davide Peccioli},
        pdfsubject={},
        allcolors=black,
        citecolor=black,
%	colorlinks=true,
        bookmarksopen=true}
\setcounter{secnumdepth}{0} % rimuove i numeri di sezione senza rimuovere le ref
\renewcommand{\href}[2]{\textcolor{blue}{#2}} % disabilita il comando href
\use{enotez} %
\setenotez{%
 mark-format = \footnotemarkformatting % Mette i numeri tra parentesi quadre%
}\let\footnote=\endnote % rende tutte le note a pié pagina come delle note a fine file 


\let\olddocument\document % modifico l'ambiende documenti per non dover stampare \printendnote
\let\oldenddocument\enddocument
\renewenvironment{document}%
{%
  \olddocument
}{%
  \printendnotes\oldenddocument
}
\renewcommand{\thethm}{\arabic{thm}}

\usepackage[hyperref]{biblatex}
\addbibresource{~/Documents/org/roam/bib/master.bib}
\author{Davide Peccioli}
\date{\today}
\title{}
\begin{document}

\section{Reverse Mathematics [CORSO SEL2025]}
\label{sec:org8e56f5d}
\subsection{Introduzione}
\label{sec:orgeeec0e6}

Partiamo dal Teorema di Pitagora

\begin{thm}
Dato un triangolo rettangolo di lati \(a,b,c\), si ha \(a^{2}+b^{2}=c^{2}\).
\end{thm}

Questo è dimostrato a partire dagli \uline{assiomi di Euclide}.

L'idea generale della matematica è risolvere questo problema:
\begin{equation*}
\text{Assiomi} \implies \text{Teoremi}
\end{equation*}

L'obiettivo della Reverse Mathematics è fare il processo inverso, ovvero rispondere a:
\begin{description}
\item[{(D1)}] Dato un teorema \(\varphi\), trovare gli assiomi minimi/minimali per dimostrarlo.
\end{description}

Una situazione ottimale è che esiste un gruppo di assiomi minimo \uline{equivalente} al teorema preso in questione. Questo può succedere per teoremi diversi: ad esempio, il Teorema di Pitagora è equivalente ad un gruppo degli assiomi di Euclide, e questo gruppo è equivalente al Teorema di Euclide, quindi Pitagora è equivalente ad Euclide.

Quindi, la seconda domanda è:

\begin{description}
\item[{(D2)}] Dato un teorema \(\varphi\), trovare i teoremi equivalenti.
\end{description}

Formalmente, fissato un teorema \(A\), si fissa una teoria base tale che \(T\not\vdash A\). Si cerca un insieme di assiomi \(S\) tali che
\begin{equation*}
T+S\vdash A,\qquad T+A\vdash S
\end{equation*}
e quindi \(T\vdash T\iff S\).
\subsection{Aritmetica del second'ordine \(\mathcal{L}_{2}\) e \(Z_{2}\).}
\label{sec:orgf6dc5ba}

Si consideri il linguaggio del prim'ordine
\begin{equation*}
\mathcal{L}_{2} \coloneqq \set{0,S,+,\cdot,=,\le,\in}
\end{equation*}
con due tipi di variabili:
\begin{itemize}
\item variabili numeriche: \(x,y,z,n,m,\dots\).
\item variabili insiemi: \(X,Y,Z\).
\end{itemize}

\uline{Assiomi}:
\begin{itemize}
\item assiomi di base (ovvero l'Aritmetica di Robinson)
\begin{equation*}
  0+x=x,\ 0\cdot x=0,\ 0\le x\ x+y=y+x
\end{equation*}
\item assioma di induzione
\begin{description}
\item[{\(\mathrm{Ind}\)}] \(0 \in X\land \forall n [n \in X\implies S(n) \in X]\implies \forall  n (n \in X)\)
\end{description}
\item schema di induzione: per ogni formula \(\varphi \in \mathcal{L}_{2}\)
\begin{description}
\item[{\(\mathrm{Ind-Schema}\)}] \(\varphi(0)\land \forall  n[\varphi(n)\implies \varphi\big(S(n)\big)]\implies \forall n\ \varphi(n).\)
\end{description}
\item schema di comprensione: per ogni formula \(\varphi \in \mathcal{L}_{2}\)
\begin{description}
\item[{\(\mathrm{CA}\)}] \(\exists X\ \forall  n\ [n \in X\iff \varphi(n)]\)
\end{description}
\end{itemize}

\begin{esempio}
Gli assiomi di base + l'assioma di comprensione dimostrano che l'assioma di induzione è equivalente allora schema di induzione
\end{esempio}

Quindi \(Z_{2}\) è dato dagli assiomi di base + \(\mathrm{Ca}\) + \(\mathrm{Ind}\). Questa teoria però è \uline{troppo forte}, quindi si considerano, per un insieme\(\Gamma \subseteq \mathcal{L}_{2}\), gli assiomi:
\begin{description}
\item[{\(\Gamma-\mathrm{Ind}\)}] \(\varphi(0)\land \forall  n[\varphi(n)\implies \varphi\big(S(n)\big)]\implies \forall n\ \varphi(n)\) per ogni \(\varphi \in \Gamma\);
\item[{\(\Gamma-\mathrm{CA}\)}] \(\exists X\ \forall  n\ [n \in X\iff \varphi(n)]\) per ogni \(\varphi \in \Gamma\).
\end{description}

\begin{definizione}
Quantificatori limitati: \href{20250603170559-complessita_di_una_formula_del_modello_standard.org}{Quantificatore limitato nel linguaggio dell'aritmetica}
\end{definizione}

\begin{definizione}
Famiglia \(\Sigma_{n}^{0}, \Pi_{n}^{0}\): \href{20250603170559-complessita_di_una_formula_del_modello_standard.org}{Complessità di una formula nel linguaggio dell'aritmetica}
\end{definizione}

\begin{esempio}
Alcune formule
\begin{align*}
x\cdot y = y\cdot x &\in \Sigma_{0}^{0}\\
\mathrm{Pr}(t):\quad t>1 \land \forall  n\le t\ \forall  m\le t\ (t=n\cdot m \implies n=1\lor m=1) &\in \Sigma_{0}^{0}\\
\forall n\ \exists p\ (p>n \land \mathrm{Pr}(p)) \in \Pi_{2}^{0}\\
\forall n\ \big[n \in X\implies \exists m \le n\ (n=m+m)\big]&\in \Pi^{0}_{1}
\end{align*}
\end{esempio}

\begin{definizione}
Si definiscono le famiglie \(\Sigma_{n}^{1},\Pi_{n}^{1}\):
\begin{itemize}
\item \(\Sigma_{0}^{1} = \Pi_{0}^{1}\coloneqq \bigcup_{n} \Sigma_{n}^{1} = \bigcup_{n} \Pi_{n}^{1}\).
\item \(\Sigma_{n}^{1} =\)
\item \(\Pi_{n}^{1} =\) (definite come al solito)
\end{itemize}
\end{definizione}
\subsection{``Big Five''}
\label{sec:org7e6cb23}

L'idea è che quasi tutti i teoremi sono equivalenti a uno dei seguenti gruppi di assiomi
\subsubsection{RCA\textsubscript{0}}
\label{sec:org9529963}

Questa lista di assiomi è data da: Assiomi di base + \(\Sigma_{1}^{0}-\mathrm{Ind}\) + \(\Delta_{1}^{0}-\mathrm{CA}\) dove
\begin{description}
\item[{(\(\Delta_{1}^{0}-\mathrm{CA}\))}] \(\forall n\ (\varphi(n)\iff\psi(n))\implies \exists X\ \forall n [n \in X\iff \varphi(n)]\) per ogni \(\varphi \in \Sigma_{1}^{0}\) e \(\psi \in \Pi_{1}^{0}\).
\end{description}

L'idea è che, a meno di equivalenza logica, \(\Delta_{n}^{0} = \Sigma_{n}^{0}\cap \Pi_{n}^{0}\).

\begin{prop}
Sono fatti equivalenti
\begin{itemize}
\item \(\varphi(n)\in\Delta_{1}^{0}\)
\item \(\varphi(n)\) è decidibile.
\end{itemize}
\end{prop}

\begin{thm}
RCA\textsubscript{0} prova:
\begin{itemize}
\item \(\R\) non è numerabile
\item Teorema di Bolzano
\item Ogni campo numerabile ha una chiusura algebrica
\end{itemize}
e non prova:
\begin{itemize}
\item ogni anello ha un ideale primo
\item ogni sequenza non limitata di reali ha limite
\item non unicità.
\end{itemize}
\end{thm}
\subsubsection{WKL\textsubscript{0}}
\label{sec:org6a996ad}

Questa lista di assiomi è data da RCA\textsubscript{0} + WKL dove:
\begin{description}
\item[{\textbf{\textbf{WKL}}}] Sia \(T\) un albero binario infinito. Allora \(T\) ha un cammino.
\end{description}

\begin{thm}
In RCA\textsubscript{0} sono fatti equivalenti:
\begin{itemize}
\item WKL
\item ogni anello numerabile ha un ideale primo;
\item ogni funzione continua in \([0,1]\) è limitata
\item Teorema di completezza di Godel
\item Ogni campo numerabile ha un'unica chiusura algebrica
\end{itemize}
\end{thm}
\subsubsection{ACA\textsubscript{0}}
\label{sec:org690667f}

Questa lista di assiomi è data da RCA\textsubscript{0} + \(\mathrm{Ind}\) + \(\Pi^{1}_{0}-\mathrm{CA}\).

\begin{thm}
In RCA\textsubscript{0} sono fatti equivalenti:
\begin{itemize}
\item ACA\textsubscript{0}
\item Lemma di Koning
\item Ogni anello numerabile ha un ideale
\end{itemize}
\end{thm}

Idea: ACA\textsubscript{0} è equivalente a PA
\subsubsection{ATR\textsubscript{0}}
\label{sec:org1addbc2}
\subsubsection{\(\Pi_{1}^{1}-\mathrm{CA}_{0}\)}
\label{sec:org8e164a9}

\subsubsection{Riferimenti:}
\label{sec:org8b44697}

\begin{itemize}
\item Simpson, Subsystems of second order aritmetic
\item Stillwell, Reverse Mathematics
\item Hitschelot, Slicing the truth
\item Dzhafarov and Mummert, Reverse Mathematics
\end{itemize}
\subsection{Esempio}
\label{sec:org3a41d69}

\begin{thm}
In RCA\textsubscript{0} sono fatti equivalenti:
\begin{enumerate}
\item ACA\textsubscript{0}
\item \(\Sigma^{0}_{1}-\mathrm{CA}\)
\item per ogni \(f:\N\to \N\) iniettiva esiste \(\operatorname{rng}(f)\) come insieme, ovvero
\begin{equation*}
 \exists X\ \forall n\ \big[n \in X\iff \exists m(f(m=n))\big]
\end{equation*}
\end{enumerate}
\end{thm}
\end{document}
