% Created 2026-02-07 Sat 19:33
% Intended LaTeX compiler: pdflatex
\documentclass[10pt]{article}
%% CREATO CON ORG - EMACS
\newcommand{\use}[2][]{\usepackage[#1]{#2}}
% PACCHETTI FONDAMENTLAI
\use[utf8]{inputenc}
\use[T1]{fontenc}
\use{graphicx}
\use{longtable}
\use{wrapfig}
\use{rotating}
\use[normalem]{ulem}
\use{amsmath}
\use{amsthm}
\use{amssymb}

\use{eucal} % Cambia mathcal{...}

\use{capt-of}
\use[italian]{babel}
\use[babel]{csquotes}
% bib la TEX lo carica in automatico org-cite
\use{microtype}
\use{lmodern}
\use{subfig} % sottofigure
\use{multicol} % due colonne
\use{lipsum} % lorem ipsum
\use{color} % colori in latex
\use{parskip} % rimuove l'indentazione dei nuovi paragrafi %% Add parbox=false to all new tcolorbox
\use{centernot}
\use[outline]{contour}\contourlength{3pt}
\use{fancyhdr}
\use{layout}
\use[most]{tcolorbox} % Riquadri colorati
\use{ifthen} % IFTHEN
\use{geometry}

% pacchetti matematica
\use{yhmath}
\use{dsfont}
\use{mathrsfs}
\use{cancel} % semplificare
\use{polynom} %divisione tra polinomi
\use{forest} % grafi ad albero
\use{booktabs} % tabelle
\use{commath} %simboli e differenziali
\use{bm} %bold
\use[fulladjust]{marginnote} %to use marginnote for date notes
\use{arrayjobx}%array
\use[intlimits]{empheq} % Riquadri colorati attorno alle equazioni
\use{mathtools}
\use{circuitikz} % Disegnare i circuiti
\use{mathtools}
\use{stmaryrd} % [[ \llbracket ]] \rrbracket
\use{bussproofs} % dimostrazioni

%%%%%%%%%%%%%


%%%% QUIVER
\newcommand{\duepunti}{\,\mathchar\numexpr"6000+`:\relax\,}
% A TikZ style for curved arrows of a fixed height, due to AndréC.
\tikzset{curve/.style={settings={#1},to path={(\tikztostart)
    .. controls ($(\tikztostart)!\pv{pos}!(\tikztotarget)!\pv{height}!270:(\tikztotarget)$)
    and ($(\tikztostart)!1-\pv{pos}!(\tikztotarget)!\pv{height}!270:(\tikztotarget)$)
    .. (\tikztotarget)\tikztonodes}},
    settings/.code={\tikzset{quiver/.cd,#1}
        \def\pv##1{\pgfkeysvalueof{/tikz/quiver/##1}}},
    quiver/.cd,pos/.initial=0.35,height/.initial=0}

% TikZ arrowhead/tail styles.
\tikzset{tail reversed/.code={\pgfsetarrowsstart{tikzcd to}}}
\tikzset{2tail/.code={\pgfsetarrowsstart{Implies[reversed]}}}
\tikzset{2tail reversed/.code={\pgfsetarrowsstart{Implies}}}
% TikZ arrow styles.
\tikzset{no body/.style={/tikz/dash pattern=on 0 off 1mm}}
%%%%%%%%%%


%% DEFINIZIONI COMANDI MATEMATICI
\let\sin\relax %TOGLIE LA DEFINIZIONE SU "\sin"

% cambia la definizione di empty set
% ---
\let\oldemptyset\emptyset
% ---
% \let\emptyset\varnothing
% ---
% \let\emptyset\relax
% \newcommand{\emptyset}{\text{\textnormal{\O}}}
% ---

\DeclareMathOperator{\bounded}{bd}
\DeclareMathOperator{\sin}{sen}
\DeclareMathOperator{\epi}{Epi}
\DeclareMathOperator{\cl}{cl}
\DeclareMathOperator{\graph}{graph}
\DeclareMathOperator{\arcsec}{arcsec}
\DeclareMathOperator{\arccot}{arccot}
\DeclareMathOperator{\arccsc}{arccsc}
\DeclareMathOperator{\spettro}{Spettro}
\DeclareMathOperator{\nulls}{nullspace}
\DeclareMathOperator{\dom}{dom}
\DeclareMathOperator{\ar}{ar}
\DeclareMathOperator{\const}{Const}
\DeclareMathOperator{\fun}{Fun}
\DeclareMathOperator{\rel}{Rel}
\DeclareMathOperator{\altezza}{ht}
\let\det\relax %TOGLIE LA DEFINIZIONE SU "\det"
\DeclareMathOperator{\det}{det}
\DeclareMathOperator{\End}{End}
\DeclareMathOperator{\gl}{GL}
\def\Id{\mathrm{Id}}
\def\id{\mathrm{id}}
\DeclareMathOperator{\I}{\mathds{1}}
\DeclareMathOperator{\II}{II}
\DeclareMathOperator{\rank}{rank}
\DeclareMathOperator{\tr}{tr}
\DeclareMathOperator{\tc}{t.c.}
\DeclareMathOperator{\T}{T}
\DeclareMathOperator{\var}{Var}
\DeclareMathOperator{\cov}{Cov}
\DeclareMathOperator{\st}{st}
\DeclareMathOperator{\mon}{Mon}
\newcommand{\card}[1]{\left\vert #1 \right\vert}
\newcommand{\trasposta}[1]{\prescript{\text{T}}{}{#1}}
\newcommand{\1}{\mathds{1}}
\newcommand{\R}{\mathds{R}}
\newcommand{\diesis}{\#}
\newcommand{\bemolle}{\flat}
\newcommand{\nonstandard}[1]{\prescript{*}{}{#1}}
\newcommand{\starR}{\nonstandard{\R}}
\newcommand{\borel}{\mathscr{B}}
\newcommand{\lebesgue}[1]{\mathscr{L}\left(#1\right)}
\newcommand{\media}{\mathds{E}}
\newcommand{\K}{\mathds{K}}
\newcommand{\A}{\mathds{A}}
\newcommand{\Q}{\mathds{Q}}
\newcommand{\N}{\mathds{N}}
\newcommand{\C}{\mathds{C}}
\newcommand{\Z}{\mathds{Z}}
\newcommand{\qo}{\hspace{1em}\text{q.o.}\,}
\renewcommand{\tilde}[1]{\widetilde{#1}}
\renewcommand{\parallel}{\mathrel{/\mkern-5mu/}}
\newcommand{\parti}[2][]{\wp_{#1}(#2)}
\newcommand{\diff}[1]{\operatorname{d}_{#1}}
\let\oldvec\vec
\renewcommand{\vec}[1]{\overrightarrow{\vphantom{i}#1}}
\newcommand{\floor}[1]{\left\lfloor #1 \right\rfloor}
\newcommand{\cat}[1]{\mathbf{#1}}
\newcommand{\dfreccia}[1]{\xrightarrow{\ #1 \ }}
\newcommand{\sfreccia}[1]{\xleftarrow{\ #1 \ }}
\newcommand{\formalsum}[2]{{\sum_{#1}^{#2}}{\vphantom{\sum}}'}
\newcommand{\minim}[2]{\mu_{#1}\, \left(#2\right)}
\newcommand{\concat}{\null^{\frown}} % concatenazione di stringe
\newcommand{\godelcode}[1]{\langle\!\langle #1 \rangle\!\rangle}
\newcommand{\godeldec}[1]{(\!(#1)\!)}
\newcommand{\termcode}[1]{\ulcorner #1\urcorner}
\newcommand{\partialto}{\dashrightarrow}
\newcommand{\restricted}{\upharpoonright}
\newcommand{\embeds}{\precsim}
\newcommand{\surjects}{\twoheadrightarrow}
\newcommand{\equipotenti}{\asymp}
%% \newcommand{\dotplus}{\mathbin{\dot{+}}} %% A quanto pare esiste già
\newcommand{\bigdot}{\mathbin{\boldsymbol{\cdot}}}
\newcommand{\dotexp}[1]{^{.#1}}
\newcommand{\conv}{\mathbin{*}}
\newcommand{\convolution}[2]{(#1\conv #2)}
\newcommand{\nil}{\mathfrak{N}}
\newcommand{\divisore}{\mathrel{|}}
\newcommand{\simplesso}[1]{\mathrm{e}_{#1}}

\renewcommand{\iff}{\mathrel{\longleftrightarrow}} %% Notazione Logica.
\newcommand{\oldiff}{\mathrel{\Longleftrightarrow}}
\renewcommand{\implies}{\mathrel{\rightarrow}} %% Notazione Logica
\newcommand{\oldimplies}{\mathrel{\Longrightarrow}}
\renewcommand{\impliedby}{\mathrel{\leftarrow}} %% Notazione Logica
\newcommand{\oldimpliedby}{\mathrel{\Longleftarrow}}

\newcommand{\IFF}{\quad\Longleftrightarrow\quad}
\newcommand{\IMPLICA}{\quad\Longrightarrow\quad}


\renewcommand{\descriptionlabel}[1]{\hspace{\labelsep}\normalfont #1} % remove bold from description


%% Definizione di Divergenza di K-L

\DeclarePairedDelimiterX{\infdivx}[2]{(}{)}{%
  #1\;\delimsize\|\;#2%
}
\newcommand{\kldiv}{D_{KL}\infdivx}

%% Definizione di \dotminus

\makeatletter
\newcommand{\dotminus}{\mathbin{\text{\@dotminus}}}

\newcommand{\@dotminus}{%
  \ooalign{\hidewidth\raise1ex\hbox{.}\hidewidth\cr$\m@th-$\cr}%
}
\makeatother

%tramite i prossimi due comandi posso decidere come scrivere i logaritmi naturali in tutti i documenti: ho infatti eliminato qualsiasi differenza tra "ln" e "log": se si vuole qualcosa di diverso bisogna inserire manualmente il tutto
\let\ln\relax
\DeclareMathOperator{\ln}{ln}
\let\log\relax
\DeclareMathOperator{\log}{log}
%%%%%%

%% NUOVI COMANDI
\newcommand{\straniero}[1]{\textit{#1}} %parole straniere
\newcommand{\titolo}[1]{\textsc{#1}} %titoli
\newcommand{\qedd}{\tag*{$\blacksquare$}} %qed per ambienti matemastici
\renewcommand{\qedsymbol}{$\blacksquare$} %modifica colore qed
\newcommand{\ooverline}[1]{\overline{\overline{#1}}}
\newcommand{\circoletto}[1]{\left(#1\right)^{\text{o}}}
%
\newcommand{\qmatrice}[1]{\begin{pmatrix}
#1_{11} & \cdots & #1_{1n}\\
\vdots & \ddots & \vdots \\
#1_{m1} & \cdots & #1_{mn}
\end{pmatrix}}
%
\newcommand{\parentesi}[2]{%
\underset{#1}{\underbrace{#2}}%
}
%
\newcommand{\norma}[1]{% Norma
\left\lVert#1\right\rVert%
}
\newcommand{\scalare}[2]{% Scalare
\left\langle #1, #2\right\rangle
}
%%%%%

%% RESTRIZIONI
\newcommand{\referenze}[2]{
        \phantomsection{}#2\textsuperscript{\textcolor{blue}{\textbf{#1}}}
}

\let\restriction\relax

\def\restriction#1#2{\mathchoice
              {\setbox1\hbox{${\displaystyle #1}_{\scriptstyle #2}$}
              \restrictionaux{#1}{#2}}
              {\setbox1\hbox{${\textstyle #1}_{\scriptstyle #2}$}
              \restrictionaux{#1}{#2}}
              {\setbox1\hbox{${\scriptstyle #1}_{\scriptscriptstyle #2}$}
              \restrictionaux{#1}{#2}}
              {\setbox1\hbox{${\scriptscriptstyle #1}_{\scriptscriptstyle #2}$}
              \restrictionaux{#1}{#2}}}
\def\restrictionaux#1#2{{#1\,\smash{\vrule height .8\ht1 depth .85\dp1}}_{\,#2}}
%%%%%%%%%%%

%%% FORMATTAZIONE FOOTNOTEMARK

\def\footnotemarkformatting#1{[#1]}
\renewcommand{\thefootnote}{\footnotemarkformatting{\arabic{footnote}}}

%% SEZIONE GRAFICA
\use{tikz}
\usetikzlibrary{matrix, patterns, calc, decorations.pathreplacing, hobby, decorations.markings, decorations.pathmorphing, babel}
\use{tikz-3dplot}
\use{mathrsfs} %per geogebra
\use{tikz-cd}
\tikzset
{
  %surface/.style={fill=black!10, shading=ball,fill opacity=0.4},
  plane/.style={black,pattern=north east lines},
  curve/.style={black,line width=0.5mm},
  dritto/.style={decoration={markings,mark=at position 0.5 with {\arrow{Stealth}}}, postaction=decorate},
  rovescio/.style={decoration={markings,mark=at position 0.5 with {\arrow{Stealth[reversed]}}}, postaction=decorate}
}
\use{pgfplots} % stampare le funzioni
        \pgfplotsset{/pgf/number format/use comma,compat=1.15}
        %\pgfplotsset{compat=1.15} %per geogebra
        \usepgfplotslibrary{fillbetween, polar}
%%%%%%

%% CITAZIONI
\use{lineno}

\newcommand{\citazione}[1]{%
  \begin{quotation}
  \begin{linenumbers}
  \modulolinenumbers[5]
  \begingroup
  \setlength{\parindent}{0cm}
  \noindent #1
  \endgroup
  \end{linenumbers}
  \end{quotation}\setcounter{linenumber}{1}
  }
%%%%%%

%%%%%%%%%%%%%%%%%%%%%%%%%%%%%%%%%%%%%%%%%%%%
%%%%%%%%%%%%%%%%%%%%%%%%%%%%%%%%%%%%%%%%%%%%

%% AMS THM

\theoremstyle{definition}% default
\newtheorem{thm}{Teorema}[section]
\newtheorem{lem}[thm]{Lemma}
\newtheorem{prop}[thm]{Proposizione}
\newtheorem{cor}[thm]{Corollario}
\newtheorem{esempio}[thm]{Esempio}
\theoremstyle{plain}
\newtheorem{definizione}[thm]{Definizione}
\theoremstyle{remark}
\newtheorem*{oss}{Osservazione}


%%%%%%%%%%%%%%%%%%%%%%%%%%%%%%%%%%%%%%%%%%%%
%%%%%%%%%%%%%%%%%%%%%%%%%%%%%%%%%%%%%%%%%%%%

\use{hyperref}
\hypersetup{%
        pdfauthor={Davide Peccioli},
        pdfsubject={},
        allcolors=black,
        citecolor=black,
%	colorlinks=true,
        bookmarksopen=true}
\setcounter{secnumdepth}{0} % rimuove i numeri di sezione senza rimuovere le ref
\renewcommand{\href}[2]{\textcolor{blue}{#2}} % disabilita il comando href
\use{enotez} %
\setenotez{%
 mark-format = \footnotemarkformatting % Mette i numeri tra parentesi quadre%
}\let\footnote=\endnote % rende tutte le note a pié pagina come delle note a fine file 


\let\olddocument\document % modifico l'ambiende documenti per non dover stampare \printendnote
\let\oldenddocument\enddocument
\renewenvironment{document}%
{%
  \olddocument
}{%
  \printendnotes\oldenddocument
}
\renewcommand{\thethm}{\arabic{thm}}

\usepackage[hyperref]{biblatex}
\addbibresource{~/Documents/org/roam/bib/master.bib}
\author{Davide Peccioli}
\date{\today}
\title{}
\begin{document}

\section{Coomologia dello spazio proiettivo reale}
\label{sec:org350759d}
\begin{thm}
Sia \(\mathds{P}^{n} \R\) lo \href{20241231115051-spazio_proiettivo.org}{spazio proiettivo} reale di dimensione \(n\). La sua \href{20251115172442-gruppo_di_coomologia_di_de_rham.org}{coomologia di De Rham} è
\begin{equation*}
H^{k}(\mathds{P}^{n}\R) =%
\begin{cases}
\R & k = 0\\
\R & k = n\text{ dispari}\\
0 & \text{altrimenti}
\end{cases}
\end{equation*}
\end{thm}
\begin{proof}
Consideriamo le seguenti mappe:
\begin{itemize}
\item Si consideri \(\mathds{S}^{n} \subseteq \R^{n+1}\), e si consideri la mappa antipodale:
\begin{align*}
a: \mathds{S}^{n} &\longrightarrow \mathds{S}^{n}\\
\bm{x} &\longmapsto -\bm{x}
\end{align*}
Questa induce i seguenti pullback\footnote{Vedi:
\begin{itemize}
\item \href{20251115174001-pullback_di_una_funzione_tra_varieta_differenziabili.org}{Pullback di una funzione tra varietà differenziabili}
\item \href{20251121174615-pullback_di_una_funzione_tra_varieta_differenziabili_in_coomologia.org}{Pullback di una funzione tra varietà differenziabili in coomologia}
\end{itemize}}:
\begin{align*}
  a^{*}: \Omega^{k}(\mathds{S}^{n}) & \longrightarrow \Omega^{k}(\mathds{S}^{n})\\
  \bm{a}^{*}: H^{k}(\mathds{S}^{n}) & \longrightarrow H^{k}(\mathds{S}^{n}).
\end{align*}
\item Si consideri la proiezione al \href{20250129155316-spazio_topologico_quoziente.org}{quoziente}:
\begin{equation*}
  \pi : \mathds{S}^{n} \to \mathds{P}^{n}\R = \mathds{S}^{n}/\sim_{a}
\end{equation*}
Che induce i seguenti pullback:
\begin{align*}
  \pi^{*}: \Omega^{k}(\mathds{P}^{n}\R) & \longrightarrow \Omega^{k}(\mathds{S}^{n})\\
  \bm{\pi}^{*}: H^{k}(\mathds{P}^{n}\R) & \longrightarrow H^{k}(\mathds{S}^{n}).
\end{align*}
\end{itemize}

Se \(k=0\) allora, siccome \(\mathds{P}^{n}\R\) è connesso, si ha che \(H^{0}(\mathds{P}^{n}\R) = \R\).

Sia quindi ora \(k>0\).
\begin{enumerate}
\item \textbf{Claim}:
\(\Omega^{k}(\mathds{S}^{n}) = \Omega^{k}(\mathds{S}^{n})_{+} \oplus \Omega^{k}(\mathds{S}^{n})_{-}\)\footnote{Vedi ``\href{20241213095808-somma_diretta.org}{Somma Diretta}''}, e inoltre
\(\pi^{*} [\Omega^{k}(\mathds{P}^{n}\R)] \subseteq \Omega^{k}(\mathds{S}^{n})_{ +}\)\footnote{Vedi:
\begin{itemize}
\item \href{20250202173528-dominio_range_e_campo_di_una_classe_relazione.org}{Range di una funzione}
\item \href{20250114103118-sottospazio_vettoriale.org}{Sottospazio vettoriale}
\end{itemize}}.

\emph{dim.}: Si definiscono:
\begin{align*}
 \Omega^{k}(\mathds{S}^{n})_{+} &\coloneqq %
 \set{\omega \in \Omega^{k}(\mathds{S}^{n}) \mid a^{*}\omega = \omega};\\
 \Omega^{k}(\mathds{S}^{n})_{-} &\coloneqq %
 \set{\omega \in \Omega^{k}(\mathds{S}^{n}) \mid a^{*}\omega = - \omega}
\end{align*}
gli insiemi delle forme \uline{invarianti} e \uline{antiinvarianti}.

Siccome \(a^{*}\circ a^{*} = \Id_{\Omega^{k}(\mathds{S}^{n})}\), segue la decomposizione. In particolare:
\begin{equation*}
 \omega = \bigg(\frac{\omega+a^{*}\omega}{2}\bigg) + %
 	\bigg(\frac{\omega-a^{*}\omega}{2}\bigg) %
 \tag{\(\star\)}
\end{equation*}

Inoltre, siccome per definizione di \(\mathds{P}^{n}\R\): \(\pi \circ a = \pi\): \href{20251115174001-pullback_di_una_funzione_tra_varieta_differenziabili.org}{allora}
\begin{equation*}
 (\pi \circ a)^{*} = \pi^{*}%
 \IMPLICA %
 a^{*} \circ \pi^{*} = \pi^{*}
\end{equation*}
e pertanto, per ogni \(\omega \in \Omega^{k}(\mathds{P}^{n}\R)\):
\begin{equation*}
 a^{*}(\pi^{*}\omega) = \pi^{*}\omega.
\end{equation*}
Segue che
\(\pi^{*} [\Omega^{k}(\mathds{P}^{n}\R)] \subseteq \Omega^{k}(\mathds{S}^{n})_{ +}\)
\item \textbf{Claim}: In realtà \(\pi^{*}: \Omega^{k}(\mathds{P}^{n}\R) \to \Omega^{k}(\mathds{S}^{n})_{+}\) è isomorfismo.

\emph{Non dimostrato}.

\item \textbf{Claim}: Segue che \(H^{k}(\mathds{S}^{n}) = H^{k}(\mathds{S}^{n})_{+} \oplus H^{k}(\mathds{S}^{n})_{-}\), e inoltre \(H^{k}(\mathds{P}^{n}\R) \cong H^{k}(\mathds{S}^{n})_{ +}\).

\emph{dim.}:
Si definiscano:
\begin{align*}
 H^{k}(\mathds{S}^{n})_{+} &\coloneqq \set{[\omega] \in H^{k}(\mathds{S}^{n}) \mid \bm{a}^{*}[\omega] = [\omega]}\\
 H^{k}(\mathds{S}^{n})_{-} &\coloneqq \set{[\omega] \in H^{k}(\mathds{S}^{n}) \mid \bm{a}^{*}[\omega] = -[\omega]}
\end{align*}

Per dimostrare la somma diretta è sufficiente:
\begin{itemize}
\item Mostrare che \(H^{k}(\mathds{S}^{n})_{+} \cap H^{k}(\mathds{S}^{n})_{-} = \set{0}\)

Sia quindi \([\omega] \in H^{k}(\mathds{S}^{n})_{+} \cap H^{k}(\mathds{S}^{n})_{-}\): allora
\begin{equation*}
   -[\omega] = \bm{a}^{*}[\omega] =[ \omega]
\end{equation*}
e pertanto \([\omega]= 0\).

\item Mostrare che se \([\omega] \in H^{k}(\mathds{S}^{n})\) allora esistono \([\omega_{1}] \in H^{k}(\mathds{S}^{n})_{+}\) e \([\omega_{2}] \in H^{k}(\mathds{S}^{n})_{-}\) tali che:
\begin{equation*}
   [\omega] = [\omega_{1}] + [\omega_{2}].
\end{equation*}

Ma in particolare, per (\(\star\)):
\begin{equation*}
   \omega = \parentesi{\null \in \Omega^{k}(\mathds{S}^{n})_{+}}{\bigg(\frac{\omega+a^{*}\omega}{2}\bigg)} + %
   \parentesi{\null \in \Omega^{k}(\mathds{S}^{n})_{-}}{\bigg(\frac{\omega-a^{*}\omega}{2}\bigg)} %
\end{equation*}
e dunque, ponendo:
\begin{align*}
   \omega_{1} &\coloneqq \frac{\omega+a^{*}\omega}{2}\\
   \omega_{2} &\coloneqq \frac{\omega-a^{*}\omega}{2}
\end{align*}
si ottiene che
\begin{align*}
   \bm{a}^{*}[\omega_{1}] = [a^{*}\omega_{1}] &= [\omega_{1}] \\
   \bm{a}^{*}[\omega_{2}] = [a^{*}\omega_{2}] = [- \omega_{2}] &= -[\omega_{2}]\\
   [\omega_{1}]+[\omega_{2}] &= [\omega].
\end{align*}
\end{itemize}

Rispetto all'isomorfismo \(H^{k}(\mathds{P}^{n}\R) \cong H^{k}(\mathds{S}^{n})_{+}\), questo è indotto da \(\bm{\pi}^{*}\):
\begin{align*}
\bm{\pi}^{*}: H^{k}(\mathds{P}^{n}\R) &\longrightarrow H^{k}(\mathds{S}^{n})_{+}\\
[\omega] &\longmapsto [\pi^{*}\omega]
\end{align*}
\begin{itemize}
\item La mappa è ben definita in quanto corestrizione di un pullback, e se \([\omega] \in H^{k}(\mathds{P}^{n}\R)\) allora
\begin{equation*}
   a^{*}\circ \pi^{*}\omega = \pi^{*}\omega
\end{equation*}
e dunque \(\bm{\pi}^{*}[\omega] = [\pi^{*}\omega] \in H^{k}(\mathds{P}^{n}\R)_{+}\).
\item La mappa è iniettiva: siano \([\omega], [\tau] \in H^{k}(\mathds{P}^{n}\R)\) tali che \(\bm{\pi}^{*}[\omega] = \bm{\pi}^{*}[\tau]\).
Allora
\begin{equation*}
   \pi^{*}(\omega-\tau) = \dif \eta.
\end{equation*}
Siccome \(\pi^{*}\) è un isomorfismo, esiste \(\nu\) tale che
\begin{equation*}
   \eta = \pi^{*} \nu
\end{equation*}
e in particolare:
\begin{equation*}
   \pi^{*}(\omega-\tau) = \dif \pi^{*} \nu = \pi^{*} \dif \nu
\end{equation*}
e dunque \(\omega-\tau = \dif \nu\) per l'iniettività di \(\pi^{*}\):
\begin{equation*}
   [\omega]= [\tau].
\end{equation*}

\item La mappa è suriettiva: se \([\eta] \in H^{k}(\mathds{S}^{n})_{+}\), allora WLOG\footnote{Infatti, se \([\eta] \in H^{k}(\mathds{S}^{n})_{+}\), allora
\begin{equation*}
[\eta] = \bm{a}^{*}[\eta] = [a^{*}\eta]%
\IMPLICA a^{*}\eta - \eta = \dif\beta.
\end{equation*}
Si definisce quindi \(\tilde{\eta} \coloneqq \frac{\eta + a^{*}\eta}{2}\).
\begin{itemize}
\item Dimostro che \([\tilde{\eta}] = [\eta]\):
\begin{equation*}
  \tilde{\eta} = \frac{\eta + a^{*}\eta}{2} = \frac{\eta+\eta+\dif\beta}{2} = \eta + \operatorname{d}(\beta/2).
\end{equation*}
\item Dimostro che \(a^{*}\tilde{\eta} = \eta\):
\begin{equation*}
        a^{*}\tilde{\eta} = \frac{a^{*}\eta + a^{*}a^{*}\eta}{2} = \frac{a^{*}\eta+\eta}{2} = \tilde{\eta}.
\end{equation*}
\end{itemize}
Dunque \([\tilde{\eta}]=[\eta]\) e \(a^{*}\tilde{\eta}=\tilde{\eta}\).} si ha
\begin{equation*}
   a^{*}\eta = \eta
\end{equation*}
e pertanto \(\eta \in \Omega^{k}(\mathds{S}^{n})_{+}\). Per il Claim 2., esiste \(\omega \in \Omega^{k}(\mathds{P}^{n} \R)\) tale che \(\pi^{*}\omega = \eta\), e inoltre
\begin{equation*}
   0 = \dif\eta = \dif \pi^{*}\omega = \pi^{*} \dif \omega
\end{equation*}
e siccome \(\pi^{*}\) isomorfismo per ogni \(k\), \(\dif\omega= 0\) e quindi \([\omega] \in H^{k}(\mathds{P}^{n}\R)\) e \(\bm{\pi}^{*}[\omega] = [\eta]\).
\end{itemize}
\end{enumerate}

Dunque, se \(k \neq n\), allora \(H^{k}(\mathds{S}^{n}) = 0\) (per la \href{20251115184248-coomologia_delle_sfere.org}{coomologia delle sfere}), ma
\begin{equation*}
H^{k}(\mathds{P}^{n}\R) \cong  H^{k}(\mathds{S}^{n})_{+} \subseteq H^{k}(\mathds{S}^{n}) = 0
\end{equation*}
e quindi \(H^{k}(\mathds{P}^{n}\R) = 0\).

Sia ora quindi \(k=n\). La situazione è la seguente\footnote{Vedi ``\href{20251115184248-coomologia_delle_sfere.org}{Coomologia delle sfere}''}:
\begin{equation*}
H^{n}(\mathds{P}^{n}\R) \cong H^{n}(\mathds{S}^{n})_{+} \subseteq H^{n}(\mathds{S}^{n}) = \langle [\nu] \rangle
\end{equation*}
dove \(\nu\) è la \href{20251115184544-forma_volume_su_una_varieta_differenziabile.org}{forma volume} su \(\mathds{S}^{n}\). È possibile calcolare che
\begin{equation*}
a^{*}\nu = (-1)^{n+1}\nu
\end{equation*}
e pertanto: \(\bm{a}^{*}[\nu] = (-1)^{n+1}[\nu]\).
\begin{itemize}
\item Se \(n\) è pari, allora \(n+1\) è dispari, e quindi \(\bm{a}^{*}[\nu] = - [\nu]\), e quindi \([\nu] \notin H^{n}(\mathds{S}^{n})_{+}\):
\begin{equation*}
  H^{n}(\mathds{S}^{n})_{+} = 0 %
  \IMPLICA %
  H^{n}(\mathds{P}^{n}\R) = 0.
\end{equation*}
\item Se \(n\) è dispari, allora \(n+1\) è pari, e quindi \(\bm{a}^{*}[\nu] = [\nu]\), e quindi \([\nu] \in H^{n}(\mathds{S}^{n})_{+}\):
\begin{equation*}
  H^{n}(\mathds{S}^{n})_{+} = \langle [\nu] \rangle \cong \R %
  \IMPLICA %
  H^{n}(\mathds{P}^{n}\R) \cong \R.%
  \qedhere
\end{equation*}
\end{itemize}
\end{proof}
\end{document}
