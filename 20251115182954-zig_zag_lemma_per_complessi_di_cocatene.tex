% Created 2026-02-07 Sat 19:31
% Intended LaTeX compiler: pdflatex
\documentclass[10pt]{article}
%% CREATO CON ORG - EMACS
\newcommand{\use}[2][]{\usepackage[#1]{#2}}
% PACCHETTI FONDAMENTLAI
\use[utf8]{inputenc}
\use[T1]{fontenc}
\use{graphicx}
\use{longtable}
\use{wrapfig}
\use{rotating}
\use[normalem]{ulem}
\use{amsmath}
\use{amsthm}
\use{amssymb}

\use{eucal} % Cambia mathcal{...}

\use{capt-of}
\use[italian]{babel}
\use[babel]{csquotes}
% bib la TEX lo carica in automatico org-cite
\use{microtype}
\use{lmodern}
\use{subfig} % sottofigure
\use{multicol} % due colonne
\use{lipsum} % lorem ipsum
\use{color} % colori in latex
\use{parskip} % rimuove l'indentazione dei nuovi paragrafi %% Add parbox=false to all new tcolorbox
\use{centernot}
\use[outline]{contour}\contourlength{3pt}
\use{fancyhdr}
\use{layout}
\use[most]{tcolorbox} % Riquadri colorati
\use{ifthen} % IFTHEN
\use{geometry}

% pacchetti matematica
\use{yhmath}
\use{dsfont}
\use{mathrsfs}
\use{cancel} % semplificare
\use{polynom} %divisione tra polinomi
\use{forest} % grafi ad albero
\use{booktabs} % tabelle
\use{commath} %simboli e differenziali
\use{bm} %bold
\use[fulladjust]{marginnote} %to use marginnote for date notes
\use{arrayjobx}%array
\use[intlimits]{empheq} % Riquadri colorati attorno alle equazioni
\use{mathtools}
\use{circuitikz} % Disegnare i circuiti
\use{mathtools}
\use{stmaryrd} % [[ \llbracket ]] \rrbracket
\use{bussproofs} % dimostrazioni

%%%%%%%%%%%%%


%%%% QUIVER
\newcommand{\duepunti}{\,\mathchar\numexpr"6000+`:\relax\,}
% A TikZ style for curved arrows of a fixed height, due to AndréC.
\tikzset{curve/.style={settings={#1},to path={(\tikztostart)
    .. controls ($(\tikztostart)!\pv{pos}!(\tikztotarget)!\pv{height}!270:(\tikztotarget)$)
    and ($(\tikztostart)!1-\pv{pos}!(\tikztotarget)!\pv{height}!270:(\tikztotarget)$)
    .. (\tikztotarget)\tikztonodes}},
    settings/.code={\tikzset{quiver/.cd,#1}
        \def\pv##1{\pgfkeysvalueof{/tikz/quiver/##1}}},
    quiver/.cd,pos/.initial=0.35,height/.initial=0}

% TikZ arrowhead/tail styles.
\tikzset{tail reversed/.code={\pgfsetarrowsstart{tikzcd to}}}
\tikzset{2tail/.code={\pgfsetarrowsstart{Implies[reversed]}}}
\tikzset{2tail reversed/.code={\pgfsetarrowsstart{Implies}}}
% TikZ arrow styles.
\tikzset{no body/.style={/tikz/dash pattern=on 0 off 1mm}}
%%%%%%%%%%


%% DEFINIZIONI COMANDI MATEMATICI
\let\sin\relax %TOGLIE LA DEFINIZIONE SU "\sin"

% cambia la definizione di empty set
% ---
\let\oldemptyset\emptyset
% ---
% \let\emptyset\varnothing
% ---
% \let\emptyset\relax
% \newcommand{\emptyset}{\text{\textnormal{\O}}}
% ---

\DeclareMathOperator{\bounded}{bd}
\DeclareMathOperator{\sin}{sen}
\DeclareMathOperator{\epi}{Epi}
\DeclareMathOperator{\cl}{cl}
\DeclareMathOperator{\graph}{graph}
\DeclareMathOperator{\arcsec}{arcsec}
\DeclareMathOperator{\arccot}{arccot}
\DeclareMathOperator{\arccsc}{arccsc}
\DeclareMathOperator{\spettro}{Spettro}
\DeclareMathOperator{\nulls}{nullspace}
\DeclareMathOperator{\dom}{dom}
\DeclareMathOperator{\ar}{ar}
\DeclareMathOperator{\const}{Const}
\DeclareMathOperator{\fun}{Fun}
\DeclareMathOperator{\rel}{Rel}
\DeclareMathOperator{\altezza}{ht}
\let\det\relax %TOGLIE LA DEFINIZIONE SU "\det"
\DeclareMathOperator{\det}{det}
\DeclareMathOperator{\End}{End}
\DeclareMathOperator{\gl}{GL}
\def\Id{\mathrm{Id}}
\def\id{\mathrm{id}}
\DeclareMathOperator{\I}{\mathds{1}}
\DeclareMathOperator{\II}{II}
\DeclareMathOperator{\rank}{rank}
\DeclareMathOperator{\tr}{tr}
\DeclareMathOperator{\tc}{t.c.}
\DeclareMathOperator{\T}{T}
\DeclareMathOperator{\var}{Var}
\DeclareMathOperator{\cov}{Cov}
\DeclareMathOperator{\st}{st}
\DeclareMathOperator{\mon}{Mon}
\newcommand{\card}[1]{\left\vert #1 \right\vert}
\newcommand{\trasposta}[1]{\prescript{\text{T}}{}{#1}}
\newcommand{\1}{\mathds{1}}
\newcommand{\R}{\mathds{R}}
\newcommand{\diesis}{\#}
\newcommand{\bemolle}{\flat}
\newcommand{\nonstandard}[1]{\prescript{*}{}{#1}}
\newcommand{\starR}{\nonstandard{\R}}
\newcommand{\borel}{\mathscr{B}}
\newcommand{\lebesgue}[1]{\mathscr{L}\left(#1\right)}
\newcommand{\media}{\mathds{E}}
\newcommand{\K}{\mathds{K}}
\newcommand{\A}{\mathds{A}}
\newcommand{\Q}{\mathds{Q}}
\newcommand{\N}{\mathds{N}}
\newcommand{\C}{\mathds{C}}
\newcommand{\Z}{\mathds{Z}}
\newcommand{\qo}{\hspace{1em}\text{q.o.}\,}
\renewcommand{\tilde}[1]{\widetilde{#1}}
\renewcommand{\parallel}{\mathrel{/\mkern-5mu/}}
\newcommand{\parti}[2][]{\wp_{#1}(#2)}
\newcommand{\diff}[1]{\operatorname{d}_{#1}}
\let\oldvec\vec
\renewcommand{\vec}[1]{\overrightarrow{\vphantom{i}#1}}
\newcommand{\floor}[1]{\left\lfloor #1 \right\rfloor}
\newcommand{\cat}[1]{\mathbf{#1}}
\newcommand{\dfreccia}[1]{\xrightarrow{\ #1 \ }}
\newcommand{\sfreccia}[1]{\xleftarrow{\ #1 \ }}
\newcommand{\formalsum}[2]{{\sum_{#1}^{#2}}{\vphantom{\sum}}'}
\newcommand{\minim}[2]{\mu_{#1}\, \left(#2\right)}
\newcommand{\concat}{\null^{\frown}} % concatenazione di stringe
\newcommand{\godelcode}[1]{\langle\!\langle #1 \rangle\!\rangle}
\newcommand{\godeldec}[1]{(\!(#1)\!)}
\newcommand{\termcode}[1]{\ulcorner #1\urcorner}
\newcommand{\partialto}{\dashrightarrow}
\newcommand{\restricted}{\upharpoonright}
\newcommand{\embeds}{\precsim}
\newcommand{\surjects}{\twoheadrightarrow}
\newcommand{\equipotenti}{\asymp}
%% \newcommand{\dotplus}{\mathbin{\dot{+}}} %% A quanto pare esiste già
\newcommand{\bigdot}{\mathbin{\boldsymbol{\cdot}}}
\newcommand{\dotexp}[1]{^{.#1}}
\newcommand{\conv}{\mathbin{*}}
\newcommand{\convolution}[2]{(#1\conv #2)}
\newcommand{\nil}{\mathfrak{N}}
\newcommand{\divisore}{\mathrel{|}}
\newcommand{\simplesso}[1]{\mathrm{e}_{#1}}

\renewcommand{\iff}{\mathrel{\longleftrightarrow}} %% Notazione Logica.
\newcommand{\oldiff}{\mathrel{\Longleftrightarrow}}
\renewcommand{\implies}{\mathrel{\rightarrow}} %% Notazione Logica
\newcommand{\oldimplies}{\mathrel{\Longrightarrow}}
\renewcommand{\impliedby}{\mathrel{\leftarrow}} %% Notazione Logica
\newcommand{\oldimpliedby}{\mathrel{\Longleftarrow}}

\newcommand{\IFF}{\quad\Longleftrightarrow\quad}
\newcommand{\IMPLICA}{\quad\Longrightarrow\quad}


\renewcommand{\descriptionlabel}[1]{\hspace{\labelsep}\normalfont #1} % remove bold from description


%% Definizione di Divergenza di K-L

\DeclarePairedDelimiterX{\infdivx}[2]{(}{)}{%
  #1\;\delimsize\|\;#2%
}
\newcommand{\kldiv}{D_{KL}\infdivx}

%% Definizione di \dotminus

\makeatletter
\newcommand{\dotminus}{\mathbin{\text{\@dotminus}}}

\newcommand{\@dotminus}{%
  \ooalign{\hidewidth\raise1ex\hbox{.}\hidewidth\cr$\m@th-$\cr}%
}
\makeatother

%tramite i prossimi due comandi posso decidere come scrivere i logaritmi naturali in tutti i documenti: ho infatti eliminato qualsiasi differenza tra "ln" e "log": se si vuole qualcosa di diverso bisogna inserire manualmente il tutto
\let\ln\relax
\DeclareMathOperator{\ln}{ln}
\let\log\relax
\DeclareMathOperator{\log}{log}
%%%%%%

%% NUOVI COMANDI
\newcommand{\straniero}[1]{\textit{#1}} %parole straniere
\newcommand{\titolo}[1]{\textsc{#1}} %titoli
\newcommand{\qedd}{\tag*{$\blacksquare$}} %qed per ambienti matemastici
\renewcommand{\qedsymbol}{$\blacksquare$} %modifica colore qed
\newcommand{\ooverline}[1]{\overline{\overline{#1}}}
\newcommand{\circoletto}[1]{\left(#1\right)^{\text{o}}}
%
\newcommand{\qmatrice}[1]{\begin{pmatrix}
#1_{11} & \cdots & #1_{1n}\\
\vdots & \ddots & \vdots \\
#1_{m1} & \cdots & #1_{mn}
\end{pmatrix}}
%
\newcommand{\parentesi}[2]{%
\underset{#1}{\underbrace{#2}}%
}
%
\newcommand{\norma}[1]{% Norma
\left\lVert#1\right\rVert%
}
\newcommand{\scalare}[2]{% Scalare
\left\langle #1, #2\right\rangle
}
%%%%%

%% RESTRIZIONI
\newcommand{\referenze}[2]{
        \phantomsection{}#2\textsuperscript{\textcolor{blue}{\textbf{#1}}}
}

\let\restriction\relax

\def\restriction#1#2{\mathchoice
              {\setbox1\hbox{${\displaystyle #1}_{\scriptstyle #2}$}
              \restrictionaux{#1}{#2}}
              {\setbox1\hbox{${\textstyle #1}_{\scriptstyle #2}$}
              \restrictionaux{#1}{#2}}
              {\setbox1\hbox{${\scriptstyle #1}_{\scriptscriptstyle #2}$}
              \restrictionaux{#1}{#2}}
              {\setbox1\hbox{${\scriptscriptstyle #1}_{\scriptscriptstyle #2}$}
              \restrictionaux{#1}{#2}}}
\def\restrictionaux#1#2{{#1\,\smash{\vrule height .8\ht1 depth .85\dp1}}_{\,#2}}
%%%%%%%%%%%

%%% FORMATTAZIONE FOOTNOTEMARK

\def\footnotemarkformatting#1{[#1]}
\renewcommand{\thefootnote}{\footnotemarkformatting{\arabic{footnote}}}

%% SEZIONE GRAFICA
\use{tikz}
\usetikzlibrary{matrix, patterns, calc, decorations.pathreplacing, hobby, decorations.markings, decorations.pathmorphing, babel}
\use{tikz-3dplot}
\use{mathrsfs} %per geogebra
\use{tikz-cd}
\tikzset
{
  %surface/.style={fill=black!10, shading=ball,fill opacity=0.4},
  plane/.style={black,pattern=north east lines},
  curve/.style={black,line width=0.5mm},
  dritto/.style={decoration={markings,mark=at position 0.5 with {\arrow{Stealth}}}, postaction=decorate},
  rovescio/.style={decoration={markings,mark=at position 0.5 with {\arrow{Stealth[reversed]}}}, postaction=decorate}
}
\use{pgfplots} % stampare le funzioni
        \pgfplotsset{/pgf/number format/use comma,compat=1.15}
        %\pgfplotsset{compat=1.15} %per geogebra
        \usepgfplotslibrary{fillbetween, polar}
%%%%%%

%% CITAZIONI
\use{lineno}

\newcommand{\citazione}[1]{%
  \begin{quotation}
  \begin{linenumbers}
  \modulolinenumbers[5]
  \begingroup
  \setlength{\parindent}{0cm}
  \noindent #1
  \endgroup
  \end{linenumbers}
  \end{quotation}\setcounter{linenumber}{1}
  }
%%%%%%

%%%%%%%%%%%%%%%%%%%%%%%%%%%%%%%%%%%%%%%%%%%%
%%%%%%%%%%%%%%%%%%%%%%%%%%%%%%%%%%%%%%%%%%%%

%% AMS THM

\theoremstyle{definition}% default
\newtheorem{thm}{Teorema}[section]
\newtheorem{lem}[thm]{Lemma}
\newtheorem{prop}[thm]{Proposizione}
\newtheorem{cor}[thm]{Corollario}
\newtheorem{esempio}[thm]{Esempio}
\theoremstyle{plain}
\newtheorem{definizione}[thm]{Definizione}
\theoremstyle{remark}
\newtheorem*{oss}{Osservazione}


%%%%%%%%%%%%%%%%%%%%%%%%%%%%%%%%%%%%%%%%%%%%
%%%%%%%%%%%%%%%%%%%%%%%%%%%%%%%%%%%%%%%%%%%%

\use{hyperref}
\hypersetup{%
        pdfauthor={Davide Peccioli},
        pdfsubject={},
        allcolors=black,
        citecolor=black,
%	colorlinks=true,
        bookmarksopen=true}
\setcounter{secnumdepth}{0} % rimuove i numeri di sezione senza rimuovere le ref
\renewcommand{\href}[2]{\textcolor{blue}{#2}} % disabilita il comando href
\use{enotez} %
\setenotez{%
 mark-format = \footnotemarkformatting % Mette i numeri tra parentesi quadre%
}\let\footnote=\endnote % rende tutte le note a pié pagina come delle note a fine file 


\let\olddocument\document % modifico l'ambiende documenti per non dover stampare \printendnote
\let\oldenddocument\enddocument
\renewenvironment{document}%
{%
  \olddocument
}{%
  \printendnotes\oldenddocument
}
\renewcommand{\thethm}{\arabic{thm}}

\usepackage[hyperref]{biblatex}
\addbibresource{~/Documents/org/roam/bib/master.bib}
\author{Davide Peccioli}
\date{\today}
\title{Zig-Zag Lemma (per complessi di cocatene)}
\begin{document}

\section{Zig-Zag Lemma (per complessi di cocatene)}
\label{sec:org218a427}
\begin{prop}
Sia
\begin{equation*}
\begin{tikzcd}
0 \arrow[r] & A^\bullet \arrow[r, "F^\bullet"] & B^\bullet \arrow[r, "G^\bullet"] & C^\bullet \arrow[r] & 0 \qquad (*)
\end{tikzcd}
\end{equation*}
una \href{20251115182707-sec_di_complessi_di_cocatene.org}{SEC} di \href{20251115182320-complesso_di_cocatene.org}{complessi di cocatene}. Allora esiste \(\delta\), detto ``morfismo di connessione'', tale che\footnote{Nota:
\begin{equation*}
H^k(C^\bullet) = \frac{\ker d_C^k}{\operatorname{Im} d_C^{k-1}}
\end{equation*}
Questa è la \href{20251115182537-coomologia_di_un_complesso_di_cocatene.org}{Coomologia di un complesso di cocatene}.}
\begin{equation*}
\begin{tikzcd}[row sep=large, column sep=large]
H^k(A^\bullet) \arrow[r, "{F^{*}}"] & H^k(B^\bullet) \arrow[r, "{G^{*}}"]
    \arrow[phantom, d, ""{coordinate, name=Z}]
    & H^k(C^\bullet) \arrow[dll, "\delta", out=-30, in=150, overlay] \\
H^{k+1}(A^\bullet) \arrow[r, "{F^{*}}"] & H^{k+1}(B^\bullet) \arrow[r, "{G^{*}}"] & H^{k+1}(C^\bullet)
\end{tikzcd}
\end{equation*}
è una \href{20251115182707-sec_di_complessi_di_cocatene.org}{successione esatta lunga}.
\end{prop}


\begin{proof}
La dimostrazione si svolge in due fasi:
\begin{enumerate}
\item Devo trovare \(\delta\) (buona def + applicazione lineare).
\item Esattezza della successione (solo \(\ker F^* = \operatorname{Im} \delta\)).
\end{enumerate}

\textbf{FASE 1.}

Se \(*\) è SEC, allora il seguente è diagramma commutativo a righe esatte.
\begin{equation*}
\begin{tikzcd}[sep=small]
	& \vdots && \vdots && \vdots \\
	0 & {A^{k+1}} && {B^{k+1}} && {C^{k+1}} & 0 \\
	&&&& {(1)} \\
	0 & {A^k} && {B^k} && {C^k} & 0 \\
	& \vdots && \vdots && \vdots
	\arrow[from=2-1, to=2-2]
	\arrow[from=2-2, to=1-2]
	\arrow["{F_{k+1}}", from=2-2, to=2-4]
	\arrow[from=2-4, to=1-4]
	\arrow["{G_{k+1}}", from=2-4, to=2-6]
	\arrow[from=2-6, to=1-6]
	\arrow[from=2-6, to=2-7]
	\arrow[from=4-1, to=4-2]
	\arrow["{d_A^k}", from=4-2, to=2-2]
	\arrow["{F_k}", from=4-2, to=4-4]
	\arrow["{d_B^k}"', from=4-4, to=2-4]
	\arrow["{G_k}", from=4-4, to=4-6]
	\arrow["{d_C^k}", from=4-6, to=2-6]
	\arrow["{G_k}", from=4-6, to=4-7]
	\arrow[from=5-2, to=4-2]
	\arrow[from=5-4, to=4-4]
	\arrow[from=5-6, to=4-6]
\end{tikzcd}
\end{equation*}

Sia quindi \([c] \in H^k(C^\bullet)\) (i.e. \(c \in C^k\) t.c. \(dc = 0\)).

Poiché la seconda riga è esatta allora \(G^k\) è \href{20241213105600-funzione_suriettiva.org}{suriettiva}, e pertanto esiste \(b \in B^k\) t.c.
\begin{equation*}
G b = c.
\end{equation*}

Considero ora \(\operatorname{d}_B^k(b)\) (abbreviato \(\dif b\)).
\begin{equation*}
G \dif b = \dif G b =\dif c = 0
\end{equation*}
\emph{(per commutatività di (1))}

Segue quindi che
\begin{equation*}
\dif b \in \ker G^{k+1} = \operatorname{Im} F^{k+1}
\end{equation*}
dove l'ultima uguaglianza sussiste per esattezza della prima riga.
Pertanto esiste \(a \in A^{k+1}\) t.c.
\begin{equation*}
F a = b' \coloneqq \dif b
\end{equation*}

Questo \(a\) è unico poiché \(F\) iniettiva (per esattezza della I\textsuperscript{a} riga).
Inoltre
\begin{equation*}
F \dif a = \dif Fa = \dif\dif b = 0 \IMPLICA \dif a = 0.
\end{equation*}

Definisco quindi
\begin{equation*}
\boxed{ \delta([c]) := [a] }
\end{equation*}

Abbiamo fatto:
\begin{equation*}
\begin{tikzcd}
	a & {\dif b} \\
	& b & c
	\arrow["F", maps to, from=1-1, to=1-2]
	\arrow["{\operatorname{d}}", maps to, from=2-2, to=1-2]
	\arrow["G", maps to, from=2-2, to=2-3]
\end{tikzcd}
\end{equation*}

Resta da dimostrare:
\begin{enumerate}
\item Che l'unica scelta fatta (ovvero \(b\)) non influisce sulla definizione.
\item Che \(\delta\) è ben definita (rispetto al quoziente).
\item Che \(\delta\) è lineare.
\end{enumerate}

\textbf{1.} Siano \(b, b' \in B^k\) tali che
\begin{equation*}
G^k(b) = G^k(b') = c.
\end{equation*}
e sia \(a, a' \in A^{k+1}\) t.c. \(F a = \dif b\), \(F a' = \dif b'\). Allora
\begin{equation*}
0 = G(b) - G(b') = G(b - b')
\end{equation*}
e pertanto (per esattezza della seconda riga) esiste \(\tilde{a} \in A^k\) t.c.
\begin{align*}
b - b' &= F \tilde{a}\\
\operatorname{d}(b - b') &= \dif F \tilde{a}\\
\dif b - \dif b' &= F \dif \tilde{a}
\end{align*}
Pertanto, ricordando che \(Fa = \dif b\):
\begin{align*}
Fa %
	&= \dif b = \dif b' + F \dif \tilde{a}\\
	&= F a' + F\dif \tilde{a}\\
	&= F (a' +\dif \tilde{a})
\end{align*}
e, poiché \(F\) è iniettiva, segue che:
\begin{equation*}
a = a' + \dif \tilde{a} \IMPLICA [a] = [a'].
\end{equation*}

\textbf{2.} Sia \(c \in \operatorname{Im} d^{k-1}\). Dimostriamo \(a \in \operatorname{Im} d^k\), dove la nomenclatura è la solita:
\begin{equation*}
\begin{tikzcd}
	a & {\dif b} \\
	& b & c
	\arrow["F", maps to, from=1-1, to=1-2]
	\arrow["{\operatorname{d}}", maps to, from=2-2, to=1-2]
	\arrow["G", maps to, from=2-2, to=2-3]
\end{tikzcd}
\end{equation*}

Sia quindi \(c = \dif \gamma\). Per esattezza \(G\) suriettiva, e quindi \(\gamma = G \beta\).
\begin{equation*}
c = \dif G \beta = G (\dif \beta).
\end{equation*}

Per il punto precedente, la scelta di \(b\) è ininfluente, pertanto pongo
\begin{equation*}
b := \dif \beta, \quad \text{e } G(\dif \dif\beta) = c
\end{equation*}
La situazione quindi è la seguente:
\begin{equation*}
\begin{tikzcd}
	a & {\dif b} \\
	& b & c \\
	& \beta & \gamma
	\arrow["F", maps to, from=1-1, to=1-2]
	\arrow["{\operatorname{d}}", maps to, from=2-2, to=1-2]
	\arrow["G", maps to, from=2-2, to=2-3]
	\arrow["{\operatorname{d}}", maps to, from=3-2, to=2-2]
	\arrow["G"', maps to, from=3-2, to=3-3]
	\arrow["{\operatorname{d}}"', maps to, from=3-3, to=2-3]
\end{tikzcd}
\end{equation*}

Segue che \(F(a) = \dif b = \dif\dif \beta = 0\) e dunque, per iniettività, \(a=0\).
Siccome \(0 \in \operatorname{Im} d^k\), la tesi.

\textbf{3.} \(\delta\) è lineare poiché ?

\textbf{FASE 2.}

Resta da dimostrare l'esattezza. Noi dimostriamo soltanto
\begin{equation*}
\ker F^* = \operatorname{Im} \delta.
\end{equation*}

(\(\supseteq\)): Sia \([a] = \delta^*[c] \in \operatorname{Im} \delta^*\):
\begin{equation*}
\begin{tikzcd}
	a & {\dif b} \\
	& b & c
	\arrow["F", maps to, from=1-1, to=1-2]
	\arrow["{\operatorname{d}}", maps to, from=2-2, to=1-2]
	\arrow["G", maps to, from=2-2, to=2-3]
\end{tikzcd}
\end{equation*}

Quindi si ha che
\begin{equation*}
Fa = \dif b \IMPLICA [Fa] = 0
\end{equation*}
Ma \(0 = [Fa] \eqqcolon F^*[a]\) e quindi \([a] \in \ker F^*\).

(\(\subseteq\)): Sia \([a]\) t.c. \(F^*[a] = 0\), \([a] \in \ker F^{*}\).
i.e. \(Fa = \dif b\).

Detto quindi \(c = G b\)
\begin{equation*}
\dif c = \dif G b = G \dif b = G F a = 0
\end{equation*}
\emph{(righe esatte: \(GF = 0\))}

Pertanto, è possibile utilizzare la costruzione di prima:
\begin{equation*}
\begin{tikzcd}
	a & {\dif b} \\
	& b & c
	\arrow["F", maps to, from=1-1, to=1-2]
	\arrow["{\operatorname{d}}", maps to, from=2-2, to=1-2]
	\arrow["G", maps to, from=2-2, to=2-3]
\end{tikzcd}
\end{equation*}
ottenendo \(\delta[c]=[a]\). Segue che \([a] \in\operatorname{Im}\delta\).
\end{proof}
\end{document}
