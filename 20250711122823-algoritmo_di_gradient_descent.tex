% Created 2026-02-07 Sat 19:30
% Intended LaTeX compiler: pdflatex
\documentclass[10pt]{article}
%% CREATO CON ORG - EMACS
\newcommand{\use}[2][]{\usepackage[#1]{#2}}
% PACCHETTI FONDAMENTLAI
\use[utf8]{inputenc}
\use[T1]{fontenc}
\use{graphicx}
\use{longtable}
\use{wrapfig}
\use{rotating}
\use[normalem]{ulem}
\use{amsmath}
\use{amsthm}
\use{amssymb}

\use{eucal} % Cambia mathcal{...}

\use{capt-of}
\use[italian]{babel}
\use[babel]{csquotes}
% bib la TEX lo carica in automatico org-cite
\use{microtype}
\use{lmodern}
\use{subfig} % sottofigure
\use{multicol} % due colonne
\use{lipsum} % lorem ipsum
\use{color} % colori in latex
\use{parskip} % rimuove l'indentazione dei nuovi paragrafi %% Add parbox=false to all new tcolorbox
\use{centernot}
\use[outline]{contour}\contourlength{3pt}
\use{fancyhdr}
\use{layout}
\use[most]{tcolorbox} % Riquadri colorati
\use{ifthen} % IFTHEN
\use{geometry}

% pacchetti matematica
\use{yhmath}
\use{dsfont}
\use{mathrsfs}
\use{cancel} % semplificare
\use{polynom} %divisione tra polinomi
\use{forest} % grafi ad albero
\use{booktabs} % tabelle
\use{commath} %simboli e differenziali
\use{bm} %bold
\use[fulladjust]{marginnote} %to use marginnote for date notes
\use{arrayjobx}%array
\use[intlimits]{empheq} % Riquadri colorati attorno alle equazioni
\use{mathtools}
\use{circuitikz} % Disegnare i circuiti
\use{mathtools}
\use{stmaryrd} % [[ \llbracket ]] \rrbracket
\use{bussproofs} % dimostrazioni

%%%%%%%%%%%%%


%%%% QUIVER
\newcommand{\duepunti}{\,\mathchar\numexpr"6000+`:\relax\,}
% A TikZ style for curved arrows of a fixed height, due to AndréC.
\tikzset{curve/.style={settings={#1},to path={(\tikztostart)
    .. controls ($(\tikztostart)!\pv{pos}!(\tikztotarget)!\pv{height}!270:(\tikztotarget)$)
    and ($(\tikztostart)!1-\pv{pos}!(\tikztotarget)!\pv{height}!270:(\tikztotarget)$)
    .. (\tikztotarget)\tikztonodes}},
    settings/.code={\tikzset{quiver/.cd,#1}
        \def\pv##1{\pgfkeysvalueof{/tikz/quiver/##1}}},
    quiver/.cd,pos/.initial=0.35,height/.initial=0}

% TikZ arrowhead/tail styles.
\tikzset{tail reversed/.code={\pgfsetarrowsstart{tikzcd to}}}
\tikzset{2tail/.code={\pgfsetarrowsstart{Implies[reversed]}}}
\tikzset{2tail reversed/.code={\pgfsetarrowsstart{Implies}}}
% TikZ arrow styles.
\tikzset{no body/.style={/tikz/dash pattern=on 0 off 1mm}}
%%%%%%%%%%


%% DEFINIZIONI COMANDI MATEMATICI
\let\sin\relax %TOGLIE LA DEFINIZIONE SU "\sin"

% cambia la definizione di empty set
% ---
\let\oldemptyset\emptyset
% ---
% \let\emptyset\varnothing
% ---
% \let\emptyset\relax
% \newcommand{\emptyset}{\text{\textnormal{\O}}}
% ---

\DeclareMathOperator{\bounded}{bd}
\DeclareMathOperator{\sin}{sen}
\DeclareMathOperator{\epi}{Epi}
\DeclareMathOperator{\cl}{cl}
\DeclareMathOperator{\graph}{graph}
\DeclareMathOperator{\arcsec}{arcsec}
\DeclareMathOperator{\arccot}{arccot}
\DeclareMathOperator{\arccsc}{arccsc}
\DeclareMathOperator{\spettro}{Spettro}
\DeclareMathOperator{\nulls}{nullspace}
\DeclareMathOperator{\dom}{dom}
\DeclareMathOperator{\ar}{ar}
\DeclareMathOperator{\const}{Const}
\DeclareMathOperator{\fun}{Fun}
\DeclareMathOperator{\rel}{Rel}
\DeclareMathOperator{\altezza}{ht}
\let\det\relax %TOGLIE LA DEFINIZIONE SU "\det"
\DeclareMathOperator{\det}{det}
\DeclareMathOperator{\End}{End}
\DeclareMathOperator{\gl}{GL}
\def\Id{\mathrm{Id}}
\def\id{\mathrm{id}}
\DeclareMathOperator{\I}{\mathds{1}}
\DeclareMathOperator{\II}{II}
\DeclareMathOperator{\rank}{rank}
\DeclareMathOperator{\tr}{tr}
\DeclareMathOperator{\tc}{t.c.}
\DeclareMathOperator{\T}{T}
\DeclareMathOperator{\var}{Var}
\DeclareMathOperator{\cov}{Cov}
\DeclareMathOperator{\st}{st}
\DeclareMathOperator{\mon}{Mon}
\newcommand{\card}[1]{\left\vert #1 \right\vert}
\newcommand{\trasposta}[1]{\prescript{\text{T}}{}{#1}}
\newcommand{\1}{\mathds{1}}
\newcommand{\R}{\mathds{R}}
\newcommand{\diesis}{\#}
\newcommand{\bemolle}{\flat}
\newcommand{\nonstandard}[1]{\prescript{*}{}{#1}}
\newcommand{\starR}{\nonstandard{\R}}
\newcommand{\borel}{\mathscr{B}}
\newcommand{\lebesgue}[1]{\mathscr{L}\left(#1\right)}
\newcommand{\media}{\mathds{E}}
\newcommand{\K}{\mathds{K}}
\newcommand{\A}{\mathds{A}}
\newcommand{\Q}{\mathds{Q}}
\newcommand{\N}{\mathds{N}}
\newcommand{\C}{\mathds{C}}
\newcommand{\Z}{\mathds{Z}}
\newcommand{\qo}{\hspace{1em}\text{q.o.}\,}
\renewcommand{\tilde}[1]{\widetilde{#1}}
\renewcommand{\parallel}{\mathrel{/\mkern-5mu/}}
\newcommand{\parti}[2][]{\wp_{#1}(#2)}
\newcommand{\diff}[1]{\operatorname{d}_{#1}}
\let\oldvec\vec
\renewcommand{\vec}[1]{\overrightarrow{\vphantom{i}#1}}
\newcommand{\floor}[1]{\left\lfloor #1 \right\rfloor}
\newcommand{\cat}[1]{\mathbf{#1}}
\newcommand{\dfreccia}[1]{\xrightarrow{\ #1 \ }}
\newcommand{\sfreccia}[1]{\xleftarrow{\ #1 \ }}
\newcommand{\formalsum}[2]{{\sum_{#1}^{#2}}{\vphantom{\sum}}'}
\newcommand{\minim}[2]{\mu_{#1}\, \left(#2\right)}
\newcommand{\concat}{\null^{\frown}} % concatenazione di stringe
\newcommand{\godelcode}[1]{\langle\!\langle #1 \rangle\!\rangle}
\newcommand{\godeldec}[1]{(\!(#1)\!)}
\newcommand{\termcode}[1]{\ulcorner #1\urcorner}
\newcommand{\partialto}{\dashrightarrow}
\newcommand{\restricted}{\upharpoonright}
\newcommand{\embeds}{\precsim}
\newcommand{\surjects}{\twoheadrightarrow}
\newcommand{\equipotenti}{\asymp}
%% \newcommand{\dotplus}{\mathbin{\dot{+}}} %% A quanto pare esiste già
\newcommand{\bigdot}{\mathbin{\boldsymbol{\cdot}}}
\newcommand{\dotexp}[1]{^{.#1}}
\newcommand{\conv}{\mathbin{*}}
\newcommand{\convolution}[2]{(#1\conv #2)}
\newcommand{\nil}{\mathfrak{N}}
\newcommand{\divisore}{\mathrel{|}}
\newcommand{\simplesso}[1]{\mathrm{e}_{#1}}

\renewcommand{\iff}{\mathrel{\longleftrightarrow}} %% Notazione Logica.
\newcommand{\oldiff}{\mathrel{\Longleftrightarrow}}
\renewcommand{\implies}{\mathrel{\rightarrow}} %% Notazione Logica
\newcommand{\oldimplies}{\mathrel{\Longrightarrow}}
\renewcommand{\impliedby}{\mathrel{\leftarrow}} %% Notazione Logica
\newcommand{\oldimpliedby}{\mathrel{\Longleftarrow}}

\newcommand{\IFF}{\quad\Longleftrightarrow\quad}
\newcommand{\IMPLICA}{\quad\Longrightarrow\quad}


\renewcommand{\descriptionlabel}[1]{\hspace{\labelsep}\normalfont #1} % remove bold from description


%% Definizione di Divergenza di K-L

\DeclarePairedDelimiterX{\infdivx}[2]{(}{)}{%
  #1\;\delimsize\|\;#2%
}
\newcommand{\kldiv}{D_{KL}\infdivx}

%% Definizione di \dotminus

\makeatletter
\newcommand{\dotminus}{\mathbin{\text{\@dotminus}}}

\newcommand{\@dotminus}{%
  \ooalign{\hidewidth\raise1ex\hbox{.}\hidewidth\cr$\m@th-$\cr}%
}
\makeatother

%tramite i prossimi due comandi posso decidere come scrivere i logaritmi naturali in tutti i documenti: ho infatti eliminato qualsiasi differenza tra "ln" e "log": se si vuole qualcosa di diverso bisogna inserire manualmente il tutto
\let\ln\relax
\DeclareMathOperator{\ln}{ln}
\let\log\relax
\DeclareMathOperator{\log}{log}
%%%%%%

%% NUOVI COMANDI
\newcommand{\straniero}[1]{\textit{#1}} %parole straniere
\newcommand{\titolo}[1]{\textsc{#1}} %titoli
\newcommand{\qedd}{\tag*{$\blacksquare$}} %qed per ambienti matemastici
\renewcommand{\qedsymbol}{$\blacksquare$} %modifica colore qed
\newcommand{\ooverline}[1]{\overline{\overline{#1}}}
\newcommand{\circoletto}[1]{\left(#1\right)^{\text{o}}}
%
\newcommand{\qmatrice}[1]{\begin{pmatrix}
#1_{11} & \cdots & #1_{1n}\\
\vdots & \ddots & \vdots \\
#1_{m1} & \cdots & #1_{mn}
\end{pmatrix}}
%
\newcommand{\parentesi}[2]{%
\underset{#1}{\underbrace{#2}}%
}
%
\newcommand{\norma}[1]{% Norma
\left\lVert#1\right\rVert%
}
\newcommand{\scalare}[2]{% Scalare
\left\langle #1, #2\right\rangle
}
%%%%%

%% RESTRIZIONI
\newcommand{\referenze}[2]{
        \phantomsection{}#2\textsuperscript{\textcolor{blue}{\textbf{#1}}}
}

\let\restriction\relax

\def\restriction#1#2{\mathchoice
              {\setbox1\hbox{${\displaystyle #1}_{\scriptstyle #2}$}
              \restrictionaux{#1}{#2}}
              {\setbox1\hbox{${\textstyle #1}_{\scriptstyle #2}$}
              \restrictionaux{#1}{#2}}
              {\setbox1\hbox{${\scriptstyle #1}_{\scriptscriptstyle #2}$}
              \restrictionaux{#1}{#2}}
              {\setbox1\hbox{${\scriptscriptstyle #1}_{\scriptscriptstyle #2}$}
              \restrictionaux{#1}{#2}}}
\def\restrictionaux#1#2{{#1\,\smash{\vrule height .8\ht1 depth .85\dp1}}_{\,#2}}
%%%%%%%%%%%

%%% FORMATTAZIONE FOOTNOTEMARK

\def\footnotemarkformatting#1{[#1]}
\renewcommand{\thefootnote}{\footnotemarkformatting{\arabic{footnote}}}

%% SEZIONE GRAFICA
\use{tikz}
\usetikzlibrary{matrix, patterns, calc, decorations.pathreplacing, hobby, decorations.markings, decorations.pathmorphing, babel}
\use{tikz-3dplot}
\use{mathrsfs} %per geogebra
\use{tikz-cd}
\tikzset
{
  %surface/.style={fill=black!10, shading=ball,fill opacity=0.4},
  plane/.style={black,pattern=north east lines},
  curve/.style={black,line width=0.5mm},
  dritto/.style={decoration={markings,mark=at position 0.5 with {\arrow{Stealth}}}, postaction=decorate},
  rovescio/.style={decoration={markings,mark=at position 0.5 with {\arrow{Stealth[reversed]}}}, postaction=decorate}
}
\use{pgfplots} % stampare le funzioni
        \pgfplotsset{/pgf/number format/use comma,compat=1.15}
        %\pgfplotsset{compat=1.15} %per geogebra
        \usepgfplotslibrary{fillbetween, polar}
%%%%%%

%% CITAZIONI
\use{lineno}

\newcommand{\citazione}[1]{%
  \begin{quotation}
  \begin{linenumbers}
  \modulolinenumbers[5]
  \begingroup
  \setlength{\parindent}{0cm}
  \noindent #1
  \endgroup
  \end{linenumbers}
  \end{quotation}\setcounter{linenumber}{1}
  }
%%%%%%

%%%%%%%%%%%%%%%%%%%%%%%%%%%%%%%%%%%%%%%%%%%%
%%%%%%%%%%%%%%%%%%%%%%%%%%%%%%%%%%%%%%%%%%%%

%% AMS THM

\theoremstyle{definition}% default
\newtheorem{thm}{Teorema}[section]
\newtheorem{lem}[thm]{Lemma}
\newtheorem{prop}[thm]{Proposizione}
\newtheorem{cor}[thm]{Corollario}
\newtheorem{esempio}[thm]{Esempio}
\theoremstyle{plain}
\newtheorem{definizione}[thm]{Definizione}
\theoremstyle{remark}
\newtheorem*{oss}{Osservazione}


%%%%%%%%%%%%%%%%%%%%%%%%%%%%%%%%%%%%%%%%%%%%
%%%%%%%%%%%%%%%%%%%%%%%%%%%%%%%%%%%%%%%%%%%%

\use{hyperref}
\hypersetup{%
        pdfauthor={Davide Peccioli},
        pdfsubject={},
        allcolors=black,
        citecolor=black,
%	colorlinks=true,
        bookmarksopen=true}
\setcounter{secnumdepth}{0} % rimuove i numeri di sezione senza rimuovere le ref
\renewcommand{\href}[2]{\textcolor{blue}{#2}} % disabilita il comando href
\use{enotez} %
\setenotez{%
 mark-format = \footnotemarkformatting % Mette i numeri tra parentesi quadre%
}\let\footnote=\endnote % rende tutte le note a pié pagina come delle note a fine file 


\let\olddocument\document % modifico l'ambiende documenti per non dover stampare \printendnote
\let\oldenddocument\enddocument
\renewenvironment{document}%
{%
  \olddocument
}{%
  \printendnotes\oldenddocument
}
\renewcommand{\thethm}{\arabic{thm}}

\usepackage[hyperref]{biblatex}
\addbibresource{~/Documents/org/roam/bib/master.bib}
\author{Davide Peccioli}
\date{\today}
\title{}
\begin{document}

\section{Algoritmo di Gradient Descent}
\label{sec:org29ff716}
L'algoritmo di \uline{Gradient Descent} è un algoritmo per trovare iterativamente il \href{20250627153543-massimo_e_minimo_di_una_funzione_reale.org}{minimo} \(x^{*}\) di una funzione reale \(f: D \subseteq \R^{n}\to \R\).

Per il \href{20250627130923-esistenza_di_una_curva_perpendicolare_a_tutte_le_curve_di_livello.org}{teorema}\footnote{AGGIUNGERE RIFERIMENTO PUNTUALE}, per ogni punto di partenza \(x^{(0)}\) sufficientemente vicino ad \(x^{*}\) esiste una curva \(\gamma\) che collega \(x^{(0)}\) a \(x^{*}\), perpendicolare in ogni punto alle curve di livello (\href{20250627130736-gradiente_e_perpendicolare_alle_curve_di_livello.org}{ovvero} parallela al \href{20250624171244-gradiente_di_una_funzione.org}{gradiente della funzione}).

Si vuole approssimare \(\gamma\) con una poligonale \([x^{(0)},x^{(1)},\dots,x^{(m)}]\) tale che:
\begin{enumerate}
\item detti \(c_{k}\coloneqq f(x^{(k)})\), si ha che \(c_{k+1}<c_{k}\)
\item il segmento \([x^{(j)},x^{(j+1)}]\) sia perpendicolare a \(\mathcal{S}_{c_{j}}\)\footnote{Con \(\mathcal{S}_{c}\) si indica la \href{20250627131207-curva_di_livello.org}{curva di livello}.}.
\end{enumerate}
\subsection{Gradient Descent con passo fissato}
\label{sec:org706fd10}

Il primo metodo per farlo è utilizzando il metodo della \uline{discesa più ripida}. Partendo da \(x^{(0)}\), si cerca il minimo della funzione muovendosi di un passo \(\eta\) nella direzione \(v\) in cui la funzione decresce più rapidamente.

Dunque, fissato \(\eta\), si cerca \(v \in \R^{n}\) tale che \(\norma{v} = 1\) e tale per cui
\begin{equation*}
f(x^{(i)}+\eta v) - f(x^{(i)})
\end{equation*}
ha il valore negativo più grande.

Utilizzando le \href{20250717132708-serie_di_taylor.org}{approssimazioni di Taylor}, ipotizzando che i termini quadrtici siano trascurabili (stiamo effettivamente ignorando la curvatura della superficie, data dall'hessiana), si ottiene che
\begin{equation*}
f(x^{(i)}+\eta v) - f(x^{(i)}) \approx \eta\,\langle \nabla f(x^{(i)}),v\rangle
\end{equation*}
Per la \href{20250629112810-disuguaglianza_di_cauchy_schwarz.org}{disuguaglianza di Cauchy-Schwartz} si ottiene che
\begin{equation*}
|\langle \nabla f(x^{(i)}),v\rangle|^{2} \le \norma{\nabla f(x^{(i)}}\, \norma{v} = \norma{\nabla f(x^{(i)})}
\end{equation*}
ovvero
\begin{equation*}
-{\norma{\nabla f(x^{(i)})}}\le \langle \nabla f(x^{(i)}),v\rangle\le {\norma{\nabla f(x^{(i)})}}.
\end{equation*}
Inoltre per \(v=-\frac{\nabla f(x^{(i)})}{\norma{\nabla f(x^{(i)})}}\) si ottiene
\begin{align*}
\langle \nabla f(x^{(i)}), v\rangle &= - \frac{1}{\norma{\nabla f(x^{(i)})}} \langle \nabla f(x^{(i)}),\nabla f(x^{(i)})\rangle=\\
&= - \frac{1}{\norma{\nabla f(x^{(i)})}} \norma{\nabla f(x^{(i)})}^{2} =\\
&= -\norma{\nabla f(x^{(i)})}
\end{align*}
e pertanto \(-\frac{\nabla f(x^{(i)})}{\norma{\nabla f(x^{(i)})}}\) è la direzione di massima decrescita, e la sopracitata decrescita è
\begin{align*}
f(x^{(i)}+\eta v) - f(x^{(i)}) \approx \eta\,\langle \nabla f(x^{(i)}),v\rangle = -\eta\norma{\nabla f(x^{(i)})}
\end{align*}

Questo dà luogo ad una \href{20250115100904-successione.org}{successione} \(\langle x^{(n)}\rangle_{n \in \N}\):
\begin{equation}
x^{(n+1)} \coloneqq x^{(n)}-\eta \frac{\nabla f(x^{(n)})}{\norma{\nabla f(x^{(n)})}}.\label{eq:succ:grdscfisso}
\end{equation}
Si noti che:
\begin{enumerate}
\item \(f(x^{(n+1)}) - f(x^{(n)})=-\eta\norma{\nabla f(x^{(n)})}<0\) e quindi \(f(x^{(n+1)}) < f(x^{(n)})\)
\item \([x^{(n)},x^{(n+1)}]\) è parallelo a \(\nabla f(x^{(n)})\), e pertanto perpendicolare alla curva di livello di \(f\) passante per \(x^{(n)}\)
\end{enumerate}
e dunque la successione così costruita dà luogo ad una approssimazione di \(\gamma\) con una poligonale, come richiesto all'inizio.

Al passo \(m\)-esimo:
\begin{equation*}
\norma{x^{*}-x^{(0)}}-m\eta \le \norma{x^{*}-x^{(m)}}\le \operatorname{diam}(\mathcal{S}_{f(x^{m})})
\end{equation*}
dove \(\operatorname{diam}\) è il \href{20250327131547-diametro_di_un_insieme.org}{diametro} dell'insieme.

Questo metodo ha un problema: siccome \(\norma{x^{(n+1)}-x^{(n)}}=\eta\) fissato, la successione \(\langle x^{(n)}\rangle\) non è di \href{20250303134529-successione_di_cauchy.org}{Cauchy}, \href{20250301194153-spazio_metrico_completo.org}{e dunque} \uline{non \href{20250115100930-convergenza_per_una_successione.org}{converge}}.
\subsection{Gradient Descent a passo variabile}
\label{sec:org28fa5a9}

Per ovviare al problema di cui sopra, si sostituisce la~\eqref{eq:succ:grdscfisso} con
\begin{equation}
x^{(n+1)} \coloneqq x^{(n)}-\eta \nabla f(x^{(n)}).\label{eq:succ:grdsc}
\end{equation}
Questo algoritmo è quello che si intende comunemente con \uline{gradient descent method}.
\begin{prop}
La successione \(\langle x^{(n)}\rangle_{n \in \N}\) definita in~\eqref{eq:succ:grdsc} \href{20250115100930-convergenza_per_una_successione.org}{converge} se e solo se \(\nabla f(x^{(n)})\to 0\) per \(n\to \infty\).
\end{prop}
\subsection{Line Search Method}
\label{sec:orgdaf57b0}
Una variante del metodo di cui sopra è dato dallo scegliere una successione \(\langle x^{(n)}\rangle_{n \in \N}\):
\begin{equation}
x^{(n+1)} \coloneqq x^{(n)}-\eta_{n} \nabla f(x^{(n)})\label{eq:succ:linsrc}
\end{equation}
dove \(\eta_{n}\) è scelto come
\begin{equation*}
\eta_{n} =\arg \min_{\eta \in \R} f\left(x^{(n)}-\eta\nabla f(x^{(n)})\right).
\end{equation*}

L'idea è la seguente: partendo dal punto \(x^{(n)}\), si procede lungo la linea retta con direzione e verso \(-\nabla f(x^{(n)})\), e si sceglie il punto \(x^{(n+1)}\) lungo questa retta che rende minima la quantità \(f(x^(n+1))\). Geometricamente, questo significa scegliere il punto in cui la retta \(x^{(n)}-t\nabla f(x^{(n)})\) interseca tangenzialmente una curva di livello.

Questo algoritmo converge molto più velocemente del gradient descent, e la poligonale \([x^{(0)},x^{(1)}, \dots]\) contiene solamente angoli retti.
\subsection{Algoritmo di Stochastic Gradient Descent}
\label{sec:org9892514}
Una generalizzazione del \hyperref[sec:org29ff716]{GD} applicato a \href{20250624155858-neurone_artificiale.org}{funzioni costo} è l'algoritmo di \uline{Stochastic Gradient Descent}. Questo prevede di suddividere il \href{20250627110009-training_error_and_test_error.org}{training set} \(\mathscr{T}\) in \(k\) parti disgiunte:
\begin{equation*}
\mathrm{Tr}_{1},\dots,\mathrm{Tr}_{k}
\end{equation*}
e di eseguire sulla funzione costo per i punti di \(\mathrm{Tr}_{i}\) l'algoritmo di GD. I parametri ottenuti saranno i parametri iniziali per svolgere il GR sulla funzione costo per i punti di \(\mathrm{Tr}_{i+1}\), con la convenzione che \(\mathrm{Tr}_{k+1}\coloneqq\mathrm{Tr}_{1}\). L'algoritmo è illustrato in figura~\ref{fig:alg:sgd}

Ogni ciclo di GD su tutti i \(k\) training set è detto un'\uline{epoca} di apprendimento.

\begin{figure}
\begin{equation*}
\begin{tikzcd}[ampersand replacement=\&,cramped]
	{\mathrm{Tr}_1} \& {\mathrm{Tr}_2} \& {\mathrm{Tr}_3} \& \dots \& {\mathrm{Tr}_k}
	\arrow["{\text{GD}}", from=1-1, to=1-2]
	\arrow["{\text{GD}}", from=1-2, to=1-3]
	\arrow["{\text{GD}}", from=1-3, to=1-4]
	\arrow["{\text{GD}}", from=1-4, to=1-5]
	\arrow["{\text{epoca}}", bend left=20pt, from=1-5, to=1-1]
\end{tikzcd}
\end{equation*}
\caption{\label{fig:alg:sgd}L'algoritmo di Stochastic Gradient Descent}
\end{figure}
\end{document}
