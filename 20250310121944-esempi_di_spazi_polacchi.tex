% Created 2026-02-07 Sat 19:34
% Intended LaTeX compiler: pdflatex
\documentclass[10pt]{article}
%% CREATO CON ORG - EMACS
\newcommand{\use}[2][]{\usepackage[#1]{#2}}
% PACCHETTI FONDAMENTLAI
\use[utf8]{inputenc}
\use[T1]{fontenc}
\use{graphicx}
\use{longtable}
\use{wrapfig}
\use{rotating}
\use[normalem]{ulem}
\use{amsmath}
\use{amsthm}
\use{amssymb}

\use{eucal} % Cambia mathcal{...}

\use{capt-of}
\use[italian]{babel}
\use[babel]{csquotes}
% bib la TEX lo carica in automatico org-cite
\use{microtype}
\use{lmodern}
\use{subfig} % sottofigure
\use{multicol} % due colonne
\use{lipsum} % lorem ipsum
\use{color} % colori in latex
\use{parskip} % rimuove l'indentazione dei nuovi paragrafi %% Add parbox=false to all new tcolorbox
\use{centernot}
\use[outline]{contour}\contourlength{3pt}
\use{fancyhdr}
\use{layout}
\use[most]{tcolorbox} % Riquadri colorati
\use{ifthen} % IFTHEN
\use{geometry}

% pacchetti matematica
\use{yhmath}
\use{dsfont}
\use{mathrsfs}
\use{cancel} % semplificare
\use{polynom} %divisione tra polinomi
\use{forest} % grafi ad albero
\use{booktabs} % tabelle
\use{commath} %simboli e differenziali
\use{bm} %bold
\use[fulladjust]{marginnote} %to use marginnote for date notes
\use{arrayjobx}%array
\use[intlimits]{empheq} % Riquadri colorati attorno alle equazioni
\use{mathtools}
\use{circuitikz} % Disegnare i circuiti
\use{mathtools}
\use{stmaryrd} % [[ \llbracket ]] \rrbracket
\use{bussproofs} % dimostrazioni

%%%%%%%%%%%%%


%%%% QUIVER
\newcommand{\duepunti}{\,\mathchar\numexpr"6000+`:\relax\,}
% A TikZ style for curved arrows of a fixed height, due to AndréC.
\tikzset{curve/.style={settings={#1},to path={(\tikztostart)
    .. controls ($(\tikztostart)!\pv{pos}!(\tikztotarget)!\pv{height}!270:(\tikztotarget)$)
    and ($(\tikztostart)!1-\pv{pos}!(\tikztotarget)!\pv{height}!270:(\tikztotarget)$)
    .. (\tikztotarget)\tikztonodes}},
    settings/.code={\tikzset{quiver/.cd,#1}
        \def\pv##1{\pgfkeysvalueof{/tikz/quiver/##1}}},
    quiver/.cd,pos/.initial=0.35,height/.initial=0}

% TikZ arrowhead/tail styles.
\tikzset{tail reversed/.code={\pgfsetarrowsstart{tikzcd to}}}
\tikzset{2tail/.code={\pgfsetarrowsstart{Implies[reversed]}}}
\tikzset{2tail reversed/.code={\pgfsetarrowsstart{Implies}}}
% TikZ arrow styles.
\tikzset{no body/.style={/tikz/dash pattern=on 0 off 1mm}}
%%%%%%%%%%


%% DEFINIZIONI COMANDI MATEMATICI
\let\sin\relax %TOGLIE LA DEFINIZIONE SU "\sin"

% cambia la definizione di empty set
% ---
\let\oldemptyset\emptyset
% ---
% \let\emptyset\varnothing
% ---
% \let\emptyset\relax
% \newcommand{\emptyset}{\text{\textnormal{\O}}}
% ---

\DeclareMathOperator{\bounded}{bd}
\DeclareMathOperator{\sin}{sen}
\DeclareMathOperator{\epi}{Epi}
\DeclareMathOperator{\cl}{cl}
\DeclareMathOperator{\graph}{graph}
\DeclareMathOperator{\arcsec}{arcsec}
\DeclareMathOperator{\arccot}{arccot}
\DeclareMathOperator{\arccsc}{arccsc}
\DeclareMathOperator{\spettro}{Spettro}
\DeclareMathOperator{\nulls}{nullspace}
\DeclareMathOperator{\dom}{dom}
\DeclareMathOperator{\ar}{ar}
\DeclareMathOperator{\const}{Const}
\DeclareMathOperator{\fun}{Fun}
\DeclareMathOperator{\rel}{Rel}
\DeclareMathOperator{\altezza}{ht}
\let\det\relax %TOGLIE LA DEFINIZIONE SU "\det"
\DeclareMathOperator{\det}{det}
\DeclareMathOperator{\End}{End}
\DeclareMathOperator{\gl}{GL}
\def\Id{\mathrm{Id}}
\def\id{\mathrm{id}}
\DeclareMathOperator{\I}{\mathds{1}}
\DeclareMathOperator{\II}{II}
\DeclareMathOperator{\rank}{rank}
\DeclareMathOperator{\tr}{tr}
\DeclareMathOperator{\tc}{t.c.}
\DeclareMathOperator{\T}{T}
\DeclareMathOperator{\var}{Var}
\DeclareMathOperator{\cov}{Cov}
\DeclareMathOperator{\st}{st}
\DeclareMathOperator{\mon}{Mon}
\newcommand{\card}[1]{\left\vert #1 \right\vert}
\newcommand{\trasposta}[1]{\prescript{\text{T}}{}{#1}}
\newcommand{\1}{\mathds{1}}
\newcommand{\R}{\mathds{R}}
\newcommand{\diesis}{\#}
\newcommand{\bemolle}{\flat}
\newcommand{\nonstandard}[1]{\prescript{*}{}{#1}}
\newcommand{\starR}{\nonstandard{\R}}
\newcommand{\borel}{\mathscr{B}}
\newcommand{\lebesgue}[1]{\mathscr{L}\left(#1\right)}
\newcommand{\media}{\mathds{E}}
\newcommand{\K}{\mathds{K}}
\newcommand{\A}{\mathds{A}}
\newcommand{\Q}{\mathds{Q}}
\newcommand{\N}{\mathds{N}}
\newcommand{\C}{\mathds{C}}
\newcommand{\Z}{\mathds{Z}}
\newcommand{\qo}{\hspace{1em}\text{q.o.}\,}
\renewcommand{\tilde}[1]{\widetilde{#1}}
\renewcommand{\parallel}{\mathrel{/\mkern-5mu/}}
\newcommand{\parti}[2][]{\wp_{#1}(#2)}
\newcommand{\diff}[1]{\operatorname{d}_{#1}}
\let\oldvec\vec
\renewcommand{\vec}[1]{\overrightarrow{\vphantom{i}#1}}
\newcommand{\floor}[1]{\left\lfloor #1 \right\rfloor}
\newcommand{\cat}[1]{\mathbf{#1}}
\newcommand{\dfreccia}[1]{\xrightarrow{\ #1 \ }}
\newcommand{\sfreccia}[1]{\xleftarrow{\ #1 \ }}
\newcommand{\formalsum}[2]{{\sum_{#1}^{#2}}{\vphantom{\sum}}'}
\newcommand{\minim}[2]{\mu_{#1}\, \left(#2\right)}
\newcommand{\concat}{\null^{\frown}} % concatenazione di stringe
\newcommand{\godelcode}[1]{\langle\!\langle #1 \rangle\!\rangle}
\newcommand{\godeldec}[1]{(\!(#1)\!)}
\newcommand{\termcode}[1]{\ulcorner #1\urcorner}
\newcommand{\partialto}{\dashrightarrow}
\newcommand{\restricted}{\upharpoonright}
\newcommand{\embeds}{\precsim}
\newcommand{\surjects}{\twoheadrightarrow}
\newcommand{\equipotenti}{\asymp}
%% \newcommand{\dotplus}{\mathbin{\dot{+}}} %% A quanto pare esiste già
\newcommand{\bigdot}{\mathbin{\boldsymbol{\cdot}}}
\newcommand{\dotexp}[1]{^{.#1}}
\newcommand{\conv}{\mathbin{*}}
\newcommand{\convolution}[2]{(#1\conv #2)}
\newcommand{\nil}{\mathfrak{N}}
\newcommand{\divisore}{\mathrel{|}}
\newcommand{\simplesso}[1]{\mathrm{e}_{#1}}

\renewcommand{\iff}{\mathrel{\longleftrightarrow}} %% Notazione Logica.
\newcommand{\oldiff}{\mathrel{\Longleftrightarrow}}
\renewcommand{\implies}{\mathrel{\rightarrow}} %% Notazione Logica
\newcommand{\oldimplies}{\mathrel{\Longrightarrow}}
\renewcommand{\impliedby}{\mathrel{\leftarrow}} %% Notazione Logica
\newcommand{\oldimpliedby}{\mathrel{\Longleftarrow}}

\newcommand{\IFF}{\quad\Longleftrightarrow\quad}
\newcommand{\IMPLICA}{\quad\Longrightarrow\quad}


\renewcommand{\descriptionlabel}[1]{\hspace{\labelsep}\normalfont #1} % remove bold from description


%% Definizione di Divergenza di K-L

\DeclarePairedDelimiterX{\infdivx}[2]{(}{)}{%
  #1\;\delimsize\|\;#2%
}
\newcommand{\kldiv}{D_{KL}\infdivx}

%% Definizione di \dotminus

\makeatletter
\newcommand{\dotminus}{\mathbin{\text{\@dotminus}}}

\newcommand{\@dotminus}{%
  \ooalign{\hidewidth\raise1ex\hbox{.}\hidewidth\cr$\m@th-$\cr}%
}
\makeatother

%tramite i prossimi due comandi posso decidere come scrivere i logaritmi naturali in tutti i documenti: ho infatti eliminato qualsiasi differenza tra "ln" e "log": se si vuole qualcosa di diverso bisogna inserire manualmente il tutto
\let\ln\relax
\DeclareMathOperator{\ln}{ln}
\let\log\relax
\DeclareMathOperator{\log}{log}
%%%%%%

%% NUOVI COMANDI
\newcommand{\straniero}[1]{\textit{#1}} %parole straniere
\newcommand{\titolo}[1]{\textsc{#1}} %titoli
\newcommand{\qedd}{\tag*{$\blacksquare$}} %qed per ambienti matemastici
\renewcommand{\qedsymbol}{$\blacksquare$} %modifica colore qed
\newcommand{\ooverline}[1]{\overline{\overline{#1}}}
\newcommand{\circoletto}[1]{\left(#1\right)^{\text{o}}}
%
\newcommand{\qmatrice}[1]{\begin{pmatrix}
#1_{11} & \cdots & #1_{1n}\\
\vdots & \ddots & \vdots \\
#1_{m1} & \cdots & #1_{mn}
\end{pmatrix}}
%
\newcommand{\parentesi}[2]{%
\underset{#1}{\underbrace{#2}}%
}
%
\newcommand{\norma}[1]{% Norma
\left\lVert#1\right\rVert%
}
\newcommand{\scalare}[2]{% Scalare
\left\langle #1, #2\right\rangle
}
%%%%%

%% RESTRIZIONI
\newcommand{\referenze}[2]{
        \phantomsection{}#2\textsuperscript{\textcolor{blue}{\textbf{#1}}}
}

\let\restriction\relax

\def\restriction#1#2{\mathchoice
              {\setbox1\hbox{${\displaystyle #1}_{\scriptstyle #2}$}
              \restrictionaux{#1}{#2}}
              {\setbox1\hbox{${\textstyle #1}_{\scriptstyle #2}$}
              \restrictionaux{#1}{#2}}
              {\setbox1\hbox{${\scriptstyle #1}_{\scriptscriptstyle #2}$}
              \restrictionaux{#1}{#2}}
              {\setbox1\hbox{${\scriptscriptstyle #1}_{\scriptscriptstyle #2}$}
              \restrictionaux{#1}{#2}}}
\def\restrictionaux#1#2{{#1\,\smash{\vrule height .8\ht1 depth .85\dp1}}_{\,#2}}
%%%%%%%%%%%

%%% FORMATTAZIONE FOOTNOTEMARK

\def\footnotemarkformatting#1{[#1]}
\renewcommand{\thefootnote}{\footnotemarkformatting{\arabic{footnote}}}

%% SEZIONE GRAFICA
\use{tikz}
\usetikzlibrary{matrix, patterns, calc, decorations.pathreplacing, hobby, decorations.markings, decorations.pathmorphing, babel}
\use{tikz-3dplot}
\use{mathrsfs} %per geogebra
\use{tikz-cd}
\tikzset
{
  %surface/.style={fill=black!10, shading=ball,fill opacity=0.4},
  plane/.style={black,pattern=north east lines},
  curve/.style={black,line width=0.5mm},
  dritto/.style={decoration={markings,mark=at position 0.5 with {\arrow{Stealth}}}, postaction=decorate},
  rovescio/.style={decoration={markings,mark=at position 0.5 with {\arrow{Stealth[reversed]}}}, postaction=decorate}
}
\use{pgfplots} % stampare le funzioni
        \pgfplotsset{/pgf/number format/use comma,compat=1.15}
        %\pgfplotsset{compat=1.15} %per geogebra
        \usepgfplotslibrary{fillbetween, polar}
%%%%%%

%% CITAZIONI
\use{lineno}

\newcommand{\citazione}[1]{%
  \begin{quotation}
  \begin{linenumbers}
  \modulolinenumbers[5]
  \begingroup
  \setlength{\parindent}{0cm}
  \noindent #1
  \endgroup
  \end{linenumbers}
  \end{quotation}\setcounter{linenumber}{1}
  }
%%%%%%

%%%%%%%%%%%%%%%%%%%%%%%%%%%%%%%%%%%%%%%%%%%%
%%%%%%%%%%%%%%%%%%%%%%%%%%%%%%%%%%%%%%%%%%%%

%% AMS THM

\theoremstyle{definition}% default
\newtheorem{thm}{Teorema}[section]
\newtheorem{lem}[thm]{Lemma}
\newtheorem{prop}[thm]{Proposizione}
\newtheorem{cor}[thm]{Corollario}
\newtheorem{esempio}[thm]{Esempio}
\theoremstyle{plain}
\newtheorem{definizione}[thm]{Definizione}
\theoremstyle{remark}
\newtheorem*{oss}{Osservazione}


%%%%%%%%%%%%%%%%%%%%%%%%%%%%%%%%%%%%%%%%%%%%
%%%%%%%%%%%%%%%%%%%%%%%%%%%%%%%%%%%%%%%%%%%%

\use{hyperref}
\hypersetup{%
        pdfauthor={Davide Peccioli},
        pdfsubject={},
        allcolors=black,
        citecolor=black,
%	colorlinks=true,
        bookmarksopen=true}
\setcounter{secnumdepth}{0} % rimuove i numeri di sezione senza rimuovere le ref
\renewcommand{\href}[2]{\textcolor{blue}{#2}} % disabilita il comando href
\use{enotez} %
\setenotez{%
 mark-format = \footnotemarkformatting % Mette i numeri tra parentesi quadre%
}\let\footnote=\endnote % rende tutte le note a pié pagina come delle note a fine file 


\let\olddocument\document % modifico l'ambiende documenti per non dover stampare \printendnote
\let\oldenddocument\enddocument
\renewenvironment{document}%
{%
  \olddocument
}{%
  \printendnotes\oldenddocument
}
\renewcommand{\thethm}{\arabic{thm}}

\usepackage[hyperref]{biblatex}
\addbibresource{~/Documents/org/roam/bib/master.bib}
\author{Davide Peccioli}
\date{\today}
\title{Esempi di spazi polacchi}
\begin{document}

\section{Il cubo di Hilbert è uno spazio polacco}
\label{sec:org7d19073}
Il \href{20250310122026-cubo_di_hilbert.org}{cubo di Hilbert}: \([0,1]^{\N}\) (vedi \href{20250202192030-classe_delle_classi_funzioni.org}{Classe delle Classi-Funzioni} e anche \href{20250131183735-prodotto_cartesiano_di_classi_mk.org}{Prodotto cartesiano generalizzato}) è uno \href{20250301194013-spazio_polacco.org}{spazio polacco} quando dotato della \href{20250109154723-topologia_prodotto.org}{topologia prodotto}.
\section{Esercizi TDI - Foglio 1}
\label{sec:org2ab0228}
\subsection{Esercizio 1}
\label{sec:org2a16f14}

Prove that the following are Polish subspaces of the \href{20250313194836-spazio_di_baire.org}{Baire space} \(\omega^\omega\):
\begin{align*}
A &= \{x \in \omega^\omega \mid x \text{ has infinite range} \}\\
B &= \{x \in \omega^\omega \mid x^{-1}(n) \text{ is infinite, for every } n \in \omega \}.
\end{align*}

In contrast, show that
\begin{align*}
C &= \{x \in \omega^\omega \mid x \text{ is not surjective} \}\\
D &= \{x \in \omega^\omega \mid x \text{ has finite range} \}
\end{align*}
are \uline{not} Polish. (Use the fact that \href{20250313194245-sottospazi_di_spazi_polacchi_fsigma_densi_e_codensi_non_sono_gdelta.org}{in a Polish space \(X\), if \(A \subseteq X\) is \(\bm{F}_\sigma\) and both dense and codense, then \(A\) is not \(\bm{G}_\delta\)}.)
\subsubsection{Soluzione}
\label{sec:orgfb6302f}

Si fissa una \href{20250301194153-spazio_metrico_completo.org}{metrica completa} \(d\) su \(\omega^{\omega}\): per ogni \(x,y \in \omega^{\omega}\)
\begin{equation*}
d(x,y) \coloneqq \sum_{i \in \omega} 2^{-i}\ \frac{|x(i)-y(i)|}{|x(i)-y(i)|+1}
\end{equation*}
\paragraph{Insieme \(A\)}
\label{sec:org53827ba}

Si vuole dimostrare che \(A\) sia un \href{20250304152026-sottoinsiemi_gdelta_e_fsigma.org}{insieme \(\bm{G}_{\delta}\)}, visto il \href{20250306134632-caratterizzazione_dei_sottoinsiemi_polacchi_di_uno_spazio_polacco.org}{Teorema di caratterizzazione}.

Consideriamo, per ogni \(n \in \omega\):
\begin{equation*}
A_{n} \coloneqq\set{x \in \omega^{\omega}\ |\ \card{x(\omega)}>n} \subseteq \omega^{\omega}
\end{equation*}
dove con \(\card{\cdot}\) si intende la \href{20241213101756-cardinalita.org}{cardinalità}.

Allora \(A = \bigcap_{n \in \omega} A_{n}\). È sufficiente dimostrare che gli \(A_{n}\) siano \href{20250103145124-topologia.org}{aperti} nella \href{20250109154723-topologia_prodotto.org}{topologia prodotto}.

Sia \(x \in A_{n}\) fissato, e sia \(N \in \omega\) tale che \(\card{\operatorname{ran}(x\upharpoonright N )}>n\). Allora
\begin{equation*}
x \in \set{x(0)}\times \set{x(1)}\times \dots\times \set{x(N)}\times \omega\times \cdots \eqqcolon A_{x}
\end{equation*}
e \(A_{x}\) aperto della \href{20250109154723-topologia_prodotto.org}{topologia prodotto}. Inoltre, \(A_{x} \subseteq A_{n}\), quindi \(A_{n}\) è \href{20250111142313-intorno.org}{intorno} di ogni suo punto, e \href{20250317093153-insieme_aperto_sse_intorno_di_ogni_suo_punto.org}{pertanto} aperto.
\paragraph{Insieme \(B\)}
\label{sec:org1b4fc3f}

Si vuole dimostrare che \(B\) sia un \href{20250304152026-sottoinsiemi_gdelta_e_fsigma.org}{insieme \(\bm{G}_{\delta}\)}, visto il \href{20250306134632-caratterizzazione_dei_sottoinsiemi_polacchi_di_uno_spazio_polacco.org}{Teorema di caratterizzazione}.

Sia, per ogni \(n,m \in \omega\):
\begin{equation*}
B_{n,m} \coloneqq  \set{x \in \omega^{\omega} \ |\ \card{x^{-1}(n)}>m} \subseteq \omega^{\omega}
\end{equation*}
dove con \(\card{\cdot}\) si intende la \href{20241213101756-cardinalita.org}{cardinalità}.

Allora
\begin{equation*}
B=\bigcap_{(n,m) \in \omega^{2}}\! B_{n,m}
\end{equation*}
dove \(\omega\) e \(\omega^{2}\) \href{20250215151413-biiezione_canonica_tra_n_e_n2.org}{sono} \href{20250619101109-classi_equipotenti.org}{equipotenti}: \(\omega\asymp \omega^{2}\). È quindi sufficiente dimostrare che i \(B_{n,m}\) sono aperti.

Sia \(x \in B_{n,m}\) e sia \(N \in \omega\) tale che
\begin{equation*}
\card{\set{i \le N\ |\ x(i) = n}}>m
\end{equation*}
Allora
\begin{equation*}
x \in \set{x(0)}\times \set{x(1)}\times \dots\times \set{x(N)}\times \omega\times \cdots \eqqcolon B_{x}
\end{equation*}
e \(B_{x}\) aperto della \href{20250109154723-topologia_prodotto.org}{topologia prodotto}. Inoltre, \(B_{x} \subseteq B_{n,m}\), quindi \(B_{n,m}\) è \href{20250111142313-intorno.org}{intorno} di ogni suo punto, e \href{20250317093153-insieme_aperto_sse_intorno_di_ogni_suo_punto.org}{pertanto} aperto.
\paragraph{Insiemi \(C\) e \(D\)}
\label{sec:orgd327e71}

Sfruttando il suggerimento, bisogna dimostrare che:
\begin{itemize}
\item \(C\) e \(D\) sono \href{20250304152026-sottoinsiemi_gdelta_e_fsigma.org}{insiemi \(\bm{F}_{\sigma}\)};
\item \(C\) e \(D\) sono \href{20250301193045-sottoinsieme_denso.org}{densi} in \(\omega^{\omega}\);
\item \(C\) e \(D\) sono \href{20250301193045-sottoinsieme_denso.org}{codensi} in \(\omega^{\omega}\).
\end{itemize}

Siccome \(D \subseteq C \subseteq \omega^{\omega}\), si ha che \(\omega^{\omega}\setminus C \subseteq \omega^{\omega}\setminus D \subseteq \omega^{\omega}\). Pertanto, basta dimostrare che:
\begin{itemize}
\item \(C\) e \(D\) sono insiemi \(\bm{F}_{\sigma}\);
\item \(D\) è denso in \(\omega^{\omega}\);
\item \(C\) è codenso in \(\omega^{\omega}\).
\end{itemize}
\begin{enumerate}
\item Insiemi \(\bm{F}_{\sigma}\)
\label{sec:org809276b}

Posto \(C_{n}\coloneqq\set{x\in \omega^{\omega}\ |\ n\notin x(\omega)}\), è possibile scrivere
\begin{equation*}
C = \bigcup_{n \in \omega}\set{x\in \omega^{\omega}\ |\ n\notin x(\omega)}
\end{equation*}
I \(C_{n}\) sono chiusi. Infatti, sfruttando la \href{20250303120747-caratterizzazione_dei_chiusi_in_termini_di_successioni.org}{caratterizzazione dei chiusi per successioni}, sia \((x_{k})_{k \in \omega}\) una \href{20250115100904-successione.org}{successione} \href{20250115100930-convergenza_per_una_successione.org}{convergente} di elementi di \(C_{n}\), \(x_{k}\to x\). Se per assurdo \(x \notin C_{n}\), allora esiste \(i_{0} \in \omega\) tale che \(x(i_{0})=n\). Allora, per ogni \(k \in \omega\),
\begin{equation*}
|x_{k}(i_{0})-x(i_{0})| = \varepsilon_{0}\ge 1
\end{equation*}
e quindi, per ogni \(k \in \omega\)
\begin{equation*}
d(x_{k},x) = \sum_{j \in \omega} 2^{-j}\ \frac{|x_{k}(i_{0})-x(i_{0})|}{1+|x_{k}(i_{0})-x(i_{0})|}\ge 2^{-i_{0}}\ \frac{\varepsilon_{0}}{\varepsilon_{0}+1}
\end{equation*}
Questo è assurdo poiché la successione converge e la distanza induce la topologia.

L'insieme \(D\), invece, è il \href{20250317100425-complementare_di_un_insieme.org}{complementare} di \(A\) e \(A\) è un insieme \(\bm{G}_{\delta}\). Pertanto \(D\) è un insieme \(\bm{F}_{\sigma}\).
\item \(D\) è denso
\label{sec:org379619b}

\href{20250303121451-caratterizzazione_della_chiusura_in_termini_di_successioni.org}{Si dimostra che} per ogni \(x \in \omega^{\omega}\) esiste una \href{20250115100904-successione.org}{successione} \href{20250115100930-convergenza_per_una_successione.org}{convergente}
\begin{equation*}
(x_{n})_{n \in \omega} \subseteq D
\end{equation*}
tale che \(x_{n}\to x\).

Sia \(x \in \omega^{\omega}\) fissato e sia, per ogni \(n \in \omega\):
\begin{equation*}
x_{n} \coloneqq (x\upharpoonright n)\cup\set{(m,0)\ |\ n<m<\omega}
\end{equation*}
Allora \(x_{n} \in \omega^{\omega}\) e in particolare \(x_{n} \in D\), poiché ha \href{20250202173528-dominio_range_e_campo_di_una_classe_relazione.org}{range} di \href{20241213101756-cardinalita.org}{cardinalità} minore di \(n+2\).

Inoltre, \(x_{n}\to x\). \href{20250317125810-successioni_in_spazio_metrico_hanno_lo_stesso_limite_sse_limite_delle_distanze_e_nullo.org}{Infatti} la successione delle distanze tende a \(0\):
\begin{align*}
d(x_{n},x) &= \sum_{j \in \omega} 2^{-j}\ \frac{|x_{n}(j)-x(j)|}{1+|x_{n}(j)-x(j)|}\\
&= \sum_{j\ge n} 2^{-j}\ \frac{x(j)}{1+x(j)} \le \sum_{j\ge n} 2^{-j} = 2^{1-n} \to 0
\end{align*}
\item \(C\) è codenso
\label{sec:orgf8c9632}

\href{20250303121451-caratterizzazione_della_chiusura_in_termini_di_successioni.org}{Si dimostra che} per ogni \(x \in \omega^{\omega}\) esiste una \href{20250115100904-successione.org}{successione} \href{20250115100930-convergenza_per_una_successione.org}{convergente}
\begin{equation*}
(x_{n})_{n \in \omega} \subseteq \omega^{\omega}\setminus C
\end{equation*}
tale che \(x_{n}\to x\).

Sia \(x \in \omega^{\omega}\) fissato, e sia, per ogni \(n \in \omega\):
\begin{equation*}
x_{n} \coloneqq (x\upharpoonright n) \cup \set{(m,m-n)\ |\ n\le m <\omega}
\end{equation*}
Ciascuna \(x_{n}\notin \omega^{\omega}\setminus C\), in quanto \(x_{n}\) è suriettiva. Inoltre, \(x_{n}\to x\). \href{20250317125810-successioni_in_spazio_metrico_hanno_lo_stesso_limite_sse_limite_delle_distanze_e_nullo.org}{Infatti} la successione delle distanze tende a \(0\):
\begin{align*}
d(x_{n},x) &= \sum_{j \in \omega} 2^{-j}\ \frac{|x_{n}(j)-x(j)|}{1+|x_{n}(j)-x(j)|}\\
&= \sum_{j \ge n}  2^{-j}\ \frac{|x_{n}(j)-x(j)|}{1+|x_{n}(j)-x(j)|}\\
&\le \sum_{j\ge n} 2^{-j} = 2^{1-n}\to 0
\end{align*}
\end{enumerate}
\subsection{Esercizio 2}
\label{sec:orgc93d5ac}

Let \(2^{(\omega^{<\omega})}\) be endowed with the \href{20250109154723-topologia_prodotto.org}{product} over the \href{20250111143651-insieme_numerabile.org}{countable} index set \(\omega^{<\omega}\) of the \href{20250317165247-topologia_discreta.org}{discrete topology} on \(2 =\set{0, 1}\). Let \(\operatorname{Tr} \subseteq 2^{(\omega^{<\omega})}\) be the set consisting of all \href{20250215160218-funzione_caratteristica.org}{characteristic functions} of trees on \(\omega\). Show that \(\operatorname{Tr}\) is closed in \(2^{(\omega^{<\omega})}\) and thus it is a \href{20250301194013-spazio_polacco.org}{Polish space}. Show also that the set \(\operatorname{PTr} \subseteq \operatorname{Tr}\) of (the characteristic functions of) pruned trees is \(\bm{G}_{\delta}\) and thus Polish as well. Finally, prove that \(\operatorname{Tr}\setminus \operatorname{PTr}\) is not a Polish space.
\subsubsection{Soluzione}
\label{sec:orgbc31b8a}

La topologia prodotto di \(2^{(\omega^{<\omega})}\) è generata dai seguenti aperti:
\begin{equation*}
\mathscr{B}\coloneqq\set{
	\prod_{\eta \in \omega^{<\omega}} U_{\eta} \ \left|\
		\parbox{16.5em}{
			$U_{\eta} \subseteq 2$\\ $U_{\eta}\neq 2$ per un numero finito di indici
		}\right.
}.
\end{equation*}
Data la topologia discreta di \(2\), infatti, \(U_{\eta} \subseteq 2\) è sempre aperto. In particolare, \(\mathscr{B}\) è esattamente la topologia di \(2^{(\omega^{<\omega})}\), in quanto:
\begin{itemize}
\item \(\mathscr{B}\) è chiuso per intersezioni finite;
\item \(\mathscr{B}\) è chiuso per unioni arbitrarie.
\end{itemize}
\paragraph{\(\operatorname{Tr}\) è chiuso in \(2^{(\omega^{<\omega})}\)}
\label{sec:org98e8c88}
Per la definizione di albero, per ogni \(f \in 2^{(\omega^{<\omega})}\), \(f\notin \operatorname{Tr}\) se e solo se esiste \(s \in \omega^{<\omega}\) tale che
\begin{equation*}
f(s)=1;\qquad f\left(s\upharpoonright (\operatorname{lh}(s)-1)\right) = 0
\end{equation*}
Si denoti per questa prima parte dell'esercizio: \(s\upharpoonright (\operatorname{lh}(s)-1) \eqqcolon s^{*}\).

In previsione di sfruttare la \href{20250303120747-caratterizzazione_dei_chiusi_in_termini_di_successioni.org}{caratterizzazione dei chiusi per successioni}, sia \((\chi_{n})_{n \in \omega} \subseteq \operatorname{Tr}\) una \href{20250115100904-successione.org}{successione} \href{20250115100930-convergenza_per_una_successione.org}{convergente}:
\begin{equation*}
\chi_{n}\to \chi
\end{equation*}

Si supponga per assurdo che \(\chi\notin \operatorname{Tr}\). Allora esiste \(s \in \omega^{<\omega}\) tale che
\begin{equation*}
\chi(s) = 1;\qquad \chi(s^{*}) = 0
\end{equation*}
Allora \(\chi\) appartiene all'aperto \(V\coloneqq \prod_{\eta \in \omega^{\omega}} V_{\eta}\), con
\begin{equation*}
V_{\eta} \coloneqq \begin{cases}
2 & \eta\neq s,s^{*}\\
\set{1} & \eta=s\\
\set{0} & \eta = s^{*}
\end{cases}
\end{equation*}
e pertanto esiste \(N \in \omega\) tale che \(\chi_{N} \in V\). Assurdo, poiché questo implica che \(\chi_{N}\notin \operatorname{Tr}\).

Pertanto \(\chi \in \operatorname{Tr}\) e dunque \(\operatorname{Tr}\) chiuso.
\paragraph{\(\operatorname{PTr}\) è uno spazio polacco}
\label{sec:orgec89039}

Per definizione di albero potato, si ha che \(\chi \in \operatorname{PTr}\) se e solo se \(\chi \in \operatorname{Tr}\) e per ogni \(s \in \omega^{<\omega}\):
\begin{equation*}
\chi(s) = 1\,\implies\, \exists\,m \in \omega\mid \chi\left(s\concat m\right)=1
\end{equation*}

Si definisce, per ogni \((\eta, j) \in \omega^{<\omega}\times \omega\)
\begin{equation*}
\Lambda_{\eta,j}' \coloneqq \set{\chi \in \omega^{<\omega}\mid \chi(\eta)=1 \,\land\, \chi(\eta\concat j) = 1 }
\end{equation*}
Questi sono aperti in \(\omega^{<\omega}\) e pertanto i \(\Lambda_{\eta,j} \coloneqq \Lambda_{\eta,j}'\cap \operatorname{Tr}\) sono aperti in \(\operatorname{Tr}\) con la topologia di sottospazio. Si può considerare ulteriormente
\begin{equation*}
\Theta_{\eta} \coloneqq \set{\chi \in \omega^{<\omega}\mid \chi(\eta) = 0}\cap \operatorname{Tr}
\end{equation*}
anche questo aperto in \(\operatorname{Tr}\).

Si considerino ora gli aperti:
\begin{equation*}
\Gamma_{\eta} \coloneqq \Theta_{\eta} \cup \bigcup_{j \in \omega} \Lambda_{\eta,j}
\end{equation*}
Si ha che \(\operatorname{PTr} = \bigcap_{\eta \in \omega^{<\omega}} \Gamma_{\eta}\). Infatti:
\begin{itemize}
\item se \(\chi \in \operatorname{PTr}\) allora per ogni \(\eta \in \omega^{<\omega}\):
\begin{itemize}
\item se \(\chi(\eta) = 1\) allora esiste \(j \in \omega\) tale che \(\chi(\eta\concat j) =1\), e allora \(\chi \in \Lambda_{\eta,j}\) e dunque \(\chi \in \Gamma_{\eta}\)
\item se \(\chi(\eta) = 0\) allora \(\chi \in \Theta_{\eta}\) e dunque \(\chi \in \Gamma_{\eta}\);
\end{itemize}
pertanto \(\chi \in \bigcap_{\eta \in \omega^{<\omega}}\Gamma_{\eta}\);
\item se, viceversa, \(\chi \in \bigcap_{\eta \in \omega^{<\omega}}\Gamma_{\eta}\), allora
\begin{itemize}
\item \(\chi \in \operatorname{Tr}\);
\item per ogni \(\eta \in \omega^{<\omega}\), \(\chi \in \Gamma_{\eta}\) e pertanto, se \(\chi(\eta) = 1\) allora esiste \(j \in \omega\) tale che \(\chi(\eta\concat j) = 1\)
\end{itemize}
pertanto \(\chi \in \operatorname{PTr}\).
\end{itemize}

Siccome \(\omega^{<\omega}\) è numerabile, si è dimostrata la tesi.
\paragraph{\(\operatorname{Tr}\setminus\operatorname{PTr}\) non è uno spazio polacco}
\label{sec:org1174fd1}

L'insieme \(\operatorname{Tr}\setminus\operatorname{PTr}\) è \(\bm{F}_{\sigma}\), poiché \(\operatorname{PTr}\) è polacco. Sfruttando il fatto che sottoinsiemi \(\bm{F}_{\sigma}\) di un polacco, densi e codensi, non possono essere \(\bm{G}_{\delta}\), si dimostra che \(\operatorname{Tr}\setminus \operatorname{PTr}\) sia denso e codenso.

\begin{itemize}
\item \(\operatorname{Tr}\setminus \operatorname{PTr}\) è denso. Infatti, sia \(\chi \in \operatorname{Tr}\). Se \(\chi \notin\operatorname{PTr}\), allora la successione costante \((\chi)_{n \in \omega}\) converge a \(\chi\).

Se invece \(\chi \in \operatorname{PTr}\), sia \(A \in \omega^{\omega}\) tale che, per ogni \(n \in \N\), \(\chi(A\upharpoonright n) = 1\). Si definisce allora \(\chi_{n} \in \operatorname{Tr}\setminus \operatorname{PTr}\), per ogni \(s \in \omega^{<\omega}\)

\begin{equation*}
  \chi_{n}(s) \coloneqq \begin{cases}
  	0 & \operatorname{lh}(s) \ge n \,\land\, s = A\upharpoonright \operatorname{lh}(s)\\
  	\chi(s) & \text{altrimenti}
      \end{cases}
\end{equation*}

Chiaramente \(\chi_{n} \in \operatorname{Tr}\setminus \operatorname{PTr}\), ed inoltre \(\chi_{n}\to \chi\). Infatti, sia \(\emptyset \neq U \subseteq 2^{(\omega^{<\omega})}\) aperto, tale che \(\chi \in U\) con \(U\neq \emptyset\):
\begin{equation*}
  U = \operatorname{Tr}\cap \prod_{\eta \in \omega^{<\omega}} U_{\eta}
\end{equation*}
con \(\emptyset\neq U_{\eta} \subseteq 2\) tali che un numero finito di \(U_{\eta}\neq 2\).

Sia quindi:
\begin{equation*}
  N \coloneqq \max\set{\operatorname{lh}(\eta)\mid U_{\eta} \neq 2}
\end{equation*}

Allora, per ogni \(n>N\), \(\chi_{n} \in U\). Infatti, se per assurdo \(\chi_{n}\notin U\), allora esiste \(s \in \omega^{<\omega}\) tale che \(\chi_{n}(s) \notin U_{s}\):
\begin{itemize}
\item se \(\operatorname{lh}(s)<n\), allora \(\chi_{n}(s)=\chi(s)\), ma \(\chi \in U\) e pertanto \(\chi(s) \in U_{s}\); assurdo;
\item se \(\operatorname{lh}(s)\ge n\), allora \(\operatorname{lh}(s) > N\), e per massimalità quindi \(U_{s}= 2\); pertanto \(\chi_{n}(s) \in U_{s}\). Assurdo.
\end{itemize}

\item \(\operatorname{PTr}\) è denso. Infatti, se per assurdo non lo fosse, allora esisterebbe \(U \subseteq \operatorname{Tr}\) non vuoto, aperto, e tale che \(U\cap \operatorname{PTr} =\emptyset\).

Allora
\begin{equation*}
  U = \operatorname{Tr}\cap \prod_{\eta \in \omega^{<\omega}} U_{\eta}
\end{equation*}
con \(\emptyset\neq U_{\eta} \subseteq 2\) tali che un numero finito di \(U_{\eta}\neq 2\). Siano \(\set{\eta_{1},\dots,\eta_{k}} \subseteq \omega^{<\omega}\) tali che \(U_{\eta_{i}} = \set{1}\), per \(i=1,\dots,k\), e siano \(\set{\theta_{1},\dots,\theta_{h}} \subseteq \omega^{<\omega}\) tali che \(U_{\theta_{j}} = \set{0}\), per \(j=1,\dots,h\). Necessariamente, per ogni \(j = 1,\dots,h\), e per ogni \(i=1,\dots,k\):
\begin{equation*}
  \forall\, \ell\le \operatorname{lh}(\eta_{i}),\qquad \eta_{i}\upharpoonright \ell \neq \theta_{j}
\end{equation*}
altrimenti \(U \subseteq \operatorname{Tr}\) sarebbe l'insieme vuoto.

Allora si definisce \(\chi \in \operatorname{Tr}\) tale che, per ogni \(s \in\omega^{<\omega}\):
\begin{itemize}
\item se \(s = \eta_{i}\upharpoonright \ell\) per qualche \(i=1,\dots,k\) e per qualche \(\ell \le \operatorname{lh}(\eta_{i})\), allora \(\chi(s) = 1\)
\item se \(s \in \set{\theta_{1},\dots,\theta_{h}}\), allora \(\chi(s) = 0\).
\item Si costruisce ricorsivamente un insieme \(\Lambda \subseteq \omega^{<\omega}\) tale che \(\set{\eta_{1},\dots,\eta_{k}} \subseteq \Lambda\), e tale che, se \(\eta \in \Lambda\), allora anche \(\eta\concat \ell \in\Lambda\), dove \(\ell \in \omega\) è il più piccolo naturale tale che \(\eta\concat \ell \notin \set{\theta_{1},\dots,\theta_{h}}\); tale \(\ell\) esiste sempre, poiché \(\set{\theta_{1},\dots,\theta_{h}}\) è un insieme finito. Per ogni \(s \in \Lambda\), \(\chi(s)=1\).
\item Per tutti gli altri \(s \in \omega^{<\omega}\), \(\chi(s) = 0\).
\end{itemize}

Si mostra che \(\chi\) genera un assurdo.
\begin{itemize}
\item \(\chi\) è ben definita, poiché i quattro casi considerati sono disgiunti.
\item \(\chi \in \operatorname{Tr}\), in quanto, se \(\chi(s) = 1\), allora per ogni \(\ell \le \operatorname{lh}(s)\), \(\chi(s\upharpoonright \ell) = 1\).
\item \(\chi \in \operatorname{PTr}\), in quanto, se \(\chi(s) = 1\), allora esiste sempre \(\eta\) che estende \(s\) tale che \(\chi(\eta)=1\).
\end{itemize}
\end{itemize}

Quindi \(\chi \in U\cap \operatorname{PTr} =\emptyset\). Assurdo.
\subsection{Esercizio 3}
\label{sec:orge3ac8a4}

Let
\[
\mathscr{L} = \{ R_i \mid i < I \} \cup \{ f_j \mid j < J \} \cup \{ a_k \mid k < K \}
\]
with \(I, J, K \leq \omega\) be an at most \href{20250111143651-insieme_numerabile.org}{countable} \href{20250130162057-linguaggio_del_prim_ordine.org}{first-order language}, and let \(M\) be a \href{20250111143651-insieme_numerabile.org}{countable} \(\mathscr{L}\)-\href{20250131103035-struttura_del_prim_ordine.org}{structure}. Without loss of generality, we may assume that the \href{20250131103035-struttura_del_prim_ordine.org}{domain} of \(M\) is \(\omega\) itself. Prove that the \href{20241205141146-gruppo_abeliano.org}{group} of \href{20250214120959-mappe_tra_strutture_del_prim_ordine.org}{automorphisms} \(\operatorname{Aut}(M)\) of \(M\) is a \href{20250304151924-gruppo_polacco.org}{Polish} \href{20241206143051-sottogruppo.org}{subgroup} of \(S_\infty\).
\subsubsection{Soluzione}
\label{sec:orgcbfd798}

Si fissi su \(\omega^{\omega}\supseteq S_{\infty}\) la \href{20250301194153-spazio_metrico_completo.org}{metrica completa} \(d\):
\begin{equation*}
d(x,y) =  \begin{cases}
0 & x=y\\
2^{-(n+1)} & x\neq y\text{ e }n\text{ il più piccolo naturale tale che }x(n)\neq y(n)
\end{cases}
\end{equation*}
e si denoti con
\begin{equation*}
B_{d}(x,\varepsilon) \coloneqq \set{y \in \omega^{\omega}\mid d(x,y)<\varepsilon}
\end{equation*}
la palla aperta.
\paragraph{\(\operatorname{Aut}(M)\) è uno spazio polacco}
\label{sec:orgd49d3db}

È possibile scrivere \(\operatorname{Aut}(M)\) come intersezione (numerabile) dei seguenti insiemi:
\begin{itemize}
\item per ogni \(i<I\): \(\set{f \in S_{\infty}\mid (a_{1},\dots,a_{\operatorname{ar}(R_{i})}) \in R_{i}\text{ sse }\left(f(a_{1}),\dots,f(a_{\operatorname{ar}(R_{i})})\right) \in R}\);
\item per ogni \(j < J\): \(\set{f \in S_{\infty}\mid f\left(f_{j}(a_{1},\dots,a_{\operatorname{ar}(f_{j})})\right) = f_{j}\left(f(a_{1}),\dots,f(a_{\operatorname{ar}(f_{j})})\right)}\);
\item per ogni \(k<K\): \(\set{f \in S_{\infty}\mid f(a_{k}) = a_{k}}\).
\end{itemize}

Volendo dimostrare che \(\operatorname{Aut}(M)\) sia polacco, si sfrutta la caratterizzazione, dimostrando che sia \(\bm{G}_{\delta}\), ovvero che tutti gli insiemi elencati sopra siano aperti, o, al più, \(\bm{G}_{\delta}\).

\begin{itemize}
\item Sia \(R \in \set{R_{i}\mid i<I}\) di arietà \(n\). L'insieme
\begin{equation*}
  \mathscr{R}_{R} \coloneqq \set{f \in S_{\infty}\mid (a_{1},\dots,a_{n}) \in R \text{ sse }\left(f(a_{1}),\dots,f(a_{n})\right) \in R,\, \forall\, a_{i} \in\omega}
\end{equation*}
è un \(\bm{G}_{\delta}\).

Infatti, posto
\begin{equation*}
  \mathscr{R}_{R,m} \coloneqq  \set{f \in S_{\infty}\mid (a_{1},\dots,a_{n}) \in R \text{ sse }\left(f(a_{1}),\dots,f(a_{n})\right) \in R,\, \forall\, a_{i} < m}
\end{equation*}
questo è un aperto.
\begin{itemize}
\item Presa \(f \in \mathscr{R}_{R,m}\), sia \(L_{f} \coloneqq \max_{a_{i}<m}\set{a_{1},\dots,a_{n}}\), e sia \(\varepsilon_{f} \coloneqq 2^{-(L_{f}+2)}\).

Allora \(f \in B_{d}(f, \varepsilon_{f}) \cap S_{\infty} \eqqcolon V\), e \(V \subseteq \mathscr{R}_{R,m}\). Se \(g \in V\) e \(g\neq f\), allora il più piccolo \(L\) tale che \(f(L)\neq g(L)\) è tale che
\begin{equation*}
2^{-(L+1)}<2^{-(L_{f}+2)}\,\implies\, L>L_{f}+1
\end{equation*}
e dunque, per ogni \(n<L\), compresi tutti gli \(a_{1},\dots,a_{n}\), si ha \(g(n) = f(n)\), e pertanto
\begin{align*}
(a_{1},\dots, a_n) \in R\,&\text{ sse }\, \left(f(a_{1}),\dots,f(a_{n})\right) \in R\\
&\text{ sse }\, \left(g(a_{1}),\dots,g(a_{n})\right) \in R
\end{align*}
\end{itemize}

Allora \(\mathscr{R}_{R} = \bigcap_{m \in\omega} \mathscr{R}_{R,m}\) e pertanto è un insieme \(\bm{G}_{\delta}\).

\item Sia \(G \in \set{F_{j}\mid j<J}\) di arietà \(n\). L'insieme
\begin{equation*}
  \mathscr{F}_{G}\coloneqq\set{f \in S_{\infty}\mid f\left(G(a_{1},\dots,a_{n})\right) = G\left(f(a_{1}),\dots,f(a_{n})\right),\, \forall\, a_{i} \in\omega}
\end{equation*}
è un \(\bm{G}_{\delta}\).

Infatti, posto
\begin{equation*}
  \mathscr{F}_{G, m} \coloneqq \set{f \in S_{\infty}\mid f\left(G(a_{1},\dots,a_{n})\right) = G\left(f(a_{1}),\dots,f(a_{n})\right),\, \forall\, a_{i} <m}
\end{equation*}
questo è un aperto.
\begin{itemize}
\item Preso \(f \in \mathscr{F}_{G,m}\), sia \(L_{f} \coloneqq \max_{a_{i}<m}\set{G(a_{1},\dots,a_{n}), a_{1},\dots,a_{n}}\), e sia \(\varepsilon_{f} \coloneqq 2^{-(L_{f}+2)}\).

Allora \(f \in B_{d}(f, \varepsilon_{f}) \cap S_{\infty} \eqqcolon V\), e \(V \subseteq \mathscr{F}_{G,m}\). Se \(g \in V\) e \(g\neq f\), allora il più piccolo \(L\) tale che \(f(L)\neq g(L)\) è tale che
\begin{equation*}
2^{-(L+1)}<2^{-(L_{f}+2)}\,\implies\, L>L_{f}+1
\end{equation*}
e dunque, per ogni \(n<L\), compresi tutti gli \(G(a_{1},\dots,a_{n}), a_{1},\dots,a_{n}\), vale \(g(n) = f(n)\), e pertanto
\begin{align*}
g\left(G(a_{1},\dots,a_{n})\right) &= f\left(G(a_{1},\dots,a_{n})\right)\\
&= G\left(f(a_{1}),\dots,f(a_{n})\right) = G\left(g(a_{1}),\dots,g(a_{n})\right)
\end{align*}
e quindi \(g \in \mathscr{F}_{G,m}\).
\end{itemize}

Siccome
\begin{equation*}
  \mathscr{F}_{G} = \bigcap_{m \in \omega} \mathscr{F}_{G,m}
\end{equation*}
allora \(\mathscr{F}_{G}\) è \(\bm{G}_{\delta}\).

\item Sia \(a \in \set{a_{k}\mid k<K}\). L'insieme
\begin{equation*}
  \mathscr{C}_{a} \coloneqq\set{f \in S_{\infty}\mid a = f(a)}
\end{equation*}
è aperto.

Infatti, per ogni \(f \in \mathscr{C}_{a}\), si ha che
\begin{equation*}
  f \in S_{\infty} \cap B_{d}(f, 2^{-(a+2)}) \eqqcolon V
\end{equation*}
ed inoltre \(V \subseteq \mathscr{C}_{a}\). Infatti, sia \(g \in V\). Allora \(d(f,g)<2^{-(a+2)}\) e quindi, se \(f\neq g\), allora
\begin{equation*}
  2^{-(n+1)} < 2^{-(a+2)} \, \implies\, n > a+1
\end{equation*}
dove \(n\) è il più piccolo naturale t.c. \(f(n)\neq g(n)\). Pertanto \(a=f(a)=g(a)\) e quindi \(g \in \mathscr{C}_{a}\).
\end{itemize}
\paragraph{\(\operatorname{Aut}(M)\) è sottogruppo di \(S_{\infty}\)}
\label{sec:orgaed68fa}

Visto che \(\operatorname{Aut}(M)\) è intersezione dei seguenti insiemi
\begin{itemize}
\item per ogni \(i<I\): \(\set{f \in S_{\infty}\mid (a_{1},\dots,a_{\operatorname{ar}(R_{i})}) \in R_{i}\text{ sse }\left(f(a_{1}),\dots,f(a_{\operatorname{ar}(R_{i})})\right) \in R}\);
\item per ogni \(j < J\): \(\set{f \in S_{\infty}\mid f\left(f_{j}(a_{1},\dots,a_{\operatorname{ar}(f_{j})})\right) = f_{j}\left(f(a_{1}),\dots,f(a_{\operatorname{ar}(f_{j})})\right)}\);
\item per ogni \(k<K\): \(\set{f \in S_{\infty}\mid f(a_{k}) = a_{k}}\);
\end{itemize}
è sufficiente mostrare che ciascuno di questi sia chiuso per composizione di funzioni e per inversa.

Consideriamo gli insiemi \(\mathscr{F}_{G}, \mathscr{R}_{R}, \mathscr{C}_{a}\) come sopra.
\begin{itemize}
\item Siano \(f,g \in \mathscr{F}_{G}\). Per ogni \(a_{1},\dots,a_{n} \in \omega\)
\begin{align*}
  	f\circ g \left(G(a_{1},\dots,a_{n})\right) &= f\left[g\left(G(a_{1},\dots,a_{n})\right)\right]\\
  	&= f\left[G\left(g(a_{1}),\dots,g(a_{n})\right)\right]\\
  	&= G\left(f\circ g(a_{1}),\dots,f\circ g(a_{n})\right)
        \end{align*}
e pertanto \(f\circ g \in \mathscr{F}_{G}\).
\item Sia \(f \in \mathscr{F}_{G}\). Per ogni \(a_{1},\dots,a_{n} \in \omega\)
\begin{align*}
  G(a_{1},\dots,a_{n}) &= G\left[f\circ f^{-1}(a_{1}),\dots,f\circ f^{-1}(a_{n})\right]\\
  &= f\left[G\left(f^{-1}(a_{1}),\dots,f^{-1}(a_{n})\right)\right]
\end{align*}
e quindi
\begin{equation*}
  f^{-1}\left[G(a_{1},\dots,a_{n})\right] = G\left(f^{-1}(a_{1}),\dots,f^{-1}(a_{n})\right).
\end{equation*}
Pertanto \(f^{-1} \in \mathscr{F}_{G}\).

\item Siano \(f,g \in \mathscr{R}_{R}\). Per ogni \(a_{1},\dots,a_{n} \in \omega\)
\begin{align*}
  (a_{1},\dots, a_{n}) \in R \,&\,\text{ sse }\, \left(g(a_{1}),\dots,g(a_{n})\right) \in R\\
  &\,\text{ sse }\,\left(f\circ g(a_{1}),\dots,f\circ g(a_{n})\right) \in R
\end{align*}
e pertanto \(f\circ g \in \mathscr{R}_{R}\).
\item Sia \(f \in \mathscr{R}_{R}\). Per ogni \(a_{1},\dots,a_{n} \in\omega\),
\begin{align*}
  (a_{1},\dots,a_{n}) \in R &\,\text{ sse }\, \left(f\circ f^{-1}(a_{1}),\dots,f\circ f^{-1}(a_{n})\right) \in R\\
  &\,\text{ sse }\, \left(f^{-1}(a_{1}),\dots,f^{-1}(a_{n})\right) \in R,
\end{align*}
e quindi \(f^{-1} \in \mathscr{R}_{R}\).

\item Siano \(f,g \in \mathscr{C}_{a}\). Allora \(f\circ g (a) = f\left(g(a)\right) = f(a) = a\), e quindi \(f\circ g \in \mathscr{C}_{a}\).
\item Sia \(f \in \mathscr{C}_{a}\). Allora \(f(a)=a\) e pertanto \(f^{-1}(a)=a\), quindi \(f^{-1} \in \mathscr{C}_{a}\).\qed
\end{itemize}
\subsection{Esercizio 4}
\label{sec:org809a06c}

Consider the \href{20250301194013-spazio_polacco.org}{Polish space} \(X = \omega^{\omega \times \omega} \times \omega\). Let \(\operatorname{Gp}\) be the space of those \((f,a) \in X\) such that \(\langle \omega, f, a \rangle\) is a \href{20241205141146-gruppo_abeliano.org}{group} with operation \(f\) and neutral element \(a\).
\begin{enumerate}
\item Prove that \(\operatorname{Gp}\) is a Polish subspace of \(X\).
\item Prove that the subspace of \(\operatorname{Gp}\) consisting of \href{20250127093245-gruppo_abeliano.org}{Abelian groups} is Polish, and similarly for the subspace of non-Abelian groups.
\item Prove that the subspace of \(\operatorname{Gp}\) consisting of \href{20250320150051-gruppo_archimedeo.org}{Archimedean groups} is Polish.
\end{enumerate}
\subsubsection{Soluzione}
\label{sec:org19c4c94}

\paragraph{Parte a.}
\label{sec:orgff788ba}
Posti:
\begin{align*}
A_{x,y,z} &= \set{(f,a) \in X\mid f\left(f(x,y),z\right) = f\left(x,f(y,z)\right)}\\
N_{x} &= \set{(f,a) \in X\mid f(x,a)=f(a,x) = a}\\
I_{x,y} &= \set{(f,a) \in X\mid f(x,y) = f(y,x) = a}\\
I_{x} &= \bigcup_{y \in \omega} I_{x,y}
\end{align*}
allora, per definizione di gruppo, si ha che:
\begin{equation*}
\operatorname{Gp} = \bigcap_{x,y,z \in\omega} A_{x,y,z} \cap \bigcap_{x \in \omega} {N_{x}}\cap \bigcap_{x \in \omega} I_{x}.
\end{equation*}

È dunque necessario dimostrare che ciascuno degli insiemi \(A_{x,y,z}, N_{x}, I_{x}\) siano degli \href{20250103145124-topologia.org}{aperti}, o almeno dei \(\bm{G}_{\delta}\). Questo implica che \(\operatorname{Gp}\) sia un \href{20250304152026-sottoinsiemi_gdelta_e_fsigma.org}{insieme \(\bm{G}_{\delta}\)} di \(\omega^{\omega\times \omega}\times \omega\) (che è un \href{20250301194013-spazio_polacco.org}{polacco}) \href{20250306134632-caratterizzazione_dei_sottoinsiemi_polacchi_di_uno_spazio_polacco.org}{e quindi} uno spazio polacco.

\begin{itemize}
\item \(A_{x,y,z}\) è aperto.

Siano, per ogni \(x,y,z,\mu,\lambda,\gamma \in\omega\):
\begin{equation*}
  A_{x,y,z}^{\mu,\lambda,\gamma} \coloneqq \set{
  	(f,a) \in X\mid f(x,y)=\mu, f(y,z) = \lambda, f(\mu, z) = \gamma, f(x,\lambda)=\gamma
  }
\end{equation*}
Allora \(A_{x,y,z} = \bigcup_{\mu,\lambda,\gamma} A_{x,y,z}^{\mu,\lambda,\gamma}\).

Inoltre, sia \((f,a) \in A_{x,y,z}^{\mu,\lambda,\gamma}\), e si consideri l'aperto \(U_{(f,a)} \subseteq \omega^{\omega\times\omega}\times\omega\)
\begin{equation*}
  U_{(f,a)} \coloneqq \prod_{(i,j) \in \omega\times\omega} U_{ij} \times V
\end{equation*}
con \(U_{xy} = \set{\mu}\), \(U_{yz} = \set{\lambda}\), \(U_{\mu z}=\set{\gamma} = U_{x\lambda}\) e \(V = \set{a}\), e tutti gli altri \(U_{ij} = \omega\).

Allora \((f,a) \in U_{(f,a)}\) e \(U_{(f,a)} \subseteq A_{x,y,z}^{\mu,\lambda,\gamma}\). Dunque \(A_{x,y,z}^{\mu,\lambda,\gamma}\) è \href{20250111142313-intorno.org}{intorno} di ogni suo punto, \href{20250317093153-insieme_aperto_sse_intorno_di_ogni_suo_punto.org}{e quindi} \href{20250103145124-topologia.org}{aperto}.

Segue che \(A_{x,y,z}\) è unione di aperti, e pertanto aperto.

\item \(N_{x}\) è aperto. Sia \((f,a) \in N_{x}\), e si consideri l'aperto \(U_{(f,a)} \subseteq \omega^{\omega\times\omega}\times\omega\):
\begin{equation*}
  U_{(f,a)} \coloneqq \prod_{(i,j) \in \omega\times\omega} U_{ij} \times V
\end{equation*}
con \(U_{ax} = U_{xa} = \set{x}\) e \(V = \set{a}\), e tutti gli altri \(U_{ij} = \omega\).

Allora \((f,a) \in U_{(f,a)}\) e \(U_{(f,a)} \subseteq N_{x}\). Dunque \(N_{x}\) è \href{20250111142313-intorno.org}{intorno} di ogni suo punto, \href{20250317093153-insieme_aperto_sse_intorno_di_ogni_suo_punto.org}{e quindi} \href{20250103145124-topologia.org}{aperto}.

\item \(I_{x}\) è aperto. Infatti \(I_{x,y}\) sono aperti, e unione numerabile di aperti è aperta. Sia \((f,a) \in I_{x,y}\) e si consideri l'aperto \(U_{(f,a)} \subseteq \omega^{\omega\times\omega}\times\omega\):
\begin{equation*}
  U_{(f,a)} \coloneqq \prod_{(i,j) \in \omega\times\omega} U_{ij} \times V
\end{equation*}
con \(V=U_{xy} = U_{yx} = \set{a}\), e tutti gli altri \(U_{ij} =\omega\).

Allora \((f,a) \in U_{(f,a)}\) e \(U_{(f,a)} \subseteq I_{x,y}\). Dunque \(I_{x,y}\) è \href{20250111142313-intorno.org}{intorno} di ogni suo punto, \href{20250317093153-insieme_aperto_sse_intorno_di_ogni_suo_punto.org}{e quindi} \href{20250103145124-topologia.org}{aperto}.
\end{itemize}
\paragraph{Parte b.}
\label{sec:org77c5066}

\((f,a) \in X\) descrive un gruppo abeliano se e solo se:
\begin{itemize}
\item \((f,a) \in \operatorname{Gp}\);
\item per ogni \(x,y \in \omega\), \(f(x,y) = f(y,x)\).
\end{itemize}

Si denoti con \(\operatorname{Ab}, \operatorname{NAb} \subseteq \operatorname{Gp}\) gli insiemi, rispettivamente, dei gruppi abeliani e dei gruppi non abeliani. Vale
\begin{equation*}
\operatorname{Ab} = \operatorname{Gp}\setminus \operatorname{NAb}
\end{equation*}

Sia quindi, per ogni \(x,y,\lambda,\mu \in\omega\):
\begin{equation*}
C_{x,y}^{\lambda,\mu} = \set{(f,a) \in X\mid f(x,y)=\lambda, f(y,x) = \mu}.
\end{equation*}
Questo è un aperto. Infatti, sia \((f,a) \in C_{x,y}^{\lambda,\mu}\) e si consideri l'aperto \(U_{(f,a)}\):
\begin{equation*}
	U_{(f,a)} \coloneqq \prod_{(i,j) \in \omega\times\omega} U_{ij} \times V
\end{equation*}
con \(U_{xy}=\set{\lambda}\), \(U_{yx} =\set{\mu}\) e con \(V=\set{a}\), e tutti gli altri \(U_{ij}=\omega\).

Allora \((f,a) \in U_{(f,a)}\) e \(U_{(f,a)} \subseteq C_{x,y}^{\lambda,\mu}\), quindi \(C_{x,y}^{\lambda,\mu}\) è \href{20250111142313-intorno.org}{intorno} di ogni suo punto, \href{20250317093153-insieme_aperto_sse_intorno_di_ogni_suo_punto.org}{e quindi} \href{20250103145124-topologia.org}{aperto}.

Si consideri ora l'aperto (in quanto unione di aperti):
\begin{equation*}
\operatorname{NAb}_{x,y} \coloneqq \bigcup_{\lambda\neq\mu} C_{x,y}^{\lambda,\mu} = \set{(f,a) \in X\mid f(x,y) \neq f(y,x)}.
\end{equation*}

L'insieme \(\operatorname{NAb} \subseteq \operatorname{Gp}\) è dato da
\begin{equation*}
\operatorname{NAb} = \operatorname{Gp}\cap \bigcup_{x,y \in \omega} \operatorname{NAb}_{x,y}
\end{equation*}
e quindi è un aperto in \(\operatorname{Gp}\). Pertanto, \(\operatorname{NAb}\) è uno spazio polacco. Inoltre \(\operatorname{Ab} = \operatorname{Gp}\setminus \operatorname{NAb}\) è un chiuso di \(\operatorname{Gp}\) e pertanto è uno spazio polacco.
\paragraph{Parte c.}
\label{sec:orgbd64860}
Si consideri ora lo spazio topologico \(Y \coloneqq X\times 2^{\omega\times \omega}\), e le due proiezioni continue per definizione di topologia prodotto:
\begin{align*}
\pi_{X}: Y&\to X\\
\pi_{2^{\omega\times\omega}} : Y &\to 2^{\omega\times\omega}
\end{align*}

Si consideri l'insieme \(\operatorname{OGp}\) delle triple \((f,a,\le)\) tali che \(\langle \omega,f,a,\le\rangle\) sia un \href{20250320184931-gruppo_ordinato.org}{gruppo ordinato}, e quindi \(\le\) un \href{20250203101604-ordine.org}{ordine totale}. Si dimostra che l'insieme seguente è uno spazio polacco:
\begin{equation*}
\operatorname{Ar} \coloneqq \set{(f,a, \le) \in \operatorname{OGp}\mid \forall\, x \in\omega \ \exists\, n \in \omega: x\le f^{(n)}(a)}
\end{equation*}
dove con \(f^{(n)}(a)\) si intende, per ricorsione:
\begin{align*}
f^{(2)} (a) &\coloneqq f(a,a)\\
f^{(n+1)}(a) &\coloneqq f\left(f^{(n)}(a), a\right).
\end{align*}

\begin{itemize}
\item \(\operatorname{OGp}\) è uno spazio polacco.

Infatti, si considerino gli insiemi, per \(x,y,z,\mu,\lambda,\delta,\gamma \in \omega\)
\begin{align*}
  G_{x,y,z}^{\mu,\lambda, \delta, \gamma} &\coloneqq \set{(f,a,\le) \in Y\mid f(x,z) = \mu \,\land\, f(y,z) = \lambda \,\land\, f(z,x) = \delta \,\land\, f(z,y) = \gamma \,\land\, \mu\le \lambda \,\land\, \delta\le \gamma}\\
  G_{x,y,z}^{2} &\coloneqq \set{(f,a,\le) \in Y\mid x\le y}
\end{align*}
Questi sono entrambi aperti.
\begin{itemize}
\item Se \((f,a,\le) \in G_{x,y,z}^{\mu,\lambda,\delta,\gamma}\), sia \(U_{(f,a,\le)}\) un aperto,
\begin{equation*}
U_{(f,a,\le)} \coloneqq \prod_{(i,j) \in \omega\times\omega} U_{ij} \times \tilde{U} \times \prod_{(s,t) \in\omega\times\omega} W_{st}
\end{equation*}
con \(U_{xz}=\set{\mu}\), \(U_{yz} = \set{\lambda}\), \(U_{zx} = \set{\delta}\), \(U_{zy} = \set{\gamma}\), \(W_{\mu\lambda} = \set{1}\) e infine \(W_{\delta\gamma}=\set{1}\). Tutti gli altri \(U_{ij} = \omega\), \(\tilde{U} = \omega\) e i restanti \(W_{st} = 2\).

Allora \((f,a,\le) \in U_{(f,a,\le)}\), ed inoltre \(U_{(f,a,\le)} \subseteq G_{x,y,z}^{\mu,\lambda,\delta,\gamma}\). Quindi \(G_{x,y,z}^{\mu,\lambda,\delta,\gamma}\) è \href{20250111142313-intorno.org}{intorno} di ogni suo punto, \href{20250317093153-insieme_aperto_sse_intorno_di_ogni_suo_punto.org}{e pertanto} \href{20250103145124-topologia.org}{aperto}.
\item Se \((f,a,\le) \in G^{2}_{x,y,z}\), sia \(V_{(f,a,\le)}\) un aperto,
\begin{equation*}
V_{(f,a,\le)} \coloneqq \prod_{(i,j) \in\omega\times\omega} V_{ij}\times \tilde{V} \times \prod_{(s,t) \in \omega\times\omega} W_{st}
\end{equation*}
con \(\tilde{V}= V_{ij} = \omega\) per ogni \(i,j\), e con \(W_{xy} =\set{1}\). Per tutti gli altri \(W_{st} =2\).

Allora \((f,a,\le) \in V_{(f,a,\le)}\), ed inoltre \(V_{(f,a,\le)} \subseteq G_{x,y,z}^{2}\). Quindi \(G_{x,y,z}^{2}\) è \href{20250111142313-intorno.org}{intorno} di ogni suo punto, \href{20250317093153-insieme_aperto_sse_intorno_di_ogni_suo_punto.org}{e pertanto} \href{20250103145124-topologia.org}{aperto}.
\end{itemize}

È inoltre aperto l'insieme \(G_{x,y,z}^{1} \coloneqq \set{(f,a,\le) \in Y\mid f(x,z)\le f(y,z) \,\land\, f(z,x)\le f(z,y)}\), in quanto
\begin{equation*}
  G^{1}_{x,y,z} = \bigcup_{\mu,\lambda,\delta,\gamma \in \omega} G_{x,y,z}^{\mu,\lambda,\delta,\gamma}.
\end{equation*}

Segue che l'insieme \(G_{x,y,z} \coloneqq G^{1}_{x,y,z} \cup \left[Y\setminus G_{x,y,z}^{2}\right]\) sia un \(\bm{G}_{\delta}\), in quanto unione di un aperto e di un chiuso (i chiusi negli spazi polacchi sono \(\bm{G}_{\delta}\)).

Inoltre, si è dimostrato che \(\operatorname{Gp} \subseteq X\) è un insieme \(\bm{G}_{\delta}\) (nel punto precedente) e che l'insieme degli ordini \(\operatorname{LO} \subseteq 2^{\omega\times\omega}\) è un insieme \(\bm{G}_{\delta}\) (a lezione). Pertanto, siccome la preimmagine continua di \(\bm{G}_{\delta}\) è ancora \(\bm{G}_{\delta}\), si ha che sono \(\bm{G}_{\delta}\) di \(Y\) i seguenti insiemi:
\begin{equation*}
  \pi_{X}^{-1}(\operatorname{Gp}),\quad \pi_{2^{\omega\times\omega}}^{-1} (\operatorname{LO}).
\end{equation*}

Siccome vale questa uguaglianza
\begin{equation*}
  \operatorname{OGp} = \pi_{X}^{-1}(\operatorname{Gp}) \cap \pi_{2^{\omega\times\omega}}^{-1} (\operatorname{LO}) \cap \bigcap_{x,y,z \in \omega} G_{x,y,z}
\end{equation*}
si è scritto \(\operatorname{OGp}\) come intersezione numerabile di \(\bm{G}_{\delta}\).Quindi \(\operatorname{OGp}\) è uno spazio polacco, in quanto sottoinsieme \(\bm{G}_{\delta}\) dello spazio polacco \(Y\).

\item L'insieme:
\begin{equation*}
\tilde{\operatorname{Ar}}_{x}^{n} \coloneqq \set{(f,a, \le) \in Y\mid x\le f^{(n)}(a)}
\end{equation*}
è aperto. Infatti, si consideri
\begin{equation*}
  \Gamma_{x,n}^{\lambda_{2},\dots,\lambda_{n}} \coloneqq \set{
  	(f,a,\le) \in Y\mid f(a,a) = \lambda_{2} \,\land\, f(\lambda_{2},a) = \lambda_{3} \,\land\, \dots \,\land\, f(\lambda_{n-1}, a) = \lambda_{n} \,\land\, x\le \lambda_{n}
  }
\end{equation*}
Ciascun \(\Gamma_{x,n}^{\lambda_{2},\dots,\lambda_{n}}\) è aperto, poiché, presa \((f,a,\le) \in \Gamma_{x,n}^{\lambda_{2},\dots,\lambda_{n}}\), è possibile prendere l'aperto \(U_{(f,a,\le)}\):
\begin{equation*}
  U_{(f,a,\le)} \coloneqq \prod_{(i,j) \in \omega\times\omega} U_{ij} \times \tilde{U} \times \prod_{(s,t) \in\omega\times\omega} W_{st}
\end{equation*}
con \(\tilde{U} = \set{a}\), \(U_{aa}=\lambda_{2}\), per ogni \(i=2,\dots,n-1\): \(U_{\lambda_{i}a} = \set{\lambda_{i+1}}\), e \(W_{x\lambda_{n}} = \set{1}\). Si pongono tutti gli altri \(U_{ij} = \omega\)  e \(W_{st} = 2\). Si ha quindi che \((f,a,\le) \in U_{(f,a,\le)} \subseteq \Gamma_{x,n}^{\lambda_{2},\dots,\lambda_{n}}\), e pertanto \(\Gamma_{x,n}^{\lambda_{2},\dots,\lambda_{n}}\) è \href{20250111142313-intorno.org}{intorno} di ogni suo punto \href{20250317093153-insieme_aperto_sse_intorno_di_ogni_suo_punto.org}{e quindi} \href{20250103145124-topologia.org}{aperto}.

Dunque \(\tilde{\operatorname{Ar}}_{x}^{n} = \bigcup_{\lambda_{2},\dots,\lambda_{n} \in \omega} \Gamma_{x,n}^{\lambda_{2},\dots,\lambda_{n}}\) è a sua volta aperto.
\end{itemize}

Allora \(\operatorname{Ar}_{x}^{n} = \tilde{\operatorname{Ar}}_{x}^{n} \cap \operatorname{OGp}\) è aperto in \(\operatorname{OGp}\). Quindi
\begin{equation*}
\operatorname{Ar} = \bigcap_{x \in \omega}\bigcup_{n \in \omega} \operatorname{Ar}_{x}^{n}
\end{equation*}
è un \href{20250304152026-sottoinsiemi_gdelta_e_fsigma.org}{insieme \(\bm{G}_{\delta}\)} di \(\operatorname{OGp}\), \href{20250306134632-caratterizzazione_dei_sottoinsiemi_polacchi_di_uno_spazio_polacco.org}{e quindi} uno \href{20250301194013-spazio_polacco.org}{spazio polacco}.\qed
\end{document}
