% Created 2026-02-07 Sat 19:33
% Intended LaTeX compiler: pdflatex
\documentclass[10pt]{article}
%% CREATO CON ORG - EMACS
\newcommand{\use}[2][]{\usepackage[#1]{#2}}
% PACCHETTI FONDAMENTLAI
\use[utf8]{inputenc}
\use[T1]{fontenc}
\use{graphicx}
\use{longtable}
\use{wrapfig}
\use{rotating}
\use[normalem]{ulem}
\use{amsmath}
\use{amsthm}
\use{amssymb}

\use{eucal} % Cambia mathcal{...}

\use{capt-of}
\use[italian]{babel}
\use[babel]{csquotes}
% bib la TEX lo carica in automatico org-cite
\use{microtype}
\use{lmodern}
\use{subfig} % sottofigure
\use{multicol} % due colonne
\use{lipsum} % lorem ipsum
\use{color} % colori in latex
\use{parskip} % rimuove l'indentazione dei nuovi paragrafi %% Add parbox=false to all new tcolorbox
\use{centernot}
\use[outline]{contour}\contourlength{3pt}
\use{fancyhdr}
\use{layout}
\use[most]{tcolorbox} % Riquadri colorati
\use{ifthen} % IFTHEN
\use{geometry}

% pacchetti matematica
\use{yhmath}
\use{dsfont}
\use{mathrsfs}
\use{cancel} % semplificare
\use{polynom} %divisione tra polinomi
\use{forest} % grafi ad albero
\use{booktabs} % tabelle
\use{commath} %simboli e differenziali
\use{bm} %bold
\use[fulladjust]{marginnote} %to use marginnote for date notes
\use{arrayjobx}%array
\use[intlimits]{empheq} % Riquadri colorati attorno alle equazioni
\use{mathtools}
\use{circuitikz} % Disegnare i circuiti
\use{mathtools}
\use{stmaryrd} % [[ \llbracket ]] \rrbracket
\use{bussproofs} % dimostrazioni

%%%%%%%%%%%%%


%%%% QUIVER
\newcommand{\duepunti}{\,\mathchar\numexpr"6000+`:\relax\,}
% A TikZ style for curved arrows of a fixed height, due to AndréC.
\tikzset{curve/.style={settings={#1},to path={(\tikztostart)
    .. controls ($(\tikztostart)!\pv{pos}!(\tikztotarget)!\pv{height}!270:(\tikztotarget)$)
    and ($(\tikztostart)!1-\pv{pos}!(\tikztotarget)!\pv{height}!270:(\tikztotarget)$)
    .. (\tikztotarget)\tikztonodes}},
    settings/.code={\tikzset{quiver/.cd,#1}
        \def\pv##1{\pgfkeysvalueof{/tikz/quiver/##1}}},
    quiver/.cd,pos/.initial=0.35,height/.initial=0}

% TikZ arrowhead/tail styles.
\tikzset{tail reversed/.code={\pgfsetarrowsstart{tikzcd to}}}
\tikzset{2tail/.code={\pgfsetarrowsstart{Implies[reversed]}}}
\tikzset{2tail reversed/.code={\pgfsetarrowsstart{Implies}}}
% TikZ arrow styles.
\tikzset{no body/.style={/tikz/dash pattern=on 0 off 1mm}}
%%%%%%%%%%


%% DEFINIZIONI COMANDI MATEMATICI
\let\sin\relax %TOGLIE LA DEFINIZIONE SU "\sin"

% cambia la definizione di empty set
% ---
\let\oldemptyset\emptyset
% ---
% \let\emptyset\varnothing
% ---
% \let\emptyset\relax
% \newcommand{\emptyset}{\text{\textnormal{\O}}}
% ---

\DeclareMathOperator{\bounded}{bd}
\DeclareMathOperator{\sin}{sen}
\DeclareMathOperator{\epi}{Epi}
\DeclareMathOperator{\cl}{cl}
\DeclareMathOperator{\graph}{graph}
\DeclareMathOperator{\arcsec}{arcsec}
\DeclareMathOperator{\arccot}{arccot}
\DeclareMathOperator{\arccsc}{arccsc}
\DeclareMathOperator{\spettro}{Spettro}
\DeclareMathOperator{\nulls}{nullspace}
\DeclareMathOperator{\dom}{dom}
\DeclareMathOperator{\ar}{ar}
\DeclareMathOperator{\const}{Const}
\DeclareMathOperator{\fun}{Fun}
\DeclareMathOperator{\rel}{Rel}
\DeclareMathOperator{\altezza}{ht}
\let\det\relax %TOGLIE LA DEFINIZIONE SU "\det"
\DeclareMathOperator{\det}{det}
\DeclareMathOperator{\End}{End}
\DeclareMathOperator{\gl}{GL}
\def\Id{\mathrm{Id}}
\def\id{\mathrm{id}}
\DeclareMathOperator{\I}{\mathds{1}}
\DeclareMathOperator{\II}{II}
\DeclareMathOperator{\rank}{rank}
\DeclareMathOperator{\tr}{tr}
\DeclareMathOperator{\tc}{t.c.}
\DeclareMathOperator{\T}{T}
\DeclareMathOperator{\var}{Var}
\DeclareMathOperator{\cov}{Cov}
\DeclareMathOperator{\st}{st}
\DeclareMathOperator{\mon}{Mon}
\newcommand{\card}[1]{\left\vert #1 \right\vert}
\newcommand{\trasposta}[1]{\prescript{\text{T}}{}{#1}}
\newcommand{\1}{\mathds{1}}
\newcommand{\R}{\mathds{R}}
\newcommand{\diesis}{\#}
\newcommand{\bemolle}{\flat}
\newcommand{\nonstandard}[1]{\prescript{*}{}{#1}}
\newcommand{\starR}{\nonstandard{\R}}
\newcommand{\borel}{\mathscr{B}}
\newcommand{\lebesgue}[1]{\mathscr{L}\left(#1\right)}
\newcommand{\media}{\mathds{E}}
\newcommand{\K}{\mathds{K}}
\newcommand{\A}{\mathds{A}}
\newcommand{\Q}{\mathds{Q}}
\newcommand{\N}{\mathds{N}}
\newcommand{\C}{\mathds{C}}
\newcommand{\Z}{\mathds{Z}}
\newcommand{\qo}{\hspace{1em}\text{q.o.}\,}
\renewcommand{\tilde}[1]{\widetilde{#1}}
\renewcommand{\parallel}{\mathrel{/\mkern-5mu/}}
\newcommand{\parti}[2][]{\wp_{#1}(#2)}
\newcommand{\diff}[1]{\operatorname{d}_{#1}}
\let\oldvec\vec
\renewcommand{\vec}[1]{\overrightarrow{\vphantom{i}#1}}
\newcommand{\floor}[1]{\left\lfloor #1 \right\rfloor}
\newcommand{\cat}[1]{\mathbf{#1}}
\newcommand{\dfreccia}[1]{\xrightarrow{\ #1 \ }}
\newcommand{\sfreccia}[1]{\xleftarrow{\ #1 \ }}
\newcommand{\formalsum}[2]{{\sum_{#1}^{#2}}{\vphantom{\sum}}'}
\newcommand{\minim}[2]{\mu_{#1}\, \left(#2\right)}
\newcommand{\concat}{\null^{\frown}} % concatenazione di stringe
\newcommand{\godelcode}[1]{\langle\!\langle #1 \rangle\!\rangle}
\newcommand{\godeldec}[1]{(\!(#1)\!)}
\newcommand{\termcode}[1]{\ulcorner #1\urcorner}
\newcommand{\partialto}{\dashrightarrow}
\newcommand{\restricted}{\upharpoonright}
\newcommand{\embeds}{\precsim}
\newcommand{\surjects}{\twoheadrightarrow}
\newcommand{\equipotenti}{\asymp}
%% \newcommand{\dotplus}{\mathbin{\dot{+}}} %% A quanto pare esiste già
\newcommand{\bigdot}{\mathbin{\boldsymbol{\cdot}}}
\newcommand{\dotexp}[1]{^{.#1}}
\newcommand{\conv}{\mathbin{*}}
\newcommand{\convolution}[2]{(#1\conv #2)}
\newcommand{\nil}{\mathfrak{N}}
\newcommand{\divisore}{\mathrel{|}}
\newcommand{\simplesso}[1]{\mathrm{e}_{#1}}

\renewcommand{\iff}{\mathrel{\longleftrightarrow}} %% Notazione Logica.
\newcommand{\oldiff}{\mathrel{\Longleftrightarrow}}
\renewcommand{\implies}{\mathrel{\rightarrow}} %% Notazione Logica
\newcommand{\oldimplies}{\mathrel{\Longrightarrow}}
\renewcommand{\impliedby}{\mathrel{\leftarrow}} %% Notazione Logica
\newcommand{\oldimpliedby}{\mathrel{\Longleftarrow}}

\newcommand{\IFF}{\quad\Longleftrightarrow\quad}
\newcommand{\IMPLICA}{\quad\Longrightarrow\quad}


\renewcommand{\descriptionlabel}[1]{\hspace{\labelsep}\normalfont #1} % remove bold from description


%% Definizione di Divergenza di K-L

\DeclarePairedDelimiterX{\infdivx}[2]{(}{)}{%
  #1\;\delimsize\|\;#2%
}
\newcommand{\kldiv}{D_{KL}\infdivx}

%% Definizione di \dotminus

\makeatletter
\newcommand{\dotminus}{\mathbin{\text{\@dotminus}}}

\newcommand{\@dotminus}{%
  \ooalign{\hidewidth\raise1ex\hbox{.}\hidewidth\cr$\m@th-$\cr}%
}
\makeatother

%tramite i prossimi due comandi posso decidere come scrivere i logaritmi naturali in tutti i documenti: ho infatti eliminato qualsiasi differenza tra "ln" e "log": se si vuole qualcosa di diverso bisogna inserire manualmente il tutto
\let\ln\relax
\DeclareMathOperator{\ln}{ln}
\let\log\relax
\DeclareMathOperator{\log}{log}
%%%%%%

%% NUOVI COMANDI
\newcommand{\straniero}[1]{\textit{#1}} %parole straniere
\newcommand{\titolo}[1]{\textsc{#1}} %titoli
\newcommand{\qedd}{\tag*{$\blacksquare$}} %qed per ambienti matemastici
\renewcommand{\qedsymbol}{$\blacksquare$} %modifica colore qed
\newcommand{\ooverline}[1]{\overline{\overline{#1}}}
\newcommand{\circoletto}[1]{\left(#1\right)^{\text{o}}}
%
\newcommand{\qmatrice}[1]{\begin{pmatrix}
#1_{11} & \cdots & #1_{1n}\\
\vdots & \ddots & \vdots \\
#1_{m1} & \cdots & #1_{mn}
\end{pmatrix}}
%
\newcommand{\parentesi}[2]{%
\underset{#1}{\underbrace{#2}}%
}
%
\newcommand{\norma}[1]{% Norma
\left\lVert#1\right\rVert%
}
\newcommand{\scalare}[2]{% Scalare
\left\langle #1, #2\right\rangle
}
%%%%%

%% RESTRIZIONI
\newcommand{\referenze}[2]{
        \phantomsection{}#2\textsuperscript{\textcolor{blue}{\textbf{#1}}}
}

\let\restriction\relax

\def\restriction#1#2{\mathchoice
              {\setbox1\hbox{${\displaystyle #1}_{\scriptstyle #2}$}
              \restrictionaux{#1}{#2}}
              {\setbox1\hbox{${\textstyle #1}_{\scriptstyle #2}$}
              \restrictionaux{#1}{#2}}
              {\setbox1\hbox{${\scriptstyle #1}_{\scriptscriptstyle #2}$}
              \restrictionaux{#1}{#2}}
              {\setbox1\hbox{${\scriptscriptstyle #1}_{\scriptscriptstyle #2}$}
              \restrictionaux{#1}{#2}}}
\def\restrictionaux#1#2{{#1\,\smash{\vrule height .8\ht1 depth .85\dp1}}_{\,#2}}
%%%%%%%%%%%

%%% FORMATTAZIONE FOOTNOTEMARK

\def\footnotemarkformatting#1{[#1]}
\renewcommand{\thefootnote}{\footnotemarkformatting{\arabic{footnote}}}

%% SEZIONE GRAFICA
\use{tikz}
\usetikzlibrary{matrix, patterns, calc, decorations.pathreplacing, hobby, decorations.markings, decorations.pathmorphing, babel}
\use{tikz-3dplot}
\use{mathrsfs} %per geogebra
\use{tikz-cd}
\tikzset
{
  %surface/.style={fill=black!10, shading=ball,fill opacity=0.4},
  plane/.style={black,pattern=north east lines},
  curve/.style={black,line width=0.5mm},
  dritto/.style={decoration={markings,mark=at position 0.5 with {\arrow{Stealth}}}, postaction=decorate},
  rovescio/.style={decoration={markings,mark=at position 0.5 with {\arrow{Stealth[reversed]}}}, postaction=decorate}
}
\use{pgfplots} % stampare le funzioni
        \pgfplotsset{/pgf/number format/use comma,compat=1.15}
        %\pgfplotsset{compat=1.15} %per geogebra
        \usepgfplotslibrary{fillbetween, polar}
%%%%%%

%% CITAZIONI
\use{lineno}

\newcommand{\citazione}[1]{%
  \begin{quotation}
  \begin{linenumbers}
  \modulolinenumbers[5]
  \begingroup
  \setlength{\parindent}{0cm}
  \noindent #1
  \endgroup
  \end{linenumbers}
  \end{quotation}\setcounter{linenumber}{1}
  }
%%%%%%

%%%%%%%%%%%%%%%%%%%%%%%%%%%%%%%%%%%%%%%%%%%%
%%%%%%%%%%%%%%%%%%%%%%%%%%%%%%%%%%%%%%%%%%%%

%% AMS THM

\theoremstyle{definition}% default
\newtheorem{thm}{Teorema}[section]
\newtheorem{lem}[thm]{Lemma}
\newtheorem{prop}[thm]{Proposizione}
\newtheorem{cor}[thm]{Corollario}
\newtheorem{esempio}[thm]{Esempio}
\theoremstyle{plain}
\newtheorem{definizione}[thm]{Definizione}
\theoremstyle{remark}
\newtheorem*{oss}{Osservazione}


%%%%%%%%%%%%%%%%%%%%%%%%%%%%%%%%%%%%%%%%%%%%
%%%%%%%%%%%%%%%%%%%%%%%%%%%%%%%%%%%%%%%%%%%%

\use{hyperref}
\hypersetup{%
        pdfauthor={Davide Peccioli},
        pdfsubject={},
        allcolors=black,
        citecolor=black,
%	colorlinks=true,
        bookmarksopen=true}
\setcounter{secnumdepth}{0} % rimuove i numeri di sezione senza rimuovere le ref
\renewcommand{\href}[2]{\textcolor{blue}{#2}} % disabilita il comando href
\use{enotez} %
\setenotez{%
 mark-format = \footnotemarkformatting % Mette i numeri tra parentesi quadre%
}\let\footnote=\endnote % rende tutte le note a pié pagina come delle note a fine file 


\let\olddocument\document % modifico l'ambiende documenti per non dover stampare \printendnote
\let\oldenddocument\enddocument
\renewenvironment{document}%
{%
  \olddocument
}{%
  \printendnotes\oldenddocument
}
\renewcommand{\thethm}{\arabic{thm}}

\usepackage[hyperref]{biblatex}
\addbibresource{~/Documents/org/roam/bib/master.bib}
\author{Davide Peccioli}
\date{\today}
\title{Stile - Scala di Formalità}
\begin{document}

\href{20251111165459-stile\_scala\_di\_formalita.html}{Vedi HTML}
\href{20250515141706-da_finire.org}{DA FINIRE}

Prompt:

CONTESTO: Eleganza classica maschile

Ho bisogno di ordinare in base alla formalità diversi outfit. Suddividi
\begin{itemize}
\item gli abiti in 4 o 5 livelli di formalità, spiegando come riconoscerli;
\item le giacche in 4 o 5 livelli di formalità, spiegando come riconoscerle;
\item i pantaloni in 2 o 3 livelli di formalità, spiegando come riconoscerli
\end{itemize}

Dopodiché, ordina le diverse combinazioni possibili dalla più formale alla meno formale, includendo le variazioni date da:
\begin{itemize}
\item dolcevita sottile / camicia sotto
\item gilet in maglia o meno
\item colori
\end{itemize}

Dopodiché a questa lista aggiungi, nel corretto livello di formalità, outfit senza giacca, con maglioni, cardigan e dolcevita spessi
\section{Catalogazione dei capi formalità – Città}
\label{sec:orgfd01ea9}

\begin{center}
\begin{tabular}{rllll}
Livello & Tipo di outfit & Caratteristiche & Accessori consentiti & Situazione ideale\\
\hline
1 & Abito formale liscio & Worsted liscio, zero pattern & Cravatta sobria, gilet fine & Eventi accademici, conferenze formali\\
2 & Abito formale liscio + dolcevita & Worsted liscio, look serale & Nessuno & Serate formali senza cravatta\\
3 & Abito formale con pattern leggerissimo & Micro-pattern molto tenue & Cravatta semplice & Presentazioni importanti\\
4 & Abito semiformale liscio & Flanella leggera o worsted meno liscio & Cravatta & Giornate accademiche eleganti\\
5 & Abito semiformale liscio + dolcevita & Liscio ma non istituzionale & Nessuno & Sera urbana d’inverno\\
6 & Abito semiformale con micro-pattern & Gessato fine, micro-check & Cravatta semplice & Ospiti esterni, pomeriggi eleganti\\
7 & Abito semiformale con micro-pattern + dolcevita & Pattern tenue & Nessuno & Cene eleganti sobrie\\
8 & Abito medio con texture leggera & Flanella fine, lana “dry” & Cravatta, gilet & Lezioni importanti\\
14 & Spezzato formale & Giacca liscia, pantalone formale & Cravatta, gilet & Versatilità urbana\\
10 & Abito medio pattern evidente & POW moderato, windowpane fine & Cravatta scura & Semi-formale urbano\\
9 & Abito medio texture leggera + dolcevita & Texture urbana sottile & Nessuno & Inverno urbano elegante\\
11 & Abito medio pattern evidente + dolcevita & Pattern medio, look intellettuale & Nessuno & Pomeriggio/sera invernale\\
15 & Spezzato formale + dolcevita & Look formale ma moderno & Nessuno & Sera urbana elegante\\
16 & Giacca formale + chino & Spezzato urbano pulito & Cravatta & Lezioni normali\\
18 & Giacca media + pantaloni formali & Hopsack, twill, flanella leggera & Cravatta & Università quotidiana\\
17 & Giacca formale + chino + dolcevita & Moderno, leggermente ridotto in formalità & Nessuno & Giornate fredde\\
19 & Giacca media + pantaloni formali + dolcevita & Texture minima & Nessuno & Sera invernale\\
20 & Giacca media + chino & Smart casual urbano & Cravatta opzionale & Giornate accademiche normali\\
21 & Giacca media + chino + dolcevita & Ottimo mix urbano & Nessuno & Pomeriggi e studio\\
22 & Giacca media + velluto fine & Texture moderata & Nessun gilet & Inverno, aspetto accademico\\
23 & Giacca media + velluto fine + dolcevita & Texture + maglia & Nessuno & Look professorale urbano\\
24 & Giacca informale (tweed/panno) + pantaloni formali & Rustico-urbano bilanciato & Cravatta scura & Inverno accademico\\
25 & Giacca informale + pantaloni formali + dolcevita & Accademico moderno & Nessuno & Mattina/pomeriggio invernale\\
26 & Giacca informale + chino & Casual urbano composto & Nessun gilet & Giorni rilassati\\
28 & Giacca informale + velluto & Urban-country & Nessun gilet & Giornate fredde\\
27 & Giacca informale + chino + dolcevita & Smart winter casual & Nessuno & Studio e città\\
29 & Giacca informale + velluto + dolcevita & Look naturale e ricco & Nessuno & Inverno completo\\
30 & Giacca informale + jeans & Casual curato & Nessuno & Campus, spostamenti\\
31 & Giacca informale + jeans + dolcevita & Casual invernale & Nessuno & Vita quotidiana\\
36 & Maglione texturizzato + chino & Casual con texture & Nessuno & Studio pomeridiano\\
37 & Maglione texturizzato + jeans & Molto casual & Nessuno & Informale\\
\hline
\end{tabular}
\end{center}
\section{Scala di formalità – Campagna}
\label{sec:org40b7a3f}

\begin{center}
\begin{tabular}{rllll}
Livello & Tipo di outfit & Caratteristiche & Accessori consentiti & Situazione ideale\\
\hline
1 & Abito formale liscio + camicia & Worsteds lisci & Cravatta sobria & Eventi molto formali\\
2 & Abito semiformale liscio + camicia & Flanella fine, worsted poco testurizzata & Cravatta & Eventi diurni importanti\\
3 & Abito semiformale liscio + dolcevita fine & Liscio ma non urbano & Nessuno & Sera invernale\\
4 & Abito medio texturizzato + camicia & Flanella, lana dry & Cravatta regimental & Eleganza country classica\\
5 & Abito medio pattern evidente + camicia & POW moderato & Cravatta scura & Eventi rurali eleganti\\
6 & Abito medio pattern evidente + dolcevita & Pattern + dolcevita & Nessuno & Inverno rurale\\
7 & Abito informale in tweed fine + camicia & Texture marcata & Cravatta tweed & Perfetto in campagna\\
8 & Abito informale in tweed fine + dolcevita & Tweed + maglia & Nessuno & Giornate fredde\\
9 & Abito informale in flanellone/tweed spesso + camicia & Ruralità & Nessun gilet & Attività accademiche in contesti naturali\\
10 & Abito informale spesso + dolcevita & Massima coerenza country & Nessuno & Prof in campagna\\
11 & Giacca media + pantaloni formali + camicia & Hopsack o flanella leggera & Cravatta & Semi-formale rurale\\
12 & Giacca media + pantaloni formali + dolcevita & Texture + maglia & Nessuno & Pomeriggi freddi\\
13 & Giacca media + velluto + camicia & Country-chic & Cravatta opzionale & Autunno/inverno\\
14 & Giacca media + velluto + dolcevita & Molto naturale & Nessuno & Inverno pieno\\
15 & Giacca media + chino + camicia & Casual rurale elegante & Nessun gilet & Attività diurne\\
16 & Giacca media + chino + dolcevita & Perfetto equilibrio & Nessuno & Quotidianità invernale\\
17 & Giacca informale (tweed/panno) + pantaloni formali + camicia & British country & Cravatta tweed & Seminari in campagna\\
18 & Giacca informale + pantaloni formali + dolcevita & Country-intellettuale & Nessuno & Lezioni in contesti rurali\\
19 & Giacca informale + chino + camicia & Rustico ma composto & Nessuno & Routine giornaliera\\
20 & Giacca informale + chino + dolcevita & Più tecnico & Nessuno & Inverno\\
21 & Giacca informale + velluto + camicia & Molto country & Nessuno & Escursioni accademiche\\
22 & Giacca informale + velluto + dolcevita & Massima coerenza rurale & Nessuno & Giornate fredde\\
23 & Giacca informale + jeans + camicia & Casual rustico & Nessuno & Vita quotidiana\\
24 & Giacca informale + jeans + dolcevita & Casual invernale & Nessuno & Giorni informali\\
25 & Maglione fine + pantaloni formali & Pulito e pratico & Nessuno & Mattine fresche\\
26 & Maglione fine + chino & Casual elegante & Nessuno & Studio e lezioni\\
27 & Dolcevita fine + chino & Elegante-casual rurale & Nessuno & Inverno\\
28 & Dolcevita fine + jeans & Casual & Nessuno & Quotidianità informale\\
29 & Maglione texturizzato (shetland) + chino & Country classico & Nessuno & Campus rurali\\
30 & Maglione texturizzato + jeans & Molto casual rurale & Nessuno & Weekend, attività pratiche\\
\end{tabular}
\end{center}
\section{Scaletta CITTÀ}
\label{sec:orgae2adaf}

\begin{center}
\begin{tabular}{rllll}
Livello & Tipo di outfit & Caratteristiche & Accessori consentiti & Situazione ideale\\
\hline
1 & Abito liscio & Tessuto liscio, zero texture, zero pattern & Cravatta sobria, gilet fine & Eventi accademici molto formali, conferenze importanti\\
2 & Abito liscio + dolcevita fine & Elegante serale urbano & No gilet, dolcevita scuro & Serate formali senza cravatta\\
3 & Abito liscio flanella fine & Texture minima, molto elegante & Cravatta, gilet fine & Lezioni importanti, incontri accademici seri\\
4 & Spezzato formale (giacca liscia + pantalone formale) & Quasi formale, molto urbano & Gilet fine, cravatta sobria & Ricevimento studenti, eventi semi-istituzionali\\
5 & Abito liscio flanella fine + dolcevita & Elegante ma meno formale del worsted & No gilet & Sera d’inverno, eventi urbani semiformali\\
6 & Abito con micro-pattern (gessato fine, micro-check) & Pattern basso contrasto & Cravatta semplice, gilet fine & Presentazioni, ospiti esterni, pomeriggi eleganti\\
7 & Spezzato formale + dolcevita & Elegante ma non “istituzionale” & No gilet & Sera urbana elegante\\
8 & Abito con micro-pattern + dolcevita & Pattern tenue + sera urbana & Nessun gilet & Serate eleganti, cene sobrie\\
9 & Spezzato con giacca texturizzata + pantalone formale & Texture moderata, ma ancora serio & Gilet neutro, cravatta regimental & Università, biblioteche, seminari\\
10 & Abito con pattern medio (POW moderato, micro-windowpane fine) & Pattern evidente ma non informale & Cravatta scura, gilet neutro & Lezioni, dipartimento, eventi semi-formali\\
11 & Spezzato con giacca texturizzata + pantalone formale + dolcevita & Buona coesione urbana & Nessun gilet & Inverno, giornata intera in città\\
12 & Abito con pattern medio + dolcevita & Riduce il rumore, vibe urbano & Nessun gilet & Pomeriggi/serate invernali urbane\\
13 & Abito con texture marcata (flanellone, tweed leggero, donegal fine) & Aspetto accademico classico & Gilet scurissimo, cravatta testurizzata & Giornate in università, ambiente accademico classico\\
14 & Spezzato con giacca con pattern + pantalone formale & Pattern controllato & Cravatta semplice, gilet scuro & Lezioni, giornate accademiche tipiche\\
15 & Abito con texture marcata + dolcevita & Accademico urbano elegante & Nessun gilet & Inverno, lezioni, conferenze non formali\\
16 & Spezzato con giacca con pattern + pantalone formale + dolcevita & Ottima coesione con tonalità scure & Nessun gilet & Pomeriggi urbani\\
17 & Abito con pattern forte (POW grande, windowpane evidente) & Informalità urbana elevata & Cravatta tinta unita scura & Lezioni regolari, incontri informali\\
18 & Abito con pattern forte + dolcevita & Molto accademico, poco formale & Nessun gilet & Mattina/pomeriggio invernale, città\\
19 & Spezzato chino + giacca elegante liscia & Casual elegante urbano & Cravatta possibile, gilet sconsigliato & Giornate universitarie normali, città\\
20 & Spezzato chino + giacca elegante liscia + dolcevita & Elegante ma rilassato & Nessun gilet & Inverno, lezioni non impegnative\\
21 & Spezzato chino + giacca texturizzata & Texture marcata, casual urbano & Cravatta opzionale, gilet fine & Look accademico urbano informale\\
22 & Spezzato chino + giacca texturizzata + dolcevita & Ottima combinazione urbana invernale & Nessun gilet & Pomeriggi, città, studio\\
23 & Maglione standalone (merino/cashmere fine) & Eleganza minima ma pulita & Cravatta solo con girocollo fine & Studio, biblioteca, mattine leggere\\
24 & Dolcevita standalone fine & Elegante-casual & Nessun gilet & Inverno urbano, lezioni rilassate\\
25 & Maglione texturizzato (shetland leggero) & Texture evidente, urbano-casual & No cravatta, no gilet & Campus, attività diurne informali\\
26 & Maglione texturizzato pesante (trecce, lambswool) & Minima formalità urbana & No cravatta, no gilet & Uscite informali, weekend\\
\end{tabular}
\end{center}
\section{Scaletta CAMPAGNA}
\label{sec:org916e393}

\begin{center}
\begin{tabular}{rllll}
Livello & Tipo di outfit & Caratteristiche & Accessori consentiti & Situazione ideale\\
\hline
1 & Abito liscio worsted & Massima eleganza anche fuori città & Cravatta sobria, gilet fine & Matrimoni, funzioni, eventi importanti\\
2 & Abito liscio flanella fine & Elegante e compatto & Cravatta, gilet fine & Eventi formali invernali, incontri istituzionali\\
3 & Abito liscio worsted + dolcevita & Molto elegante nel rurale & No gilet & Serate eleganti fuori città\\
4 & Abito con micro-pattern & Pattern discreto, classicità country & Cravatta semplice, gilet fine & Eventi semi-formali, pranzi importanti\\
5 & Abito con micro-pattern + dolcevita & Eleganza moderna country & Nessun gilet & Serate, visite formali\\
6 & Abito con pattern medio (POW moderato, windowpane fine) & Perfetto in campagna, molto classic country & Cravatta scura & Ricevimenti, eventi di rappresentanza rurale\\
7 & Abito con texture marcata (flanellone, tweed leggero, donegal fine) & Tessuti “naturali”, molto eleganti in campagna & Gilet scuro, cravatta testurizzata & Eventi importanti locali, riunioni\\
8 & Abito con pattern medio + dolcevita & Eleganza country invernale & Nessun gilet & Serate e pomeriggi eleganti\\
9 & Abito con pattern forte (POW grande, windowpane evidente) & Classico gentleman rurale & Cravatta scura & Cacce simulate, eventi tradizionali\\
10 & Abito con pattern forte + dolcevita & Moderno e molto coerente & Nessun gilet & Pomeriggi e serate invernali\\
11 & Spezzato con giacca texturizzata + pantalone formale & Molto elegante nel rurale & Gilet neutro, cravatta regimental & Ricevimenti, visite ufficiali\\
12 & Spezzato con giacca con pattern + pantalone formale & Pattern naturale = elegante & Cravatta semplice & Lezioni, funzioni, incontri\\
13 & Spezzato con giacca texturizzata + pantalone formale + dolcevita & Coerentissimo in inverno & Nessun gilet & Inverno, borghi, castelli\\
14 & Spezzato con giacca con pattern + pantalone formale + dolcevita & Country classico & Nessun gilet & Pomeriggi, circoli, club\\
15 & Spezzato chino + giacca elegante liscia & Casual medio & Cravatta opzionale & Giorni tranquilli, paesi, visite leggere\\
16 & Spezzato chino + giacca elegante liscia + dolcevita & Casual elegante & Nessun gilet & Inverno nel rurale\\
17 & Spezzato chino + giacca texturizzata & Texture tipica, molto coerente & Gilet fine possibile & Pomeriggi, escursioni leggere\\
18 & Spezzato chino + giacca texturizzata + dolcevita & Country informale elegante & Nessun gilet & Giornate fredde, pranzi informali\\
19 & Maglione standalone fine (merino/cashmere fine) & Eleganza minima & Nessun gilet & Casa di campagna, lezioni rilassate\\
20 & Dolcevita standalone fine & Eleganza-casual & Nessun gilet & Inverno, pomeriggi tranquilli\\
21 & Maglione texturizzato (shetland leggero) & Standard country & No cravatta & Attività diurne, vita quotidiana\\
22 & Maglione texturizzato pesante (trecce, lambswool) & Tipico “uniforme rurale” & No cravatta & Passeggiate, pub, vita invernale tradizionale\\
\end{tabular}
\end{center}
\section{Scaletta Mista}
\label{sec:orgde91fa5}

\begin{center}
\begin{tabular}{rllll}
Livello & Tipo di outfit & Caratteristiche & Accessori consentiti & Situazione ideale\\
\hline
1 & Abito liscio worsted & Tessuto liscio, zero texture, zero pattern & Cravatta sobria, gilet fine & Eventi accademici molto formali, conferenze importanti\\
2 & Abito liscio worsted + dolcevita fine & Elegante serale urbano & No gilet, dolcevita scuro & Serate formali senza cravatta\\
3 & Abito liscio flanella fine & Texture minima, molto elegante & Cravatta, gilet fine & Lezioni importanti, incontri accademici seri\\
6 & Abito con micro-pattern (gessato fine, micro-check) & Pattern basso contrasto & Cravatta semplice, gilet fine & Presentazioni, ospiti esterni, pomeriggi eleganti\\
4 & Spezzato formale (giacca liscia + pantalone formale) & Quasi formale, molto urbano & Gilet fine, cravatta sobria & Ricevimento studenti, eventi semi-istituzionali\\
5 & Abito liscio flanella fine + dolcevita & Elegante ma meno formale del worsted & No gilet & Sera d’inverno, eventi urbani semiformali\\
8 & Abito con micro-pattern + dolcevita & Pattern tenue + sera urbana & Nessun gilet & Serate eleganti, cene sobrie\\
7 & Spezzato formale + dolcevita & Elegante ma non “istituzionale” & No gilet & Sera urbana elegante\\
9 & Spezzato con giacca texturizzata + pantalone formale & Texture moderata, ma ancora serio & Gilet neutro, cravatta regimental & Università, biblioteche, seminari\\
10 & Abito con pattern medio (POW moderato, micro-windowpane fine) & Pattern evidente ma non informale & Cravatta scura, gilet neutro & Lezioni, dipartimento, eventi semi-formali\\
11 & Spezzato con giacca texturizzata + pantalone formale + dolcevita & Buona coesione urbana & Nessun gilet & Inverno, giornata intera in città\\
12 & Abito con pattern medio + dolcevita & Riduce il rumore, vibe urbano & Nessun gilet & Pomeriggi/serate invernali urbane\\
14 & Spezzato con giacca con pattern + pantalone formale & Pattern controllato & Cravatta semplice, gilet scuro & Lezioni, giornate accademiche tipiche\\
15 & Abito con texture marcata + dolcevita & Accademico urbano elegante & Nessun gilet & Inverno, lezioni, conferenze non formali\\
16 & Spezzato con giacca con pattern + pantalone formale + dolcevita & Ottima coesione con tonalità scure & Nessun gilet & Pomeriggi urbani\\
19 & Spezzato chino + giacca elegante liscia & Casual elegante urbano & Cravatta possibile, gilet sconsigliato & Giornate universitarie normali, città\\
20 & Spezzato chino + giacca elegante liscia + dolcevita & Elegante ma rilassato & Nessun gilet & Inverno, lezioni non impegnative\\
21 & Spezzato chino + giacca texturizzata & Texture marcata, casual urbano & Cravatta opzionale, gilet fine & Look accademico urbano informale\\
22 & Spezzato chino + giacca texturizzata + dolcevita & Ottima combinazione urbana invernale & Nessun gilet & Pomeriggi, città, studio\\
23 & Maglione standalone (merino/cashmere fine) & Eleganza minima ma pulita & Cravatta solo con girocollo fine & Studio, biblioteca, mattine leggere\\
24 & Dolcevita standalone fine & Elegante-casual & Nessun gilet & Inverno urbano, lezioni rilassate\\
25 & Maglione texturizzato (shetland leggero) & Texture evidente, urbano-casual & No cravatta, no gilet & Campus, attività diurne informali\\
26 & Maglione texturizzato pesante (trecce, lambswool) & Estetica rurale & No cravatta, no gilet & Contesti rurali, weekend, uscite informali\\
\end{tabular}
\end{center}
\end{document}
