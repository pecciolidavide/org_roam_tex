% Created 2026-02-07 Sat 19:30
% Intended LaTeX compiler: pdflatex
\documentclass[10pt]{article}
%% CREATO CON ORG - EMACS
\newcommand{\use}[2][]{\usepackage[#1]{#2}}
% PACCHETTI FONDAMENTLAI
\use[utf8]{inputenc}
\use[T1]{fontenc}
\use{graphicx}
\use{longtable}
\use{wrapfig}
\use{rotating}
\use[normalem]{ulem}
\use{amsmath}
\use{amsthm}
\use{amssymb}

\use{eucal} % Cambia mathcal{...}

\use{capt-of}
\use[italian]{babel}
\use[babel]{csquotes}
% bib la TEX lo carica in automatico org-cite
\use{microtype}
\use{lmodern}
\use{subfig} % sottofigure
\use{multicol} % due colonne
\use{lipsum} % lorem ipsum
\use{color} % colori in latex
\use{parskip} % rimuove l'indentazione dei nuovi paragrafi %% Add parbox=false to all new tcolorbox
\use{centernot}
\use[outline]{contour}\contourlength{3pt}
\use{fancyhdr}
\use{layout}
\use[most]{tcolorbox} % Riquadri colorati
\use{ifthen} % IFTHEN
\use{geometry}

% pacchetti matematica
\use{yhmath}
\use{dsfont}
\use{mathrsfs}
\use{cancel} % semplificare
\use{polynom} %divisione tra polinomi
\use{forest} % grafi ad albero
\use{booktabs} % tabelle
\use{commath} %simboli e differenziali
\use{bm} %bold
\use[fulladjust]{marginnote} %to use marginnote for date notes
\use{arrayjobx}%array
\use[intlimits]{empheq} % Riquadri colorati attorno alle equazioni
\use{mathtools}
\use{circuitikz} % Disegnare i circuiti
\use{mathtools}
\use{stmaryrd} % [[ \llbracket ]] \rrbracket
\use{bussproofs} % dimostrazioni

%%%%%%%%%%%%%


%%%% QUIVER
\newcommand{\duepunti}{\,\mathchar\numexpr"6000+`:\relax\,}
% A TikZ style for curved arrows of a fixed height, due to AndréC.
\tikzset{curve/.style={settings={#1},to path={(\tikztostart)
    .. controls ($(\tikztostart)!\pv{pos}!(\tikztotarget)!\pv{height}!270:(\tikztotarget)$)
    and ($(\tikztostart)!1-\pv{pos}!(\tikztotarget)!\pv{height}!270:(\tikztotarget)$)
    .. (\tikztotarget)\tikztonodes}},
    settings/.code={\tikzset{quiver/.cd,#1}
        \def\pv##1{\pgfkeysvalueof{/tikz/quiver/##1}}},
    quiver/.cd,pos/.initial=0.35,height/.initial=0}

% TikZ arrowhead/tail styles.
\tikzset{tail reversed/.code={\pgfsetarrowsstart{tikzcd to}}}
\tikzset{2tail/.code={\pgfsetarrowsstart{Implies[reversed]}}}
\tikzset{2tail reversed/.code={\pgfsetarrowsstart{Implies}}}
% TikZ arrow styles.
\tikzset{no body/.style={/tikz/dash pattern=on 0 off 1mm}}
%%%%%%%%%%


%% DEFINIZIONI COMANDI MATEMATICI
\let\sin\relax %TOGLIE LA DEFINIZIONE SU "\sin"

% cambia la definizione di empty set
% ---
\let\oldemptyset\emptyset
% ---
% \let\emptyset\varnothing
% ---
% \let\emptyset\relax
% \newcommand{\emptyset}{\text{\textnormal{\O}}}
% ---

\DeclareMathOperator{\bounded}{bd}
\DeclareMathOperator{\sin}{sen}
\DeclareMathOperator{\epi}{Epi}
\DeclareMathOperator{\cl}{cl}
\DeclareMathOperator{\graph}{graph}
\DeclareMathOperator{\arcsec}{arcsec}
\DeclareMathOperator{\arccot}{arccot}
\DeclareMathOperator{\arccsc}{arccsc}
\DeclareMathOperator{\spettro}{Spettro}
\DeclareMathOperator{\nulls}{nullspace}
\DeclareMathOperator{\dom}{dom}
\DeclareMathOperator{\ar}{ar}
\DeclareMathOperator{\const}{Const}
\DeclareMathOperator{\fun}{Fun}
\DeclareMathOperator{\rel}{Rel}
\DeclareMathOperator{\altezza}{ht}
\let\det\relax %TOGLIE LA DEFINIZIONE SU "\det"
\DeclareMathOperator{\det}{det}
\DeclareMathOperator{\End}{End}
\DeclareMathOperator{\gl}{GL}
\def\Id{\mathrm{Id}}
\def\id{\mathrm{id}}
\DeclareMathOperator{\I}{\mathds{1}}
\DeclareMathOperator{\II}{II}
\DeclareMathOperator{\rank}{rank}
\DeclareMathOperator{\tr}{tr}
\DeclareMathOperator{\tc}{t.c.}
\DeclareMathOperator{\T}{T}
\DeclareMathOperator{\var}{Var}
\DeclareMathOperator{\cov}{Cov}
\DeclareMathOperator{\st}{st}
\DeclareMathOperator{\mon}{Mon}
\newcommand{\card}[1]{\left\vert #1 \right\vert}
\newcommand{\trasposta}[1]{\prescript{\text{T}}{}{#1}}
\newcommand{\1}{\mathds{1}}
\newcommand{\R}{\mathds{R}}
\newcommand{\diesis}{\#}
\newcommand{\bemolle}{\flat}
\newcommand{\nonstandard}[1]{\prescript{*}{}{#1}}
\newcommand{\starR}{\nonstandard{\R}}
\newcommand{\borel}{\mathscr{B}}
\newcommand{\lebesgue}[1]{\mathscr{L}\left(#1\right)}
\newcommand{\media}{\mathds{E}}
\newcommand{\K}{\mathds{K}}
\newcommand{\A}{\mathds{A}}
\newcommand{\Q}{\mathds{Q}}
\newcommand{\N}{\mathds{N}}
\newcommand{\C}{\mathds{C}}
\newcommand{\Z}{\mathds{Z}}
\newcommand{\qo}{\hspace{1em}\text{q.o.}\,}
\renewcommand{\tilde}[1]{\widetilde{#1}}
\renewcommand{\parallel}{\mathrel{/\mkern-5mu/}}
\newcommand{\parti}[2][]{\wp_{#1}(#2)}
\newcommand{\diff}[1]{\operatorname{d}_{#1}}
\let\oldvec\vec
\renewcommand{\vec}[1]{\overrightarrow{\vphantom{i}#1}}
\newcommand{\floor}[1]{\left\lfloor #1 \right\rfloor}
\newcommand{\cat}[1]{\mathbf{#1}}
\newcommand{\dfreccia}[1]{\xrightarrow{\ #1 \ }}
\newcommand{\sfreccia}[1]{\xleftarrow{\ #1 \ }}
\newcommand{\formalsum}[2]{{\sum_{#1}^{#2}}{\vphantom{\sum}}'}
\newcommand{\minim}[2]{\mu_{#1}\, \left(#2\right)}
\newcommand{\concat}{\null^{\frown}} % concatenazione di stringe
\newcommand{\godelcode}[1]{\langle\!\langle #1 \rangle\!\rangle}
\newcommand{\godeldec}[1]{(\!(#1)\!)}
\newcommand{\termcode}[1]{\ulcorner #1\urcorner}
\newcommand{\partialto}{\dashrightarrow}
\newcommand{\restricted}{\upharpoonright}
\newcommand{\embeds}{\precsim}
\newcommand{\surjects}{\twoheadrightarrow}
\newcommand{\equipotenti}{\asymp}
%% \newcommand{\dotplus}{\mathbin{\dot{+}}} %% A quanto pare esiste già
\newcommand{\bigdot}{\mathbin{\boldsymbol{\cdot}}}
\newcommand{\dotexp}[1]{^{.#1}}
\newcommand{\conv}{\mathbin{*}}
\newcommand{\convolution}[2]{(#1\conv #2)}
\newcommand{\nil}{\mathfrak{N}}
\newcommand{\divisore}{\mathrel{|}}
\newcommand{\simplesso}[1]{\mathrm{e}_{#1}}

\renewcommand{\iff}{\mathrel{\longleftrightarrow}} %% Notazione Logica.
\newcommand{\oldiff}{\mathrel{\Longleftrightarrow}}
\renewcommand{\implies}{\mathrel{\rightarrow}} %% Notazione Logica
\newcommand{\oldimplies}{\mathrel{\Longrightarrow}}
\renewcommand{\impliedby}{\mathrel{\leftarrow}} %% Notazione Logica
\newcommand{\oldimpliedby}{\mathrel{\Longleftarrow}}

\newcommand{\IFF}{\quad\Longleftrightarrow\quad}
\newcommand{\IMPLICA}{\quad\Longrightarrow\quad}


\renewcommand{\descriptionlabel}[1]{\hspace{\labelsep}\normalfont #1} % remove bold from description


%% Definizione di Divergenza di K-L

\DeclarePairedDelimiterX{\infdivx}[2]{(}{)}{%
  #1\;\delimsize\|\;#2%
}
\newcommand{\kldiv}{D_{KL}\infdivx}

%% Definizione di \dotminus

\makeatletter
\newcommand{\dotminus}{\mathbin{\text{\@dotminus}}}

\newcommand{\@dotminus}{%
  \ooalign{\hidewidth\raise1ex\hbox{.}\hidewidth\cr$\m@th-$\cr}%
}
\makeatother

%tramite i prossimi due comandi posso decidere come scrivere i logaritmi naturali in tutti i documenti: ho infatti eliminato qualsiasi differenza tra "ln" e "log": se si vuole qualcosa di diverso bisogna inserire manualmente il tutto
\let\ln\relax
\DeclareMathOperator{\ln}{ln}
\let\log\relax
\DeclareMathOperator{\log}{log}
%%%%%%

%% NUOVI COMANDI
\newcommand{\straniero}[1]{\textit{#1}} %parole straniere
\newcommand{\titolo}[1]{\textsc{#1}} %titoli
\newcommand{\qedd}{\tag*{$\blacksquare$}} %qed per ambienti matemastici
\renewcommand{\qedsymbol}{$\blacksquare$} %modifica colore qed
\newcommand{\ooverline}[1]{\overline{\overline{#1}}}
\newcommand{\circoletto}[1]{\left(#1\right)^{\text{o}}}
%
\newcommand{\qmatrice}[1]{\begin{pmatrix}
#1_{11} & \cdots & #1_{1n}\\
\vdots & \ddots & \vdots \\
#1_{m1} & \cdots & #1_{mn}
\end{pmatrix}}
%
\newcommand{\parentesi}[2]{%
\underset{#1}{\underbrace{#2}}%
}
%
\newcommand{\norma}[1]{% Norma
\left\lVert#1\right\rVert%
}
\newcommand{\scalare}[2]{% Scalare
\left\langle #1, #2\right\rangle
}
%%%%%

%% RESTRIZIONI
\newcommand{\referenze}[2]{
        \phantomsection{}#2\textsuperscript{\textcolor{blue}{\textbf{#1}}}
}

\let\restriction\relax

\def\restriction#1#2{\mathchoice
              {\setbox1\hbox{${\displaystyle #1}_{\scriptstyle #2}$}
              \restrictionaux{#1}{#2}}
              {\setbox1\hbox{${\textstyle #1}_{\scriptstyle #2}$}
              \restrictionaux{#1}{#2}}
              {\setbox1\hbox{${\scriptstyle #1}_{\scriptscriptstyle #2}$}
              \restrictionaux{#1}{#2}}
              {\setbox1\hbox{${\scriptscriptstyle #1}_{\scriptscriptstyle #2}$}
              \restrictionaux{#1}{#2}}}
\def\restrictionaux#1#2{{#1\,\smash{\vrule height .8\ht1 depth .85\dp1}}_{\,#2}}
%%%%%%%%%%%

%%% FORMATTAZIONE FOOTNOTEMARK

\def\footnotemarkformatting#1{[#1]}
\renewcommand{\thefootnote}{\footnotemarkformatting{\arabic{footnote}}}

%% SEZIONE GRAFICA
\use{tikz}
\usetikzlibrary{matrix, patterns, calc, decorations.pathreplacing, hobby, decorations.markings, decorations.pathmorphing, babel}
\use{tikz-3dplot}
\use{mathrsfs} %per geogebra
\use{tikz-cd}
\tikzset
{
  %surface/.style={fill=black!10, shading=ball,fill opacity=0.4},
  plane/.style={black,pattern=north east lines},
  curve/.style={black,line width=0.5mm},
  dritto/.style={decoration={markings,mark=at position 0.5 with {\arrow{Stealth}}}, postaction=decorate},
  rovescio/.style={decoration={markings,mark=at position 0.5 with {\arrow{Stealth[reversed]}}}, postaction=decorate}
}
\use{pgfplots} % stampare le funzioni
        \pgfplotsset{/pgf/number format/use comma,compat=1.15}
        %\pgfplotsset{compat=1.15} %per geogebra
        \usepgfplotslibrary{fillbetween, polar}
%%%%%%

%% CITAZIONI
\use{lineno}

\newcommand{\citazione}[1]{%
  \begin{quotation}
  \begin{linenumbers}
  \modulolinenumbers[5]
  \begingroup
  \setlength{\parindent}{0cm}
  \noindent #1
  \endgroup
  \end{linenumbers}
  \end{quotation}\setcounter{linenumber}{1}
  }
%%%%%%

%%%%%%%%%%%%%%%%%%%%%%%%%%%%%%%%%%%%%%%%%%%%
%%%%%%%%%%%%%%%%%%%%%%%%%%%%%%%%%%%%%%%%%%%%

%% AMS THM

\theoremstyle{definition}% default
\newtheorem{thm}{Teorema}[section]
\newtheorem{lem}[thm]{Lemma}
\newtheorem{prop}[thm]{Proposizione}
\newtheorem{cor}[thm]{Corollario}
\newtheorem{esempio}[thm]{Esempio}
\theoremstyle{plain}
\newtheorem{definizione}[thm]{Definizione}
\theoremstyle{remark}
\newtheorem*{oss}{Osservazione}


%%%%%%%%%%%%%%%%%%%%%%%%%%%%%%%%%%%%%%%%%%%%
%%%%%%%%%%%%%%%%%%%%%%%%%%%%%%%%%%%%%%%%%%%%

\use{hyperref}
\hypersetup{%
        pdfauthor={Davide Peccioli},
        pdfsubject={},
        allcolors=black,
        citecolor=black,
%	colorlinks=true,
        bookmarksopen=true}
\setcounter{secnumdepth}{0} % rimuove i numeri di sezione senza rimuovere le ref
\renewcommand{\href}[2]{\textcolor{blue}{#2}} % disabilita il comando href
\use{enotez} %
\setenotez{%
 mark-format = \footnotemarkformatting % Mette i numeri tra parentesi quadre%
}\let\footnote=\endnote % rende tutte le note a pié pagina come delle note a fine file 


\let\olddocument\document % modifico l'ambiende documenti per non dover stampare \printendnote
\let\oldenddocument\enddocument
\renewenvironment{document}%
{%
  \olddocument
}{%
  \printendnotes\oldenddocument
}
\renewcommand{\thethm}{\arabic{thm}}

\usepackage[hyperref]{biblatex}
\addbibresource{~/Documents/org/roam/bib/master.bib}
\author{Davide Peccioli}
\date{\today}
\title{}
\begin{document}

\section{Aritmetizzazione della sintassi}
\label{sec:org0b21ad6}
Sia \(L=\operatorname{Rel}\cup \operatorname{Fun}\cup \operatorname{Cost}\) un \href{20250130162057-linguaggio_del_prim_ordine.org}{linguaggio del prim'ordine} \href{20250111143651-insieme_numerabile.org}{numerabile}, dove \(\operatorname{Rel}, \operatorname{Fun}, \operatorname{ Cost}\) sono gli insiemi di simboli di, rispettivamente, relazione, funzione e costanti, con \(\operatorname{ar}(\cdot)\) la funzione di \href{20250130162057-linguaggio_del_prim_ordine.org}{arietà}, e sia \(\operatorname{Vbl}=\set{v_{i}\mid i \in \omega}\) l'insieme delle \href{20250130162057-linguaggio_del_prim_ordine.org}{variabili}.
\subsection{Buona codifica di un linguaggio}
\label{sec:orgc77f614}
Una buona codifica per \(L\) è una \href{20250531104048-codifica_di_un_insieme_numerabile.org}{codifica ricorsiva} dell'insieme \(D=L\cup \operatorname{Vbl}\cup \set{\lnot, \land,\lor, \implies,\iff, \exists, \forall, =}\)
\begin{equation*}
\#: D\to \N
\end{equation*}
tale che
\begin{itemize}
\item \(\#(v_{i}) = 2{i}\) per ogni \(v_{i} \in \operatorname{Vbl}\);
\item \(\#(\lnot)=1\), \(\#(\land) =3\), \(\#(\lor)=5\), \(\#(\implies)=7\), \(\#(\iff)=9\), \(\#(\exists)=11\), \(\#(\forall) = 13\), \(\#(=)=15\);
\item le \href{20250202190147-immagine_punto_a_punto_di_due_classi.org}{immagini} \(\operatorname{Rel}^{\#}\coloneqq\#[\operatorname{Rel}]\), \(\operatorname{Fun}^{\#}\coloneqq\#[\operatorname{Fun}]\) e \(\operatorname{Cost}^{\#}\coloneqq\#[\operatorname{Cost}]\) sono ricorsivi primitivi;
\item la funzione \(a:\N\to \N\) tale che
\begin{equation*}
  a(n) =\begin{cases}
  	\operatorname{ar}(s) & \text{se }n=\#(s)\text{ per qualche }s \in \operatorname{Rel}\cup \operatorname{Fun}\\
  	0 &\text{altrimenti}
      \end{cases}
\end{equation*}
è ricorsiva primitiva.
\end{itemize}
\subsection{Buona codifica dei termini di un linguaggio}
\label{sec:org7207b46}
A partire da \(\#\), si definiscono delle codifiche \(\termcode{t}\) per gli \(L\)-\href{20250130162316-termine_del_prim_ordine.org}{termini} \(t\) per ricorsione sull'\href{20250130162316-termine_del_prim_ordine.org}{altezza} di \(t\):
\begin{itemize}
\item se \(t\) è della forma \(s\) per qualche simbolo \(s \in \operatorname{Cost}\cup \operatorname{Vbl}\), si pone
\begin{equation*}
  \termcode{t} \coloneqq \godelcode{\#(s)}
\end{equation*}
dove \(\godelcode{\cdot}\) è la \href{20250531110737-codifica_delle_sequenze_finite_tramite_beta_di_godel.org}{codifica di Godel per le sequenze finite};
\item se \(t\) è della forma \(f(t_{1},\dots,t_{k})\) per qualche \(f \in \operatorname{Fun}\) e con \(\operatorname{ar}(f)=k\), si pone
\begin{equation*}
  \termcode{t} \coloneqq \godelcode{\#(f),\termcode{t_{1}},\dots,\termcode{t_{k}}}.
\end{equation*}
\end{itemize}

Si indica con \(\operatorname{Term}\) l'insieme degli \(L\)-termini e con \(\operatorname{Term}^{\#}\) l'insieme delle loro codifiche, ovvero
\begin{equation*}
\operatorname{Term}^{\#} \coloneqq \set{\termcode{t}\mid t \in \operatorname{Term}}.
\end{equation*}
\begin{prop}
Valgono le seguenti proprietà:
\begin{itemize}
\item \(0\notin \operatorname{Term}^{\#}\), poiché \(0=\godelcode{}\) mentre ogni elemento di \(\operatorname{Term}^{\#}\) è codifica di una sequenza non vuota;
\item per ogni \(t,t' \in \operatorname{Term}\) si ha
\begin{itemize}
\item \(\operatorname{ht}(t)\le \termcode{t}\);
\item se \(t'\) è un \href{20250609111619-sottotermine.org}{sottotermine} di \(t\), allora \(\termcode{t'}\le\termcode{t}\);
\item \(\term{t}=\term{t'}\) se e solo se \(t=t'\);
\item se il simbolo \(s\) occorre in \(t\), allora \(\#(s)\le \termcode{t}\);
\end{itemize}
\item l'insieme \(\operatorname{Term}^{\#}\) è ricorsivo primitivo.
\end{itemize}
\end{prop}
\subsection{Buona codifica delle formule di un linguaggio}
\label{sec:orgcbafa71}
A partire da \(\#\), si definiscono delle codifiche \(\termcode{\varphi}\) per le \(L\)-\href{20250131103317-formula_del_prim_ordine.org}{formule} \(\varphi\) per ricorsione sull'\href{20250131103317-formula_del_prim_ordine.org}{altezza} di \(\varphi\):
\begin{itemize}
\item se \(\varphi\) è della forma \(t_{1}=t_{2}\) con \(t_{1},t_{2} \in \operatorname{Term}\), allora
\begin{equation*}
  \termcode{\varphi} = \godelcode{\#(=), \termcode{t_{1}},\termcode{t_{2}}};
\end{equation*}
\item se \(\varphi\) è della forma \(R(t_{1},\dots,t_{k})\) con \(R \in \operatorname{Rel}\), \(\operatorname{ar}(R)=k\) e \(t_{1},\dots,t_{k} \in \operatorname{Term}\), allora si pone
\begin{equation*}
  \termcode{\varphi} = \godelcode{\#(R), \termcode{t_{1}},\dots,\termcode{t_{k}}};
\end{equation*}
\item se \(\varphi\) è della forma \(\lnot\psi\) poniamo \(\termcode{\varphi}=\godelcode{\#(\lnot), \termcode{\varphi}}\);
\item se \(\varphi\) è della forma \(\psi_{1}\mathrel{\square} \psi_{2}\) con \(\square \in \set{\land,\lor,\implies,\iff}\), si pone
\begin{equation*}
  \termcode{\varphi} = \godelcode{\#(\square), \termcode{\psi_{1}},\termcode{\psi_{2}}};
\end{equation*}
\item se \(\varphi\) è della forma \(Q\,v_{i}\ \psi\) per qualche \(Q \in \set{\exists,\forall}\) e \(v_{i} \in \operatorname{Vbl}\), si pone
\begin{equation*}
  \termcode{\varphi} = \godelcode{\#(Q), \#(v_{i}), \termcode{\psi}}.
\end{equation*}
\end{itemize}

Si indica con \(\operatorname{Fml}\) l'insieme delle \(L\)-formule e con \(\operatorname{Fml}^{\#}\) l'insieme delle loro codifiche, ovvero
\begin{equation*}
\operatorname{Fml}^{\#} \coloneqq \set{\termcode{\varphi}\mid \varphi \in \operatorname{Fml}}.
\end{equation*}
\subsubsection{Proprietà}
\label{sec:orgf34b5b6}
\begin{itemize}
\item \(0\notin \operatorname{Fml}^{\#}\), poiché \(0=\godelcode{}\) mentre ogni elemento di \(\operatorname{Form}^{\#}\) è codifica di una sequenza non vuota;
\item per ogni \(\varphi,\psi \in \operatorname{Fml}\) si ha
\begin{itemize}
\item \(\operatorname{ht}(\varphi)\le \termcode{\varphi}\);
\item se \(\psi\) è una \href{20250609113022-sottoformula_del_prim_ordine.org}{sottoformula} di \(\varphi\), allora \(\termcode{\psi}\le\termcode{\varphi}\);
\item \(\term{\varphi}=\term{\psi}\) se e solo se \(\varphi=\psi\);
\item se il simbolo \(s\) occorre in \(\varphi\), allora \(\#(s)\le \termcode{\varphi}\);
\end{itemize}
\item l'insieme \(\operatorname{Fml}^{\#}\) è ricorsivo primitivo.
\end{itemize}
\subsection{Codifica della sostituzione di termini a variabili}
\label{sec:org7edb6d1}
Dati due termini \(s,t \in \operatorname{Term}\) ed una variabile \(v_{i}\), il termine \(s(t/v_{i})\) ottenuto \href{20250609125154-sostituzione_di_termini_in_un_termine.org}{sostituendo} \(t\) a \(v_{i}\) è definito per ricorsione sull'altezza di \(s\) come segue:
\begin{itemize}
\item se \(s=v_{i}\), allora \(s(t/v_{i}) = t\);
\item se \(s \in \operatorname{Cost}\cup \operatorname{Vbl}\) ma \(s\neq v_{i}\), allora \(s(t/v_{i}) = s\);
\item se \(s=f(t_{1},\dots,t_{k})\) con \(f \in \operatorname{Fun}\) di arietà \(k\) e \(t_{1},\dots,t_{k} \in \operatorname{Term}\), allora
\begin{equation*}
  s(t/v_{i}) = f\left(t_{1}(t/v_{i}), \dots,t_{k}(t/v_{i})\right).
\end{equation*}
\end{itemize}

Allo stesso modo, data una formula \(\varphi \in \operatorname{Fml}\) e un termine \(t \in \operatorname{Term}\) ed una variabile \(v_{i} \in \operatorname{Vbl}\), la formula \(\varphi(t/v_{i})\) ottenuta sostituendo \(t\) ad ogni occorrenza libera di \(v_{i}\) in \(\varphi\), è definita per ricorsione sull'altezza della formula:
\begin{itemize}
\item se \(\varphi\) è della forma \(t_{1}=t_{2}\) allora \(\varphi(t/v_{i})\) è \(t_{1}(t/v_{i}) = t_{2}(t/v_{i})\);
\item se \(\varphi\) è della forma \(R(t_{1},\dots,t_{k})\) con \(R \in \operatorname{Rel}\) di arietà \(k\) e \(t_{1},\dots,t_{k} \in \operatorname{Term}\) allora \(\varphi(t/v_{i})\) è \(R\left(t_{1}(t/v_{i}),\dots,t_{k}(t/v_{i})\right)\);
\item se \(\varphi\) è della forma \(\lnot\psi\) allora \(\varphi(t/v_{i})\) è \(\lnot \psi(t/v_{i})\);
\item se \(\varphi\) è della forma \(\psi_{1}\mathrel{\square}\psi_{2}\) con \(\square \in \set{\land,\lor, \implies,\iff}\) allora \(\varphi(t/v_{i})\) è \(\psi_{1}(t/v_{i})\mathrel{\square}\psi_{2}(t/v_{i})\);
\item se \(\varphi\) è della forma \(Q\, v_{i}\ \psi\) con \(Q \in \set{\forall,\exists}\), allora \(\varphi(t/v_{i})\) è \(\varphi\) stessa;
\item se \(\varphi\) è della forma \(Q\, v_{j}\ \psi\) con \(Q \in \set{\forall,\exists}\) e \(j\neq i\), allora \(\varphi(t/v_{i})\) è \(Q\, v_{j}\ \psi(t/v_{i})\).
\end{itemize}
\subsubsection{Proposizione}
\label{sec:org16ae478}
Le funzioni \(\operatorname{sub}_{\operatorname{Term}}, \operatorname{sub}_{\operatorname{Fml}}:\N^{3}\to \N\), definite come segue, sono \href{20250215141024-funzioni_primitive_ricordive.org}{ricorsive primitive}:
\begin{align*}
\operatorname{sub}_{\operatorname{Term}}(n,i,m) &\coloneqq \begin{cases}
\termcode{s(t/v_{i})} & \text{se }n=\termcode{s}, m=\termcode{t} \in \operatorname{Term}^{\#}\\
0 &\text{altrimenti}.
\end{cases}\\
\operatorname{sub}_{\operatorname{Fml}}(n,i,m) &\coloneqq \begin{cases}
\termcode{\varphi(t/v_{i})} & \text{se }n=\termcode{\varphi} \in \operatorname{Fml}^{\#}, m=\termcode{t} \in \operatorname{Term}^{\#}\\
0 &\text{altrimenti}.
\end{cases}
\end{align*}

Inoltre sono ricorsivi primitivi gli insiemi
\begin{align*}
\operatorname{Free}_{\operatorname{Term}}^{\#} &\coloneqq \set{(i,\termcode{t}) \in \N^{2}\mid v_{i}\text{ occorre libera nel termine }t}\\
\operatorname{Free}_{\operatorname{Fml}}^{\#} &\coloneqq \set{(i,\termcode{\varphi}) \in \N^{2}\mid v_{i}\text{ occorre libera nella formula  }\varphi}\\
\operatorname{Enum}^{\#} &\coloneqq\set{\termcode{\sigma}\mid \sigma\text{ è un enunciato}} \subseteq \N
\end{align*}
\subsection{Notazioni}
\label{sec:org56853cb}

Data una \(L\)-teoria del prim'ordine \(T\), indichiamo con
\begin{align*}
\operatorname{Teor}_{T} &\coloneqq\set{\varphi\mid T\vDash \varphi}\\
\operatorname{Teor}_{T}^{\#} &\coloneqq\set{\termcode{\varphi}\mid \varphi \in \operatorname{Teor}_{T}}
\end{align*}
gli insiemi delle \href{20250131123011-conseguenza_logica.org}{conseguenze logiche} di \(T\) e dei loro codici.

Inoltre, se \(\operatorname{Ax}(T)\) è un \href{20250131123109-insieme_di_assiomi_per_una_teoria.org}{sistema di assiomi} per \(T\), allora poniamo
\begin{equation*}
\operatorname{Ax}^{\#}(T) :=\set{\termcode{\varphi}\mid\varphi \in \operatorname{Ax}(T)}.
\end{equation*}
\end{document}
