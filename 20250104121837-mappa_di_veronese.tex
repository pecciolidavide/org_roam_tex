% Created 2026-02-07 Sat 19:30
% Intended LaTeX compiler: pdflatex
\documentclass[10pt]{article}
%% CREATO CON ORG - EMACS
\newcommand{\use}[2][]{\usepackage[#1]{#2}}
% PACCHETTI FONDAMENTLAI
\use[utf8]{inputenc}
\use[T1]{fontenc}
\use{graphicx}
\use{longtable}
\use{wrapfig}
\use{rotating}
\use[normalem]{ulem}
\use{amsmath}
\use{amsthm}
\use{amssymb}

\use{eucal} % Cambia mathcal{...}

\use{capt-of}
\use[italian]{babel}
\use[babel]{csquotes}
% bib la TEX lo carica in automatico org-cite
\use{microtype}
\use{lmodern}
\use{subfig} % sottofigure
\use{multicol} % due colonne
\use{lipsum} % lorem ipsum
\use{color} % colori in latex
\use{parskip} % rimuove l'indentazione dei nuovi paragrafi %% Add parbox=false to all new tcolorbox
\use{centernot}
\use[outline]{contour}\contourlength{3pt}
\use{fancyhdr}
\use{layout}
\use[most]{tcolorbox} % Riquadri colorati
\use{ifthen} % IFTHEN
\use{geometry}

% pacchetti matematica
\use{yhmath}
\use{dsfont}
\use{mathrsfs}
\use{cancel} % semplificare
\use{polynom} %divisione tra polinomi
\use{forest} % grafi ad albero
\use{booktabs} % tabelle
\use{commath} %simboli e differenziali
\use{bm} %bold
\use[fulladjust]{marginnote} %to use marginnote for date notes
\use{arrayjobx}%array
\use[intlimits]{empheq} % Riquadri colorati attorno alle equazioni
\use{mathtools}
\use{circuitikz} % Disegnare i circuiti
\use{mathtools}
\use{stmaryrd} % [[ \llbracket ]] \rrbracket
\use{bussproofs} % dimostrazioni

%%%%%%%%%%%%%


%%%% QUIVER
\newcommand{\duepunti}{\,\mathchar\numexpr"6000+`:\relax\,}
% A TikZ style for curved arrows of a fixed height, due to AndréC.
\tikzset{curve/.style={settings={#1},to path={(\tikztostart)
    .. controls ($(\tikztostart)!\pv{pos}!(\tikztotarget)!\pv{height}!270:(\tikztotarget)$)
    and ($(\tikztostart)!1-\pv{pos}!(\tikztotarget)!\pv{height}!270:(\tikztotarget)$)
    .. (\tikztotarget)\tikztonodes}},
    settings/.code={\tikzset{quiver/.cd,#1}
        \def\pv##1{\pgfkeysvalueof{/tikz/quiver/##1}}},
    quiver/.cd,pos/.initial=0.35,height/.initial=0}

% TikZ arrowhead/tail styles.
\tikzset{tail reversed/.code={\pgfsetarrowsstart{tikzcd to}}}
\tikzset{2tail/.code={\pgfsetarrowsstart{Implies[reversed]}}}
\tikzset{2tail reversed/.code={\pgfsetarrowsstart{Implies}}}
% TikZ arrow styles.
\tikzset{no body/.style={/tikz/dash pattern=on 0 off 1mm}}
%%%%%%%%%%


%% DEFINIZIONI COMANDI MATEMATICI
\let\sin\relax %TOGLIE LA DEFINIZIONE SU "\sin"

% cambia la definizione di empty set
% ---
\let\oldemptyset\emptyset
% ---
% \let\emptyset\varnothing
% ---
% \let\emptyset\relax
% \newcommand{\emptyset}{\text{\textnormal{\O}}}
% ---

\DeclareMathOperator{\bounded}{bd}
\DeclareMathOperator{\sin}{sen}
\DeclareMathOperator{\epi}{Epi}
\DeclareMathOperator{\cl}{cl}
\DeclareMathOperator{\graph}{graph}
\DeclareMathOperator{\arcsec}{arcsec}
\DeclareMathOperator{\arccot}{arccot}
\DeclareMathOperator{\arccsc}{arccsc}
\DeclareMathOperator{\spettro}{Spettro}
\DeclareMathOperator{\nulls}{nullspace}
\DeclareMathOperator{\dom}{dom}
\DeclareMathOperator{\ar}{ar}
\DeclareMathOperator{\const}{Const}
\DeclareMathOperator{\fun}{Fun}
\DeclareMathOperator{\rel}{Rel}
\DeclareMathOperator{\altezza}{ht}
\let\det\relax %TOGLIE LA DEFINIZIONE SU "\det"
\DeclareMathOperator{\det}{det}
\DeclareMathOperator{\End}{End}
\DeclareMathOperator{\gl}{GL}
\def\Id{\mathrm{Id}}
\def\id{\mathrm{id}}
\DeclareMathOperator{\I}{\mathds{1}}
\DeclareMathOperator{\II}{II}
\DeclareMathOperator{\rank}{rank}
\DeclareMathOperator{\tr}{tr}
\DeclareMathOperator{\tc}{t.c.}
\DeclareMathOperator{\T}{T}
\DeclareMathOperator{\var}{Var}
\DeclareMathOperator{\cov}{Cov}
\DeclareMathOperator{\st}{st}
\DeclareMathOperator{\mon}{Mon}
\newcommand{\card}[1]{\left\vert #1 \right\vert}
\newcommand{\trasposta}[1]{\prescript{\text{T}}{}{#1}}
\newcommand{\1}{\mathds{1}}
\newcommand{\R}{\mathds{R}}
\newcommand{\diesis}{\#}
\newcommand{\bemolle}{\flat}
\newcommand{\nonstandard}[1]{\prescript{*}{}{#1}}
\newcommand{\starR}{\nonstandard{\R}}
\newcommand{\borel}{\mathscr{B}}
\newcommand{\lebesgue}[1]{\mathscr{L}\left(#1\right)}
\newcommand{\media}{\mathds{E}}
\newcommand{\K}{\mathds{K}}
\newcommand{\A}{\mathds{A}}
\newcommand{\Q}{\mathds{Q}}
\newcommand{\N}{\mathds{N}}
\newcommand{\C}{\mathds{C}}
\newcommand{\Z}{\mathds{Z}}
\newcommand{\qo}{\hspace{1em}\text{q.o.}\,}
\renewcommand{\tilde}[1]{\widetilde{#1}}
\renewcommand{\parallel}{\mathrel{/\mkern-5mu/}}
\newcommand{\parti}[2][]{\wp_{#1}(#2)}
\newcommand{\diff}[1]{\operatorname{d}_{#1}}
\let\oldvec\vec
\renewcommand{\vec}[1]{\overrightarrow{\vphantom{i}#1}}
\newcommand{\floor}[1]{\left\lfloor #1 \right\rfloor}
\newcommand{\cat}[1]{\mathbf{#1}}
\newcommand{\dfreccia}[1]{\xrightarrow{\ #1 \ }}
\newcommand{\sfreccia}[1]{\xleftarrow{\ #1 \ }}
\newcommand{\formalsum}[2]{{\sum_{#1}^{#2}}{\vphantom{\sum}}'}
\newcommand{\minim}[2]{\mu_{#1}\, \left(#2\right)}
\newcommand{\concat}{\null^{\frown}} % concatenazione di stringe
\newcommand{\godelcode}[1]{\langle\!\langle #1 \rangle\!\rangle}
\newcommand{\godeldec}[1]{(\!(#1)\!)}
\newcommand{\termcode}[1]{\ulcorner #1\urcorner}
\newcommand{\partialto}{\dashrightarrow}
\newcommand{\restricted}{\upharpoonright}
\newcommand{\embeds}{\precsim}
\newcommand{\surjects}{\twoheadrightarrow}
\newcommand{\equipotenti}{\asymp}
%% \newcommand{\dotplus}{\mathbin{\dot{+}}} %% A quanto pare esiste già
\newcommand{\bigdot}{\mathbin{\boldsymbol{\cdot}}}
\newcommand{\dotexp}[1]{^{.#1}}
\newcommand{\conv}{\mathbin{*}}
\newcommand{\convolution}[2]{(#1\conv #2)}
\newcommand{\nil}{\mathfrak{N}}
\newcommand{\divisore}{\mathrel{|}}
\newcommand{\simplesso}[1]{\mathrm{e}_{#1}}

\renewcommand{\iff}{\mathrel{\longleftrightarrow}} %% Notazione Logica.
\newcommand{\oldiff}{\mathrel{\Longleftrightarrow}}
\renewcommand{\implies}{\mathrel{\rightarrow}} %% Notazione Logica
\newcommand{\oldimplies}{\mathrel{\Longrightarrow}}
\renewcommand{\impliedby}{\mathrel{\leftarrow}} %% Notazione Logica
\newcommand{\oldimpliedby}{\mathrel{\Longleftarrow}}

\newcommand{\IFF}{\quad\Longleftrightarrow\quad}
\newcommand{\IMPLICA}{\quad\Longrightarrow\quad}


\renewcommand{\descriptionlabel}[1]{\hspace{\labelsep}\normalfont #1} % remove bold from description


%% Definizione di Divergenza di K-L

\DeclarePairedDelimiterX{\infdivx}[2]{(}{)}{%
  #1\;\delimsize\|\;#2%
}
\newcommand{\kldiv}{D_{KL}\infdivx}

%% Definizione di \dotminus

\makeatletter
\newcommand{\dotminus}{\mathbin{\text{\@dotminus}}}

\newcommand{\@dotminus}{%
  \ooalign{\hidewidth\raise1ex\hbox{.}\hidewidth\cr$\m@th-$\cr}%
}
\makeatother

%tramite i prossimi due comandi posso decidere come scrivere i logaritmi naturali in tutti i documenti: ho infatti eliminato qualsiasi differenza tra "ln" e "log": se si vuole qualcosa di diverso bisogna inserire manualmente il tutto
\let\ln\relax
\DeclareMathOperator{\ln}{ln}
\let\log\relax
\DeclareMathOperator{\log}{log}
%%%%%%

%% NUOVI COMANDI
\newcommand{\straniero}[1]{\textit{#1}} %parole straniere
\newcommand{\titolo}[1]{\textsc{#1}} %titoli
\newcommand{\qedd}{\tag*{$\blacksquare$}} %qed per ambienti matemastici
\renewcommand{\qedsymbol}{$\blacksquare$} %modifica colore qed
\newcommand{\ooverline}[1]{\overline{\overline{#1}}}
\newcommand{\circoletto}[1]{\left(#1\right)^{\text{o}}}
%
\newcommand{\qmatrice}[1]{\begin{pmatrix}
#1_{11} & \cdots & #1_{1n}\\
\vdots & \ddots & \vdots \\
#1_{m1} & \cdots & #1_{mn}
\end{pmatrix}}
%
\newcommand{\parentesi}[2]{%
\underset{#1}{\underbrace{#2}}%
}
%
\newcommand{\norma}[1]{% Norma
\left\lVert#1\right\rVert%
}
\newcommand{\scalare}[2]{% Scalare
\left\langle #1, #2\right\rangle
}
%%%%%

%% RESTRIZIONI
\newcommand{\referenze}[2]{
        \phantomsection{}#2\textsuperscript{\textcolor{blue}{\textbf{#1}}}
}

\let\restriction\relax

\def\restriction#1#2{\mathchoice
              {\setbox1\hbox{${\displaystyle #1}_{\scriptstyle #2}$}
              \restrictionaux{#1}{#2}}
              {\setbox1\hbox{${\textstyle #1}_{\scriptstyle #2}$}
              \restrictionaux{#1}{#2}}
              {\setbox1\hbox{${\scriptstyle #1}_{\scriptscriptstyle #2}$}
              \restrictionaux{#1}{#2}}
              {\setbox1\hbox{${\scriptscriptstyle #1}_{\scriptscriptstyle #2}$}
              \restrictionaux{#1}{#2}}}
\def\restrictionaux#1#2{{#1\,\smash{\vrule height .8\ht1 depth .85\dp1}}_{\,#2}}
%%%%%%%%%%%

%%% FORMATTAZIONE FOOTNOTEMARK

\def\footnotemarkformatting#1{[#1]}
\renewcommand{\thefootnote}{\footnotemarkformatting{\arabic{footnote}}}

%% SEZIONE GRAFICA
\use{tikz}
\usetikzlibrary{matrix, patterns, calc, decorations.pathreplacing, hobby, decorations.markings, decorations.pathmorphing, babel}
\use{tikz-3dplot}
\use{mathrsfs} %per geogebra
\use{tikz-cd}
\tikzset
{
  %surface/.style={fill=black!10, shading=ball,fill opacity=0.4},
  plane/.style={black,pattern=north east lines},
  curve/.style={black,line width=0.5mm},
  dritto/.style={decoration={markings,mark=at position 0.5 with {\arrow{Stealth}}}, postaction=decorate},
  rovescio/.style={decoration={markings,mark=at position 0.5 with {\arrow{Stealth[reversed]}}}, postaction=decorate}
}
\use{pgfplots} % stampare le funzioni
        \pgfplotsset{/pgf/number format/use comma,compat=1.15}
        %\pgfplotsset{compat=1.15} %per geogebra
        \usepgfplotslibrary{fillbetween, polar}
%%%%%%

%% CITAZIONI
\use{lineno}

\newcommand{\citazione}[1]{%
  \begin{quotation}
  \begin{linenumbers}
  \modulolinenumbers[5]
  \begingroup
  \setlength{\parindent}{0cm}
  \noindent #1
  \endgroup
  \end{linenumbers}
  \end{quotation}\setcounter{linenumber}{1}
  }
%%%%%%

%%%%%%%%%%%%%%%%%%%%%%%%%%%%%%%%%%%%%%%%%%%%
%%%%%%%%%%%%%%%%%%%%%%%%%%%%%%%%%%%%%%%%%%%%

%% AMS THM

\theoremstyle{definition}% default
\newtheorem{thm}{Teorema}[section]
\newtheorem{lem}[thm]{Lemma}
\newtheorem{prop}[thm]{Proposizione}
\newtheorem{cor}[thm]{Corollario}
\newtheorem{esempio}[thm]{Esempio}
\theoremstyle{plain}
\newtheorem{definizione}[thm]{Definizione}
\theoremstyle{remark}
\newtheorem*{oss}{Osservazione}


%%%%%%%%%%%%%%%%%%%%%%%%%%%%%%%%%%%%%%%%%%%%
%%%%%%%%%%%%%%%%%%%%%%%%%%%%%%%%%%%%%%%%%%%%

\use{hyperref}
\hypersetup{%
        pdfauthor={Davide Peccioli},
        pdfsubject={},
        allcolors=black,
        citecolor=black,
%	colorlinks=true,
        bookmarksopen=true}
\setcounter{secnumdepth}{0} % rimuove i numeri di sezione senza rimuovere le ref
\renewcommand{\href}[2]{\textcolor{blue}{#2}} % disabilita il comando href
\use{enotez} %
\setenotez{%
 mark-format = \footnotemarkformatting % Mette i numeri tra parentesi quadre%
}\let\footnote=\endnote % rende tutte le note a pié pagina come delle note a fine file 


\let\olddocument\document % modifico l'ambiende documenti per non dover stampare \printendnote
\let\oldenddocument\enddocument
\renewenvironment{document}%
{%
  \olddocument
}{%
  \printendnotes\oldenddocument
}
\renewcommand{\thethm}{\arabic{thm}}

\usepackage[hyperref]{biblatex}
\addbibresource{~/Documents/org/roam/bib/master.bib}
\author{Davide Peccioli}
\date{\today}
\title{Mappa di Veronese}
\begin{document}

Sia \(\K\) un \href{20241231112713-campo_algebricamente_chiuso.org}{campo algebricamente chiuso}.
Si consideri \(\mathds{P}^{n}\) con coordinate \([x_{0}:\dots:x_{n}]\) (si veda \href{20241231115051-spazio_proiettivo.org}{Spazio Proiettivo}).

Sia \(S=\K[x_{0},\dots,x_{n}]\) l'\href{20241219113434-anello_dei_polinomi.org}{anello dei polinomi}, e si \(S_{d} \subseteq S\) l'insieme dei \href{20241231121125-polinomi_omogenei.org}{polinomi omogenei} di \href{20241231124742-grado_polinomi.org}{grado} \(d\). Sia \(N\coloneqq \dim S_{d}-1\) (vedi \href{20241231121125-polinomi_omogenei.org}{Dimensione di \(S_{d}\) come spazio vettoriale}):
\begin{equation*}
N=\binom{n+d}{d}-1
\end{equation*}

Consideriamo \(I=(i_{0},\dots,i_{n})\) un \href{20250105122522-multi_indice.org}{multi-indice}; definiamo \(x^{I} \in S\) come
\begin{equation*}
x^{I}\coloneqq x_{0}^{i_{0}}\cdot \dots\cdot x_{n}^{i_{n}}
\end{equation*}
Su \(\mathds{P}^{N}\) usiamo le coordinate \(\set{z_{I}}_{I \in D}\), dove
\begin{equation*}
D=\set{I=(i_{0},\dots,i_{n}): \sum_{j=0}^{n}i_{j}=d,\ i_{j}\ge 0}
\end{equation*}
ordinate in ordine lessicografico. Questo è semplicemente il ``nome'' della coordinata, per semplicità.
Ad esempio, se \(n=d=2\) e \(N=5\), le coordinate di \(\mathds{P}^{5}\) saranno
\begin{equation*}
[z_{(0,0,2)}: z_{(0,1,1)}: z_{(0,2,0)}:z_{(1,0,1)}: z_{(1,1,0)}: z_{(2,0,0)}]
\end{equation*}
I multiindici sono sono soltanto un ``nome'' per le coordinate di \(\mathds{P}^{N}\).

Si definisce la \textbf{mappa di Veronese}
\begin{align*}
\nu_{n,d}: \mathds{P}^{n} &\longrightarrow \mathds{P}^{N}=\mathds{P}(S_{d}^{*})\\
[x_{0}:\dots:x_{n}] &\longmapsto [\dots:x^{I}:\dots]
\end{align*}
(vedi \href{20250105124008-spazio_vettoriale_duale.org}{Spazio vettoriale duale})

Questa mappa è un \href{20250104120600-morfismo_tra_varieta_algebriche_proiettive.org}{morfismo} tra \href{20241231123223-varieta_algebrica_proiettiva.org}{varietà algebriche proiettive}, con una definizione globale sul suo dominio.
\section{Varietà di Veronese}
\label{sec:org6c9c7fd}
L'immagine \(V_{n,d}\coloneqq\nu_{n,d}(\mathds{P}^{n})\) è detta \textbf{varietà di Veronese} di tipo \((n,d)\).
\subsection{Dimostrazione che sia realmente una varietà}
\label{sec:orgbdcbe5a}
Per \(I,J \in D\), poniamo \(I+J=(i_{0}+j_{0},\dots,i_{n}+j_{n})\).

Posto \(Y\) il \href{20241231112823-radici_polinomiali.org}{luogo delle soluzioni} delle equazioni quadratiche
\begin{equation*}
z_{I}z_{J}=z_{K}z_{L}\qquad \forall\, I,J,K,L: \ I+J=K+L
\end{equation*}
si ha che \(V_{n,d}=Y\).
\subsubsection{\(V_{n,d} \subseteq Y\)}
\label{sec:org31976a6}
Sia \(p=[p_{0}:\dots:p_{n}] \in \mathds{P}^{n}\), e consideriamo \(\nu_{n,d}(p)=[\dots:P_{I}:\dots]\) con le coordinate di cui sopra. Siano
\begin{align*}
I&=(i_{0},\dots,i_{n}) & i_{0}+\dots+i_{n}&=d\\
J&=(j_{0},\dots,j_{n}) & j_{0}+\dots+j_{n}&=d\\
K&=(k_{0},\dots,k_{n}) & k_{0}+\dots+k_{n}&=d\\
L&=(l_{0},\dots,l_{n}) & l_{0}+\dots+l_{n}&=d
\end{align*}
tali che \(I+J=K+L\).
\begin{align*}
P_{I}P_{J}&=(x_{0})^{i_{0}}\dots (x_{n})^{i_{n}} \cdot (x_{0})^{j_{0}}\dots(x_{n})^{j_{n}}\\
&= (x_{0})^{i_{0}+j_{0}}\dots(x_{n})^{i_{n}+j_{n}}\\
&= (x_{0})^{k_{0}+l_{0}}\dots(x_{n})^{k_{n}+l_{n}}\\
&= (x_{0})^{k_{0}}\dots(x_{n})^{k_{n}} \cdot (x_{0})^{l_{0}}\dots(x_{n})^{l_{n}} = P_{K}P_{L}
\end{align*}
\subsubsection{\(Y \subseteq V_{n,d}\)}
\label{sec:org43a1528}
Sia \(K_{\ell}\) il multi-indice composto da soli zeri, ma con \(d\) nella posizione \(\ell\)-esima:
\begin{equation*}
K_{\ell}=(0,\dots,0,d,0,\dots,0)
\end{equation*}
per \(0\le \ell\le n\).

Sia dunque \(P=[\dots:P_{I}:\dots] \in Y\).
\paragraph{Claim: esiste \(\ell_{0}\) tale \(P_{K_{\ell_{0}}}\neq 0\)}
\label{sec:org886ba39}
Sia \(i_{\max}\) il massimo indice che compare tra tutti i multiindici \(I\) tali che \(P_{I}\neq 0\). La tesi è dimostrare che \(i_{\max}=d\).

Sia \(P_{(\dots:i_{\max}:\dots)}\neq 0\). Se \(i_{\max}\le d-1\), allora esiste un indice \(j>0\) nel multiindice \((\dots:i_{\max}:\dots)\). Sia dunque \(I\) questo multiindice:
\begin{equation*}
I=(\dots:i_{\max}:\dots:j:\dots)
\end{equation*}
Siccome \(j\ge 0\) e \(i_{\max}\le d-1\), entrambi questi indici sono elementi di \(D\):
\begin{align*}
J&= (\dots,i_{\max}+1,\dots,j-1,\dots)\\
K&= (\dots,i_{\max}-1,\dots,j+1,\dots)
\end{align*}
e inoltre \(J+K=I+I\). Dunque è soddisfatta l'equazione
\begin{equation*}
P_{I}^{2}=P_{J}P_{K}
\end{equation*}
Siccome \(P_{I}\neq 0\) , necessariamente \(P_{J}\neq 0 \neq P_{K}\), e pertanto è presente un indice \(>i_{\max}\). Assurdo.

Dunque \(i_{\max}=d\), e la tesi è dimostrata.
\paragraph{{\bfseries\sffamily TODO} \(P \in V_{n,d}\)\hfill{}\textsc{matematica\_lm:geo\_alg}}
\label{sec:org846fe82}
Supponiamo che \(P_{K_{0}}\neq 0\). Siano:
\begin{align*}
q_{0}&\coloneqq P_{(d,0,\dots,0)} = P_{K_{0}}\neq 0\\
q_{1} &\coloneqq P_{(d-1,1,0,\dots,0)}\\
\vdots&\\
q_{i}&\coloneqq P_{(d-1,0,\dots,1,\dots,0)}
\end{align*}

dove ad \(q_{i}\), il multiindice è composto da \(d-1\) in posizione zero, \(1\) in posizione \(i\), e \(0\) altrove.

Sicuramente \([q_{0}:\dots:q_{n}] \in \mathds{P}^{n}\), siccome \(q_{0}\neq 0\) per ipotesi.

Sia \(\nu_{n,d}\left([q_{0}:\dots:q_{n}]\right)=Q \coloneqq [\dots:Q_{I}:\dots]\)
Vale che \(P=Q\). Infatti, se \(I=(i_{0},\dots,i_{n})\):
\begin{align*}
Q_{I}&=(q_{0})^{i_{0}}\ \dots\ (q_{n})^{i_{n}}\\
&= (P_{(d,0,\dots,0)})^{i_{0}}\ (P_{(d-1,1,0,\dots,0)})^{i_{1}}\ \dots\ (P_{(d-1,0,\dots,0,1)})^{i_{n}}
\end{align*}
\section{{\bfseries\sffamily TODO} \(\nu_{n,d}\) è un \href{20250104120600-morfismo_tra_varieta_algebriche_proiettive.org}{Isomorfismo tra varietà algebriche proiettive}\hfill{}\textsc{matematica\_lm:geo\_alg}}
\label{sec:org37a22c4}
\end{document}
