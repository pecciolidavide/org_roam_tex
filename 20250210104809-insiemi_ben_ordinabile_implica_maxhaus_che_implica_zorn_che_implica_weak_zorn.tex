% Created 2026-02-07 Sat 19:34
% Intended LaTeX compiler: pdflatex
\documentclass[10pt]{article}
%% CREATO CON ORG - EMACS
\newcommand{\use}[2][]{\usepackage[#1]{#2}}
% PACCHETTI FONDAMENTLAI
\use[utf8]{inputenc}
\use[T1]{fontenc}
\use{graphicx}
\use{longtable}
\use{wrapfig}
\use{rotating}
\use[normalem]{ulem}
\use{amsmath}
\use{amsthm}
\use{amssymb}

\use{eucal} % Cambia mathcal{...}

\use{capt-of}
\use[italian]{babel}
\use[babel]{csquotes}
% bib la TEX lo carica in automatico org-cite
\use{microtype}
\use{lmodern}
\use{subfig} % sottofigure
\use{multicol} % due colonne
\use{lipsum} % lorem ipsum
\use{color} % colori in latex
\use{parskip} % rimuove l'indentazione dei nuovi paragrafi %% Add parbox=false to all new tcolorbox
\use{centernot}
\use[outline]{contour}\contourlength{3pt}
\use{fancyhdr}
\use{layout}
\use[most]{tcolorbox} % Riquadri colorati
\use{ifthen} % IFTHEN
\use{geometry}

% pacchetti matematica
\use{yhmath}
\use{dsfont}
\use{mathrsfs}
\use{cancel} % semplificare
\use{polynom} %divisione tra polinomi
\use{forest} % grafi ad albero
\use{booktabs} % tabelle
\use{commath} %simboli e differenziali
\use{bm} %bold
\use[fulladjust]{marginnote} %to use marginnote for date notes
\use{arrayjobx}%array
\use[intlimits]{empheq} % Riquadri colorati attorno alle equazioni
\use{mathtools}
\use{circuitikz} % Disegnare i circuiti
\use{mathtools}
\use{stmaryrd} % [[ \llbracket ]] \rrbracket
\use{bussproofs} % dimostrazioni

%%%%%%%%%%%%%


%%%% QUIVER
\newcommand{\duepunti}{\,\mathchar\numexpr"6000+`:\relax\,}
% A TikZ style for curved arrows of a fixed height, due to AndréC.
\tikzset{curve/.style={settings={#1},to path={(\tikztostart)
    .. controls ($(\tikztostart)!\pv{pos}!(\tikztotarget)!\pv{height}!270:(\tikztotarget)$)
    and ($(\tikztostart)!1-\pv{pos}!(\tikztotarget)!\pv{height}!270:(\tikztotarget)$)
    .. (\tikztotarget)\tikztonodes}},
    settings/.code={\tikzset{quiver/.cd,#1}
        \def\pv##1{\pgfkeysvalueof{/tikz/quiver/##1}}},
    quiver/.cd,pos/.initial=0.35,height/.initial=0}

% TikZ arrowhead/tail styles.
\tikzset{tail reversed/.code={\pgfsetarrowsstart{tikzcd to}}}
\tikzset{2tail/.code={\pgfsetarrowsstart{Implies[reversed]}}}
\tikzset{2tail reversed/.code={\pgfsetarrowsstart{Implies}}}
% TikZ arrow styles.
\tikzset{no body/.style={/tikz/dash pattern=on 0 off 1mm}}
%%%%%%%%%%


%% DEFINIZIONI COMANDI MATEMATICI
\let\sin\relax %TOGLIE LA DEFINIZIONE SU "\sin"

% cambia la definizione di empty set
% ---
\let\oldemptyset\emptyset
% ---
% \let\emptyset\varnothing
% ---
% \let\emptyset\relax
% \newcommand{\emptyset}{\text{\textnormal{\O}}}
% ---

\DeclareMathOperator{\bounded}{bd}
\DeclareMathOperator{\sin}{sen}
\DeclareMathOperator{\epi}{Epi}
\DeclareMathOperator{\cl}{cl}
\DeclareMathOperator{\graph}{graph}
\DeclareMathOperator{\arcsec}{arcsec}
\DeclareMathOperator{\arccot}{arccot}
\DeclareMathOperator{\arccsc}{arccsc}
\DeclareMathOperator{\spettro}{Spettro}
\DeclareMathOperator{\nulls}{nullspace}
\DeclareMathOperator{\dom}{dom}
\DeclareMathOperator{\ar}{ar}
\DeclareMathOperator{\const}{Const}
\DeclareMathOperator{\fun}{Fun}
\DeclareMathOperator{\rel}{Rel}
\DeclareMathOperator{\altezza}{ht}
\let\det\relax %TOGLIE LA DEFINIZIONE SU "\det"
\DeclareMathOperator{\det}{det}
\DeclareMathOperator{\End}{End}
\DeclareMathOperator{\gl}{GL}
\def\Id{\mathrm{Id}}
\def\id{\mathrm{id}}
\DeclareMathOperator{\I}{\mathds{1}}
\DeclareMathOperator{\II}{II}
\DeclareMathOperator{\rank}{rank}
\DeclareMathOperator{\tr}{tr}
\DeclareMathOperator{\tc}{t.c.}
\DeclareMathOperator{\T}{T}
\DeclareMathOperator{\var}{Var}
\DeclareMathOperator{\cov}{Cov}
\DeclareMathOperator{\st}{st}
\DeclareMathOperator{\mon}{Mon}
\newcommand{\card}[1]{\left\vert #1 \right\vert}
\newcommand{\trasposta}[1]{\prescript{\text{T}}{}{#1}}
\newcommand{\1}{\mathds{1}}
\newcommand{\R}{\mathds{R}}
\newcommand{\diesis}{\#}
\newcommand{\bemolle}{\flat}
\newcommand{\nonstandard}[1]{\prescript{*}{}{#1}}
\newcommand{\starR}{\nonstandard{\R}}
\newcommand{\borel}{\mathscr{B}}
\newcommand{\lebesgue}[1]{\mathscr{L}\left(#1\right)}
\newcommand{\media}{\mathds{E}}
\newcommand{\K}{\mathds{K}}
\newcommand{\A}{\mathds{A}}
\newcommand{\Q}{\mathds{Q}}
\newcommand{\N}{\mathds{N}}
\newcommand{\C}{\mathds{C}}
\newcommand{\Z}{\mathds{Z}}
\newcommand{\qo}{\hspace{1em}\text{q.o.}\,}
\renewcommand{\tilde}[1]{\widetilde{#1}}
\renewcommand{\parallel}{\mathrel{/\mkern-5mu/}}
\newcommand{\parti}[2][]{\wp_{#1}(#2)}
\newcommand{\diff}[1]{\operatorname{d}_{#1}}
\let\oldvec\vec
\renewcommand{\vec}[1]{\overrightarrow{\vphantom{i}#1}}
\newcommand{\floor}[1]{\left\lfloor #1 \right\rfloor}
\newcommand{\cat}[1]{\mathbf{#1}}
\newcommand{\dfreccia}[1]{\xrightarrow{\ #1 \ }}
\newcommand{\sfreccia}[1]{\xleftarrow{\ #1 \ }}
\newcommand{\formalsum}[2]{{\sum_{#1}^{#2}}{\vphantom{\sum}}'}
\newcommand{\minim}[2]{\mu_{#1}\, \left(#2\right)}
\newcommand{\concat}{\null^{\frown}} % concatenazione di stringe
\newcommand{\godelcode}[1]{\langle\!\langle #1 \rangle\!\rangle}
\newcommand{\godeldec}[1]{(\!(#1)\!)}
\newcommand{\termcode}[1]{\ulcorner #1\urcorner}
\newcommand{\partialto}{\dashrightarrow}
\newcommand{\restricted}{\upharpoonright}
\newcommand{\embeds}{\precsim}
\newcommand{\surjects}{\twoheadrightarrow}
\newcommand{\equipotenti}{\asymp}
%% \newcommand{\dotplus}{\mathbin{\dot{+}}} %% A quanto pare esiste già
\newcommand{\bigdot}{\mathbin{\boldsymbol{\cdot}}}
\newcommand{\dotexp}[1]{^{.#1}}
\newcommand{\conv}{\mathbin{*}}
\newcommand{\convolution}[2]{(#1\conv #2)}
\newcommand{\nil}{\mathfrak{N}}
\newcommand{\divisore}{\mathrel{|}}
\newcommand{\simplesso}[1]{\mathrm{e}_{#1}}

\renewcommand{\iff}{\mathrel{\longleftrightarrow}} %% Notazione Logica.
\newcommand{\oldiff}{\mathrel{\Longleftrightarrow}}
\renewcommand{\implies}{\mathrel{\rightarrow}} %% Notazione Logica
\newcommand{\oldimplies}{\mathrel{\Longrightarrow}}
\renewcommand{\impliedby}{\mathrel{\leftarrow}} %% Notazione Logica
\newcommand{\oldimpliedby}{\mathrel{\Longleftarrow}}

\newcommand{\IFF}{\quad\Longleftrightarrow\quad}
\newcommand{\IMPLICA}{\quad\Longrightarrow\quad}


\renewcommand{\descriptionlabel}[1]{\hspace{\labelsep}\normalfont #1} % remove bold from description


%% Definizione di Divergenza di K-L

\DeclarePairedDelimiterX{\infdivx}[2]{(}{)}{%
  #1\;\delimsize\|\;#2%
}
\newcommand{\kldiv}{D_{KL}\infdivx}

%% Definizione di \dotminus

\makeatletter
\newcommand{\dotminus}{\mathbin{\text{\@dotminus}}}

\newcommand{\@dotminus}{%
  \ooalign{\hidewidth\raise1ex\hbox{.}\hidewidth\cr$\m@th-$\cr}%
}
\makeatother

%tramite i prossimi due comandi posso decidere come scrivere i logaritmi naturali in tutti i documenti: ho infatti eliminato qualsiasi differenza tra "ln" e "log": se si vuole qualcosa di diverso bisogna inserire manualmente il tutto
\let\ln\relax
\DeclareMathOperator{\ln}{ln}
\let\log\relax
\DeclareMathOperator{\log}{log}
%%%%%%

%% NUOVI COMANDI
\newcommand{\straniero}[1]{\textit{#1}} %parole straniere
\newcommand{\titolo}[1]{\textsc{#1}} %titoli
\newcommand{\qedd}{\tag*{$\blacksquare$}} %qed per ambienti matemastici
\renewcommand{\qedsymbol}{$\blacksquare$} %modifica colore qed
\newcommand{\ooverline}[1]{\overline{\overline{#1}}}
\newcommand{\circoletto}[1]{\left(#1\right)^{\text{o}}}
%
\newcommand{\qmatrice}[1]{\begin{pmatrix}
#1_{11} & \cdots & #1_{1n}\\
\vdots & \ddots & \vdots \\
#1_{m1} & \cdots & #1_{mn}
\end{pmatrix}}
%
\newcommand{\parentesi}[2]{%
\underset{#1}{\underbrace{#2}}%
}
%
\newcommand{\norma}[1]{% Norma
\left\lVert#1\right\rVert%
}
\newcommand{\scalare}[2]{% Scalare
\left\langle #1, #2\right\rangle
}
%%%%%

%% RESTRIZIONI
\newcommand{\referenze}[2]{
        \phantomsection{}#2\textsuperscript{\textcolor{blue}{\textbf{#1}}}
}

\let\restriction\relax

\def\restriction#1#2{\mathchoice
              {\setbox1\hbox{${\displaystyle #1}_{\scriptstyle #2}$}
              \restrictionaux{#1}{#2}}
              {\setbox1\hbox{${\textstyle #1}_{\scriptstyle #2}$}
              \restrictionaux{#1}{#2}}
              {\setbox1\hbox{${\scriptstyle #1}_{\scriptscriptstyle #2}$}
              \restrictionaux{#1}{#2}}
              {\setbox1\hbox{${\scriptscriptstyle #1}_{\scriptscriptstyle #2}$}
              \restrictionaux{#1}{#2}}}
\def\restrictionaux#1#2{{#1\,\smash{\vrule height .8\ht1 depth .85\dp1}}_{\,#2}}
%%%%%%%%%%%

%%% FORMATTAZIONE FOOTNOTEMARK

\def\footnotemarkformatting#1{[#1]}
\renewcommand{\thefootnote}{\footnotemarkformatting{\arabic{footnote}}}

%% SEZIONE GRAFICA
\use{tikz}
\usetikzlibrary{matrix, patterns, calc, decorations.pathreplacing, hobby, decorations.markings, decorations.pathmorphing, babel}
\use{tikz-3dplot}
\use{mathrsfs} %per geogebra
\use{tikz-cd}
\tikzset
{
  %surface/.style={fill=black!10, shading=ball,fill opacity=0.4},
  plane/.style={black,pattern=north east lines},
  curve/.style={black,line width=0.5mm},
  dritto/.style={decoration={markings,mark=at position 0.5 with {\arrow{Stealth}}}, postaction=decorate},
  rovescio/.style={decoration={markings,mark=at position 0.5 with {\arrow{Stealth[reversed]}}}, postaction=decorate}
}
\use{pgfplots} % stampare le funzioni
        \pgfplotsset{/pgf/number format/use comma,compat=1.15}
        %\pgfplotsset{compat=1.15} %per geogebra
        \usepgfplotslibrary{fillbetween, polar}
%%%%%%

%% CITAZIONI
\use{lineno}

\newcommand{\citazione}[1]{%
  \begin{quotation}
  \begin{linenumbers}
  \modulolinenumbers[5]
  \begingroup
  \setlength{\parindent}{0cm}
  \noindent #1
  \endgroup
  \end{linenumbers}
  \end{quotation}\setcounter{linenumber}{1}
  }
%%%%%%

%%%%%%%%%%%%%%%%%%%%%%%%%%%%%%%%%%%%%%%%%%%%
%%%%%%%%%%%%%%%%%%%%%%%%%%%%%%%%%%%%%%%%%%%%

%% AMS THM

\theoremstyle{definition}% default
\newtheorem{thm}{Teorema}[section]
\newtheorem{lem}[thm]{Lemma}
\newtheorem{prop}[thm]{Proposizione}
\newtheorem{cor}[thm]{Corollario}
\newtheorem{esempio}[thm]{Esempio}
\theoremstyle{plain}
\newtheorem{definizione}[thm]{Definizione}
\theoremstyle{remark}
\newtheorem*{oss}{Osservazione}


%%%%%%%%%%%%%%%%%%%%%%%%%%%%%%%%%%%%%%%%%%%%
%%%%%%%%%%%%%%%%%%%%%%%%%%%%%%%%%%%%%%%%%%%%

\use{hyperref}
\hypersetup{%
        pdfauthor={Davide Peccioli},
        pdfsubject={},
        allcolors=black,
        citecolor=black,
%	colorlinks=true,
        bookmarksopen=true}
\setcounter{secnumdepth}{0} % rimuove i numeri di sezione senza rimuovere le ref
\renewcommand{\href}[2]{\textcolor{blue}{#2}} % disabilita il comando href
\use{enotez} %
\setenotez{%
 mark-format = \footnotemarkformatting % Mette i numeri tra parentesi quadre%
}\let\footnote=\endnote % rende tutte le note a pié pagina come delle note a fine file 


\let\olddocument\document % modifico l'ambiende documenti per non dover stampare \printendnote
\let\oldenddocument\enddocument
\renewenvironment{document}%
{%
  \olddocument
}{%
  \printendnotes\oldenddocument
}
\renewcommand{\thethm}{\arabic{thm}}

\usepackage[hyperref]{biblatex}
\addbibresource{~/Documents/org/roam/bib/master.bib}
\author{Davide Peccioli}
\date{\today}
\title{Catena di implicazioni uniformi MaxHaus, Zorn e wZorn}
\begin{document}

Contesto: \href{20250130104245-morse_kelly_set_theory.org}{Morse Kelly Set Theory}
\section{Proposizione}
\label{sec:orgc40f3e8}
Sia \(X\) un \href{20250130104331-insieme_mk.org}{insieme} \href{20250131161811-insieme_vuoto_mk.org}{non vuoto}.

\(X\) è \href{20250203161431-classe_ben_ordinabile_mk.org}{ben ordinabile} implica \(\textsc{MaxHaus}(X)\) implica \(\textsc{Zorn}(X)\) implica \(\textsc{wZorn}(X)\) .
(vedi \href{20250210104707-principio_di_massimalita_di_hausdorff.org}{MaxHaus}, \href{20250210104633-lemma_di_zorn.org}{Zorn Lemma} e \href{20250210104633-lemma_di_zorn.org}{Weak Zorn Lemma})

Inoltre \(\textsc{wZorn}\left(\parti{X\times X}\right)\) implica \(X\) ben ordinabile (vedi \href{20250130104245-morse_kelly_set_theory.org}{Insieme delle parti MK} e \href{20250131183735-prodotto_cartesiano_di_classi_mk.org}{Prodotto cartesiano di classi MK})
\subsection{Dimostrazione}
\label{sec:orgbf3826d}

\subsubsection{Punto 1.}
\label{sec:org6dd0e2d}
Supponiamo che \(X\) sia ben ordinabile, e supponiamo per assurdo che \(\le\) sia un \href{20250203104134-buon_ordine_mk.org}{buon ordine} senza \href{20250102120836-catena.org}{catene massimali}.

Sia dunque \(C \subseteq X\) una \href{20250102120836-catena.org}{catena}, e si definisca l'insieme
\begin{equation*}
K(C) \coloneqq \set{x \in X\setminus C\ |\ C\cup\set{x}\text{ è una catena}}
\end{equation*}
Questo è non vuoto poiché \(C\) non è una catena massimale.

Sia \(F:\parti{X}\setminus\set{\emptyset}\to X\) una \href{20250203105434-funzione_di_scelta.org}{funzione di scelta} fissata (\href{20250210104534-ac_e_classi_ben_ordinabili.org}{che esiste}). Allora la funzione (vedi \href{20250205152531-numeri_di_hartogs.org}{Numeri di Hartogs})
\begin{equation*}
g: \operatorname{Hrtg}(X)\to X:\quad \alpha\mapsto F\left(
K\left(\set{g(\beta)\ |\ \beta<\alpha}\right)
\right)
\end{equation*}
che esiste per il \href{20250207121906-teorema_di_ricorsione.org}{Teorema di Ricorsione}, è iniettiva\footnote{Bisogna dimostrare che sia iniettiva}. \href{20250205152531-numeri_di_hartogs.org}{Assurdo}.
\subsubsection{Punto 2.}
\label{sec:org17d4f85}
Supponiamo ora \(\textsc{MaxHaus}(X)\), e sia \(\le\) un \href{20250203101604-ordine.org}{ordine} su \(X\) tale che ogni \href{20250102120836-catena.org}{catena} abbia un \href{20250203102516-massimo_e_minimo.org}{estremo superiore}.
Sia \(C \subseteq X\) è una \href{20250102120836-catena.org}{catena massimale}, allora l'estremo superiore di \(C\) appartiene a \(C\), e pertanto è un \href{20250203102516-massimo_e_minimo.org}{elemento massimale} di \(X\).
\subsubsection{Punto 3.}
\label{sec:orgffb6ac1}
L'implicazione \href{20250210104633-lemma_di_zorn.org}{Zorn} implica \href{20250210104633-lemma_di_zorn.org}{wZorn} è immediata.
\subsubsection{Punto 4.}
\label{sec:org520432a}
Sia \(\mathcal{P} \subseteq \parti{X\times X}\) l'insieme di tutti i buoni ordini \(R\) su \(X\) tali che \(\operatorname{fld}(R) \subseteq X\) (vedi \href{20250202173528-dominio_range_e_campo_di_una_classe_relazione.org}{Dominio, Range e Campo di una Classe Relazione}).

Se \(R \in \mathcal{P}\) allora \(R\) è un \href{20250203104134-buon_ordine_mk.org}{buon ordine} su un \href{20250131155822-operazioni_insiemistiche_tra_classi_mk.org}{sottoinsieme} \(\operatorname{fld}(R) \subseteq X\). \(\mathcal{P}\neq \emptyset\) poiché \href{20250203161431-classe_ben_ordinabile_mk.org}{ogni insieme finito è ben ordinabile} (vedi anche \href{20250203161341-cardinali.org}{Numeri naturali sono cardinali})

Per \(R, S \in \mathcal{P}\), si ponga \(R\trianglelefteq S\) se e solo se
\begin{equation*}
\exists\, a \in \operatorname{fld}S\ \left[\operatorname{fld}(R) = \operatorname{pred}(a; S) \,\land\, R = S\cap \operatorname{fld}(R)\times \operatorname{fld}(R) \right]
\end{equation*}
(vedi \href{20250206120526-segmento_iniziale_per_un_ordine.org}{Insieme dei predecessori} e \href{20250131183735-prodotto_cartesiano_di_classi_mk.org}{Prodotto cartesiano di classi MK})

Per \(\textsc{wZorn}\left(\parti{X\times X}\right)\) esiste\footnote{\(\textsc{wZorn}\left(\parti{X\times X}\right)\) non richiede che l'ordine sia totale, e pertanto si può applicare a \(\trianglelefteq\). Inoltre, ovviamente, un elemento massimale di \(\parti{X\times X}\) rispetto a \(\trianglelefteq\) è necessariamente in \(\mathcal{P}\), poiché l'ordine è definito solo lì.
Bisogna dimostrare che ogni insieme superiormente diretto di \(\mathcal{P}\) abbia un estremo superiore. Se \(R \subseteq \mathcal{P}\) allora \(\bigcup R\) è l'estremo superiore cercato.} \(\overline{R} \in \mathcal{P}\) elemento \(\trianglelefteq\)-\href{20250203102516-massimo_e_minimo.org}{massimale}. Se \(\operatorname{fld}(\overline{R}) = X\) allora \(\overline{R}\) è un \href{20250203104134-buon_ordine_mk.org}{buon ordine}\footnote{Infatti \(\overline{R}\) è un buon ordine su un sottoinsieme di \(X\). Se inoltre per ogni \(x \in X\) esiste \(y \in X\) tale che \((x,y) \in R\) oppure \((y,x) \in R\) si ha che questo sia un buon ordine su \(X\) poiché è \href{20250203101604-ordine.org}{totale}.} su \(X\).

Supponiamo per assurdo che \(\operatorname{fld}(\overline{R}) \neq X\). Sia dunque \(a \in X\setminus\operatorname{fld}(\overline{R})\), e si consideri (vedi \href{20250131162451-coppia_ordinata_mk.org}{Coppia ordinata MK})
\begin{equation*}
S \coloneqq \overline{R}\cup \set{
(y,a)\ |\ y \in \operatorname{fld}(\overline{R})
}\cup\set{(a,a)}
\end{equation*}

Allora\footnote{Per dimostrare che \(S \in \mathcal{P}\) bisogna dimostrare che \(S\) sia \href{20250203095749-relazione_left_narrow_mk.org}{left-narrow} e \href{20250203100901-relazione_well_founded_mk.org}{well-founded} e che sia \href{20250203101604-ordine.org}{totale} su \(\operatorname{fld}(S)\subseteq X\).
\begin{itemize}
\item Sia \(x \in \operatorname{fld}(S)\). Allora, se \(x\neq a\)
\begin{equation*}
  \set{y \in X\ |\ (y,x) \in S} = \set{y \in X\ |\ (y,x) \in \overline{R}} \
\end{equation*}
che è un insieme poiché \(\overline{R}\) è left-narrow siccome buon ordine.
Se invece \(x=a\) allora
\begin{equation*}
  \set{y \in X\ |\ (y,x) \in S} = \operatorname{fld}(\overline{R})\cup\set{a}
\end{equation*}
che è un insieme \href{20250202173528-dominio_range_e_campo_di_una_classe_relazione.org}{poiché} \(\operatorname{fld}(\overline{R})\) è un insieme e per l'Axiom of Union (vedi \href{20250131155822-operazioni_insiemistiche_tra_classi_mk.org}{Classe Unione Generalizzata})
\item Osserviamo che \(\operatorname{fld}(S) = \operatorname{fld}(\overline{R})\cup\set{a}\). Sia dunque \(Y \subseteq \operatorname{fld}(S)\). Sia \(\overline{y}\) l'elemento \(\overline{R}\)-minimale di \(Y\setminus \set{a}\) (dunque \(\overline{y}\neq a\)), e sia \(y \in Y\), \(y\neq \overline{y}\). Si deve dimostrare che \((y,\overline{y})\notin S\). Supponiamo per assurdo che \((y,\overline{y}) \in S\).
Siccome \(\overline{y}\neq a\), si deve avere \((y,\overline{y}) \in \overline{R}\). Ma questo è assurdo. Infatti, se \(y = a\) allora \((a,\overline{y})\notin \overline{R}\) poiché \(a \notin \operatorname{fld}(\overline{R})\). Se invece \(y\neq a\) allora \((y,\overline{y})\notin \overline{R}\) per \(\overline{R}\)-minimalità di \(\overline{y}\) rispetto a \(Y\setminus\set{a}\).
\end{itemize}
Inoltre, sicuramente \(\overline{R} =  S\cap \operatorname{fld}(\overline{R})\times\operatorname{fld}(\overline{R})\) in quando, siccome \(a\notin \operatorname{fld}(\overline{R})\).
Si ha anche che \(\operatorname{fld}(\overline{R}) = \operatorname{pred}(a;S)\), dove
\begin{equation*}
\operatorname{pred}(a;S) = \set{y \in X\ |\ (y,a) \in S}
\end{equation*}
Infatti, se \(y_{0} \in \operatorname{pred}(a;S)\) allora \((y_{0},a) \in S\). Siccome \(a \notin\operatorname{fld}(\overline{R})\), si che che
\begin{equation*}
(y_{0},a) \in \set{(y,a)\ |\ y \in \operatorname{fld}(\overline{R})}\cup\set{(a,a)}
\end{equation*}
Allora o \(y_{0} \in \operatorname{fld}(\overline{R})\) oppure \(y_{0} = a\). ????
Se invece \(y_{0} \in \operatorname{fld}(\overline{R})\) allora \((y_{0},a) \in S\) e dunque \((y_{0},a) \in \operatorname{pred}(a;S)\).} \(S \in \mathcal{P}\) e \(\overline{R}\trianglelefteq S\), contro la massimalità di \(\overline{R}\). Assurdo
\end{document}
