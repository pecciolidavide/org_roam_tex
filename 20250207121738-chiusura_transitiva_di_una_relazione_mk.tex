% Created 2026-02-07 Sat 19:32
% Intended LaTeX compiler: pdflatex
\documentclass[10pt]{article}
%% CREATO CON ORG - EMACS
\newcommand{\use}[2][]{\usepackage[#1]{#2}}
% PACCHETTI FONDAMENTLAI
\use[utf8]{inputenc}
\use[T1]{fontenc}
\use{graphicx}
\use{longtable}
\use{wrapfig}
\use{rotating}
\use[normalem]{ulem}
\use{amsmath}
\use{amsthm}
\use{amssymb}

\use{eucal} % Cambia mathcal{...}

\use{capt-of}
\use[italian]{babel}
\use[babel]{csquotes}
% bib la TEX lo carica in automatico org-cite
\use{microtype}
\use{lmodern}
\use{subfig} % sottofigure
\use{multicol} % due colonne
\use{lipsum} % lorem ipsum
\use{color} % colori in latex
\use{parskip} % rimuove l'indentazione dei nuovi paragrafi %% Add parbox=false to all new tcolorbox
\use{centernot}
\use[outline]{contour}\contourlength{3pt}
\use{fancyhdr}
\use{layout}
\use[most]{tcolorbox} % Riquadri colorati
\use{ifthen} % IFTHEN
\use{geometry}

% pacchetti matematica
\use{yhmath}
\use{dsfont}
\use{mathrsfs}
\use{cancel} % semplificare
\use{polynom} %divisione tra polinomi
\use{forest} % grafi ad albero
\use{booktabs} % tabelle
\use{commath} %simboli e differenziali
\use{bm} %bold
\use[fulladjust]{marginnote} %to use marginnote for date notes
\use{arrayjobx}%array
\use[intlimits]{empheq} % Riquadri colorati attorno alle equazioni
\use{mathtools}
\use{circuitikz} % Disegnare i circuiti
\use{mathtools}
\use{stmaryrd} % [[ \llbracket ]] \rrbracket
\use{bussproofs} % dimostrazioni

%%%%%%%%%%%%%


%%%% QUIVER
\newcommand{\duepunti}{\,\mathchar\numexpr"6000+`:\relax\,}
% A TikZ style for curved arrows of a fixed height, due to AndréC.
\tikzset{curve/.style={settings={#1},to path={(\tikztostart)
    .. controls ($(\tikztostart)!\pv{pos}!(\tikztotarget)!\pv{height}!270:(\tikztotarget)$)
    and ($(\tikztostart)!1-\pv{pos}!(\tikztotarget)!\pv{height}!270:(\tikztotarget)$)
    .. (\tikztotarget)\tikztonodes}},
    settings/.code={\tikzset{quiver/.cd,#1}
        \def\pv##1{\pgfkeysvalueof{/tikz/quiver/##1}}},
    quiver/.cd,pos/.initial=0.35,height/.initial=0}

% TikZ arrowhead/tail styles.
\tikzset{tail reversed/.code={\pgfsetarrowsstart{tikzcd to}}}
\tikzset{2tail/.code={\pgfsetarrowsstart{Implies[reversed]}}}
\tikzset{2tail reversed/.code={\pgfsetarrowsstart{Implies}}}
% TikZ arrow styles.
\tikzset{no body/.style={/tikz/dash pattern=on 0 off 1mm}}
%%%%%%%%%%


%% DEFINIZIONI COMANDI MATEMATICI
\let\sin\relax %TOGLIE LA DEFINIZIONE SU "\sin"

% cambia la definizione di empty set
% ---
\let\oldemptyset\emptyset
% ---
% \let\emptyset\varnothing
% ---
% \let\emptyset\relax
% \newcommand{\emptyset}{\text{\textnormal{\O}}}
% ---

\DeclareMathOperator{\bounded}{bd}
\DeclareMathOperator{\sin}{sen}
\DeclareMathOperator{\epi}{Epi}
\DeclareMathOperator{\cl}{cl}
\DeclareMathOperator{\graph}{graph}
\DeclareMathOperator{\arcsec}{arcsec}
\DeclareMathOperator{\arccot}{arccot}
\DeclareMathOperator{\arccsc}{arccsc}
\DeclareMathOperator{\spettro}{Spettro}
\DeclareMathOperator{\nulls}{nullspace}
\DeclareMathOperator{\dom}{dom}
\DeclareMathOperator{\ar}{ar}
\DeclareMathOperator{\const}{Const}
\DeclareMathOperator{\fun}{Fun}
\DeclareMathOperator{\rel}{Rel}
\DeclareMathOperator{\altezza}{ht}
\let\det\relax %TOGLIE LA DEFINIZIONE SU "\det"
\DeclareMathOperator{\det}{det}
\DeclareMathOperator{\End}{End}
\DeclareMathOperator{\gl}{GL}
\def\Id{\mathrm{Id}}
\def\id{\mathrm{id}}
\DeclareMathOperator{\I}{\mathds{1}}
\DeclareMathOperator{\II}{II}
\DeclareMathOperator{\rank}{rank}
\DeclareMathOperator{\tr}{tr}
\DeclareMathOperator{\tc}{t.c.}
\DeclareMathOperator{\T}{T}
\DeclareMathOperator{\var}{Var}
\DeclareMathOperator{\cov}{Cov}
\DeclareMathOperator{\st}{st}
\DeclareMathOperator{\mon}{Mon}
\newcommand{\card}[1]{\left\vert #1 \right\vert}
\newcommand{\trasposta}[1]{\prescript{\text{T}}{}{#1}}
\newcommand{\1}{\mathds{1}}
\newcommand{\R}{\mathds{R}}
\newcommand{\diesis}{\#}
\newcommand{\bemolle}{\flat}
\newcommand{\nonstandard}[1]{\prescript{*}{}{#1}}
\newcommand{\starR}{\nonstandard{\R}}
\newcommand{\borel}{\mathscr{B}}
\newcommand{\lebesgue}[1]{\mathscr{L}\left(#1\right)}
\newcommand{\media}{\mathds{E}}
\newcommand{\K}{\mathds{K}}
\newcommand{\A}{\mathds{A}}
\newcommand{\Q}{\mathds{Q}}
\newcommand{\N}{\mathds{N}}
\newcommand{\C}{\mathds{C}}
\newcommand{\Z}{\mathds{Z}}
\newcommand{\qo}{\hspace{1em}\text{q.o.}\,}
\renewcommand{\tilde}[1]{\widetilde{#1}}
\renewcommand{\parallel}{\mathrel{/\mkern-5mu/}}
\newcommand{\parti}[2][]{\wp_{#1}(#2)}
\newcommand{\diff}[1]{\operatorname{d}_{#1}}
\let\oldvec\vec
\renewcommand{\vec}[1]{\overrightarrow{\vphantom{i}#1}}
\newcommand{\floor}[1]{\left\lfloor #1 \right\rfloor}
\newcommand{\cat}[1]{\mathbf{#1}}
\newcommand{\dfreccia}[1]{\xrightarrow{\ #1 \ }}
\newcommand{\sfreccia}[1]{\xleftarrow{\ #1 \ }}
\newcommand{\formalsum}[2]{{\sum_{#1}^{#2}}{\vphantom{\sum}}'}
\newcommand{\minim}[2]{\mu_{#1}\, \left(#2\right)}
\newcommand{\concat}{\null^{\frown}} % concatenazione di stringe
\newcommand{\godelcode}[1]{\langle\!\langle #1 \rangle\!\rangle}
\newcommand{\godeldec}[1]{(\!(#1)\!)}
\newcommand{\termcode}[1]{\ulcorner #1\urcorner}
\newcommand{\partialto}{\dashrightarrow}
\newcommand{\restricted}{\upharpoonright}
\newcommand{\embeds}{\precsim}
\newcommand{\surjects}{\twoheadrightarrow}
\newcommand{\equipotenti}{\asymp}
%% \newcommand{\dotplus}{\mathbin{\dot{+}}} %% A quanto pare esiste già
\newcommand{\bigdot}{\mathbin{\boldsymbol{\cdot}}}
\newcommand{\dotexp}[1]{^{.#1}}
\newcommand{\conv}{\mathbin{*}}
\newcommand{\convolution}[2]{(#1\conv #2)}
\newcommand{\nil}{\mathfrak{N}}
\newcommand{\divisore}{\mathrel{|}}
\newcommand{\simplesso}[1]{\mathrm{e}_{#1}}

\renewcommand{\iff}{\mathrel{\longleftrightarrow}} %% Notazione Logica.
\newcommand{\oldiff}{\mathrel{\Longleftrightarrow}}
\renewcommand{\implies}{\mathrel{\rightarrow}} %% Notazione Logica
\newcommand{\oldimplies}{\mathrel{\Longrightarrow}}
\renewcommand{\impliedby}{\mathrel{\leftarrow}} %% Notazione Logica
\newcommand{\oldimpliedby}{\mathrel{\Longleftarrow}}

\newcommand{\IFF}{\quad\Longleftrightarrow\quad}
\newcommand{\IMPLICA}{\quad\Longrightarrow\quad}


\renewcommand{\descriptionlabel}[1]{\hspace{\labelsep}\normalfont #1} % remove bold from description


%% Definizione di Divergenza di K-L

\DeclarePairedDelimiterX{\infdivx}[2]{(}{)}{%
  #1\;\delimsize\|\;#2%
}
\newcommand{\kldiv}{D_{KL}\infdivx}

%% Definizione di \dotminus

\makeatletter
\newcommand{\dotminus}{\mathbin{\text{\@dotminus}}}

\newcommand{\@dotminus}{%
  \ooalign{\hidewidth\raise1ex\hbox{.}\hidewidth\cr$\m@th-$\cr}%
}
\makeatother

%tramite i prossimi due comandi posso decidere come scrivere i logaritmi naturali in tutti i documenti: ho infatti eliminato qualsiasi differenza tra "ln" e "log": se si vuole qualcosa di diverso bisogna inserire manualmente il tutto
\let\ln\relax
\DeclareMathOperator{\ln}{ln}
\let\log\relax
\DeclareMathOperator{\log}{log}
%%%%%%

%% NUOVI COMANDI
\newcommand{\straniero}[1]{\textit{#1}} %parole straniere
\newcommand{\titolo}[1]{\textsc{#1}} %titoli
\newcommand{\qedd}{\tag*{$\blacksquare$}} %qed per ambienti matemastici
\renewcommand{\qedsymbol}{$\blacksquare$} %modifica colore qed
\newcommand{\ooverline}[1]{\overline{\overline{#1}}}
\newcommand{\circoletto}[1]{\left(#1\right)^{\text{o}}}
%
\newcommand{\qmatrice}[1]{\begin{pmatrix}
#1_{11} & \cdots & #1_{1n}\\
\vdots & \ddots & \vdots \\
#1_{m1} & \cdots & #1_{mn}
\end{pmatrix}}
%
\newcommand{\parentesi}[2]{%
\underset{#1}{\underbrace{#2}}%
}
%
\newcommand{\norma}[1]{% Norma
\left\lVert#1\right\rVert%
}
\newcommand{\scalare}[2]{% Scalare
\left\langle #1, #2\right\rangle
}
%%%%%

%% RESTRIZIONI
\newcommand{\referenze}[2]{
        \phantomsection{}#2\textsuperscript{\textcolor{blue}{\textbf{#1}}}
}

\let\restriction\relax

\def\restriction#1#2{\mathchoice
              {\setbox1\hbox{${\displaystyle #1}_{\scriptstyle #2}$}
              \restrictionaux{#1}{#2}}
              {\setbox1\hbox{${\textstyle #1}_{\scriptstyle #2}$}
              \restrictionaux{#1}{#2}}
              {\setbox1\hbox{${\scriptstyle #1}_{\scriptscriptstyle #2}$}
              \restrictionaux{#1}{#2}}
              {\setbox1\hbox{${\scriptscriptstyle #1}_{\scriptscriptstyle #2}$}
              \restrictionaux{#1}{#2}}}
\def\restrictionaux#1#2{{#1\,\smash{\vrule height .8\ht1 depth .85\dp1}}_{\,#2}}
%%%%%%%%%%%

%%% FORMATTAZIONE FOOTNOTEMARK

\def\footnotemarkformatting#1{[#1]}
\renewcommand{\thefootnote}{\footnotemarkformatting{\arabic{footnote}}}

%% SEZIONE GRAFICA
\use{tikz}
\usetikzlibrary{matrix, patterns, calc, decorations.pathreplacing, hobby, decorations.markings, decorations.pathmorphing, babel}
\use{tikz-3dplot}
\use{mathrsfs} %per geogebra
\use{tikz-cd}
\tikzset
{
  %surface/.style={fill=black!10, shading=ball,fill opacity=0.4},
  plane/.style={black,pattern=north east lines},
  curve/.style={black,line width=0.5mm},
  dritto/.style={decoration={markings,mark=at position 0.5 with {\arrow{Stealth}}}, postaction=decorate},
  rovescio/.style={decoration={markings,mark=at position 0.5 with {\arrow{Stealth[reversed]}}}, postaction=decorate}
}
\use{pgfplots} % stampare le funzioni
        \pgfplotsset{/pgf/number format/use comma,compat=1.15}
        %\pgfplotsset{compat=1.15} %per geogebra
        \usepgfplotslibrary{fillbetween, polar}
%%%%%%

%% CITAZIONI
\use{lineno}

\newcommand{\citazione}[1]{%
  \begin{quotation}
  \begin{linenumbers}
  \modulolinenumbers[5]
  \begingroup
  \setlength{\parindent}{0cm}
  \noindent #1
  \endgroup
  \end{linenumbers}
  \end{quotation}\setcounter{linenumber}{1}
  }
%%%%%%

%%%%%%%%%%%%%%%%%%%%%%%%%%%%%%%%%%%%%%%%%%%%
%%%%%%%%%%%%%%%%%%%%%%%%%%%%%%%%%%%%%%%%%%%%

%% AMS THM

\theoremstyle{definition}% default
\newtheorem{thm}{Teorema}[section]
\newtheorem{lem}[thm]{Lemma}
\newtheorem{prop}[thm]{Proposizione}
\newtheorem{cor}[thm]{Corollario}
\newtheorem{esempio}[thm]{Esempio}
\theoremstyle{plain}
\newtheorem{definizione}[thm]{Definizione}
\theoremstyle{remark}
\newtheorem*{oss}{Osservazione}


%%%%%%%%%%%%%%%%%%%%%%%%%%%%%%%%%%%%%%%%%%%%
%%%%%%%%%%%%%%%%%%%%%%%%%%%%%%%%%%%%%%%%%%%%

\use{hyperref}
\hypersetup{%
        pdfauthor={Davide Peccioli},
        pdfsubject={},
        allcolors=black,
        citecolor=black,
%	colorlinks=true,
        bookmarksopen=true}
\setcounter{secnumdepth}{0} % rimuove i numeri di sezione senza rimuovere le ref
\renewcommand{\href}[2]{\textcolor{blue}{#2}} % disabilita il comando href
\use{enotez} %
\setenotez{%
 mark-format = \footnotemarkformatting % Mette i numeri tra parentesi quadre%
}\let\footnote=\endnote % rende tutte le note a pié pagina come delle note a fine file 


\let\olddocument\document % modifico l'ambiende documenti per non dover stampare \printendnote
\let\oldenddocument\enddocument
\renewenvironment{document}%
{%
  \olddocument
}{%
  \printendnotes\oldenddocument
}
\renewcommand{\thethm}{\arabic{thm}}

\usepackage[hyperref]{biblatex}
\addbibresource{~/Documents/org/roam/bib/master.bib}
\author{Davide Peccioli}
\date{\today}
\title{Chiusura transitiva di una relazione}
\begin{document}

\textbf{\textbf{\uline{NOTA}: quando si parla di \uline{classi}, se ne parla nell'ambito della \href{20250130104245-morse_kelly_set_theory.org}{Morse Kelly Set Theory}; quando si parla di insiemi, il discorso ha validità più generale.}}
\section{Definizione}
\label{sec:org1f27771}
Sia \(X\) un \href{20250130104331-insieme_mk.org}{insieme} (o una \href{20250130104320-classe_mk.org}{classe}), e sia \(R\) una \href{20250202170607-classe_relazione_binaria.org}{relazione} su \(X\).

La chiusura transitiva di \(R\) è la \href{20250202170607-classe_relazione_binaria.org}{relazione}:\footnote{Vedi:
\begin{itemize}
\item \href{20250131162451-coppia_ordinata_mk.org}{Coppia ordinata}
\item \href{20250131183735-prodotto_cartesiano_di_classi_mk.org}{Prodotto cartesiano}
\item \(\operatorname{S}(n)\): \href{20250202124648-successore_di_un_insieme_mk.org}{Successore di un numero naturale} e \href{20250202130045-insieme_dei_numeri_naturali_mk.org}{Insieme dei numeri naturali}
\item \(\prescript{\operatorname{S}(n)}{}{X}\): \href{20250202192030-classe_delle_classi_funzioni.org}{Insieme delle funzioni}
\end{itemize}}
\begin{align*}
\tilde{R} &= \left\{
(x,y) \in X\times X\ |\ \exists\, n>0\ \exists\, f \in \prescript{\operatorname{S}(n)}{}{X}\ \left[ x = f(0) \,\land\,
\right.\right.\\ &\qquad\quad \left.\left.
y = f(n) \,\land\,  \forall\, i < n\ \left[ f(i), f\left(\operatorname{S}(i)\right) \in R
\right]\right]\right\}.
\end{align*}

In altre parole, \(x\mathrel{\tilde{R}}y\) se e solo se esistono \(x_{0},\dots,x_{n} \in X\) tali che
\begin{equation*}
x=x_{0}\mathrel{R}x_{1},\ x_{1}\mathrel{R}x_{2}\ \dots,\ x_{n-1}\mathrel{R}x_{n} = y
\end{equation*}
\subsection{Osservazione}
\label{sec:orgbb80a26}
Per costruzione \(\tilde{R}\) è la più piccola \href{20250202170607-classe_relazione_binaria.org}{relazione} \href{20250203110714-classe_transitiva.org}{transitiva} su \(X\) tale che\footnote{Vedi ``\href{20250131155822-operazioni_insiemistiche_tra_classi_mk.org}{Sottoclasse MK}''} \(R \subseteq \tilde{R}\)
\section{Proprietà}
\label{sec:orgbda475e}
Sia \(X\) una \href{20250130104320-classe_mk.org}{classe}, e sia \(R\) una \href{20250202170607-classe_relazione_binaria.org}{relazione} su \(X\). Sia \(\tilde{R}\) la sua \href{20250207121738-chiusura_transitiva_di_una_relazione_mk.org}{chiusura transitiva}.

\begin{enumerate}
\item \(R\) è \href{20250203095749-relazione_left_narrow_mk.org}{regolare} su \(X\) se e solo se \(\tilde{R}\) è \href{20250203095749-relazione_left_narrow_mk.org}{regolare} su \(X\).
\item \(R\) è \href{20250203100901-relazione_well_founded_mk.org}{ben fondata} su \(X\) se e solo se \(\tilde{R}\) è \href{20250203100901-relazione_well_founded_mk.org}{ben fondata} su \(X\).
\end{enumerate}
\subsection{Dimostrazione}
\label{sec:orgd39c39a}
\subsubsection{Dimostrazione 1}
\label{sec:org4517ec3}

Siccome \(R \subseteq \tilde{R}\) è sufficiente dimostrare che \(\tilde{R}\) è left-narrow se lo è \(R\), dal momento che il viceversa è immediato.
\paragraph{Claim}
\label{sec:org1f2772a}

Sia \(\overline{x} \in X\) fissato. Esiste una \href{20250206170922-sequenze_e_stringhe.org}{sequenza} di \href{20250130104331-insieme_mk.org}{insiemi} \(\langle Z_{n}\ |\ n \in \omega\rangle\) indicizzata da \href{20250203161110-numeri_naturali_sono_ordinali.org}{\(\omega\)} tali che:
\begin{align*}
Z_{0} &= \set{y \in X\ |\ y\mathrel{R}\overline{x}}\\
Z_{n+1} &=  \set{y \in X\ |\ \exists\, z \in Z_{n}\ (y\mathrel{R}z)} = \bigcup_{z \in Z_{n}} \set{y \in X\ |\ y\mathrel{R}z}
\end{align*}
\paragraph{Dimostrazione del Claim}
\label{sec:orgcb93315}

Si applichi il \href{20250207121712-teorema_di_ricorsione_caso_speciale.org}{Teorema di ricorsione (caso speciale)}, con
\begin{align*}
A &= V\\
\overline{a} &= Z_{0}\\
F(n,a) &=\set{x \in X\ |\ \exists\, y \in a\ (x\mathrel{R}y)}
\end{align*}
ottenendo \(G(n) = Z_{n}\), dove \(V\) è la \href{20250203104513-classe_totale.org}{classe di tutti gli insiemi}.
\paragraph{Fine della dimostrazione}
\label{sec:org9cec5ed}

Tramite gli assiomi di Replacement e dell'Unione,
\begin{equation*}
\bigcup_{n \in \omega} Z_{n}
\end{equation*}
è un insieme, ed è uguale a \(\set{y \in X\ |\ y\mathrel{\tilde{R}}\overline{x}}\).
\subsubsection{Dimostrazione 2}
\label{sec:org44b5f1a}

Siccome \(R \subseteq \tilde{R}\) è sufficiente dimostrare che \(\tilde{R}\) è well-founded se lo è \(R\), dal momento che il viceversa è immediato.

Sia \(\emptyset \neq Y \subseteq X\) fissato. Si mostra che vi è un \(\tilde{R}\)-\href{20250203102516-massimo_e_minimo.org}{elemento minimale}.

Si definisce \uline{percorso} da \(Y\) in se stesso come una \href{20250206170922-sequenze_e_stringhe.org}{sequenza finita} \(\langle z_{0},\dots,z_{n}\rangle\) di elementi di \(X\) tali che
\begin{enumerate}
\item \(z_{0},z_{n} \in Y\)
\item per ogni \(i<n\) si ha \(z_{i}\mathrel{R} z_{i+1}\)
\end{enumerate}

Sia dunque
\begin{equation*}
\overline{Y} = \set{x \in X\ |\ \exists\, s\ (s\text{ è un percorso da } Y\text{ in se stesso} \,\land\, x \in \operatorname{ran}(s))}
\end{equation*}
(dove \(\operatorname{ran}(s)\) indica il \href{20250202173528-dominio_range_e_campo_di_una_classe_relazione.org}{range di \(s\)}) la classe degli elementi di \(X\) visitati da un percorso da \(Y\) in se stesso.

Per costruzione, \(Y \subseteq \overline{Y}\). Siccome \(R\) è well-founded e \(\overline{Y} \subseteq X\), sia \(\overline{y}\) il \(R\)-elemento minimale di \(\overline{Y}\).

Nessun elemento di \(\overline{Y}\setminus Y\) può essere minimale rispetto a \(\overline{Y}\) (ovvio, siccome vi è una catena di relazioni). Pertanto \(\overline{y} \in Y\).

Dimostriamo che \(\overline{y}\) sia \(\tilde{R}\)-minimale in \(Y\). Supponiamo per assurdo che vi sia \(\overline{x} \in Y\) tale che
\begin{equation*}
\overline{x} \mathrel{\tilde{R}}\overline{y}
\end{equation*}
Dunque esistono \(z_{0},\dots,z_{n+1}\) elementi \(\overline{Y}\) tali che
\begin{equation*}\overline{x}\mathrel{R}z_{0}\mathrel{R}z_{1}\mathrel{R}\dots\mathrel{R}z_{n+1}\mathrel{R}\overline{y}
\end{equation*}
Dunque \(z_{n+1}\mathrel{R}\overline{y}\), contro l'ipotesi che \(\overline{y}\) fosse minimale rispetto a \(R\) in \(\overline{Y}\).
\end{document}
