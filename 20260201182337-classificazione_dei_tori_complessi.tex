% Created 2026-02-07 Sat 19:32
% Intended LaTeX compiler: pdflatex
\documentclass[10pt]{article}
%% CREATO CON ORG - EMACS
\newcommand{\use}[2][]{\usepackage[#1]{#2}}
% PACCHETTI FONDAMENTLAI
\use[utf8]{inputenc}
\use[T1]{fontenc}
\use{graphicx}
\use{longtable}
\use{wrapfig}
\use{rotating}
\use[normalem]{ulem}
\use{amsmath}
\use{amsthm}
\use{amssymb}

\use{eucal} % Cambia mathcal{...}

\use{capt-of}
\use[italian]{babel}
\use[babel]{csquotes}
% bib la TEX lo carica in automatico org-cite
\use{microtype}
\use{lmodern}
\use{subfig} % sottofigure
\use{multicol} % due colonne
\use{lipsum} % lorem ipsum
\use{color} % colori in latex
\use{parskip} % rimuove l'indentazione dei nuovi paragrafi %% Add parbox=false to all new tcolorbox
\use{centernot}
\use[outline]{contour}\contourlength{3pt}
\use{fancyhdr}
\use{layout}
\use[most]{tcolorbox} % Riquadri colorati
\use{ifthen} % IFTHEN
\use{geometry}

% pacchetti matematica
\use{yhmath}
\use{dsfont}
\use{mathrsfs}
\use{cancel} % semplificare
\use{polynom} %divisione tra polinomi
\use{forest} % grafi ad albero
\use{booktabs} % tabelle
\use{commath} %simboli e differenziali
\use{bm} %bold
\use[fulladjust]{marginnote} %to use marginnote for date notes
\use{arrayjobx}%array
\use[intlimits]{empheq} % Riquadri colorati attorno alle equazioni
\use{mathtools}
\use{circuitikz} % Disegnare i circuiti
\use{mathtools}
\use{stmaryrd} % [[ \llbracket ]] \rrbracket
\use{bussproofs} % dimostrazioni

%%%%%%%%%%%%%


%%%% QUIVER
\newcommand{\duepunti}{\,\mathchar\numexpr"6000+`:\relax\,}
% A TikZ style for curved arrows of a fixed height, due to AndréC.
\tikzset{curve/.style={settings={#1},to path={(\tikztostart)
    .. controls ($(\tikztostart)!\pv{pos}!(\tikztotarget)!\pv{height}!270:(\tikztotarget)$)
    and ($(\tikztostart)!1-\pv{pos}!(\tikztotarget)!\pv{height}!270:(\tikztotarget)$)
    .. (\tikztotarget)\tikztonodes}},
    settings/.code={\tikzset{quiver/.cd,#1}
        \def\pv##1{\pgfkeysvalueof{/tikz/quiver/##1}}},
    quiver/.cd,pos/.initial=0.35,height/.initial=0}

% TikZ arrowhead/tail styles.
\tikzset{tail reversed/.code={\pgfsetarrowsstart{tikzcd to}}}
\tikzset{2tail/.code={\pgfsetarrowsstart{Implies[reversed]}}}
\tikzset{2tail reversed/.code={\pgfsetarrowsstart{Implies}}}
% TikZ arrow styles.
\tikzset{no body/.style={/tikz/dash pattern=on 0 off 1mm}}
%%%%%%%%%%


%% DEFINIZIONI COMANDI MATEMATICI
\let\sin\relax %TOGLIE LA DEFINIZIONE SU "\sin"

% cambia la definizione di empty set
% ---
\let\oldemptyset\emptyset
% ---
% \let\emptyset\varnothing
% ---
% \let\emptyset\relax
% \newcommand{\emptyset}{\text{\textnormal{\O}}}
% ---

\DeclareMathOperator{\bounded}{bd}
\DeclareMathOperator{\sin}{sen}
\DeclareMathOperator{\epi}{Epi}
\DeclareMathOperator{\cl}{cl}
\DeclareMathOperator{\graph}{graph}
\DeclareMathOperator{\arcsec}{arcsec}
\DeclareMathOperator{\arccot}{arccot}
\DeclareMathOperator{\arccsc}{arccsc}
\DeclareMathOperator{\spettro}{Spettro}
\DeclareMathOperator{\nulls}{nullspace}
\DeclareMathOperator{\dom}{dom}
\DeclareMathOperator{\ar}{ar}
\DeclareMathOperator{\const}{Const}
\DeclareMathOperator{\fun}{Fun}
\DeclareMathOperator{\rel}{Rel}
\DeclareMathOperator{\altezza}{ht}
\let\det\relax %TOGLIE LA DEFINIZIONE SU "\det"
\DeclareMathOperator{\det}{det}
\DeclareMathOperator{\End}{End}
\DeclareMathOperator{\gl}{GL}
\def\Id{\mathrm{Id}}
\def\id{\mathrm{id}}
\DeclareMathOperator{\I}{\mathds{1}}
\DeclareMathOperator{\II}{II}
\DeclareMathOperator{\rank}{rank}
\DeclareMathOperator{\tr}{tr}
\DeclareMathOperator{\tc}{t.c.}
\DeclareMathOperator{\T}{T}
\DeclareMathOperator{\var}{Var}
\DeclareMathOperator{\cov}{Cov}
\DeclareMathOperator{\st}{st}
\DeclareMathOperator{\mon}{Mon}
\newcommand{\card}[1]{\left\vert #1 \right\vert}
\newcommand{\trasposta}[1]{\prescript{\text{T}}{}{#1}}
\newcommand{\1}{\mathds{1}}
\newcommand{\R}{\mathds{R}}
\newcommand{\diesis}{\#}
\newcommand{\bemolle}{\flat}
\newcommand{\nonstandard}[1]{\prescript{*}{}{#1}}
\newcommand{\starR}{\nonstandard{\R}}
\newcommand{\borel}{\mathscr{B}}
\newcommand{\lebesgue}[1]{\mathscr{L}\left(#1\right)}
\newcommand{\media}{\mathds{E}}
\newcommand{\K}{\mathds{K}}
\newcommand{\A}{\mathds{A}}
\newcommand{\Q}{\mathds{Q}}
\newcommand{\N}{\mathds{N}}
\newcommand{\C}{\mathds{C}}
\newcommand{\Z}{\mathds{Z}}
\newcommand{\qo}{\hspace{1em}\text{q.o.}\,}
\renewcommand{\tilde}[1]{\widetilde{#1}}
\renewcommand{\parallel}{\mathrel{/\mkern-5mu/}}
\newcommand{\parti}[2][]{\wp_{#1}(#2)}
\newcommand{\diff}[1]{\operatorname{d}_{#1}}
\let\oldvec\vec
\renewcommand{\vec}[1]{\overrightarrow{\vphantom{i}#1}}
\newcommand{\floor}[1]{\left\lfloor #1 \right\rfloor}
\newcommand{\cat}[1]{\mathbf{#1}}
\newcommand{\dfreccia}[1]{\xrightarrow{\ #1 \ }}
\newcommand{\sfreccia}[1]{\xleftarrow{\ #1 \ }}
\newcommand{\formalsum}[2]{{\sum_{#1}^{#2}}{\vphantom{\sum}}'}
\newcommand{\minim}[2]{\mu_{#1}\, \left(#2\right)}
\newcommand{\concat}{\null^{\frown}} % concatenazione di stringe
\newcommand{\godelcode}[1]{\langle\!\langle #1 \rangle\!\rangle}
\newcommand{\godeldec}[1]{(\!(#1)\!)}
\newcommand{\termcode}[1]{\ulcorner #1\urcorner}
\newcommand{\partialto}{\dashrightarrow}
\newcommand{\restricted}{\upharpoonright}
\newcommand{\embeds}{\precsim}
\newcommand{\surjects}{\twoheadrightarrow}
\newcommand{\equipotenti}{\asymp}
%% \newcommand{\dotplus}{\mathbin{\dot{+}}} %% A quanto pare esiste già
\newcommand{\bigdot}{\mathbin{\boldsymbol{\cdot}}}
\newcommand{\dotexp}[1]{^{.#1}}
\newcommand{\conv}{\mathbin{*}}
\newcommand{\convolution}[2]{(#1\conv #2)}
\newcommand{\nil}{\mathfrak{N}}
\newcommand{\divisore}{\mathrel{|}}
\newcommand{\simplesso}[1]{\mathrm{e}_{#1}}

\renewcommand{\iff}{\mathrel{\longleftrightarrow}} %% Notazione Logica.
\newcommand{\oldiff}{\mathrel{\Longleftrightarrow}}
\renewcommand{\implies}{\mathrel{\rightarrow}} %% Notazione Logica
\newcommand{\oldimplies}{\mathrel{\Longrightarrow}}
\renewcommand{\impliedby}{\mathrel{\leftarrow}} %% Notazione Logica
\newcommand{\oldimpliedby}{\mathrel{\Longleftarrow}}

\newcommand{\IFF}{\quad\Longleftrightarrow\quad}
\newcommand{\IMPLICA}{\quad\Longrightarrow\quad}


\renewcommand{\descriptionlabel}[1]{\hspace{\labelsep}\normalfont #1} % remove bold from description


%% Definizione di Divergenza di K-L

\DeclarePairedDelimiterX{\infdivx}[2]{(}{)}{%
  #1\;\delimsize\|\;#2%
}
\newcommand{\kldiv}{D_{KL}\infdivx}

%% Definizione di \dotminus

\makeatletter
\newcommand{\dotminus}{\mathbin{\text{\@dotminus}}}

\newcommand{\@dotminus}{%
  \ooalign{\hidewidth\raise1ex\hbox{.}\hidewidth\cr$\m@th-$\cr}%
}
\makeatother

%tramite i prossimi due comandi posso decidere come scrivere i logaritmi naturali in tutti i documenti: ho infatti eliminato qualsiasi differenza tra "ln" e "log": se si vuole qualcosa di diverso bisogna inserire manualmente il tutto
\let\ln\relax
\DeclareMathOperator{\ln}{ln}
\let\log\relax
\DeclareMathOperator{\log}{log}
%%%%%%

%% NUOVI COMANDI
\newcommand{\straniero}[1]{\textit{#1}} %parole straniere
\newcommand{\titolo}[1]{\textsc{#1}} %titoli
\newcommand{\qedd}{\tag*{$\blacksquare$}} %qed per ambienti matemastici
\renewcommand{\qedsymbol}{$\blacksquare$} %modifica colore qed
\newcommand{\ooverline}[1]{\overline{\overline{#1}}}
\newcommand{\circoletto}[1]{\left(#1\right)^{\text{o}}}
%
\newcommand{\qmatrice}[1]{\begin{pmatrix}
#1_{11} & \cdots & #1_{1n}\\
\vdots & \ddots & \vdots \\
#1_{m1} & \cdots & #1_{mn}
\end{pmatrix}}
%
\newcommand{\parentesi}[2]{%
\underset{#1}{\underbrace{#2}}%
}
%
\newcommand{\norma}[1]{% Norma
\left\lVert#1\right\rVert%
}
\newcommand{\scalare}[2]{% Scalare
\left\langle #1, #2\right\rangle
}
%%%%%

%% RESTRIZIONI
\newcommand{\referenze}[2]{
        \phantomsection{}#2\textsuperscript{\textcolor{blue}{\textbf{#1}}}
}

\let\restriction\relax

\def\restriction#1#2{\mathchoice
              {\setbox1\hbox{${\displaystyle #1}_{\scriptstyle #2}$}
              \restrictionaux{#1}{#2}}
              {\setbox1\hbox{${\textstyle #1}_{\scriptstyle #2}$}
              \restrictionaux{#1}{#2}}
              {\setbox1\hbox{${\scriptstyle #1}_{\scriptscriptstyle #2}$}
              \restrictionaux{#1}{#2}}
              {\setbox1\hbox{${\scriptscriptstyle #1}_{\scriptscriptstyle #2}$}
              \restrictionaux{#1}{#2}}}
\def\restrictionaux#1#2{{#1\,\smash{\vrule height .8\ht1 depth .85\dp1}}_{\,#2}}
%%%%%%%%%%%

%%% FORMATTAZIONE FOOTNOTEMARK

\def\footnotemarkformatting#1{[#1]}
\renewcommand{\thefootnote}{\footnotemarkformatting{\arabic{footnote}}}

%% SEZIONE GRAFICA
\use{tikz}
\usetikzlibrary{matrix, patterns, calc, decorations.pathreplacing, hobby, decorations.markings, decorations.pathmorphing, babel}
\use{tikz-3dplot}
\use{mathrsfs} %per geogebra
\use{tikz-cd}
\tikzset
{
  %surface/.style={fill=black!10, shading=ball,fill opacity=0.4},
  plane/.style={black,pattern=north east lines},
  curve/.style={black,line width=0.5mm},
  dritto/.style={decoration={markings,mark=at position 0.5 with {\arrow{Stealth}}}, postaction=decorate},
  rovescio/.style={decoration={markings,mark=at position 0.5 with {\arrow{Stealth[reversed]}}}, postaction=decorate}
}
\use{pgfplots} % stampare le funzioni
        \pgfplotsset{/pgf/number format/use comma,compat=1.15}
        %\pgfplotsset{compat=1.15} %per geogebra
        \usepgfplotslibrary{fillbetween, polar}
%%%%%%

%% CITAZIONI
\use{lineno}

\newcommand{\citazione}[1]{%
  \begin{quotation}
  \begin{linenumbers}
  \modulolinenumbers[5]
  \begingroup
  \setlength{\parindent}{0cm}
  \noindent #1
  \endgroup
  \end{linenumbers}
  \end{quotation}\setcounter{linenumber}{1}
  }
%%%%%%

%%%%%%%%%%%%%%%%%%%%%%%%%%%%%%%%%%%%%%%%%%%%
%%%%%%%%%%%%%%%%%%%%%%%%%%%%%%%%%%%%%%%%%%%%

%% AMS THM

\theoremstyle{definition}% default
\newtheorem{thm}{Teorema}[section]
\newtheorem{lem}[thm]{Lemma}
\newtheorem{prop}[thm]{Proposizione}
\newtheorem{cor}[thm]{Corollario}
\newtheorem{esempio}[thm]{Esempio}
\theoremstyle{plain}
\newtheorem{definizione}[thm]{Definizione}
\theoremstyle{remark}
\newtheorem*{oss}{Osservazione}


%%%%%%%%%%%%%%%%%%%%%%%%%%%%%%%%%%%%%%%%%%%%
%%%%%%%%%%%%%%%%%%%%%%%%%%%%%%%%%%%%%%%%%%%%

\use{hyperref}
\hypersetup{%
        pdfauthor={Davide Peccioli},
        pdfsubject={},
        allcolors=black,
        citecolor=black,
%	colorlinks=true,
        bookmarksopen=true}
\setcounter{secnumdepth}{0} % rimuove i numeri di sezione senza rimuovere le ref
\renewcommand{\href}[2]{\textcolor{blue}{#2}} % disabilita il comando href
\use{enotez} %
\setenotez{%
 mark-format = \footnotemarkformatting % Mette i numeri tra parentesi quadre%
}\let\footnote=\endnote % rende tutte le note a pié pagina come delle note a fine file 


\let\olddocument\document % modifico l'ambiende documenti per non dover stampare \printendnote
\let\oldenddocument\enddocument
\renewenvironment{document}%
{%
  \olddocument
}{%
  \printendnotes\oldenddocument
}
\renewcommand{\thethm}{\arabic{thm}}

\usepackage[hyperref]{biblatex}
\addbibresource{~/Documents/org/roam/bib/master.bib}
\use{upgreek}
\def\tau{\uptau}
\author{Davide Peccioli}
\date{\today}
\title{}
\begin{document}

\section{Classificazione dei tori complessi}
\label{sec:orga3855ff}
Si consideri il semipiano superiore di \(\C\):
\begin{equation*}
\mathds{H} \coloneqq \set{z \in \C \mid \operatorname{Im} z > 0}.
\end{equation*}
Per ogni \(\tau \in \mathds{H}\), sia \(L_{\tau}\) il reticolo generato da \(1\) e \(\tau\):
\begin{equation*}
L_{\tau} \coloneqq \set{z_{1} + z_{2}\,\tau \mid z_{1},z_{2} \in \Z}
\end{equation*}
e sia \(X_{\tau} \coloneqq \C/L_{\tau}\) il \href{20260127113001-toro_complesso.org}{toro complesso} dato dal \href{20250127093819-quoziente_di_gruppo_e_sottogruppo.org}{quoziente} con \(L_{\tau}\).

\begin{prop}
Sia \(Y\) un toro complesso qualsiasi. Allora esiste \(\tau \in \mathds{H}\) tale che
\begin{equation*}
Y = X_{\tau}.
\end{equation*}
\end{prop}

\begin{proof}
Sia \(Y=\C/M\), con \(M\), e siano \(w_{1},w_{2} \in \C\) tale che
\begin{equation*}
M \coloneqq \set{z_{1}\,w_{1}+z_{2}\,w_{2} \mid z_{1},z_{2} \in \Z}
\end{equation*}
Allora \(w_{1},w_{2}\) sono \(\R\)-\href{20241212142019-insiemi_linearmente_indipendenti.org}{linearmente indipendenti}, e \(w_{1} \neq 0\). Sia \(\gamma \coloneqq \frac{1}{w_{1}}\).

Allora \(\gamma\cdot M\) è il reticolo generato da \(1,\frac{\omega_{2}}{\omega_{1}}\), linearmente indipendenti su \(\R\). Pertanto, \(\frac{w_{2}}{w_{1}}\notin \R\).

\begin{itemize}
\item Se \(\operatorname{Im}\frac{w_{2}}{w_{1}} > 0\), allora \(\tau\coloneqq \frac{w_{2}}{w_{1}}\) e
\begin{equation*}
  \gamma \cdot M = L_{\tau}
\end{equation*}
e \href{20260201182313-isomorfismo_tra_tori_complessi.org}{quindi} \(Y \cong X_{\tau}\).

\item Se \(\operatorname{Im}\frac{w_{2}}{w_{1}} < 0\), allora \(\tau\coloneqq - \frac{w_{2}}{w_{1}}\) e\footnote{Ovviamente \(\left(1,\frac{w_{2}}{w_{1}}\right)\) e \(\left(1,-\frac{w_{2}}{w_{1}}\right)\) generano lo stesso reticolo.}
\begin{equation*}
  \gamma \cdot M = L_{\tau}
\end{equation*}
e \href{20260201182313-isomorfismo_tra_tori_complessi.org}{quindi} \(Y \cong X_{\tau}\).
\qedhere
\end{itemize}
\end{proof}

La situazione quindi è la seguente:\footnote{A meno di \href{20260128144717-isomorfismo_tra_superfici_di_riemann.org}{isomorfismo} tra \href{20260127112828-superficie_di_riemann.org}{superfici di Riemann}.}
\begin{equation*}
\begin{tikzcd}[ampersand replacement=\&]
	{\set{\text{tori complessi}}/\text{iso}} \&\&\& {\set{X_{\tau}}_{\tau \in \mathds{H}}/\text{iso}}
	\arrow["{1\duepunti 1}", tail reversed, from=1-1, to=1-4]
\end{tikzcd}
\end{equation*}

\begin{prop}
Siano \(\tau,\tau' \in \mathds{H}\). Sono fatti equivalenti:
\begin{enumerate}
\item sono isomorfi: \(X_{\tau} \cong X_{\tau'}\);
\item esiste \(\left(\begin{smallmatrix} a & b\\ c & d\end{smallmatrix}\right) \in \operatorname{SL}(2;\Z)\)\footnote{\(\operatorname{SL}(n, \K)\) è il \href{20250113142317-gruppo_lineare_speciale.org}{gruppo lineare speciale}} tale che
\begin{equation*}
 \tau = \frac{a\tau' + b}{c\tau' + d}.
\end{equation*}
\end{enumerate}
\end{prop}

\begin{proof}
Per la \href{20260201182313-isomorfismo_tra_tori_complessi.org}{caratterizzazione dei tori isomorfi}, \(X_{\tau}\cong X_{\tau'}\) se e solo se esiste \(\gamma \in \C\setminus \set{0}\) tale che \(\gamma\cdot L_{\tau} = L_{\tau'}\), dove
\begin{align*}
L_{\tau} &= \set{z_{1}+\tau \, z_{2} \mid z_{1},z_{2} \in \Z}\\
L_{\tau'} &= \set{z_{1}+\tau' \, z_{2} \mid z_{1},z_{2} \in \Z}\\
\gamma \cdot L_{\tau} &= \set{\gamma\cdot(z_{1}+\tau \, z_{2}) \mid z_{1},z_{2} \in \Z} = \set{z_{1}\,\gamma + z_{2}\, (\gamma\tau) \mid z_{1},z_{2} \in \Z}
\end{align*}
ovvero se e solo se esiste \(\gamma \in \C\setminus\set{0}\) tale che \(\gamma,\gamma\tau \in L_{\tau'}\), e lo generano come gruppo additivo.

(\(\Rightarrow\)): Siccome \(\gamma, \gamma\tau \in L_{\tau'}\) allora esistono \(a,b,c,d \in \Z\) tali che
\begin{align*}
0 \neq \gamma &= c\tau' + d\\
\gamma\tau &= a\tau' + b
\end{align*}
Allora \(\tau = \frac{a\tau'+b}{c\tau' + d}\), e \(\left(\begin{smallmatrix} a & b\\ c & d \end{smallmatrix}\right) \in \Z^{2\times 2}\):
\begin{equation*}
\begin{pmatrix}
\gamma\tau\\
\gamma
\end{pmatrix} =
\begin{pmatrix}
a & b\\
c & d
\end{pmatrix}\
\begin{pmatrix}
\tau'\\
1
\end{pmatrix}
\end{equation*}

Siccome \(1, \tau' \in \gamma L_{\tau}\) allora esistono \(a',b',c',d' \in \Z\) tali che
\begin{align*}
1 &= \gamma(c'\tau + d')\\
\tau' &= \gamma(a'\tau + b')
\end{align*}
Allora \(\left(\begin{smallmatrix} a' & b'\\ c' & d' \end{smallmatrix}\right) \in \Z^{2\times 2}\) è tale che :
\begin{equation*}
\begin{pmatrix}
\tau'\\
1
\end{pmatrix} =
\begin{pmatrix}
a' & b'\\
c' & d'
\end{pmatrix}\
\begin{pmatrix}
\gamma \tau\\
\gamma
\end{pmatrix}
\end{equation*}

Segue che \(\left(\begin{smallmatrix} a & b\\ c & d \end{smallmatrix}\right)\) è invertibile, e con inversa in \(\Z^{2\times 2}\), pertanto
\begin{equation*}
\det \begin{pmatrix} a' & b'\\ c' & d' \end{pmatrix} \cdot \det \begin{pmatrix} a & b\\ c & d \end{pmatrix} = 1
\end{equation*}
e devono entrambi essere numeri interi. Segue che
\begin{equation*}
\det \begin{pmatrix} a & b\\ c & d \end{pmatrix} = \pm 1
\end{equation*}

Inoltre si ha che\footnote{\(\overline{\tau'}\) è il \href{20260201232630-complesso_coniugato.org}{complesso coniugato}}
\begin{align*}
\operatorname{Im} \tau &= \operatorname{Im}\left( %
\frac{a\tau'+ b}{c\tau' + d}
\right) = \operatorname{Im}\left( %
\frac{(a\tau'+ b)(c\overline{\tau'}+d)}{|c\tau' + d|^{2}}
\right)\\[2ex]
&= \operatorname{Im}\left( %
\frac{%
	ac \, |\tau'|^{2} + ad \,\tau' + bc\, \overline{\tau'} + bc%
}{|c\tau' + d|^{2}}
\right)\\
&= \frac{1}{|c\tau' + d|^{2}} \cdot \operatorname{Im}(ac \, |\tau'|^{2} + ad \,\tau' + bc\, \overline{\tau'} + bc)\\
&= \frac{1}{|c\tau' + d|^{2}} \cdot \operatorname{Im}(ad \,\tau' + bc\, \overline{\tau'})\\
&= \frac{1}{|c\tau' + d|^{2}} \cdot \bigg[ad\,\operatorname{Im}(\tau') + bc\, \operatorname{Im}(\overline{\tau'})\bigg]\\
&= \frac{1}{|c\tau' + d|^{2}} \cdot \bigg[ad\,\operatorname{Im}(\tau') - bc\, \operatorname{Im}({\tau'})\bigg]\\
&= \frac{1}{|c\tau' + d|^{2}} \cdot (ad-bc)\operatorname{Im}(\tau')\\
&= \frac{1}{|c\tau' + d|^{2}} \cdot \det\begin{pmatrix}
a & b\\
c & d
\end{pmatrix}\cdot \operatorname{Im}(\tau')
\end{align*}
e quindi, per concordanza dei segni (\(\operatorname{Im}\tau > 0\), \(\operatorname{Im}\tau' >0\)), si deve avere \(\det\left(\begin{smallmatrix} a & b\\c & d \end{smallmatrix}\right) > 0'\).

(\(\Leftarrow\)): Supponiamo quindi che \(\tau = \frac{a\tau'+b}{c\tau' + d}\), con
\begin{equation*}
\begin{pmatrix}
a & b\\
c & d
\end{pmatrix} \in \operatorname{SL}(2;\Z).
\end{equation*}

Sia \(\gamma \coloneqq c\tau' + d\).
\begin{itemize}
\item \(\gamma \neq 0\), poiché \((c,d) \neq (0,0)\) visto che la matrice è invertibile, e \(\tau'\notin \R\).
\item Ovviamente \(\gamma, \gamma\,\tau \in L_{\tau'}\), in quanto
\begin{equation*}
  \gamma\,\tau = \gamma\, \frac{a\tau'+b}{c\tau' + d} = a\tau' + b.
\end{equation*}
\item Per dimostrare che \(\gamma,\gamma\,\tau\) generino \(L_{\tau'}\) come gruppo additivo, è sufficiente scrivere \(1,\tau'\) come \(\Z\)-combinazione lineare di \(\gamma,\gamma\,\tau\). In particolare, sia \(A' \in \Z^{2\times 2}\) la matrice inversa di \(\left(\begin{smallmatrix} a & b\\ c & d\end{smallmatrix}\right)\). Allora
\begin{equation*}
  A'\cdot \begin{pmatrix}
  	\gamma\,\tau\\
  	\gamma
  \end{pmatrix} = A' \cdot \begin{pmatrix}
  	a & b\\
  	c & d
  \end{pmatrix} \cdot \begin{pmatrix}
  	\tau'\\
  	1
  \end{pmatrix} = \begin{pmatrix}
  	\tau'\\
  	1
  \end{pmatrix}.
  \qedhere
\end{equation*}
\end{itemize}
\end{proof}

\begin{prop}
Il gruppo lineare speciale \(\operatorname{SL}(2;\Z)\) \href{20260201185059-azione_di_gruppo.org}{agisce} su \(\mathds{H}\) tramite
\begin{equation*}
\begin{pmatrix}
a & b\\
c & d
\end{pmatrix} \in \operatorname{SL}(2;\Z),\qquad \begin{aligned}
\mathds{H} & \longrightarrow \mathds{H}\\
\tau &\longmapsto \frac{a\tau + b}{c\tau + d}.
\end{aligned}
\end{equation*}
\end{prop}

\begin{oss}
In particolare, quindi, \(X_{\tau} \cong X_{\tau'}\) se e solo se \(\tau,\tau'\) sono nella stessa \href{20260201185059-azione_di_gruppo.org}{orbita} per questa azione. In definitiva, si ha
\begin{equation*}
\begin{tikzcd}[ampersand replacement=\&]
	{\set{\text{tori complessi}}/\text{iso}} \&\& {\set{X_{\tau}}_{\tau \in \mathds{H}}/\text{iso}} \&\& {\mathds{H}/\operatorname{SL}(2;\Z)}
	\arrow["{{1\duepunti 1}}", tail reversed, from=1-1, to=1-3]
	\arrow["{1\duepunti 1}", tail reversed, from=1-3, to=1-5]
\end{tikzcd}
\end{equation*}
\end{oss}
\end{document}
