% Created 2026-02-07 Sat 19:30
% Intended LaTeX compiler: pdflatex
\documentclass[10pt]{article}
%% CREATO CON ORG - EMACS
\newcommand{\use}[2][]{\usepackage[#1]{#2}}
% PACCHETTI FONDAMENTLAI
\use[utf8]{inputenc}
\use[T1]{fontenc}
\use{graphicx}
\use{longtable}
\use{wrapfig}
\use{rotating}
\use[normalem]{ulem}
\use{amsmath}
\use{amsthm}
\use{amssymb}

\use{eucal} % Cambia mathcal{...}

\use{capt-of}
\use[italian]{babel}
\use[babel]{csquotes}
% bib la TEX lo carica in automatico org-cite
\use{microtype}
\use{lmodern}
\use{subfig} % sottofigure
\use{multicol} % due colonne
\use{lipsum} % lorem ipsum
\use{color} % colori in latex
\use{parskip} % rimuove l'indentazione dei nuovi paragrafi %% Add parbox=false to all new tcolorbox
\use{centernot}
\use[outline]{contour}\contourlength{3pt}
\use{fancyhdr}
\use{layout}
\use[most]{tcolorbox} % Riquadri colorati
\use{ifthen} % IFTHEN
\use{geometry}

% pacchetti matematica
\use{yhmath}
\use{dsfont}
\use{mathrsfs}
\use{cancel} % semplificare
\use{polynom} %divisione tra polinomi
\use{forest} % grafi ad albero
\use{booktabs} % tabelle
\use{commath} %simboli e differenziali
\use{bm} %bold
\use[fulladjust]{marginnote} %to use marginnote for date notes
\use{arrayjobx}%array
\use[intlimits]{empheq} % Riquadri colorati attorno alle equazioni
\use{mathtools}
\use{circuitikz} % Disegnare i circuiti
\use{mathtools}
\use{stmaryrd} % [[ \llbracket ]] \rrbracket
\use{bussproofs} % dimostrazioni

%%%%%%%%%%%%%


%%%% QUIVER
\newcommand{\duepunti}{\,\mathchar\numexpr"6000+`:\relax\,}
% A TikZ style for curved arrows of a fixed height, due to AndréC.
\tikzset{curve/.style={settings={#1},to path={(\tikztostart)
    .. controls ($(\tikztostart)!\pv{pos}!(\tikztotarget)!\pv{height}!270:(\tikztotarget)$)
    and ($(\tikztostart)!1-\pv{pos}!(\tikztotarget)!\pv{height}!270:(\tikztotarget)$)
    .. (\tikztotarget)\tikztonodes}},
    settings/.code={\tikzset{quiver/.cd,#1}
        \def\pv##1{\pgfkeysvalueof{/tikz/quiver/##1}}},
    quiver/.cd,pos/.initial=0.35,height/.initial=0}

% TikZ arrowhead/tail styles.
\tikzset{tail reversed/.code={\pgfsetarrowsstart{tikzcd to}}}
\tikzset{2tail/.code={\pgfsetarrowsstart{Implies[reversed]}}}
\tikzset{2tail reversed/.code={\pgfsetarrowsstart{Implies}}}
% TikZ arrow styles.
\tikzset{no body/.style={/tikz/dash pattern=on 0 off 1mm}}
%%%%%%%%%%


%% DEFINIZIONI COMANDI MATEMATICI
\let\sin\relax %TOGLIE LA DEFINIZIONE SU "\sin"

% cambia la definizione di empty set
% ---
\let\oldemptyset\emptyset
% ---
% \let\emptyset\varnothing
% ---
% \let\emptyset\relax
% \newcommand{\emptyset}{\text{\textnormal{\O}}}
% ---

\DeclareMathOperator{\bounded}{bd}
\DeclareMathOperator{\sin}{sen}
\DeclareMathOperator{\epi}{Epi}
\DeclareMathOperator{\cl}{cl}
\DeclareMathOperator{\graph}{graph}
\DeclareMathOperator{\arcsec}{arcsec}
\DeclareMathOperator{\arccot}{arccot}
\DeclareMathOperator{\arccsc}{arccsc}
\DeclareMathOperator{\spettro}{Spettro}
\DeclareMathOperator{\nulls}{nullspace}
\DeclareMathOperator{\dom}{dom}
\DeclareMathOperator{\ar}{ar}
\DeclareMathOperator{\const}{Const}
\DeclareMathOperator{\fun}{Fun}
\DeclareMathOperator{\rel}{Rel}
\DeclareMathOperator{\altezza}{ht}
\let\det\relax %TOGLIE LA DEFINIZIONE SU "\det"
\DeclareMathOperator{\det}{det}
\DeclareMathOperator{\End}{End}
\DeclareMathOperator{\gl}{GL}
\def\Id{\mathrm{Id}}
\def\id{\mathrm{id}}
\DeclareMathOperator{\I}{\mathds{1}}
\DeclareMathOperator{\II}{II}
\DeclareMathOperator{\rank}{rank}
\DeclareMathOperator{\tr}{tr}
\DeclareMathOperator{\tc}{t.c.}
\DeclareMathOperator{\T}{T}
\DeclareMathOperator{\var}{Var}
\DeclareMathOperator{\cov}{Cov}
\DeclareMathOperator{\st}{st}
\DeclareMathOperator{\mon}{Mon}
\newcommand{\card}[1]{\left\vert #1 \right\vert}
\newcommand{\trasposta}[1]{\prescript{\text{T}}{}{#1}}
\newcommand{\1}{\mathds{1}}
\newcommand{\R}{\mathds{R}}
\newcommand{\diesis}{\#}
\newcommand{\bemolle}{\flat}
\newcommand{\nonstandard}[1]{\prescript{*}{}{#1}}
\newcommand{\starR}{\nonstandard{\R}}
\newcommand{\borel}{\mathscr{B}}
\newcommand{\lebesgue}[1]{\mathscr{L}\left(#1\right)}
\newcommand{\media}{\mathds{E}}
\newcommand{\K}{\mathds{K}}
\newcommand{\A}{\mathds{A}}
\newcommand{\Q}{\mathds{Q}}
\newcommand{\N}{\mathds{N}}
\newcommand{\C}{\mathds{C}}
\newcommand{\Z}{\mathds{Z}}
\newcommand{\qo}{\hspace{1em}\text{q.o.}\,}
\renewcommand{\tilde}[1]{\widetilde{#1}}
\renewcommand{\parallel}{\mathrel{/\mkern-5mu/}}
\newcommand{\parti}[2][]{\wp_{#1}(#2)}
\newcommand{\diff}[1]{\operatorname{d}_{#1}}
\let\oldvec\vec
\renewcommand{\vec}[1]{\overrightarrow{\vphantom{i}#1}}
\newcommand{\floor}[1]{\left\lfloor #1 \right\rfloor}
\newcommand{\cat}[1]{\mathbf{#1}}
\newcommand{\dfreccia}[1]{\xrightarrow{\ #1 \ }}
\newcommand{\sfreccia}[1]{\xleftarrow{\ #1 \ }}
\newcommand{\formalsum}[2]{{\sum_{#1}^{#2}}{\vphantom{\sum}}'}
\newcommand{\minim}[2]{\mu_{#1}\, \left(#2\right)}
\newcommand{\concat}{\null^{\frown}} % concatenazione di stringe
\newcommand{\godelcode}[1]{\langle\!\langle #1 \rangle\!\rangle}
\newcommand{\godeldec}[1]{(\!(#1)\!)}
\newcommand{\termcode}[1]{\ulcorner #1\urcorner}
\newcommand{\partialto}{\dashrightarrow}
\newcommand{\restricted}{\upharpoonright}
\newcommand{\embeds}{\precsim}
\newcommand{\surjects}{\twoheadrightarrow}
\newcommand{\equipotenti}{\asymp}
%% \newcommand{\dotplus}{\mathbin{\dot{+}}} %% A quanto pare esiste già
\newcommand{\bigdot}{\mathbin{\boldsymbol{\cdot}}}
\newcommand{\dotexp}[1]{^{.#1}}
\newcommand{\conv}{\mathbin{*}}
\newcommand{\convolution}[2]{(#1\conv #2)}
\newcommand{\nil}{\mathfrak{N}}
\newcommand{\divisore}{\mathrel{|}}
\newcommand{\simplesso}[1]{\mathrm{e}_{#1}}

\renewcommand{\iff}{\mathrel{\longleftrightarrow}} %% Notazione Logica.
\newcommand{\oldiff}{\mathrel{\Longleftrightarrow}}
\renewcommand{\implies}{\mathrel{\rightarrow}} %% Notazione Logica
\newcommand{\oldimplies}{\mathrel{\Longrightarrow}}
\renewcommand{\impliedby}{\mathrel{\leftarrow}} %% Notazione Logica
\newcommand{\oldimpliedby}{\mathrel{\Longleftarrow}}

\newcommand{\IFF}{\quad\Longleftrightarrow\quad}
\newcommand{\IMPLICA}{\quad\Longrightarrow\quad}


\renewcommand{\descriptionlabel}[1]{\hspace{\labelsep}\normalfont #1} % remove bold from description


%% Definizione di Divergenza di K-L

\DeclarePairedDelimiterX{\infdivx}[2]{(}{)}{%
  #1\;\delimsize\|\;#2%
}
\newcommand{\kldiv}{D_{KL}\infdivx}

%% Definizione di \dotminus

\makeatletter
\newcommand{\dotminus}{\mathbin{\text{\@dotminus}}}

\newcommand{\@dotminus}{%
  \ooalign{\hidewidth\raise1ex\hbox{.}\hidewidth\cr$\m@th-$\cr}%
}
\makeatother

%tramite i prossimi due comandi posso decidere come scrivere i logaritmi naturali in tutti i documenti: ho infatti eliminato qualsiasi differenza tra "ln" e "log": se si vuole qualcosa di diverso bisogna inserire manualmente il tutto
\let\ln\relax
\DeclareMathOperator{\ln}{ln}
\let\log\relax
\DeclareMathOperator{\log}{log}
%%%%%%

%% NUOVI COMANDI
\newcommand{\straniero}[1]{\textit{#1}} %parole straniere
\newcommand{\titolo}[1]{\textsc{#1}} %titoli
\newcommand{\qedd}{\tag*{$\blacksquare$}} %qed per ambienti matemastici
\renewcommand{\qedsymbol}{$\blacksquare$} %modifica colore qed
\newcommand{\ooverline}[1]{\overline{\overline{#1}}}
\newcommand{\circoletto}[1]{\left(#1\right)^{\text{o}}}
%
\newcommand{\qmatrice}[1]{\begin{pmatrix}
#1_{11} & \cdots & #1_{1n}\\
\vdots & \ddots & \vdots \\
#1_{m1} & \cdots & #1_{mn}
\end{pmatrix}}
%
\newcommand{\parentesi}[2]{%
\underset{#1}{\underbrace{#2}}%
}
%
\newcommand{\norma}[1]{% Norma
\left\lVert#1\right\rVert%
}
\newcommand{\scalare}[2]{% Scalare
\left\langle #1, #2\right\rangle
}
%%%%%

%% RESTRIZIONI
\newcommand{\referenze}[2]{
        \phantomsection{}#2\textsuperscript{\textcolor{blue}{\textbf{#1}}}
}

\let\restriction\relax

\def\restriction#1#2{\mathchoice
              {\setbox1\hbox{${\displaystyle #1}_{\scriptstyle #2}$}
              \restrictionaux{#1}{#2}}
              {\setbox1\hbox{${\textstyle #1}_{\scriptstyle #2}$}
              \restrictionaux{#1}{#2}}
              {\setbox1\hbox{${\scriptstyle #1}_{\scriptscriptstyle #2}$}
              \restrictionaux{#1}{#2}}
              {\setbox1\hbox{${\scriptscriptstyle #1}_{\scriptscriptstyle #2}$}
              \restrictionaux{#1}{#2}}}
\def\restrictionaux#1#2{{#1\,\smash{\vrule height .8\ht1 depth .85\dp1}}_{\,#2}}
%%%%%%%%%%%

%%% FORMATTAZIONE FOOTNOTEMARK

\def\footnotemarkformatting#1{[#1]}
\renewcommand{\thefootnote}{\footnotemarkformatting{\arabic{footnote}}}

%% SEZIONE GRAFICA
\use{tikz}
\usetikzlibrary{matrix, patterns, calc, decorations.pathreplacing, hobby, decorations.markings, decorations.pathmorphing, babel}
\use{tikz-3dplot}
\use{mathrsfs} %per geogebra
\use{tikz-cd}
\tikzset
{
  %surface/.style={fill=black!10, shading=ball,fill opacity=0.4},
  plane/.style={black,pattern=north east lines},
  curve/.style={black,line width=0.5mm},
  dritto/.style={decoration={markings,mark=at position 0.5 with {\arrow{Stealth}}}, postaction=decorate},
  rovescio/.style={decoration={markings,mark=at position 0.5 with {\arrow{Stealth[reversed]}}}, postaction=decorate}
}
\use{pgfplots} % stampare le funzioni
        \pgfplotsset{/pgf/number format/use comma,compat=1.15}
        %\pgfplotsset{compat=1.15} %per geogebra
        \usepgfplotslibrary{fillbetween, polar}
%%%%%%

%% CITAZIONI
\use{lineno}

\newcommand{\citazione}[1]{%
  \begin{quotation}
  \begin{linenumbers}
  \modulolinenumbers[5]
  \begingroup
  \setlength{\parindent}{0cm}
  \noindent #1
  \endgroup
  \end{linenumbers}
  \end{quotation}\setcounter{linenumber}{1}
  }
%%%%%%

%%%%%%%%%%%%%%%%%%%%%%%%%%%%%%%%%%%%%%%%%%%%
%%%%%%%%%%%%%%%%%%%%%%%%%%%%%%%%%%%%%%%%%%%%

%% AMS THM

\theoremstyle{definition}% default
\newtheorem{thm}{Teorema}[section]
\newtheorem{lem}[thm]{Lemma}
\newtheorem{prop}[thm]{Proposizione}
\newtheorem{cor}[thm]{Corollario}
\newtheorem{esempio}[thm]{Esempio}
\theoremstyle{plain}
\newtheorem{definizione}[thm]{Definizione}
\theoremstyle{remark}
\newtheorem*{oss}{Osservazione}


%%%%%%%%%%%%%%%%%%%%%%%%%%%%%%%%%%%%%%%%%%%%
%%%%%%%%%%%%%%%%%%%%%%%%%%%%%%%%%%%%%%%%%%%%

\use{hyperref}
\hypersetup{%
        pdfauthor={Davide Peccioli},
        pdfsubject={},
        allcolors=black,
        citecolor=black,
%	colorlinks=true,
        bookmarksopen=true}
\setcounter{secnumdepth}{0} % rimuove i numeri di sezione senza rimuovere le ref
\renewcommand{\href}[2]{\textcolor{blue}{#2}} % disabilita il comando href
\use{enotez} %
\setenotez{%
 mark-format = \footnotemarkformatting % Mette i numeri tra parentesi quadre%
}\let\footnote=\endnote % rende tutte le note a pié pagina come delle note a fine file 


\let\olddocument\document % modifico l'ambiende documenti per non dover stampare \printendnote
\let\oldenddocument\enddocument
\renewenvironment{document}%
{%
  \olddocument
}{%
  \printendnotes\oldenddocument
}
\renewcommand{\thethm}{\arabic{thm}}

\usepackage[hyperref]{biblatex}
\addbibresource{~/Documents/org/roam/bib/master.bib}
\author{Davide Peccioli}
\date{\today}
\title{Ogni chiuso di uno spazio metrico secondo numerabile ammette un ricoprimento numerabile di diametro arbitrariamente piccolo}
\begin{document}

\section{Proposizione}
\label{sec:org98a1ef2}
Se \(X\) è uno \href{20250301193511-spazio_metrico.org}{spazio metrico} \href{20250111142303-spazio_topologico_a_base_numerabile.org}{secondo numerabile}, allora per ogni \(U \subseteq X\) \href{20250103145124-topologia.org}{aperto} e per ogni \(\varepsilon \in \R^{+}\), esiste un \href{20250103164252-ricoprimento.org}{ricoprimento} \href{20250103145124-topologia.org}{aperto} \href{20250111143651-insieme_numerabile.org}{numerabile} \((U_{n})_{n \in \omega}\) di \(U\) tale che \(\operatorname{Cl}(U_{n}) \subseteq U\) e \(\operatorname{diam}(U_{n})< \varepsilon\), per ogni \(n \in \omega\).
\subsection{Dimostrazione}
\label{sec:org8d6ca4e}
Sia \((X,d)\) lo \href{20250301193511-spazio_metrico.org}{spazio metrico} in considerazione. Si denotino con
\begin{equation*}
B_{d}(x,r) \coloneqq \set{y \in X\mid d(x,y)< r}.
\end{equation*}

Siccome \(X\) è secondo numerabile \href{20250331095811-spazio_topologico_secondo_numerabile_implica_separabile.org}{allora} \(X\) è \href{20250301192908-spazio_topologico_separabile.org}{separabile}, e \href{20250301192908-spazio_topologico_separabile.org}{pertanto} \(U \subseteq X\) è \href{20250301192908-spazio_topologico_separabile.org}{separabile}. Sia quindi \(C\) sottoinsieme \href{20250301193045-sottoinsieme_denso.org}{denso} di \(U\), numerabile. Allora, per ogni \(c \in C\) esiste \(0<r_{c}<\varepsilon\) tale che \(B_{d}(c,r_{c}) \subseteq U\), poiché \(U\) aperto \href{20250317093153-insieme_aperto_sse_intorno_di_ogni_suo_punto.org}{e quindi} intorno di ogni suo punto. In particolare, si richiede che
\begin{equation*}
r_{c}\coloneqq \sup \set{ r \in \left(0,\frac{\varepsilon}{2}\right)\mid B_{d}(c,r) \subseteq U}.
\end{equation*}

Si consideri quindi:
\begin{equation*}
\mathcal{B}_{U} \coloneqq \set{B_{d}\left(c,\frac{r_{c}}{2}\right)\mid c \in C}
\end{equation*}
\begin{itemize}
\item Sia ora \(c \in C\) fissato, e sia \((x_{n})_{n \in \omega}\subseteq B_{d}\left(c,\frac{r_{c}}{2}\right)\) \href{20250115100904-successione.org}{successione} \href{20250115100930-convergenza_per_una_successione.org}{convergente} a \(x\). Allora, siccome \(d(x,x_{n})<\frac{r_{c}}{2}\) \href{20250304141512-proprieta_vere_definitivamente.org}{definitivamente}:
\begin{align*}
d(x,c) &\le d(x,x_{n}) + d(x_{n}, c)\\
&< \frac{r_{c}}{2} + \frac{r_{c}}{2} < r_{c}
\end{align*}
e pertanto \(x \in U\). Quindi, per la \href{20250303121451-caratterizzazione_della_chiusura_in_termini_di_successioni.org}{caratterizzazione della chiusura per successioni}, \(\operatorname{cl}\left(B_{d}\left(c,\frac{r_{c}}{2}\right)\right) \subseteq U\).
\item Infine, si ha che \href{20250131155822-operazioni_insiemistiche_tra_classi_mk.org}{l'unione} \(\bigcup \mathcal{B}_{U} = U\). Infatti, se \(y \in U \setminus C\) allora esiste \(\delta<\frac{\varepsilon}{2}\) tale che \(B_{d}(y,\delta) \subseteq U\). In particolare, esiste \(c \in B_{d}(y,\delta/2)\cap C\) (poiché \(C\) è denso in \(U\)). Si ha quindi che \(y \in B_{d}(c, \delta/2) \subseteq U\): infatti, se per assurdo esistesse \(x \in B_{d}(c,\delta/2)\setminus U\) allora
\begin{align*}
  d(x,y) &\le d(x,c) + d(c,y)\\
  	&< \frac{\delta}{2}+\frac{\delta}{2} < \delta
\end{align*}
e pertanto \(x \in B_{d}(y,\delta) \subseteq U\). Assurdo.

Dunque \(\delta/2<\varepsilon/2\) e \(B_{d}(c,\delta/2) \subseteq U\), e dunque \(r_{c}\ge \delta/2\) per massimalità. Pertanto \(B_{d}(c,r_{c})\supseteq B_{d}(c,\delta/2)\ni y\), e quindi \(y \in \bigcup\mathcal{B}_{U}\).
\end{itemize}
\section{NON Proposizione}
\label{sec:orgbfc5c0f}
If \(X\) is a \href{20250301193511-spazio_metrico.org}{metric space}, then for every \href{20250103145124-topologia.org}{open} \(U\) and every \(\varepsilon \in \R^{+}\) there is a \href{20250111143651-insieme_numerabile.org}{countable} \href{20250103164252-ricoprimento.org}{covering} \(\left(U_n\right)_{n \in \omega}\) of \(U\) such that \(\operatorname{Cl}\left(U_n\right) \subseteq U\) (vedi \href{20250103144944-chiusura_topologica.org}{Chiusura Topologica}) and \(\operatorname{diam}\left(U_n\right)<\varepsilon\) (vedi \href{20250327131547-diametro_di_un_insieme.org}{Diametro di un insieme}), for all \(n \in \omega\)
\subsection{Controesempio}
\label{sec:org14d7f40}

Sia \(X \coloneqq \R\times [0,1]\), dotato della \href{20250301193511-spazio_metrico.org}{distanza}:
\begin{equation*}
d\left((x,t),(y,s)\right) \coloneqq \begin{cases}
2 & x\neq y\\
|t-s| & x=y
\end{cases}
\end{equation*}

La funzione \(d\) è realmente una distanza: per ogni \((x,t),(y,s), (z,k) \in X\)
\begin{enumerate}
\item \(d\left((x,t),(y,s)\right)\ge 0\);
\item \(d\left((x,t),(y,s)\right) = 0\) se e solo se \(x=y\) e \(|t-s|=0\) se e solo se \((x,t) = (y,s)\);
\item \(d\left((x,t),(y,s)\right) = d\left((y,s), (x,t)\right)\);
\item la \href{20250306115949-disuguaglianza_triangolare.org}{disuguaglianza triangolare}:
\begin{equation*}
 	d\left((x,t),(y,s)\right)\le d\left((x,t),(z,k)\right)+d\left((z,k),(y,s)\right)
\end{equation*}
per casi:
\begin{itemize}
\item se \(x\neq y \neq z\) allora
\begin{align*}
   d\left((x,t),(y,s)\right) &= 2\\
   d\left((x,t),(z,k)\right) &= 2\\
   d\left((z,k),(y,s)\right) &= 2
\end{align*}
e quindi
\begin{equation*}
   2=d\left((x,t),(y,s)\right) \le d\left((x,t),(z,k)\right)+d\left((z,k),(y,s)\right)=4
\end{equation*}
\item se \(x=y\neq z\) allora
\begin{align*}
   d\left((x,t),(y,s)\right)&\le 1\\
   d\left((x,t),(z,k)\right) = 2 &= d\left((z,k),(y,s)\right)
\end{align*}
e quindi si ha la tesi;
\item se \(x=z\neq y\) oppure \(y=z \neq x\) (solo il primo per simmetria):
\begin{align*}
   d\left((x,t),(y,s)\right) &= 2\\
   d\left((x,t),(z,k)\right) &= \ell \le 1\\
   d\left((z,k),(y,s)\right) &= 2
\end{align*}
e quindi
\begin{equation*}
   2 = d\left((x,t),(y,s)\right)\le d\left((x,t),(z,k)\right)+d\left((z,k),(y,s)\right) = \ell + 2
\end{equation*}
\item se \(x=y=z\) allora
\begin{equation*}
   |t-s|\le |t-k| + |k-s|.
\end{equation*}
\end{itemize}
\end{enumerate}

Dunque \(\left(\R\times[0,1], d\right)\) è uno \href{20250301193511-spazio_metrico.org}{spazio metrico}.

L'aperto \(\R\times [0,1]\) non ammette alcun \href{20250103164252-ricoprimento.org}{ricoprimento} \((U_{n})_{n \in \omega}\) tale che, per ogni \(n \in \omega\): \(\operatorname{diam}(U_{n})< 1/2\). Si supponga per assurdo che esista.

Fissato \(U_{n}\): per ogni \((x,t), (y,s) \in U_{n}\): \(d\left((x,t),(y,s)\right)<1/2\), e quindi \(x=y\). Pertanto esiste \(x_{n} \in \R\) tale che \(U_{n} \subseteq \set{x_{n}}\times [0,1]\).

Si ha quindi che
\begin{equation*}
\bigcup_{n \in \omega} U_{n} \subseteq \set{x_{n}\mid n \in \omega}\times [0,1] \subsetneqq \R\times[0,1]
\end{equation*}
e pertanto \((U_{n})_{n \in \omega}\) non è un ricoprimento. Assurdo.
\end{document}
