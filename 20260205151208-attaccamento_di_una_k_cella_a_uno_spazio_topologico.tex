% Created 2026-02-07 Sat 19:32
% Intended LaTeX compiler: pdflatex
\documentclass[10pt]{article}
%% CREATO CON ORG - EMACS
\newcommand{\use}[2][]{\usepackage[#1]{#2}}
% PACCHETTI FONDAMENTLAI
\use[utf8]{inputenc}
\use[T1]{fontenc}
\use{graphicx}
\use{longtable}
\use{wrapfig}
\use{rotating}
\use[normalem]{ulem}
\use{amsmath}
\use{amsthm}
\use{amssymb}

\use{eucal} % Cambia mathcal{...}

\use{capt-of}
\use[italian]{babel}
\use[babel]{csquotes}
% bib la TEX lo carica in automatico org-cite
\use{microtype}
\use{lmodern}
\use{subfig} % sottofigure
\use{multicol} % due colonne
\use{lipsum} % lorem ipsum
\use{color} % colori in latex
\use{parskip} % rimuove l'indentazione dei nuovi paragrafi %% Add parbox=false to all new tcolorbox
\use{centernot}
\use[outline]{contour}\contourlength{3pt}
\use{fancyhdr}
\use{layout}
\use[most]{tcolorbox} % Riquadri colorati
\use{ifthen} % IFTHEN
\use{geometry}

% pacchetti matematica
\use{yhmath}
\use{dsfont}
\use{mathrsfs}
\use{cancel} % semplificare
\use{polynom} %divisione tra polinomi
\use{forest} % grafi ad albero
\use{booktabs} % tabelle
\use{commath} %simboli e differenziali
\use{bm} %bold
\use[fulladjust]{marginnote} %to use marginnote for date notes
\use{arrayjobx}%array
\use[intlimits]{empheq} % Riquadri colorati attorno alle equazioni
\use{mathtools}
\use{circuitikz} % Disegnare i circuiti
\use{mathtools}
\use{stmaryrd} % [[ \llbracket ]] \rrbracket
\use{bussproofs} % dimostrazioni

%%%%%%%%%%%%%


%%%% QUIVER
\newcommand{\duepunti}{\,\mathchar\numexpr"6000+`:\relax\,}
% A TikZ style for curved arrows of a fixed height, due to AndréC.
\tikzset{curve/.style={settings={#1},to path={(\tikztostart)
    .. controls ($(\tikztostart)!\pv{pos}!(\tikztotarget)!\pv{height}!270:(\tikztotarget)$)
    and ($(\tikztostart)!1-\pv{pos}!(\tikztotarget)!\pv{height}!270:(\tikztotarget)$)
    .. (\tikztotarget)\tikztonodes}},
    settings/.code={\tikzset{quiver/.cd,#1}
        \def\pv##1{\pgfkeysvalueof{/tikz/quiver/##1}}},
    quiver/.cd,pos/.initial=0.35,height/.initial=0}

% TikZ arrowhead/tail styles.
\tikzset{tail reversed/.code={\pgfsetarrowsstart{tikzcd to}}}
\tikzset{2tail/.code={\pgfsetarrowsstart{Implies[reversed]}}}
\tikzset{2tail reversed/.code={\pgfsetarrowsstart{Implies}}}
% TikZ arrow styles.
\tikzset{no body/.style={/tikz/dash pattern=on 0 off 1mm}}
%%%%%%%%%%


%% DEFINIZIONI COMANDI MATEMATICI
\let\sin\relax %TOGLIE LA DEFINIZIONE SU "\sin"

% cambia la definizione di empty set
% ---
\let\oldemptyset\emptyset
% ---
% \let\emptyset\varnothing
% ---
% \let\emptyset\relax
% \newcommand{\emptyset}{\text{\textnormal{\O}}}
% ---

\DeclareMathOperator{\bounded}{bd}
\DeclareMathOperator{\sin}{sen}
\DeclareMathOperator{\epi}{Epi}
\DeclareMathOperator{\cl}{cl}
\DeclareMathOperator{\graph}{graph}
\DeclareMathOperator{\arcsec}{arcsec}
\DeclareMathOperator{\arccot}{arccot}
\DeclareMathOperator{\arccsc}{arccsc}
\DeclareMathOperator{\spettro}{Spettro}
\DeclareMathOperator{\nulls}{nullspace}
\DeclareMathOperator{\dom}{dom}
\DeclareMathOperator{\ar}{ar}
\DeclareMathOperator{\const}{Const}
\DeclareMathOperator{\fun}{Fun}
\DeclareMathOperator{\rel}{Rel}
\DeclareMathOperator{\altezza}{ht}
\let\det\relax %TOGLIE LA DEFINIZIONE SU "\det"
\DeclareMathOperator{\det}{det}
\DeclareMathOperator{\End}{End}
\DeclareMathOperator{\gl}{GL}
\def\Id{\mathrm{Id}}
\def\id{\mathrm{id}}
\DeclareMathOperator{\I}{\mathds{1}}
\DeclareMathOperator{\II}{II}
\DeclareMathOperator{\rank}{rank}
\DeclareMathOperator{\tr}{tr}
\DeclareMathOperator{\tc}{t.c.}
\DeclareMathOperator{\T}{T}
\DeclareMathOperator{\var}{Var}
\DeclareMathOperator{\cov}{Cov}
\DeclareMathOperator{\st}{st}
\DeclareMathOperator{\mon}{Mon}
\newcommand{\card}[1]{\left\vert #1 \right\vert}
\newcommand{\trasposta}[1]{\prescript{\text{T}}{}{#1}}
\newcommand{\1}{\mathds{1}}
\newcommand{\R}{\mathds{R}}
\newcommand{\diesis}{\#}
\newcommand{\bemolle}{\flat}
\newcommand{\nonstandard}[1]{\prescript{*}{}{#1}}
\newcommand{\starR}{\nonstandard{\R}}
\newcommand{\borel}{\mathscr{B}}
\newcommand{\lebesgue}[1]{\mathscr{L}\left(#1\right)}
\newcommand{\media}{\mathds{E}}
\newcommand{\K}{\mathds{K}}
\newcommand{\A}{\mathds{A}}
\newcommand{\Q}{\mathds{Q}}
\newcommand{\N}{\mathds{N}}
\newcommand{\C}{\mathds{C}}
\newcommand{\Z}{\mathds{Z}}
\newcommand{\qo}{\hspace{1em}\text{q.o.}\,}
\renewcommand{\tilde}[1]{\widetilde{#1}}
\renewcommand{\parallel}{\mathrel{/\mkern-5mu/}}
\newcommand{\parti}[2][]{\wp_{#1}(#2)}
\newcommand{\diff}[1]{\operatorname{d}_{#1}}
\let\oldvec\vec
\renewcommand{\vec}[1]{\overrightarrow{\vphantom{i}#1}}
\newcommand{\floor}[1]{\left\lfloor #1 \right\rfloor}
\newcommand{\cat}[1]{\mathbf{#1}}
\newcommand{\dfreccia}[1]{\xrightarrow{\ #1 \ }}
\newcommand{\sfreccia}[1]{\xleftarrow{\ #1 \ }}
\newcommand{\formalsum}[2]{{\sum_{#1}^{#2}}{\vphantom{\sum}}'}
\newcommand{\minim}[2]{\mu_{#1}\, \left(#2\right)}
\newcommand{\concat}{\null^{\frown}} % concatenazione di stringe
\newcommand{\godelcode}[1]{\langle\!\langle #1 \rangle\!\rangle}
\newcommand{\godeldec}[1]{(\!(#1)\!)}
\newcommand{\termcode}[1]{\ulcorner #1\urcorner}
\newcommand{\partialto}{\dashrightarrow}
\newcommand{\restricted}{\upharpoonright}
\newcommand{\embeds}{\precsim}
\newcommand{\surjects}{\twoheadrightarrow}
\newcommand{\equipotenti}{\asymp}
%% \newcommand{\dotplus}{\mathbin{\dot{+}}} %% A quanto pare esiste già
\newcommand{\bigdot}{\mathbin{\boldsymbol{\cdot}}}
\newcommand{\dotexp}[1]{^{.#1}}
\newcommand{\conv}{\mathbin{*}}
\newcommand{\convolution}[2]{(#1\conv #2)}
\newcommand{\nil}{\mathfrak{N}}
\newcommand{\divisore}{\mathrel{|}}
\newcommand{\simplesso}[1]{\mathrm{e}_{#1}}

\renewcommand{\iff}{\mathrel{\longleftrightarrow}} %% Notazione Logica.
\newcommand{\oldiff}{\mathrel{\Longleftrightarrow}}
\renewcommand{\implies}{\mathrel{\rightarrow}} %% Notazione Logica
\newcommand{\oldimplies}{\mathrel{\Longrightarrow}}
\renewcommand{\impliedby}{\mathrel{\leftarrow}} %% Notazione Logica
\newcommand{\oldimpliedby}{\mathrel{\Longleftarrow}}

\newcommand{\IFF}{\quad\Longleftrightarrow\quad}
\newcommand{\IMPLICA}{\quad\Longrightarrow\quad}


\renewcommand{\descriptionlabel}[1]{\hspace{\labelsep}\normalfont #1} % remove bold from description


%% Definizione di Divergenza di K-L

\DeclarePairedDelimiterX{\infdivx}[2]{(}{)}{%
  #1\;\delimsize\|\;#2%
}
\newcommand{\kldiv}{D_{KL}\infdivx}

%% Definizione di \dotminus

\makeatletter
\newcommand{\dotminus}{\mathbin{\text{\@dotminus}}}

\newcommand{\@dotminus}{%
  \ooalign{\hidewidth\raise1ex\hbox{.}\hidewidth\cr$\m@th-$\cr}%
}
\makeatother

%tramite i prossimi due comandi posso decidere come scrivere i logaritmi naturali in tutti i documenti: ho infatti eliminato qualsiasi differenza tra "ln" e "log": se si vuole qualcosa di diverso bisogna inserire manualmente il tutto
\let\ln\relax
\DeclareMathOperator{\ln}{ln}
\let\log\relax
\DeclareMathOperator{\log}{log}
%%%%%%

%% NUOVI COMANDI
\newcommand{\straniero}[1]{\textit{#1}} %parole straniere
\newcommand{\titolo}[1]{\textsc{#1}} %titoli
\newcommand{\qedd}{\tag*{$\blacksquare$}} %qed per ambienti matemastici
\renewcommand{\qedsymbol}{$\blacksquare$} %modifica colore qed
\newcommand{\ooverline}[1]{\overline{\overline{#1}}}
\newcommand{\circoletto}[1]{\left(#1\right)^{\text{o}}}
%
\newcommand{\qmatrice}[1]{\begin{pmatrix}
#1_{11} & \cdots & #1_{1n}\\
\vdots & \ddots & \vdots \\
#1_{m1} & \cdots & #1_{mn}
\end{pmatrix}}
%
\newcommand{\parentesi}[2]{%
\underset{#1}{\underbrace{#2}}%
}
%
\newcommand{\norma}[1]{% Norma
\left\lVert#1\right\rVert%
}
\newcommand{\scalare}[2]{% Scalare
\left\langle #1, #2\right\rangle
}
%%%%%

%% RESTRIZIONI
\newcommand{\referenze}[2]{
        \phantomsection{}#2\textsuperscript{\textcolor{blue}{\textbf{#1}}}
}

\let\restriction\relax

\def\restriction#1#2{\mathchoice
              {\setbox1\hbox{${\displaystyle #1}_{\scriptstyle #2}$}
              \restrictionaux{#1}{#2}}
              {\setbox1\hbox{${\textstyle #1}_{\scriptstyle #2}$}
              \restrictionaux{#1}{#2}}
              {\setbox1\hbox{${\scriptstyle #1}_{\scriptscriptstyle #2}$}
              \restrictionaux{#1}{#2}}
              {\setbox1\hbox{${\scriptscriptstyle #1}_{\scriptscriptstyle #2}$}
              \restrictionaux{#1}{#2}}}
\def\restrictionaux#1#2{{#1\,\smash{\vrule height .8\ht1 depth .85\dp1}}_{\,#2}}
%%%%%%%%%%%

%%% FORMATTAZIONE FOOTNOTEMARK

\def\footnotemarkformatting#1{[#1]}
\renewcommand{\thefootnote}{\footnotemarkformatting{\arabic{footnote}}}

%% SEZIONE GRAFICA
\use{tikz}
\usetikzlibrary{matrix, patterns, calc, decorations.pathreplacing, hobby, decorations.markings, decorations.pathmorphing, babel}
\use{tikz-3dplot}
\use{mathrsfs} %per geogebra
\use{tikz-cd}
\tikzset
{
  %surface/.style={fill=black!10, shading=ball,fill opacity=0.4},
  plane/.style={black,pattern=north east lines},
  curve/.style={black,line width=0.5mm},
  dritto/.style={decoration={markings,mark=at position 0.5 with {\arrow{Stealth}}}, postaction=decorate},
  rovescio/.style={decoration={markings,mark=at position 0.5 with {\arrow{Stealth[reversed]}}}, postaction=decorate}
}
\use{pgfplots} % stampare le funzioni
        \pgfplotsset{/pgf/number format/use comma,compat=1.15}
        %\pgfplotsset{compat=1.15} %per geogebra
        \usepgfplotslibrary{fillbetween, polar}
%%%%%%

%% CITAZIONI
\use{lineno}

\newcommand{\citazione}[1]{%
  \begin{quotation}
  \begin{linenumbers}
  \modulolinenumbers[5]
  \begingroup
  \setlength{\parindent}{0cm}
  \noindent #1
  \endgroup
  \end{linenumbers}
  \end{quotation}\setcounter{linenumber}{1}
  }
%%%%%%

%%%%%%%%%%%%%%%%%%%%%%%%%%%%%%%%%%%%%%%%%%%%
%%%%%%%%%%%%%%%%%%%%%%%%%%%%%%%%%%%%%%%%%%%%

%% AMS THM

\theoremstyle{definition}% default
\newtheorem{thm}{Teorema}[section]
\newtheorem{lem}[thm]{Lemma}
\newtheorem{prop}[thm]{Proposizione}
\newtheorem{cor}[thm]{Corollario}
\newtheorem{esempio}[thm]{Esempio}
\theoremstyle{plain}
\newtheorem{definizione}[thm]{Definizione}
\theoremstyle{remark}
\newtheorem*{oss}{Osservazione}


%%%%%%%%%%%%%%%%%%%%%%%%%%%%%%%%%%%%%%%%%%%%
%%%%%%%%%%%%%%%%%%%%%%%%%%%%%%%%%%%%%%%%%%%%

\use{hyperref}
\hypersetup{%
        pdfauthor={Davide Peccioli},
        pdfsubject={},
        allcolors=black,
        citecolor=black,
%	colorlinks=true,
        bookmarksopen=true}
\setcounter{secnumdepth}{0} % rimuove i numeri di sezione senza rimuovere le ref
\renewcommand{\href}[2]{\textcolor{blue}{#2}} % disabilita il comando href
\use{enotez} %
\setenotez{%
 mark-format = \footnotemarkformatting % Mette i numeri tra parentesi quadre%
}\let\footnote=\endnote % rende tutte le note a pié pagina come delle note a fine file 


\let\olddocument\document % modifico l'ambiende documenti per non dover stampare \printendnote
\let\oldenddocument\enddocument
\renewenvironment{document}%
{%
  \olddocument
}{%
  \printendnotes\oldenddocument
}
\renewcommand{\thethm}{\arabic{thm}}

\usepackage[hyperref]{biblatex}
\addbibresource{~/Documents/org/roam/bib/master.bib}
\author{Davide Peccioli}
\date{\today}
\title{}
\begin{document}

\section{Attaccamento di una k-cella a uno spazio topologico}
\label{sec:org9696b4f}
\begin{definizione}
Una \textbf{\(k\)-cella} (o cella di dimensione \(k\)), denotata solitamente con \(e^k\), è uno \href{20250103145124-topologia.org}{spazio topologico} \href{20250111142332-omeomorfismo.org}{omeomorfo} alla \href{20250127170831-disco_n_dimensionale.org}{palla chiusa unitaria} in \(\R^k\):
\begin{equation*}
e^{k} \mathrel{\overset{\text{omeo}}{\sim}}\mathds{D}^k = \{ x \in \R^k : \|x\| \le 1 \}
\end{equation*}
Il \href{20250129161026-bordo.org}{bordo} \(\partial e^{k}\) è omeomorfo alla \href{20250115150754-sfera_n_dimensionale.org}{sfera} \(\mathds{S}^{k-1}\)
\end{definizione}

\begin{definizione}
Un \textbf{collare} (o \(k\)-collare) di \(e^{k}\) è un sottoinsieme \(C \subseteq e^{k}\) tale che \(\partial e^{k} \subseteq C\) è un \href{20250129112659-retratto_di_deformazione_forte_di_uno_spazio_topologico.org}{retratto di deformazione forte}.
\end{definizione}

\begin{definizione}
Dato uno \href{20250103145124-topologia.org}{spazio topologico} \(X\), \(e^{k}\) una \(k\)-cella e una \href{20250103103252-funzione_continua.org}{mappa continua} \(\varphi : \partial e^{k} \to X\), detta \emph{mappa di attaccamento}, l'\textbf{attaccamento di \(e^{k}\) ad \(X\) tramite \(\varphi\)} è lo spazio \href{20250114100810-quoziente_rispetto_a_relazione_di_equivalenza.org}{quoziente}:
\begin{equation*}
X \mathrel{\cup_{\varphi}} e^k \coloneqq (X \amalg e^k) / \sim
\end{equation*}
dove la \href{20250113110148-relazione_di_equivalenza.org}{relazione di equivalenza} \(\sim\):
\begin{equation*}
x\sim y \IFF%
\begin{cases}
x = y\\
x \in \partial e^{k},\ y = \varphi(x)\\
y \in \partial e^{k},\ x=\varphi(y)\\
x,y \in \partial e^{k},\ \varphi(x)=\varphi(y).
\end{cases}
\end{equation*}

La \href{20250103145124-topologia.org}{topologia} su \(X\mathrel{\cup_{\varphi}} e^k\) è la \href{20250129155316-spazio_topologico_quoziente.org}{topologia quoziente} rispetto alla proiezione \(\pi:X \amalg e^{k} \longrightarrow (X \amalg e^k) / \sim\).
\end{definizione}

\textbf{Mappe indotte.}
Sia quindi \(\varphi: \partial e^{k}\to X\) fissata, \(Y\coloneqq X \mathrel{\cup_{\varphi}} e^k\). Questa definisce:
\begin{enumerate}
\item \(\varphi\) si chiama \uline{mappa di attaccamento};
\item \(\Phi: e^{k}\longrightarrow Y\) è la \uline{mappa caratteristica} di \(e^{k}\):
\begin{equation*}
 e^{k} \hookrightarrow X \amalg e^{k} \xrightarrow{\ \pi\ } X \mathrel{\cup_{\varphi}} e^k.
\end{equation*}
\item \(j: X \hookrightarrow X \amalg e^{k} \xrightarrow{\ \pi\ } X \mathrel{\cup_{\varphi}} e^k\) è un omeomorfismo sull'\href{20250202190147-immagine_punto_a_punto_di_due_classi.org}{immagine}.
\end{enumerate}

\begin{oss}
Si hanno le seguenti proprietà:
\begin{enumerate}
\item \(\Phi(\mathring{e}^{k})\) è aperto in \(Y\);
\item se l'\href{20250202190147-immagine_punto_a_punto_di_due_classi.org}{immagine} \(\varphi(\partial e^{k})\) è chiusa in \(X\), allora \(j\big(X\setminus \varphi(\partial e^{k})\big)\) è aperto in \(Y\).
\end{enumerate}
\end{oss}

\begin{prop}
Sia \(X\) \href{20250103145124-topologia.org}{spazio topologico} di \href{20250109155715-spazio_topologico_di_hausdorff.org}{Haussdorff}, \(\varphi: \partial e^{k}\longrightarrow X\) mappa di attaccamento, \(Y\coloneqq X\mathrel{\cup_{\varphi}} e^k\). Allora
\begin{enumerate}
\item \(Y\) è di \href{20250109155715-spazio_topologico_di_hausdorff.org}{Hausdorff};
\item Se \(C \subseteq e^{k}\) è un collare di \(\partial e^{k}\), allora
\begin{equation*}
 j(X) \subseteq j(X)\cup \Phi(C)
\end{equation*}
è un \href{20250129112659-retratto_di_deformazione_forte_di_uno_spazio_topologico.org}{retratto di deformazione forte}.
\end{enumerate}
\end{prop}

\begin{proof}
\begin{enumerate}
\item Siano \(x,y \in Y\), \(x\neq y\). Si studiano diversi casi.
\begin{itemize}
\item Se sono nella \href{20250122181431-parte_interna.org}{parte interna}:
\begin{equation*}
     x,y \in \Phi(\mathring{e}^{k})
\end{equation*}
allora esistono due aperti \(U_{x}, U_{y} \subseteq \mathring{e}^{k}\) disgiunti, tali che \(\Phi(U_{x}), \Phi(U_{y}) \subseteq \Phi(\mathring{e}^{k})\) sono intorni aperti disgiunti di \(x,y\). \href{20250103163814-sottospazio_topologico.org}{Segue} che lo sono in \(Y\).
\item Se \(x,y \in j\big(X\setminus \varphi(\partial e^{k})\big)\):
\begin{itemize}
\item siccome \(X\) è \href{20250109155715-spazio_topologico_di_hausdorff.org}{Hausdorff} e \(\varphi(\partial e^{k})\) è \href{20250103163701-spazio_topologico_compatto.org}{compatto}\footnote{Infatti \href{20251229125103-immagine_continua_di_spazio_compatto_e_compatto.org}{immagine continua di un compatto è compatto}.}, \href{20250331174140-compatto_in_un_haussdorf_e_chiuso.org}{allora} \(\varphi(\partial e^{k})\) è chiuso;
\item quindi \(j\big(X\setminus \varphi(\partial e^{k})\big)\) è aperto di \(Y\);
\item come per il punto precedente, si trovano due intorni aperti disgiunti di \(x,y\) in \(j\big(X\setminus \varphi(\partial e^{k})\big)\), che lo sono anche per \(Y\).
\end{itemize}
\item Se \(x \in j\big(X\setminus \varphi(\partial e^{k})\big)\) e \(y \in \Phi(\mathring{e}^{k})\), per quanto detto nei punti precedenti
 \begin{equation*}
j\big(X\setminus \varphi(\partial e^{k})\big),\qquad %
\Phi(\mathring{e}^{k})
 \end{equation*}
sono entrambi aperti, e disgiunti.
\item Se \(x,y \in j\circ \varphi(\partial e^{k}) = \Phi(\partial e^{k})\): siano \(A_{x}, A_{y} \subseteq X\) intorni aperti disgiunti delle retroimmagini di \(x,y\), e consideriamo
 \begin{equation*}
V_{x}' \coloneqq \varphi^{-1}(A_{x}),\qquad V_{y}' \coloneqq \varphi^{-1}(A_{y}),\qquad V_{x}', V_{y}' \subseteq \partial e^{k}.
 \end{equation*}
Poiché \(A_{x}\) e \(A_{y}\) sono disgiunti, allora anche \(V_{x}, V_{y}\) sono disgiunti. Inoltre necessariamente esistono \(V_{x}, V_{y} \subseteq e^{k}\) aperti disgiunti tali che
 \begin{equation*}
V_{x}\cap \partial e^{k} = V_{x}',\qquad V_{y}\cap \partial e^{k} = V_{y}'.
 \end{equation*}

Gli aperti \(j(A_{x}) \cup \Phi(V_{x})\) e \(j(A_{y}) \cup \Phi(V_{y})\) sono quelli cercati.
\end{itemize}

\item Sia \(H: C\times [0,1] \longrightarrow C\) la mappa che rende \(\partial e^{k} \subseteq C\) un \href{20250129112659-retratto_di_deformazione_forte_di_uno_spazio_topologico.org}{retratto di deformazione forte}, \(r: C \longrightarrow \partial e^{k}\).

Definiamo \(\hat{H}:X \amalg C \times [0,1] \longrightarrow X \amalg C\) come segue:
\begin{equation*}
 \hat{H}(x,t) = \begin{cases}
 	H(x,t) & x \in C\\
 	x & x \in X
 \end{cases}
\end{equation*}
\begin{itemize}
\item \uline{\(\hat{H}\) è passa al quoziente}: se \(x\sim y\), allora \(\hat{H}(x,t) \sim \hat{H}(y,t)\).
\begin{itemize}
\item se \(x \in X\) e \(y \in \partial e^{k}\) tali che \(x = \varphi(y)\), allora
\begin{align*}
 \hat{H}(x,t) &= x\\
 \hat{H}(y,t) &= H(y,t) = y
\end{align*}

\item se \(x,y \in \partial e^{k}\) tali che \(\varphi(x)=\varphi(y)\), allora
\begin{equation*}
 \hat{H}(x,t) = x,\qquad \hat{H}(y,t) = y
\end{equation*}
\end{itemize}

\item Lo spazio \((X\amalg C) / \sim\ =\ j(X) \cup \Phi(C)\).

\item Posso quindi definire una funzione
\begin{align*}
\bm{H}: \big(j(X) \cup \Phi(C)\big) \times [0,1] &\longrightarrow j(X) \cup \Phi(C)\\
\big([x]_{\sim},t\big) &\longmapsto \big[H(x,t)\big]_{\sim}
\end{align*}
che è l'identità su \(j(X)\).

\item \(\bm{H}\) è una omotopia:
 \begin{align*}
\bm{H}(x,0) &= \begin{cases}
	\Phi\big(H(\Phi^{-1}(x),0)\big) = x & x \in \Phi(C)\\
	x & x \in j(X)
\end{cases}\\[3ex]
\bm{H}(x,1) &= \begin{cases}
	\Phi\big(H(\Phi^{-1}(x),1)\big) & x \in \Phi(C)\\
	x & x \in j(X)
\end{cases}\\
&= \begin{cases}
	\Phi\big(r\big(\Phi^{-1}(x)\big)\big) & x \in \Phi(C)\\
	x & x \in j(X)
\end{cases}
 \end{align*}
e quindi \(\bm{H}(x,1) \in j(X)\), in quanto per ogni \(y \in \partial e^{k}\), \(\Phi(y) \in j(X)\).
\qedhere
\end{itemize}
\end{enumerate}
\end{proof}

D'ora in avanti si potrà considerare che \(X \subseteq X\mathrel{\cup_{\varphi}} e^k\).
\subsection{Attaccamento di una famiglia di k-celle ad uno spazio topologico}
\label{sec:org16d9825}
\begin{definizione}
Sia \(X\) uno \href{20250103145124-topologia.org}{spazio topologico} e sia \(\{ e^k_\alpha \}_{\alpha \in A}\) una famiglia di \(k\)-\hyperref[sec:org9696b4f]{celle} indicizzata da un insieme \(A\).
Per ogni \(\alpha \in A\), sia data una \href{20250103103252-funzione_continua.org}{mappa continua} dal \href{20250129161026-bordo.org}{bordo} di \(e^{k}_{\alpha}\) per ogni \(\alpha\): \(\varphi_\alpha: \partial e^k_\alpha \to X\).

L'\textbf{attaccamento della famiglia di celle \(\{ e^k_\alpha \}\) ad \(X\)} è lo spazio \href{20250114100810-quoziente_rispetto_a_relazione_di_equivalenza.org}{quoziente}:
\begin{equation*}
Y= X \mathrel{\cup_{\{\varphi_\alpha\}}} \left( \coprod_{\alpha \in A} e^k_\alpha \right) \coloneqq \left( X \amalg \coprod_{\alpha \in A} e^k_\alpha \right) \bigg/ \sim
\end{equation*}
dove \(\amalg\) denota l'\href{20250113175700-unione_disgiunta.org}{unione disgiunta topologica} e la relazione \(\sim\) è data da
\begin{equation*}
x \sim y \IFF %
\begin{cases}
x = y \\[2ex]
x \in \partial e^k_\alpha, \ y = \varphi_\alpha(x) \in X\\[2ex]
y \in \partial e^k_\alpha, \ x = \varphi_\alpha(y) \in X\\[2ex]
\exists \alpha, \beta \in A :\ x \in \partial e^k_\alpha, \ y \in \partial e^k_\beta \text{ e } \varphi_\alpha(x) = \varphi_\beta(y)
\end{cases}
\end{equation*}
\end{definizione}

Per ogni \(\alpha \in A\), è indotta la \uline{mappa caratteristica di \(e^{k}_{\alpha}\)}.
\begin{equation*}
	\Phi_{\alpha} :\qquad e^{k}_{\alpha} \hookrightarrow X \amalg \coprod_{\alpha \in A} e^k_\alpha \xrightarrow{\ \pi\ } \left( X \amalg \coprod_{\alpha \in A} e^k_\alpha \right) \bigg/ \sim.
\end{equation*}

La topologia su \(Y\) è la \href{20250129155316-spazio_topologico_quoziente.org}{topologia quoziente} indotta dalla proiezione
\begin{equation*}
\pi: \left(X \amalg \coprod_{\alpha \in A} e^k_\alpha \right) \longrightarrow \left( X \amalg \coprod_{\alpha \in A} e^k_\alpha \right) \bigg/ \sim
\end{equation*}

Si può considerare \(X \subseteq X \mathrel{\cup_{\{\varphi_\alpha\}}} \left( \coprod_{\alpha \in A} e^k_\alpha \right)\)
\end{document}
