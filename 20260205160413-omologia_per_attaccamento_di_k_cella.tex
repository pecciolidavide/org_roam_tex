% Created 2026-02-07 Sat 19:30
% Intended LaTeX compiler: pdflatex
\documentclass[10pt]{article}
%% CREATO CON ORG - EMACS
\newcommand{\use}[2][]{\usepackage[#1]{#2}}
% PACCHETTI FONDAMENTLAI
\use[utf8]{inputenc}
\use[T1]{fontenc}
\use{graphicx}
\use{longtable}
\use{wrapfig}
\use{rotating}
\use[normalem]{ulem}
\use{amsmath}
\use{amsthm}
\use{amssymb}

\use{eucal} % Cambia mathcal{...}

\use{capt-of}
\use[italian]{babel}
\use[babel]{csquotes}
% bib la TEX lo carica in automatico org-cite
\use{microtype}
\use{lmodern}
\use{subfig} % sottofigure
\use{multicol} % due colonne
\use{lipsum} % lorem ipsum
\use{color} % colori in latex
\use{parskip} % rimuove l'indentazione dei nuovi paragrafi %% Add parbox=false to all new tcolorbox
\use{centernot}
\use[outline]{contour}\contourlength{3pt}
\use{fancyhdr}
\use{layout}
\use[most]{tcolorbox} % Riquadri colorati
\use{ifthen} % IFTHEN
\use{geometry}

% pacchetti matematica
\use{yhmath}
\use{dsfont}
\use{mathrsfs}
\use{cancel} % semplificare
\use{polynom} %divisione tra polinomi
\use{forest} % grafi ad albero
\use{booktabs} % tabelle
\use{commath} %simboli e differenziali
\use{bm} %bold
\use[fulladjust]{marginnote} %to use marginnote for date notes
\use{arrayjobx}%array
\use[intlimits]{empheq} % Riquadri colorati attorno alle equazioni
\use{mathtools}
\use{circuitikz} % Disegnare i circuiti
\use{mathtools}
\use{stmaryrd} % [[ \llbracket ]] \rrbracket
\use{bussproofs} % dimostrazioni

%%%%%%%%%%%%%


%%%% QUIVER
\newcommand{\duepunti}{\,\mathchar\numexpr"6000+`:\relax\,}
% A TikZ style for curved arrows of a fixed height, due to AndréC.
\tikzset{curve/.style={settings={#1},to path={(\tikztostart)
    .. controls ($(\tikztostart)!\pv{pos}!(\tikztotarget)!\pv{height}!270:(\tikztotarget)$)
    and ($(\tikztostart)!1-\pv{pos}!(\tikztotarget)!\pv{height}!270:(\tikztotarget)$)
    .. (\tikztotarget)\tikztonodes}},
    settings/.code={\tikzset{quiver/.cd,#1}
        \def\pv##1{\pgfkeysvalueof{/tikz/quiver/##1}}},
    quiver/.cd,pos/.initial=0.35,height/.initial=0}

% TikZ arrowhead/tail styles.
\tikzset{tail reversed/.code={\pgfsetarrowsstart{tikzcd to}}}
\tikzset{2tail/.code={\pgfsetarrowsstart{Implies[reversed]}}}
\tikzset{2tail reversed/.code={\pgfsetarrowsstart{Implies}}}
% TikZ arrow styles.
\tikzset{no body/.style={/tikz/dash pattern=on 0 off 1mm}}
%%%%%%%%%%


%% DEFINIZIONI COMANDI MATEMATICI
\let\sin\relax %TOGLIE LA DEFINIZIONE SU "\sin"

% cambia la definizione di empty set
% ---
\let\oldemptyset\emptyset
% ---
% \let\emptyset\varnothing
% ---
% \let\emptyset\relax
% \newcommand{\emptyset}{\text{\textnormal{\O}}}
% ---

\DeclareMathOperator{\bounded}{bd}
\DeclareMathOperator{\sin}{sen}
\DeclareMathOperator{\epi}{Epi}
\DeclareMathOperator{\cl}{cl}
\DeclareMathOperator{\graph}{graph}
\DeclareMathOperator{\arcsec}{arcsec}
\DeclareMathOperator{\arccot}{arccot}
\DeclareMathOperator{\arccsc}{arccsc}
\DeclareMathOperator{\spettro}{Spettro}
\DeclareMathOperator{\nulls}{nullspace}
\DeclareMathOperator{\dom}{dom}
\DeclareMathOperator{\ar}{ar}
\DeclareMathOperator{\const}{Const}
\DeclareMathOperator{\fun}{Fun}
\DeclareMathOperator{\rel}{Rel}
\DeclareMathOperator{\altezza}{ht}
\let\det\relax %TOGLIE LA DEFINIZIONE SU "\det"
\DeclareMathOperator{\det}{det}
\DeclareMathOperator{\End}{End}
\DeclareMathOperator{\gl}{GL}
\def\Id{\mathrm{Id}}
\def\id{\mathrm{id}}
\DeclareMathOperator{\I}{\mathds{1}}
\DeclareMathOperator{\II}{II}
\DeclareMathOperator{\rank}{rank}
\DeclareMathOperator{\tr}{tr}
\DeclareMathOperator{\tc}{t.c.}
\DeclareMathOperator{\T}{T}
\DeclareMathOperator{\var}{Var}
\DeclareMathOperator{\cov}{Cov}
\DeclareMathOperator{\st}{st}
\DeclareMathOperator{\mon}{Mon}
\newcommand{\card}[1]{\left\vert #1 \right\vert}
\newcommand{\trasposta}[1]{\prescript{\text{T}}{}{#1}}
\newcommand{\1}{\mathds{1}}
\newcommand{\R}{\mathds{R}}
\newcommand{\diesis}{\#}
\newcommand{\bemolle}{\flat}
\newcommand{\nonstandard}[1]{\prescript{*}{}{#1}}
\newcommand{\starR}{\nonstandard{\R}}
\newcommand{\borel}{\mathscr{B}}
\newcommand{\lebesgue}[1]{\mathscr{L}\left(#1\right)}
\newcommand{\media}{\mathds{E}}
\newcommand{\K}{\mathds{K}}
\newcommand{\A}{\mathds{A}}
\newcommand{\Q}{\mathds{Q}}
\newcommand{\N}{\mathds{N}}
\newcommand{\C}{\mathds{C}}
\newcommand{\Z}{\mathds{Z}}
\newcommand{\qo}{\hspace{1em}\text{q.o.}\,}
\renewcommand{\tilde}[1]{\widetilde{#1}}
\renewcommand{\parallel}{\mathrel{/\mkern-5mu/}}
\newcommand{\parti}[2][]{\wp_{#1}(#2)}
\newcommand{\diff}[1]{\operatorname{d}_{#1}}
\let\oldvec\vec
\renewcommand{\vec}[1]{\overrightarrow{\vphantom{i}#1}}
\newcommand{\floor}[1]{\left\lfloor #1 \right\rfloor}
\newcommand{\cat}[1]{\mathbf{#1}}
\newcommand{\dfreccia}[1]{\xrightarrow{\ #1 \ }}
\newcommand{\sfreccia}[1]{\xleftarrow{\ #1 \ }}
\newcommand{\formalsum}[2]{{\sum_{#1}^{#2}}{\vphantom{\sum}}'}
\newcommand{\minim}[2]{\mu_{#1}\, \left(#2\right)}
\newcommand{\concat}{\null^{\frown}} % concatenazione di stringe
\newcommand{\godelcode}[1]{\langle\!\langle #1 \rangle\!\rangle}
\newcommand{\godeldec}[1]{(\!(#1)\!)}
\newcommand{\termcode}[1]{\ulcorner #1\urcorner}
\newcommand{\partialto}{\dashrightarrow}
\newcommand{\restricted}{\upharpoonright}
\newcommand{\embeds}{\precsim}
\newcommand{\surjects}{\twoheadrightarrow}
\newcommand{\equipotenti}{\asymp}
%% \newcommand{\dotplus}{\mathbin{\dot{+}}} %% A quanto pare esiste già
\newcommand{\bigdot}{\mathbin{\boldsymbol{\cdot}}}
\newcommand{\dotexp}[1]{^{.#1}}
\newcommand{\conv}{\mathbin{*}}
\newcommand{\convolution}[2]{(#1\conv #2)}
\newcommand{\nil}{\mathfrak{N}}
\newcommand{\divisore}{\mathrel{|}}
\newcommand{\simplesso}[1]{\mathrm{e}_{#1}}

\renewcommand{\iff}{\mathrel{\longleftrightarrow}} %% Notazione Logica.
\newcommand{\oldiff}{\mathrel{\Longleftrightarrow}}
\renewcommand{\implies}{\mathrel{\rightarrow}} %% Notazione Logica
\newcommand{\oldimplies}{\mathrel{\Longrightarrow}}
\renewcommand{\impliedby}{\mathrel{\leftarrow}} %% Notazione Logica
\newcommand{\oldimpliedby}{\mathrel{\Longleftarrow}}

\newcommand{\IFF}{\quad\Longleftrightarrow\quad}
\newcommand{\IMPLICA}{\quad\Longrightarrow\quad}


\renewcommand{\descriptionlabel}[1]{\hspace{\labelsep}\normalfont #1} % remove bold from description


%% Definizione di Divergenza di K-L

\DeclarePairedDelimiterX{\infdivx}[2]{(}{)}{%
  #1\;\delimsize\|\;#2%
}
\newcommand{\kldiv}{D_{KL}\infdivx}

%% Definizione di \dotminus

\makeatletter
\newcommand{\dotminus}{\mathbin{\text{\@dotminus}}}

\newcommand{\@dotminus}{%
  \ooalign{\hidewidth\raise1ex\hbox{.}\hidewidth\cr$\m@th-$\cr}%
}
\makeatother

%tramite i prossimi due comandi posso decidere come scrivere i logaritmi naturali in tutti i documenti: ho infatti eliminato qualsiasi differenza tra "ln" e "log": se si vuole qualcosa di diverso bisogna inserire manualmente il tutto
\let\ln\relax
\DeclareMathOperator{\ln}{ln}
\let\log\relax
\DeclareMathOperator{\log}{log}
%%%%%%

%% NUOVI COMANDI
\newcommand{\straniero}[1]{\textit{#1}} %parole straniere
\newcommand{\titolo}[1]{\textsc{#1}} %titoli
\newcommand{\qedd}{\tag*{$\blacksquare$}} %qed per ambienti matemastici
\renewcommand{\qedsymbol}{$\blacksquare$} %modifica colore qed
\newcommand{\ooverline}[1]{\overline{\overline{#1}}}
\newcommand{\circoletto}[1]{\left(#1\right)^{\text{o}}}
%
\newcommand{\qmatrice}[1]{\begin{pmatrix}
#1_{11} & \cdots & #1_{1n}\\
\vdots & \ddots & \vdots \\
#1_{m1} & \cdots & #1_{mn}
\end{pmatrix}}
%
\newcommand{\parentesi}[2]{%
\underset{#1}{\underbrace{#2}}%
}
%
\newcommand{\norma}[1]{% Norma
\left\lVert#1\right\rVert%
}
\newcommand{\scalare}[2]{% Scalare
\left\langle #1, #2\right\rangle
}
%%%%%

%% RESTRIZIONI
\newcommand{\referenze}[2]{
        \phantomsection{}#2\textsuperscript{\textcolor{blue}{\textbf{#1}}}
}

\let\restriction\relax

\def\restriction#1#2{\mathchoice
              {\setbox1\hbox{${\displaystyle #1}_{\scriptstyle #2}$}
              \restrictionaux{#1}{#2}}
              {\setbox1\hbox{${\textstyle #1}_{\scriptstyle #2}$}
              \restrictionaux{#1}{#2}}
              {\setbox1\hbox{${\scriptstyle #1}_{\scriptscriptstyle #2}$}
              \restrictionaux{#1}{#2}}
              {\setbox1\hbox{${\scriptscriptstyle #1}_{\scriptscriptstyle #2}$}
              \restrictionaux{#1}{#2}}}
\def\restrictionaux#1#2{{#1\,\smash{\vrule height .8\ht1 depth .85\dp1}}_{\,#2}}
%%%%%%%%%%%

%%% FORMATTAZIONE FOOTNOTEMARK

\def\footnotemarkformatting#1{[#1]}
\renewcommand{\thefootnote}{\footnotemarkformatting{\arabic{footnote}}}

%% SEZIONE GRAFICA
\use{tikz}
\usetikzlibrary{matrix, patterns, calc, decorations.pathreplacing, hobby, decorations.markings, decorations.pathmorphing, babel}
\use{tikz-3dplot}
\use{mathrsfs} %per geogebra
\use{tikz-cd}
\tikzset
{
  %surface/.style={fill=black!10, shading=ball,fill opacity=0.4},
  plane/.style={black,pattern=north east lines},
  curve/.style={black,line width=0.5mm},
  dritto/.style={decoration={markings,mark=at position 0.5 with {\arrow{Stealth}}}, postaction=decorate},
  rovescio/.style={decoration={markings,mark=at position 0.5 with {\arrow{Stealth[reversed]}}}, postaction=decorate}
}
\use{pgfplots} % stampare le funzioni
        \pgfplotsset{/pgf/number format/use comma,compat=1.15}
        %\pgfplotsset{compat=1.15} %per geogebra
        \usepgfplotslibrary{fillbetween, polar}
%%%%%%

%% CITAZIONI
\use{lineno}

\newcommand{\citazione}[1]{%
  \begin{quotation}
  \begin{linenumbers}
  \modulolinenumbers[5]
  \begingroup
  \setlength{\parindent}{0cm}
  \noindent #1
  \endgroup
  \end{linenumbers}
  \end{quotation}\setcounter{linenumber}{1}
  }
%%%%%%

%%%%%%%%%%%%%%%%%%%%%%%%%%%%%%%%%%%%%%%%%%%%
%%%%%%%%%%%%%%%%%%%%%%%%%%%%%%%%%%%%%%%%%%%%

%% AMS THM

\theoremstyle{definition}% default
\newtheorem{thm}{Teorema}[section]
\newtheorem{lem}[thm]{Lemma}
\newtheorem{prop}[thm]{Proposizione}
\newtheorem{cor}[thm]{Corollario}
\newtheorem{esempio}[thm]{Esempio}
\theoremstyle{plain}
\newtheorem{definizione}[thm]{Definizione}
\theoremstyle{remark}
\newtheorem*{oss}{Osservazione}


%%%%%%%%%%%%%%%%%%%%%%%%%%%%%%%%%%%%%%%%%%%%
%%%%%%%%%%%%%%%%%%%%%%%%%%%%%%%%%%%%%%%%%%%%

\use{hyperref}
\hypersetup{%
        pdfauthor={Davide Peccioli},
        pdfsubject={},
        allcolors=black,
        citecolor=black,
%	colorlinks=true,
        bookmarksopen=true}
\setcounter{secnumdepth}{0} % rimuove i numeri di sezione senza rimuovere le ref
\renewcommand{\href}[2]{\textcolor{blue}{#2}} % disabilita il comando href
\use{enotez} %
\setenotez{%
 mark-format = \footnotemarkformatting % Mette i numeri tra parentesi quadre%
}\let\footnote=\endnote % rende tutte le note a pié pagina come delle note a fine file 


\let\olddocument\document % modifico l'ambiende documenti per non dover stampare \printendnote
\let\oldenddocument\enddocument
\renewenvironment{document}%
{%
  \olddocument
}{%
  \printendnotes\oldenddocument
}
\renewcommand{\thethm}{\arabic{thm}}

\usepackage[hyperref]{biblatex}
\addbibresource{~/Documents/org/roam/bib/master.bib}
\author{Davide Peccioli}
\date{\today}
\title{}
\begin{document}

\section{Omologia per attaccamento di k-cella}
\label{sec:org1a37850}
Sia \(R\) un \href{20241219112842-pid.org}{PID}.

Sia \(X\) uno \href{20250103145124-topologia.org}{spazio topologico} di \href{20250109155715-spazio_topologico_di_hausdorff.org}{Haussrdof}, \(e^{k}\) una \(k\)-cella, \(\varphi: \partial e^{k}\to X\) \href{20250103103252-funzione_continua.org}{continua} e
\begin{equation*}
Y \coloneqq X \mathrel{\cup_{\varphi}} e^k
\end{equation*}
l'\href{20250215122759-export.org_archive}{attaccamento di \(e^{k}\) ad \(X\)}.

\begin{prop}
La mappa \(\Phi: (e^{k}, \partial e^{k})\longrightarrow \big(Y, j(X)\big)\) \href{20250126191208-funtore_da_topp_a_rmod_di_omologia.org}{induce} un \href{20241206115416-morfismi_r_moduli.org}{isomorfismo} in \href{20250122154903-omologia_singolare_relativa.org}{omologia relativa}:
\begin{equation*}
\forall q:\qquad H_{q}(e^{k},\partial e^{k}) \cong H_{q}\big(Y,j(X)\big).
\end{equation*}
\end{prop}
\begin{proof}
Sia \(C\) un \href{20250215122759-export.org_archive}{collare} di \(\partial e^{k}\).
\begin{enumerate}
\item Abbiamo il seguente \href{20250111142332-omeomorfismo.org}{omeomorfismo}:
\begin{equation*}
\restriction{\Phi}{e^{k}\setminus \partial e^{k}}:(e^{k}\setminus \partial e^{k}, C\setminus \partial e^{k})\longrightarrow \big(
 Y \setminus j(X),\ (j(X) \cup \Phi(C))\setminus j(X)
\big)
\end{equation*}
che \href{20250126191208-funtore_da_topp_a_rmod_di_omologia.org}{induce} un \href{20241206115416-morfismi_r_moduli.org}{isomorfismo}
\begin{equation*}
\forall  n:\qquad H_{n}(e^{k}\setminus \partial e^{k}, C\setminus \partial e^{k})\cong H_{n}\big(
 Y \setminus j(X),\ (j(X) \cup \Phi(C))\setminus j(X)
\big)
\end{equation*}

\item Poiché \(\partial e^{k} \subseteq C\) e \(j(X) \subseteq j(X) \cup \Phi(C)\) sono \href{20250122155727-retratto_di_deformazione_di_uno_spazio_topologico.org}{retratti di deformazione}, \href{20250126190440-equivalenze_omotopiche_tra_coppie_topologiche_induce_isomorfismo_tra_omologia_singolare_relativa.org}{allora}
\begin{align*}
 \forall n:\qquad H_{n}\big(Y,j(X)\big) & \cong \big(Y, j(X) \cup \Phi(C)\big)\\
 \forall n:\qquad H_{n}(e^{k},\partial e^{k}) & \cong H_{n}(e^{k}, C)
\end{align*}
\item Per il \href{20250126223310-teorema_di_escissione.org}{Teorema di Escissione}:
\begin{align*}
  H_{n}(e^{k}\setminus \partial e^{k}, C\setminus \partial e^{k}) & \cong H_{n}(e^{k}, C)\\
  H_{n}\big(
         Y \setminus j(X),\ (j(X) \cup \Phi(C))\setminus j(X)
  \big) & \cong H_{n}\big(
         Y,\ (j(X) \cup \Phi(C))
  \big)
\end{align*}
\end{enumerate}
Tutto questo dà l'isomorfismo richiesto:
\begin{equation*}
\begin{tikzcd}[ampersand replacement=\&]
	{H_{n}(e^{k},\partial e^{k})} \&\& {H_{n}(e^{k}, C)} \&\& {H_{n}(e^{k}\setminus \partial e^{k}, C\setminus \partial e^{k})} \\
	\\
	{H_{n}\big(Y,j(X)\big)} \&\& {H_n\big(Y, j(X) \cup \Phi(C)\big)} \&\& {H_{n}\big(Y \setminus j(X),\ (j(X) \cup \Phi(C))\setminus j(X)\big)}
	\arrow["{{\cong }}", from=1-1, to=1-3]
	\arrow["{{2.}}"', draw=none, from=1-1, to=1-3]
	\arrow["\cong"', from=1-5, to=1-3]
	\arrow["{{3.}}", draw=none, from=1-5, to=1-3]
	\arrow["\cong", from=1-5, to=3-5]
	\arrow["{{1.}}"', draw=none, from=1-5, to=3-5]
	\arrow["\cong"', from=3-1, to=3-3]
	\arrow["{{2.}}", draw=none, from=3-1, to=3-3]
	\arrow["\cong", from=3-5, to=3-3]
	\arrow["{{3.}}"', draw=none, from=3-5, to=3-3]
\end{tikzcd}
\qedhere
\end{equation*}
\end{proof}

\begin{prop}
Per l'\href{20250122133631-omologia_singolare.org}{omologia singolare}:
\begin{enumerate}
\item Per ogni \(q \neq k,k-1\):
\begin{equation*}
 H_{q}(Y) \cong H_{q}(X);
\end{equation*}
\item Vale una e una sola delle seguenti:
\begin{itemize}
\item si ha un \href{20241206115416-morfismi_r_moduli.org}{isomorfismo} con la \href{20241213095808-somma_diretta.org}{somma diretta}
\begin{equation*}
   H_{k}(Y) \cong H_{k}(X) \oplus R
\end{equation*}
e, se \(H_{k-1}(X)\) è \href{20241213100845-modulo_finitamente_generato.org}{finitamente generato}, allora lo è anche \(H_{k-1}(Y)\) e vale la seguente formula per il \href{20260109173733-rango_di_un_modulo.org}{rango}:
\begin{equation*}
   \operatorname{rk}\big(H_{k-1}(Y)\big) = \operatorname{rk}\big(H_{k-1}(X)\big);
\end{equation*}
\item si ha un \href{20241206115416-morfismi_r_moduli.org}{isomorfismo}:
\begin{equation*}
   H_{k}(Y) \cong H_{k}(X)
\end{equation*}
e, se \(H_{k-1}(X)\) è \href{20241213100845-modulo_finitamente_generato.org}{finitamente generato}, allora lo è anche \(H_{k-1}(Y)\) e vale la seguente formula per il \href{20260109173733-rango_di_un_modulo.org}{rango}:
\begin{equation*}
   \operatorname{rk}\big(H_{k-1}(Y)\big) = \operatorname{rk}\big(H_{k-1}(X)\big) - 1.
\end{equation*}
\end{itemize}
\end{enumerate}
\end{prop}
\begin{proof}
Si noti che per la proposizione precedente:
\begin{equation*}
     H_{q}\big(Y,j(X)\big) \cong H_{q}(e^{k}, \partial e^{k}) = \begin{cases}
             R & q = k\\
             0 & q \neq k
     \end{cases}
\end{equation*}
per l'\href{20250127162702-calcolo_dell_omologia_singolare_della_sfera_e_dell_omologia_singolare_relativa_del_disco_rispetto_alla_sfera.org}{omologia relativa già calcolata}.

\begin{enumerate}
\item Si scriva la \href{20250122154927-successione_esatta_di_una_coppia_topologica.org}{SEL per l'omologia relativa}:
\begin{equation*}
\begin{tikzcd}[ampersand replacement=\&]
        {H_{q+1}\big(Y,j(X)\big)} \& {H_{q}\big(j(X)\big)} \& {H_{q}(Y)} \& {H_{q}\big(Y,j(X)\big)} \\
        \& {H_q(X)}
        \arrow[from=1-1, to=1-2]
        \arrow[from=1-2, to=1-3]
        \arrow["\cong"{marking, allow upside down}, draw=none, from=1-2, to=2-2]
        \arrow[from=1-3, to=1-4]
\end{tikzcd}
\end{equation*}
Per \(q \neq k\), \(q\neq k-1\), allora \(q \neq k\) e \(q+1 \neq k\), e quindi
\begin{equation*}
 H_{q+1}\big(Y,j(X)\big) = 0,\qquad H_{q}\big(Y,j(X)\big) = 0.
\end{equation*}
\href{20250120130155-caratterizzazione_di_alcune_successioni_esatte_di_r_moduli.org}{Pertanto}, \(H_{q}(X) \cong H_{q}(j(X)) \cong H_{q}(Y)\)
\item Si consideri sempre la SEL per l'omologia relativa, vicino a \(q=k\).
\begin{equation*}
\scalebox{0.8}{%
\begin{tikzcd}[ampersand replacement=\&,column sep=small]
        {H_{k+1}\big(Y,j(X)\big)} \& {H_{k}\big(j(X)\big)} \& {H_{k}(Y)} \& {H_{k}\big(Y,j(X)\big)} \& {H_{k-1}\big(j(X)\big)} \& {H_{k-1}(Y)} \& {H_{k-1}\big(Y,j(X)\big)} \\
        \& {H_k(X)} \&\&\& {H_{k-1}(X)}
        \arrow[from=1-1, to=1-2]
        \arrow[from=1-2, to=1-3]
        \arrow["\cong"{marking, allow upside down}, draw=none, from=1-2, to=2-2]
        \arrow[from=1-3, to=1-4]
        \arrow[from=1-4, to=1-5]
        \arrow[from=1-5, to=1-6]
        \arrow["\cong"{marking, allow upside down}, draw=none, from=1-5, to=2-5]
        \arrow[from=1-6, to=1-7]
\end{tikzcd} %
}
\end{equation*}
e si facciano le dovute sostituzioni
\begin{equation*}
\begin{tikzcd}[ampersand replacement=\&]
        0 \& {H_k(X)} \& {H_{k}(Y)} \& R \& {H_{k-1}(X)} \& {H_{k-1}(Y)} \& 0
        \arrow[from=1-1, to=1-2]
        \arrow[from=1-2, to=1-3]
        \arrow[from=1-3, to=1-4]
        \arrow["\alpha", from=1-4, to=1-5]
        \arrow[from=1-5, to=1-6]
        \arrow[from=1-6, to=1-7]
\end{tikzcd} %
\end{equation*}
Siccome \(R\) è un \(R\)-modulo libero, ed \(R\) è un PID, \href{20241219192830-pid_sottomoduli_sono_liberi.org}{allora ogni suo sottomodulo è libero}. In particolare lo è il \href{20241213105201-kernel.org}{kernel} \(\ker\alpha\).
Ci sono solo due casi possibili:

\begin{enumerate}
\item \(\boxed{\ker \alpha \cong R}\): allora sicuramente la seguente è esatta:
\begin{equation*}
\begin{tikzcd}[ampersand replacement=\&]
        0 \& {H_k(X)} \& {H_{k}(Y)} \& {\ker\alpha} \& 0
        \arrow[from=1-1, to=1-2]
        \arrow[from=1-2, to=1-3]
        \arrow[from=1-3, to=1-4]
        \arrow[from=1-4, to=1-5]
\end{tikzcd}
\end{equation*}
e siccome \(\ker\alpha \cong R\) è \href{20241213094625-modulo_libero.org}{libero}, \href{20250120131729-teorema_di_spezzamento_sec.org}{si ha che}
\begin{equation*}
  H_{k}(Y) \cong H_{k}(X) \oplus R.
\end{equation*}

Per la seconda parte, per il \href{20250120155457-morfismo_iniettivo_di_r_moduli_induce_isomorfismo.org}{primo teorema di isomorfismo} questa è una \href{20250120125004-successione_di_r_moduli_esatta.org}{successione esatta corta}:
\begin{equation*}
\begin{tikzcd}[ampersand replacement=\&]
        0 \& {R/\ker\alpha} \&\& {H_{k-1}(X)} \&\& {H_{k-1}(Y)} \& 0
        \arrow[from=1-1, to=1-2]
        \arrow["{\overline{\alpha}}"', from=1-2, to=1-4]
        \arrow[from=1-4, to=1-6]
        \arrow[from=1-6, to=1-7]
\end{tikzcd}
\end{equation*}
\begin{itemize}
\item \href{20250120121333-quoziente_di_modulo_fg_e_fg.org}{\(R/\ker \alpha\) è finitamente generato}. Inoltre è \href{20250120103129-modulo_di_torsione.org}{di torsione}.

Infatti, siccome \(\ker\alpha \subseteq R\) è sottomodulo, allora è ideale di un PID, e pertanto esiste \(r \in R\), \(r \neq 0\) tale che\footnote{Vedi l'\href{20241219113154-ideale_generato.org}{ideale generato da un elemento}}
\begin{equation*}
\ker\alpha = (r)
\end{equation*}
Allora per ogni \(p + \ker \alpha \in R/\alpha\), \(r\cdot [p] = 0\).

Pertanto ha \href{20260109173733-rango_di_un_modulo.org}{rango nullo}.

\item Se \(H_{k-1}(X)\) è finitamente generato, allora lo è anche \(H_{k-1}(Y)\).\footnote{Questo segue dal fatto che \(H_{k-1}(Y)\) è l'immagine di \(H_{k-1}(X)\) tramite morfismo.}

Pertanto \href{20260206182305-rango_in_una_successione_di_r_moduli_finitamente_generati.org}{vale la formula}
\begin{equation*}
\operatorname{rk}H_{k-1}(X) = \operatorname{rk}H_{k-1}(Y) + \parentesi{=0}{\operatorname{rk}(R/\ker\alpha)}.
\end{equation*}
\end{itemize}
\end{enumerate}
\end{enumerate}


\begin{enumerate}
\item \(\boxed{\ker \alpha = 0}\): allora la seguente è esatta:
\begin{equation*}
\scalebox{0.93}{%
\begin{tikzcd}[ampersand replacement=\&]
        0 \& {H_k(X)} \& {H_{k}(Y)} \&\& R \& {H_{k-1}(X)} \& {H_{k-1}(Y)} \& 0 \\
        \&\&\& {\ker\alpha}
        \arrow[from=1-1, to=1-2]
        \arrow[from=1-2, to=1-3]
        \arrow["0"', from=1-3, to=2-4]
        \arrow["\alpha", from=1-5, to=1-6]
        \arrow[from=1-6, to=1-7]
        \arrow[from=1-7, to=1-8]
        \arrow[hook, from=2-4, to=1-5]
\end{tikzcd}%
}
\end{equation*}
e pertanto lo sono entrambe le seguenti:
\begin{equation*}
\begin{tikzcd}[ampersand replacement=\&]
        \& 0 \& {H_k(X)} \& {H_{k}(Y)} \& 0 \\
        0 \& R \& {H_{k-1}(X)} \& {H_{k-1}(Y)} \& 0
        \arrow[from=1-2, to=1-3]
        \arrow[from=1-3, to=1-4]
        \arrow[from=1-4, to=1-5]
        \arrow[from=2-1, to=2-2]
        \arrow["\alpha", from=2-2, to=2-3]
        \arrow[from=2-3, to=2-4]
        \arrow[from=2-4, to=2-5]
\end{tikzcd}
\end{equation*}
\href{20250120130155-caratterizzazione_di_alcune_successioni_esatte_di_r_moduli.org}{Segue} che \(H_{k}(X) \cong H_{k}(Y)\), e \(H_{k-1}(Y) \cong H_{k-1}(X) / R\).

Quindi, se \(H_{k-1}(X)\) è finitamente generato, \href{20250120121333-quoziente_di_modulo_fg_e_fg.org}{allora} \(H_{k-1}(Y)\) è finitamente generato, e inoltre per il \href{20260109173733-rango_di_un_modulo.org}{rango} \href{20260206182305-rango_in_una_successione_di_r_moduli_finitamente_generati.org}{vale}:
\begin{equation*}
  \operatorname{rk} H_{k-1}(X) = \operatorname{rk} H_{k-1}(Y) + \parentesi{=1}{\operatorname{rk} R}.
\end{equation*}
\end{enumerate}
\end{proof}
\end{document}
