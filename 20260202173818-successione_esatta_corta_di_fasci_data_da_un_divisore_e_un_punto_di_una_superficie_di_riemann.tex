% Created 2026-02-07 Sat 19:33
% Intended LaTeX compiler: pdflatex
\documentclass[10pt]{article}
%% CREATO CON ORG - EMACS
\newcommand{\use}[2][]{\usepackage[#1]{#2}}
% PACCHETTI FONDAMENTLAI
\use[utf8]{inputenc}
\use[T1]{fontenc}
\use{graphicx}
\use{longtable}
\use{wrapfig}
\use{rotating}
\use[normalem]{ulem}
\use{amsmath}
\use{amsthm}
\use{amssymb}

\use{eucal} % Cambia mathcal{...}

\use{capt-of}
\use[italian]{babel}
\use[babel]{csquotes}
% bib la TEX lo carica in automatico org-cite
\use{microtype}
\use{lmodern}
\use{subfig} % sottofigure
\use{multicol} % due colonne
\use{lipsum} % lorem ipsum
\use{color} % colori in latex
\use{parskip} % rimuove l'indentazione dei nuovi paragrafi %% Add parbox=false to all new tcolorbox
\use{centernot}
\use[outline]{contour}\contourlength{3pt}
\use{fancyhdr}
\use{layout}
\use[most]{tcolorbox} % Riquadri colorati
\use{ifthen} % IFTHEN
\use{geometry}

% pacchetti matematica
\use{yhmath}
\use{dsfont}
\use{mathrsfs}
\use{cancel} % semplificare
\use{polynom} %divisione tra polinomi
\use{forest} % grafi ad albero
\use{booktabs} % tabelle
\use{commath} %simboli e differenziali
\use{bm} %bold
\use[fulladjust]{marginnote} %to use marginnote for date notes
\use{arrayjobx}%array
\use[intlimits]{empheq} % Riquadri colorati attorno alle equazioni
\use{mathtools}
\use{circuitikz} % Disegnare i circuiti
\use{mathtools}
\use{stmaryrd} % [[ \llbracket ]] \rrbracket
\use{bussproofs} % dimostrazioni

%%%%%%%%%%%%%


%%%% QUIVER
\newcommand{\duepunti}{\,\mathchar\numexpr"6000+`:\relax\,}
% A TikZ style for curved arrows of a fixed height, due to AndréC.
\tikzset{curve/.style={settings={#1},to path={(\tikztostart)
    .. controls ($(\tikztostart)!\pv{pos}!(\tikztotarget)!\pv{height}!270:(\tikztotarget)$)
    and ($(\tikztostart)!1-\pv{pos}!(\tikztotarget)!\pv{height}!270:(\tikztotarget)$)
    .. (\tikztotarget)\tikztonodes}},
    settings/.code={\tikzset{quiver/.cd,#1}
        \def\pv##1{\pgfkeysvalueof{/tikz/quiver/##1}}},
    quiver/.cd,pos/.initial=0.35,height/.initial=0}

% TikZ arrowhead/tail styles.
\tikzset{tail reversed/.code={\pgfsetarrowsstart{tikzcd to}}}
\tikzset{2tail/.code={\pgfsetarrowsstart{Implies[reversed]}}}
\tikzset{2tail reversed/.code={\pgfsetarrowsstart{Implies}}}
% TikZ arrow styles.
\tikzset{no body/.style={/tikz/dash pattern=on 0 off 1mm}}
%%%%%%%%%%


%% DEFINIZIONI COMANDI MATEMATICI
\let\sin\relax %TOGLIE LA DEFINIZIONE SU "\sin"

% cambia la definizione di empty set
% ---
\let\oldemptyset\emptyset
% ---
% \let\emptyset\varnothing
% ---
% \let\emptyset\relax
% \newcommand{\emptyset}{\text{\textnormal{\O}}}
% ---

\DeclareMathOperator{\bounded}{bd}
\DeclareMathOperator{\sin}{sen}
\DeclareMathOperator{\epi}{Epi}
\DeclareMathOperator{\cl}{cl}
\DeclareMathOperator{\graph}{graph}
\DeclareMathOperator{\arcsec}{arcsec}
\DeclareMathOperator{\arccot}{arccot}
\DeclareMathOperator{\arccsc}{arccsc}
\DeclareMathOperator{\spettro}{Spettro}
\DeclareMathOperator{\nulls}{nullspace}
\DeclareMathOperator{\dom}{dom}
\DeclareMathOperator{\ar}{ar}
\DeclareMathOperator{\const}{Const}
\DeclareMathOperator{\fun}{Fun}
\DeclareMathOperator{\rel}{Rel}
\DeclareMathOperator{\altezza}{ht}
\let\det\relax %TOGLIE LA DEFINIZIONE SU "\det"
\DeclareMathOperator{\det}{det}
\DeclareMathOperator{\End}{End}
\DeclareMathOperator{\gl}{GL}
\def\Id{\mathrm{Id}}
\def\id{\mathrm{id}}
\DeclareMathOperator{\I}{\mathds{1}}
\DeclareMathOperator{\II}{II}
\DeclareMathOperator{\rank}{rank}
\DeclareMathOperator{\tr}{tr}
\DeclareMathOperator{\tc}{t.c.}
\DeclareMathOperator{\T}{T}
\DeclareMathOperator{\var}{Var}
\DeclareMathOperator{\cov}{Cov}
\DeclareMathOperator{\st}{st}
\DeclareMathOperator{\mon}{Mon}
\newcommand{\card}[1]{\left\vert #1 \right\vert}
\newcommand{\trasposta}[1]{\prescript{\text{T}}{}{#1}}
\newcommand{\1}{\mathds{1}}
\newcommand{\R}{\mathds{R}}
\newcommand{\diesis}{\#}
\newcommand{\bemolle}{\flat}
\newcommand{\nonstandard}[1]{\prescript{*}{}{#1}}
\newcommand{\starR}{\nonstandard{\R}}
\newcommand{\borel}{\mathscr{B}}
\newcommand{\lebesgue}[1]{\mathscr{L}\left(#1\right)}
\newcommand{\media}{\mathds{E}}
\newcommand{\K}{\mathds{K}}
\newcommand{\A}{\mathds{A}}
\newcommand{\Q}{\mathds{Q}}
\newcommand{\N}{\mathds{N}}
\newcommand{\C}{\mathds{C}}
\newcommand{\Z}{\mathds{Z}}
\newcommand{\qo}{\hspace{1em}\text{q.o.}\,}
\renewcommand{\tilde}[1]{\widetilde{#1}}
\renewcommand{\parallel}{\mathrel{/\mkern-5mu/}}
\newcommand{\parti}[2][]{\wp_{#1}(#2)}
\newcommand{\diff}[1]{\operatorname{d}_{#1}}
\let\oldvec\vec
\renewcommand{\vec}[1]{\overrightarrow{\vphantom{i}#1}}
\newcommand{\floor}[1]{\left\lfloor #1 \right\rfloor}
\newcommand{\cat}[1]{\mathbf{#1}}
\newcommand{\dfreccia}[1]{\xrightarrow{\ #1 \ }}
\newcommand{\sfreccia}[1]{\xleftarrow{\ #1 \ }}
\newcommand{\formalsum}[2]{{\sum_{#1}^{#2}}{\vphantom{\sum}}'}
\newcommand{\minim}[2]{\mu_{#1}\, \left(#2\right)}
\newcommand{\concat}{\null^{\frown}} % concatenazione di stringe
\newcommand{\godelcode}[1]{\langle\!\langle #1 \rangle\!\rangle}
\newcommand{\godeldec}[1]{(\!(#1)\!)}
\newcommand{\termcode}[1]{\ulcorner #1\urcorner}
\newcommand{\partialto}{\dashrightarrow}
\newcommand{\restricted}{\upharpoonright}
\newcommand{\embeds}{\precsim}
\newcommand{\surjects}{\twoheadrightarrow}
\newcommand{\equipotenti}{\asymp}
%% \newcommand{\dotplus}{\mathbin{\dot{+}}} %% A quanto pare esiste già
\newcommand{\bigdot}{\mathbin{\boldsymbol{\cdot}}}
\newcommand{\dotexp}[1]{^{.#1}}
\newcommand{\conv}{\mathbin{*}}
\newcommand{\convolution}[2]{(#1\conv #2)}
\newcommand{\nil}{\mathfrak{N}}
\newcommand{\divisore}{\mathrel{|}}
\newcommand{\simplesso}[1]{\mathrm{e}_{#1}}

\renewcommand{\iff}{\mathrel{\longleftrightarrow}} %% Notazione Logica.
\newcommand{\oldiff}{\mathrel{\Longleftrightarrow}}
\renewcommand{\implies}{\mathrel{\rightarrow}} %% Notazione Logica
\newcommand{\oldimplies}{\mathrel{\Longrightarrow}}
\renewcommand{\impliedby}{\mathrel{\leftarrow}} %% Notazione Logica
\newcommand{\oldimpliedby}{\mathrel{\Longleftarrow}}

\newcommand{\IFF}{\quad\Longleftrightarrow\quad}
\newcommand{\IMPLICA}{\quad\Longrightarrow\quad}


\renewcommand{\descriptionlabel}[1]{\hspace{\labelsep}\normalfont #1} % remove bold from description


%% Definizione di Divergenza di K-L

\DeclarePairedDelimiterX{\infdivx}[2]{(}{)}{%
  #1\;\delimsize\|\;#2%
}
\newcommand{\kldiv}{D_{KL}\infdivx}

%% Definizione di \dotminus

\makeatletter
\newcommand{\dotminus}{\mathbin{\text{\@dotminus}}}

\newcommand{\@dotminus}{%
  \ooalign{\hidewidth\raise1ex\hbox{.}\hidewidth\cr$\m@th-$\cr}%
}
\makeatother

%tramite i prossimi due comandi posso decidere come scrivere i logaritmi naturali in tutti i documenti: ho infatti eliminato qualsiasi differenza tra "ln" e "log": se si vuole qualcosa di diverso bisogna inserire manualmente il tutto
\let\ln\relax
\DeclareMathOperator{\ln}{ln}
\let\log\relax
\DeclareMathOperator{\log}{log}
%%%%%%

%% NUOVI COMANDI
\newcommand{\straniero}[1]{\textit{#1}} %parole straniere
\newcommand{\titolo}[1]{\textsc{#1}} %titoli
\newcommand{\qedd}{\tag*{$\blacksquare$}} %qed per ambienti matemastici
\renewcommand{\qedsymbol}{$\blacksquare$} %modifica colore qed
\newcommand{\ooverline}[1]{\overline{\overline{#1}}}
\newcommand{\circoletto}[1]{\left(#1\right)^{\text{o}}}
%
\newcommand{\qmatrice}[1]{\begin{pmatrix}
#1_{11} & \cdots & #1_{1n}\\
\vdots & \ddots & \vdots \\
#1_{m1} & \cdots & #1_{mn}
\end{pmatrix}}
%
\newcommand{\parentesi}[2]{%
\underset{#1}{\underbrace{#2}}%
}
%
\newcommand{\norma}[1]{% Norma
\left\lVert#1\right\rVert%
}
\newcommand{\scalare}[2]{% Scalare
\left\langle #1, #2\right\rangle
}
%%%%%

%% RESTRIZIONI
\newcommand{\referenze}[2]{
        \phantomsection{}#2\textsuperscript{\textcolor{blue}{\textbf{#1}}}
}

\let\restriction\relax

\def\restriction#1#2{\mathchoice
              {\setbox1\hbox{${\displaystyle #1}_{\scriptstyle #2}$}
              \restrictionaux{#1}{#2}}
              {\setbox1\hbox{${\textstyle #1}_{\scriptstyle #2}$}
              \restrictionaux{#1}{#2}}
              {\setbox1\hbox{${\scriptstyle #1}_{\scriptscriptstyle #2}$}
              \restrictionaux{#1}{#2}}
              {\setbox1\hbox{${\scriptscriptstyle #1}_{\scriptscriptstyle #2}$}
              \restrictionaux{#1}{#2}}}
\def\restrictionaux#1#2{{#1\,\smash{\vrule height .8\ht1 depth .85\dp1}}_{\,#2}}
%%%%%%%%%%%

%%% FORMATTAZIONE FOOTNOTEMARK

\def\footnotemarkformatting#1{[#1]}
\renewcommand{\thefootnote}{\footnotemarkformatting{\arabic{footnote}}}

%% SEZIONE GRAFICA
\use{tikz}
\usetikzlibrary{matrix, patterns, calc, decorations.pathreplacing, hobby, decorations.markings, decorations.pathmorphing, babel}
\use{tikz-3dplot}
\use{mathrsfs} %per geogebra
\use{tikz-cd}
\tikzset
{
  %surface/.style={fill=black!10, shading=ball,fill opacity=0.4},
  plane/.style={black,pattern=north east lines},
  curve/.style={black,line width=0.5mm},
  dritto/.style={decoration={markings,mark=at position 0.5 with {\arrow{Stealth}}}, postaction=decorate},
  rovescio/.style={decoration={markings,mark=at position 0.5 with {\arrow{Stealth[reversed]}}}, postaction=decorate}
}
\use{pgfplots} % stampare le funzioni
        \pgfplotsset{/pgf/number format/use comma,compat=1.15}
        %\pgfplotsset{compat=1.15} %per geogebra
        \usepgfplotslibrary{fillbetween, polar}
%%%%%%

%% CITAZIONI
\use{lineno}

\newcommand{\citazione}[1]{%
  \begin{quotation}
  \begin{linenumbers}
  \modulolinenumbers[5]
  \begingroup
  \setlength{\parindent}{0cm}
  \noindent #1
  \endgroup
  \end{linenumbers}
  \end{quotation}\setcounter{linenumber}{1}
  }
%%%%%%

%%%%%%%%%%%%%%%%%%%%%%%%%%%%%%%%%%%%%%%%%%%%
%%%%%%%%%%%%%%%%%%%%%%%%%%%%%%%%%%%%%%%%%%%%

%% AMS THM

\theoremstyle{definition}% default
\newtheorem{thm}{Teorema}[section]
\newtheorem{lem}[thm]{Lemma}
\newtheorem{prop}[thm]{Proposizione}
\newtheorem{cor}[thm]{Corollario}
\newtheorem{esempio}[thm]{Esempio}
\theoremstyle{plain}
\newtheorem{definizione}[thm]{Definizione}
\theoremstyle{remark}
\newtheorem*{oss}{Osservazione}


%%%%%%%%%%%%%%%%%%%%%%%%%%%%%%%%%%%%%%%%%%%%
%%%%%%%%%%%%%%%%%%%%%%%%%%%%%%%%%%%%%%%%%%%%

\use{hyperref}
\hypersetup{%
        pdfauthor={Davide Peccioli},
        pdfsubject={},
        allcolors=black,
        citecolor=black,
%	colorlinks=true,
        bookmarksopen=true}
\setcounter{secnumdepth}{0} % rimuove i numeri di sezione senza rimuovere le ref
\renewcommand{\href}[2]{\textcolor{blue}{#2}} % disabilita il comando href
\use{enotez} %
\setenotez{%
 mark-format = \footnotemarkformatting % Mette i numeri tra parentesi quadre%
}\let\footnote=\endnote % rende tutte le note a pié pagina come delle note a fine file 


\let\olddocument\document % modifico l'ambiende documenti per non dover stampare \printendnote
\let\oldenddocument\enddocument
\renewenvironment{document}%
{%
  \olddocument
}{%
  \printendnotes\oldenddocument
}
\renewcommand{\thethm}{\arabic{thm}}

\usepackage[hyperref]{biblatex}
\addbibresource{~/Documents/org/roam/bib/master.bib}
\author{Davide Peccioli}
\date{\today}
\title{}
\begin{document}

\section{Successione esatta corta di fasci data da un divisore e un punto di una superficie di Riemann}
\label{sec:org1828408}
Sia \(X\) una \href{20260127112828-superficie_di_riemann.org}{superficie di Riemann}, e sia \(\operatorname{Div}(X)\) il \href{20260201235551-divisore_di_una_superficie_di_riemann.org}{gruppo dei divisori di \(X\)}. Siano:
\begin{itemize}
\item per ogni \(D \in \operatorname{Div}(X)\), \(\mathcal{O}_{X}(D)\) il \href{20260202101128-fascio_associato_ad_un_divisore_su_una_superficie_di_riemann.org}{fascio associato a \(D\)};
\item \(\mathcal{O}_{X}\) il \href{20260128143847-fascio_delle_funzioni_olomorfe_su_una_superficie_di_riemann.org}{fascio delle funzioni olomorfe};
\item per ogni \(p \in X\), \(\C_{p}\) il \href{20250325171249-fascio_grattacielo.org}{fascio grattacielo centrato in \(p\)}.
\end{itemize}

\begin{oss}
Per \(p \in X\):
\begin{itemize}
\item \(\mathcal{O}_{X}(-p)\) è il fascio delle funzioni olomorfe nulle in \(p\), ed è pertanto sottofascio di \(\mathcal{O}_{X}\):
\begin{equation*}
\mathcal{O}_{X}(-p) \hookrightarrow \mathcal{O}_{X}
\end{equation*}
\item Inoltre, si ha la mappa di valutazione in \(p\), morfismo di fasci \(\operatorname{ev}_{p}: \mathcal{O}_{X}\to \C_{p}\): se \(p \in U\):
\begin{align*}
\operatorname{ev}_{p,U}: \mathcal{O}_{X}(U) &\longrightarrow \C_{p}(U)\\
f &\longmapsto f(p)
\end{align*}
se \(p\notin U\) allora \(\operatorname{ev}_{p,U}=0\).

È possibile dimostrare che il \href{20250327114922-fascio_nucleo.org}{fascio nucleo} \(\ker \operatorname{ev}_{p} = \mathcal{O}_{X}(-p)\) e che \(\operatorname{ev}_{p}\) è \href{20250325180613-morfismo_di_prefasci.org}{morfismo di fasci} \href{20250327115214-morfismo_di_fasci_suriettivo.org}{suriettivo}.
\end{itemize}
Si ottiene quindi la \href{20250327150404-successione_di_fasci_esatta.org}{SEC} di \href{20250324174728-fascio.org}{fasci} di \href{20241205142027-spazio_vettoriale.org}{spazi vettoriali}:
\begin{equation*}
\begin{tikzcd}[ampersand replacement=\&]
	0 \& {\mathcal{O}_X(-p)} \& {\mathcal{O}_X} \& {\C_p} \& 0
	\arrow[from=1-1, to=1-2]
	\arrow[hook, from=1-2, to=1-3]
	\arrow["{\operatorname{ev}_p}", from=1-3, to=1-4]
	\arrow[from=1-4, to=1-5]
\end{tikzcd}
\end{equation*}
\end{oss}

\begin{prop}
Per ogni \(D \in \operatorname{Div}(X)\) e \(p \in X\) esistono \(i\) e \(\varphi\) tali che
\begin{equation*}
\begin{tikzcd}[ampersand replacement=\&]
	0 \& {\mathcal{O}_X(D-p)} \& {\mathcal{O}_X(D)} \& {\C_p} \& 0
	\arrow[from=1-1, to=1-2]
	\arrow["i", from=1-2, to=1-3]
	\arrow["\varphi", from=1-3, to=1-4]
	\arrow[from=1-4, to=1-5]
\end{tikzcd}
\end{equation*}
sia una \href{20250327150404-successione_di_fasci_esatta.org}{SEC} di \href{20250324174728-fascio.org}{fasci} di \href{20241205142027-spazio_vettoriale.org}{spazi vettoriali}.
\end{prop}
\begin{proof}
Siccome \(D-p \le D\), \href{20260202101128-fascio_associato_ad_un_divisore_su_una_superficie_di_riemann.org}{allora} \(\mathcal{O}_{X}(D-p) \subseteq \mathcal{O}_{X}(D)\) è \href{20250325150647-sottoprefascio.org}{sottofascio}, e pertanto è possibile prendere come \(i\) l'inclusione.

È necessario quindi costruire \(\mathcal{O}_{X}(D)\xrightarrow{\varphi} \C_{p}\) \href{20250327115214-morfismo_di_fasci_suriettivo.org}{suriettiva} il cui \href{20250327114922-fascio_nucleo.org}{nucleo} sia \(\mathcal{O}_{X}(D-p)\). Si fissi \(U_{p} \subseteq X\) \href{20250111142313-intorno.org}{intorno} \href{20250103145124-topologia.org}{aperto} e \href{20250103165325-spazio_topologico_connesso.org}{connesso} di \(p\), e \(z_{p}:U_{p}\to \C\) \href{20260127112715-atlante_complesso.org}{carta locale} tale che \(z_{p}(p) = 0\).

Si definisce \(\varphi\). Sia ora \(U \subseteq X\) aperto.
\begin{itemize}
\item Se \(p\notin U\), allora si pone \(\varphi_{U} \coloneqq 0\).
\item Se \(p \in U\), sia \(f \in \mathcal{O}_{X}(D)(U)\). Allora l'\href{20260128144450-ordine_di_una_funzione_meromorfa_su_una_superficie_di_riemann.org}{ordine}
\begin{equation*}
  \mathrm{ord}_{p} f \ge - D (p)
\end{equation*}
\begin{itemize}
\item se \(f\) è nulla in un intorno di \(p\), si pone \(\varphi_{U}(f) = 0\);
\item se \(f\) è non nulla in un intorno di \(p\), allora \(f\) nella carta locale ha uno sviluppo in \href{20260128163831-serie_di_laurent.org}{serie di Laurent}:
\begin{equation*}
f = \sum_{n\ge -D(p)} a_{n}\,z_{p}^{n}
\end{equation*}
Si pone \(\varphi_{U}(f) \coloneqq a_{-D(p)} \in \C\). Si noti che
\begin{align*}
a_{-D(p)} \neq 0 & \IFF \mathrm{ord}_{p} f = -D(p)\\
a_{-D(p)} = 0 & \IFF \mathrm{ord}_{p} f > -D(p)\\
&\IFF \mathrm{ord}_{p} f \ge -D(p) + 1\\
&\IFF f \in \mathcal{O}_{X}(D-p) \subseteq \mathcal{O}_{X}(D).
\end{align*}
\end{itemize}
\end{itemize}

Si è dimostrato anche che \(\ker \varphi = \mathcal{O}_{X}(D-p)\).

Resta da dimostrare che \(\varphi\) sia un \href{20250327115214-morfismo_di_fasci_suriettivo.org}{morfismo di fasci suriettivo}. Sia \(U \subseteq X\) aperto.
\begin{itemize}
\item Se \(p \notin U\), allora \(\varphi_{U}(U) = 0 = \C_{p}(U)\), e rispetta la suriettività.
\item Se \(p \in U\), sia \(\lambda \in \C = \C_{p}(U)\). Vogliamo dimostrare che per ogni \(q \in U\) esiste \(W \subseteq U\) intorno aperto di \(q\), ed esiste \(F \in \mathcal{O}_{X}(D)(W)\) tale che \(\restriction{\lambda}{W} = \varphi_{W} (F)\).

Per ogni \(q \neq p\) questo è ovvio: è sufficiente prendere un intorno \(W\) di \(q\) che non contenga \(p\), ottenendo che \(0 \in \mathcal{O}_{X}(D)(W)\) e \(\varphi_{W}(0) = 0 = \restriction{\lambda}{W}\).

Sia quindi \(U_{p}\) l'intorno di \(p\) di cui sopra: \(z_{p}:U_{p}\to \C\) carta locale tale che \(z_{p}(p) = 0\). A meno di restringere \(U_{p}\), è possibile supporre che
\begin{equation*}
  U_{p}\cap \operatorname{supp}D \subsete\set{p}
\end{equation*}
in quanto il supporto è insieme discreto in uno spazio di Haussdorf. Allora \(\restriction{D}{U_{p}} = D(p)\cdot p\).

Sia \(g \in \mathcal{M}_{X}(U_{p})\) data da \(g \coloneqq \lambda z_{p}^{-D(p)}\).
\begin{itemize}
\item \(g\) è olomorfa su \(U_{p}\setminus\set{p}\);

\item \(\mathrm{ord}_{p}\,g = - D(p)\)
\end{itemize}
e pertanto \(\operatorname{div}g + \restriction{D}{U_{p}}\ge 0\): \(g \in \mathcal{O}_{X}(D)(U_{p})\) e \(\varphi_{U_{p}}(g) = \lambda\).
\qedhere
\end{itemize}
\end{proof}
\end{document}
