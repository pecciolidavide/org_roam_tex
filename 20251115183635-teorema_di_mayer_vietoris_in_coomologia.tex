% Created 2026-02-07 Sat 19:32
% Intended LaTeX compiler: pdflatex
\documentclass[10pt]{article}
%% CREATO CON ORG - EMACS
\newcommand{\use}[2][]{\usepackage[#1]{#2}}
% PACCHETTI FONDAMENTLAI
\use[utf8]{inputenc}
\use[T1]{fontenc}
\use{graphicx}
\use{longtable}
\use{wrapfig}
\use{rotating}
\use[normalem]{ulem}
\use{amsmath}
\use{amsthm}
\use{amssymb}

\use{eucal} % Cambia mathcal{...}

\use{capt-of}
\use[italian]{babel}
\use[babel]{csquotes}
% bib la TEX lo carica in automatico org-cite
\use{microtype}
\use{lmodern}
\use{subfig} % sottofigure
\use{multicol} % due colonne
\use{lipsum} % lorem ipsum
\use{color} % colori in latex
\use{parskip} % rimuove l'indentazione dei nuovi paragrafi %% Add parbox=false to all new tcolorbox
\use{centernot}
\use[outline]{contour}\contourlength{3pt}
\use{fancyhdr}
\use{layout}
\use[most]{tcolorbox} % Riquadri colorati
\use{ifthen} % IFTHEN
\use{geometry}

% pacchetti matematica
\use{yhmath}
\use{dsfont}
\use{mathrsfs}
\use{cancel} % semplificare
\use{polynom} %divisione tra polinomi
\use{forest} % grafi ad albero
\use{booktabs} % tabelle
\use{commath} %simboli e differenziali
\use{bm} %bold
\use[fulladjust]{marginnote} %to use marginnote for date notes
\use{arrayjobx}%array
\use[intlimits]{empheq} % Riquadri colorati attorno alle equazioni
\use{mathtools}
\use{circuitikz} % Disegnare i circuiti
\use{mathtools}
\use{stmaryrd} % [[ \llbracket ]] \rrbracket
\use{bussproofs} % dimostrazioni

%%%%%%%%%%%%%


%%%% QUIVER
\newcommand{\duepunti}{\,\mathchar\numexpr"6000+`:\relax\,}
% A TikZ style for curved arrows of a fixed height, due to AndréC.
\tikzset{curve/.style={settings={#1},to path={(\tikztostart)
    .. controls ($(\tikztostart)!\pv{pos}!(\tikztotarget)!\pv{height}!270:(\tikztotarget)$)
    and ($(\tikztostart)!1-\pv{pos}!(\tikztotarget)!\pv{height}!270:(\tikztotarget)$)
    .. (\tikztotarget)\tikztonodes}},
    settings/.code={\tikzset{quiver/.cd,#1}
        \def\pv##1{\pgfkeysvalueof{/tikz/quiver/##1}}},
    quiver/.cd,pos/.initial=0.35,height/.initial=0}

% TikZ arrowhead/tail styles.
\tikzset{tail reversed/.code={\pgfsetarrowsstart{tikzcd to}}}
\tikzset{2tail/.code={\pgfsetarrowsstart{Implies[reversed]}}}
\tikzset{2tail reversed/.code={\pgfsetarrowsstart{Implies}}}
% TikZ arrow styles.
\tikzset{no body/.style={/tikz/dash pattern=on 0 off 1mm}}
%%%%%%%%%%


%% DEFINIZIONI COMANDI MATEMATICI
\let\sin\relax %TOGLIE LA DEFINIZIONE SU "\sin"

% cambia la definizione di empty set
% ---
\let\oldemptyset\emptyset
% ---
% \let\emptyset\varnothing
% ---
% \let\emptyset\relax
% \newcommand{\emptyset}{\text{\textnormal{\O}}}
% ---

\DeclareMathOperator{\bounded}{bd}
\DeclareMathOperator{\sin}{sen}
\DeclareMathOperator{\epi}{Epi}
\DeclareMathOperator{\cl}{cl}
\DeclareMathOperator{\graph}{graph}
\DeclareMathOperator{\arcsec}{arcsec}
\DeclareMathOperator{\arccot}{arccot}
\DeclareMathOperator{\arccsc}{arccsc}
\DeclareMathOperator{\spettro}{Spettro}
\DeclareMathOperator{\nulls}{nullspace}
\DeclareMathOperator{\dom}{dom}
\DeclareMathOperator{\ar}{ar}
\DeclareMathOperator{\const}{Const}
\DeclareMathOperator{\fun}{Fun}
\DeclareMathOperator{\rel}{Rel}
\DeclareMathOperator{\altezza}{ht}
\let\det\relax %TOGLIE LA DEFINIZIONE SU "\det"
\DeclareMathOperator{\det}{det}
\DeclareMathOperator{\End}{End}
\DeclareMathOperator{\gl}{GL}
\def\Id{\mathrm{Id}}
\def\id{\mathrm{id}}
\DeclareMathOperator{\I}{\mathds{1}}
\DeclareMathOperator{\II}{II}
\DeclareMathOperator{\rank}{rank}
\DeclareMathOperator{\tr}{tr}
\DeclareMathOperator{\tc}{t.c.}
\DeclareMathOperator{\T}{T}
\DeclareMathOperator{\var}{Var}
\DeclareMathOperator{\cov}{Cov}
\DeclareMathOperator{\st}{st}
\DeclareMathOperator{\mon}{Mon}
\newcommand{\card}[1]{\left\vert #1 \right\vert}
\newcommand{\trasposta}[1]{\prescript{\text{T}}{}{#1}}
\newcommand{\1}{\mathds{1}}
\newcommand{\R}{\mathds{R}}
\newcommand{\diesis}{\#}
\newcommand{\bemolle}{\flat}
\newcommand{\nonstandard}[1]{\prescript{*}{}{#1}}
\newcommand{\starR}{\nonstandard{\R}}
\newcommand{\borel}{\mathscr{B}}
\newcommand{\lebesgue}[1]{\mathscr{L}\left(#1\right)}
\newcommand{\media}{\mathds{E}}
\newcommand{\K}{\mathds{K}}
\newcommand{\A}{\mathds{A}}
\newcommand{\Q}{\mathds{Q}}
\newcommand{\N}{\mathds{N}}
\newcommand{\C}{\mathds{C}}
\newcommand{\Z}{\mathds{Z}}
\newcommand{\qo}{\hspace{1em}\text{q.o.}\,}
\renewcommand{\tilde}[1]{\widetilde{#1}}
\renewcommand{\parallel}{\mathrel{/\mkern-5mu/}}
\newcommand{\parti}[2][]{\wp_{#1}(#2)}
\newcommand{\diff}[1]{\operatorname{d}_{#1}}
\let\oldvec\vec
\renewcommand{\vec}[1]{\overrightarrow{\vphantom{i}#1}}
\newcommand{\floor}[1]{\left\lfloor #1 \right\rfloor}
\newcommand{\cat}[1]{\mathbf{#1}}
\newcommand{\dfreccia}[1]{\xrightarrow{\ #1 \ }}
\newcommand{\sfreccia}[1]{\xleftarrow{\ #1 \ }}
\newcommand{\formalsum}[2]{{\sum_{#1}^{#2}}{\vphantom{\sum}}'}
\newcommand{\minim}[2]{\mu_{#1}\, \left(#2\right)}
\newcommand{\concat}{\null^{\frown}} % concatenazione di stringe
\newcommand{\godelcode}[1]{\langle\!\langle #1 \rangle\!\rangle}
\newcommand{\godeldec}[1]{(\!(#1)\!)}
\newcommand{\termcode}[1]{\ulcorner #1\urcorner}
\newcommand{\partialto}{\dashrightarrow}
\newcommand{\restricted}{\upharpoonright}
\newcommand{\embeds}{\precsim}
\newcommand{\surjects}{\twoheadrightarrow}
\newcommand{\equipotenti}{\asymp}
%% \newcommand{\dotplus}{\mathbin{\dot{+}}} %% A quanto pare esiste già
\newcommand{\bigdot}{\mathbin{\boldsymbol{\cdot}}}
\newcommand{\dotexp}[1]{^{.#1}}
\newcommand{\conv}{\mathbin{*}}
\newcommand{\convolution}[2]{(#1\conv #2)}
\newcommand{\nil}{\mathfrak{N}}
\newcommand{\divisore}{\mathrel{|}}
\newcommand{\simplesso}[1]{\mathrm{e}_{#1}}

\renewcommand{\iff}{\mathrel{\longleftrightarrow}} %% Notazione Logica.
\newcommand{\oldiff}{\mathrel{\Longleftrightarrow}}
\renewcommand{\implies}{\mathrel{\rightarrow}} %% Notazione Logica
\newcommand{\oldimplies}{\mathrel{\Longrightarrow}}
\renewcommand{\impliedby}{\mathrel{\leftarrow}} %% Notazione Logica
\newcommand{\oldimpliedby}{\mathrel{\Longleftarrow}}

\newcommand{\IFF}{\quad\Longleftrightarrow\quad}
\newcommand{\IMPLICA}{\quad\Longrightarrow\quad}


\renewcommand{\descriptionlabel}[1]{\hspace{\labelsep}\normalfont #1} % remove bold from description


%% Definizione di Divergenza di K-L

\DeclarePairedDelimiterX{\infdivx}[2]{(}{)}{%
  #1\;\delimsize\|\;#2%
}
\newcommand{\kldiv}{D_{KL}\infdivx}

%% Definizione di \dotminus

\makeatletter
\newcommand{\dotminus}{\mathbin{\text{\@dotminus}}}

\newcommand{\@dotminus}{%
  \ooalign{\hidewidth\raise1ex\hbox{.}\hidewidth\cr$\m@th-$\cr}%
}
\makeatother

%tramite i prossimi due comandi posso decidere come scrivere i logaritmi naturali in tutti i documenti: ho infatti eliminato qualsiasi differenza tra "ln" e "log": se si vuole qualcosa di diverso bisogna inserire manualmente il tutto
\let\ln\relax
\DeclareMathOperator{\ln}{ln}
\let\log\relax
\DeclareMathOperator{\log}{log}
%%%%%%

%% NUOVI COMANDI
\newcommand{\straniero}[1]{\textit{#1}} %parole straniere
\newcommand{\titolo}[1]{\textsc{#1}} %titoli
\newcommand{\qedd}{\tag*{$\blacksquare$}} %qed per ambienti matemastici
\renewcommand{\qedsymbol}{$\blacksquare$} %modifica colore qed
\newcommand{\ooverline}[1]{\overline{\overline{#1}}}
\newcommand{\circoletto}[1]{\left(#1\right)^{\text{o}}}
%
\newcommand{\qmatrice}[1]{\begin{pmatrix}
#1_{11} & \cdots & #1_{1n}\\
\vdots & \ddots & \vdots \\
#1_{m1} & \cdots & #1_{mn}
\end{pmatrix}}
%
\newcommand{\parentesi}[2]{%
\underset{#1}{\underbrace{#2}}%
}
%
\newcommand{\norma}[1]{% Norma
\left\lVert#1\right\rVert%
}
\newcommand{\scalare}[2]{% Scalare
\left\langle #1, #2\right\rangle
}
%%%%%

%% RESTRIZIONI
\newcommand{\referenze}[2]{
        \phantomsection{}#2\textsuperscript{\textcolor{blue}{\textbf{#1}}}
}

\let\restriction\relax

\def\restriction#1#2{\mathchoice
              {\setbox1\hbox{${\displaystyle #1}_{\scriptstyle #2}$}
              \restrictionaux{#1}{#2}}
              {\setbox1\hbox{${\textstyle #1}_{\scriptstyle #2}$}
              \restrictionaux{#1}{#2}}
              {\setbox1\hbox{${\scriptstyle #1}_{\scriptscriptstyle #2}$}
              \restrictionaux{#1}{#2}}
              {\setbox1\hbox{${\scriptscriptstyle #1}_{\scriptscriptstyle #2}$}
              \restrictionaux{#1}{#2}}}
\def\restrictionaux#1#2{{#1\,\smash{\vrule height .8\ht1 depth .85\dp1}}_{\,#2}}
%%%%%%%%%%%

%%% FORMATTAZIONE FOOTNOTEMARK

\def\footnotemarkformatting#1{[#1]}
\renewcommand{\thefootnote}{\footnotemarkformatting{\arabic{footnote}}}

%% SEZIONE GRAFICA
\use{tikz}
\usetikzlibrary{matrix, patterns, calc, decorations.pathreplacing, hobby, decorations.markings, decorations.pathmorphing, babel}
\use{tikz-3dplot}
\use{mathrsfs} %per geogebra
\use{tikz-cd}
\tikzset
{
  %surface/.style={fill=black!10, shading=ball,fill opacity=0.4},
  plane/.style={black,pattern=north east lines},
  curve/.style={black,line width=0.5mm},
  dritto/.style={decoration={markings,mark=at position 0.5 with {\arrow{Stealth}}}, postaction=decorate},
  rovescio/.style={decoration={markings,mark=at position 0.5 with {\arrow{Stealth[reversed]}}}, postaction=decorate}
}
\use{pgfplots} % stampare le funzioni
        \pgfplotsset{/pgf/number format/use comma,compat=1.15}
        %\pgfplotsset{compat=1.15} %per geogebra
        \usepgfplotslibrary{fillbetween, polar}
%%%%%%

%% CITAZIONI
\use{lineno}

\newcommand{\citazione}[1]{%
  \begin{quotation}
  \begin{linenumbers}
  \modulolinenumbers[5]
  \begingroup
  \setlength{\parindent}{0cm}
  \noindent #1
  \endgroup
  \end{linenumbers}
  \end{quotation}\setcounter{linenumber}{1}
  }
%%%%%%

%%%%%%%%%%%%%%%%%%%%%%%%%%%%%%%%%%%%%%%%%%%%
%%%%%%%%%%%%%%%%%%%%%%%%%%%%%%%%%%%%%%%%%%%%

%% AMS THM

\theoremstyle{definition}% default
\newtheorem{thm}{Teorema}[section]
\newtheorem{lem}[thm]{Lemma}
\newtheorem{prop}[thm]{Proposizione}
\newtheorem{cor}[thm]{Corollario}
\newtheorem{esempio}[thm]{Esempio}
\theoremstyle{plain}
\newtheorem{definizione}[thm]{Definizione}
\theoremstyle{remark}
\newtheorem*{oss}{Osservazione}


%%%%%%%%%%%%%%%%%%%%%%%%%%%%%%%%%%%%%%%%%%%%
%%%%%%%%%%%%%%%%%%%%%%%%%%%%%%%%%%%%%%%%%%%%

\use{hyperref}
\hypersetup{%
        pdfauthor={Davide Peccioli},
        pdfsubject={},
        allcolors=black,
        citecolor=black,
%	colorlinks=true,
        bookmarksopen=true}
\setcounter{secnumdepth}{0} % rimuove i numeri di sezione senza rimuovere le ref
\renewcommand{\href}[2]{\textcolor{blue}{#2}} % disabilita il comando href
\use{enotez} %
\setenotez{%
 mark-format = \footnotemarkformatting % Mette i numeri tra parentesi quadre%
}\let\footnote=\endnote % rende tutte le note a pié pagina come delle note a fine file 


\let\olddocument\document % modifico l'ambiende documenti per non dover stampare \printendnote
\let\oldenddocument\enddocument
\renewenvironment{document}%
{%
  \olddocument
}{%
  \printendnotes\oldenddocument
}
\renewcommand{\thethm}{\arabic{thm}}

\usepackage[hyperref]{biblatex}
\addbibresource{~/Documents/org/roam/bib/master.bib}
\author{Davide Peccioli}
\date{\today}
\title{}
\begin{document}

\section{Teorema di Mayer-Vietoris (in coomologia)}
\label{sec:orgdb85fc1}
Sia \(M\) una \href{20250113115909-struttura_differenziabile.org}{varietà differenziabile}, e sia \(\set{U_{0},U_{1}}\) un suo \href{20250103164252-ricoprimento.org}{ricoprimento} \href{20250103145124-topologia.org}{aperto}. Si considerino le seguenti inclusioni:
\begin{equation*}
\begin{tikzcd}
	& {U_0} \\
	{U_0\cap U_1} && {U_0\cup U_1 = M } \\
	& {U_1}
	\arrow["{\jmath_0}", from=1-2, to=2-3]
	\arrow["{\iota_0}", from=2-1, to=1-2]
	\arrow["{\iota_1}"', from=2-1, to=3-2]
	\arrow["{\jmath_1}"', from=3-2, to=2-3]
\end{tikzcd}
\end{equation*}
È possibile farne il \href{20251115174001-pullback_di_una_funzione_tra_varieta_differenziabili.org}{pullback} alle \href{20251115155511-forma_differenziale_in_un_punto.org}{forme differenziali} per ogni \(k \in \N\), ottenendo:
\begin{equation*}
\begin{tikzcd}
	& {A^k(U_0)} \\
	{A^k(U_0\cap U_1)} && {A^k(M )} \\
	& {A^k(U_1)}
	\arrow["{\iota_0^*}"', from=1-2, to=2-1]
	\arrow["{\jmath_0^*}"', from=2-3, to=1-2]
	\arrow["{\jmath_1^*}", from=2-3, to=3-2]
	\arrow["{\iota_1^*}", from=3-2, to=2-1]
\end{tikzcd}
\end{equation*}
Si \href{20251223102452-pullback_di_una_inclusione_tra_varieta_differenziabili.org}{ricorda} che il pullback di una inclusione è la \href{20251201155413-restrizione_di_una_forma_ad_una_sottovarieta.org}{restrizione della forma all'aperto considerato}, ovvero:
\begin{align*}
\iota_{0}^{*}\omega &= \restriction{\omega}{U_{0}\cap U_{1}}\\
\iota_{1}^{*}\omega &= \restriction{\omega}{U_{0}\cap U_{1}}\\
\jmath_{0}^{*}\omega &= \restriction{\omega}{U_{0}}\\
\jmath_{1}^{*}\omega &= \restriction{\omega}{U_{1}}
\end{align*}
Si consideri ora la \href{20241213095808-somma_diretta.org}{somma diretta tra spazi vettoriali} \(A^{k}(U_{0}) \oplus A^{k}(U_{1})\): si definiscono le seguenti funzioni:
\begin{align*}
\jmath^{*}: A^{k}(M) &\longrightarrow A^{k}(U_{0}) \oplus A^{k}(U_{1})\\
\omega &\longmapsto (\jmath_{1}^{*}\omega, \jmath_{2}^{*}\omega) = ( \restriction{\omega}{U_{0}},  \restriction{\omega}{U_{1}})\\[1em]
\iota_{1}^{*}-\iota_{0}^{*} : A^{k}(U_{0}) \oplus A^{k}(U_{1}) &\longrightarrow A^{k}(U_{0}\cap U_{1})\\
(\eta,\tau) &\longmapsto \iota_{1}^{*}\eta - \iota_{0}^{*}\tau = \restriction{\eta}{U_{0}\cap U_{1}} - \restriction{\tau}{U_{0}\cap U_{1}}.
\end{align*}
Si ottiene quindi, considerando queste funzioni per ogni \(k \in \N\) (e ricordando che la somma diretta è commutativa e associativa):
\begin{equation*}
\begin{tikzcd}[row sep=small]
	0 & {A^{\bullet}(M)} & {A^{\bullet}(U_0)\oplus A^{\bullet}(U_1)} & {A^{\bullet}(U_0\cap U_1)} & 0 \\
	& \omega & {( \restriction{\omega}{U_{0}},  \restriction{\omega}{U_{1}})} \\
	&& {(\eta,\tau)} & {\restriction{\eta}{U_{0}\cap U_{1}} - \restriction{\tau}{U_{0}\cap U_{1}}}
	\arrow[from=1-1, to=1-2]
	\arrow["{\jmath^*}", from=1-2, to=1-3]
	\arrow["{\iota_1^*-\iota_0^*}", from=1-3, to=1-4]
	\arrow[from=1-4, to=1-5]
	\arrow[maps to, from=2-2, to=2-3]
	\arrow[maps to, from=3-3, to=3-4]
\end{tikzcd}
\end{equation*}

\begin{thm}
Questa \href{20251115182133-successione_di_spazi_vettoriali_esatta.org}{successione è esatta}.
\end{thm}
\begin{proof}
La dimostrazione si articola in tre fasi:
\begin{enumerate}
\item \uline{\(\jmath^{*}\) è \href{20241219101956-funzione_iniettiva.org}{iniettiva}.}

Sia \(\eta \in A^{\bullet}(M)\) tale che \(\jmath^{*}\eta = 0\). Allora
\begin{equation*}
\restriction{\eta}{U_{0}} = 0,\qquad \restriction{\eta}{U_{1}} = 0
\end{equation*}
e \href{20251201155413-restrizione_di_una_forma_ad_una_sottovarieta.org}{pertanto} \(j = 0\) in \(M=U_{0}\cup U_{1}\).

\item \uline{\(\operatorname{Im}\jmath^{*} = \ker (\iota_{1}^{*}-\iota_{0}^{*})\)}\footnote{Vedi
\begin{itemize}
\item \href{20250202173528-dominio_range_e_campo_di_una_classe_relazione.org}{Range di una funzione}
\item \href{20251121143525-kernel_di_una_funzione_tra_spazi_vettoriali.org}{Kernel di una funzione tra spazi vettoriali}
\end{itemize}}

(\(\subseteq\)): è ovvio.

(\(\supseteq\)): sia \((\eta,\tau) \in \ker (\iota_{1}^{*}-\iota_{0}^{*})\). Allora in \(U_{0}\cap U_{1}\) si ha che \(\eta \equiv \tau\). Definendo quindi
\begin{equation*}
\omega = %
\begin{cases}
\eta & \text{su } U_{0}\\
\tau & \text{su } U_{1}
\end{cases}
\end{equation*}
questa è ben definita e \(\jmath^{*} \omega = (\eta,\tau)\).

\item \uline{\((\iota_{1}^{*}-\iota_{0}^{*})\) è \href{20241213105600-funzione_suriettiva.org}{suriettiva}}.

Si utilizza una partizione dell'unità subordinata al ricoprimento \(\mathcal{U} = \set{U_{0}, U_{1}}\): esistono \href{20250113144722-funzioni_cinfinito_tra_varieta_differenziabili.org}{due funzioni} \(\rho_{0},\rho_{1} \in C^{\infty}(M) = A^{0}(M)\)  tali che i loro \href{20250701115005-supporto_di_una_funzione.org}{supporti}
\begin{align*}
 \operatorname{supp} \rho_{0} &= \overline{\set{
 	x \in M \mid \rho_{0}(x) \neq 0
 }} \subseteq U_{0}\\
 \operatorname{supp} \rho_{1} &= \overline{\set{
 	x \in M \mid \rho_{1}(x) \neq 0
 }} \subseteq U_{1}
\end{align*}
e per ogni \(x \in M\): \(\rho_{0}(x) + \rho_{1}(x) = 1\).

Sia quindi \(\eta \in A^{\bullet}(U_{0}\cap U_{1})\). Si definiscono:
\begin{align*}
 \omega &= -\rho_{1}\eta\quad\text{ su }U_{0}\\
 \tau &= \rho_{0}\eta\quad\text{ su }U_{1}
\end{align*}
(estese a \(0\) dove non sono definite), e si ottiene che
\begin{align*}
 (\iota_{1}^{*}-\iota_{0}^{*})(\omega,\tau) &=%
 	\rho_{0}\eta+\rho_{1}\eta \\
 	&= (\rho_{0}+\rho_{1})\eta = \eta.%
 	\qedhere
\end{align*}
\end{enumerate}
\end{proof}

\begin{oss}
È possibile vedere questa come una \href{20251115182707-sec_di_complessi_di_cocatene.org}{successione} di \href{20251115182320-complesso_di_cocatene.org}{complessi di cocatene}
\end{oss}

\begin{oss}
Tramite lo \href{20251115182954-zig_zag_lemma_per_complessi_di_cocatene.org}{Zig-Zag Lemma}, otteniamo una \href{20251115182133-successione_di_spazi_vettoriali_esatta.org}{successione esatta} in \href{20251115182537-coomologia_di_un_complesso_di_cocatene.org}{coomologia} (\href{20250122122650-quoziente_di_somma_diretta_di_moduli.org}{somma diretta commuta con quoziente}):
\begin{equation*}
\begin{tikzcd}
	\dots \\
	{H^k(M)} & {H^k(U_0)\oplus H^k(U_1)} & {H^k(U_0\cap U_1)} \\
	\\
	{H^{k+1}(M)} & {H^{k+1}(U_0)\oplus H^{k+1}(U_1)} & \dots
	\arrow[from=1-1, to=2-1]
	\arrow["{\jmath^*}", from=2-1, to=2-2]
	\arrow["{\iota_1^*-\iota_0^*}", from=2-2, to=2-3]
	\arrow["{\partial^*}", from=2-3, to=4-1]
	\arrow["{\jmath^*}"', from=4-1, to=4-2]
	\arrow[from=4-2, to=4-3]
\end{tikzcd}
\end{equation*}
ed è possibile calcolare esplicitamente il morfismo di connessione \(\partial^{*}\) (\href{20251115182904-morfismo_tra_complessi_di_cocatene_induce_morfismo_in_coomologia.org}{oltre che gli altri morfismi}), seguendo la costruizione: dato \([\eta] \in H^{k}(U_{0}\cap U_{1})\),
\begin{equation*}
\eta = (\iota_{1}^{*}-\iota_{0}^{*})(-\rho_{1} \eta, \rho_{0} \eta)
\end{equation*}
Facendo il differenziale di \((-\rho_{1} \eta, \rho_{0} \eta)\) su ambo le componenti, si \href{20251115160537-differenziale_di_una_forma.org}{ottiene}
\begin{equation*}
(- \dif \rho_{1} \wedge \eta, \dif \rho_{0} \wedge \eta)
\end{equation*}
in quanto \(\dif\eta = 0\)\footnote{Vedi forme \href{20251115172517-forma_differenziale_chiusa.org}{chiuse} ed \href{20251115172517-forma_differenziale_chiusa.org}{esatte}.}. Ponendo ora
\begin{equation*}
\omega = %
\begin{cases}
-\dif \rho_{1} \wedge \eta & \text{su } U_{0}\\
\dif \rho_{0} \wedge \eta & \text{su } U_{1}
\end{cases}
\end{equation*}
che è ben definita in quanto in \(p \in U_{0}\cap U_{1}\), se le coordinate in un intorno sono \((x^{1},\dots,x^{n})\) si ha
\begin{equation*}
-\dif \rho_{1} \wedge \eta
	= - \dpd{\rho_{1}}{{x^{\mu}}} \dif x^{\mu} \wedge \eta %
	= - \dpd{}{{x^{\mu}}} (1-\rho_{0}) \dif x^{\mu} \wedge \eta
	= \dpd{\rho_{0}}{{x^{\mu}}} \dif x^{\mu} \wedge \eta = \dif \rho_{0} \wedge \eta
\end{equation*}
e si ottiene per definizione che \(\jmath^{*} \omega = (- \dif \rho_{1} \wedge \eta, \dif \rho_{0} \wedge \eta)\).

Quindi \(\partial^{*}[\eta] = [\omega]\), con
\begin{equation*}
\omega = %
\begin{cases}
-\dif \rho_{1} \wedge \eta & \text{su } U_{0}\\
\dif \rho_{0} \wedge \eta & \text{su } U_{1}
\end{cases}
\end{equation*}
\end{oss}
\end{document}
