% Created 2026-02-07 Sat 19:30
% Intended LaTeX compiler: pdflatex
\documentclass[10pt]{article}
%% CREATO CON ORG - EMACS
\newcommand{\use}[2][]{\usepackage[#1]{#2}}
% PACCHETTI FONDAMENTLAI
\use[utf8]{inputenc}
\use[T1]{fontenc}
\use{graphicx}
\use{longtable}
\use{wrapfig}
\use{rotating}
\use[normalem]{ulem}
\use{amsmath}
\use{amsthm}
\use{amssymb}

\use{eucal} % Cambia mathcal{...}

\use{capt-of}
\use[italian]{babel}
\use[babel]{csquotes}
% bib la TEX lo carica in automatico org-cite
\use{microtype}
\use{lmodern}
\use{subfig} % sottofigure
\use{multicol} % due colonne
\use{lipsum} % lorem ipsum
\use{color} % colori in latex
\use{parskip} % rimuove l'indentazione dei nuovi paragrafi %% Add parbox=false to all new tcolorbox
\use{centernot}
\use[outline]{contour}\contourlength{3pt}
\use{fancyhdr}
\use{layout}
\use[most]{tcolorbox} % Riquadri colorati
\use{ifthen} % IFTHEN
\use{geometry}

% pacchetti matematica
\use{yhmath}
\use{dsfont}
\use{mathrsfs}
\use{cancel} % semplificare
\use{polynom} %divisione tra polinomi
\use{forest} % grafi ad albero
\use{booktabs} % tabelle
\use{commath} %simboli e differenziali
\use{bm} %bold
\use[fulladjust]{marginnote} %to use marginnote for date notes
\use{arrayjobx}%array
\use[intlimits]{empheq} % Riquadri colorati attorno alle equazioni
\use{mathtools}
\use{circuitikz} % Disegnare i circuiti
\use{mathtools}
\use{stmaryrd} % [[ \llbracket ]] \rrbracket
\use{bussproofs} % dimostrazioni

%%%%%%%%%%%%%


%%%% QUIVER
\newcommand{\duepunti}{\,\mathchar\numexpr"6000+`:\relax\,}
% A TikZ style for curved arrows of a fixed height, due to AndréC.
\tikzset{curve/.style={settings={#1},to path={(\tikztostart)
    .. controls ($(\tikztostart)!\pv{pos}!(\tikztotarget)!\pv{height}!270:(\tikztotarget)$)
    and ($(\tikztostart)!1-\pv{pos}!(\tikztotarget)!\pv{height}!270:(\tikztotarget)$)
    .. (\tikztotarget)\tikztonodes}},
    settings/.code={\tikzset{quiver/.cd,#1}
        \def\pv##1{\pgfkeysvalueof{/tikz/quiver/##1}}},
    quiver/.cd,pos/.initial=0.35,height/.initial=0}

% TikZ arrowhead/tail styles.
\tikzset{tail reversed/.code={\pgfsetarrowsstart{tikzcd to}}}
\tikzset{2tail/.code={\pgfsetarrowsstart{Implies[reversed]}}}
\tikzset{2tail reversed/.code={\pgfsetarrowsstart{Implies}}}
% TikZ arrow styles.
\tikzset{no body/.style={/tikz/dash pattern=on 0 off 1mm}}
%%%%%%%%%%


%% DEFINIZIONI COMANDI MATEMATICI
\let\sin\relax %TOGLIE LA DEFINIZIONE SU "\sin"

% cambia la definizione di empty set
% ---
\let\oldemptyset\emptyset
% ---
% \let\emptyset\varnothing
% ---
% \let\emptyset\relax
% \newcommand{\emptyset}{\text{\textnormal{\O}}}
% ---

\DeclareMathOperator{\bounded}{bd}
\DeclareMathOperator{\sin}{sen}
\DeclareMathOperator{\epi}{Epi}
\DeclareMathOperator{\cl}{cl}
\DeclareMathOperator{\graph}{graph}
\DeclareMathOperator{\arcsec}{arcsec}
\DeclareMathOperator{\arccot}{arccot}
\DeclareMathOperator{\arccsc}{arccsc}
\DeclareMathOperator{\spettro}{Spettro}
\DeclareMathOperator{\nulls}{nullspace}
\DeclareMathOperator{\dom}{dom}
\DeclareMathOperator{\ar}{ar}
\DeclareMathOperator{\const}{Const}
\DeclareMathOperator{\fun}{Fun}
\DeclareMathOperator{\rel}{Rel}
\DeclareMathOperator{\altezza}{ht}
\let\det\relax %TOGLIE LA DEFINIZIONE SU "\det"
\DeclareMathOperator{\det}{det}
\DeclareMathOperator{\End}{End}
\DeclareMathOperator{\gl}{GL}
\def\Id{\mathrm{Id}}
\def\id{\mathrm{id}}
\DeclareMathOperator{\I}{\mathds{1}}
\DeclareMathOperator{\II}{II}
\DeclareMathOperator{\rank}{rank}
\DeclareMathOperator{\tr}{tr}
\DeclareMathOperator{\tc}{t.c.}
\DeclareMathOperator{\T}{T}
\DeclareMathOperator{\var}{Var}
\DeclareMathOperator{\cov}{Cov}
\DeclareMathOperator{\st}{st}
\DeclareMathOperator{\mon}{Mon}
\newcommand{\card}[1]{\left\vert #1 \right\vert}
\newcommand{\trasposta}[1]{\prescript{\text{T}}{}{#1}}
\newcommand{\1}{\mathds{1}}
\newcommand{\R}{\mathds{R}}
\newcommand{\diesis}{\#}
\newcommand{\bemolle}{\flat}
\newcommand{\nonstandard}[1]{\prescript{*}{}{#1}}
\newcommand{\starR}{\nonstandard{\R}}
\newcommand{\borel}{\mathscr{B}}
\newcommand{\lebesgue}[1]{\mathscr{L}\left(#1\right)}
\newcommand{\media}{\mathds{E}}
\newcommand{\K}{\mathds{K}}
\newcommand{\A}{\mathds{A}}
\newcommand{\Q}{\mathds{Q}}
\newcommand{\N}{\mathds{N}}
\newcommand{\C}{\mathds{C}}
\newcommand{\Z}{\mathds{Z}}
\newcommand{\qo}{\hspace{1em}\text{q.o.}\,}
\renewcommand{\tilde}[1]{\widetilde{#1}}
\renewcommand{\parallel}{\mathrel{/\mkern-5mu/}}
\newcommand{\parti}[2][]{\wp_{#1}(#2)}
\newcommand{\diff}[1]{\operatorname{d}_{#1}}
\let\oldvec\vec
\renewcommand{\vec}[1]{\overrightarrow{\vphantom{i}#1}}
\newcommand{\floor}[1]{\left\lfloor #1 \right\rfloor}
\newcommand{\cat}[1]{\mathbf{#1}}
\newcommand{\dfreccia}[1]{\xrightarrow{\ #1 \ }}
\newcommand{\sfreccia}[1]{\xleftarrow{\ #1 \ }}
\newcommand{\formalsum}[2]{{\sum_{#1}^{#2}}{\vphantom{\sum}}'}
\newcommand{\minim}[2]{\mu_{#1}\, \left(#2\right)}
\newcommand{\concat}{\null^{\frown}} % concatenazione di stringe
\newcommand{\godelcode}[1]{\langle\!\langle #1 \rangle\!\rangle}
\newcommand{\godeldec}[1]{(\!(#1)\!)}
\newcommand{\termcode}[1]{\ulcorner #1\urcorner}
\newcommand{\partialto}{\dashrightarrow}
\newcommand{\restricted}{\upharpoonright}
\newcommand{\embeds}{\precsim}
\newcommand{\surjects}{\twoheadrightarrow}
\newcommand{\equipotenti}{\asymp}
%% \newcommand{\dotplus}{\mathbin{\dot{+}}} %% A quanto pare esiste già
\newcommand{\bigdot}{\mathbin{\boldsymbol{\cdot}}}
\newcommand{\dotexp}[1]{^{.#1}}
\newcommand{\conv}{\mathbin{*}}
\newcommand{\convolution}[2]{(#1\conv #2)}
\newcommand{\nil}{\mathfrak{N}}
\newcommand{\divisore}{\mathrel{|}}
\newcommand{\simplesso}[1]{\mathrm{e}_{#1}}

\renewcommand{\iff}{\mathrel{\longleftrightarrow}} %% Notazione Logica.
\newcommand{\oldiff}{\mathrel{\Longleftrightarrow}}
\renewcommand{\implies}{\mathrel{\rightarrow}} %% Notazione Logica
\newcommand{\oldimplies}{\mathrel{\Longrightarrow}}
\renewcommand{\impliedby}{\mathrel{\leftarrow}} %% Notazione Logica
\newcommand{\oldimpliedby}{\mathrel{\Longleftarrow}}

\newcommand{\IFF}{\quad\Longleftrightarrow\quad}
\newcommand{\IMPLICA}{\quad\Longrightarrow\quad}


\renewcommand{\descriptionlabel}[1]{\hspace{\labelsep}\normalfont #1} % remove bold from description


%% Definizione di Divergenza di K-L

\DeclarePairedDelimiterX{\infdivx}[2]{(}{)}{%
  #1\;\delimsize\|\;#2%
}
\newcommand{\kldiv}{D_{KL}\infdivx}

%% Definizione di \dotminus

\makeatletter
\newcommand{\dotminus}{\mathbin{\text{\@dotminus}}}

\newcommand{\@dotminus}{%
  \ooalign{\hidewidth\raise1ex\hbox{.}\hidewidth\cr$\m@th-$\cr}%
}
\makeatother

%tramite i prossimi due comandi posso decidere come scrivere i logaritmi naturali in tutti i documenti: ho infatti eliminato qualsiasi differenza tra "ln" e "log": se si vuole qualcosa di diverso bisogna inserire manualmente il tutto
\let\ln\relax
\DeclareMathOperator{\ln}{ln}
\let\log\relax
\DeclareMathOperator{\log}{log}
%%%%%%

%% NUOVI COMANDI
\newcommand{\straniero}[1]{\textit{#1}} %parole straniere
\newcommand{\titolo}[1]{\textsc{#1}} %titoli
\newcommand{\qedd}{\tag*{$\blacksquare$}} %qed per ambienti matemastici
\renewcommand{\qedsymbol}{$\blacksquare$} %modifica colore qed
\newcommand{\ooverline}[1]{\overline{\overline{#1}}}
\newcommand{\circoletto}[1]{\left(#1\right)^{\text{o}}}
%
\newcommand{\qmatrice}[1]{\begin{pmatrix}
#1_{11} & \cdots & #1_{1n}\\
\vdots & \ddots & \vdots \\
#1_{m1} & \cdots & #1_{mn}
\end{pmatrix}}
%
\newcommand{\parentesi}[2]{%
\underset{#1}{\underbrace{#2}}%
}
%
\newcommand{\norma}[1]{% Norma
\left\lVert#1\right\rVert%
}
\newcommand{\scalare}[2]{% Scalare
\left\langle #1, #2\right\rangle
}
%%%%%

%% RESTRIZIONI
\newcommand{\referenze}[2]{
        \phantomsection{}#2\textsuperscript{\textcolor{blue}{\textbf{#1}}}
}

\let\restriction\relax

\def\restriction#1#2{\mathchoice
              {\setbox1\hbox{${\displaystyle #1}_{\scriptstyle #2}$}
              \restrictionaux{#1}{#2}}
              {\setbox1\hbox{${\textstyle #1}_{\scriptstyle #2}$}
              \restrictionaux{#1}{#2}}
              {\setbox1\hbox{${\scriptstyle #1}_{\scriptscriptstyle #2}$}
              \restrictionaux{#1}{#2}}
              {\setbox1\hbox{${\scriptscriptstyle #1}_{\scriptscriptstyle #2}$}
              \restrictionaux{#1}{#2}}}
\def\restrictionaux#1#2{{#1\,\smash{\vrule height .8\ht1 depth .85\dp1}}_{\,#2}}
%%%%%%%%%%%

%%% FORMATTAZIONE FOOTNOTEMARK

\def\footnotemarkformatting#1{[#1]}
\renewcommand{\thefootnote}{\footnotemarkformatting{\arabic{footnote}}}

%% SEZIONE GRAFICA
\use{tikz}
\usetikzlibrary{matrix, patterns, calc, decorations.pathreplacing, hobby, decorations.markings, decorations.pathmorphing, babel}
\use{tikz-3dplot}
\use{mathrsfs} %per geogebra
\use{tikz-cd}
\tikzset
{
  %surface/.style={fill=black!10, shading=ball,fill opacity=0.4},
  plane/.style={black,pattern=north east lines},
  curve/.style={black,line width=0.5mm},
  dritto/.style={decoration={markings,mark=at position 0.5 with {\arrow{Stealth}}}, postaction=decorate},
  rovescio/.style={decoration={markings,mark=at position 0.5 with {\arrow{Stealth[reversed]}}}, postaction=decorate}
}
\use{pgfplots} % stampare le funzioni
        \pgfplotsset{/pgf/number format/use comma,compat=1.15}
        %\pgfplotsset{compat=1.15} %per geogebra
        \usepgfplotslibrary{fillbetween, polar}
%%%%%%

%% CITAZIONI
\use{lineno}

\newcommand{\citazione}[1]{%
  \begin{quotation}
  \begin{linenumbers}
  \modulolinenumbers[5]
  \begingroup
  \setlength{\parindent}{0cm}
  \noindent #1
  \endgroup
  \end{linenumbers}
  \end{quotation}\setcounter{linenumber}{1}
  }
%%%%%%

%%%%%%%%%%%%%%%%%%%%%%%%%%%%%%%%%%%%%%%%%%%%
%%%%%%%%%%%%%%%%%%%%%%%%%%%%%%%%%%%%%%%%%%%%

%% AMS THM

\theoremstyle{definition}% default
\newtheorem{thm}{Teorema}[section]
\newtheorem{lem}[thm]{Lemma}
\newtheorem{prop}[thm]{Proposizione}
\newtheorem{cor}[thm]{Corollario}
\newtheorem{esempio}[thm]{Esempio}
\theoremstyle{plain}
\newtheorem{definizione}[thm]{Definizione}
\theoremstyle{remark}
\newtheorem*{oss}{Osservazione}


%%%%%%%%%%%%%%%%%%%%%%%%%%%%%%%%%%%%%%%%%%%%
%%%%%%%%%%%%%%%%%%%%%%%%%%%%%%%%%%%%%%%%%%%%

\use{hyperref}
\hypersetup{%
        pdfauthor={Davide Peccioli},
        pdfsubject={},
        allcolors=black,
        citecolor=black,
%	colorlinks=true,
        bookmarksopen=true}
\setcounter{secnumdepth}{0} % rimuove i numeri di sezione senza rimuovere le ref
\renewcommand{\href}[2]{\textcolor{blue}{#2}} % disabilita il comando href
\use{enotez} %
\setenotez{%
 mark-format = \footnotemarkformatting % Mette i numeri tra parentesi quadre%
}\let\footnote=\endnote % rende tutte le note a pié pagina come delle note a fine file 


\let\olddocument\document % modifico l'ambiende documenti per non dover stampare \printendnote
\let\oldenddocument\enddocument
\renewenvironment{document}%
{%
  \olddocument
}{%
  \printendnotes\oldenddocument
}
\renewcommand{\thethm}{\arabic{thm}}

\usepackage[hyperref]{biblatex}
\addbibresource{~/Documents/org/roam/bib/master.bib}
\renewcommand{\href}[2]{#2}
\author{Davide Peccioli}
\date{\today}
\title{Proprietà di chiusura degli Spazi Polacchi}
\begin{document}

\section{Chiusura per omeomorfismi}
\label{sec:orgaaefa41}
Se \(X\) uno \href{20250301194013-spazio_polacco.org}{spazio polacco} e \(r: X\to Y\) è un \href{20250111142332-omeomorfismo.org}{omeomorfismo}, allora \(Y\) è uno \href{20250301194013-spazio_polacco.org}{spazio polacco}.
\section{Chiusura per sottoinsiemi chiusi}
\label{sec:org6b16e6d}
Se \(X\) uno \href{20250301194013-spazio_polacco.org}{spazio polacco} e \(C \subseteq X\) è un \href{20250131155822-operazioni_insiemistiche_tra_classi_mk.org}{sottoinsieme} \href{20250103145124-topologia.org}{chiuso}, allora \(C\) è uno \href{20250301194013-spazio_polacco.org}{spazio polacco} con la \href{20250103163814-sottospazio_topologico.org}{topologia di sottospazio}.
\subsection{Dimostrazione}
\label{sec:org53c85d9}
Segue dalla \href{20250303120747-caratterizzazione_dei_chiusi_in_termini_di_successioni.org}{caratterizzazione dei chiusi in termini di successioni}.
\section{Chiusura per prodotto cartesiano numerabile}
\label{sec:orgfb558de}
Se \(\langle X_{n}\rangle_{n \in \omega}\) è una \href{20250206170922-sequenze_e_stringhe.org}{sequenza} di \href{20250301194013-spazio_polacco.org}{spazi polacchi} (vedi \href{20250203161110-numeri_naturali_sono_ordinali.org}{Ordinale omega}), allora il \href{20250131183735-prodotto_cartesiano_di_classi_mk.org}{prodotto}
\begin{equation*}
\prod_{n \in \omega} X_{n}
\end{equation*}
è uno spazio polacco con la \href{20250109154723-topologia_prodotto.org}{topologia prodotto}.
\subsection{{\bfseries\sffamily TODO} Dimostrazione}
\label{sec:orgc65e207}
\section{Chiusura per unione disgiunta numerabile}
\label{sec:orgad67629}
Se \(\langle X_{n}\rangle_{n \in \omega}\) è una \href{20250206170922-sequenze_e_stringhe.org}{sequenza} di \href{20250301194013-spazio_polacco.org}{spazi polacchi} (vedi \href{20250203161110-numeri_naturali_sono_ordinali.org}{Ordinale omega}), allora l'\href{20250113175700-unione_disgiunta.org}{unione disgiunta}
\begin{equation*}
\coprod_{n \in \omega} X_{n}
\end{equation*}
è uno \href{20250301194013-spazio_polacco.org}{spazio polacco}
\subsection{{\bfseries\sffamily TODO} Dimostrazione}
\label{sec:orgab0858e}
\section{Chisura per intersezione numerabile}
\label{sec:org54fb67c}
Sia \(X\) uno \href{20250301194013-spazio_polacco.org}{spazio polacco} e sia \(\langle Y_{n}\rangle_{n \in \omega}\) una \href{20250206170922-sequenze_e_stringhe.org}{sequenza} di \href{20250301194013-spazio_polacco.org}{spazi polacchi}, \(Y_{n} \subseteq X\) (vedi \href{20250203161110-numeri_naturali_sono_ordinali.org}{Ordinale omega} e \href{20250103163814-sottospazio_topologico.org}{Sottospazio topologico}).

Allora \(Y\coloneqq\bigcap_{n \in\omega} Y_{i}\) è uno spazio polacco dotato della \href{20250103163814-sottospazio_topologico.org}{topologia di sottospazio} (vedi \href{20250131155822-operazioni_insiemistiche_tra_classi_mk.org}{Intersezione di classi MK}).
\subsection{{\bfseries\sffamily TODO} Dimostrazione}
\label{sec:org668d3dc}
\section{Sottoinsiemi aperti di spazi polacchi sono polacchi}
\label{sec:org8253e16}
Sia \(X\) uno \href{20250301194013-spazio_polacco.org}{spazio polacco} e sia \(Y \subseteq X\) un \href{20250103163814-sottospazio_topologico.org}{sottoinsieme} \href{20250103145124-topologia.org}{aperto}. Allora \(Y\) è \href{20250301194013-spazio_polacco.org}{polacco} con la \href{20250103163814-sottospazio_topologico.org}{topologia di sottospazio}.
\subsection{Dimostrazione}
\label{sec:org2f05363}
Sicuramente \(Y\) è \href{20250111142303-spazio_topologico_a_base_numerabile.org}{secondo numerabile}, e pertanto è necessario mostrare solamente che sia \href{20250301193401-spazio_topologico_metrizzabile.org}{completamente metrizzabile}.

Sia \(d: X\times X\to \R\) una \href{20250301193511-spazio_metrico.org}{distanza} su \(X\) tale che \((X,d)\) sia uno \href{20250301194153-spazio_metrico_completo.org}{spazio metrico completo}, e tale che \href{20250301193530-topologia_indotta_da_una_distanza.org}{induca} la topologia di \(X\).
Si supponga \(d\) limitata\footnote{Questo si può sempre fare, vedi l'osservazione di \href{20250301193401-spazio_topologico_metrizzabile.org}{Spazio topologico metrizzabile}}

Sia \(F\coloneqq X\setminus Y\) (vedi \href{20250131155822-operazioni_insiemistiche_tra_classi_mk.org}{Sottrazione di classi MK}) e per ogni \(x \in X\) si ponga (vedi \href{20250203102516-massimo_e_minimo.org}{Estremo superiore ed inferiore})
\begin{equation*}
d(x, F) \coloneqq \operatorname{inf}\set{d(x,y)\ |\ y \in F}
\end{equation*}
e si definisca su \(Y\):
\begin{equation*}
d'(x,y) \coloneqq d(x,y) + \left|\frac{1}{d(x,F)} - \frac{1}{d(y,F)}\right|
\end{equation*}

Osserviamo\footnote{Sarebbe da dimostrare questa continuità\label{org8fe9048}} che \(d(\cdot, F): X\to [0,1]\) è una funzione \href{20250103103252-funzione_continua.org}{continua}.
\subsubsection{\(d\) e \(d'\) inducono la stessa topologia su \(Y\)}
\label{sec:org606ce38}
Per definizione di \(d'\), si ha che, \(\forall\, x,y \in Y\):
\begin{equation*}
d'(x,y) \le d(x,y)
\end{equation*}
e pertanto, posti
\begin{align*}
B_{d}(x,\varepsilon) &\coloneqq \set{y \in X\ |\ d(x,y)<\varepsilon}\\
B_{d'}(x,\varepsilon') &\coloneqq \set{y \in Y\ |\ d'(x,y)<\varepsilon'}
\end{align*}

Si ha che, per ogni \(x \in Y\) e per ogni \(\varepsilon>0\):
\begin{equation*}
B_{d'}(x,\varepsilon) \subseteq B_{d}(x,\varepsilon)\cap Y.
\end{equation*}
È sufficiente dimostrare che per ogni \(x \in Y\) e per ogni \(\varepsilon>0\) esiste \(\varepsilon'>0\) tale che
\begin{equation*}
B_{d}(x,\varepsilon')\cap Y \subseteq B_{d'}(x,\varepsilon)
\end{equation*}

Siano \(x \in Y\) e \(\varepsilon>0\) fissati.
Esiste\footnote{Infatti \(d(\cdot, F): Y\to (0, 1]\) è continua e  \(\frac{1}{\cdot}: (0,1]\to [1,+\infty)\) è continua, e pertanto
\begin{equation*}
\frac{1}{d(\cdot, F)}: Y\to [1,+\infty)
\end{equation*}
è continua e dunque, per ogni \(x \in Y\) e \(\varepsilon>0\) esiste \(\varepsilon'>0\) tale che per ogni \(y \in Y\cap B_{d}(x,\varepsilon')\) si ha
\begin{equation*}
\left|\frac{1}{d(x,F)}-\frac{1}{d(y,F)}\right|<\frac{\varepsilon}{2}
\end{equation*}
Senza perdità di generalità si può porre \(\varepsilon'<\frac{\varepsilon}{2}\).} \(\varepsilon'>0\) tale che: \(\varepsilon'<\frac{\varepsilon}{2}\) e per ogni \(y \in Y\cap B_{d}(x,\varepsilon')\):
\begin{equation*}
\left|\frac{1}{d(x,F)}-\frac{1}{d(y,F)}\right|<\frac{\varepsilon}{2}
\end{equation*}

Allora, per ogni \(y \in Y\cap B_{d}(x,\varepsilon')\) si ha
\begin{equation*}
d'(x,y) = d(x,y) + \left|\frac{1}{d(x,F)}-\frac{1}{d(y,F)}\right|<\varepsilon'+\frac{\varepsilon}{2} < \varepsilon
\end{equation*}
e quindi \(y \in B_{d'}(x,\varepsilon)\).
\subsubsection{\((Y,d')\) è uno spazio metrico completo.}
\label{sec:orgcff6b8c}
Siccome \(d\) e \(d'\) \href{20250301193530-topologia_indotta_da_una_distanza.org}{inducono} la stessa topologia, allora ogni \href{20250115100904-successione.org}{successione} di \(d'\)-\href{20250303134529-successione_di_cauchy.org}{Cauchy} è \(d\)-Cauchy.

Sia dunque \((y_{i})_{i \in \N} \subseteq Y\) una successione di \(d'\)-Cauchy. Allora\footnote{Infatti \(d'\)-Cauchy implica \(d\)-Cauchy, ma \((X,d)\) è uno \href{20250301194153-spazio_metrico_completo.org}{spazio metrico completo}, e quindi (per definizione) ogni successione di Cauchy è convergente.} \(y_{i}\to y\) \href{20250115100930-convergenza_per_una_successione.org}{converge} a qualche \(y \in X\).

Consideriamo ora la successione \(\left(\frac{1}{d(y_{i},F)}\right)_{i \in \N}\): questa è di Cauchy in \(\R\), infatti per ogni \(i,j\)
\begin{align*}
\left|\frac{1}{d(y_{i},F)}-\frac{1}{d(y_{j}, F)}\right| &= \left|d'(y_{i}, y_{j})- d(y_{i}, y_{j})\right|\\
&=\le \left|d'(y_{i},y_{j})\right|+\left|d(y_{i},y_{j})\right|
\end{align*}
e \((y_{i})_{i \in \N}\) è sia \(d\)-Cauchy che \(d'\)-Cauchy, per cui si ha la tesi.

Siccome \(\left(\frac{1}{d(y_{i},F)}\right)_{i \in \N}\) è di Cauchy in \(\R\), allora converge a \(\ell \in \R\).
\begin{itemize}
\item Poiché \(y_{i} \in Y\), allora \(d(y_{i}, F) \neq 0\) per ogni \(i \in \N\), dunque la successione esiste per ogni \(i\); dunque per ogni \(i \in \N\):
\begin{equation*}
  d(y_{i}, F) >0
\end{equation*}
Necessariamente \(\ell\ge 0\), per il \href{20250304161634-teorema_di_permanenza_del_segno.org}{Teorema di permanenza del segno}.
\item Poiché \(d\) è limitata, allora \(\ell \neq 0\).
\end{itemize}

Quindi \(\ell > 0\).

Poiché
\begin{align*}
f: (0,+\infty) &\longrightarrow (0,+\infty)\\
t &\longmapsto \frac{1}{t}
\end{align*}
è continua, \href{20250304142114-funzione_continua_e_continua_per_successioni.org}{allora} è \href{20250304142114-funzione_continua_e_continua_per_successioni.org}{continua per successioni}, e pertanto
\begin{equation*}
d(y_{i}, F)\to \frac{1}{\ell}\neq 0
\end{equation*}

Per \href{20250103103252-funzione_continua.org}{continuità}\textsuperscript{\ref{org8fe9048}} di \(d(\cdot, F)\) (vedi \href{20250304142114-funzione_continua_e_continua_per_successioni.org}{Funzione continua è continua per successioni})
\begin{equation*}
d(y_{i}, F)\to d(y,F)
\end{equation*}

Per l'\href{20250304162602-unicita_del_limite.org}{unicità del limite}, \(d(y,F)=1/\ell\neq 0\), e quindi \(y \notin F\) e quindi \(y \in Y\).
\end{document}
