% Created 2026-02-07 Sat 19:30
% Intended LaTeX compiler: pdflatex
\documentclass[10pt]{article}
%% CREATO CON ORG - EMACS
\newcommand{\use}[2][]{\usepackage[#1]{#2}}
% PACCHETTI FONDAMENTLAI
\use[utf8]{inputenc}
\use[T1]{fontenc}
\use{graphicx}
\use{longtable}
\use{wrapfig}
\use{rotating}
\use[normalem]{ulem}
\use{amsmath}
\use{amsthm}
\use{amssymb}

\use{eucal} % Cambia mathcal{...}

\use{capt-of}
\use[italian]{babel}
\use[babel]{csquotes}
% bib la TEX lo carica in automatico org-cite
\use{microtype}
\use{lmodern}
\use{subfig} % sottofigure
\use{multicol} % due colonne
\use{lipsum} % lorem ipsum
\use{color} % colori in latex
\use{parskip} % rimuove l'indentazione dei nuovi paragrafi %% Add parbox=false to all new tcolorbox
\use{centernot}
\use[outline]{contour}\contourlength{3pt}
\use{fancyhdr}
\use{layout}
\use[most]{tcolorbox} % Riquadri colorati
\use{ifthen} % IFTHEN
\use{geometry}

% pacchetti matematica
\use{yhmath}
\use{dsfont}
\use{mathrsfs}
\use{cancel} % semplificare
\use{polynom} %divisione tra polinomi
\use{forest} % grafi ad albero
\use{booktabs} % tabelle
\use{commath} %simboli e differenziali
\use{bm} %bold
\use[fulladjust]{marginnote} %to use marginnote for date notes
\use{arrayjobx}%array
\use[intlimits]{empheq} % Riquadri colorati attorno alle equazioni
\use{mathtools}
\use{circuitikz} % Disegnare i circuiti
\use{mathtools}
\use{stmaryrd} % [[ \llbracket ]] \rrbracket
\use{bussproofs} % dimostrazioni

%%%%%%%%%%%%%


%%%% QUIVER
\newcommand{\duepunti}{\,\mathchar\numexpr"6000+`:\relax\,}
% A TikZ style for curved arrows of a fixed height, due to AndréC.
\tikzset{curve/.style={settings={#1},to path={(\tikztostart)
    .. controls ($(\tikztostart)!\pv{pos}!(\tikztotarget)!\pv{height}!270:(\tikztotarget)$)
    and ($(\tikztostart)!1-\pv{pos}!(\tikztotarget)!\pv{height}!270:(\tikztotarget)$)
    .. (\tikztotarget)\tikztonodes}},
    settings/.code={\tikzset{quiver/.cd,#1}
        \def\pv##1{\pgfkeysvalueof{/tikz/quiver/##1}}},
    quiver/.cd,pos/.initial=0.35,height/.initial=0}

% TikZ arrowhead/tail styles.
\tikzset{tail reversed/.code={\pgfsetarrowsstart{tikzcd to}}}
\tikzset{2tail/.code={\pgfsetarrowsstart{Implies[reversed]}}}
\tikzset{2tail reversed/.code={\pgfsetarrowsstart{Implies}}}
% TikZ arrow styles.
\tikzset{no body/.style={/tikz/dash pattern=on 0 off 1mm}}
%%%%%%%%%%


%% DEFINIZIONI COMANDI MATEMATICI
\let\sin\relax %TOGLIE LA DEFINIZIONE SU "\sin"

% cambia la definizione di empty set
% ---
\let\oldemptyset\emptyset
% ---
% \let\emptyset\varnothing
% ---
% \let\emptyset\relax
% \newcommand{\emptyset}{\text{\textnormal{\O}}}
% ---

\DeclareMathOperator{\bounded}{bd}
\DeclareMathOperator{\sin}{sen}
\DeclareMathOperator{\epi}{Epi}
\DeclareMathOperator{\cl}{cl}
\DeclareMathOperator{\graph}{graph}
\DeclareMathOperator{\arcsec}{arcsec}
\DeclareMathOperator{\arccot}{arccot}
\DeclareMathOperator{\arccsc}{arccsc}
\DeclareMathOperator{\spettro}{Spettro}
\DeclareMathOperator{\nulls}{nullspace}
\DeclareMathOperator{\dom}{dom}
\DeclareMathOperator{\ar}{ar}
\DeclareMathOperator{\const}{Const}
\DeclareMathOperator{\fun}{Fun}
\DeclareMathOperator{\rel}{Rel}
\DeclareMathOperator{\altezza}{ht}
\let\det\relax %TOGLIE LA DEFINIZIONE SU "\det"
\DeclareMathOperator{\det}{det}
\DeclareMathOperator{\End}{End}
\DeclareMathOperator{\gl}{GL}
\def\Id{\mathrm{Id}}
\def\id{\mathrm{id}}
\DeclareMathOperator{\I}{\mathds{1}}
\DeclareMathOperator{\II}{II}
\DeclareMathOperator{\rank}{rank}
\DeclareMathOperator{\tr}{tr}
\DeclareMathOperator{\tc}{t.c.}
\DeclareMathOperator{\T}{T}
\DeclareMathOperator{\var}{Var}
\DeclareMathOperator{\cov}{Cov}
\DeclareMathOperator{\st}{st}
\DeclareMathOperator{\mon}{Mon}
\newcommand{\card}[1]{\left\vert #1 \right\vert}
\newcommand{\trasposta}[1]{\prescript{\text{T}}{}{#1}}
\newcommand{\1}{\mathds{1}}
\newcommand{\R}{\mathds{R}}
\newcommand{\diesis}{\#}
\newcommand{\bemolle}{\flat}
\newcommand{\nonstandard}[1]{\prescript{*}{}{#1}}
\newcommand{\starR}{\nonstandard{\R}}
\newcommand{\borel}{\mathscr{B}}
\newcommand{\lebesgue}[1]{\mathscr{L}\left(#1\right)}
\newcommand{\media}{\mathds{E}}
\newcommand{\K}{\mathds{K}}
\newcommand{\A}{\mathds{A}}
\newcommand{\Q}{\mathds{Q}}
\newcommand{\N}{\mathds{N}}
\newcommand{\C}{\mathds{C}}
\newcommand{\Z}{\mathds{Z}}
\newcommand{\qo}{\hspace{1em}\text{q.o.}\,}
\renewcommand{\tilde}[1]{\widetilde{#1}}
\renewcommand{\parallel}{\mathrel{/\mkern-5mu/}}
\newcommand{\parti}[2][]{\wp_{#1}(#2)}
\newcommand{\diff}[1]{\operatorname{d}_{#1}}
\let\oldvec\vec
\renewcommand{\vec}[1]{\overrightarrow{\vphantom{i}#1}}
\newcommand{\floor}[1]{\left\lfloor #1 \right\rfloor}
\newcommand{\cat}[1]{\mathbf{#1}}
\newcommand{\dfreccia}[1]{\xrightarrow{\ #1 \ }}
\newcommand{\sfreccia}[1]{\xleftarrow{\ #1 \ }}
\newcommand{\formalsum}[2]{{\sum_{#1}^{#2}}{\vphantom{\sum}}'}
\newcommand{\minim}[2]{\mu_{#1}\, \left(#2\right)}
\newcommand{\concat}{\null^{\frown}} % concatenazione di stringe
\newcommand{\godelcode}[1]{\langle\!\langle #1 \rangle\!\rangle}
\newcommand{\godeldec}[1]{(\!(#1)\!)}
\newcommand{\termcode}[1]{\ulcorner #1\urcorner}
\newcommand{\partialto}{\dashrightarrow}
\newcommand{\restricted}{\upharpoonright}
\newcommand{\embeds}{\precsim}
\newcommand{\surjects}{\twoheadrightarrow}
\newcommand{\equipotenti}{\asymp}
%% \newcommand{\dotplus}{\mathbin{\dot{+}}} %% A quanto pare esiste già
\newcommand{\bigdot}{\mathbin{\boldsymbol{\cdot}}}
\newcommand{\dotexp}[1]{^{.#1}}
\newcommand{\conv}{\mathbin{*}}
\newcommand{\convolution}[2]{(#1\conv #2)}
\newcommand{\nil}{\mathfrak{N}}
\newcommand{\divisore}{\mathrel{|}}
\newcommand{\simplesso}[1]{\mathrm{e}_{#1}}

\renewcommand{\iff}{\mathrel{\longleftrightarrow}} %% Notazione Logica.
\newcommand{\oldiff}{\mathrel{\Longleftrightarrow}}
\renewcommand{\implies}{\mathrel{\rightarrow}} %% Notazione Logica
\newcommand{\oldimplies}{\mathrel{\Longrightarrow}}
\renewcommand{\impliedby}{\mathrel{\leftarrow}} %% Notazione Logica
\newcommand{\oldimpliedby}{\mathrel{\Longleftarrow}}

\newcommand{\IFF}{\quad\Longleftrightarrow\quad}
\newcommand{\IMPLICA}{\quad\Longrightarrow\quad}


\renewcommand{\descriptionlabel}[1]{\hspace{\labelsep}\normalfont #1} % remove bold from description


%% Definizione di Divergenza di K-L

\DeclarePairedDelimiterX{\infdivx}[2]{(}{)}{%
  #1\;\delimsize\|\;#2%
}
\newcommand{\kldiv}{D_{KL}\infdivx}

%% Definizione di \dotminus

\makeatletter
\newcommand{\dotminus}{\mathbin{\text{\@dotminus}}}

\newcommand{\@dotminus}{%
  \ooalign{\hidewidth\raise1ex\hbox{.}\hidewidth\cr$\m@th-$\cr}%
}
\makeatother

%tramite i prossimi due comandi posso decidere come scrivere i logaritmi naturali in tutti i documenti: ho infatti eliminato qualsiasi differenza tra "ln" e "log": se si vuole qualcosa di diverso bisogna inserire manualmente il tutto
\let\ln\relax
\DeclareMathOperator{\ln}{ln}
\let\log\relax
\DeclareMathOperator{\log}{log}
%%%%%%

%% NUOVI COMANDI
\newcommand{\straniero}[1]{\textit{#1}} %parole straniere
\newcommand{\titolo}[1]{\textsc{#1}} %titoli
\newcommand{\qedd}{\tag*{$\blacksquare$}} %qed per ambienti matemastici
\renewcommand{\qedsymbol}{$\blacksquare$} %modifica colore qed
\newcommand{\ooverline}[1]{\overline{\overline{#1}}}
\newcommand{\circoletto}[1]{\left(#1\right)^{\text{o}}}
%
\newcommand{\qmatrice}[1]{\begin{pmatrix}
#1_{11} & \cdots & #1_{1n}\\
\vdots & \ddots & \vdots \\
#1_{m1} & \cdots & #1_{mn}
\end{pmatrix}}
%
\newcommand{\parentesi}[2]{%
\underset{#1}{\underbrace{#2}}%
}
%
\newcommand{\norma}[1]{% Norma
\left\lVert#1\right\rVert%
}
\newcommand{\scalare}[2]{% Scalare
\left\langle #1, #2\right\rangle
}
%%%%%

%% RESTRIZIONI
\newcommand{\referenze}[2]{
        \phantomsection{}#2\textsuperscript{\textcolor{blue}{\textbf{#1}}}
}

\let\restriction\relax

\def\restriction#1#2{\mathchoice
              {\setbox1\hbox{${\displaystyle #1}_{\scriptstyle #2}$}
              \restrictionaux{#1}{#2}}
              {\setbox1\hbox{${\textstyle #1}_{\scriptstyle #2}$}
              \restrictionaux{#1}{#2}}
              {\setbox1\hbox{${\scriptstyle #1}_{\scriptscriptstyle #2}$}
              \restrictionaux{#1}{#2}}
              {\setbox1\hbox{${\scriptscriptstyle #1}_{\scriptscriptstyle #2}$}
              \restrictionaux{#1}{#2}}}
\def\restrictionaux#1#2{{#1\,\smash{\vrule height .8\ht1 depth .85\dp1}}_{\,#2}}
%%%%%%%%%%%

%%% FORMATTAZIONE FOOTNOTEMARK

\def\footnotemarkformatting#1{[#1]}
\renewcommand{\thefootnote}{\footnotemarkformatting{\arabic{footnote}}}

%% SEZIONE GRAFICA
\use{tikz}
\usetikzlibrary{matrix, patterns, calc, decorations.pathreplacing, hobby, decorations.markings, decorations.pathmorphing, babel}
\use{tikz-3dplot}
\use{mathrsfs} %per geogebra
\use{tikz-cd}
\tikzset
{
  %surface/.style={fill=black!10, shading=ball,fill opacity=0.4},
  plane/.style={black,pattern=north east lines},
  curve/.style={black,line width=0.5mm},
  dritto/.style={decoration={markings,mark=at position 0.5 with {\arrow{Stealth}}}, postaction=decorate},
  rovescio/.style={decoration={markings,mark=at position 0.5 with {\arrow{Stealth[reversed]}}}, postaction=decorate}
}
\use{pgfplots} % stampare le funzioni
        \pgfplotsset{/pgf/number format/use comma,compat=1.15}
        %\pgfplotsset{compat=1.15} %per geogebra
        \usepgfplotslibrary{fillbetween, polar}
%%%%%%

%% CITAZIONI
\use{lineno}

\newcommand{\citazione}[1]{%
  \begin{quotation}
  \begin{linenumbers}
  \modulolinenumbers[5]
  \begingroup
  \setlength{\parindent}{0cm}
  \noindent #1
  \endgroup
  \end{linenumbers}
  \end{quotation}\setcounter{linenumber}{1}
  }
%%%%%%

%%%%%%%%%%%%%%%%%%%%%%%%%%%%%%%%%%%%%%%%%%%%
%%%%%%%%%%%%%%%%%%%%%%%%%%%%%%%%%%%%%%%%%%%%

%% AMS THM

\theoremstyle{definition}% default
\newtheorem{thm}{Teorema}[section]
\newtheorem{lem}[thm]{Lemma}
\newtheorem{prop}[thm]{Proposizione}
\newtheorem{cor}[thm]{Corollario}
\newtheorem{esempio}[thm]{Esempio}
\theoremstyle{plain}
\newtheorem{definizione}[thm]{Definizione}
\theoremstyle{remark}
\newtheorem*{oss}{Osservazione}


%%%%%%%%%%%%%%%%%%%%%%%%%%%%%%%%%%%%%%%%%%%%
%%%%%%%%%%%%%%%%%%%%%%%%%%%%%%%%%%%%%%%%%%%%

\use{hyperref}
\hypersetup{%
        pdfauthor={Davide Peccioli},
        pdfsubject={},
        allcolors=black,
        citecolor=black,
%	colorlinks=true,
        bookmarksopen=true}
\setcounter{secnumdepth}{0} % rimuove i numeri di sezione senza rimuovere le ref
\renewcommand{\href}[2]{\textcolor{blue}{#2}} % disabilita il comando href
\use{enotez} %
\setenotez{%
 mark-format = \footnotemarkformatting % Mette i numeri tra parentesi quadre%
}\let\footnote=\endnote % rende tutte le note a pié pagina come delle note a fine file 


\let\olddocument\document % modifico l'ambiende documenti per non dover stampare \printendnote
\let\oldenddocument\enddocument
\renewenvironment{document}%
{%
  \olddocument
}{%
  \printendnotes\oldenddocument
}
\renewcommand{\thethm}{\arabic{thm}}

\usepackage[hyperref]{biblatex}
\addbibresource{~/Documents/org/roam/bib/master.bib}
\author{Davide Peccioli}
\date{\today}
\title{}
\begin{document}

\section{Forma normale locale per superfici di Riemann}
\label{sec:orgaaa213e}
\begin{prop}
Siano \(X,Y\) \href{20260127112828-superficie_di_riemann.org}{superfici di Riemann}, \(F:X\to Y\) \href{20260126110551-funzione_olomorfa.org}{olomorfa} non costante. Sia \(p \in X\).

Allora esiste un unico \(m \in \N\), \(m\ge 1\) tale che per ogni \href{20260127112715-atlante_complesso.org}{carta locale}
\begin{equation*}
\varphi_{2}:U_{2}\to V_{2}
\end{equation*}
di \(Y\) centrata in \(F(p)\)\footnote{Ovvero \(F(p) \in U_{2}\) e \(\varphi_{2}\big(F(p)\big) = 0\).} esiste \(\varphi_{1}:U_{1}\to V_{1}\) carta locale di \(X\) centrata in \(p\)\footnote{Ovvero \(p \in U_{1}\) e \(\varphi_{1}(p) = 0\)} tale che l'espressione di \(F\) nelle carte locali sia \(z\mapsto z^{m}\), ovvero
\begin{equation*}
\varphi_{2}\circ F\circ\varphi_{1}^{-1}(z) = z^{m}.
\end{equation*}
\label{prop_formanormaleriem}
\end{prop}

\begin{proof}
Fissiamo \(\varphi_{2}:U_{2}\to V_{2}\) carta locale centrata in \(F(p)\), e scegliamo \(\psi:U\to V\) carta locale per \(X\) centrata in \(p\).

Sia \(T:V\to V_{2}\) l'espressione di \(F\) nelle carte locali \(\psi\) e \(\varphi_{2}\):
\begin{equation*}
\forall  w \in V:\qquad T(w) = \varphi_{2}\circ F\circ \psi^{-2}(w)
\end{equation*}
In particolare:
\begin{itemize}
\item \(T(0) = 0\);
\item \(T\) è \href{20260126110551-funzione_olomorfa.org}{olomorfa}
\end{itemize}

Siccome \(F\) non è costante, allora \(T\) non è costante, e \href{20260128131354-principio_di_identita_per_funzioni_olomorfe.org}{pertanto} \(w=0\) è uno \href{20250403131856-punto_isolato.org}{zero isolato} per \(T\): esiste quindi \(m \coloneqq \mathrm{ord}_{0} T\)\footnote{Questo è l'\href{20260128124105-ordine_di_una_funzione_olomorfa.org}{ordine di \(T\) in \(0\)}.} finito. In particolare, \(m\ge 1\) poiché \(T(0) = 0\).

Per il \href{20260128130443-comportamento_locale_di_una_funzione_analitica.org}{Teorema di Forma Normale nel caso di funzioni olomorfe}, esiste \(\eta: U_{0}\to U_{0}\) \href{20260127132618-funzione_biolomorfa.org}{biolomorfismo}, per \(U_{0} \subseteq \C\) intorno di \(0\), tale che
\begin{equation*}
\eta(0) = 0,\qquad T(w) = \big(\eta(w)\big)^{m}.
\end{equation*}
Sia quindi \(U_{1} \coloneqq \psi^{-1}[U_{0}] \ni p\), \(U_{1} \subseteq U\) \href{20250103145124-topologia.org}{aperto}, e poniamo
\begin{equation*}
\varphi_{1}\coloneqq \eta \circ \psi : U_{1}\to V_{1} \subseteq \C
\end{equation*}
carta locale di \(X\) centrata in \(p\). Allora,
\begin{align*}
\varphi_{2}\circ F \circ \varphi_{1}^{-1}(x) &= %
\varphi_{2}\circ F \circ \psi^{-1}\circ \eta^{-1} (z) = \\
&= (\varphi_{2}\circ F \circ \psi^{-1})\circ \eta^{-1} (z) = \\
&= T\circ \eta^{-1}(z) = \big(\eta\circ\eta^{-1}(x)\big)^{m} = z^{m}.
\end{align*}

Resta da dimostrare l'unicità di \(m\). Notiamo che per ogni aperto \(A\) di \(p \in X\) esiste \(\tilde{A} \subseteq A\) intorno aperto di \(p\) tale che, considerando\footnote{Vedi ``\href{20250202190147-immagine_punto_a_punto_di_due_classi.org}{Immagine e retroimmagine tramite una funzione}''}
\begin{equation*}
\restriction{F}{\tilde{A}}: \tilde{A} \to F[\tilde{A}]
\end{equation*}
si abbia:
\begin{enumerate}
\item per \(p\) si abbia \((\restriction{F}{\tilde{A}})^{-1} \big[F(p)\big] = \set{p}\)
\item per ogni \(q \neq F(p)\), \(q \in F[\tilde{A}]\) si abbia che \((\restriction{F}{\tilde{A}})^{-1}(q)\) ha \href{20241213101756-cardinalita.org}{cardinalità} \(m\).
\end{enumerate}

Infatti, è sufficiente prendere \(\tilde{A}\coloneqq \varphi_{1}^{-1}[U_{0}] \cap A\):
\begin{enumerate}
\item se esistesse \(p' \in \tilde{A}\) tale che \(F(p)= F(p')\), allora in particolare \(\varphi_{2}\circ F(p) = \varphi_{2}\circ F(p')\), e quindi
\begin{equation*}
 0 = \big(\varphi_{1}(p)\big)^{m} = \big(\varphi_{1}(p')\big)^{m}
\end{equation*}
e pertanto \(\varphi_{1}(p') = 0\), ovvero \(p'=p\);
\item sia \(q \neq F(p)\), \(q \in F[\tilde{A}] \subseteq U_{2}\). Allora \(0 = \varphi_{2}\circ F(p) \neq \varphi_{2}(q)\). In particolare, esistono esattamente \(m\) radici \(r_{1},\dots,r_{m} \in U_{0} \cap \varphi_{1}[\tilde{A}]\) tali che
\begin{equation*}
 (r_{i})^{m} = \varphi_{2}(q)
\end{equation*}
Poiché \(\varphi_{1}\) è omeomorfismo, ci sono esattamente \(m\) elementi dentro la fibra di \(q\).
\end{enumerate}

Quindi \(m\) non dipende dalla scelta di \(\varphi_{2}\) o dalla costruzione, ma solo da \(F\). Quindi \(m\) è unico.
\end{proof}

\begin{prop}
Per ogni intorno aperto \(U\) di \(p\) in \(X\), esiste \(U'\) intorno aperto di \(p\) in \(X\) tale che
\begin{enumerate}
\item la \href{20260128184014-fibra_di_una_funzione.org}{fibra} di \(F(p)\) in \(\tilde{U}\) contenga solo \(p\);
\item per ogni \(p \neq q \in F[\tilde{U}]\), la \href{20260128184014-fibra_di_una_funzione.org}{fibra} di \(q\) in \(\tilde{U}\) ha cardinalità \(m\).
\end{enumerate}
\end{prop}
\subsection{Molteplicità di una funzione tra superfici di Riemann}
\label{sec:org91793bc}
\begin{definizione}
Tale \(m \in \N\), \(m\ge 1\) si dice \uline{molteplicità di \(F\) in \(p\)}, e si indica con
\begin{equation*}
m \eqqcolon \mathrm{molt}_{p} F
\end{equation*}
\end{definizione}

\begin{cor}
Se esistono delle carte per cui la scrittura locale di \(F\) in un intorno di \(p\) è
\begin{equation*}
f(z) = z^{n}\cdot g(z)
\end{equation*}
con \(g(z)\) olomorfa e \(g(0) \neq 0\), allora \(n= \mathrm{molt}_{p} F\).
\end{cor}
\begin{proof}
Poiché \(g(z)\) è \href{20250103103252-funzione_continua.org}{continua} e \(g(0) \neq 0\), \href{20250306140014-funzione_continua_in_un_punto.org}{allora} esiste un \href{20250111142313-intorno.org}{intorno} \(U\) di \(0\) tale che \(g(z)\) è non nulla su \(0\). WLOG si consideri \(U\) \href{20250113100451-spazio_topologico_semplicemente_connesso.org}{semplicemente connesso}.

\href{20260126184450-funzione_olomorfa_su_un_aperto_semplicemente_connesso_ammette_radice_m_esima.org}{Allora} esiste \(h(z)\) tale che \(g(z) = \big(h(z)\big)^{n}\):
\begin{equation*}
f(z) = z^{n} \cdot g(z) = \big(z\cdot h(z)\big)^{n}
\end{equation*}
Se \(\theta(z) \coloneqq z\cdot h(z)\) è un \href{20260127132618-funzione_biolomorfa.org}{biolomorfismo} locale in \(0\), allora definisce un \href{20260127112715-atlante_complesso.org}{cambio di coordinate} su \(Y\) tale per cui la scrittura locale di \(F\) diventa \(w\mapsto w^{n}\), e per il \hyperref[sec:orgaaa213e]{Teorema di Forma Normale}, \(n= \mathrm{molt}_{p} F\).
\begin{itemize}
\item \(\theta(0) = 0\);
\item \(\theta'(z) = h(z) + z\cdot h'(z)\), e quindi \(\theta'(0) = h(0) \neq 0\).
\end{itemize}
Per il \href{20260128143021-teorema_di_inversione_locale.org}{Teorema della funzione inversa} \(\theta(z)\) è un biolomorfismo locale.
\end{proof}

\begin{oss}
Quindi si ha che:
\begin{enumerate}
\item Se \(m=1\), allora \(F\) è un \href{20260128144717-isomorfismo_tra_superfici_di_riemann.org}{biolomorfismo} locale in \(p\);
\item Se \(m>1\), allora \(F\) non è \href{20241219101956-funzione_iniettiva.org}{iniettiva} in nessun \href{20250111142313-intorno.org}{intorno} di \(p\);
\end{enumerate}
e dunque sono equivalenti\footnote{Si confronti con ``\href{20260128143427-funzione_olomorfa_iniettiva_e_biolomorfismo_locale.org}{Funzione olomorfa iniettiva è biolomorfismo locale}''}:
\begin{itemize}
\item \(\mathrm{mult}_{p} F = 1\);
\item \(F\) è un \href{20260128144717-isomorfismo_tra_superfici_di_riemann.org}{biolomorfismo} locale in \(p\);
\item \(F\) è iniettiva in un intorno di \(p\).
\end{itemize}
\end{oss}

\begin{oss}
Siccome essere biolomorfismo locale è una condizione aperta, si ha che
\begin{equation*}
\set{p \in X \mid \mathrm{mult}_{p}\, F = 1}
\end{equation*}
è un \href{20250103145124-topologia.org}{aperto} di \(X\).
\end{oss}

\begin{esempio}
Vedi un \href{20260129141700-esempio_di_calcolo_delle_moltiplicita_di_una_funzione_tra_superfici_di_riemann.org}{esempio di calcolo delle moltiplicità}.
\end{esempio}
\end{document}
