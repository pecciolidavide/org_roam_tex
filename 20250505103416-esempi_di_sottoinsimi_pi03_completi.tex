% Created 2026-02-07 Sat 19:30
% Intended LaTeX compiler: pdflatex
\documentclass[10pt]{article}
%% CREATO CON ORG - EMACS
\newcommand{\use}[2][]{\usepackage[#1]{#2}}
% PACCHETTI FONDAMENTLAI
\use[utf8]{inputenc}
\use[T1]{fontenc}
\use{graphicx}
\use{longtable}
\use{wrapfig}
\use{rotating}
\use[normalem]{ulem}
\use{amsmath}
\use{amsthm}
\use{amssymb}

\use{eucal} % Cambia mathcal{...}

\use{capt-of}
\use[italian]{babel}
\use[babel]{csquotes}
% bib la TEX lo carica in automatico org-cite
\use{microtype}
\use{lmodern}
\use{subfig} % sottofigure
\use{multicol} % due colonne
\use{lipsum} % lorem ipsum
\use{color} % colori in latex
\use{parskip} % rimuove l'indentazione dei nuovi paragrafi %% Add parbox=false to all new tcolorbox
\use{centernot}
\use[outline]{contour}\contourlength{3pt}
\use{fancyhdr}
\use{layout}
\use[most]{tcolorbox} % Riquadri colorati
\use{ifthen} % IFTHEN
\use{geometry}

% pacchetti matematica
\use{yhmath}
\use{dsfont}
\use{mathrsfs}
\use{cancel} % semplificare
\use{polynom} %divisione tra polinomi
\use{forest} % grafi ad albero
\use{booktabs} % tabelle
\use{commath} %simboli e differenziali
\use{bm} %bold
\use[fulladjust]{marginnote} %to use marginnote for date notes
\use{arrayjobx}%array
\use[intlimits]{empheq} % Riquadri colorati attorno alle equazioni
\use{mathtools}
\use{circuitikz} % Disegnare i circuiti
\use{mathtools}
\use{stmaryrd} % [[ \llbracket ]] \rrbracket
\use{bussproofs} % dimostrazioni

%%%%%%%%%%%%%


%%%% QUIVER
\newcommand{\duepunti}{\,\mathchar\numexpr"6000+`:\relax\,}
% A TikZ style for curved arrows of a fixed height, due to AndréC.
\tikzset{curve/.style={settings={#1},to path={(\tikztostart)
    .. controls ($(\tikztostart)!\pv{pos}!(\tikztotarget)!\pv{height}!270:(\tikztotarget)$)
    and ($(\tikztostart)!1-\pv{pos}!(\tikztotarget)!\pv{height}!270:(\tikztotarget)$)
    .. (\tikztotarget)\tikztonodes}},
    settings/.code={\tikzset{quiver/.cd,#1}
        \def\pv##1{\pgfkeysvalueof{/tikz/quiver/##1}}},
    quiver/.cd,pos/.initial=0.35,height/.initial=0}

% TikZ arrowhead/tail styles.
\tikzset{tail reversed/.code={\pgfsetarrowsstart{tikzcd to}}}
\tikzset{2tail/.code={\pgfsetarrowsstart{Implies[reversed]}}}
\tikzset{2tail reversed/.code={\pgfsetarrowsstart{Implies}}}
% TikZ arrow styles.
\tikzset{no body/.style={/tikz/dash pattern=on 0 off 1mm}}
%%%%%%%%%%


%% DEFINIZIONI COMANDI MATEMATICI
\let\sin\relax %TOGLIE LA DEFINIZIONE SU "\sin"

% cambia la definizione di empty set
% ---
\let\oldemptyset\emptyset
% ---
% \let\emptyset\varnothing
% ---
% \let\emptyset\relax
% \newcommand{\emptyset}{\text{\textnormal{\O}}}
% ---

\DeclareMathOperator{\bounded}{bd}
\DeclareMathOperator{\sin}{sen}
\DeclareMathOperator{\epi}{Epi}
\DeclareMathOperator{\cl}{cl}
\DeclareMathOperator{\graph}{graph}
\DeclareMathOperator{\arcsec}{arcsec}
\DeclareMathOperator{\arccot}{arccot}
\DeclareMathOperator{\arccsc}{arccsc}
\DeclareMathOperator{\spettro}{Spettro}
\DeclareMathOperator{\nulls}{nullspace}
\DeclareMathOperator{\dom}{dom}
\DeclareMathOperator{\ar}{ar}
\DeclareMathOperator{\const}{Const}
\DeclareMathOperator{\fun}{Fun}
\DeclareMathOperator{\rel}{Rel}
\DeclareMathOperator{\altezza}{ht}
\let\det\relax %TOGLIE LA DEFINIZIONE SU "\det"
\DeclareMathOperator{\det}{det}
\DeclareMathOperator{\End}{End}
\DeclareMathOperator{\gl}{GL}
\def\Id{\mathrm{Id}}
\def\id{\mathrm{id}}
\DeclareMathOperator{\I}{\mathds{1}}
\DeclareMathOperator{\II}{II}
\DeclareMathOperator{\rank}{rank}
\DeclareMathOperator{\tr}{tr}
\DeclareMathOperator{\tc}{t.c.}
\DeclareMathOperator{\T}{T}
\DeclareMathOperator{\var}{Var}
\DeclareMathOperator{\cov}{Cov}
\DeclareMathOperator{\st}{st}
\DeclareMathOperator{\mon}{Mon}
\newcommand{\card}[1]{\left\vert #1 \right\vert}
\newcommand{\trasposta}[1]{\prescript{\text{T}}{}{#1}}
\newcommand{\1}{\mathds{1}}
\newcommand{\R}{\mathds{R}}
\newcommand{\diesis}{\#}
\newcommand{\bemolle}{\flat}
\newcommand{\nonstandard}[1]{\prescript{*}{}{#1}}
\newcommand{\starR}{\nonstandard{\R}}
\newcommand{\borel}{\mathscr{B}}
\newcommand{\lebesgue}[1]{\mathscr{L}\left(#1\right)}
\newcommand{\media}{\mathds{E}}
\newcommand{\K}{\mathds{K}}
\newcommand{\A}{\mathds{A}}
\newcommand{\Q}{\mathds{Q}}
\newcommand{\N}{\mathds{N}}
\newcommand{\C}{\mathds{C}}
\newcommand{\Z}{\mathds{Z}}
\newcommand{\qo}{\hspace{1em}\text{q.o.}\,}
\renewcommand{\tilde}[1]{\widetilde{#1}}
\renewcommand{\parallel}{\mathrel{/\mkern-5mu/}}
\newcommand{\parti}[2][]{\wp_{#1}(#2)}
\newcommand{\diff}[1]{\operatorname{d}_{#1}}
\let\oldvec\vec
\renewcommand{\vec}[1]{\overrightarrow{\vphantom{i}#1}}
\newcommand{\floor}[1]{\left\lfloor #1 \right\rfloor}
\newcommand{\cat}[1]{\mathbf{#1}}
\newcommand{\dfreccia}[1]{\xrightarrow{\ #1 \ }}
\newcommand{\sfreccia}[1]{\xleftarrow{\ #1 \ }}
\newcommand{\formalsum}[2]{{\sum_{#1}^{#2}}{\vphantom{\sum}}'}
\newcommand{\minim}[2]{\mu_{#1}\, \left(#2\right)}
\newcommand{\concat}{\null^{\frown}} % concatenazione di stringe
\newcommand{\godelcode}[1]{\langle\!\langle #1 \rangle\!\rangle}
\newcommand{\godeldec}[1]{(\!(#1)\!)}
\newcommand{\termcode}[1]{\ulcorner #1\urcorner}
\newcommand{\partialto}{\dashrightarrow}
\newcommand{\restricted}{\upharpoonright}
\newcommand{\embeds}{\precsim}
\newcommand{\surjects}{\twoheadrightarrow}
\newcommand{\equipotenti}{\asymp}
%% \newcommand{\dotplus}{\mathbin{\dot{+}}} %% A quanto pare esiste già
\newcommand{\bigdot}{\mathbin{\boldsymbol{\cdot}}}
\newcommand{\dotexp}[1]{^{.#1}}
\newcommand{\conv}{\mathbin{*}}
\newcommand{\convolution}[2]{(#1\conv #2)}
\newcommand{\nil}{\mathfrak{N}}
\newcommand{\divisore}{\mathrel{|}}
\newcommand{\simplesso}[1]{\mathrm{e}_{#1}}

\renewcommand{\iff}{\mathrel{\longleftrightarrow}} %% Notazione Logica.
\newcommand{\oldiff}{\mathrel{\Longleftrightarrow}}
\renewcommand{\implies}{\mathrel{\rightarrow}} %% Notazione Logica
\newcommand{\oldimplies}{\mathrel{\Longrightarrow}}
\renewcommand{\impliedby}{\mathrel{\leftarrow}} %% Notazione Logica
\newcommand{\oldimpliedby}{\mathrel{\Longleftarrow}}

\newcommand{\IFF}{\quad\Longleftrightarrow\quad}
\newcommand{\IMPLICA}{\quad\Longrightarrow\quad}


\renewcommand{\descriptionlabel}[1]{\hspace{\labelsep}\normalfont #1} % remove bold from description


%% Definizione di Divergenza di K-L

\DeclarePairedDelimiterX{\infdivx}[2]{(}{)}{%
  #1\;\delimsize\|\;#2%
}
\newcommand{\kldiv}{D_{KL}\infdivx}

%% Definizione di \dotminus

\makeatletter
\newcommand{\dotminus}{\mathbin{\text{\@dotminus}}}

\newcommand{\@dotminus}{%
  \ooalign{\hidewidth\raise1ex\hbox{.}\hidewidth\cr$\m@th-$\cr}%
}
\makeatother

%tramite i prossimi due comandi posso decidere come scrivere i logaritmi naturali in tutti i documenti: ho infatti eliminato qualsiasi differenza tra "ln" e "log": se si vuole qualcosa di diverso bisogna inserire manualmente il tutto
\let\ln\relax
\DeclareMathOperator{\ln}{ln}
\let\log\relax
\DeclareMathOperator{\log}{log}
%%%%%%

%% NUOVI COMANDI
\newcommand{\straniero}[1]{\textit{#1}} %parole straniere
\newcommand{\titolo}[1]{\textsc{#1}} %titoli
\newcommand{\qedd}{\tag*{$\blacksquare$}} %qed per ambienti matemastici
\renewcommand{\qedsymbol}{$\blacksquare$} %modifica colore qed
\newcommand{\ooverline}[1]{\overline{\overline{#1}}}
\newcommand{\circoletto}[1]{\left(#1\right)^{\text{o}}}
%
\newcommand{\qmatrice}[1]{\begin{pmatrix}
#1_{11} & \cdots & #1_{1n}\\
\vdots & \ddots & \vdots \\
#1_{m1} & \cdots & #1_{mn}
\end{pmatrix}}
%
\newcommand{\parentesi}[2]{%
\underset{#1}{\underbrace{#2}}%
}
%
\newcommand{\norma}[1]{% Norma
\left\lVert#1\right\rVert%
}
\newcommand{\scalare}[2]{% Scalare
\left\langle #1, #2\right\rangle
}
%%%%%

%% RESTRIZIONI
\newcommand{\referenze}[2]{
        \phantomsection{}#2\textsuperscript{\textcolor{blue}{\textbf{#1}}}
}

\let\restriction\relax

\def\restriction#1#2{\mathchoice
              {\setbox1\hbox{${\displaystyle #1}_{\scriptstyle #2}$}
              \restrictionaux{#1}{#2}}
              {\setbox1\hbox{${\textstyle #1}_{\scriptstyle #2}$}
              \restrictionaux{#1}{#2}}
              {\setbox1\hbox{${\scriptstyle #1}_{\scriptscriptstyle #2}$}
              \restrictionaux{#1}{#2}}
              {\setbox1\hbox{${\scriptscriptstyle #1}_{\scriptscriptstyle #2}$}
              \restrictionaux{#1}{#2}}}
\def\restrictionaux#1#2{{#1\,\smash{\vrule height .8\ht1 depth .85\dp1}}_{\,#2}}
%%%%%%%%%%%

%%% FORMATTAZIONE FOOTNOTEMARK

\def\footnotemarkformatting#1{[#1]}
\renewcommand{\thefootnote}{\footnotemarkformatting{\arabic{footnote}}}

%% SEZIONE GRAFICA
\use{tikz}
\usetikzlibrary{matrix, patterns, calc, decorations.pathreplacing, hobby, decorations.markings, decorations.pathmorphing, babel}
\use{tikz-3dplot}
\use{mathrsfs} %per geogebra
\use{tikz-cd}
\tikzset
{
  %surface/.style={fill=black!10, shading=ball,fill opacity=0.4},
  plane/.style={black,pattern=north east lines},
  curve/.style={black,line width=0.5mm},
  dritto/.style={decoration={markings,mark=at position 0.5 with {\arrow{Stealth}}}, postaction=decorate},
  rovescio/.style={decoration={markings,mark=at position 0.5 with {\arrow{Stealth[reversed]}}}, postaction=decorate}
}
\use{pgfplots} % stampare le funzioni
        \pgfplotsset{/pgf/number format/use comma,compat=1.15}
        %\pgfplotsset{compat=1.15} %per geogebra
        \usepgfplotslibrary{fillbetween, polar}
%%%%%%

%% CITAZIONI
\use{lineno}

\newcommand{\citazione}[1]{%
  \begin{quotation}
  \begin{linenumbers}
  \modulolinenumbers[5]
  \begingroup
  \setlength{\parindent}{0cm}
  \noindent #1
  \endgroup
  \end{linenumbers}
  \end{quotation}\setcounter{linenumber}{1}
  }
%%%%%%

%%%%%%%%%%%%%%%%%%%%%%%%%%%%%%%%%%%%%%%%%%%%
%%%%%%%%%%%%%%%%%%%%%%%%%%%%%%%%%%%%%%%%%%%%

%% AMS THM

\theoremstyle{definition}% default
\newtheorem{thm}{Teorema}[section]
\newtheorem{lem}[thm]{Lemma}
\newtheorem{prop}[thm]{Proposizione}
\newtheorem{cor}[thm]{Corollario}
\newtheorem{esempio}[thm]{Esempio}
\theoremstyle{plain}
\newtheorem{definizione}[thm]{Definizione}
\theoremstyle{remark}
\newtheorem*{oss}{Osservazione}


%%%%%%%%%%%%%%%%%%%%%%%%%%%%%%%%%%%%%%%%%%%%
%%%%%%%%%%%%%%%%%%%%%%%%%%%%%%%%%%%%%%%%%%%%

\use{hyperref}
\hypersetup{%
        pdfauthor={Davide Peccioli},
        pdfsubject={},
        allcolors=black,
        citecolor=black,
%	colorlinks=true,
        bookmarksopen=true}
\setcounter{secnumdepth}{0} % rimuove i numeri di sezione senza rimuovere le ref
\renewcommand{\href}[2]{\textcolor{blue}{#2}} % disabilita il comando href
\use{enotez} %
\setenotez{%
 mark-format = \footnotemarkformatting % Mette i numeri tra parentesi quadre%
}\let\footnote=\endnote % rende tutte le note a pié pagina come delle note a fine file 


\let\olddocument\document % modifico l'ambiende documenti per non dover stampare \printendnote
\let\oldenddocument\enddocument
\renewenvironment{document}%
{%
  \olddocument
}{%
  \printendnotes\oldenddocument
}
\renewcommand{\thethm}{\arabic{thm}}

\usepackage[hyperref]{biblatex}
\addbibresource{~/Documents/org/roam/bib/master.bib}
\author{Davide Peccioli}
\date{\today}
\title{Esempi di sottoinsiemi pi03 completi}
\begin{document}

\section{Esercizio 3}
\label{sec:orgb39e495}

Prove that the sets
\begin{align*}
C_0 &= c_0 \cap [0,1]^\omega = \left\{(x_n)_{n \in \omega} \in [0,1]^\omega \,\middle|\, x_n \to 0 \right\}\\
C &= \left\{(x_n)_{n \in \omega} \in [0,1]^\omega \,\middle|\, (x_n)_{n \in \omega} \text{ converges} \right\}
\end{align*}
are both \(\bm{\Pi}^0_3\)-complete.

\emph{Hint.} For the hardness part, compare these sets with the \(\bm{\Pi}^0_3\)-complete set \(C_3\) from Exercise 2.1.27 in the notes.
\subsection{Soluzione}
\label{sec:org0d4ee7c}

\subsubsection{\(C_{0}\) e \(C\) sono degli insiemi \(\mathbf{\Pi}_{0}^{3}\).}
\label{sec:org641e00a}

\paragraph{Insieme \(C_{0}\).}
\label{sec:orga9a86ac}

Si ha che \((x_{j})_{j \in \omega} \in C_{0}\) se e solo se \((x_{j})_{j \in \omega} \in [0,1]^{\omega}\) e:
\begin{equation*}
	\forall\, \varepsilon \in \Q^{+}\ \exists\,N \in \N \ \forall\, n > N\ \left( |x_{n}|\le\varepsilon\right)
\end{equation*}
ovvero, se \(U_{n, \varepsilon} \coloneqq \set{(x_{j})_{j \in \omega} \in [0,1]^{\omega}: |x_{n}|\le\varepsilon}\), allora
\begin{equation*}
	C_{0} = \bigcap_{\varepsilon \in \Q^{+}} \bigcup_{N \in \N} \bigcap_{n>N} U_{n,\varepsilon}.
\end{equation*}

Quindi, dette \(\pi_{m} : [0,1]^{\omega}\to [0,1]\) le \(m\)-esime proiezioni (continue per definizione di topologia prodotto):
\begin{equation*}
	U_{n,\varepsilon}= \pi_{n}^{-1}\left([-\varepsilon,\varepsilon]\right)
\end{equation*}
e pertanto \(U_{n,\varepsilon}\) è chiuso. Per il Lemma 2.1.5:
\begin{align*}
	\bigcap_{n > N} U_{n,\varepsilon} &\in \bm{\Pi}_{1}^{0}\\
	\bigcup_{N \in \N}\bigcap_{n >N} U_{n,\varepsilon} &\in \bm{\Sigma}_{2}^{0}\\
	C_{0} = \bigcap_{\varepsilon \in \Q^{+}} \bigcup_{N \in \N} \bigcap_{n>N} U_{n,\varepsilon} &\in \bm{\Pi}_{3}^{0}.
\end{align*}
e si ottiene che \(C_{0} \in \bm{\Pi}_{3}^{0}\left([0,1]^{\omega}\right)\).
\paragraph{Insieme \(C\).}
\label{sec:org4cc935d}

Si ha che \((x_{j})_{j \in\omega} \in C\) se e solo se \((x_{j})_{j \in\omega} \in [0,1]^{\omega}\) e
\begin{equation*}
\forall\, \varepsilon \in \Q^{+}\ \exists\, N \in \N \ \forall\,n,m> N\ (|x_{n}-x_{m}|\le\varepsilon)
\end{equation*}
ovvero, se \(V_{m,n}^{\varepsilon} \coloneqq \set{(x_{j})_{j \in \omega} \in [0,1]^{\omega}: |x_{n}-x_{m}|\le\varepsilon}}\), allora
\begin{equation*}
C = \bigcap_{\varepsilon \in \Q^{+}}\bigcup_{N \in \N}\bigcap_{n,m>N} V_{n,m}^{\varepsilon}.
\end{equation*}

Poiché la funzione \((\pi_{n}-\pi_{m}):[0,1]^{\omega}\to \R\) è continua, allora
\begin{equation*}
V_{n,m}^{\varepsilon} \coloneqq (\pi_{n}-\pi_{m})^{-1}\left([-\varepsilon,\varepsilon]\right)
\end{equation*}
e quindi \(V_{n,m}^{\varepsilon}\) è chiuso. Per il Lemma 2.1.5:
\begin{align*}
\bigcap_{n,m > N} V_{n,m}^{\varepsilon} &\in \bm{\Pi}_{1}^{0}\\
\bigcup_{N \in \N}\bigcap_{n,m > N} V_{n,m}^{\varepsilon} &\in \bm{\Sigma}_{2}^{0}\\
C = \bigcap_{\varepsilon \in \Q^{+}}\bigcup_{N \in \N}\bigcap_{n,m>N} V_{n,m}^{\varepsilon} &\in \bm{\Pi}_{3}^{0}
\end{align*}
e si ottiene che \(C \in \bm{\Pi}_{3}^{0}\left([0,1]^{\omega}\right)\).
\subsubsection{Hardness}
\label{sec:org7e7a528}

È noto (Esercizio 2.1.27) che l'insieme \(C_{3} \coloneqq \set{x \in \omega^{\omega}\mid \lim_{n\to\infty}x(n) = \infty}\) sia \(\bm{\Pi}_{3}^{0}\)-hard. Pertanto si cercano delle funzioni continue
\begin{equation*}
\begin{tikzcd}[ampersand replacement=\&,cramped]
	{\omega^\omega} \& {[0,1]^\omega} \& {[0,1]^\omega}
	\arrow["F", from=1-1, to=1-2]
	\arrow["G", from=1-2, to=1-3]
\end{tikzcd}
\end{equation*}
tali che
\begin{equation*}
F^{-1}(C_{0}) = C_{3},\qquad G^{-1}(C) = C_{0}.
\end{equation*}
Questo, per mezzo del Lemma 2.1.23, garantisce che \(C_{0},C\) siano insiemi \(\bm{\Pi}_{3}^{0}\)-hard (e quindi, per il punto precedente, completi).

Le due funzioni si definiscono come segue:
\begin{align*}
&\begin{aligned}
F: \omega^{\omega} &\longrightarrow[0,1]^{\omega}\\
(x_{j})_{j \in \omega} &\longmapsto \left(\phi(x_{j})\right)_{j \in \omega}
\end{aligned} & &\text{dove} & &\begin{aligned}
\phi: \N &\longrightarrow [0,1]\\
m &\longmapsto \begin{cases}
1/m & m\neq 0\\
1 & m=0.
\end{cases}
\end{aligned}\\[1.5em]
&\begin{aligned}
G: [0,1]^{\omega} &\longrightarrow [0,1]^{\omega}\\
(x_{j})_{j \in \omega} &\longmapsto (y_{j})_{j \in \omega}
\end{aligned} & &\text{dove} &
&y_{j} \coloneqq \begin{cases}
0 & j\text{ dispari}\\
x_{j/2} & j\text{ pari}.
\end{cases}
\end{align*}
\paragraph{\underline{\(F\) è continua.}}
\label{sec:org58f8bce}

La funzione \(F\) è continua poiché lo è su ciascuna componente (in quanto \(\N\) ha la topologia discreta).
\paragraph{\underline{\(G\) è continua.}}
\label{sec:orga06a756}

La funzione \(F\) è continua poiché lo è su ciascuna componente:
\begin{itemize}
\item la componente \(j\)-esima di \(G\), con \(j\) dispari, è data dalla funzione costante nulla, continua;
\item la componente \(j\)-esima di \(G\), con \(j\) pari, è data dalla funzione proiezione \(\pi_{j/2}: [0,1]^{\omega}\to [0,1]\), continua per definizione di topologia prodotto.
\end{itemize}
\paragraph{\underline{\(F^{-1}(C_{0})=C_{3}\).}}
\label{sec:org6b1c9a5}

Si dimostra che \(\alpha \in C_{3}\) sse \(F(\alpha) \in C_{0}\).

\begin{itemize}
\item Se \(\alpha = (x_{j})_{j \in \omega}\in C_{3}\) allora esiste \(N \in \N\) tale che, per ogni \(j>N\) si ha\(x_{j}\neq 1\).

Pertanto, per ogni \(j>N\), \(\phi(x_{j}) = 1/x_{j}\) e, siccome \(x_{j}\to \infty\), \(\phi(x_{j})\to 0\). Quindi \(F(\alpha) \in C_{0}\).
\item Viceversa, sia \(\alpha = (x_{j})_{j \in \omega} \notin C_{3}\). Si supponga per assurdo che \((y_{j})_{j \in \omega} = F(\alpha) \in C_{0}\).

Allora, definitivamente, \(y_{j} = 1/x_{j}\) (e in particolare \(x_{j}\neq 0\neq y_{j}\)), poiché altrimenti non si avrebbe convergenza a \(0\). In particolare, \(x_{j} = 1/y_{j}\), definitivamente:
\begin{equation*}
  \lim_{j\to \infty} x_{j} = \lim_{j\to\infty}\frac{1}{y_{j}} = \infty
\end{equation*}
poiché \(y_{j}\to 0\). Quindi \((x_{j})_{j \in \omega} \in C_{3}\). Assurdo.

Si ottiene perciò che \(F(\alpha) \notin C_{0}\).
\end{itemize}
\paragraph{\underline{\(G^{-1}(C)=C_{0}\).}}
\label{sec:org684be79}

Si dimostra che \(\alpha \in C_{0}\) sse \(G(\alpha) \in C\).

\begin{itemize}
\item Se \(\alpha = (x_{j})_{j \in \omega} \in C_{0}\) allora la successione \(\beta= (y_{j})_{j \in \omega} \coloneqq G(\alpha)\) converge a \(0\), e pertanto converge: \(G(\alpha) \in C\).
\item Viceversa, se \(\alpha = (x_{j})_{j \in \omega}\notin C_{0}\) , allora la successione \(\beta= (y_{j})_{j \in \omega} \coloneqq G(\alpha)\) non converge, in quanto presenta due sottosuccessioni (\((y_{2j+1})_{j \in \omega}\) e \((y_{2j})_{j \in \omega}\)) con caratteri diversi: \(G(\alpha)\notin C\).\qed
\end{itemize}
\end{document}
