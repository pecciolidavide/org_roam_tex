% Created 2026-02-07 Sat 19:30
% Intended LaTeX compiler: pdflatex
\documentclass[10pt]{article}
%% CREATO CON ORG - EMACS
\newcommand{\use}[2][]{\usepackage[#1]{#2}}
% PACCHETTI FONDAMENTLAI
\use[utf8]{inputenc}
\use[T1]{fontenc}
\use{graphicx}
\use{longtable}
\use{wrapfig}
\use{rotating}
\use[normalem]{ulem}
\use{amsmath}
\use{amsthm}
\use{amssymb}

\use{eucal} % Cambia mathcal{...}

\use{capt-of}
\use[italian]{babel}
\use[babel]{csquotes}
% bib la TEX lo carica in automatico org-cite
\use{microtype}
\use{lmodern}
\use{subfig} % sottofigure
\use{multicol} % due colonne
\use{lipsum} % lorem ipsum
\use{color} % colori in latex
\use{parskip} % rimuove l'indentazione dei nuovi paragrafi %% Add parbox=false to all new tcolorbox
\use{centernot}
\use[outline]{contour}\contourlength{3pt}
\use{fancyhdr}
\use{layout}
\use[most]{tcolorbox} % Riquadri colorati
\use{ifthen} % IFTHEN
\use{geometry}

% pacchetti matematica
\use{yhmath}
\use{dsfont}
\use{mathrsfs}
\use{cancel} % semplificare
\use{polynom} %divisione tra polinomi
\use{forest} % grafi ad albero
\use{booktabs} % tabelle
\use{commath} %simboli e differenziali
\use{bm} %bold
\use[fulladjust]{marginnote} %to use marginnote for date notes
\use{arrayjobx}%array
\use[intlimits]{empheq} % Riquadri colorati attorno alle equazioni
\use{mathtools}
\use{circuitikz} % Disegnare i circuiti
\use{mathtools}
\use{stmaryrd} % [[ \llbracket ]] \rrbracket
\use{bussproofs} % dimostrazioni

%%%%%%%%%%%%%


%%%% QUIVER
\newcommand{\duepunti}{\,\mathchar\numexpr"6000+`:\relax\,}
% A TikZ style for curved arrows of a fixed height, due to AndréC.
\tikzset{curve/.style={settings={#1},to path={(\tikztostart)
    .. controls ($(\tikztostart)!\pv{pos}!(\tikztotarget)!\pv{height}!270:(\tikztotarget)$)
    and ($(\tikztostart)!1-\pv{pos}!(\tikztotarget)!\pv{height}!270:(\tikztotarget)$)
    .. (\tikztotarget)\tikztonodes}},
    settings/.code={\tikzset{quiver/.cd,#1}
        \def\pv##1{\pgfkeysvalueof{/tikz/quiver/##1}}},
    quiver/.cd,pos/.initial=0.35,height/.initial=0}

% TikZ arrowhead/tail styles.
\tikzset{tail reversed/.code={\pgfsetarrowsstart{tikzcd to}}}
\tikzset{2tail/.code={\pgfsetarrowsstart{Implies[reversed]}}}
\tikzset{2tail reversed/.code={\pgfsetarrowsstart{Implies}}}
% TikZ arrow styles.
\tikzset{no body/.style={/tikz/dash pattern=on 0 off 1mm}}
%%%%%%%%%%


%% DEFINIZIONI COMANDI MATEMATICI
\let\sin\relax %TOGLIE LA DEFINIZIONE SU "\sin"

% cambia la definizione di empty set
% ---
\let\oldemptyset\emptyset
% ---
% \let\emptyset\varnothing
% ---
% \let\emptyset\relax
% \newcommand{\emptyset}{\text{\textnormal{\O}}}
% ---

\DeclareMathOperator{\bounded}{bd}
\DeclareMathOperator{\sin}{sen}
\DeclareMathOperator{\epi}{Epi}
\DeclareMathOperator{\cl}{cl}
\DeclareMathOperator{\graph}{graph}
\DeclareMathOperator{\arcsec}{arcsec}
\DeclareMathOperator{\arccot}{arccot}
\DeclareMathOperator{\arccsc}{arccsc}
\DeclareMathOperator{\spettro}{Spettro}
\DeclareMathOperator{\nulls}{nullspace}
\DeclareMathOperator{\dom}{dom}
\DeclareMathOperator{\ar}{ar}
\DeclareMathOperator{\const}{Const}
\DeclareMathOperator{\fun}{Fun}
\DeclareMathOperator{\rel}{Rel}
\DeclareMathOperator{\altezza}{ht}
\let\det\relax %TOGLIE LA DEFINIZIONE SU "\det"
\DeclareMathOperator{\det}{det}
\DeclareMathOperator{\End}{End}
\DeclareMathOperator{\gl}{GL}
\def\Id{\mathrm{Id}}
\def\id{\mathrm{id}}
\DeclareMathOperator{\I}{\mathds{1}}
\DeclareMathOperator{\II}{II}
\DeclareMathOperator{\rank}{rank}
\DeclareMathOperator{\tr}{tr}
\DeclareMathOperator{\tc}{t.c.}
\DeclareMathOperator{\T}{T}
\DeclareMathOperator{\var}{Var}
\DeclareMathOperator{\cov}{Cov}
\DeclareMathOperator{\st}{st}
\DeclareMathOperator{\mon}{Mon}
\newcommand{\card}[1]{\left\vert #1 \right\vert}
\newcommand{\trasposta}[1]{\prescript{\text{T}}{}{#1}}
\newcommand{\1}{\mathds{1}}
\newcommand{\R}{\mathds{R}}
\newcommand{\diesis}{\#}
\newcommand{\bemolle}{\flat}
\newcommand{\nonstandard}[1]{\prescript{*}{}{#1}}
\newcommand{\starR}{\nonstandard{\R}}
\newcommand{\borel}{\mathscr{B}}
\newcommand{\lebesgue}[1]{\mathscr{L}\left(#1\right)}
\newcommand{\media}{\mathds{E}}
\newcommand{\K}{\mathds{K}}
\newcommand{\A}{\mathds{A}}
\newcommand{\Q}{\mathds{Q}}
\newcommand{\N}{\mathds{N}}
\newcommand{\C}{\mathds{C}}
\newcommand{\Z}{\mathds{Z}}
\newcommand{\qo}{\hspace{1em}\text{q.o.}\,}
\renewcommand{\tilde}[1]{\widetilde{#1}}
\renewcommand{\parallel}{\mathrel{/\mkern-5mu/}}
\newcommand{\parti}[2][]{\wp_{#1}(#2)}
\newcommand{\diff}[1]{\operatorname{d}_{#1}}
\let\oldvec\vec
\renewcommand{\vec}[1]{\overrightarrow{\vphantom{i}#1}}
\newcommand{\floor}[1]{\left\lfloor #1 \right\rfloor}
\newcommand{\cat}[1]{\mathbf{#1}}
\newcommand{\dfreccia}[1]{\xrightarrow{\ #1 \ }}
\newcommand{\sfreccia}[1]{\xleftarrow{\ #1 \ }}
\newcommand{\formalsum}[2]{{\sum_{#1}^{#2}}{\vphantom{\sum}}'}
\newcommand{\minim}[2]{\mu_{#1}\, \left(#2\right)}
\newcommand{\concat}{\null^{\frown}} % concatenazione di stringe
\newcommand{\godelcode}[1]{\langle\!\langle #1 \rangle\!\rangle}
\newcommand{\godeldec}[1]{(\!(#1)\!)}
\newcommand{\termcode}[1]{\ulcorner #1\urcorner}
\newcommand{\partialto}{\dashrightarrow}
\newcommand{\restricted}{\upharpoonright}
\newcommand{\embeds}{\precsim}
\newcommand{\surjects}{\twoheadrightarrow}
\newcommand{\equipotenti}{\asymp}
%% \newcommand{\dotplus}{\mathbin{\dot{+}}} %% A quanto pare esiste già
\newcommand{\bigdot}{\mathbin{\boldsymbol{\cdot}}}
\newcommand{\dotexp}[1]{^{.#1}}
\newcommand{\conv}{\mathbin{*}}
\newcommand{\convolution}[2]{(#1\conv #2)}
\newcommand{\nil}{\mathfrak{N}}
\newcommand{\divisore}{\mathrel{|}}
\newcommand{\simplesso}[1]{\mathrm{e}_{#1}}

\renewcommand{\iff}{\mathrel{\longleftrightarrow}} %% Notazione Logica.
\newcommand{\oldiff}{\mathrel{\Longleftrightarrow}}
\renewcommand{\implies}{\mathrel{\rightarrow}} %% Notazione Logica
\newcommand{\oldimplies}{\mathrel{\Longrightarrow}}
\renewcommand{\impliedby}{\mathrel{\leftarrow}} %% Notazione Logica
\newcommand{\oldimpliedby}{\mathrel{\Longleftarrow}}

\newcommand{\IFF}{\quad\Longleftrightarrow\quad}
\newcommand{\IMPLICA}{\quad\Longrightarrow\quad}


\renewcommand{\descriptionlabel}[1]{\hspace{\labelsep}\normalfont #1} % remove bold from description


%% Definizione di Divergenza di K-L

\DeclarePairedDelimiterX{\infdivx}[2]{(}{)}{%
  #1\;\delimsize\|\;#2%
}
\newcommand{\kldiv}{D_{KL}\infdivx}

%% Definizione di \dotminus

\makeatletter
\newcommand{\dotminus}{\mathbin{\text{\@dotminus}}}

\newcommand{\@dotminus}{%
  \ooalign{\hidewidth\raise1ex\hbox{.}\hidewidth\cr$\m@th-$\cr}%
}
\makeatother

%tramite i prossimi due comandi posso decidere come scrivere i logaritmi naturali in tutti i documenti: ho infatti eliminato qualsiasi differenza tra "ln" e "log": se si vuole qualcosa di diverso bisogna inserire manualmente il tutto
\let\ln\relax
\DeclareMathOperator{\ln}{ln}
\let\log\relax
\DeclareMathOperator{\log}{log}
%%%%%%

%% NUOVI COMANDI
\newcommand{\straniero}[1]{\textit{#1}} %parole straniere
\newcommand{\titolo}[1]{\textsc{#1}} %titoli
\newcommand{\qedd}{\tag*{$\blacksquare$}} %qed per ambienti matemastici
\renewcommand{\qedsymbol}{$\blacksquare$} %modifica colore qed
\newcommand{\ooverline}[1]{\overline{\overline{#1}}}
\newcommand{\circoletto}[1]{\left(#1\right)^{\text{o}}}
%
\newcommand{\qmatrice}[1]{\begin{pmatrix}
#1_{11} & \cdots & #1_{1n}\\
\vdots & \ddots & \vdots \\
#1_{m1} & \cdots & #1_{mn}
\end{pmatrix}}
%
\newcommand{\parentesi}[2]{%
\underset{#1}{\underbrace{#2}}%
}
%
\newcommand{\norma}[1]{% Norma
\left\lVert#1\right\rVert%
}
\newcommand{\scalare}[2]{% Scalare
\left\langle #1, #2\right\rangle
}
%%%%%

%% RESTRIZIONI
\newcommand{\referenze}[2]{
        \phantomsection{}#2\textsuperscript{\textcolor{blue}{\textbf{#1}}}
}

\let\restriction\relax

\def\restriction#1#2{\mathchoice
              {\setbox1\hbox{${\displaystyle #1}_{\scriptstyle #2}$}
              \restrictionaux{#1}{#2}}
              {\setbox1\hbox{${\textstyle #1}_{\scriptstyle #2}$}
              \restrictionaux{#1}{#2}}
              {\setbox1\hbox{${\scriptstyle #1}_{\scriptscriptstyle #2}$}
              \restrictionaux{#1}{#2}}
              {\setbox1\hbox{${\scriptscriptstyle #1}_{\scriptscriptstyle #2}$}
              \restrictionaux{#1}{#2}}}
\def\restrictionaux#1#2{{#1\,\smash{\vrule height .8\ht1 depth .85\dp1}}_{\,#2}}
%%%%%%%%%%%

%%% FORMATTAZIONE FOOTNOTEMARK

\def\footnotemarkformatting#1{[#1]}
\renewcommand{\thefootnote}{\footnotemarkformatting{\arabic{footnote}}}

%% SEZIONE GRAFICA
\use{tikz}
\usetikzlibrary{matrix, patterns, calc, decorations.pathreplacing, hobby, decorations.markings, decorations.pathmorphing, babel}
\use{tikz-3dplot}
\use{mathrsfs} %per geogebra
\use{tikz-cd}
\tikzset
{
  %surface/.style={fill=black!10, shading=ball,fill opacity=0.4},
  plane/.style={black,pattern=north east lines},
  curve/.style={black,line width=0.5mm},
  dritto/.style={decoration={markings,mark=at position 0.5 with {\arrow{Stealth}}}, postaction=decorate},
  rovescio/.style={decoration={markings,mark=at position 0.5 with {\arrow{Stealth[reversed]}}}, postaction=decorate}
}
\use{pgfplots} % stampare le funzioni
        \pgfplotsset{/pgf/number format/use comma,compat=1.15}
        %\pgfplotsset{compat=1.15} %per geogebra
        \usepgfplotslibrary{fillbetween, polar}
%%%%%%

%% CITAZIONI
\use{lineno}

\newcommand{\citazione}[1]{%
  \begin{quotation}
  \begin{linenumbers}
  \modulolinenumbers[5]
  \begingroup
  \setlength{\parindent}{0cm}
  \noindent #1
  \endgroup
  \end{linenumbers}
  \end{quotation}\setcounter{linenumber}{1}
  }
%%%%%%

%%%%%%%%%%%%%%%%%%%%%%%%%%%%%%%%%%%%%%%%%%%%
%%%%%%%%%%%%%%%%%%%%%%%%%%%%%%%%%%%%%%%%%%%%

%% AMS THM

\theoremstyle{definition}% default
\newtheorem{thm}{Teorema}[section]
\newtheorem{lem}[thm]{Lemma}
\newtheorem{prop}[thm]{Proposizione}
\newtheorem{cor}[thm]{Corollario}
\newtheorem{esempio}[thm]{Esempio}
\theoremstyle{plain}
\newtheorem{definizione}[thm]{Definizione}
\theoremstyle{remark}
\newtheorem*{oss}{Osservazione}


%%%%%%%%%%%%%%%%%%%%%%%%%%%%%%%%%%%%%%%%%%%%
%%%%%%%%%%%%%%%%%%%%%%%%%%%%%%%%%%%%%%%%%%%%

\use{hyperref}
\hypersetup{%
        pdfauthor={Davide Peccioli},
        pdfsubject={},
        allcolors=black,
        citecolor=black,
%	colorlinks=true,
        bookmarksopen=true}
\setcounter{secnumdepth}{0} % rimuove i numeri di sezione senza rimuovere le ref
\renewcommand{\href}[2]{\textcolor{blue}{#2}} % disabilita il comando href
\use{enotez} %
\setenotez{%
 mark-format = \footnotemarkformatting % Mette i numeri tra parentesi quadre%
}\let\footnote=\endnote % rende tutte le note a pié pagina come delle note a fine file 


\let\olddocument\document % modifico l'ambiende documenti per non dover stampare \printendnote
\let\oldenddocument\enddocument
\renewenvironment{document}%
{%
  \olddocument
}{%
  \printendnotes\oldenddocument
}
\renewcommand{\thethm}{\arabic{thm}}

\usepackage[hyperref]{biblatex}
\addbibresource{~/Documents/org/roam/bib/master.bib}
\author{Davide Peccioli}
\date{\today}
\title{}
\begin{document}

\section{Dualità di Poincaré}
\label{sec:orgf19828d}
\begin{thm}
Sia \(M\) una \href{20250113115909-struttura_differenziabile.org}{varietà differenziabile} \href{20251223152054-varieta_differenziabile_orientabile.org}{orientabile}. Allora, per ogni \(0\le k \le n\) il seguente \href{20251115185654-integrazione_di_forme_su_varieta_differenziabile_orientata.org}{integrale} (per il \href{20251229120415-prodotto_tensoriale.org}{prodotto tensoriale} tra \href{20251115172442-gruppo_di_coomologia_di_de_rham.org}{coomologie di De Rham} \(H^{k}\) e \href{20251115192241-coomologia_a_supporto_compatto.org}{a supporto compatto} \(H^{k}_{\text{c}}\)) è ben definito\footnote{Vedi ``\href{20251115155511-forma_differenziale_in_un_punto.org}{Prodotto wedge tra forme differenziali}''}
\begin{align*}
\int_{M}: H^{k}(M) \otimes H^{n-k}_{\text{c}}(M) &\longrightarrow \R\\
([\omega],[\tau]) &\longmapsto \int_{M} \omega\wedge \tau
\end{align*}
e induce un \href{20250113125833-isomorfismo_tra_spazi_vettoriali.org}{isomorfismo} sullo \href{20250105124008-spazio_vettoriale_duale.org}{spazio duale}:
\begin{align*}
\int_{M}: H^{k}(M) &\xrightarrow{\hphantom{cia}\cong\hphantom{cia}} H^{n-k}_{\text{c}}(M)^{*}\\
[\omega] &\longmapsto \bigg([\tau] \mapsto \int_{M} \omega\wedge \tau\bigg)
\end{align*}
\end{thm}

\begin{proof}
La mappa è ben definita per il \href{20251115190058-teorema_di_stokes.org}{Teorema di Stokes}. Si dimostra l'isomorfismo solo per \href{20251115191303-varieta_differenziabile_di_tipo_finito.org}{varietà di tipo finito}.

\uline{Passo 0}: Considero un ricoprimento \(\set{U,V}\) di \(M\), e scrivo le due successioni Mayer-Vietoris (\href{20251115183635-teorema_di_mayer_vietoris_in_coomologia.org}{quella normale} e \href{20251229105636-successione_di_mayer_vietoris_per_coomologia_a_supporto_compatto.org}{quella a supporto compatto}), si ottiene un diagramma in cui tutti i quadrati sono commutativi, tranne (\(\star\)), che è commutativo a meno del segno (\emph{non dimostrato}):
\begin{equation*}
\begin{tikzcd}[sep=tiny]
	\cdots && {H^k(M)} && {H^k(U)\oplus H^k(V)} && {H^k(U\cap V)} && {H^{k+1}(M)} && \cdots \\
	&&&&&&& {(\star)} \\
	\cdots && {H_{\text{c}}^{n-k}(M)^*} && {H_{\text{c}}^{n-k}(U)^*\oplus H_{\text{c}}^{n-k}(V)^*} && {H_{\text{c}}^{n-k}(U\cap V)^*} && {H_{\text{c}}^{n-k-1}(M)^*} && \cdots
	\arrow[from=1-1, to=1-3]
	\arrow[from=1-3, to=1-5]
	\arrow["{\int_M}"', from=1-3, to=3-3]
	\arrow[from=1-5, to=1-7]
	\arrow["{\int_U}"', shift right=5, from=1-5, to=3-5]
	\arrow["{\int_V}", shift left=5, from=1-5, to=3-5]
	\arrow[from=1-7, to=1-9]
	\arrow["{\int_{U\cap V}}", from=1-7, to=3-7]
	\arrow[from=1-9, to=1-11]
	\arrow["{\int_M}", from=1-9, to=3-9]
	\arrow[from=3-1, to=3-3]
	\arrow[from=3-3, to=3-5]
	\arrow[from=3-5, to=3-7]
	\arrow[from=3-7, to=3-9]
	\arrow[from=3-9, to=3-11]
\end{tikzcd}
\end{equation*}

\uline{Passo 1}: Si dimostra per induzione sulla cardinalità \(s\) di un \href{20251115191224-ricoprimento_aciclico_di_una_varieta_differenziabile.org}{ricoprimento aciclico} (ovvero \(\set{U_{\alpha}}_{\alpha \in A}\) tale che ogni intersezione finita non vuota di aperti sia diffeomorfa ad \(\R^{n}\), e \href{20251115173810-varieta_differenziabili_diffeomorfe_hanno_stessa_coomologia_di_de_rham.org}{quindi} abbia la stessa coomologia) di \(M\).
\begin{itemize}
\item Passo base \(s=1\): \(M\) è ricoperto da un solo aperto che ha la \href{20251115173611-coomologia_di_de_rham_di_r.org}{Coomologia di \(\R\)}.

Quindi, per \(k\neq 0\) le coomologie sono nulle, mentre\footnote{Vedi anche ``\href{20251115192308-coomologia_a_supporto_compatto_di_r.org}{Coomologia a supporto compatto di R}''}:
\begin{equation*}
  H^{0}(M) \cong \R,\qquad H^{n-0}_{\text{c}}(M) \cong \R
\end{equation*}

Pertanto \(H^{n}_{\text{c}}(M)^{*} \cong \R\), e \(\int_{M}\) è effettivamente l'isomorfismo.

\item Passo induttivo: Supponiamo la tesi vera per tutte le varietà che ammettono un ricoprimento aciclico di cardinalità \(s-1\), e la dimostriamo per una varietà che ammette un ricprimento aciclico di \(s\) aperti.

Sia quindi \(M\) una varietà con un ricoprimento aciclico \(\set{U_{1},\dots,U_{s}}\), e siano
\begin{equation*}
  U\coloneqq U_{1}\cup\dots\cup U_{s-1},\qquad V\coloneqq U_{s}
\end{equation*}
Allora \(U\cap V\) è ricoperta da \(\set{U_{1}\cap U_{s},\dots,U_{s-1}\cap U_{s}}\). Scriviamo il diagramma commutativo di cui sopra:
\begin{equation*}
\scriptsize
\begin{tikzcd}[column sep=tiny]
        {H^{k-1}(U)\oplus H^{k-1}(V)} && {H^{k-1}(U\cap V)} && {H^k(M)} && {H^k(U)\oplus H^k(V)} && {H^k(U\cap V)} \\
        \\
        {H_{\text{c}}^{n-k+1}(U)^*\oplus H_{\text{c}}^{n-k+1}(V)^*} && {H_{\text{c}}^{n-k+1}(U\cap V)^*} && {H_{\text{c}}^{n-k}(M)^*} && {H_{\text{c}}^{n-k}(U)^*\oplus H_{\text{c}}^{n-k}(V)^*} && {H_{\text{c}}^{n-k}(U\cap V)^*}
        \arrow[from=1-1, to=1-3]
        \arrow["{\int_U}"'{text={rgb,255:red,255;green,54;blue,51}}, shift right=5, from=1-1, to=3-1]
        \arrow["{\int_V}"{text={rgb,255:red,255;green,54;blue,51}}, shift left=5, from=1-1, to=3-1]
        \arrow[from=1-3, to=1-5]
        \arrow["{\int_{U\cap V}}"'{text={rgb,255:red,255;green,54;blue,51}}, from=1-3, to=3-3]
        \arrow[from=1-5, to=1-7]
        \arrow["{\int_M}"', from=1-5, to=3-5]
        \arrow[from=1-7, to=1-9]
        \arrow["{\int_U}"'{text={rgb,255:red,255;green,54;blue,51}}, shift right=5, from=1-7, to=3-7]
        \arrow["{\int_V}"{text={rgb,255:red,255;green,54;blue,51}}, shift left=5, from=1-7, to=3-7]
        \arrow["{\int_{U\cap V}}"{text={rgb,255:red,255;green,54;blue,51}}, from=1-9, to=3-9]
        \arrow[from=3-1, to=3-3]
        \arrow[from=3-3, to=3-5]
        \arrow[from=3-5, to=3-7]
        \arrow[from=3-7, to=3-9]
\end{tikzcd}
\end{equation*}

Per ipotesi induttiva, tutti i morfismi evidenziati in rosso sono isomorfismi, e quindi, per il \href{20250120155801-lemma_del_cinque.org}{Lemma del cinque}, anche \(\int_{M}\) è isomorfismo.
\qedhere
\end{itemize}
\end{proof}
\begin{oss}
Se \(M\) è una \href{20250113115909-struttura_differenziabile.org}{varietà differenziabile} \href{20251223152054-varieta_differenziabile_orientabile.org}{orientabile} e \href{20250103163701-spazio_topologico_compatto.org}{compatta}, allora \(H^{k}_{\text{c}}(M) = H^{k}(M)\), e pertanto, per DP:
\begin{equation*}
H^{k}(M) \cong H^{n-k}(M)^{*}
\end{equation*}
\end{oss}
\begin{oss}
Sempre nel caso in cui \(M\) è una \href{20250113115909-struttura_differenziabile.org}{varietà differenziabile} \href{20251223152054-varieta_differenziabile_orientabile.org}{orientabile} e \href{20250103163701-spazio_topologico_compatto.org}{compatta}, allora\footnote{Questo si ha perché:
\begin{itemize}
\item \href{20251115191303-varieta_differenziabile_di_tipo_finito.org}{Varietà differenziabile compatta ha tipo finito}
\item \href{20251115191355-varieta_differenziabile_di_tipo_finito_ha_coomologia_di_dimensione_finita.org}{Varietà differenziabile di tipo finito ha coomologia di dimensione finita}
\end{itemize}} la \href{20241205142027-spazio_vettoriale.org}{dimensione} \(\dim H^{k}(M) < +\infty\) e quindi, applicando nuovamente DP
\begin{equation*}
	H^{n-k}(M) \cong H^{k}(M)^{*} \cong H^{k}(M)
\end{equation*}
\end{oss}
\end{document}
