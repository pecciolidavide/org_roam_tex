% Created 2026-02-07 Sat 19:31
% Intended LaTeX compiler: pdflatex
\documentclass[10pt]{article}
%% CREATO CON ORG - EMACS
\newcommand{\use}[2][]{\usepackage[#1]{#2}}
% PACCHETTI FONDAMENTLAI
\use[utf8]{inputenc}
\use[T1]{fontenc}
\use{graphicx}
\use{longtable}
\use{wrapfig}
\use{rotating}
\use[normalem]{ulem}
\use{amsmath}
\use{amsthm}
\use{amssymb}

\use{eucal} % Cambia mathcal{...}

\use{capt-of}
\use[italian]{babel}
\use[babel]{csquotes}
% bib la TEX lo carica in automatico org-cite
\use{microtype}
\use{lmodern}
\use{subfig} % sottofigure
\use{multicol} % due colonne
\use{lipsum} % lorem ipsum
\use{color} % colori in latex
\use{parskip} % rimuove l'indentazione dei nuovi paragrafi %% Add parbox=false to all new tcolorbox
\use{centernot}
\use[outline]{contour}\contourlength{3pt}
\use{fancyhdr}
\use{layout}
\use[most]{tcolorbox} % Riquadri colorati
\use{ifthen} % IFTHEN
\use{geometry}

% pacchetti matematica
\use{yhmath}
\use{dsfont}
\use{mathrsfs}
\use{cancel} % semplificare
\use{polynom} %divisione tra polinomi
\use{forest} % grafi ad albero
\use{booktabs} % tabelle
\use{commath} %simboli e differenziali
\use{bm} %bold
\use[fulladjust]{marginnote} %to use marginnote for date notes
\use{arrayjobx}%array
\use[intlimits]{empheq} % Riquadri colorati attorno alle equazioni
\use{mathtools}
\use{circuitikz} % Disegnare i circuiti
\use{mathtools}
\use{stmaryrd} % [[ \llbracket ]] \rrbracket
\use{bussproofs} % dimostrazioni

%%%%%%%%%%%%%


%%%% QUIVER
\newcommand{\duepunti}{\,\mathchar\numexpr"6000+`:\relax\,}
% A TikZ style for curved arrows of a fixed height, due to AndréC.
\tikzset{curve/.style={settings={#1},to path={(\tikztostart)
    .. controls ($(\tikztostart)!\pv{pos}!(\tikztotarget)!\pv{height}!270:(\tikztotarget)$)
    and ($(\tikztostart)!1-\pv{pos}!(\tikztotarget)!\pv{height}!270:(\tikztotarget)$)
    .. (\tikztotarget)\tikztonodes}},
    settings/.code={\tikzset{quiver/.cd,#1}
        \def\pv##1{\pgfkeysvalueof{/tikz/quiver/##1}}},
    quiver/.cd,pos/.initial=0.35,height/.initial=0}

% TikZ arrowhead/tail styles.
\tikzset{tail reversed/.code={\pgfsetarrowsstart{tikzcd to}}}
\tikzset{2tail/.code={\pgfsetarrowsstart{Implies[reversed]}}}
\tikzset{2tail reversed/.code={\pgfsetarrowsstart{Implies}}}
% TikZ arrow styles.
\tikzset{no body/.style={/tikz/dash pattern=on 0 off 1mm}}
%%%%%%%%%%


%% DEFINIZIONI COMANDI MATEMATICI
\let\sin\relax %TOGLIE LA DEFINIZIONE SU "\sin"

% cambia la definizione di empty set
% ---
\let\oldemptyset\emptyset
% ---
% \let\emptyset\varnothing
% ---
% \let\emptyset\relax
% \newcommand{\emptyset}{\text{\textnormal{\O}}}
% ---

\DeclareMathOperator{\bounded}{bd}
\DeclareMathOperator{\sin}{sen}
\DeclareMathOperator{\epi}{Epi}
\DeclareMathOperator{\cl}{cl}
\DeclareMathOperator{\graph}{graph}
\DeclareMathOperator{\arcsec}{arcsec}
\DeclareMathOperator{\arccot}{arccot}
\DeclareMathOperator{\arccsc}{arccsc}
\DeclareMathOperator{\spettro}{Spettro}
\DeclareMathOperator{\nulls}{nullspace}
\DeclareMathOperator{\dom}{dom}
\DeclareMathOperator{\ar}{ar}
\DeclareMathOperator{\const}{Const}
\DeclareMathOperator{\fun}{Fun}
\DeclareMathOperator{\rel}{Rel}
\DeclareMathOperator{\altezza}{ht}
\let\det\relax %TOGLIE LA DEFINIZIONE SU "\det"
\DeclareMathOperator{\det}{det}
\DeclareMathOperator{\End}{End}
\DeclareMathOperator{\gl}{GL}
\def\Id{\mathrm{Id}}
\def\id{\mathrm{id}}
\DeclareMathOperator{\I}{\mathds{1}}
\DeclareMathOperator{\II}{II}
\DeclareMathOperator{\rank}{rank}
\DeclareMathOperator{\tr}{tr}
\DeclareMathOperator{\tc}{t.c.}
\DeclareMathOperator{\T}{T}
\DeclareMathOperator{\var}{Var}
\DeclareMathOperator{\cov}{Cov}
\DeclareMathOperator{\st}{st}
\DeclareMathOperator{\mon}{Mon}
\newcommand{\card}[1]{\left\vert #1 \right\vert}
\newcommand{\trasposta}[1]{\prescript{\text{T}}{}{#1}}
\newcommand{\1}{\mathds{1}}
\newcommand{\R}{\mathds{R}}
\newcommand{\diesis}{\#}
\newcommand{\bemolle}{\flat}
\newcommand{\nonstandard}[1]{\prescript{*}{}{#1}}
\newcommand{\starR}{\nonstandard{\R}}
\newcommand{\borel}{\mathscr{B}}
\newcommand{\lebesgue}[1]{\mathscr{L}\left(#1\right)}
\newcommand{\media}{\mathds{E}}
\newcommand{\K}{\mathds{K}}
\newcommand{\A}{\mathds{A}}
\newcommand{\Q}{\mathds{Q}}
\newcommand{\N}{\mathds{N}}
\newcommand{\C}{\mathds{C}}
\newcommand{\Z}{\mathds{Z}}
\newcommand{\qo}{\hspace{1em}\text{q.o.}\,}
\renewcommand{\tilde}[1]{\widetilde{#1}}
\renewcommand{\parallel}{\mathrel{/\mkern-5mu/}}
\newcommand{\parti}[2][]{\wp_{#1}(#2)}
\newcommand{\diff}[1]{\operatorname{d}_{#1}}
\let\oldvec\vec
\renewcommand{\vec}[1]{\overrightarrow{\vphantom{i}#1}}
\newcommand{\floor}[1]{\left\lfloor #1 \right\rfloor}
\newcommand{\cat}[1]{\mathbf{#1}}
\newcommand{\dfreccia}[1]{\xrightarrow{\ #1 \ }}
\newcommand{\sfreccia}[1]{\xleftarrow{\ #1 \ }}
\newcommand{\formalsum}[2]{{\sum_{#1}^{#2}}{\vphantom{\sum}}'}
\newcommand{\minim}[2]{\mu_{#1}\, \left(#2\right)}
\newcommand{\concat}{\null^{\frown}} % concatenazione di stringe
\newcommand{\godelcode}[1]{\langle\!\langle #1 \rangle\!\rangle}
\newcommand{\godeldec}[1]{(\!(#1)\!)}
\newcommand{\termcode}[1]{\ulcorner #1\urcorner}
\newcommand{\partialto}{\dashrightarrow}
\newcommand{\restricted}{\upharpoonright}
\newcommand{\embeds}{\precsim}
\newcommand{\surjects}{\twoheadrightarrow}
\newcommand{\equipotenti}{\asymp}
%% \newcommand{\dotplus}{\mathbin{\dot{+}}} %% A quanto pare esiste già
\newcommand{\bigdot}{\mathbin{\boldsymbol{\cdot}}}
\newcommand{\dotexp}[1]{^{.#1}}
\newcommand{\conv}{\mathbin{*}}
\newcommand{\convolution}[2]{(#1\conv #2)}
\newcommand{\nil}{\mathfrak{N}}
\newcommand{\divisore}{\mathrel{|}}
\newcommand{\simplesso}[1]{\mathrm{e}_{#1}}

\renewcommand{\iff}{\mathrel{\longleftrightarrow}} %% Notazione Logica.
\newcommand{\oldiff}{\mathrel{\Longleftrightarrow}}
\renewcommand{\implies}{\mathrel{\rightarrow}} %% Notazione Logica
\newcommand{\oldimplies}{\mathrel{\Longrightarrow}}
\renewcommand{\impliedby}{\mathrel{\leftarrow}} %% Notazione Logica
\newcommand{\oldimpliedby}{\mathrel{\Longleftarrow}}

\newcommand{\IFF}{\quad\Longleftrightarrow\quad}
\newcommand{\IMPLICA}{\quad\Longrightarrow\quad}


\renewcommand{\descriptionlabel}[1]{\hspace{\labelsep}\normalfont #1} % remove bold from description


%% Definizione di Divergenza di K-L

\DeclarePairedDelimiterX{\infdivx}[2]{(}{)}{%
  #1\;\delimsize\|\;#2%
}
\newcommand{\kldiv}{D_{KL}\infdivx}

%% Definizione di \dotminus

\makeatletter
\newcommand{\dotminus}{\mathbin{\text{\@dotminus}}}

\newcommand{\@dotminus}{%
  \ooalign{\hidewidth\raise1ex\hbox{.}\hidewidth\cr$\m@th-$\cr}%
}
\makeatother

%tramite i prossimi due comandi posso decidere come scrivere i logaritmi naturali in tutti i documenti: ho infatti eliminato qualsiasi differenza tra "ln" e "log": se si vuole qualcosa di diverso bisogna inserire manualmente il tutto
\let\ln\relax
\DeclareMathOperator{\ln}{ln}
\let\log\relax
\DeclareMathOperator{\log}{log}
%%%%%%

%% NUOVI COMANDI
\newcommand{\straniero}[1]{\textit{#1}} %parole straniere
\newcommand{\titolo}[1]{\textsc{#1}} %titoli
\newcommand{\qedd}{\tag*{$\blacksquare$}} %qed per ambienti matemastici
\renewcommand{\qedsymbol}{$\blacksquare$} %modifica colore qed
\newcommand{\ooverline}[1]{\overline{\overline{#1}}}
\newcommand{\circoletto}[1]{\left(#1\right)^{\text{o}}}
%
\newcommand{\qmatrice}[1]{\begin{pmatrix}
#1_{11} & \cdots & #1_{1n}\\
\vdots & \ddots & \vdots \\
#1_{m1} & \cdots & #1_{mn}
\end{pmatrix}}
%
\newcommand{\parentesi}[2]{%
\underset{#1}{\underbrace{#2}}%
}
%
\newcommand{\norma}[1]{% Norma
\left\lVert#1\right\rVert%
}
\newcommand{\scalare}[2]{% Scalare
\left\langle #1, #2\right\rangle
}
%%%%%

%% RESTRIZIONI
\newcommand{\referenze}[2]{
        \phantomsection{}#2\textsuperscript{\textcolor{blue}{\textbf{#1}}}
}

\let\restriction\relax

\def\restriction#1#2{\mathchoice
              {\setbox1\hbox{${\displaystyle #1}_{\scriptstyle #2}$}
              \restrictionaux{#1}{#2}}
              {\setbox1\hbox{${\textstyle #1}_{\scriptstyle #2}$}
              \restrictionaux{#1}{#2}}
              {\setbox1\hbox{${\scriptstyle #1}_{\scriptscriptstyle #2}$}
              \restrictionaux{#1}{#2}}
              {\setbox1\hbox{${\scriptscriptstyle #1}_{\scriptscriptstyle #2}$}
              \restrictionaux{#1}{#2}}}
\def\restrictionaux#1#2{{#1\,\smash{\vrule height .8\ht1 depth .85\dp1}}_{\,#2}}
%%%%%%%%%%%

%%% FORMATTAZIONE FOOTNOTEMARK

\def\footnotemarkformatting#1{[#1]}
\renewcommand{\thefootnote}{\footnotemarkformatting{\arabic{footnote}}}

%% SEZIONE GRAFICA
\use{tikz}
\usetikzlibrary{matrix, patterns, calc, decorations.pathreplacing, hobby, decorations.markings, decorations.pathmorphing, babel}
\use{tikz-3dplot}
\use{mathrsfs} %per geogebra
\use{tikz-cd}
\tikzset
{
  %surface/.style={fill=black!10, shading=ball,fill opacity=0.4},
  plane/.style={black,pattern=north east lines},
  curve/.style={black,line width=0.5mm},
  dritto/.style={decoration={markings,mark=at position 0.5 with {\arrow{Stealth}}}, postaction=decorate},
  rovescio/.style={decoration={markings,mark=at position 0.5 with {\arrow{Stealth[reversed]}}}, postaction=decorate}
}
\use{pgfplots} % stampare le funzioni
        \pgfplotsset{/pgf/number format/use comma,compat=1.15}
        %\pgfplotsset{compat=1.15} %per geogebra
        \usepgfplotslibrary{fillbetween, polar}
%%%%%%

%% CITAZIONI
\use{lineno}

\newcommand{\citazione}[1]{%
  \begin{quotation}
  \begin{linenumbers}
  \modulolinenumbers[5]
  \begingroup
  \setlength{\parindent}{0cm}
  \noindent #1
  \endgroup
  \end{linenumbers}
  \end{quotation}\setcounter{linenumber}{1}
  }
%%%%%%

%%%%%%%%%%%%%%%%%%%%%%%%%%%%%%%%%%%%%%%%%%%%
%%%%%%%%%%%%%%%%%%%%%%%%%%%%%%%%%%%%%%%%%%%%

%% AMS THM

\theoremstyle{definition}% default
\newtheorem{thm}{Teorema}[section]
\newtheorem{lem}[thm]{Lemma}
\newtheorem{prop}[thm]{Proposizione}
\newtheorem{cor}[thm]{Corollario}
\newtheorem{esempio}[thm]{Esempio}
\theoremstyle{plain}
\newtheorem{definizione}[thm]{Definizione}
\theoremstyle{remark}
\newtheorem*{oss}{Osservazione}


%%%%%%%%%%%%%%%%%%%%%%%%%%%%%%%%%%%%%%%%%%%%
%%%%%%%%%%%%%%%%%%%%%%%%%%%%%%%%%%%%%%%%%%%%

\use{hyperref}
\hypersetup{%
        pdfauthor={Davide Peccioli},
        pdfsubject={},
        allcolors=black,
        citecolor=black,
%	colorlinks=true,
        bookmarksopen=true}
\setcounter{secnumdepth}{0} % rimuove i numeri di sezione senza rimuovere le ref
\renewcommand{\href}[2]{\textcolor{blue}{#2}} % disabilita il comando href
\use{enotez} %
\setenotez{%
 mark-format = \footnotemarkformatting % Mette i numeri tra parentesi quadre%
}\let\footnote=\endnote % rende tutte le note a pié pagina come delle note a fine file 


\let\olddocument\document % modifico l'ambiende documenti per non dover stampare \printendnote
\let\oldenddocument\enddocument
\renewenvironment{document}%
{%
  \olddocument
}{%
  \printendnotes\oldenddocument
}
\renewcommand{\thethm}{\arabic{thm}}

\usepackage[hyperref]{biblatex}
\addbibresource{~/Documents/org/roam/bib/master.bib}
\use{upgreek}
\def\tau{\uptau}
\author{Davide Peccioli}
\date{\today}
\title{}
\begin{document}

\section{Mappa di suddivisione tra complessi di catene singolari}
\label{sec:org2a16f94}
Sia \(R\) un \href{20241219112842-pid.org}{PID}. Sia \(X\) uno \href{20250103145124-topologia.org}{spazio topologico}, e sia
\begin{equation*}
\mathcal{S}_{\bullet}(X) = \set{\left(S_{q}(X),\partial_{q}^{X}\right)}_{q}
\end{equation*}
il \href{20250120163114-complesso_di_catene.org}{complesso} di \href{20250122133614-mappa_di_bordo_tra_moduli_di_catene_singolari.org}{catene singolari}, \(S_{q}\) \href{20250122133435-simplesso_singolare.org}{modulo di catene singolari}.

\begin{definizione}
Per ogni \(q \in \N\), si construisce la \textbf{\href{20241206115416-morfismi_r_moduli.org}{mappa} di suddivisione}
\begin{equation*}
\operatorname{sd}_{q}^{(X)}:S_{q}(X) \to S_{q}(X)
\end{equation*}
come segue, per induzione.
\begin{itemize}
\item Per \(q = 0\), si pone \(\operatorname{sd}_{0}^{(X)} = \Id_{X}\).
\item Per ipotesi induttiva, si supponga costruita per ogni \(X\) la mappa: \(\operatorname{sd}_{q-1}^{(X)} : S_{q-1}(X)\to S_{q-1}(X)\).

\begin{itemize}
\item \uline{Per \(X = \Delta_{q}\) \href{20250121122324-simplesso_standard.org}{simplesso standard} e \(\iota_{q} \in S_{q}(\Delta_{q})\), \(\iota = \operatorname{id}_{\Delta_{q}}\).}

Sia \(b\) il \href{20250128131908-suddivisione_baricentrica.org}{baricentro} di \(\Delta_{q}\), e sia
\begin{equation*}
\operatorname{J}_{b}: S_{q-1}(\Delta_{q})\longrightarrow S_{q}(\Delta_{q})
\end{equation*}
il \href{20250122154711-estensione_di_un_simplesso_singolare_in_uno_spazio_stellato.org}{join ad un punto} (poiché \(\Delta_{q}\) è stellato rispetto a \(b\)).

Definiamo
\begin{equation*}
\operatorname{sd}_{q}(\iota_{q}) \coloneqq \operatorname{J}_{b}\left(\operatorname{sd}_{q-1}(\partial_{q}^{\Delta_{q}}\iota_{q})\right).
\end{equation*}

\item Per \(X\) qualsiasi, si \href{20241213094625-modulo_libero.org}{definisce} il \href{20241206115416-morfismi_r_moduli.org}{morfismo}
\begin{equation*}
\operatorname{sd}_{q}^{(X)}: S_{q}(X) \longrightarrow S_{q}(X)
\end{equation*}
sulla \href{20241213094625-modulo_libero.org}{base} \(\Sigma_{q}(X)\) di \(S_{q}(X) = R^{\Sigma_{q}(X)}\)\footnote{Vedi ``\href{20241213095808-somma_diretta.org}{Somma Diretta}''}

Se \(\sigma \in \Sigma_{q}(X)\) \href{20250122133435-simplesso_singolare.org}{catena singolare}, allora \(\sigma:\Delta_{q}\longrightarrow X\) \href{20250103103252-funzione_continua.org}{continua}, ed è pertanto possibile applicare il \href{20241205131958-funtore.org}{funtore} \href{20250122154136-funtorialita_dell_omologia_singolare.org}{diesis} ottenendo
\begin{equation*}
\sigma_{\diesis}: \mathcal{S}_{\bullet}(\Delta_{q}) \longrightarrow \mathcal{S}_{\bullet}(X)
\end{equation*}
dove il \href{20250120163759-categoria_complessi_di_catene.org}{morfismo di complessi di catene} è definito come segue:
\begin{equation*}
\sigma_{\diesis} = \set{
\begin{aligned}
\sigma_{\diesis}^{n}: S_{n}(\Delta_{q}) &\longrightarrow S_{n}(X)\\
\Sigma_{n}(\Delta_{q})\ni\tau &\longmapsto \sigma\circ\tau
\end{aligned}
}
\end{equation*}

Dunque si definisce \(\operatorname{sd}_{q}^{(X)}\sigma \coloneqq \sigma_{\diesis}^{q}\left(\operatorname{sd}_{q}^{(\Delta_{n})}\iota_{q}\right)\).
\end{itemize}
\end{itemize}
\end{definizione}

\begin{prop}
L'insieme dei \href{20241206115416-morfismi_r_moduli.org}{morfismi}
\(\operatorname{sd}_{q}: S_{q}(X) \longrightarrow S_{q}(X)\)
induce un \href{20250120163759-categoria_complessi_di_catene.org}{morfismo}
\begin{equation*}
\operatorname{sd}_{\bullet} : \mathcal{S}_{\bullet}(X)\longrightarrow \mathcal{S}_{\bullet}(X)
\end{equation*}
ovvero il seguente diagramma commuta:
\begin{equation*}
\begin{tikzcd}[ampersand replacement=\&,cramped,sep=large]
	\cdots \& {S_{q+1}(X)} \& {S_{q}(X)} \& {S_{q-1}(X)} \& \cdots \\
	\cdots \& {S_{q+1}(X)} \& {S_{q}(X)} \& {S_{q-1}(X)} \& \cdots
	\arrow["{\partial_{q+2}}", from=1-1, to=1-2]
	\arrow["{\partial_{q+1}}", from=1-2, to=1-3]
	\arrow["{\operatorname{sd}_{q+1}}", from=1-2, to=2-2]
	\arrow["{\partial_q}", from=1-3, to=1-4]
	\arrow["{\operatorname{sd}_{q}}", from=1-3, to=2-3]
	\arrow["{\partial_{q-1}}", from=1-4, to=1-5]
	\arrow["{\operatorname{sd}_{q-1}}", from=1-4, to=2-4]
	\arrow["{\partial_{q+2}}"', from=2-1, to=2-2]
	\arrow["{\partial_{q+1}}"', from=2-2, to=2-3]
	\arrow["{\partial_q}"', from=2-3, to=2-4]
	\arrow["{\partial_{q-1}}"', from=2-4, to=2-5]
\end{tikzcd}
\end{equation*}
\end{prop}
\begin{cor}
È ben definita una mappa tra i \href{20250122133631-omologia_singolare.org}{moduli di omologia singolare}:
\begin{equation*}
\operatorname{sd}_{\star}: H_{q}(X) \longrightarrow H_{q}(X)\\
\end{equation*}
tramite il \href{20250123115927-funtore_di_omologia_singolare.org}{funtore di omologia}.
\end{cor}
\subsection{Legame con le trasformazioni affini}
\label{sec:org6779dcb}

\begin{oss}
\begin{enumerate}
\item Se \(\iota_{q} :\Delta_{q} \longrightarrow\Delta_{q}\) è l'identità, allora \(\operatorname{sd}_{q}(\iota_{q}) \in S_{q}(\Delta_{q})\) è
\begin{equation*}
 \operatorname{sd}_{q}(\iota_{q}) = \sum a_{i}\ \tau_{i}
\end{equation*}
con \(\tau_{i} \in \Sigma_{q}(\Delta_{q})\) \href{20250129094132-trasformazione_affine.org}{trasformazioni affini}.
\item Se \(\tau_{0}: \Delta_{q} \longrightarrow\Delta_{q}\) è una \href{20250129094132-trasformazione_affine.org}{trasformazione affine}, allora
\begin{align*}
 \operatorname{sd}_{q}(\tau_{0}) &= (\tau_{0})^{q}_{\diesis}\ \operatorname{sd}_{q}(\iota_{q})\\
 &= (\tau_{0})^{q}_{\diesis}\ \sum a_{i}\ \tau_{i} = \sum a_{i}\ \tau_{0}\circ \tau_{i}
\end{align*}
dove, ovviamente, \(\tau_{0}\circ\tau_{i}\) è ancora una \href{20250129094132-trasformazione_affine.org}{trasformazione affine}.

Pertanto, si ha che \(\operatorname{sd}_{q}^{2}(\iota_{q}) = \sum a_{j}a_{h}\ \tau_{i}\circ\tau_{j}\) è ancora somma di \href{20250129094132-trasformazione_affine.org}{trasformazioni affini}.
\item Se \(\tau:\Delta_{q}\to \Delta_{q}\) \href{20250129094132-trasformazione_affine.org}{trasformazione affine}, con \(P_{i} \coloneqq \tau(e_{i})\)\footnote{Con \(e_{i}\) si indicano \href{20250121122324-simplesso_standard.org}{i punti base di \(\Delta_{q}\)}}, allora \(\operatorname{sd}_{q}\tau = \sum_{i} a_{i}\tau_{i}\), dove \(\tau_{i}\) affine e tale che:
\begin{align*}
\tau_{i}: \Delta_{q} &\longrightarrow \Delta_{q}\\
e_{0} &\longmapsto P_{i_{0}}\\
e_{1} &\longmapsto \frac{P_{i_{0}}+P_{i_{1}}}{2}\\
&\vdots\\
e_{q} &\longmapsto \frac{P_{i_{0}} + \dots + P_{i_{{q}}}}{q+1}.
\end{align*}

TODO fare bene tutti i calcoli
\end{enumerate}
\end{oss}
\end{document}
