% Created 2026-02-07 Sat 19:31
% Intended LaTeX compiler: pdflatex
\documentclass[10pt]{article}
%% CREATO CON ORG - EMACS
\newcommand{\use}[2][]{\usepackage[#1]{#2}}
% PACCHETTI FONDAMENTLAI
\use[utf8]{inputenc}
\use[T1]{fontenc}
\use{graphicx}
\use{longtable}
\use{wrapfig}
\use{rotating}
\use[normalem]{ulem}
\use{amsmath}
\use{amsthm}
\use{amssymb}

\use{eucal} % Cambia mathcal{...}

\use{capt-of}
\use[italian]{babel}
\use[babel]{csquotes}
% bib la TEX lo carica in automatico org-cite
\use{microtype}
\use{lmodern}
\use{subfig} % sottofigure
\use{multicol} % due colonne
\use{lipsum} % lorem ipsum
\use{color} % colori in latex
\use{parskip} % rimuove l'indentazione dei nuovi paragrafi %% Add parbox=false to all new tcolorbox
\use{centernot}
\use[outline]{contour}\contourlength{3pt}
\use{fancyhdr}
\use{layout}
\use[most]{tcolorbox} % Riquadri colorati
\use{ifthen} % IFTHEN
\use{geometry}

% pacchetti matematica
\use{yhmath}
\use{dsfont}
\use{mathrsfs}
\use{cancel} % semplificare
\use{polynom} %divisione tra polinomi
\use{forest} % grafi ad albero
\use{booktabs} % tabelle
\use{commath} %simboli e differenziali
\use{bm} %bold
\use[fulladjust]{marginnote} %to use marginnote for date notes
\use{arrayjobx}%array
\use[intlimits]{empheq} % Riquadri colorati attorno alle equazioni
\use{mathtools}
\use{circuitikz} % Disegnare i circuiti
\use{mathtools}
\use{stmaryrd} % [[ \llbracket ]] \rrbracket
\use{bussproofs} % dimostrazioni

%%%%%%%%%%%%%


%%%% QUIVER
\newcommand{\duepunti}{\,\mathchar\numexpr"6000+`:\relax\,}
% A TikZ style for curved arrows of a fixed height, due to AndréC.
\tikzset{curve/.style={settings={#1},to path={(\tikztostart)
    .. controls ($(\tikztostart)!\pv{pos}!(\tikztotarget)!\pv{height}!270:(\tikztotarget)$)
    and ($(\tikztostart)!1-\pv{pos}!(\tikztotarget)!\pv{height}!270:(\tikztotarget)$)
    .. (\tikztotarget)\tikztonodes}},
    settings/.code={\tikzset{quiver/.cd,#1}
        \def\pv##1{\pgfkeysvalueof{/tikz/quiver/##1}}},
    quiver/.cd,pos/.initial=0.35,height/.initial=0}

% TikZ arrowhead/tail styles.
\tikzset{tail reversed/.code={\pgfsetarrowsstart{tikzcd to}}}
\tikzset{2tail/.code={\pgfsetarrowsstart{Implies[reversed]}}}
\tikzset{2tail reversed/.code={\pgfsetarrowsstart{Implies}}}
% TikZ arrow styles.
\tikzset{no body/.style={/tikz/dash pattern=on 0 off 1mm}}
%%%%%%%%%%


%% DEFINIZIONI COMANDI MATEMATICI
\let\sin\relax %TOGLIE LA DEFINIZIONE SU "\sin"

% cambia la definizione di empty set
% ---
\let\oldemptyset\emptyset
% ---
% \let\emptyset\varnothing
% ---
% \let\emptyset\relax
% \newcommand{\emptyset}{\text{\textnormal{\O}}}
% ---

\DeclareMathOperator{\bounded}{bd}
\DeclareMathOperator{\sin}{sen}
\DeclareMathOperator{\epi}{Epi}
\DeclareMathOperator{\cl}{cl}
\DeclareMathOperator{\graph}{graph}
\DeclareMathOperator{\arcsec}{arcsec}
\DeclareMathOperator{\arccot}{arccot}
\DeclareMathOperator{\arccsc}{arccsc}
\DeclareMathOperator{\spettro}{Spettro}
\DeclareMathOperator{\nulls}{nullspace}
\DeclareMathOperator{\dom}{dom}
\DeclareMathOperator{\ar}{ar}
\DeclareMathOperator{\const}{Const}
\DeclareMathOperator{\fun}{Fun}
\DeclareMathOperator{\rel}{Rel}
\DeclareMathOperator{\altezza}{ht}
\let\det\relax %TOGLIE LA DEFINIZIONE SU "\det"
\DeclareMathOperator{\det}{det}
\DeclareMathOperator{\End}{End}
\DeclareMathOperator{\gl}{GL}
\def\Id{\mathrm{Id}}
\def\id{\mathrm{id}}
\DeclareMathOperator{\I}{\mathds{1}}
\DeclareMathOperator{\II}{II}
\DeclareMathOperator{\rank}{rank}
\DeclareMathOperator{\tr}{tr}
\DeclareMathOperator{\tc}{t.c.}
\DeclareMathOperator{\T}{T}
\DeclareMathOperator{\var}{Var}
\DeclareMathOperator{\cov}{Cov}
\DeclareMathOperator{\st}{st}
\DeclareMathOperator{\mon}{Mon}
\newcommand{\card}[1]{\left\vert #1 \right\vert}
\newcommand{\trasposta}[1]{\prescript{\text{T}}{}{#1}}
\newcommand{\1}{\mathds{1}}
\newcommand{\R}{\mathds{R}}
\newcommand{\diesis}{\#}
\newcommand{\bemolle}{\flat}
\newcommand{\nonstandard}[1]{\prescript{*}{}{#1}}
\newcommand{\starR}{\nonstandard{\R}}
\newcommand{\borel}{\mathscr{B}}
\newcommand{\lebesgue}[1]{\mathscr{L}\left(#1\right)}
\newcommand{\media}{\mathds{E}}
\newcommand{\K}{\mathds{K}}
\newcommand{\A}{\mathds{A}}
\newcommand{\Q}{\mathds{Q}}
\newcommand{\N}{\mathds{N}}
\newcommand{\C}{\mathds{C}}
\newcommand{\Z}{\mathds{Z}}
\newcommand{\qo}{\hspace{1em}\text{q.o.}\,}
\renewcommand{\tilde}[1]{\widetilde{#1}}
\renewcommand{\parallel}{\mathrel{/\mkern-5mu/}}
\newcommand{\parti}[2][]{\wp_{#1}(#2)}
\newcommand{\diff}[1]{\operatorname{d}_{#1}}
\let\oldvec\vec
\renewcommand{\vec}[1]{\overrightarrow{\vphantom{i}#1}}
\newcommand{\floor}[1]{\left\lfloor #1 \right\rfloor}
\newcommand{\cat}[1]{\mathbf{#1}}
\newcommand{\dfreccia}[1]{\xrightarrow{\ #1 \ }}
\newcommand{\sfreccia}[1]{\xleftarrow{\ #1 \ }}
\newcommand{\formalsum}[2]{{\sum_{#1}^{#2}}{\vphantom{\sum}}'}
\newcommand{\minim}[2]{\mu_{#1}\, \left(#2\right)}
\newcommand{\concat}{\null^{\frown}} % concatenazione di stringe
\newcommand{\godelcode}[1]{\langle\!\langle #1 \rangle\!\rangle}
\newcommand{\godeldec}[1]{(\!(#1)\!)}
\newcommand{\termcode}[1]{\ulcorner #1\urcorner}
\newcommand{\partialto}{\dashrightarrow}
\newcommand{\restricted}{\upharpoonright}
\newcommand{\embeds}{\precsim}
\newcommand{\surjects}{\twoheadrightarrow}
\newcommand{\equipotenti}{\asymp}
%% \newcommand{\dotplus}{\mathbin{\dot{+}}} %% A quanto pare esiste già
\newcommand{\bigdot}{\mathbin{\boldsymbol{\cdot}}}
\newcommand{\dotexp}[1]{^{.#1}}
\newcommand{\conv}{\mathbin{*}}
\newcommand{\convolution}[2]{(#1\conv #2)}
\newcommand{\nil}{\mathfrak{N}}
\newcommand{\divisore}{\mathrel{|}}
\newcommand{\simplesso}[1]{\mathrm{e}_{#1}}

\renewcommand{\iff}{\mathrel{\longleftrightarrow}} %% Notazione Logica.
\newcommand{\oldiff}{\mathrel{\Longleftrightarrow}}
\renewcommand{\implies}{\mathrel{\rightarrow}} %% Notazione Logica
\newcommand{\oldimplies}{\mathrel{\Longrightarrow}}
\renewcommand{\impliedby}{\mathrel{\leftarrow}} %% Notazione Logica
\newcommand{\oldimpliedby}{\mathrel{\Longleftarrow}}

\newcommand{\IFF}{\quad\Longleftrightarrow\quad}
\newcommand{\IMPLICA}{\quad\Longrightarrow\quad}


\renewcommand{\descriptionlabel}[1]{\hspace{\labelsep}\normalfont #1} % remove bold from description


%% Definizione di Divergenza di K-L

\DeclarePairedDelimiterX{\infdivx}[2]{(}{)}{%
  #1\;\delimsize\|\;#2%
}
\newcommand{\kldiv}{D_{KL}\infdivx}

%% Definizione di \dotminus

\makeatletter
\newcommand{\dotminus}{\mathbin{\text{\@dotminus}}}

\newcommand{\@dotminus}{%
  \ooalign{\hidewidth\raise1ex\hbox{.}\hidewidth\cr$\m@th-$\cr}%
}
\makeatother

%tramite i prossimi due comandi posso decidere come scrivere i logaritmi naturali in tutti i documenti: ho infatti eliminato qualsiasi differenza tra "ln" e "log": se si vuole qualcosa di diverso bisogna inserire manualmente il tutto
\let\ln\relax
\DeclareMathOperator{\ln}{ln}
\let\log\relax
\DeclareMathOperator{\log}{log}
%%%%%%

%% NUOVI COMANDI
\newcommand{\straniero}[1]{\textit{#1}} %parole straniere
\newcommand{\titolo}[1]{\textsc{#1}} %titoli
\newcommand{\qedd}{\tag*{$\blacksquare$}} %qed per ambienti matemastici
\renewcommand{\qedsymbol}{$\blacksquare$} %modifica colore qed
\newcommand{\ooverline}[1]{\overline{\overline{#1}}}
\newcommand{\circoletto}[1]{\left(#1\right)^{\text{o}}}
%
\newcommand{\qmatrice}[1]{\begin{pmatrix}
#1_{11} & \cdots & #1_{1n}\\
\vdots & \ddots & \vdots \\
#1_{m1} & \cdots & #1_{mn}
\end{pmatrix}}
%
\newcommand{\parentesi}[2]{%
\underset{#1}{\underbrace{#2}}%
}
%
\newcommand{\norma}[1]{% Norma
\left\lVert#1\right\rVert%
}
\newcommand{\scalare}[2]{% Scalare
\left\langle #1, #2\right\rangle
}
%%%%%

%% RESTRIZIONI
\newcommand{\referenze}[2]{
        \phantomsection{}#2\textsuperscript{\textcolor{blue}{\textbf{#1}}}
}

\let\restriction\relax

\def\restriction#1#2{\mathchoice
              {\setbox1\hbox{${\displaystyle #1}_{\scriptstyle #2}$}
              \restrictionaux{#1}{#2}}
              {\setbox1\hbox{${\textstyle #1}_{\scriptstyle #2}$}
              \restrictionaux{#1}{#2}}
              {\setbox1\hbox{${\scriptstyle #1}_{\scriptscriptstyle #2}$}
              \restrictionaux{#1}{#2}}
              {\setbox1\hbox{${\scriptscriptstyle #1}_{\scriptscriptstyle #2}$}
              \restrictionaux{#1}{#2}}}
\def\restrictionaux#1#2{{#1\,\smash{\vrule height .8\ht1 depth .85\dp1}}_{\,#2}}
%%%%%%%%%%%

%%% FORMATTAZIONE FOOTNOTEMARK

\def\footnotemarkformatting#1{[#1]}
\renewcommand{\thefootnote}{\footnotemarkformatting{\arabic{footnote}}}

%% SEZIONE GRAFICA
\use{tikz}
\usetikzlibrary{matrix, patterns, calc, decorations.pathreplacing, hobby, decorations.markings, decorations.pathmorphing, babel}
\use{tikz-3dplot}
\use{mathrsfs} %per geogebra
\use{tikz-cd}
\tikzset
{
  %surface/.style={fill=black!10, shading=ball,fill opacity=0.4},
  plane/.style={black,pattern=north east lines},
  curve/.style={black,line width=0.5mm},
  dritto/.style={decoration={markings,mark=at position 0.5 with {\arrow{Stealth}}}, postaction=decorate},
  rovescio/.style={decoration={markings,mark=at position 0.5 with {\arrow{Stealth[reversed]}}}, postaction=decorate}
}
\use{pgfplots} % stampare le funzioni
        \pgfplotsset{/pgf/number format/use comma,compat=1.15}
        %\pgfplotsset{compat=1.15} %per geogebra
        \usepgfplotslibrary{fillbetween, polar}
%%%%%%

%% CITAZIONI
\use{lineno}

\newcommand{\citazione}[1]{%
  \begin{quotation}
  \begin{linenumbers}
  \modulolinenumbers[5]
  \begingroup
  \setlength{\parindent}{0cm}
  \noindent #1
  \endgroup
  \end{linenumbers}
  \end{quotation}\setcounter{linenumber}{1}
  }
%%%%%%

%%%%%%%%%%%%%%%%%%%%%%%%%%%%%%%%%%%%%%%%%%%%
%%%%%%%%%%%%%%%%%%%%%%%%%%%%%%%%%%%%%%%%%%%%

%% AMS THM

\theoremstyle{definition}% default
\newtheorem{thm}{Teorema}[section]
\newtheorem{lem}[thm]{Lemma}
\newtheorem{prop}[thm]{Proposizione}
\newtheorem{cor}[thm]{Corollario}
\newtheorem{esempio}[thm]{Esempio}
\theoremstyle{plain}
\newtheorem{definizione}[thm]{Definizione}
\theoremstyle{remark}
\newtheorem*{oss}{Osservazione}


%%%%%%%%%%%%%%%%%%%%%%%%%%%%%%%%%%%%%%%%%%%%
%%%%%%%%%%%%%%%%%%%%%%%%%%%%%%%%%%%%%%%%%%%%

\use{hyperref}
\hypersetup{%
        pdfauthor={Davide Peccioli},
        pdfsubject={},
        allcolors=black,
        citecolor=black,
%	colorlinks=true,
        bookmarksopen=true}
\setcounter{secnumdepth}{0} % rimuove i numeri di sezione senza rimuovere le ref
\renewcommand{\href}[2]{\textcolor{blue}{#2}} % disabilita il comando href
\use{enotez} %
\setenotez{%
 mark-format = \footnotemarkformatting % Mette i numeri tra parentesi quadre%
}\let\footnote=\endnote % rende tutte le note a pié pagina come delle note a fine file 


\let\olddocument\document % modifico l'ambiende documenti per non dover stampare \printendnote
\let\oldenddocument\enddocument
\renewenvironment{document}%
{%
  \olddocument
}{%
  \printendnotes\oldenddocument
}
\renewcommand{\thethm}{\arabic{thm}}

\usepackage[hyperref]{biblatex}
\addbibresource{~/Documents/org/roam/bib/master.bib}
\def\Hk#1{\mathrm{H}^{#1}_{\text{dR}}}
\def\Toro{\mathds{T}}
\def\ToroBucato{\tilde{\Toro}}
\author{Davide Peccioli}
\date{\today}
\title{}
\begin{document}

\section{Coomologia delle superfici topologiche compatte orientabili}
\label{sec:org8b5e570}
\begin{thm}
Se \(\Sigma_{g}\) è la \href{20251230172241-superficie_topologica.org}{superficie topologica} \href{20250103163701-spazio_topologico_compatto.org}{compatta} \href{20251223152054-varieta_differenziabile_orientabile.org}{orientabile} di genere \(g\), allora la \href{20251115172442-gruppo_di_coomologia_di_de_rham.org}{Coomologia di De Rham} è\footnote{Vedi ``\href{20241213095808-somma_diretta.org}{Somma Diretta}''}
\begin{equation*}
\Hk{k}(\Sigma_{g}) \cong \begin{cases}
\R & k = 0,2\\
\R^{g} \oplus \R^{g} & k = 1\\
0 & \text{altrimenti}
\end{cases}
\end{equation*}
dove ``\(\cong\)'' è un \href{20250113125833-isomorfismo_tra_spazi_vettoriali.org}{isomorfismo}.
\label{thm:coomsigmag}
\end{thm}
\begin{lem}
La coomologia del toro bucato, \(\ToroBucato \coloneqq \Toro \setminus\set{\text{punto}}\) è
\begin{equation*}
\Hk{k}(\ToroBucato) \cong %
\begin{cases}
\R & k = 0\\
\R^{2} & k = 1\\
0 & k\ge 0
\end{cases}
\end{equation*}
\label{lem:coomtorobuc}
\end{lem}
\begin{proof}
Dato il Toro \(\Toro\), \href{20251115190941-coomologia_del_toro.org}{di cui si conosce la coomologia}\footnote{Si ha che
\begin{equation*}
\Hk{k}(\Toro) \cong
\begin{cases}
\R & k = 0,2\\
\R^{2} & k=1
\end{cases}
\end{equation*}}, si considera:
\begin{equation*}
     \Toro = \ToroBucato \cup D
\end{equation*}
dove \(\ToroBucato = \Toro\setminus\set{p}\) e \(D\) è un intorno aperto attorno a \(p\), omotopo a \(\R^{2}\), \href{20251115173712-coomologia_di_de_rham_di_r2.org}{di cui si conosca le coomologia}\footnote{Si ha che
\begin{equation*}
\Hk{k}(\R^{2}) \cong %
\begin{cases}
\R & k=0\\
O & k\neq 0
\end{cases}
\end{equation*}}. Si noti che \(\ToroBucato\cap D\) è omotopo a \(\mathds{S}^{1}\), \href{20251115184223-coomologia_della_circonferenza.org}{di cui si conosce la coomologia}\footnote{Si ha che
\begin{equation*}
\Hk{k}(\mathds{S}^{1}) \cong %
\begin{cases}
\R & k = 0,1\\
0 & k \ge 2
\end{cases}
\end{equation*}}. Si scrive quindi \href{20251115183635-teorema_di_mayer_vietoris_in_coomologia.org}{Mayer-Vietoris}:
\begin{equation*}
\begin{tikzcd}[column sep=small]
        0 & {\Hk{0}(\Toro)} & {\Hk{0}(\ToroBucato)\oplus\Hk{0}(D)} & {\Hk{0}(\ToroBucato \cap D)} \\
        & {\Hk{1}(\Toro)} & {\Hk{1}(\ToroBucato)\oplus\Hk{1}(D)} & {\Hk{1}(\ToroBucato \cap D)} \\
        & {\Hk{2}(\Toro)} & {\Hk{2}(\ToroBucato)\oplus\Hk{2}(D)} & {\Hk{2}(\ToroBucato \cap D)} & 0
        \arrow[from=1-1, to=1-2]
        \arrow[from=1-2, to=1-3]
        \arrow[from=1-3, to=1-4]
        \arrow[from=1-4, to=2-2]
        \arrow[from=2-2, to=2-3]
        \arrow[from=2-3, to=2-4]
        \arrow[from=2-4, to=3-2]
        \arrow[from=3-2, to=3-3]
        \arrow[from=3-3, to=3-4]
        \arrow[from=3-4, to=3-5]
\end{tikzcd}
\end{equation*}
Si possono semplificare alcuni termini:
\begin{itemize}
\item siccome tutti gli spazi sono connessi, \href{20251115174538-0_gruppo_di_coomologia_di_de_rham_di_una_varieta_connessa.org}{ogni \(\Hk{0}\) è isomorfo a \(\R\)} (in particolare, \(\boxed{\Hk{0}(\ToroBucato) \cong \R}\));
\item siccome \(\ToroBucato\) è \href{20251223152054-varieta_differenziabile_orientabile.org}{orientabile} ma \href{20250103163701-spazio_topologico_compatto.org}{non compatta}, \href{20251117121206-coomologia_in_dimensione_massima.org}{allora} \(\boxed{\Hk{2}(\ToroBucato) = 0}\);
\item si sostituiscono gli \(\Hk{k}(D) \cong \Hk{k}(\R^{2})\) e \(\Hk{k}(\ToroBucato\cap D)\cong \Hk{k}(\mathds{S}^{1})\), nonché gli elementi della coomologia di \(\Toro\).
\end{itemize}
La successione esatta è necessaria solo più per calcolare \(\Hk{1}(\ToroBucato)\):
\begin{equation*}
\begin{tikzcd}[column sep=small]
        0 & \R & {\R\oplus\R} & \R \\
        & {\R^2} & {\Hk{1}(\ToroBucato)} & \R \\
        & \R & 0 & 0 & 0
        \arrow[from=1-1, to=1-2]
        \arrow[from=1-2, to=1-3]
        \arrow[from=1-3, to=1-4]
        \arrow[from=1-4, to=2-2]
        \arrow[from=2-2, to=2-3]
        \arrow[from=2-3, to=2-4]
        \arrow[from=2-4, to=3-2]
        \arrow["\beta"', from=3-2, to=3-3]
        \arrow[from=3-3, to=3-4]
        \arrow[from=3-4, to=3-5]
\end{tikzcd}
\end{equation*}
(\emph{Non dimostrato}: \(\Hk{1}(\ToroBucato)\) ha dimensione finita). Utilizzando la \href{20251115182133-successione_di_spazi_vettoriali_esatta.org}{somma alterna delle dimensioni}, si ottiene che
\begin{equation*}
     -1+2-1+2-\dim \Hk{1}(\ToroBucato) +1 -1 = 0
\end{equation*}
ovvero \(\dim \Hk{1}(\ToroBucato) = 2\), \(\boxed{\Hk{1}(\ToroBucato) \cong \R^{2}}\).
\end{proof}
\begin{proof}
(Del Teorema~\ref{thm:coomsigmag})
\uline{Caso \(k=0\)}: siccome \(\Sigma_{g}\) è \href{20250103165325-spazio_topologico_connesso.org}{connessa}, \href{20251115174538-0_gruppo_di_coomologia_di_de_rham_di_una_varieta_connessa.org}{allora} \(\boxed{\Hk{0}(\Sigma_{g}) = \R}\).

\uline{Caso \(k=2\) e \(k\ge 2\)}: \(\Sigma_{g}\) è una \href{20250113115909-struttura_differenziabile.org}{varietà differenziabile} di dimensione \(2\).
\begin{itemize}
\item per ogni \(k>2\): \(\Hk{k}(\Sigma_{g}) = 0\), per questione di dimensione;
\item per \(k=2\): siccome \(\Sigma_{g}\) è compatta e orientabile, allora (\href{20251117121206-coomologia_in_dimensione_massima.org}{Coomologia in dimensione massima})
\begin{equation*}
  \boxed{\Hk{2}(\Sigma_{g}) \cong \R}
\end{equation*}
\end{itemize}

\uline{Caso \(k=1\)}: Consideriamo \(\Sigma_{g}\) rappresentato in Figura~\ref{fig:multitoro_normale}: possiamo dividerlo nei due insiemi \(U,V\), rappresentati in Figura~\ref{fig:multitoro_UeV}:
\begin{itemize}
\item \(U\) è \href{20250124155008-spazi_topologici_omotopicamente_equivalenti.org}{omotopo} alla superficie \(\tilde{\Sigma}_{g-1} \coloneqq \Sigma_{g-1} \setminus \set{\text{punto}}\);
\item \(V\) è omotopo a \(\ToroBucato \coloneqq \Toro \setminus \set{\text{punto}}\), dove \(\Toro\) è il \href{20250113125113-toro.org}{Toro}.
\item \(U\cap V\) è omotopo a \(\mathds{S}^1\), come si vede in Figura~\ref{fig:multitoro_UcapV}.
\end{itemize}

Siccome si vuole utilizzare la \href{20251115183635-teorema_di_mayer_vietoris_in_coomologia.org}{sequenza di Mayer-Vietoris}, (sfruttando l'\href{20251223145901-teorema_di_invarianza_omotopica_per_la_coomologia_di_de_rham.org}{invarianza della coomologia per omotopia}) è necessario calcolare la coomologia di \(\tilde{\Sigma}_{g-1}\). Si procede per induzione
\begin{itemize}
\item \uline{Caso base}: \(\Sigma_{1} = \Toro\), è verificato.
\item \uline{Ipotesi induttiva}: si supponga vera l'ipotesi per \(\Sigma_{g-1}\):
\begin{equation*}
\Hk{1}(\Sigma_{g-1}) \cong \R^{g-1} \oplus \R^{g-1}
\end{equation*}
Si vuole dimostrare la tesi per \(\Sigma_{g}\). Si procede scrivendo la successione di Mayer-Vietoris, utilizzando \(U\) e \(V\) come sopra.
\begin{equation*}
\begin{tikzcd}[column sep=small]
        0 & {\Hk{0}(\Sigma_g)} & {\Hk{0}(U)\oplus \Hk{0}(V)} & {\Hk{0}(U\cap V)} \\
        & {\Hk{1}(\Sigma_g)} & {\Hk{1}(U)\oplus \Hk{1}(V)} & {\Hk{1}(U\cap V)} \\
        & {\Hk{2}(\Sigma_g)} & {\Hk{2}(U)\oplus \Hk{2}(V)} & {\Hk{2}(U\cap V)} & 0
        \arrow[from=1-1, to=1-2]
        \arrow[from=1-2, to=1-3]
        \arrow[from=1-3, to=1-4]
        \arrow[from=1-4, to=2-2]
        \arrow[from=2-2, to=2-3]
        \arrow[from=2-3, to=2-4]
        \arrow[from=2-4, to=3-2]
        \arrow[from=3-2, to=3-3]
        \arrow[from=3-3, to=3-4]
        \arrow[from=3-4, to=3-5]
\end{tikzcd}
\end{equation*}
Sostituendo utilizzando le considerazioni sull'omotopia:
\begin{equation*}
\begin{tikzcd}[column sep=small]
        0 & {\Hk{0}(\Sigma_g)} & {\Hk{0}(\tilde{\Sigma}_{g-1})\oplus \Hk{0}(\ToroBucato)} & {\Hk{0}(\mathds{S}^1)} \\
        & {\Hk{1}(\Sigma_g)} & {\Hk{1}(\tilde{\Sigma}_{g-1})\oplus \Hk{1}(\ToroBucato)} & {\Hk{1}(\mathds{S}^1)} \\
        & {\Hk{2}(\Sigma_g)} & {\Hk{2}(\tilde{\Sigma}_{g-1})\oplus \Hk{2}(\ToroBucato)} & {\Hk{2}(\mathds{S}^1)} & 0
        \arrow[from=1-1, to=1-2]
        \arrow[from=1-2, to=1-3]
        \arrow[from=1-3, to=1-4]
        \arrow[from=1-4, to=2-2]
        \arrow[from=2-2, to=2-3]
        \arrow[from=2-3, to=2-4]
        \arrow[from=2-4, to=3-2]
        \arrow[from=3-2, to=3-3]
        \arrow[from=3-3, to=3-4]
        \arrow[from=3-4, to=3-5]
\end{tikzcd}
\end{equation*}
\begin{itemize}
\item Tutte le superfici in gioco sono connesse, quindi \(\Hk{0} \cong \R\);
\item Si conoscono la coomologia \href{20251115184223-coomologia_della_circonferenza.org}{di \(\mathds{S}^{1}\)} e \(\ToroBucato\)\footnote{Vedi il Lemma~\ref{lem:coomtorobuc}}, nonché le coomologie di dimensione \(0\) e \(2\) si \(\Sigma_{g}\);
\item \(\tilde{\Sigma}_{g-1}\) è una superficie orientabile non compatta, e \href{20251117121206-coomologia_in_dimensione_massima.org}{quindi} \(\Hk{2}(\tilde{\Sigma}_{g-1}) \cong 0\).
\end{itemize}
Resta quindi:
\begin{equation*}
\begin{tikzcd}[column sep=small]
        0 & \R & {\R\oplus \R} & \R \\
        & {\Hk{1}(\Sigma_g)} & {\Hk{1}(\tilde{\Sigma}_{g-1})\oplus \R^2} & \R \\
        & \R & 0 & 0 & 0
        \arrow[from=1-1, to=1-2]
        \arrow[from=1-2, to=1-3]
        \arrow[from=1-3, to=1-4]
        \arrow[from=1-4, to=2-2]
        \arrow[from=2-2, to=2-3]
        \arrow[from=2-3, to=2-4]
        \arrow[from=2-4, to=3-2]
        \arrow[from=3-2, to=3-3]
        \arrow[from=3-3, to=3-4]
        \arrow[from=3-4, to=3-5]
\end{tikzcd}
\end{equation*}
Utilizzando la \href{20251115182133-successione_di_spazi_vettoriali_esatta.org}{somma alterna delle dimensioni} si ottiene:
\begin{align*}
  0 &= 1 - 2 + 1 - \dim \Hk{1}(\Sigma_{g}) + \dim \Hk{1}({\Sigma}_{g-1}) + 2 - 1 + 1\\
  \dim \Hk{1}(\Sigma_{g}) &= 1 - 2 + 1  + \dim \Hk{1}({\Sigma}_{g-1}) + 2 - 1 + 1 = %
  \dim \Hk{1}({\Sigma}_{g-1}) + 2. %
   \tag{$\star\star$}
\end{align*}

È qui che entra in gioco l'ipotesi induttiva. Se si considera \(\Sigma_{g-1} = A \cup B\), con \(A = \Sigma_{g-1} \setminus \set{q}\) e \(B\) intorno aperto di \(q\) omotopo a \(\R^{2}\), si ottiene che:
\begin{itemize}
\item \(\Hk{k}(A) \cong \Hk{k}(\tilde{\Sigma}_{g-1})\), in quanto sono omotopi, di cui si conoscono tutte le coomologie tranne \(\Hk{1}\):
\begin{equation*}
	\Hk{0}(\tilde{\Sigma}_{g-1}) \cong \R, \qquad \Hk{2}(\tilde{\Sigma}_{g-1}) = 0;
\end{equation*}

\item \(\Hk{k}(B) \cong \Hk{k}(\R^{2})\), \href{20251115173712-coomologia_di_de_rham_di_r2.org}{coomologia nota};

\item \(\Hk{k}(A \cap B) \cong \Hk{k}(\mathds{S}^{1})\), \href{20251115184223-coomologia_della_circonferenza.org}{coomologia nota}.
\end{itemize}

Scrivendo Mayer-Vietoris:
\begin{equation*}
\begin{tikzcd}[column sep=small]
        0 & {\Hk{0}(\Sigma_{g-1})} & {\Hk{0}(\tilde{\Sigma}_{g-1})\oplus\Hk{0}(\R^2)} & {\Hk{0}(\mathds{S}^1)} \\
        & {\Hk{1}(\Sigma_{g-1})} & {\Hk{1}(\tilde{\Sigma}_{g-1})\oplus\Hk{1}(\R^2)} & {\Hk{1}(\mathds{S}^1)} \\
        & {\Hk{2}(\Sigma_{g-1})} & {\Hk{2}(\tilde{\Sigma}_{g-1})\oplus\Hk{2}(\R^2)} & {\Hk{2}(\mathds{S}^1)} & 0
        \arrow[from=1-1, to=1-2]
        \arrow[from=1-2, to=1-3]
        \arrow[from=1-3, to=1-4]
        \arrow[from=1-4, to=2-2]
        \arrow[from=2-2, to=2-3]
        \arrow[from=2-3, to=2-4]
        \arrow[from=2-4, to=3-2]
        \arrow[from=3-2, to=3-3]
        \arrow[from=3-3, to=3-4]
        \arrow[from=3-4, to=3-5]
\end{tikzcd}
\end{equation*}

Per ipotesi induttiva si conoscono tutte le \(\Hk{k}(\Sigma_{g-1})\), e pertanto, sostituendo tutte le coomologie note:
\begin{equation*}
\begin{tikzcd}[column sep=small]
        0 & \R & {\R\oplus \R} & \R \\
        & {\R^{g-1} \oplus \R^{g-1}} & {\Hk{1}(\tilde{\Sigma}_{g-1})\oplus 0} & \R \\
        & \R & {0 \oplus 0} & 0 & 0
        \arrow[from=1-1, to=1-2]
        \arrow[from=1-2, to=1-3]
        \arrow[from=1-3, to=1-4]
        \arrow[from=1-4, to=2-2]
        \arrow[from=2-2, to=2-3]
        \arrow[from=2-3, to=2-4]
        \arrow[from=2-4, to=3-2]
        \arrow[from=3-2, to=3-3]
        \arrow[from=3-3, to=3-4]
        \arrow[from=3-4, to=3-5]
\end{tikzcd}
\end{equation*}
e facendo la \href{20251115182133-successione_di_spazi_vettoriali_esatta.org}{somma alterna delle dimensioni}:
\begin{equation*}
   \dim \Hk{1}(\tilde{\Sigma}_{g-1}) = -1+2-1+2(g-1)  +1 -1 = 2(g-1)
\end{equation*}
che, applicata in (\(\star\star\)):
\begin{equation*}
  \boxed{\Hk{1}(\Sigma_{g}) \cong \R^{g} \oplus \R^{g}}.%
  \qedhere
\end{equation*}
\end{itemize}
\end{proof}
\begin{figure}
\begin{equation*}
\begin{tikzpicture}[scale=1.2, line width=0.8pt]

    % Definizione corretta del comando \buco con 2 argomenti
    \newcommand{\buco}[2]{
        \draw (#1-0.5, #2) .. controls (#1-0.1, #2-0.2) and (#1+0.1, #2-0.2) .. (#1+0.5, #2);
        \draw (#1-0.4, #2-0.04) .. controls (#1-0.1, #2+0.2) and (#1+0.1, #2+0.2) .. (#1+0.4, #2-0.04);
    }

    % -- Parte Sinistra: Sigma_{g-1} senza un punto (U) --
    \draw (-4, 0) to[out=90, in=180] (-3, 0.8)
                  to[out=0, in=180] (-2, 0.4)
                  to[out=0, in=180] (-1, 0.8)
                  to[out=0, in=180] (0, 0.4); % Collo aperto a destra

    \draw (-4, 0) to[out=-90, in=180] (-3, -0.8)
                  to[out=0, in=180] (-2, -0.4)
                  to[out=0, in=180] (-1, -0.8)
                  to[out=0, in=180] (0, -0.4); % Collo aperto a destra

    \buco{-3}{0}
    \buco{-1}{0}

    % -- Parte Centrale: Tratteggio e sovrapposizione --
    \draw[dashed] (0, 0.4) -- (1.13, 0.4);
    \draw[dashed] (0, -0.4) -- (1.13, -0.4);


    % Puntini di sospensione per il genere generico
    \node at (0.5, 0) {$\dots$};

    % -- Parte Destra: Toro bucato (V) --
    \draw (1.13, 0.4) -- (1.5, 0.4);
    \draw (1.13, -0.4) -- (1.5, -0.4);

    \draw (1.5, 0.4) to[out=0, in=180] (2.5, 0.8)
                   to[out=0, in=90] (3.5, 0);

    \draw (1.5, -0.4) to[out=0, in=180] (2.5, -0.8)
                    to[out=0, in=-90] (3.5, 0);

    \buco{2.5}{0}

\end{tikzpicture}
\end{equation*}
\caption{\label{fig:multitoro_normale}La superficie \(\Sigma_{g}\)}
\end{figure}


\begin{figure}
\begin{equation*}
\begin{tikzpicture}[scale=1.2, line width=0.8pt]

    % Definizione corretta del comando \buco con 2 argomenti
    \newcommand{\buco}[2]{
        \draw (#1-0.5, #2) .. controls (#1-0.1, #2-0.2) and (#1+0.1, #2-0.2) .. (#1+0.5, #2);
        \draw (#1-0.4, #2-0.04) .. controls (#1-0.1, #2+0.2) and (#1+0.1, #2+0.2) .. (#1+0.4, #2-0.04);
    }

    % -- Parte Sinistra: Sigma_{g-1} senza un punto (U) --
    \draw (-4, 0) to[out=90, in=180] (-3, 0.8)
                  to[out=0, in=180] (-2, 0.4)
                  to[out=0, in=180] (-1, 0.8)
                  to[out=0, in=180] (0, 0.4); % Collo aperto a destra

    \draw (-4, 0) to[out=-90, in=180] (-3, -0.8)
                  to[out=0, in=180] (-2, -0.4)
                  to[out=0, in=180] (-1, -0.8)
                  to[out=0, in=180] (0, -0.4); % Collo aperto a destra

    \buco{-3}{0}
    \buco{-1}{0}

    % -- Parte Centrale: Tratteggio e sovrapposizione --
    \draw[dashed] (0, 0.4) -- (1.13, 0.4);
    \draw[dashed] (0, -0.4) -- (1.13, -0.4);


    % Puntini di sospensione per il genere generico
    \node at (0.5, 0) {$\dots$};

    % -- Parte Destra: Toro bucato (V) --
    \draw (1.13, 0.4) -- (1.5, 0.4);
    \draw (1.13, -0.4) -- (1.5, -0.4);

    \draw (1.5, 0.4) to[out=0, in=180] (2.5, 0.8)
                   to[out=0, in=90] (3.5, 0);

    \draw (1.5, -0.4) to[out=0, in=180] (2.5, -0.8)
                    to[out=0, in=-90] (3.5, 0);

    \buco{2.5}{0}

    \draw[red, line width=1.2pt, dashed] (1.2, -0.4) -- (1.2, 0.4);
    \draw[red] (1.2, -0.5) -- (1.2, -1.1) -- (3.5, -1.1) -- (3.5, -0.6);
    \node at (2.35,-1.3) {\textcolor{red}{\(V\)}};

    \draw[blue, line width=1.2pt, dashed] (1.4, -0.4) -- (1.4, 0.4);
    \draw[blue] (1.4, 0.5) -- (1.4, 1.1) -- (-4, 1.1) -- (-4, 0.3);
    \node at (-1.3,1.3) {\textcolor{blue}{\(U\)}};
\end{tikzpicture}
\end{equation*}
\caption{\label{fig:multitoro_UeV}Gli insiemi \(U\) e \(V\)}
\end{figure}

\begin{figure}
\begin{equation*}
\begin{tikzpicture}[scale=1.2, line width=0.8pt]

    \fill[color=purple!40] (1.2, -0.4) rectangle (1.4, 0.4);

%    \fill[pattern=north west lines, pattern color=red] (1.2, -0.4) rectangle (1.4, 0.4);

    % Definizione corretta del comando \buco con 2 argomenti
    \newcommand{\buco}[2]{
        \draw (#1-0.5, #2) .. controls (#1-0.1, #2-0.2) and (#1+0.1, #2-0.2) .. (#1+0.5, #2);
        \draw (#1-0.4, #2-0.04) .. controls (#1-0.1, #2+0.2) and (#1+0.1, #2+0.2) .. (#1+0.4, #2-0.04);
    }

    % -- Parte Sinistra: Sigma_{g-1} senza un punto (U) --
    \draw (-4, 0) to[out=90, in=180] (-3, 0.8)
                  to[out=0, in=180] (-2, 0.4)
                  to[out=0, in=180] (-1, 0.8)
                  to[out=0, in=180] (0, 0.4); % Collo aperto a destra

    \draw (-4, 0) to[out=-90, in=180] (-3, -0.8)
                  to[out=0, in=180] (-2, -0.4)
                  to[out=0, in=180] (-1, -0.8)
                  to[out=0, in=180] (0, -0.4); % Collo aperto a destra

    \buco{-3}{0}
    \buco{-1}{0}

    % -- Parte Centrale: Tratteggio e sovrapposizione --
    \draw[dashed] (0, 0.4) -- (1.13, 0.4);
    \draw[dashed] (0, -0.4) -- (1.13, -0.4);


    % Puntini di sospensione per il genere generico
    \node at (0.5, 0) {$\dots$};

    % -- Parte Destra: Toro bucato (V) --
    \draw (1.13, 0.4) -- (1.5, 0.4);
    \draw (1.13, -0.4) -- (1.5, -0.4);

    \draw (1.5, 0.4) to[out=0, in=180] (2.5, 0.8)
                   to[out=0, in=90] (3.5, 0);

    \draw (1.5, -0.4) to[out=0, in=180] (2.5, -0.8)
                    to[out=0, in=-90] (3.5, 0);

    \buco{2.5}{0}

    \draw[red, line width=1.2pt, dashed] (1.2, -0.4) -- (1.2, 0.4);

    \draw[blue, line width=1.2pt, dashed] (1.4, -0.4) -- (1.4, 0.4);

% 2. ETICHETTA E FRECCIA U CAP V:
    % Definiamo un nodo per l'etichetta posizionato in basso a destra
    \node at (2.55, -1.65) {\textcolor{purple}{$U \cap V$}};

    % Disegniamo la freccia. 'shorten >=' fa fermare la punta poco prima del target.
    % Usiamo 'to[out=..., in=...]' per una freccia curva più elegante.
    \draw[<-, thick, purple, shorten >= 3pt] (2.5, -1.5) to[out=150, in=-80] (1.3, -0.1);
\end{tikzpicture}
\end{equation*}
\caption{\label{fig:multitoro_UcapV}L'insieme \(U\cap V\)}
\end{figure}
\end{document}
