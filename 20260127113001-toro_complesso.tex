% Created 2026-02-07 Sat 19:35
% Intended LaTeX compiler: pdflatex
\documentclass[10pt]{article}
%% CREATO CON ORG - EMACS
\newcommand{\use}[2][]{\usepackage[#1]{#2}}
% PACCHETTI FONDAMENTLAI
\use[utf8]{inputenc}
\use[T1]{fontenc}
\use{graphicx}
\use{longtable}
\use{wrapfig}
\use{rotating}
\use[normalem]{ulem}
\use{amsmath}
\use{amsthm}
\use{amssymb}

\use{eucal} % Cambia mathcal{...}

\use{capt-of}
\use[italian]{babel}
\use[babel]{csquotes}
% bib la TEX lo carica in automatico org-cite
\use{microtype}
\use{lmodern}
\use{subfig} % sottofigure
\use{multicol} % due colonne
\use{lipsum} % lorem ipsum
\use{color} % colori in latex
\use{parskip} % rimuove l'indentazione dei nuovi paragrafi %% Add parbox=false to all new tcolorbox
\use{centernot}
\use[outline]{contour}\contourlength{3pt}
\use{fancyhdr}
\use{layout}
\use[most]{tcolorbox} % Riquadri colorati
\use{ifthen} % IFTHEN
\use{geometry}

% pacchetti matematica
\use{yhmath}
\use{dsfont}
\use{mathrsfs}
\use{cancel} % semplificare
\use{polynom} %divisione tra polinomi
\use{forest} % grafi ad albero
\use{booktabs} % tabelle
\use{commath} %simboli e differenziali
\use{bm} %bold
\use[fulladjust]{marginnote} %to use marginnote for date notes
\use{arrayjobx}%array
\use[intlimits]{empheq} % Riquadri colorati attorno alle equazioni
\use{mathtools}
\use{circuitikz} % Disegnare i circuiti
\use{mathtools}
\use{stmaryrd} % [[ \llbracket ]] \rrbracket
\use{bussproofs} % dimostrazioni

%%%%%%%%%%%%%


%%%% QUIVER
\newcommand{\duepunti}{\,\mathchar\numexpr"6000+`:\relax\,}
% A TikZ style for curved arrows of a fixed height, due to AndréC.
\tikzset{curve/.style={settings={#1},to path={(\tikztostart)
    .. controls ($(\tikztostart)!\pv{pos}!(\tikztotarget)!\pv{height}!270:(\tikztotarget)$)
    and ($(\tikztostart)!1-\pv{pos}!(\tikztotarget)!\pv{height}!270:(\tikztotarget)$)
    .. (\tikztotarget)\tikztonodes}},
    settings/.code={\tikzset{quiver/.cd,#1}
        \def\pv##1{\pgfkeysvalueof{/tikz/quiver/##1}}},
    quiver/.cd,pos/.initial=0.35,height/.initial=0}

% TikZ arrowhead/tail styles.
\tikzset{tail reversed/.code={\pgfsetarrowsstart{tikzcd to}}}
\tikzset{2tail/.code={\pgfsetarrowsstart{Implies[reversed]}}}
\tikzset{2tail reversed/.code={\pgfsetarrowsstart{Implies}}}
% TikZ arrow styles.
\tikzset{no body/.style={/tikz/dash pattern=on 0 off 1mm}}
%%%%%%%%%%


%% DEFINIZIONI COMANDI MATEMATICI
\let\sin\relax %TOGLIE LA DEFINIZIONE SU "\sin"

% cambia la definizione di empty set
% ---
\let\oldemptyset\emptyset
% ---
% \let\emptyset\varnothing
% ---
% \let\emptyset\relax
% \newcommand{\emptyset}{\text{\textnormal{\O}}}
% ---

\DeclareMathOperator{\bounded}{bd}
\DeclareMathOperator{\sin}{sen}
\DeclareMathOperator{\epi}{Epi}
\DeclareMathOperator{\cl}{cl}
\DeclareMathOperator{\graph}{graph}
\DeclareMathOperator{\arcsec}{arcsec}
\DeclareMathOperator{\arccot}{arccot}
\DeclareMathOperator{\arccsc}{arccsc}
\DeclareMathOperator{\spettro}{Spettro}
\DeclareMathOperator{\nulls}{nullspace}
\DeclareMathOperator{\dom}{dom}
\DeclareMathOperator{\ar}{ar}
\DeclareMathOperator{\const}{Const}
\DeclareMathOperator{\fun}{Fun}
\DeclareMathOperator{\rel}{Rel}
\DeclareMathOperator{\altezza}{ht}
\let\det\relax %TOGLIE LA DEFINIZIONE SU "\det"
\DeclareMathOperator{\det}{det}
\DeclareMathOperator{\End}{End}
\DeclareMathOperator{\gl}{GL}
\def\Id{\mathrm{Id}}
\def\id{\mathrm{id}}
\DeclareMathOperator{\I}{\mathds{1}}
\DeclareMathOperator{\II}{II}
\DeclareMathOperator{\rank}{rank}
\DeclareMathOperator{\tr}{tr}
\DeclareMathOperator{\tc}{t.c.}
\DeclareMathOperator{\T}{T}
\DeclareMathOperator{\var}{Var}
\DeclareMathOperator{\cov}{Cov}
\DeclareMathOperator{\st}{st}
\DeclareMathOperator{\mon}{Mon}
\newcommand{\card}[1]{\left\vert #1 \right\vert}
\newcommand{\trasposta}[1]{\prescript{\text{T}}{}{#1}}
\newcommand{\1}{\mathds{1}}
\newcommand{\R}{\mathds{R}}
\newcommand{\diesis}{\#}
\newcommand{\bemolle}{\flat}
\newcommand{\nonstandard}[1]{\prescript{*}{}{#1}}
\newcommand{\starR}{\nonstandard{\R}}
\newcommand{\borel}{\mathscr{B}}
\newcommand{\lebesgue}[1]{\mathscr{L}\left(#1\right)}
\newcommand{\media}{\mathds{E}}
\newcommand{\K}{\mathds{K}}
\newcommand{\A}{\mathds{A}}
\newcommand{\Q}{\mathds{Q}}
\newcommand{\N}{\mathds{N}}
\newcommand{\C}{\mathds{C}}
\newcommand{\Z}{\mathds{Z}}
\newcommand{\qo}{\hspace{1em}\text{q.o.}\,}
\renewcommand{\tilde}[1]{\widetilde{#1}}
\renewcommand{\parallel}{\mathrel{/\mkern-5mu/}}
\newcommand{\parti}[2][]{\wp_{#1}(#2)}
\newcommand{\diff}[1]{\operatorname{d}_{#1}}
\let\oldvec\vec
\renewcommand{\vec}[1]{\overrightarrow{\vphantom{i}#1}}
\newcommand{\floor}[1]{\left\lfloor #1 \right\rfloor}
\newcommand{\cat}[1]{\mathbf{#1}}
\newcommand{\dfreccia}[1]{\xrightarrow{\ #1 \ }}
\newcommand{\sfreccia}[1]{\xleftarrow{\ #1 \ }}
\newcommand{\formalsum}[2]{{\sum_{#1}^{#2}}{\vphantom{\sum}}'}
\newcommand{\minim}[2]{\mu_{#1}\, \left(#2\right)}
\newcommand{\concat}{\null^{\frown}} % concatenazione di stringe
\newcommand{\godelcode}[1]{\langle\!\langle #1 \rangle\!\rangle}
\newcommand{\godeldec}[1]{(\!(#1)\!)}
\newcommand{\termcode}[1]{\ulcorner #1\urcorner}
\newcommand{\partialto}{\dashrightarrow}
\newcommand{\restricted}{\upharpoonright}
\newcommand{\embeds}{\precsim}
\newcommand{\surjects}{\twoheadrightarrow}
\newcommand{\equipotenti}{\asymp}
%% \newcommand{\dotplus}{\mathbin{\dot{+}}} %% A quanto pare esiste già
\newcommand{\bigdot}{\mathbin{\boldsymbol{\cdot}}}
\newcommand{\dotexp}[1]{^{.#1}}
\newcommand{\conv}{\mathbin{*}}
\newcommand{\convolution}[2]{(#1\conv #2)}
\newcommand{\nil}{\mathfrak{N}}
\newcommand{\divisore}{\mathrel{|}}
\newcommand{\simplesso}[1]{\mathrm{e}_{#1}}

\renewcommand{\iff}{\mathrel{\longleftrightarrow}} %% Notazione Logica.
\newcommand{\oldiff}{\mathrel{\Longleftrightarrow}}
\renewcommand{\implies}{\mathrel{\rightarrow}} %% Notazione Logica
\newcommand{\oldimplies}{\mathrel{\Longrightarrow}}
\renewcommand{\impliedby}{\mathrel{\leftarrow}} %% Notazione Logica
\newcommand{\oldimpliedby}{\mathrel{\Longleftarrow}}

\newcommand{\IFF}{\quad\Longleftrightarrow\quad}
\newcommand{\IMPLICA}{\quad\Longrightarrow\quad}


\renewcommand{\descriptionlabel}[1]{\hspace{\labelsep}\normalfont #1} % remove bold from description


%% Definizione di Divergenza di K-L

\DeclarePairedDelimiterX{\infdivx}[2]{(}{)}{%
  #1\;\delimsize\|\;#2%
}
\newcommand{\kldiv}{D_{KL}\infdivx}

%% Definizione di \dotminus

\makeatletter
\newcommand{\dotminus}{\mathbin{\text{\@dotminus}}}

\newcommand{\@dotminus}{%
  \ooalign{\hidewidth\raise1ex\hbox{.}\hidewidth\cr$\m@th-$\cr}%
}
\makeatother

%tramite i prossimi due comandi posso decidere come scrivere i logaritmi naturali in tutti i documenti: ho infatti eliminato qualsiasi differenza tra "ln" e "log": se si vuole qualcosa di diverso bisogna inserire manualmente il tutto
\let\ln\relax
\DeclareMathOperator{\ln}{ln}
\let\log\relax
\DeclareMathOperator{\log}{log}
%%%%%%

%% NUOVI COMANDI
\newcommand{\straniero}[1]{\textit{#1}} %parole straniere
\newcommand{\titolo}[1]{\textsc{#1}} %titoli
\newcommand{\qedd}{\tag*{$\blacksquare$}} %qed per ambienti matemastici
\renewcommand{\qedsymbol}{$\blacksquare$} %modifica colore qed
\newcommand{\ooverline}[1]{\overline{\overline{#1}}}
\newcommand{\circoletto}[1]{\left(#1\right)^{\text{o}}}
%
\newcommand{\qmatrice}[1]{\begin{pmatrix}
#1_{11} & \cdots & #1_{1n}\\
\vdots & \ddots & \vdots \\
#1_{m1} & \cdots & #1_{mn}
\end{pmatrix}}
%
\newcommand{\parentesi}[2]{%
\underset{#1}{\underbrace{#2}}%
}
%
\newcommand{\norma}[1]{% Norma
\left\lVert#1\right\rVert%
}
\newcommand{\scalare}[2]{% Scalare
\left\langle #1, #2\right\rangle
}
%%%%%

%% RESTRIZIONI
\newcommand{\referenze}[2]{
        \phantomsection{}#2\textsuperscript{\textcolor{blue}{\textbf{#1}}}
}

\let\restriction\relax

\def\restriction#1#2{\mathchoice
              {\setbox1\hbox{${\displaystyle #1}_{\scriptstyle #2}$}
              \restrictionaux{#1}{#2}}
              {\setbox1\hbox{${\textstyle #1}_{\scriptstyle #2}$}
              \restrictionaux{#1}{#2}}
              {\setbox1\hbox{${\scriptstyle #1}_{\scriptscriptstyle #2}$}
              \restrictionaux{#1}{#2}}
              {\setbox1\hbox{${\scriptscriptstyle #1}_{\scriptscriptstyle #2}$}
              \restrictionaux{#1}{#2}}}
\def\restrictionaux#1#2{{#1\,\smash{\vrule height .8\ht1 depth .85\dp1}}_{\,#2}}
%%%%%%%%%%%

%%% FORMATTAZIONE FOOTNOTEMARK

\def\footnotemarkformatting#1{[#1]}
\renewcommand{\thefootnote}{\footnotemarkformatting{\arabic{footnote}}}

%% SEZIONE GRAFICA
\use{tikz}
\usetikzlibrary{matrix, patterns, calc, decorations.pathreplacing, hobby, decorations.markings, decorations.pathmorphing, babel}
\use{tikz-3dplot}
\use{mathrsfs} %per geogebra
\use{tikz-cd}
\tikzset
{
  %surface/.style={fill=black!10, shading=ball,fill opacity=0.4},
  plane/.style={black,pattern=north east lines},
  curve/.style={black,line width=0.5mm},
  dritto/.style={decoration={markings,mark=at position 0.5 with {\arrow{Stealth}}}, postaction=decorate},
  rovescio/.style={decoration={markings,mark=at position 0.5 with {\arrow{Stealth[reversed]}}}, postaction=decorate}
}
\use{pgfplots} % stampare le funzioni
        \pgfplotsset{/pgf/number format/use comma,compat=1.15}
        %\pgfplotsset{compat=1.15} %per geogebra
        \usepgfplotslibrary{fillbetween, polar}
%%%%%%

%% CITAZIONI
\use{lineno}

\newcommand{\citazione}[1]{%
  \begin{quotation}
  \begin{linenumbers}
  \modulolinenumbers[5]
  \begingroup
  \setlength{\parindent}{0cm}
  \noindent #1
  \endgroup
  \end{linenumbers}
  \end{quotation}\setcounter{linenumber}{1}
  }
%%%%%%

%%%%%%%%%%%%%%%%%%%%%%%%%%%%%%%%%%%%%%%%%%%%
%%%%%%%%%%%%%%%%%%%%%%%%%%%%%%%%%%%%%%%%%%%%

%% AMS THM

\theoremstyle{definition}% default
\newtheorem{thm}{Teorema}[section]
\newtheorem{lem}[thm]{Lemma}
\newtheorem{prop}[thm]{Proposizione}
\newtheorem{cor}[thm]{Corollario}
\newtheorem{esempio}[thm]{Esempio}
\theoremstyle{plain}
\newtheorem{definizione}[thm]{Definizione}
\theoremstyle{remark}
\newtheorem*{oss}{Osservazione}


%%%%%%%%%%%%%%%%%%%%%%%%%%%%%%%%%%%%%%%%%%%%
%%%%%%%%%%%%%%%%%%%%%%%%%%%%%%%%%%%%%%%%%%%%

\use{hyperref}
\hypersetup{%
        pdfauthor={Davide Peccioli},
        pdfsubject={},
        allcolors=black,
        citecolor=black,
%	colorlinks=true,
        bookmarksopen=true}
\setcounter{secnumdepth}{0} % rimuove i numeri di sezione senza rimuovere le ref
\renewcommand{\href}[2]{\textcolor{blue}{#2}} % disabilita il comando href
\use{enotez} %
\setenotez{%
 mark-format = \footnotemarkformatting % Mette i numeri tra parentesi quadre%
}\let\footnote=\endnote % rende tutte le note a pié pagina come delle note a fine file 


\let\olddocument\document % modifico l'ambiende documenti per non dover stampare \printendnote
\let\oldenddocument\enddocument
\renewenvironment{document}%
{%
  \olddocument
}{%
  \printendnotes\oldenddocument
}
\renewcommand{\thethm}{\arabic{thm}}

\usepackage[hyperref]{biblatex}
\addbibresource{~/Documents/org/roam/bib/master.bib}
\author{Davide Peccioli}
\date{\today}
\title{}
\begin{document}

\section{Toro complesso}
\label{sec:org2006f8e}
\subsection{Costruzione come quoziente di gruppi}
\label{sec:orgbe43676}

In \(\C^{n}\) si considerino \(2n\) vettori \(\set{w_{1},\dots,w_{2n}}\), \href{20241212142019-insiemi_linearmente_indipendenti.org}{linearmente indipendenti} su \(\R\).

Si consideri ora il \href{20241206143051-sottogruppo.org}{sottogruppo} additivo di \(\C^{n}\) \href{20260115123905-sottogruppo_generato.org}{generato} da \(\set{w_{1},\dots,w_{2n}}\):
\begin{equation*}
\Gamma \coloneqq \set{m_{1}w_{1} + \dots + m_{2n}w_{2n} \mid m_{i} \in \Z}
\end{equation*}
Alcune proprietà:
\begin{itemize}
\item \(\Gamma\) è \href{20241206115531-morfismo_di_gruppi.org}{isomorfo} a \(\Z^{2n}\) come \href{20241205141146-gruppo_abeliano.org}{gruppo},
\item \(\Gamma \subseteq \C^{n}\) è chiuso;
\item \(\Gamma\) ha la \href{20250317165247-topologia_discreta.org}{topologia discreta}, ovvero ogni suo punto è \href{20250403131856-punto_isolato.org}{isolato}.
\end{itemize}
\(\Gamma\) prende il nome di \uline{reticolo}.

Sia quindi \(X\coloneqq \C^{n}/\Gamma\) il \href{20250127093819-quoziente_di_gruppo_e_sottogruppo.org}{gruppo quoziente}, con proiezione suriettiva:
\begin{equation*}
\pi: \C^{n}\to X.
\end{equation*}
Si dota quindi \(X\) della \href{20250129155316-spazio_topologico_quoziente.org}{topologia quoziente}.

\begin{itemize}
\item \textbf{Topologia di \(X\)}:

Definiamo la mappa \(\phi: \C^{n}\to \R^{2n}:\ w_{i}\mapsto e_{i}\), dove \(\set{e_{i}}\) è la \href{20250102163502-base_di_uno_spazio_vettoriale.org}{base} canonica di \(\R^{2n}\). Questa è \(\R\)-\href{20250114101949-funzione_lineare.org}{lineare} ed è un \href{20250111142332-omeomorfismo.org}{omeomorfismo}.

Quindi è una \href{20250104114559-funzione_chiusa.org}{mappa chiusa}, e, siccome \(\Gamma\) è chiuso, \href{20250104114559-funzione_chiusa.org}{anche} \(\restriction{\phi}{\Gamma}\) è chiusa. Inoltre\footnote{Vedi ``\href{20250202173528-dominio_range_e_campo_di_una_classe_relazione.org}{Range di una funzione}''}
\begin{equation*}
  \phi(\Gamma) = \Z^{2n}
\end{equation*}
e \href{20250104114559-funzione_chiusa.org}{pertanto} \(\phi\) è un omeomorfismo tra \(\Gamma\) e \(\Z^{2n}\).

Inoltre, ovviamente, \(\phi\) induce un isomorfismo tra i gruppi:
\begin{itemize}
\item \(\R^{2n}\) e \(\C^{n}\);
\item \(\Gamma\) e \(\Z^{2n}\).
\end{itemize}

Pertanto\footnote{Questa cosa non mi è per nulla chiara}
\begin{equation*}
  X \mathrel{\overset{\text{omeo}}{\approx}} \frac{\R^{2n}}{\Z^{2n}} \mathrel{\overset{\text{omeo}}{\approx}} (\mathds{S}^{1})^{2n}
\end{equation*}
e quindi \(X\) è una \href{20250111092123-varieta_topologica.org}{varietà topologica} \href{20250103163701-spazio_topologico_compatto.org}{compatta}, orientabile e di dimensione pari.

\item \textbf{Mappa \(\pi\)}

Siccome \(X\) è il \href{20260128105613-azione_di_gruppo_su_uno_spazio_topologico.org}{quoziente per l'azione di \(\Gamma\) su \(\C^{n}\)}, allora è una \href{20250104114559-funzione_chiusa.org}{mappa aperta}.
\end{itemize}

Possiamo ora costruire un atlante complesso per \(X\).

\begin{enumerate}
\item Siccome \(0 \in \Gamma\) e \(\Gamma\) è discreto, esiste un intorno aperto \(U\) di \(0\) in \(\C\) tale che \(\Gamma\cap U = \set{0}\). Ovvero
\begin{equation*}
\exists \varepsilon >0\ \tc\quad \forall w \in \Gamma\setminus\set{0}\ \norma{w} > 2\varepsilon.
\end{equation*}

Allora, per ogni \(z_{0} \in \C^{n}\) fissato, detto
\begin{equation*}
D(z_{0};\varepsilon) \coloneqq \set{z \in \C^{n} \mid \norma{z-z_{0}}<\varepsilon}
\end{equation*}
la mappa
\begin{equation*}
\restriction{\pi}{D(z_{0};\varepsilon)}: D(z_{0};\varepsilon) \to X
\end{equation*}
è \href{20241219101956-funzione_iniettiva.org}{iniettiva}: se \(z_{1},z_{2} \in D\) sono tali che \(\pi(z_{1})=\pi(z_{2})\) allora
\begin{equation*}
z_{1}-z_{2} \in \Gamma:\qquad \norma{z_{1}-z_{2}} \le \norma{z_{1}-z_{0}} + \norma{z_{0}-z_{2}} < 2\varepsilon
\end{equation*}
e quindi \(z_{1}-z_{2} = 0\).

\item Sia quindi \(U_{z_{0}} \coloneqq \pi[D(z_{0};\varepsilon)]\). Siccome \(\pi\) è una mappa aperta, \(U_{z_{0}} \subseteq X\) è un aperto. Inoltre
\begin{equation*}
 \restriction{\pi}{D(z_{0};\varepsilon)}:D(z_{0};\varepsilon) \to U_{z_{0}}
\end{equation*}
è un \href{20250111142332-omeomorfismo.org}{omeomorfismo}, poiché è una funzione continua, biiettiva e aperta\footnote{Infatti \href{20250104114559-funzione_chiusa.org}{restrizione di mappe aperte ad un aperto sono ancora aperte}.}.

Ammette quindi un inverso \(\varphi_{z_{0}} \coloneqq \big(	\restriction{\pi}{D(z_{0};\varepsilon)}\big)^{-1}\):
\begin{equation*}
 \varphi_{z_{0}}:U_{z_{0}}\to D(z_{0};\varepsilon).
\end{equation*}

Quindi \((U_{z_{0}};\varphi_{z_{0}})\) è una carta locale.
\end{enumerate}

Si propone \(\set{(U_{z_{0}};\varphi_{z_{0}})}_{z_{0} \in \C^{n}}\) come \href{20260127112715-atlante_complesso.org}{atlante complesso}.
\begin{itemize}
\item Sicuramente \(\displaystyle X = \bigcup_{z_{0} \in \C^{n}} U_{z_{0}}\).
\item Si dimostra la compatibilità: siano \(z_{0},z_{1} \in \C^{n}\) tali che \(U_{z_{0}}\cap U_{z_{1}} \neq \emptyset\):
\begin{equation*}
\begin{tikzcd}[ampersand replacement=\&]
        \&\& {U_{z_0}\cap U_{z_1}} \\
        \\
        {D(z_0;\varepsilon)\supseteq A_0} \&\&\&\& {A_1 \subseteq D(z_1;\varepsilon)}
        \arrow["{\varphi_{z_0}}", from=1-3, to=3-1]
        \arrow["{\varphi_{z_1}}"', from=1-3, to=3-5]
        \arrow["\pi", shift left=3, from=3-1, to=1-3]
        \arrow["h"', from=3-1, to=3-5]
        \arrow["\pi"', shift right=3, from=3-5, to=1-3]
\end{tikzcd}
\end{equation*}

Per ogni \(z \in A_{0}\) si ha
\begin{equation*}
  \pi(z) = \pi(h(z))
\end{equation*}
e pertanto \(z-h(z) \in \Gamma\).

Se consideriamo l'inclusione \(\iota: A_{0} \subseteq \C^{n} \hookarrow \C^{n}\) si ha che
\begin{equation*}
  i-h : A_{0} \to \C
\end{equation*}
ha immagine in \(\Gamma\) insieme discreto. Inoltre \(i-h\) è continua, e pertanto è \href{20250325153824-funzione_localmente_costante.org}{localmente costante}.

\href{20250325154046-funzione_localmente_costante_sse_costante_sulle_componenti_connesse.org}{Pertanto}, \(i-h\) è costante sulle \href{20250325160128-componente_connessa_di_uno_spazio_topologico.org}{componenti connesse} di \(A_{0}\), ovvero per ogni \(\Omega \subseteq A_{0}\) componente connessa (aperta) esiste \(\omega_{\Omega} \in \Gamma\) tale che
\begin{equation*}
  \forall z \in \Omega:\qquad h(z) = z+\omega_{\Omega}.
\end{equation*}

Pertanto \(\restriction{h}{\Omega}\) è olomorfa, e quindi lo è \(h\).
\end{itemize}
\end{document}
