% Created 2026-02-07 Sat 19:34
% Intended LaTeX compiler: pdflatex
\documentclass[10pt]{article}
%% CREATO CON ORG - EMACS
\newcommand{\use}[2][]{\usepackage[#1]{#2}}
% PACCHETTI FONDAMENTLAI
\use[utf8]{inputenc}
\use[T1]{fontenc}
\use{graphicx}
\use{longtable}
\use{wrapfig}
\use{rotating}
\use[normalem]{ulem}
\use{amsmath}
\use{amsthm}
\use{amssymb}

\use{eucal} % Cambia mathcal{...}

\use{capt-of}
\use[italian]{babel}
\use[babel]{csquotes}
% bib la TEX lo carica in automatico org-cite
\use{microtype}
\use{lmodern}
\use{subfig} % sottofigure
\use{multicol} % due colonne
\use{lipsum} % lorem ipsum
\use{color} % colori in latex
\use{parskip} % rimuove l'indentazione dei nuovi paragrafi %% Add parbox=false to all new tcolorbox
\use{centernot}
\use[outline]{contour}\contourlength{3pt}
\use{fancyhdr}
\use{layout}
\use[most]{tcolorbox} % Riquadri colorati
\use{ifthen} % IFTHEN
\use{geometry}

% pacchetti matematica
\use{yhmath}
\use{dsfont}
\use{mathrsfs}
\use{cancel} % semplificare
\use{polynom} %divisione tra polinomi
\use{forest} % grafi ad albero
\use{booktabs} % tabelle
\use{commath} %simboli e differenziali
\use{bm} %bold
\use[fulladjust]{marginnote} %to use marginnote for date notes
\use{arrayjobx}%array
\use[intlimits]{empheq} % Riquadri colorati attorno alle equazioni
\use{mathtools}
\use{circuitikz} % Disegnare i circuiti
\use{mathtools}
\use{stmaryrd} % [[ \llbracket ]] \rrbracket
\use{bussproofs} % dimostrazioni

%%%%%%%%%%%%%


%%%% QUIVER
\newcommand{\duepunti}{\,\mathchar\numexpr"6000+`:\relax\,}
% A TikZ style for curved arrows of a fixed height, due to AndréC.
\tikzset{curve/.style={settings={#1},to path={(\tikztostart)
    .. controls ($(\tikztostart)!\pv{pos}!(\tikztotarget)!\pv{height}!270:(\tikztotarget)$)
    and ($(\tikztostart)!1-\pv{pos}!(\tikztotarget)!\pv{height}!270:(\tikztotarget)$)
    .. (\tikztotarget)\tikztonodes}},
    settings/.code={\tikzset{quiver/.cd,#1}
        \def\pv##1{\pgfkeysvalueof{/tikz/quiver/##1}}},
    quiver/.cd,pos/.initial=0.35,height/.initial=0}

% TikZ arrowhead/tail styles.
\tikzset{tail reversed/.code={\pgfsetarrowsstart{tikzcd to}}}
\tikzset{2tail/.code={\pgfsetarrowsstart{Implies[reversed]}}}
\tikzset{2tail reversed/.code={\pgfsetarrowsstart{Implies}}}
% TikZ arrow styles.
\tikzset{no body/.style={/tikz/dash pattern=on 0 off 1mm}}
%%%%%%%%%%


%% DEFINIZIONI COMANDI MATEMATICI
\let\sin\relax %TOGLIE LA DEFINIZIONE SU "\sin"

% cambia la definizione di empty set
% ---
\let\oldemptyset\emptyset
% ---
% \let\emptyset\varnothing
% ---
% \let\emptyset\relax
% \newcommand{\emptyset}{\text{\textnormal{\O}}}
% ---

\DeclareMathOperator{\bounded}{bd}
\DeclareMathOperator{\sin}{sen}
\DeclareMathOperator{\epi}{Epi}
\DeclareMathOperator{\cl}{cl}
\DeclareMathOperator{\graph}{graph}
\DeclareMathOperator{\arcsec}{arcsec}
\DeclareMathOperator{\arccot}{arccot}
\DeclareMathOperator{\arccsc}{arccsc}
\DeclareMathOperator{\spettro}{Spettro}
\DeclareMathOperator{\nulls}{nullspace}
\DeclareMathOperator{\dom}{dom}
\DeclareMathOperator{\ar}{ar}
\DeclareMathOperator{\const}{Const}
\DeclareMathOperator{\fun}{Fun}
\DeclareMathOperator{\rel}{Rel}
\DeclareMathOperator{\altezza}{ht}
\let\det\relax %TOGLIE LA DEFINIZIONE SU "\det"
\DeclareMathOperator{\det}{det}
\DeclareMathOperator{\End}{End}
\DeclareMathOperator{\gl}{GL}
\def\Id{\mathrm{Id}}
\def\id{\mathrm{id}}
\DeclareMathOperator{\I}{\mathds{1}}
\DeclareMathOperator{\II}{II}
\DeclareMathOperator{\rank}{rank}
\DeclareMathOperator{\tr}{tr}
\DeclareMathOperator{\tc}{t.c.}
\DeclareMathOperator{\T}{T}
\DeclareMathOperator{\var}{Var}
\DeclareMathOperator{\cov}{Cov}
\DeclareMathOperator{\st}{st}
\DeclareMathOperator{\mon}{Mon}
\newcommand{\card}[1]{\left\vert #1 \right\vert}
\newcommand{\trasposta}[1]{\prescript{\text{T}}{}{#1}}
\newcommand{\1}{\mathds{1}}
\newcommand{\R}{\mathds{R}}
\newcommand{\diesis}{\#}
\newcommand{\bemolle}{\flat}
\newcommand{\nonstandard}[1]{\prescript{*}{}{#1}}
\newcommand{\starR}{\nonstandard{\R}}
\newcommand{\borel}{\mathscr{B}}
\newcommand{\lebesgue}[1]{\mathscr{L}\left(#1\right)}
\newcommand{\media}{\mathds{E}}
\newcommand{\K}{\mathds{K}}
\newcommand{\A}{\mathds{A}}
\newcommand{\Q}{\mathds{Q}}
\newcommand{\N}{\mathds{N}}
\newcommand{\C}{\mathds{C}}
\newcommand{\Z}{\mathds{Z}}
\newcommand{\qo}{\hspace{1em}\text{q.o.}\,}
\renewcommand{\tilde}[1]{\widetilde{#1}}
\renewcommand{\parallel}{\mathrel{/\mkern-5mu/}}
\newcommand{\parti}[2][]{\wp_{#1}(#2)}
\newcommand{\diff}[1]{\operatorname{d}_{#1}}
\let\oldvec\vec
\renewcommand{\vec}[1]{\overrightarrow{\vphantom{i}#1}}
\newcommand{\floor}[1]{\left\lfloor #1 \right\rfloor}
\newcommand{\cat}[1]{\mathbf{#1}}
\newcommand{\dfreccia}[1]{\xrightarrow{\ #1 \ }}
\newcommand{\sfreccia}[1]{\xleftarrow{\ #1 \ }}
\newcommand{\formalsum}[2]{{\sum_{#1}^{#2}}{\vphantom{\sum}}'}
\newcommand{\minim}[2]{\mu_{#1}\, \left(#2\right)}
\newcommand{\concat}{\null^{\frown}} % concatenazione di stringe
\newcommand{\godelcode}[1]{\langle\!\langle #1 \rangle\!\rangle}
\newcommand{\godeldec}[1]{(\!(#1)\!)}
\newcommand{\termcode}[1]{\ulcorner #1\urcorner}
\newcommand{\partialto}{\dashrightarrow}
\newcommand{\restricted}{\upharpoonright}
\newcommand{\embeds}{\precsim}
\newcommand{\surjects}{\twoheadrightarrow}
\newcommand{\equipotenti}{\asymp}
%% \newcommand{\dotplus}{\mathbin{\dot{+}}} %% A quanto pare esiste già
\newcommand{\bigdot}{\mathbin{\boldsymbol{\cdot}}}
\newcommand{\dotexp}[1]{^{.#1}}
\newcommand{\conv}{\mathbin{*}}
\newcommand{\convolution}[2]{(#1\conv #2)}
\newcommand{\nil}{\mathfrak{N}}
\newcommand{\divisore}{\mathrel{|}}
\newcommand{\simplesso}[1]{\mathrm{e}_{#1}}

\renewcommand{\iff}{\mathrel{\longleftrightarrow}} %% Notazione Logica.
\newcommand{\oldiff}{\mathrel{\Longleftrightarrow}}
\renewcommand{\implies}{\mathrel{\rightarrow}} %% Notazione Logica
\newcommand{\oldimplies}{\mathrel{\Longrightarrow}}
\renewcommand{\impliedby}{\mathrel{\leftarrow}} %% Notazione Logica
\newcommand{\oldimpliedby}{\mathrel{\Longleftarrow}}

\newcommand{\IFF}{\quad\Longleftrightarrow\quad}
\newcommand{\IMPLICA}{\quad\Longrightarrow\quad}


\renewcommand{\descriptionlabel}[1]{\hspace{\labelsep}\normalfont #1} % remove bold from description


%% Definizione di Divergenza di K-L

\DeclarePairedDelimiterX{\infdivx}[2]{(}{)}{%
  #1\;\delimsize\|\;#2%
}
\newcommand{\kldiv}{D_{KL}\infdivx}

%% Definizione di \dotminus

\makeatletter
\newcommand{\dotminus}{\mathbin{\text{\@dotminus}}}

\newcommand{\@dotminus}{%
  \ooalign{\hidewidth\raise1ex\hbox{.}\hidewidth\cr$\m@th-$\cr}%
}
\makeatother

%tramite i prossimi due comandi posso decidere come scrivere i logaritmi naturali in tutti i documenti: ho infatti eliminato qualsiasi differenza tra "ln" e "log": se si vuole qualcosa di diverso bisogna inserire manualmente il tutto
\let\ln\relax
\DeclareMathOperator{\ln}{ln}
\let\log\relax
\DeclareMathOperator{\log}{log}
%%%%%%

%% NUOVI COMANDI
\newcommand{\straniero}[1]{\textit{#1}} %parole straniere
\newcommand{\titolo}[1]{\textsc{#1}} %titoli
\newcommand{\qedd}{\tag*{$\blacksquare$}} %qed per ambienti matemastici
\renewcommand{\qedsymbol}{$\blacksquare$} %modifica colore qed
\newcommand{\ooverline}[1]{\overline{\overline{#1}}}
\newcommand{\circoletto}[1]{\left(#1\right)^{\text{o}}}
%
\newcommand{\qmatrice}[1]{\begin{pmatrix}
#1_{11} & \cdots & #1_{1n}\\
\vdots & \ddots & \vdots \\
#1_{m1} & \cdots & #1_{mn}
\end{pmatrix}}
%
\newcommand{\parentesi}[2]{%
\underset{#1}{\underbrace{#2}}%
}
%
\newcommand{\norma}[1]{% Norma
\left\lVert#1\right\rVert%
}
\newcommand{\scalare}[2]{% Scalare
\left\langle #1, #2\right\rangle
}
%%%%%

%% RESTRIZIONI
\newcommand{\referenze}[2]{
        \phantomsection{}#2\textsuperscript{\textcolor{blue}{\textbf{#1}}}
}

\let\restriction\relax

\def\restriction#1#2{\mathchoice
              {\setbox1\hbox{${\displaystyle #1}_{\scriptstyle #2}$}
              \restrictionaux{#1}{#2}}
              {\setbox1\hbox{${\textstyle #1}_{\scriptstyle #2}$}
              \restrictionaux{#1}{#2}}
              {\setbox1\hbox{${\scriptstyle #1}_{\scriptscriptstyle #2}$}
              \restrictionaux{#1}{#2}}
              {\setbox1\hbox{${\scriptscriptstyle #1}_{\scriptscriptstyle #2}$}
              \restrictionaux{#1}{#2}}}
\def\restrictionaux#1#2{{#1\,\smash{\vrule height .8\ht1 depth .85\dp1}}_{\,#2}}
%%%%%%%%%%%

%%% FORMATTAZIONE FOOTNOTEMARK

\def\footnotemarkformatting#1{[#1]}
\renewcommand{\thefootnote}{\footnotemarkformatting{\arabic{footnote}}}

%% SEZIONE GRAFICA
\use{tikz}
\usetikzlibrary{matrix, patterns, calc, decorations.pathreplacing, hobby, decorations.markings, decorations.pathmorphing, babel}
\use{tikz-3dplot}
\use{mathrsfs} %per geogebra
\use{tikz-cd}
\tikzset
{
  %surface/.style={fill=black!10, shading=ball,fill opacity=0.4},
  plane/.style={black,pattern=north east lines},
  curve/.style={black,line width=0.5mm},
  dritto/.style={decoration={markings,mark=at position 0.5 with {\arrow{Stealth}}}, postaction=decorate},
  rovescio/.style={decoration={markings,mark=at position 0.5 with {\arrow{Stealth[reversed]}}}, postaction=decorate}
}
\use{pgfplots} % stampare le funzioni
        \pgfplotsset{/pgf/number format/use comma,compat=1.15}
        %\pgfplotsset{compat=1.15} %per geogebra
        \usepgfplotslibrary{fillbetween, polar}
%%%%%%

%% CITAZIONI
\use{lineno}

\newcommand{\citazione}[1]{%
  \begin{quotation}
  \begin{linenumbers}
  \modulolinenumbers[5]
  \begingroup
  \setlength{\parindent}{0cm}
  \noindent #1
  \endgroup
  \end{linenumbers}
  \end{quotation}\setcounter{linenumber}{1}
  }
%%%%%%

%%%%%%%%%%%%%%%%%%%%%%%%%%%%%%%%%%%%%%%%%%%%
%%%%%%%%%%%%%%%%%%%%%%%%%%%%%%%%%%%%%%%%%%%%

%% AMS THM

\theoremstyle{definition}% default
\newtheorem{thm}{Teorema}[section]
\newtheorem{lem}[thm]{Lemma}
\newtheorem{prop}[thm]{Proposizione}
\newtheorem{cor}[thm]{Corollario}
\newtheorem{esempio}[thm]{Esempio}
\theoremstyle{plain}
\newtheorem{definizione}[thm]{Definizione}
\theoremstyle{remark}
\newtheorem*{oss}{Osservazione}


%%%%%%%%%%%%%%%%%%%%%%%%%%%%%%%%%%%%%%%%%%%%
%%%%%%%%%%%%%%%%%%%%%%%%%%%%%%%%%%%%%%%%%%%%

\use{hyperref}
\hypersetup{%
        pdfauthor={Davide Peccioli},
        pdfsubject={},
        allcolors=black,
        citecolor=black,
%	colorlinks=true,
        bookmarksopen=true}
\setcounter{secnumdepth}{0} % rimuove i numeri di sezione senza rimuovere le ref
\renewcommand{\href}[2]{\textcolor{blue}{#2}} % disabilita il comando href
\use{enotez} %
\setenotez{%
 mark-format = \footnotemarkformatting % Mette i numeri tra parentesi quadre%
}\let\footnote=\endnote % rende tutte le note a pié pagina come delle note a fine file 


\let\olddocument\document % modifico l'ambiende documenti per non dover stampare \printendnote
\let\oldenddocument\enddocument
\renewenvironment{document}%
{%
  \olddocument
}{%
  \printendnotes\oldenddocument
}
\renewcommand{\thethm}{\arabic{thm}}

\usepackage[hyperref]{biblatex}
\addbibresource{~/Documents/org/roam/bib/master.bib}
\author{Davide Peccioli}
\date{\today}
\title{}
\begin{document}

\section{Teorema di Ramsey}
\label{sec:org69f0694}
Si veda la Sezione~15.1 di \autocite{zambellaCrecheCourseModel2025a}

\def\M{\mathcal{M}}
\def\U{\mathcal{U}}
\def\L{\mathcal{L}}
\def\equivalentover#1{\mathrel{\equiv_{#1}}}
\def\restricted#1{\,\mathord{\upharpoonright}{\scriptstyle #1}}
%% Alcuni simboli
\def\nonforkSymbol{\mathbin{\raise1.8ex\rlap{\kern0.6ex\rule{0.6ex}{0.1ex}}\rlap{\kern1.1ex\rule{0.1ex}{1.9ex}}\raise-0.3ex\hbox{$\smile$} } }
\renewcommand{\nonfork}[1][M]{\mathrel{\nonforkSymbol_{#1}}}

\uline{Notazione}:
\begin{itemize}
\item per \(k \in \N\), \((k] = \set{1,\dots,k}\)
\item per \(I\) insieme e \(\lambda\) \href{20250203161341-cardinali.org}{cardinale}, si indica con
\begin{equation*}
  \binom{I}{\lambda} = I^{(\lambda)} =  \set{J \subseteq I \mid \card{J} = \lambda}.
\end{equation*}
l'insieme dei \href{20250131155822-operazioni_insiemistiche_tra_classi_mk.org}{sottoinsiemi} di \(I\) di \href{20241213101756-cardinalita.org}{cardinalità} fissata.
\end{itemize}

\begin{definizione}
Una \uline{colorazione} per un insieme \(M^{(n)}\) è una funzione
\begin{equation*}
f: M^{(n)} \to (k]
\end{equation*}
per qualche \(n,k \in \omega\) fissato.
\end{definizione}

\begin{definizione}
Dato un insieme \(M\) ed una colorazione \(f:M^{(n)}\to (k]\), \(H \subseteq M\) è \uline{monocromatico} se la \href{20250205170515-restrizione_di_una_classe.org}{restrizione} \(f \upharpoonright H^{(n)}\) è costante.
\end{definizione}

\begin{thm}
Per ogni \(M\) insieme infinito e per ogni colorazione
\begin{equation*}
f: \binom{M}{n}\to (k]
\end{equation*}
esiste \(H \subseteq M\) infinito e monocromatico.
\end{thm}

\begin{proof}
Sia \(\L\) il \href{20250130162057-linguaggio_del_prim_ordine.org}{linguaggio del prim'ordine} contenente \(k\) relazioni \(n\)-arie:
\begin{equation*}
\L = \set{r_{i} \mid i \in (k]}
\end{equation*}
e sia \(\M\) la \(\L\)-struttura di dominio \(M\), in cui
\begin{equation*}
r_{i}^{\M} \coloneqq \set{(a_{1},\dots,a_{n}) \mid f\big(\set{a_{1},\dots,a_{n}}\big) = i}.
\end{equation*}

Sia \(\U \succeq \mathcal{M}\)\footnote{con ``\(\preceq\)'' si intende che \(\mathcal{M}\) è \href{20250212102253-sottostruttura_elementare.org}{Sottostruttura elementare} di \(\U\).} un \href{20250617095548-modello_lambda_saturo.org}{modello saturo} \href{20250617102733-modello_mostro.org}{mostro}. Si noti che esiste una \href{20250131103317-formula_del_prim_ordine.org}{formula del prim'ordine} che afferma che \(r_{i}(x_{1},\dots,x_{n})\) rappresentano una colorazione di \(M^{(n)}\) (ovvero che le relazioni sono simmetriche e sono un ``ricoprimento''), e pertanto, per elementarietà, questo vale anche in \(\U\).

Sia \(x\) una variabile di lunghezza 1 e sia \(q(x)\) il tipo \(x\notin M\). Questo è \href{20250212164424-tipo_teoria_dei_modelli.org}{finitamente soddisfacibile}, e \href{20251026195832-estensione_tipo_finitamente_soddisfacibile_in_un_insieme_a_tipo_massimale_e_completo.org}{pertanto} esiste un \href{20250617102733-modello_mostro.org}{tipo globale} \(p(x) \supseteq q(x)\), \(p(x) \in S(\U)\) finitamente soddisfacibile in \(M\). \href{20251026195746-tipo_finitamente_soddisfacibile_e_invariante.org}{Inoltre} \(p(x)\) è invariante su \(M\).

Sia \(\overline{c} = \langle c_{i} \mid i<\omega \rangle\) una \href{20251026211057-sequenza_di_coeredi.org}{sequenza di coeredi di \(p(x)\) su \(M\)}.

In particolare, siccome \(\overline{c}\) è un \href{20251026210840-sequenza_di_morley.org}{sequenza di Morley} su \(p(x)\) invariante su \(M\), \href{20251026211005-sequenza_di_morely_inviariante_e_indiscernibile.org}{allora} è una \href{20251026210908-sequenza_di_indiscernibili.org}{sequenza di indiscernibili} su \(M\). Pertanto per ogni \(I,J \in \omega^{(n)}\)
\begin{equation*}
c \restricted{I} \equivalentover{M} c \restricted{J}.
\end{equation*}
e dunque, per ogni \(i=1,\dots,k\)
\begin{equation*}
r_{i}(c\restricted{I}) \IFF r_{i}(c \restricted{J})
\end{equation*}
Segue che \(\set{c_{i} \mid i <\omega}\) è monocromatico (WLOG di colore 1). Si osservi però che questa sequenza è in \(\U\), non in \(M\).

Si vuole costruire per induzione una sequenza \(\overline{a} = \langle a_{i} : i <\omega \rangle\) in \(M\) monocromatica di colore 1.

Per induzione su \(i\): si supponga per ipotesi induttiva che tutte le sottosequenze di lunghezza \(n\) di
\begin{equation*}
(a\restricted{i}) \concat (c\restricted{n}),
\end{equation*}
dove ``\(\concat\)'' rappresenta la concatenazione tra \href{20250206170922-sequenze_e_stringhe.org}{stringhe}, abbiano lo stesso colore 1. Si cerca quindi \(a_{i} \in M\) in maniera che
\begin{equation*}
(a\restricted{i}) \textcolor{red}{\concat a_{i}} \concat (c\restricted{n}) = (a\restricted{i+1}) \concat (c\restricted{n})
\end{equation*}
abbia la stessa proprietà.

Siccome \(\overline{c}\) è sequenza di indiscernibili su \(M\)\footnote{Ovvero per ogni \(I,J \in \omega^{(n)}\)
\begin{equation*}
c \restricted{I} \equivalentover{M} c \restricted{J}.
\end{equation*}} in particolare
\begin{equation*}
(a\restricted{i}) \concat (c\restricted{n}) \concat c_{n}
\end{equation*}
ha tutte le sottostringhe di lunghezza \(n\) di colore 1 (poiché \(a\restricted{i} \subseteq M\)).

In particolare esiste una formula \(\varphi \in \L\) tale che \(\varphi(a\restricted{i};\, c \restricted{n};\, c_{n})\) affermi che tutte le sottosequenze di lunghezza \(n\) della stringa abbiano colore 1.
Questa formula per sua natura è simmetrica nelle variabili libere.

Per la \href{20251026211057-sequenza_di_coeredi.org}{caratterizzazione delle sequenze di coeredi} si ha che
\begin{equation*}
c_{n} \nonfork c \restricted{n}
\end{equation*}
e dunque per \href{20251029154447-tuple_indipendenti_rispetto_ad_un_modello_monster_model.org}{definizione} esiste \(\bm{a} \in M\) tale che
\begin{equation*}
\varphi(a\restricted{i};\, c \restricted{n};\, \textcolor{red}{\bm{a}})
\end{equation*}
e, per l'osservazione di cui sopra,
\begin{equation*}
\varphi(a\restricted{i};\, \bm{a};\, c \restricted{n}).
\end{equation*}

Quindi la stringa
\begin{equation*}
(a \restricted{i}) \concat \bm{a} \concat (c\restricted{n})
\end{equation*}
è monocromatica di colore 1. Ponendo \(a_{i} \coloneqq \bm{a}\) si ha la tesi.
\end{proof}
\printbibliography
\end{document}
