% Created 2026-02-07 Sat 19:30
% Intended LaTeX compiler: pdflatex
\documentclass[10pt]{article}
%% CREATO CON ORG - EMACS
\newcommand{\use}[2][]{\usepackage[#1]{#2}}
% PACCHETTI FONDAMENTLAI
\use[utf8]{inputenc}
\use[T1]{fontenc}
\use{graphicx}
\use{longtable}
\use{wrapfig}
\use{rotating}
\use[normalem]{ulem}
\use{amsmath}
\use{amsthm}
\use{amssymb}

\use{eucal} % Cambia mathcal{...}

\use{capt-of}
\use[italian]{babel}
\use[babel]{csquotes}
% bib la TEX lo carica in automatico org-cite
\use{microtype}
\use{lmodern}
\use{subfig} % sottofigure
\use{multicol} % due colonne
\use{lipsum} % lorem ipsum
\use{color} % colori in latex
\use{parskip} % rimuove l'indentazione dei nuovi paragrafi %% Add parbox=false to all new tcolorbox
\use{centernot}
\use[outline]{contour}\contourlength{3pt}
\use{fancyhdr}
\use{layout}
\use[most]{tcolorbox} % Riquadri colorati
\use{ifthen} % IFTHEN
\use{geometry}

% pacchetti matematica
\use{yhmath}
\use{dsfont}
\use{mathrsfs}
\use{cancel} % semplificare
\use{polynom} %divisione tra polinomi
\use{forest} % grafi ad albero
\use{booktabs} % tabelle
\use{commath} %simboli e differenziali
\use{bm} %bold
\use[fulladjust]{marginnote} %to use marginnote for date notes
\use{arrayjobx}%array
\use[intlimits]{empheq} % Riquadri colorati attorno alle equazioni
\use{mathtools}
\use{circuitikz} % Disegnare i circuiti
\use{mathtools}
\use{stmaryrd} % [[ \llbracket ]] \rrbracket
\use{bussproofs} % dimostrazioni

%%%%%%%%%%%%%


%%%% QUIVER
\newcommand{\duepunti}{\,\mathchar\numexpr"6000+`:\relax\,}
% A TikZ style for curved arrows of a fixed height, due to AndréC.
\tikzset{curve/.style={settings={#1},to path={(\tikztostart)
    .. controls ($(\tikztostart)!\pv{pos}!(\tikztotarget)!\pv{height}!270:(\tikztotarget)$)
    and ($(\tikztostart)!1-\pv{pos}!(\tikztotarget)!\pv{height}!270:(\tikztotarget)$)
    .. (\tikztotarget)\tikztonodes}},
    settings/.code={\tikzset{quiver/.cd,#1}
        \def\pv##1{\pgfkeysvalueof{/tikz/quiver/##1}}},
    quiver/.cd,pos/.initial=0.35,height/.initial=0}

% TikZ arrowhead/tail styles.
\tikzset{tail reversed/.code={\pgfsetarrowsstart{tikzcd to}}}
\tikzset{2tail/.code={\pgfsetarrowsstart{Implies[reversed]}}}
\tikzset{2tail reversed/.code={\pgfsetarrowsstart{Implies}}}
% TikZ arrow styles.
\tikzset{no body/.style={/tikz/dash pattern=on 0 off 1mm}}
%%%%%%%%%%


%% DEFINIZIONI COMANDI MATEMATICI
\let\sin\relax %TOGLIE LA DEFINIZIONE SU "\sin"

% cambia la definizione di empty set
% ---
\let\oldemptyset\emptyset
% ---
% \let\emptyset\varnothing
% ---
% \let\emptyset\relax
% \newcommand{\emptyset}{\text{\textnormal{\O}}}
% ---

\DeclareMathOperator{\bounded}{bd}
\DeclareMathOperator{\sin}{sen}
\DeclareMathOperator{\epi}{Epi}
\DeclareMathOperator{\cl}{cl}
\DeclareMathOperator{\graph}{graph}
\DeclareMathOperator{\arcsec}{arcsec}
\DeclareMathOperator{\arccot}{arccot}
\DeclareMathOperator{\arccsc}{arccsc}
\DeclareMathOperator{\spettro}{Spettro}
\DeclareMathOperator{\nulls}{nullspace}
\DeclareMathOperator{\dom}{dom}
\DeclareMathOperator{\ar}{ar}
\DeclareMathOperator{\const}{Const}
\DeclareMathOperator{\fun}{Fun}
\DeclareMathOperator{\rel}{Rel}
\DeclareMathOperator{\altezza}{ht}
\let\det\relax %TOGLIE LA DEFINIZIONE SU "\det"
\DeclareMathOperator{\det}{det}
\DeclareMathOperator{\End}{End}
\DeclareMathOperator{\gl}{GL}
\def\Id{\mathrm{Id}}
\def\id{\mathrm{id}}
\DeclareMathOperator{\I}{\mathds{1}}
\DeclareMathOperator{\II}{II}
\DeclareMathOperator{\rank}{rank}
\DeclareMathOperator{\tr}{tr}
\DeclareMathOperator{\tc}{t.c.}
\DeclareMathOperator{\T}{T}
\DeclareMathOperator{\var}{Var}
\DeclareMathOperator{\cov}{Cov}
\DeclareMathOperator{\st}{st}
\DeclareMathOperator{\mon}{Mon}
\newcommand{\card}[1]{\left\vert #1 \right\vert}
\newcommand{\trasposta}[1]{\prescript{\text{T}}{}{#1}}
\newcommand{\1}{\mathds{1}}
\newcommand{\R}{\mathds{R}}
\newcommand{\diesis}{\#}
\newcommand{\bemolle}{\flat}
\newcommand{\nonstandard}[1]{\prescript{*}{}{#1}}
\newcommand{\starR}{\nonstandard{\R}}
\newcommand{\borel}{\mathscr{B}}
\newcommand{\lebesgue}[1]{\mathscr{L}\left(#1\right)}
\newcommand{\media}{\mathds{E}}
\newcommand{\K}{\mathds{K}}
\newcommand{\A}{\mathds{A}}
\newcommand{\Q}{\mathds{Q}}
\newcommand{\N}{\mathds{N}}
\newcommand{\C}{\mathds{C}}
\newcommand{\Z}{\mathds{Z}}
\newcommand{\qo}{\hspace{1em}\text{q.o.}\,}
\renewcommand{\tilde}[1]{\widetilde{#1}}
\renewcommand{\parallel}{\mathrel{/\mkern-5mu/}}
\newcommand{\parti}[2][]{\wp_{#1}(#2)}
\newcommand{\diff}[1]{\operatorname{d}_{#1}}
\let\oldvec\vec
\renewcommand{\vec}[1]{\overrightarrow{\vphantom{i}#1}}
\newcommand{\floor}[1]{\left\lfloor #1 \right\rfloor}
\newcommand{\cat}[1]{\mathbf{#1}}
\newcommand{\dfreccia}[1]{\xrightarrow{\ #1 \ }}
\newcommand{\sfreccia}[1]{\xleftarrow{\ #1 \ }}
\newcommand{\formalsum}[2]{{\sum_{#1}^{#2}}{\vphantom{\sum}}'}
\newcommand{\minim}[2]{\mu_{#1}\, \left(#2\right)}
\newcommand{\concat}{\null^{\frown}} % concatenazione di stringe
\newcommand{\godelcode}[1]{\langle\!\langle #1 \rangle\!\rangle}
\newcommand{\godeldec}[1]{(\!(#1)\!)}
\newcommand{\termcode}[1]{\ulcorner #1\urcorner}
\newcommand{\partialto}{\dashrightarrow}
\newcommand{\restricted}{\upharpoonright}
\newcommand{\embeds}{\precsim}
\newcommand{\surjects}{\twoheadrightarrow}
\newcommand{\equipotenti}{\asymp}
%% \newcommand{\dotplus}{\mathbin{\dot{+}}} %% A quanto pare esiste già
\newcommand{\bigdot}{\mathbin{\boldsymbol{\cdot}}}
\newcommand{\dotexp}[1]{^{.#1}}
\newcommand{\conv}{\mathbin{*}}
\newcommand{\convolution}[2]{(#1\conv #2)}
\newcommand{\nil}{\mathfrak{N}}
\newcommand{\divisore}{\mathrel{|}}
\newcommand{\simplesso}[1]{\mathrm{e}_{#1}}

\renewcommand{\iff}{\mathrel{\longleftrightarrow}} %% Notazione Logica.
\newcommand{\oldiff}{\mathrel{\Longleftrightarrow}}
\renewcommand{\implies}{\mathrel{\rightarrow}} %% Notazione Logica
\newcommand{\oldimplies}{\mathrel{\Longrightarrow}}
\renewcommand{\impliedby}{\mathrel{\leftarrow}} %% Notazione Logica
\newcommand{\oldimpliedby}{\mathrel{\Longleftarrow}}

\newcommand{\IFF}{\quad\Longleftrightarrow\quad}
\newcommand{\IMPLICA}{\quad\Longrightarrow\quad}


\renewcommand{\descriptionlabel}[1]{\hspace{\labelsep}\normalfont #1} % remove bold from description


%% Definizione di Divergenza di K-L

\DeclarePairedDelimiterX{\infdivx}[2]{(}{)}{%
  #1\;\delimsize\|\;#2%
}
\newcommand{\kldiv}{D_{KL}\infdivx}

%% Definizione di \dotminus

\makeatletter
\newcommand{\dotminus}{\mathbin{\text{\@dotminus}}}

\newcommand{\@dotminus}{%
  \ooalign{\hidewidth\raise1ex\hbox{.}\hidewidth\cr$\m@th-$\cr}%
}
\makeatother

%tramite i prossimi due comandi posso decidere come scrivere i logaritmi naturali in tutti i documenti: ho infatti eliminato qualsiasi differenza tra "ln" e "log": se si vuole qualcosa di diverso bisogna inserire manualmente il tutto
\let\ln\relax
\DeclareMathOperator{\ln}{ln}
\let\log\relax
\DeclareMathOperator{\log}{log}
%%%%%%

%% NUOVI COMANDI
\newcommand{\straniero}[1]{\textit{#1}} %parole straniere
\newcommand{\titolo}[1]{\textsc{#1}} %titoli
\newcommand{\qedd}{\tag*{$\blacksquare$}} %qed per ambienti matemastici
\renewcommand{\qedsymbol}{$\blacksquare$} %modifica colore qed
\newcommand{\ooverline}[1]{\overline{\overline{#1}}}
\newcommand{\circoletto}[1]{\left(#1\right)^{\text{o}}}
%
\newcommand{\qmatrice}[1]{\begin{pmatrix}
#1_{11} & \cdots & #1_{1n}\\
\vdots & \ddots & \vdots \\
#1_{m1} & \cdots & #1_{mn}
\end{pmatrix}}
%
\newcommand{\parentesi}[2]{%
\underset{#1}{\underbrace{#2}}%
}
%
\newcommand{\norma}[1]{% Norma
\left\lVert#1\right\rVert%
}
\newcommand{\scalare}[2]{% Scalare
\left\langle #1, #2\right\rangle
}
%%%%%

%% RESTRIZIONI
\newcommand{\referenze}[2]{
        \phantomsection{}#2\textsuperscript{\textcolor{blue}{\textbf{#1}}}
}

\let\restriction\relax

\def\restriction#1#2{\mathchoice
              {\setbox1\hbox{${\displaystyle #1}_{\scriptstyle #2}$}
              \restrictionaux{#1}{#2}}
              {\setbox1\hbox{${\textstyle #1}_{\scriptstyle #2}$}
              \restrictionaux{#1}{#2}}
              {\setbox1\hbox{${\scriptstyle #1}_{\scriptscriptstyle #2}$}
              \restrictionaux{#1}{#2}}
              {\setbox1\hbox{${\scriptscriptstyle #1}_{\scriptscriptstyle #2}$}
              \restrictionaux{#1}{#2}}}
\def\restrictionaux#1#2{{#1\,\smash{\vrule height .8\ht1 depth .85\dp1}}_{\,#2}}
%%%%%%%%%%%

%%% FORMATTAZIONE FOOTNOTEMARK

\def\footnotemarkformatting#1{[#1]}
\renewcommand{\thefootnote}{\footnotemarkformatting{\arabic{footnote}}}

%% SEZIONE GRAFICA
\use{tikz}
\usetikzlibrary{matrix, patterns, calc, decorations.pathreplacing, hobby, decorations.markings, decorations.pathmorphing, babel}
\use{tikz-3dplot}
\use{mathrsfs} %per geogebra
\use{tikz-cd}
\tikzset
{
  %surface/.style={fill=black!10, shading=ball,fill opacity=0.4},
  plane/.style={black,pattern=north east lines},
  curve/.style={black,line width=0.5mm},
  dritto/.style={decoration={markings,mark=at position 0.5 with {\arrow{Stealth}}}, postaction=decorate},
  rovescio/.style={decoration={markings,mark=at position 0.5 with {\arrow{Stealth[reversed]}}}, postaction=decorate}
}
\use{pgfplots} % stampare le funzioni
        \pgfplotsset{/pgf/number format/use comma,compat=1.15}
        %\pgfplotsset{compat=1.15} %per geogebra
        \usepgfplotslibrary{fillbetween, polar}
%%%%%%

%% CITAZIONI
\use{lineno}

\newcommand{\citazione}[1]{%
  \begin{quotation}
  \begin{linenumbers}
  \modulolinenumbers[5]
  \begingroup
  \setlength{\parindent}{0cm}
  \noindent #1
  \endgroup
  \end{linenumbers}
  \end{quotation}\setcounter{linenumber}{1}
  }
%%%%%%

%%%%%%%%%%%%%%%%%%%%%%%%%%%%%%%%%%%%%%%%%%%%
%%%%%%%%%%%%%%%%%%%%%%%%%%%%%%%%%%%%%%%%%%%%

%% AMS THM

\theoremstyle{definition}% default
\newtheorem{thm}{Teorema}[section]
\newtheorem{lem}[thm]{Lemma}
\newtheorem{prop}[thm]{Proposizione}
\newtheorem{cor}[thm]{Corollario}
\newtheorem{esempio}[thm]{Esempio}
\theoremstyle{plain}
\newtheorem{definizione}[thm]{Definizione}
\theoremstyle{remark}
\newtheorem*{oss}{Osservazione}


%%%%%%%%%%%%%%%%%%%%%%%%%%%%%%%%%%%%%%%%%%%%
%%%%%%%%%%%%%%%%%%%%%%%%%%%%%%%%%%%%%%%%%%%%

\use{hyperref}
\hypersetup{%
        pdfauthor={Davide Peccioli},
        pdfsubject={},
        allcolors=black,
        citecolor=black,
%	colorlinks=true,
        bookmarksopen=true}
\setcounter{secnumdepth}{0} % rimuove i numeri di sezione senza rimuovere le ref
\renewcommand{\href}[2]{\textcolor{blue}{#2}} % disabilita il comando href
\use{enotez} %
\setenotez{%
 mark-format = \footnotemarkformatting % Mette i numeri tra parentesi quadre%
}\let\footnote=\endnote % rende tutte le note a pié pagina come delle note a fine file 


\let\olddocument\document % modifico l'ambiende documenti per non dover stampare \printendnote
\let\oldenddocument\enddocument
\renewenvironment{document}%
{%
  \olddocument
}{%
  \printendnotes\oldenddocument
}
\renewcommand{\thethm}{\arabic{thm}}

\usepackage[hyperref]{biblatex}
\addbibresource{~/Documents/org/roam/bib/master.bib}
\def\mult#1{\mathrm{mult}_{#1}\,}
\author{Davide Peccioli}
\date{\today}
\title{}
\begin{document}

\section{Teorema del Grado per olomorfismi tra superfici di Riemannn}
\label{sec:org29aa0cc}
Siano \(X,Y\) \href{20260127112828-superficie_di_riemann.org}{superfici di Riemann} \href{20250103163701-spazio_topologico_compatto.org}{compatte}, e sia \(F:X\to Y\) \href{20260128143822-funzione_olomorfa_su_una_superficie_di_riemann.org}{olomorfismo} non costante.

\begin{definizione}
Per ogni \(y \in Y\) si definisce\footnote{La somma è finita per il \href{20260128144016-principio_di_identita_per_funzioni_olomorfe_su_superfici_di_riemann.org}{Principio di identità} (poiché \href{20260128172415-sottoinsieme_discreto_in_un_compatto.org}{sottinsiemi discreti in un compatto sono finiti}), mentre \(\mathrm{mult}_{p}\, F\) è la \href{20260129104215-forma_normale_locale_per_superfici_di_riemann.org}{molteplicità}.}
\begin{equation*}
\operatorname{d}(y) \coloneqq \sum_{p \in F^{-1}(y)} \mathrm{mult}_{p}\, F
\end{equation*}
\end{definizione}

\begin{thm}
\(\operatorname{d}(y)\) non dipende da \(y\).
\end{thm}
\begin{proof}
Si fissi \(y_{0} \in Y\), e sia
\begin{equation*}
F^{-1}(y_{0}) \coloneqq \set{x_{1},\dots,x_{h}},\qquad m_{i} \coloneqq \mult{x_{i}} F.
\end{equation*}
Allora \(\operatorname{d}(y_{0}) = m_{1}+\dots+m_{h}\).

Consideriamo ora:
\begin{itemize}
\item su \(Y\) la carta locale \(w\), centrata in \(y_{0}\)\footnote{Ovvero \(w\) è definita su un intorno aperto di \(y_{0}\), e \(w(y_{0})=0\).};
\item su \(X\), le carte locali \(z_{i}\), centrate in \(x_{i}\), tali per cui l'espressione locale di \(F\) è
\begin{equation*}
  w = z_{i}^{m_{i}}.
\end{equation*}
A meno di restringermi, considero tutti i domini disgiunti.
\end{itemize}
Queste esistono per il \href{20260129104215-forma_normale_locale_per_superfici_di_riemann.org}{Teorema di Forma Normale}.

Si consideri ora, per ogni \(\varepsilon>0\), l'insieme
\begin{equation*}
D(0;\varepsilon) \coloneqq \set{|z| < \varepsilon} \subseteq \C
\end{equation*}
e si definiscano:
\begin{align*}
\Delta_{\varepsilon}&\coloneqq w^{-1}\big(D(0;\varepsilon)\big) \subseteq Y\\
\Delta_{x_{i}} &\coloneqq z_{i}^{-1}\bigg(D(0;\varepsilon^{1/m_{i}})\bigg) \subseteq X
\end{align*}

A meno di restringere \(\varepsilon\), si può supporre che \(w\) e \(z_{i}\) siano degli omeomorfismi su \(\Delta_{\varepsilon}\) e \(\Delta_{x_{i}}\).

Si ha che, per ogni \(i\), \(F(\Delta_{x_{i}}) = \Delta_{\varepsilon}\)
\begin{itemize}
\item Per ogni \(y \in \Delta_{\varepsilon} \setminus \set{y_{0}}\), ci sono \(m_{i}\) \href{20250202190147-immagine_punto_a_punto_di_due_classi.org}{retroimmagini} in \(\Delta_{x_{i}}\), ciascuna di \href{20260129104215-forma_normale_locale_per_superfici_di_riemann.org}{molteplicità} 1.

Infatti, ci sono esattamente \(m_{i}\) radici \(\eta_{1},\dots,\eta_{m_{i}} \in D(0;\varepsilon^{1/m_{i}})\) tali che
\begin{equation*}
(\eta_{\ell})^{m_{i}} = w(y) \in D(0; \varepsilon)
\end{equation*}
e siccome \(z_{i}\) è un omeomorfismo, ne esistono altrettanti in \(\Delta_{x_{i}}\).

I punti sono di molteplicità \(1\) poiché la funzione \(z\mapsto z^{m}\) \href{20260130094844-funzione_z_m.org}{è iniettiva in un intorno di ogni punto \(z\neq 0\)}, e questa cosa si trasmette ad \(F\).
\end{itemize}

Per le proprietà delle retroimmagini si ha la seguente inclusione
\begin{equation*}
\bigcup_{i=1}^{h} \Delta_{x_{i}} \subseteq F^{-1}(\Delta_{\varepsilon}).
\end{equation*}
\begin{itemize}
\item Se vale l'uguaglianza: \(\bigcup_{i=1}^{h} \Delta_{x_{i}} = F^{-1}(\Delta_{\varepsilon})\), allora per ogni \(y \in \Delta_{\varepsilon}\setminus \set{y_{0}}\)
\begin{equation*}
  \operatorname{d}(y) = m_{1}+\dots+m_{h} = \operatorname{d}(y_{0})
\end{equation*}
e quindi \(\operatorname{d}\) è \href{20250325153824-funzione_localmente_costante.org}{localmente costante}, con \(Y\) \href{20250103165325-spazio_topologico_connesso.org}{connesso}: \href{20250325154046-funzione_localmente_costante_sse_costante_sulle_componenti_connesse.org}{vale che \(\operatorname{d}\) è costante}.

\item \uline{A meno di restringere \(\varepsilon\), si ottiene l'uguaglianza}.

Per assurdo, supponiamo che per ogni \(\varepsilon > 0\) ammissibile\footnote{Si erano poste delle restrizioni sul valore massimo che potesse assumere \(\varepsilon\).}, si abbia \(\bigcup_{i=1}^{h} \Delta_{x_{i}} \subsetneqq F^{-1}(\Delta_{\varepsilon})\).

Allora, esiste un certo \(n_{0}\) tale per cui, per ogni \(n>n_{0}\), esiste \(y_{n} \in \Delta_{1/n}\) tale che
\begin{equation*}
  F^{-1}(y_{n}) \setminus \bigcup_{i=1}^{h} \Delta_{x_{i}} \neq \emptyset
\end{equation*}
e sia \(x_{n} \in F^{-1}(y_{n}) \setminus \bigcup_{i=1}^{h} \Delta_{x_{i}}\).

Si hanno due \href{20250115100904-successione.org}{successioni} \(\set{x_{n}}\), \(\set{y_{n}}\) tali per cui \(F(x_{n}) = y_{n}\), e inoltre \(y_{n} \to y\) in \(Y\)\footnote{Vedi ``\href{20260130103009-successione_in_uno_spazio_topologico.org}{Successione in uno spazio topologico}''}.

Poiché \(X\) è compatto, \(x_{n}\) ha una \href{20250115100916-sottosuccessione.org}{sottosuccessione} \(\set{x_{n_{k}}}\) convergente ad \(\tilde{x} \in X\), e siccome \(F\) è continua (e \href{20250304142114-funzione_continua_e_continua_per_successioni.org}{quindi} \href{20250310112816-funzione_continua_per_successioni.org}{continua per successioni})
\begin{equation*}
  F(\tilde{x}) = y_{0}
\end{equation*}
ovvero \(\tilde{x} \in F^{-1}(y_{0}) = \set{x_{1},\dots,x_{h}}\).

Quindi, se \(\tilde{x} = x_{j}\), allora esiste \(k_{0}\) tale che per ogni \(k>k_{0}\): \(x_{n_{k}} \in \Delta_{x_{j}}\). Questo contraddice il fatto che
\begin{equation*}
  x_{n_{k}} \in F^{-1}(y_{n_{k}}) \setminus \bigcup_{i=1}^{h} \Delta_{x_{i}} \IMPLICA x_{n_{k}} \notin \Delta_{x_{j}}.
\end{equation*}
Assurdo.\qedhere
\end{itemize}
\end{proof}

\begin{definizione}
Si definisce il \textbf{grado di \(F\)} come:
\begin{equation*}
\deg F \coloneqq \operatorname{d}(y),\qquad y \in Y.
\end{equation*}
\end{definizione}
\end{document}
