% Created 2026-02-07 Sat 19:30
% Intended LaTeX compiler: pdflatex
\documentclass[10pt]{article}
%% CREATO CON ORG - EMACS
\newcommand{\use}[2][]{\usepackage[#1]{#2}}
% PACCHETTI FONDAMENTLAI
\use[utf8]{inputenc}
\use[T1]{fontenc}
\use{graphicx}
\use{longtable}
\use{wrapfig}
\use{rotating}
\use[normalem]{ulem}
\use{amsmath}
\use{amsthm}
\use{amssymb}

\use{eucal} % Cambia mathcal{...}

\use{capt-of}
\use[italian]{babel}
\use[babel]{csquotes}
% bib la TEX lo carica in automatico org-cite
\use{microtype}
\use{lmodern}
\use{subfig} % sottofigure
\use{multicol} % due colonne
\use{lipsum} % lorem ipsum
\use{color} % colori in latex
\use{parskip} % rimuove l'indentazione dei nuovi paragrafi %% Add parbox=false to all new tcolorbox
\use{centernot}
\use[outline]{contour}\contourlength{3pt}
\use{fancyhdr}
\use{layout}
\use[most]{tcolorbox} % Riquadri colorati
\use{ifthen} % IFTHEN
\use{geometry}

% pacchetti matematica
\use{yhmath}
\use{dsfont}
\use{mathrsfs}
\use{cancel} % semplificare
\use{polynom} %divisione tra polinomi
\use{forest} % grafi ad albero
\use{booktabs} % tabelle
\use{commath} %simboli e differenziali
\use{bm} %bold
\use[fulladjust]{marginnote} %to use marginnote for date notes
\use{arrayjobx}%array
\use[intlimits]{empheq} % Riquadri colorati attorno alle equazioni
\use{mathtools}
\use{circuitikz} % Disegnare i circuiti
\use{mathtools}
\use{stmaryrd} % [[ \llbracket ]] \rrbracket
\use{bussproofs} % dimostrazioni

%%%%%%%%%%%%%


%%%% QUIVER
\newcommand{\duepunti}{\,\mathchar\numexpr"6000+`:\relax\,}
% A TikZ style for curved arrows of a fixed height, due to AndréC.
\tikzset{curve/.style={settings={#1},to path={(\tikztostart)
    .. controls ($(\tikztostart)!\pv{pos}!(\tikztotarget)!\pv{height}!270:(\tikztotarget)$)
    and ($(\tikztostart)!1-\pv{pos}!(\tikztotarget)!\pv{height}!270:(\tikztotarget)$)
    .. (\tikztotarget)\tikztonodes}},
    settings/.code={\tikzset{quiver/.cd,#1}
        \def\pv##1{\pgfkeysvalueof{/tikz/quiver/##1}}},
    quiver/.cd,pos/.initial=0.35,height/.initial=0}

% TikZ arrowhead/tail styles.
\tikzset{tail reversed/.code={\pgfsetarrowsstart{tikzcd to}}}
\tikzset{2tail/.code={\pgfsetarrowsstart{Implies[reversed]}}}
\tikzset{2tail reversed/.code={\pgfsetarrowsstart{Implies}}}
% TikZ arrow styles.
\tikzset{no body/.style={/tikz/dash pattern=on 0 off 1mm}}
%%%%%%%%%%


%% DEFINIZIONI COMANDI MATEMATICI
\let\sin\relax %TOGLIE LA DEFINIZIONE SU "\sin"

% cambia la definizione di empty set
% ---
\let\oldemptyset\emptyset
% ---
% \let\emptyset\varnothing
% ---
% \let\emptyset\relax
% \newcommand{\emptyset}{\text{\textnormal{\O}}}
% ---

\DeclareMathOperator{\bounded}{bd}
\DeclareMathOperator{\sin}{sen}
\DeclareMathOperator{\epi}{Epi}
\DeclareMathOperator{\cl}{cl}
\DeclareMathOperator{\graph}{graph}
\DeclareMathOperator{\arcsec}{arcsec}
\DeclareMathOperator{\arccot}{arccot}
\DeclareMathOperator{\arccsc}{arccsc}
\DeclareMathOperator{\spettro}{Spettro}
\DeclareMathOperator{\nulls}{nullspace}
\DeclareMathOperator{\dom}{dom}
\DeclareMathOperator{\ar}{ar}
\DeclareMathOperator{\const}{Const}
\DeclareMathOperator{\fun}{Fun}
\DeclareMathOperator{\rel}{Rel}
\DeclareMathOperator{\altezza}{ht}
\let\det\relax %TOGLIE LA DEFINIZIONE SU "\det"
\DeclareMathOperator{\det}{det}
\DeclareMathOperator{\End}{End}
\DeclareMathOperator{\gl}{GL}
\def\Id{\mathrm{Id}}
\def\id{\mathrm{id}}
\DeclareMathOperator{\I}{\mathds{1}}
\DeclareMathOperator{\II}{II}
\DeclareMathOperator{\rank}{rank}
\DeclareMathOperator{\tr}{tr}
\DeclareMathOperator{\tc}{t.c.}
\DeclareMathOperator{\T}{T}
\DeclareMathOperator{\var}{Var}
\DeclareMathOperator{\cov}{Cov}
\DeclareMathOperator{\st}{st}
\DeclareMathOperator{\mon}{Mon}
\newcommand{\card}[1]{\left\vert #1 \right\vert}
\newcommand{\trasposta}[1]{\prescript{\text{T}}{}{#1}}
\newcommand{\1}{\mathds{1}}
\newcommand{\R}{\mathds{R}}
\newcommand{\diesis}{\#}
\newcommand{\bemolle}{\flat}
\newcommand{\nonstandard}[1]{\prescript{*}{}{#1}}
\newcommand{\starR}{\nonstandard{\R}}
\newcommand{\borel}{\mathscr{B}}
\newcommand{\lebesgue}[1]{\mathscr{L}\left(#1\right)}
\newcommand{\media}{\mathds{E}}
\newcommand{\K}{\mathds{K}}
\newcommand{\A}{\mathds{A}}
\newcommand{\Q}{\mathds{Q}}
\newcommand{\N}{\mathds{N}}
\newcommand{\C}{\mathds{C}}
\newcommand{\Z}{\mathds{Z}}
\newcommand{\qo}{\hspace{1em}\text{q.o.}\,}
\renewcommand{\tilde}[1]{\widetilde{#1}}
\renewcommand{\parallel}{\mathrel{/\mkern-5mu/}}
\newcommand{\parti}[2][]{\wp_{#1}(#2)}
\newcommand{\diff}[1]{\operatorname{d}_{#1}}
\let\oldvec\vec
\renewcommand{\vec}[1]{\overrightarrow{\vphantom{i}#1}}
\newcommand{\floor}[1]{\left\lfloor #1 \right\rfloor}
\newcommand{\cat}[1]{\mathbf{#1}}
\newcommand{\dfreccia}[1]{\xrightarrow{\ #1 \ }}
\newcommand{\sfreccia}[1]{\xleftarrow{\ #1 \ }}
\newcommand{\formalsum}[2]{{\sum_{#1}^{#2}}{\vphantom{\sum}}'}
\newcommand{\minim}[2]{\mu_{#1}\, \left(#2\right)}
\newcommand{\concat}{\null^{\frown}} % concatenazione di stringe
\newcommand{\godelcode}[1]{\langle\!\langle #1 \rangle\!\rangle}
\newcommand{\godeldec}[1]{(\!(#1)\!)}
\newcommand{\termcode}[1]{\ulcorner #1\urcorner}
\newcommand{\partialto}{\dashrightarrow}
\newcommand{\restricted}{\upharpoonright}
\newcommand{\embeds}{\precsim}
\newcommand{\surjects}{\twoheadrightarrow}
\newcommand{\equipotenti}{\asymp}
%% \newcommand{\dotplus}{\mathbin{\dot{+}}} %% A quanto pare esiste già
\newcommand{\bigdot}{\mathbin{\boldsymbol{\cdot}}}
\newcommand{\dotexp}[1]{^{.#1}}
\newcommand{\conv}{\mathbin{*}}
\newcommand{\convolution}[2]{(#1\conv #2)}
\newcommand{\nil}{\mathfrak{N}}
\newcommand{\divisore}{\mathrel{|}}
\newcommand{\simplesso}[1]{\mathrm{e}_{#1}}

\renewcommand{\iff}{\mathrel{\longleftrightarrow}} %% Notazione Logica.
\newcommand{\oldiff}{\mathrel{\Longleftrightarrow}}
\renewcommand{\implies}{\mathrel{\rightarrow}} %% Notazione Logica
\newcommand{\oldimplies}{\mathrel{\Longrightarrow}}
\renewcommand{\impliedby}{\mathrel{\leftarrow}} %% Notazione Logica
\newcommand{\oldimpliedby}{\mathrel{\Longleftarrow}}

\newcommand{\IFF}{\quad\Longleftrightarrow\quad}
\newcommand{\IMPLICA}{\quad\Longrightarrow\quad}


\renewcommand{\descriptionlabel}[1]{\hspace{\labelsep}\normalfont #1} % remove bold from description


%% Definizione di Divergenza di K-L

\DeclarePairedDelimiterX{\infdivx}[2]{(}{)}{%
  #1\;\delimsize\|\;#2%
}
\newcommand{\kldiv}{D_{KL}\infdivx}

%% Definizione di \dotminus

\makeatletter
\newcommand{\dotminus}{\mathbin{\text{\@dotminus}}}

\newcommand{\@dotminus}{%
  \ooalign{\hidewidth\raise1ex\hbox{.}\hidewidth\cr$\m@th-$\cr}%
}
\makeatother

%tramite i prossimi due comandi posso decidere come scrivere i logaritmi naturali in tutti i documenti: ho infatti eliminato qualsiasi differenza tra "ln" e "log": se si vuole qualcosa di diverso bisogna inserire manualmente il tutto
\let\ln\relax
\DeclareMathOperator{\ln}{ln}
\let\log\relax
\DeclareMathOperator{\log}{log}
%%%%%%

%% NUOVI COMANDI
\newcommand{\straniero}[1]{\textit{#1}} %parole straniere
\newcommand{\titolo}[1]{\textsc{#1}} %titoli
\newcommand{\qedd}{\tag*{$\blacksquare$}} %qed per ambienti matemastici
\renewcommand{\qedsymbol}{$\blacksquare$} %modifica colore qed
\newcommand{\ooverline}[1]{\overline{\overline{#1}}}
\newcommand{\circoletto}[1]{\left(#1\right)^{\text{o}}}
%
\newcommand{\qmatrice}[1]{\begin{pmatrix}
#1_{11} & \cdots & #1_{1n}\\
\vdots & \ddots & \vdots \\
#1_{m1} & \cdots & #1_{mn}
\end{pmatrix}}
%
\newcommand{\parentesi}[2]{%
\underset{#1}{\underbrace{#2}}%
}
%
\newcommand{\norma}[1]{% Norma
\left\lVert#1\right\rVert%
}
\newcommand{\scalare}[2]{% Scalare
\left\langle #1, #2\right\rangle
}
%%%%%

%% RESTRIZIONI
\newcommand{\referenze}[2]{
        \phantomsection{}#2\textsuperscript{\textcolor{blue}{\textbf{#1}}}
}

\let\restriction\relax

\def\restriction#1#2{\mathchoice
              {\setbox1\hbox{${\displaystyle #1}_{\scriptstyle #2}$}
              \restrictionaux{#1}{#2}}
              {\setbox1\hbox{${\textstyle #1}_{\scriptstyle #2}$}
              \restrictionaux{#1}{#2}}
              {\setbox1\hbox{${\scriptstyle #1}_{\scriptscriptstyle #2}$}
              \restrictionaux{#1}{#2}}
              {\setbox1\hbox{${\scriptscriptstyle #1}_{\scriptscriptstyle #2}$}
              \restrictionaux{#1}{#2}}}
\def\restrictionaux#1#2{{#1\,\smash{\vrule height .8\ht1 depth .85\dp1}}_{\,#2}}
%%%%%%%%%%%

%%% FORMATTAZIONE FOOTNOTEMARK

\def\footnotemarkformatting#1{[#1]}
\renewcommand{\thefootnote}{\footnotemarkformatting{\arabic{footnote}}}

%% SEZIONE GRAFICA
\use{tikz}
\usetikzlibrary{matrix, patterns, calc, decorations.pathreplacing, hobby, decorations.markings, decorations.pathmorphing, babel}
\use{tikz-3dplot}
\use{mathrsfs} %per geogebra
\use{tikz-cd}
\tikzset
{
  %surface/.style={fill=black!10, shading=ball,fill opacity=0.4},
  plane/.style={black,pattern=north east lines},
  curve/.style={black,line width=0.5mm},
  dritto/.style={decoration={markings,mark=at position 0.5 with {\arrow{Stealth}}}, postaction=decorate},
  rovescio/.style={decoration={markings,mark=at position 0.5 with {\arrow{Stealth[reversed]}}}, postaction=decorate}
}
\use{pgfplots} % stampare le funzioni
        \pgfplotsset{/pgf/number format/use comma,compat=1.15}
        %\pgfplotsset{compat=1.15} %per geogebra
        \usepgfplotslibrary{fillbetween, polar}
%%%%%%

%% CITAZIONI
\use{lineno}

\newcommand{\citazione}[1]{%
  \begin{quotation}
  \begin{linenumbers}
  \modulolinenumbers[5]
  \begingroup
  \setlength{\parindent}{0cm}
  \noindent #1
  \endgroup
  \end{linenumbers}
  \end{quotation}\setcounter{linenumber}{1}
  }
%%%%%%

%%%%%%%%%%%%%%%%%%%%%%%%%%%%%%%%%%%%%%%%%%%%
%%%%%%%%%%%%%%%%%%%%%%%%%%%%%%%%%%%%%%%%%%%%

%% AMS THM

\theoremstyle{definition}% default
\newtheorem{thm}{Teorema}[section]
\newtheorem{lem}[thm]{Lemma}
\newtheorem{prop}[thm]{Proposizione}
\newtheorem{cor}[thm]{Corollario}
\newtheorem{esempio}[thm]{Esempio}
\theoremstyle{plain}
\newtheorem{definizione}[thm]{Definizione}
\theoremstyle{remark}
\newtheorem*{oss}{Osservazione}


%%%%%%%%%%%%%%%%%%%%%%%%%%%%%%%%%%%%%%%%%%%%
%%%%%%%%%%%%%%%%%%%%%%%%%%%%%%%%%%%%%%%%%%%%

\use{hyperref}
\hypersetup{%
        pdfauthor={Davide Peccioli},
        pdfsubject={},
        allcolors=black,
        citecolor=black,
%	colorlinks=true,
        bookmarksopen=true}
\setcounter{secnumdepth}{0} % rimuove i numeri di sezione senza rimuovere le ref
\renewcommand{\href}[2]{\textcolor{blue}{#2}} % disabilita il comando href
\use{enotez} %
\setenotez{%
 mark-format = \footnotemarkformatting % Mette i numeri tra parentesi quadre%
}\let\footnote=\endnote % rende tutte le note a pié pagina come delle note a fine file 


\let\olddocument\document % modifico l'ambiende documenti per non dover stampare \printendnote
\let\oldenddocument\enddocument
\renewenvironment{document}%
{%
  \olddocument
}{%
  \printendnotes\oldenddocument
}
\renewcommand{\thethm}{\arabic{thm}}

\usepackage[hyperref]{biblatex}
\addbibresource{~/Documents/org/roam/bib/master.bib}
\author{Davide Peccioli}
\date{\today}
\title{Primo Teorema di Incompletezza di Gödel}
\begin{document}

Si considerino le notazione della ``\href{20250609104647-buona_codifica_di_un_linguaggio.org}{Aritmetizzazione della sintassi}'', e si fissi una \href{20250609104647-buona_codifica_di_un_linguaggio.org}{buona codifica} \(\#\) per il \href{20250130162057-linguaggio_del_prim_ordine.org}{linguaggio} \(L_{Q}=\set{+,\cdot,S,0}\) dell'\href{20250608093604-aritmetica.org}{aritmetica di Robinson} \(Q\).
\section{Premesse}
\label{sec:org0a285f7}

\subsection{Lemma 1}
\label{sec:orga4cdf63}

La seguente funzione è \href{20250215141024-funzioni_primitive_ricordive.org}{ricorsiva primitiva}
\begin{equation*}
\operatorname{num}:\N\to \N: n\mapsto \termcode{\overline{n}}
\end{equation*}
dove \(\overline{n}\) indica il \href{20250608093604-aritmetica.org}{numerale} associato a \(n\).
\subsection{Lemma 2}
\label{sec:org28d66d1}

Esiste una funzione \href{20250215141024-funzioni_primitive_ricordive.org}{ricorsiva primitiva} \(D: \N\to \N\) tale che per ogni \(L_{Q}\)-\href{20250131103317-formula_del_prim_ordine.org}{formula} \(\rho(v_{0})\)
\begin{equation*}
D\left(\termcode{\rho}\right) = \termcode{\rho(\overline{\termcode{\rho}}/v_{0})}.
\end{equation*}
\subsection{Lemma di diagonalizzazione per le formule dell'aritmetica di Robinson}
\label{sec:orgb7c07fe}
Per ogni \(L_{Q}\)-\href{20250131103317-formula_del_prim_ordine.org}{formula} \(\varphi(v_{i})\) con \(v_{i}\) \href{20250131103429-variabile_libera_di_una_formula.org}{libera} in \(\varphi\), esiste un \(L_{Q}\)-\href{20250131103446-enunciato_del_prim_ordine.org}{enunciato} \(\sigma\) tale che l'aritmetica di Robinson \href{20250131123011-conseguenza_logica.org}{dimostri}:
\begin{equation*}
Q\vdash \sigma\iff\varphi(\overline{\termcode{\sigma}}/v_{i})
\end{equation*}
\subsubsection{Dimostrazione}
\label{sec:org258289e}

Sia \(D\) la funzione del Lemma 2; siccome è \href{20250215141024-funzioni_primitive_ricordive.org}{ricorsiva primitiva}, \href{20250608094553-aritmetica_di_robinson_rappresenta_funzioni_ricorsive_totali_e_predicati_ricorsivi.org}{allora} esiste \(\psi(x,y)\) una \(L_{Q}\)-\href{20250131103317-formula_del_prim_ordine.org}{formula} che \href{20250608094213-insieme_rappresentato_da_una_formula.org}{rappresenta \(D\) in \(Q\)}.

WLOG:
\begin{itemize}
\item si supponga che \(i\neq 0\) (sempre possibile a meno di rinominare la variabili);
\item \(\psi\) non contenga le variabili \(v_{0},v_{i}\).
\end{itemize}

Sia dunque
\begin{equation*}
\rho(v_{0}):\qquad \forall\,v_{i}\ (\psi(v_{0},v_{i})\implies \varphi(v_{i}))
\end{equation*}
e sia
\begin{equation*}
\sigma:\qquad \rho(\overline{\termcode{\rho}}/v_{0})
\end{equation*}

Si vuole dimostrare che
\begin{equation*}
Q\vdash \sigma\iff \varphi(\overline{\termcode{\sigma}}/v_{i})
\end{equation*}

\begin{itemize}
\item Si supponga quindi che \(Q\vdash \sigma\), ovvero, per definizione di \(\rho\):
\begin{equation*}
     Q\vdash \forall\,v_{i}\ \left(\psi(\overline{\termcode{\rho}},v_{i})\implies \varphi(v_{i})\right)\tag{\star}
\end{equation*}

Inoltre, siccome \(\psi(x,y)\) rappresenta \(D\), allora
\begin{equation*}
     Q\vdash \psi(\overline{\termcode{\rho}}, \overline{D(\overline{\termcode{\rho}})})
\end{equation*}
ovvero proprio
\begin{equation*}
     Q\vdash \psi(\overline{\termcode{\rho}}, \overline{\termcode{\rho(\overline{\termcode{\rho}}/v_{0})}})\tag{\star\star}
\end{equation*}
Per (\(\star\)), istanziando \(v_{i}\) con \(\overline{\termcode{\rho(\overline{\termcode{\rho}}/v_{0})}}\):
\begin{equation*}
     Q\vdash \psi(\overline{\termcode{\rho}},\overline{\termcode{\rho(\overline{\termcode{\rho}}/v_{0})}})\implies \varphi(\overline{\termcode{\rho(\overline{\termcode{\rho}}/v_{0})}})
\end{equation*}
e per Modus Ponens con  (\(\star\star\)), si ottiene
\begin{equation*}
     Q\vdash \varphi(\overline{\termcode{\rho(\overline{\termcode{\rho}}/v_{0})}}/v_{i})
\end{equation*}
ovvero per la definizione di \(\sigma\)
\begin{equation*}
     Q\vdash \varphi(\overline{\termcode{\sigma}}/v_{i})
\end{equation*}

Questo dimostra che
\begin{equation*}
  Q\vdash \sigma\implies\varphi(\overline{\termcode{\sigma}}/v_{i})
\end{equation*}
\end{itemize}


\begin{itemize}
\item Viceversa, si supponga che \(Q\vdash \varphi(\overline{\termcode{\sigma}}/v_{i})\). Sia \(M\) una \(L_{Q}\)-struttura arbitraria tale che \(M\vDash Q\).

Allora
\begin{equation*}
  M\vDash \varphi(\overline{\termcode{\sigma}}/v_{i})\tag{\star\star\star}
\end{equation*}

Mostriamo che \(M\vDash \sigma\), ovvero
\begin{equation*}
  M\vDash \forall\, v_{i}\ \left(\psi(\overline{\termcode{\rho}}, v_{i})\implies \varphi(v_{i})\right)
\end{equation*}

Sia dunque \(q \in M\) tale che \(M\vDash \psi[\overline{\termcode{\rho}},q]\). Siccome \(\psi\) rappresenta \(D\) in \(Q\) e \(M\vDash Q\), allora
\begin{equation*}
  q= \overline{D(\overline{\termcode{\rho}})}^{M} = \overline{\termcode{\rho(\overline{\termcode{\rho}}/v_{0})}}^{M} = \overline{\termcode{\sigma}}^{M}
\end{equation*}
e pertanto \(M\vDash \varphi[q]\) poiché, per (\(\star\star\star\)), \(M\vDash\varphi(\overline{\termcode{\sigma}}/v_{i})\).

Per l'arbitrarietà di \(q \in M\), si ottiene
\begin{equation*}
  M\vDash\forall\, v_{i}\ \left(\psi(\overline{\termcode{\rho}}, v_{i})\implies \varphi(v_{i})\right)\qquad M\vDash \sigma
\end{equation*}
e per arbitrarietà di \(M\vDash Q\): \(Q\vdash \sigma\).

Questo dimostra che \(Q\vdash \varphi(\overline{\term{\sigma}}/v_{i})\implies \sigma\).
\end{itemize}

I due punti sopra dimostrano la tesi
\begin{equation*}
Q\vdash \sigma\iff \overline{\term{\sigma}}/v_{i}).\qedd
\end{equation*}
\section{Primo Teorema di Incompletezza di Gödel}
\label{sec:orgfaae2d0}

Sia \(L \supseteq L_{Q}\) e sia \(T\supseteq Q\) una \(L\)-\href{20250130114950-teoria_del_prim_ordine.org}{teoria} \href{20250609162711-teoria_omega_coerente.org}{\(\omega\)-coerente} e \href{20250609135250-teoria_ricorsivamente_assiomatizzabile.org}{ricorsivamente assiomatizzabile}. Allora \(T\) è \href{20250131123151-teoria_completa.org}{incompleta}.
\subsection{Dimostrazione}
\label{sec:org8f03e7f}

Sia \(\operatorname{Ax}(T)\) un sistema di assiomi per \(T\) tale che \(\operatorname{Ax}^{\#}(T) \subseteq \N\) sia ricorsivo.

\href{20250609135524-codifica_delle_dimostrazioni_a_partire_dagli_assiomi.org}{Allora} \(\operatorname{Prov}_{\operatorname{Ax}(T)}^{\#} \subseteq \N^{2}\) è \href{20250216173925-insieme_ricorsivo.org}{ricorsivo}, e \href{20250608094553-aritmetica_di_robinson_rappresenta_funzioni_ricorsive_totali_e_predicati_ricorsivi.org}{quindi} esiste una \(L\)-formula \(\psi(x,y)\) che lo \href{20250608094213-insieme_rappresentato_da_una_formula.org}{rappresenta in \(T\)}.


Sia quindi \(\varphi(y)\) la \(L\)-formula \(\lnot \exists\, x\ \psi(x,y)\). \footnote{Si noti che \(\varphi(y)\) è logicamente equivalente ad una \href{20250603170559-complessita_di_una_formula_del_modello_standard.org}{formula \(\Pi_{1}\)}}

Per il Lemma di Diagonalizzazione, esiste \(\sigma_{G}\) tale che \footnote{Per costruzione, siccome \(\varphi(y)\) è \(\Pi_{1}\), allora anche \(\sigma_{G}\) è \(\Pi_{1}\).}
\begin{equation*}
T\supseteq Q\vdash \sigma_{G}\iff\varphi(\overline{\termcode{\sigma_{G}}}/y).\tag{\diamondsuit}
\end{equation*}

Si dimostra che \(T\not\vdash \sigma_{G}\) e \(T\not\vdash\lnot\sigma_{G}\).

\begin{itemize}
\item Se per assurdo \(T\vdash\sigma_{G}\) allora \(\sigma_{G} \in \operatorname{Teor}_{T}\), ovvero \(\termcode{\sigma_{G}} \in \operatorname{Teor}^{\#}_{T}\).

Si ricorda che
\begin{equation*}
  \operatorname{Teor}_{T}^{\#}(m)\quad\iff\quad\exists\,n \ \operatorname{Prov}_{\operatorname{Ax}(T)}^{\#}(n,m)
\end{equation*}
con \(\operatorname{Prov}_{\operatorname{Ax}(T)}^{\#}\) rappresentato da \(\psi(x,y)\).

Poiché \(T\) è \(\omega\)-coerente, \href{20250609162711-teoria_omega_coerente.org}{allora}
\begin{equation*}
  T\vdash \exists\, x\ \psi(x, \overline{\termcode{\sigma_{G}}}).
\end{equation*}
ovvero
\begin{equation*}
  T\vdash \lnot \varphi(\overline{\termcode{\sigma_{G}}})
\end{equation*}
Questo, per (\(\diamondsuit\)), implica che \(T\vdash \lnot\sigma_{G}\). Assurdo, poiché \(T\) è \(\omega\)-coerente, e quindi coerente.

Dunque \(\termcode{\sigma_{G}}\notin \operatorname{Teor}^{\#}_{T}\).

\item Siccome \(T\) è \(\omega\)-coerente, \href{20250609162711-teoria_omega_coerente.org}{allora}
\begin{equation*}
  T\not\vdash \exists\, \psi(x,\overline{\termcode{\sigma_{G}}})
\end{equation*}
poiché altrimenti \(\termcode{\sigma_{G}} \in \operatorname{Teor}^{\#}_{T}\).

Dunque \(T\not\vdash \lnot \varphi(\overline{\termcode{\sigma_{G}}}/y)\) e per (\(\diamondsuit\)), quindi \(T\not\vdash\lnot\sigma_{G}\).\qed
\end{itemize}
\section{Conseguenze}
\label{sec:org7e5d879}

\subsection{Corollario 1}
\label{sec:org8d05fda}

Sia \(T\) una teoria ricorsivamente assiomatizzabile i cui assiomi siano veri in \(\N\). Se \(T\) dimostra gli assiomi (Q1)-(Q7) di \(Q\), allora \(T\) è incompleta. In particolare, l'aritmetica di Peano è incompleta.
\subsection{Complessità e dimostrabilità nell'aritmetica di Robinson}
\label{sec:orgc924d27}
Sia \(T\supseteq Q\) una \(L_{Q}\)-\href{20250130114950-teoria_del_prim_ordine.org}{teoria} di cui il \href{20250606095019-modello_standard_dell_artimetica.org}{modello standard} sia un \href{20250131122945-modello_di_un_insieme_di_formule.org}{modello}. Esiste un \href{20250603170559-complessita_di_una_formula_del_modello_standard.org}{\(\Pi_{1}\)}-\href{20250131103446-enunciato_del_prim_ordine.org}{enunciato} \(\sigma_{G}\) nel linguaggio \(L_{Q}\) vero in \(\N\) tale che \(T\) non è in grado di dimostrare.

Tutti gli \href{20250131103446-enunciato_del_prim_ordine.org}{enunciati} si considerino ora nel linguaggio \(L_{Q}\).
\begin{itemize}
\item Gli enunciati \href{20250603170559-complessita_di_una_formula_del_modello_standard.org}{\(\Sigma_{1}\)} veri nel \href{20250606095019-modello_standard_dell_artimetica.org}{modello standard} sono dimostrabili da \(Q\).
\item Gli enunciato \(\Sigma_{1}\) falsi nel \href{20250606095019-modello_standard_dell_artimetica.org}{modello standard} \uline{non sono dimostrabili in \(Q\)}, ma non è detto che \(Q\) li possa refutare-
\item Gli enunciati \(\Delta_{0}\) veri nel \href{20250606095019-modello_standard_dell_artimetica.org}{modello standard} sono dimostrabili in \(Q\).
\item Gli enunciati \(\Delta_{0}\) falsi nel \href{20250606095019-modello_standard_dell_artimetica.org}{modello standard} sono refutabili in \(Q\).
\end{itemize}
\subsection{Teorema}
\label{sec:orgfedb331}

Ogni teoria \(T\supseteq Q\) coerente e ricorsivamente assiomatizzabile è incompleta.
\end{document}
