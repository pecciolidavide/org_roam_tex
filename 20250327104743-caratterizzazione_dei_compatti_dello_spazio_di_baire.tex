% Created 2026-02-07 Sat 19:33
% Intended LaTeX compiler: pdflatex
\documentclass[10pt]{article}
%% CREATO CON ORG - EMACS
\newcommand{\use}[2][]{\usepackage[#1]{#2}}
% PACCHETTI FONDAMENTLAI
\use[utf8]{inputenc}
\use[T1]{fontenc}
\use{graphicx}
\use{longtable}
\use{wrapfig}
\use{rotating}
\use[normalem]{ulem}
\use{amsmath}
\use{amsthm}
\use{amssymb}

\use{eucal} % Cambia mathcal{...}

\use{capt-of}
\use[italian]{babel}
\use[babel]{csquotes}
% bib la TEX lo carica in automatico org-cite
\use{microtype}
\use{lmodern}
\use{subfig} % sottofigure
\use{multicol} % due colonne
\use{lipsum} % lorem ipsum
\use{color} % colori in latex
\use{parskip} % rimuove l'indentazione dei nuovi paragrafi %% Add parbox=false to all new tcolorbox
\use{centernot}
\use[outline]{contour}\contourlength{3pt}
\use{fancyhdr}
\use{layout}
\use[most]{tcolorbox} % Riquadri colorati
\use{ifthen} % IFTHEN
\use{geometry}

% pacchetti matematica
\use{yhmath}
\use{dsfont}
\use{mathrsfs}
\use{cancel} % semplificare
\use{polynom} %divisione tra polinomi
\use{forest} % grafi ad albero
\use{booktabs} % tabelle
\use{commath} %simboli e differenziali
\use{bm} %bold
\use[fulladjust]{marginnote} %to use marginnote for date notes
\use{arrayjobx}%array
\use[intlimits]{empheq} % Riquadri colorati attorno alle equazioni
\use{mathtools}
\use{circuitikz} % Disegnare i circuiti
\use{mathtools}
\use{stmaryrd} % [[ \llbracket ]] \rrbracket
\use{bussproofs} % dimostrazioni

%%%%%%%%%%%%%


%%%% QUIVER
\newcommand{\duepunti}{\,\mathchar\numexpr"6000+`:\relax\,}
% A TikZ style for curved arrows of a fixed height, due to AndréC.
\tikzset{curve/.style={settings={#1},to path={(\tikztostart)
    .. controls ($(\tikztostart)!\pv{pos}!(\tikztotarget)!\pv{height}!270:(\tikztotarget)$)
    and ($(\tikztostart)!1-\pv{pos}!(\tikztotarget)!\pv{height}!270:(\tikztotarget)$)
    .. (\tikztotarget)\tikztonodes}},
    settings/.code={\tikzset{quiver/.cd,#1}
        \def\pv##1{\pgfkeysvalueof{/tikz/quiver/##1}}},
    quiver/.cd,pos/.initial=0.35,height/.initial=0}

% TikZ arrowhead/tail styles.
\tikzset{tail reversed/.code={\pgfsetarrowsstart{tikzcd to}}}
\tikzset{2tail/.code={\pgfsetarrowsstart{Implies[reversed]}}}
\tikzset{2tail reversed/.code={\pgfsetarrowsstart{Implies}}}
% TikZ arrow styles.
\tikzset{no body/.style={/tikz/dash pattern=on 0 off 1mm}}
%%%%%%%%%%


%% DEFINIZIONI COMANDI MATEMATICI
\let\sin\relax %TOGLIE LA DEFINIZIONE SU "\sin"

% cambia la definizione di empty set
% ---
\let\oldemptyset\emptyset
% ---
% \let\emptyset\varnothing
% ---
% \let\emptyset\relax
% \newcommand{\emptyset}{\text{\textnormal{\O}}}
% ---

\DeclareMathOperator{\bounded}{bd}
\DeclareMathOperator{\sin}{sen}
\DeclareMathOperator{\epi}{Epi}
\DeclareMathOperator{\cl}{cl}
\DeclareMathOperator{\graph}{graph}
\DeclareMathOperator{\arcsec}{arcsec}
\DeclareMathOperator{\arccot}{arccot}
\DeclareMathOperator{\arccsc}{arccsc}
\DeclareMathOperator{\spettro}{Spettro}
\DeclareMathOperator{\nulls}{nullspace}
\DeclareMathOperator{\dom}{dom}
\DeclareMathOperator{\ar}{ar}
\DeclareMathOperator{\const}{Const}
\DeclareMathOperator{\fun}{Fun}
\DeclareMathOperator{\rel}{Rel}
\DeclareMathOperator{\altezza}{ht}
\let\det\relax %TOGLIE LA DEFINIZIONE SU "\det"
\DeclareMathOperator{\det}{det}
\DeclareMathOperator{\End}{End}
\DeclareMathOperator{\gl}{GL}
\def\Id{\mathrm{Id}}
\def\id{\mathrm{id}}
\DeclareMathOperator{\I}{\mathds{1}}
\DeclareMathOperator{\II}{II}
\DeclareMathOperator{\rank}{rank}
\DeclareMathOperator{\tr}{tr}
\DeclareMathOperator{\tc}{t.c.}
\DeclareMathOperator{\T}{T}
\DeclareMathOperator{\var}{Var}
\DeclareMathOperator{\cov}{Cov}
\DeclareMathOperator{\st}{st}
\DeclareMathOperator{\mon}{Mon}
\newcommand{\card}[1]{\left\vert #1 \right\vert}
\newcommand{\trasposta}[1]{\prescript{\text{T}}{}{#1}}
\newcommand{\1}{\mathds{1}}
\newcommand{\R}{\mathds{R}}
\newcommand{\diesis}{\#}
\newcommand{\bemolle}{\flat}
\newcommand{\nonstandard}[1]{\prescript{*}{}{#1}}
\newcommand{\starR}{\nonstandard{\R}}
\newcommand{\borel}{\mathscr{B}}
\newcommand{\lebesgue}[1]{\mathscr{L}\left(#1\right)}
\newcommand{\media}{\mathds{E}}
\newcommand{\K}{\mathds{K}}
\newcommand{\A}{\mathds{A}}
\newcommand{\Q}{\mathds{Q}}
\newcommand{\N}{\mathds{N}}
\newcommand{\C}{\mathds{C}}
\newcommand{\Z}{\mathds{Z}}
\newcommand{\qo}{\hspace{1em}\text{q.o.}\,}
\renewcommand{\tilde}[1]{\widetilde{#1}}
\renewcommand{\parallel}{\mathrel{/\mkern-5mu/}}
\newcommand{\parti}[2][]{\wp_{#1}(#2)}
\newcommand{\diff}[1]{\operatorname{d}_{#1}}
\let\oldvec\vec
\renewcommand{\vec}[1]{\overrightarrow{\vphantom{i}#1}}
\newcommand{\floor}[1]{\left\lfloor #1 \right\rfloor}
\newcommand{\cat}[1]{\mathbf{#1}}
\newcommand{\dfreccia}[1]{\xrightarrow{\ #1 \ }}
\newcommand{\sfreccia}[1]{\xleftarrow{\ #1 \ }}
\newcommand{\formalsum}[2]{{\sum_{#1}^{#2}}{\vphantom{\sum}}'}
\newcommand{\minim}[2]{\mu_{#1}\, \left(#2\right)}
\newcommand{\concat}{\null^{\frown}} % concatenazione di stringe
\newcommand{\godelcode}[1]{\langle\!\langle #1 \rangle\!\rangle}
\newcommand{\godeldec}[1]{(\!(#1)\!)}
\newcommand{\termcode}[1]{\ulcorner #1\urcorner}
\newcommand{\partialto}{\dashrightarrow}
\newcommand{\restricted}{\upharpoonright}
\newcommand{\embeds}{\precsim}
\newcommand{\surjects}{\twoheadrightarrow}
\newcommand{\equipotenti}{\asymp}
%% \newcommand{\dotplus}{\mathbin{\dot{+}}} %% A quanto pare esiste già
\newcommand{\bigdot}{\mathbin{\boldsymbol{\cdot}}}
\newcommand{\dotexp}[1]{^{.#1}}
\newcommand{\conv}{\mathbin{*}}
\newcommand{\convolution}[2]{(#1\conv #2)}
\newcommand{\nil}{\mathfrak{N}}
\newcommand{\divisore}{\mathrel{|}}
\newcommand{\simplesso}[1]{\mathrm{e}_{#1}}

\renewcommand{\iff}{\mathrel{\longleftrightarrow}} %% Notazione Logica.
\newcommand{\oldiff}{\mathrel{\Longleftrightarrow}}
\renewcommand{\implies}{\mathrel{\rightarrow}} %% Notazione Logica
\newcommand{\oldimplies}{\mathrel{\Longrightarrow}}
\renewcommand{\impliedby}{\mathrel{\leftarrow}} %% Notazione Logica
\newcommand{\oldimpliedby}{\mathrel{\Longleftarrow}}

\newcommand{\IFF}{\quad\Longleftrightarrow\quad}
\newcommand{\IMPLICA}{\quad\Longrightarrow\quad}


\renewcommand{\descriptionlabel}[1]{\hspace{\labelsep}\normalfont #1} % remove bold from description


%% Definizione di Divergenza di K-L

\DeclarePairedDelimiterX{\infdivx}[2]{(}{)}{%
  #1\;\delimsize\|\;#2%
}
\newcommand{\kldiv}{D_{KL}\infdivx}

%% Definizione di \dotminus

\makeatletter
\newcommand{\dotminus}{\mathbin{\text{\@dotminus}}}

\newcommand{\@dotminus}{%
  \ooalign{\hidewidth\raise1ex\hbox{.}\hidewidth\cr$\m@th-$\cr}%
}
\makeatother

%tramite i prossimi due comandi posso decidere come scrivere i logaritmi naturali in tutti i documenti: ho infatti eliminato qualsiasi differenza tra "ln" e "log": se si vuole qualcosa di diverso bisogna inserire manualmente il tutto
\let\ln\relax
\DeclareMathOperator{\ln}{ln}
\let\log\relax
\DeclareMathOperator{\log}{log}
%%%%%%

%% NUOVI COMANDI
\newcommand{\straniero}[1]{\textit{#1}} %parole straniere
\newcommand{\titolo}[1]{\textsc{#1}} %titoli
\newcommand{\qedd}{\tag*{$\blacksquare$}} %qed per ambienti matemastici
\renewcommand{\qedsymbol}{$\blacksquare$} %modifica colore qed
\newcommand{\ooverline}[1]{\overline{\overline{#1}}}
\newcommand{\circoletto}[1]{\left(#1\right)^{\text{o}}}
%
\newcommand{\qmatrice}[1]{\begin{pmatrix}
#1_{11} & \cdots & #1_{1n}\\
\vdots & \ddots & \vdots \\
#1_{m1} & \cdots & #1_{mn}
\end{pmatrix}}
%
\newcommand{\parentesi}[2]{%
\underset{#1}{\underbrace{#2}}%
}
%
\newcommand{\norma}[1]{% Norma
\left\lVert#1\right\rVert%
}
\newcommand{\scalare}[2]{% Scalare
\left\langle #1, #2\right\rangle
}
%%%%%

%% RESTRIZIONI
\newcommand{\referenze}[2]{
        \phantomsection{}#2\textsuperscript{\textcolor{blue}{\textbf{#1}}}
}

\let\restriction\relax

\def\restriction#1#2{\mathchoice
              {\setbox1\hbox{${\displaystyle #1}_{\scriptstyle #2}$}
              \restrictionaux{#1}{#2}}
              {\setbox1\hbox{${\textstyle #1}_{\scriptstyle #2}$}
              \restrictionaux{#1}{#2}}
              {\setbox1\hbox{${\scriptstyle #1}_{\scriptscriptstyle #2}$}
              \restrictionaux{#1}{#2}}
              {\setbox1\hbox{${\scriptscriptstyle #1}_{\scriptscriptstyle #2}$}
              \restrictionaux{#1}{#2}}}
\def\restrictionaux#1#2{{#1\,\smash{\vrule height .8\ht1 depth .85\dp1}}_{\,#2}}
%%%%%%%%%%%

%%% FORMATTAZIONE FOOTNOTEMARK

\def\footnotemarkformatting#1{[#1]}
\renewcommand{\thefootnote}{\footnotemarkformatting{\arabic{footnote}}}

%% SEZIONE GRAFICA
\use{tikz}
\usetikzlibrary{matrix, patterns, calc, decorations.pathreplacing, hobby, decorations.markings, decorations.pathmorphing, babel}
\use{tikz-3dplot}
\use{mathrsfs} %per geogebra
\use{tikz-cd}
\tikzset
{
  %surface/.style={fill=black!10, shading=ball,fill opacity=0.4},
  plane/.style={black,pattern=north east lines},
  curve/.style={black,line width=0.5mm},
  dritto/.style={decoration={markings,mark=at position 0.5 with {\arrow{Stealth}}}, postaction=decorate},
  rovescio/.style={decoration={markings,mark=at position 0.5 with {\arrow{Stealth[reversed]}}}, postaction=decorate}
}
\use{pgfplots} % stampare le funzioni
        \pgfplotsset{/pgf/number format/use comma,compat=1.15}
        %\pgfplotsset{compat=1.15} %per geogebra
        \usepgfplotslibrary{fillbetween, polar}
%%%%%%

%% CITAZIONI
\use{lineno}

\newcommand{\citazione}[1]{%
  \begin{quotation}
  \begin{linenumbers}
  \modulolinenumbers[5]
  \begingroup
  \setlength{\parindent}{0cm}
  \noindent #1
  \endgroup
  \end{linenumbers}
  \end{quotation}\setcounter{linenumber}{1}
  }
%%%%%%

%%%%%%%%%%%%%%%%%%%%%%%%%%%%%%%%%%%%%%%%%%%%
%%%%%%%%%%%%%%%%%%%%%%%%%%%%%%%%%%%%%%%%%%%%

%% AMS THM

\theoremstyle{definition}% default
\newtheorem{thm}{Teorema}[section]
\newtheorem{lem}[thm]{Lemma}
\newtheorem{prop}[thm]{Proposizione}
\newtheorem{cor}[thm]{Corollario}
\newtheorem{esempio}[thm]{Esempio}
\theoremstyle{plain}
\newtheorem{definizione}[thm]{Definizione}
\theoremstyle{remark}
\newtheorem*{oss}{Osservazione}


%%%%%%%%%%%%%%%%%%%%%%%%%%%%%%%%%%%%%%%%%%%%
%%%%%%%%%%%%%%%%%%%%%%%%%%%%%%%%%%%%%%%%%%%%

\use{hyperref}
\hypersetup{%
        pdfauthor={Davide Peccioli},
        pdfsubject={},
        allcolors=black,
        citecolor=black,
%	colorlinks=true,
        bookmarksopen=true}
\setcounter{secnumdepth}{0} % rimuove i numeri di sezione senza rimuovere le ref
\renewcommand{\href}[2]{\textcolor{blue}{#2}} % disabilita il comando href
\use{enotez} %
\setenotez{%
 mark-format = \footnotemarkformatting % Mette i numeri tra parentesi quadre%
}\let\footnote=\endnote % rende tutte le note a pié pagina come delle note a fine file 


\let\olddocument\document % modifico l'ambiende documenti per non dover stampare \printendnote
\let\oldenddocument\enddocument
\renewenvironment{document}%
{%
  \olddocument
}{%
  \printendnotes\oldenddocument
}
\renewcommand{\thethm}{\arabic{thm}}

\usepackage[hyperref]{biblatex}
\addbibresource{~/Documents/org/roam/bib/master.bib}
\author{Davide Peccioli}
\date{\today}
\title{Caratterizzazione dei compatti dello spazio di Baire}
\begin{document}

\section{Proposizione}
\label{sec:org2e37175}
A set \(A \subseteq \omega^\omega\) is \textbf{bounded} if there is \(z \in \omega^\omega\) such that for all \(x \in A\) we have \(x(n) \leq z(n)\) for all \(n \in \omega\). Prove that the following conditions are equivalent for an arbitrary \(F \subseteq \omega^\omega\) :
\begin{enumerate}
\item \(F\) is \href{20250103163701-spazio_topologico_compatto.org}{compact};
\item \(F\) is \href{20250103145124-topologia.org}{closed} and \href{20250327104804-insieme_limitato_dello_spazio_di_baire.org}{bounded};
\item \(F=[T]\) with \(T\) a finitely branching tree (i.e. every node in \(T\) has only finitely many successors).
\end{enumerate}

Conclude that \(A \subseteq \omega^\omega\) is contained in a compact set (equivalently, has compact closure) if and only \(A\) is bounded, and therefore \(\omega^\omega\) is not locally compact.
\subsection{Dimostrazione}
\label{sec:orgb2d2ff3}

Sia, per ogni \(s \in \omega^{<\omega}\): \(\bm{N}_{s} \coloneqq \set{x \in\omega^{\omega}\mid x\upharpoonright s =s}\).
\subsubsection{a. implica b.}
\label{sec:org0d11e11}

Siccome \(\omega^{\omega}\) è \href{20250301193401-spazio_topologico_metrizzabile.org}{metrizzabile} è uno \href{20250109155715-spazio_topologico_di_hausdorff.org}{spazio T2}. Se \(F\) è \href{20250103163701-spazio_topologico_compatto.org}{compatto} allora è \href{20250103145124-topologia.org}{chiuso}. Resta da dimostrare che \(F\) sia \href{20250327104804-insieme_limitato_dello_spazio_di_baire.org}{limitato}.

Per ogni \(x \in F\) e per ogni \(n \in \omega\) si consideri l'aperto
\begin{equation*}
U_{x,n} \coloneqq \set{y \in F\mid y\upharpoonright n = x\upharpoonright n}
\end{equation*}
Si ottiene quindi \(U_{x} \coloneqq F\cap \bigcup_{n \in \omega}U_{x,n}\) aperto in \(F\), e pertanto \(\set{U_{x}}_{x \in F}\) è un \href{20250103164252-ricoprimento.org}{ricoprimento} aperto di \(F\).

Dal momento che \(F\) è compatto, esiste \(\set{x_{1},\dots,x_{m}} \subseteq F\) tali che
\begin{equation*}
\bigcup_{i=1,\dots,m} U_{x_{i}} = F
\end{equation*}
e pertanto è sufficiente porre \(z \in \omega^{\omega}\):
\begin{equation*}
z(\eta) \coloneqq \max \set{x_{i}(\eta)}
\end{equation*}
per ottenere la tesi.
\subsubsection{b. implica a.}
\label{sec:org4a3f206}

Siccome \(F\) è limitato, esiste \(z \in \omega^{\omega}\) tale che per ogni \(x \in F\) e per ogni \(n \in \omega\) si ha
\begin{equation*}
x(n)\le z(n)
\end{equation*}

Si consideri quindi, per ogni \(n \in \omega\): \(A_{n} \subseteq \omega\), \(A_{n} \coloneqq z(n)+1 = \set{0,1,2,\dots,z(n)}\). Questo è compatto in quanto finito con la \href{20250317165247-topologia_discreta.org}{topologia discreta}.

Per il \href{20250401124050-teorema_di_tichonov.org}{Teorema di Tichonoff} \(A\coloneqq \prod_{n \in \omega} A_{n}\) è compatto. Inoltre
\begin{equation*}
F \subseteq A
\end{equation*}
in quanto, se \(x \in F\) allora per ogni \(n \in \omega\): \(x(n)< z(n)+1\) i.e. \(x(n) \in \left(z(n) + 1\right) = A_{n}\) e pertanto \(x \in A\).

Quindi \(F\) è chiuso dentro \(A\) compatto, \href{20250401125136-chiuso_in_un_compatto_e_compatto.org}{quindi} \(F\) è compatto.
\subsubsection{b. implica c.}
\label{sec:org2676bf6}
Per la proposizione 1.3.3 esiste un albero potato \(T_{F}\) tale che \(F=[T_{F}]\), con
\begin{equation*}
T_{F} \coloneqq \set{x\upharpoonright n\mid x \in F \,\land\, n \in \omega}
\end{equation*}

Resta da dimostrare che \(T_{F}\) sia a ramificazione finita. Se per assurdo esistesse \(s \in T_{F}\) tale che, per ogni \(i \in\omega\):
\begin{equation*}
s\concat i \in T_{F}
\end{equation*}
Pertanto, per ogni \(i \in \omega\), esiste \(x_{i} \in F\) tale che \(x_{i}\upharpoonright \operatorname{lh}(s) + 1 = s\concat i\) ed in particolar modo, per ogni \(i \in \omega\) vale che \(x_{i}\left(\operatorname{lh}(s)+2\right)=i\). Per ogni \(z \in \omega^{\omega}\), quindi, esiste \(n \coloneqq \operatorname{lh}(s) +2\) ed esiste \(x \in F\), \(x \coloneqq x_{i_{0}}\) con \(i_{0}=z(n)+1\) tale per cui
\begin{equation*}
z\left(n\right) \le x(n) = x_{i_{0}}(n) = i_{0} = z(n)+1.
\end{equation*}

Assurdo poiché \(F\) è limitato.
\subsubsection{c. implica b.}
\label{sec:org9e80a77}

Sia \(T\) un albero a ramificazione finita, ovvero tale che per ogni \(s \in T\):
\begin{equation*}
R_{s} \coloneqq \set{n \in \omega\mid s\concat n \in T} \subseteq \omega
\end{equation*}
è un insieme finito, con
\begin{equation*}
F = [T] = \set{x \in\omega^{\omega}\mid \forall\, n \in \omega \ (x\upharpoonright n \in T)}
\end{equation*}

Per la proposizione 1.3.3 \(F\) è chiuso, e pertanto resta da dimostrare che \(F\) sia limitato.

\begin{itemize}
\item Per ogni \(n \in \omega\), \(T_{n} \coloneqq \set{t \in T\mid \operatorname{lh}(t) = n}\) è finito.

Per induzione, \(T_{0} = \set{\emptyset}\). Se \(T_{n}\) è finito, allora
\begin{equation*}
  	T_{n+1} = \set{t \in T\mid \exists\, s \in T_{n} \,\land\, \exists\, m \in \omega\ (s\concat m = t)}
\end{equation*}
ovvero
\begin{equation*}
  	T_{n+1} = \bigcup_{s \in T_{n}} \bigcup_{m \in R_{s}} \set{s\concat m}
\end{equation*}
unione finita di singoletti, e pertanto finito.
\item Si definisce \(z \in \omega^{\omega}\) come segue:
\begin{equation*}
  	\forall\, n \in \omega: \quad z(n) \coloneqq 1+\max_{s \in T_{n}}\max R_{s}
\end{equation*}
\item Claim: \(z\) così definito è tale che, per ogni \(x \in F\) e per ogni \(n \in \omega\): \(z(n)\ge x(n)\).

Infatti, se per assurdo esistesse \(\tilde{x} \in F\) e \(\tilde{n} \in \omega\) tali che \(\tilde{x}(\tilde{n})> z(\tilde{n})\), allora \(\tilde{x}\upharpoonright \tilde{n}+1 \in T\) poiché \(F=[T]\), ed in particolar modo,
\begin{equation*}
  	(\tilde{x}\upharpoonright \tilde{n}+1) \in T_{\tilde{n}+1}
\end{equation*}
Pertanto \(\tilde{x}(\tilde{n}) \in R_{\tilde{x}\upharpoonright \tilde{n}}\) e \(\tilde{x}\upharpoonright \tilde{n} \in T_{\tilde{n}}\). Quindi \(z(\tilde{n})\ge 1 + \tilde{x}(\tilde{n})\). Assurdo.
\end{itemize}
\subsubsection{Locale compattezza}
\label{sec:org2a41607}

Sia \(A \subseteq \omega^{\omega}\).
\begin{itemize}
\item Se esiste \(C \subseteq \omega^{\omega}\) compatto e tale che \(A \subseteq C\), allora \(C\) è limitato e quindi \(A\) è limitato.
\item Se \(A\) è limitato e \(z \in \omega^{\omega}\) ne è testimone, allora sia \((a_{n})_{n\in \omega} \subseteq A\) una successione convergente ad \(a\).

Allora per ogni \(n \in \omega\), \(a \in \bm{N}_{a\upharpoonright n+1}\), e quindi esiste \(N \in\omega\) tale che \(a_{N} \in \bm{N}_{a\upharpoonright n+1}\) e pertanto
\begin{equation*}
  	a(n) = a_{N}(n) \le z(n)
\end{equation*}
dove l'ultima disuguaglianza vale perché \(a_{N} \in A\) limitato.

Per la caratterizzazione della chiusura in termini di successioni, si è dimostrato che \(\operatorname{Cl}(A)\) è limitato (e ovviamente chiuso), quindi compatto, e
\begin{equation*}
  	A \subseteq \operatorname{Cl}(A)
\end{equation*}
\end{itemize}
\end{document}
