% Created 2026-02-07 Sat 19:31
% Intended LaTeX compiler: pdflatex
\documentclass[10pt]{article}
%% CREATO CON ORG - EMACS
\newcommand{\use}[2][]{\usepackage[#1]{#2}}
% PACCHETTI FONDAMENTLAI
\use[utf8]{inputenc}
\use[T1]{fontenc}
\use{graphicx}
\use{longtable}
\use{wrapfig}
\use{rotating}
\use[normalem]{ulem}
\use{amsmath}
\use{amsthm}
\use{amssymb}

\use{eucal} % Cambia mathcal{...}

\use{capt-of}
\use[italian]{babel}
\use[babel]{csquotes}
% bib la TEX lo carica in automatico org-cite
\use{microtype}
\use{lmodern}
\use{subfig} % sottofigure
\use{multicol} % due colonne
\use{lipsum} % lorem ipsum
\use{color} % colori in latex
\use{parskip} % rimuove l'indentazione dei nuovi paragrafi %% Add parbox=false to all new tcolorbox
\use{centernot}
\use[outline]{contour}\contourlength{3pt}
\use{fancyhdr}
\use{layout}
\use[most]{tcolorbox} % Riquadri colorati
\use{ifthen} % IFTHEN
\use{geometry}

% pacchetti matematica
\use{yhmath}
\use{dsfont}
\use{mathrsfs}
\use{cancel} % semplificare
\use{polynom} %divisione tra polinomi
\use{forest} % grafi ad albero
\use{booktabs} % tabelle
\use{commath} %simboli e differenziali
\use{bm} %bold
\use[fulladjust]{marginnote} %to use marginnote for date notes
\use{arrayjobx}%array
\use[intlimits]{empheq} % Riquadri colorati attorno alle equazioni
\use{mathtools}
\use{circuitikz} % Disegnare i circuiti
\use{mathtools}
\use{stmaryrd} % [[ \llbracket ]] \rrbracket
\use{bussproofs} % dimostrazioni

%%%%%%%%%%%%%


%%%% QUIVER
\newcommand{\duepunti}{\,\mathchar\numexpr"6000+`:\relax\,}
% A TikZ style for curved arrows of a fixed height, due to AndréC.
\tikzset{curve/.style={settings={#1},to path={(\tikztostart)
    .. controls ($(\tikztostart)!\pv{pos}!(\tikztotarget)!\pv{height}!270:(\tikztotarget)$)
    and ($(\tikztostart)!1-\pv{pos}!(\tikztotarget)!\pv{height}!270:(\tikztotarget)$)
    .. (\tikztotarget)\tikztonodes}},
    settings/.code={\tikzset{quiver/.cd,#1}
        \def\pv##1{\pgfkeysvalueof{/tikz/quiver/##1}}},
    quiver/.cd,pos/.initial=0.35,height/.initial=0}

% TikZ arrowhead/tail styles.
\tikzset{tail reversed/.code={\pgfsetarrowsstart{tikzcd to}}}
\tikzset{2tail/.code={\pgfsetarrowsstart{Implies[reversed]}}}
\tikzset{2tail reversed/.code={\pgfsetarrowsstart{Implies}}}
% TikZ arrow styles.
\tikzset{no body/.style={/tikz/dash pattern=on 0 off 1mm}}
%%%%%%%%%%


%% DEFINIZIONI COMANDI MATEMATICI
\let\sin\relax %TOGLIE LA DEFINIZIONE SU "\sin"

% cambia la definizione di empty set
% ---
\let\oldemptyset\emptyset
% ---
% \let\emptyset\varnothing
% ---
% \let\emptyset\relax
% \newcommand{\emptyset}{\text{\textnormal{\O}}}
% ---

\DeclareMathOperator{\bounded}{bd}
\DeclareMathOperator{\sin}{sen}
\DeclareMathOperator{\epi}{Epi}
\DeclareMathOperator{\cl}{cl}
\DeclareMathOperator{\graph}{graph}
\DeclareMathOperator{\arcsec}{arcsec}
\DeclareMathOperator{\arccot}{arccot}
\DeclareMathOperator{\arccsc}{arccsc}
\DeclareMathOperator{\spettro}{Spettro}
\DeclareMathOperator{\nulls}{nullspace}
\DeclareMathOperator{\dom}{dom}
\DeclareMathOperator{\ar}{ar}
\DeclareMathOperator{\const}{Const}
\DeclareMathOperator{\fun}{Fun}
\DeclareMathOperator{\rel}{Rel}
\DeclareMathOperator{\altezza}{ht}
\let\det\relax %TOGLIE LA DEFINIZIONE SU "\det"
\DeclareMathOperator{\det}{det}
\DeclareMathOperator{\End}{End}
\DeclareMathOperator{\gl}{GL}
\def\Id{\mathrm{Id}}
\def\id{\mathrm{id}}
\DeclareMathOperator{\I}{\mathds{1}}
\DeclareMathOperator{\II}{II}
\DeclareMathOperator{\rank}{rank}
\DeclareMathOperator{\tr}{tr}
\DeclareMathOperator{\tc}{t.c.}
\DeclareMathOperator{\T}{T}
\DeclareMathOperator{\var}{Var}
\DeclareMathOperator{\cov}{Cov}
\DeclareMathOperator{\st}{st}
\DeclareMathOperator{\mon}{Mon}
\newcommand{\card}[1]{\left\vert #1 \right\vert}
\newcommand{\trasposta}[1]{\prescript{\text{T}}{}{#1}}
\newcommand{\1}{\mathds{1}}
\newcommand{\R}{\mathds{R}}
\newcommand{\diesis}{\#}
\newcommand{\bemolle}{\flat}
\newcommand{\nonstandard}[1]{\prescript{*}{}{#1}}
\newcommand{\starR}{\nonstandard{\R}}
\newcommand{\borel}{\mathscr{B}}
\newcommand{\lebesgue}[1]{\mathscr{L}\left(#1\right)}
\newcommand{\media}{\mathds{E}}
\newcommand{\K}{\mathds{K}}
\newcommand{\A}{\mathds{A}}
\newcommand{\Q}{\mathds{Q}}
\newcommand{\N}{\mathds{N}}
\newcommand{\C}{\mathds{C}}
\newcommand{\Z}{\mathds{Z}}
\newcommand{\qo}{\hspace{1em}\text{q.o.}\,}
\renewcommand{\tilde}[1]{\widetilde{#1}}
\renewcommand{\parallel}{\mathrel{/\mkern-5mu/}}
\newcommand{\parti}[2][]{\wp_{#1}(#2)}
\newcommand{\diff}[1]{\operatorname{d}_{#1}}
\let\oldvec\vec
\renewcommand{\vec}[1]{\overrightarrow{\vphantom{i}#1}}
\newcommand{\floor}[1]{\left\lfloor #1 \right\rfloor}
\newcommand{\cat}[1]{\mathbf{#1}}
\newcommand{\dfreccia}[1]{\xrightarrow{\ #1 \ }}
\newcommand{\sfreccia}[1]{\xleftarrow{\ #1 \ }}
\newcommand{\formalsum}[2]{{\sum_{#1}^{#2}}{\vphantom{\sum}}'}
\newcommand{\minim}[2]{\mu_{#1}\, \left(#2\right)}
\newcommand{\concat}{\null^{\frown}} % concatenazione di stringe
\newcommand{\godelcode}[1]{\langle\!\langle #1 \rangle\!\rangle}
\newcommand{\godeldec}[1]{(\!(#1)\!)}
\newcommand{\termcode}[1]{\ulcorner #1\urcorner}
\newcommand{\partialto}{\dashrightarrow}
\newcommand{\restricted}{\upharpoonright}
\newcommand{\embeds}{\precsim}
\newcommand{\surjects}{\twoheadrightarrow}
\newcommand{\equipotenti}{\asymp}
%% \newcommand{\dotplus}{\mathbin{\dot{+}}} %% A quanto pare esiste già
\newcommand{\bigdot}{\mathbin{\boldsymbol{\cdot}}}
\newcommand{\dotexp}[1]{^{.#1}}
\newcommand{\conv}{\mathbin{*}}
\newcommand{\convolution}[2]{(#1\conv #2)}
\newcommand{\nil}{\mathfrak{N}}
\newcommand{\divisore}{\mathrel{|}}
\newcommand{\simplesso}[1]{\mathrm{e}_{#1}}

\renewcommand{\iff}{\mathrel{\longleftrightarrow}} %% Notazione Logica.
\newcommand{\oldiff}{\mathrel{\Longleftrightarrow}}
\renewcommand{\implies}{\mathrel{\rightarrow}} %% Notazione Logica
\newcommand{\oldimplies}{\mathrel{\Longrightarrow}}
\renewcommand{\impliedby}{\mathrel{\leftarrow}} %% Notazione Logica
\newcommand{\oldimpliedby}{\mathrel{\Longleftarrow}}

\newcommand{\IFF}{\quad\Longleftrightarrow\quad}
\newcommand{\IMPLICA}{\quad\Longrightarrow\quad}


\renewcommand{\descriptionlabel}[1]{\hspace{\labelsep}\normalfont #1} % remove bold from description


%% Definizione di Divergenza di K-L

\DeclarePairedDelimiterX{\infdivx}[2]{(}{)}{%
  #1\;\delimsize\|\;#2%
}
\newcommand{\kldiv}{D_{KL}\infdivx}

%% Definizione di \dotminus

\makeatletter
\newcommand{\dotminus}{\mathbin{\text{\@dotminus}}}

\newcommand{\@dotminus}{%
  \ooalign{\hidewidth\raise1ex\hbox{.}\hidewidth\cr$\m@th-$\cr}%
}
\makeatother

%tramite i prossimi due comandi posso decidere come scrivere i logaritmi naturali in tutti i documenti: ho infatti eliminato qualsiasi differenza tra "ln" e "log": se si vuole qualcosa di diverso bisogna inserire manualmente il tutto
\let\ln\relax
\DeclareMathOperator{\ln}{ln}
\let\log\relax
\DeclareMathOperator{\log}{log}
%%%%%%

%% NUOVI COMANDI
\newcommand{\straniero}[1]{\textit{#1}} %parole straniere
\newcommand{\titolo}[1]{\textsc{#1}} %titoli
\newcommand{\qedd}{\tag*{$\blacksquare$}} %qed per ambienti matemastici
\renewcommand{\qedsymbol}{$\blacksquare$} %modifica colore qed
\newcommand{\ooverline}[1]{\overline{\overline{#1}}}
\newcommand{\circoletto}[1]{\left(#1\right)^{\text{o}}}
%
\newcommand{\qmatrice}[1]{\begin{pmatrix}
#1_{11} & \cdots & #1_{1n}\\
\vdots & \ddots & \vdots \\
#1_{m1} & \cdots & #1_{mn}
\end{pmatrix}}
%
\newcommand{\parentesi}[2]{%
\underset{#1}{\underbrace{#2}}%
}
%
\newcommand{\norma}[1]{% Norma
\left\lVert#1\right\rVert%
}
\newcommand{\scalare}[2]{% Scalare
\left\langle #1, #2\right\rangle
}
%%%%%

%% RESTRIZIONI
\newcommand{\referenze}[2]{
        \phantomsection{}#2\textsuperscript{\textcolor{blue}{\textbf{#1}}}
}

\let\restriction\relax

\def\restriction#1#2{\mathchoice
              {\setbox1\hbox{${\displaystyle #1}_{\scriptstyle #2}$}
              \restrictionaux{#1}{#2}}
              {\setbox1\hbox{${\textstyle #1}_{\scriptstyle #2}$}
              \restrictionaux{#1}{#2}}
              {\setbox1\hbox{${\scriptstyle #1}_{\scriptscriptstyle #2}$}
              \restrictionaux{#1}{#2}}
              {\setbox1\hbox{${\scriptscriptstyle #1}_{\scriptscriptstyle #2}$}
              \restrictionaux{#1}{#2}}}
\def\restrictionaux#1#2{{#1\,\smash{\vrule height .8\ht1 depth .85\dp1}}_{\,#2}}
%%%%%%%%%%%

%%% FORMATTAZIONE FOOTNOTEMARK

\def\footnotemarkformatting#1{[#1]}
\renewcommand{\thefootnote}{\footnotemarkformatting{\arabic{footnote}}}

%% SEZIONE GRAFICA
\use{tikz}
\usetikzlibrary{matrix, patterns, calc, decorations.pathreplacing, hobby, decorations.markings, decorations.pathmorphing, babel}
\use{tikz-3dplot}
\use{mathrsfs} %per geogebra
\use{tikz-cd}
\tikzset
{
  %surface/.style={fill=black!10, shading=ball,fill opacity=0.4},
  plane/.style={black,pattern=north east lines},
  curve/.style={black,line width=0.5mm},
  dritto/.style={decoration={markings,mark=at position 0.5 with {\arrow{Stealth}}}, postaction=decorate},
  rovescio/.style={decoration={markings,mark=at position 0.5 with {\arrow{Stealth[reversed]}}}, postaction=decorate}
}
\use{pgfplots} % stampare le funzioni
        \pgfplotsset{/pgf/number format/use comma,compat=1.15}
        %\pgfplotsset{compat=1.15} %per geogebra
        \usepgfplotslibrary{fillbetween, polar}
%%%%%%

%% CITAZIONI
\use{lineno}

\newcommand{\citazione}[1]{%
  \begin{quotation}
  \begin{linenumbers}
  \modulolinenumbers[5]
  \begingroup
  \setlength{\parindent}{0cm}
  \noindent #1
  \endgroup
  \end{linenumbers}
  \end{quotation}\setcounter{linenumber}{1}
  }
%%%%%%

%%%%%%%%%%%%%%%%%%%%%%%%%%%%%%%%%%%%%%%%%%%%
%%%%%%%%%%%%%%%%%%%%%%%%%%%%%%%%%%%%%%%%%%%%

%% AMS THM

\theoremstyle{definition}% default
\newtheorem{thm}{Teorema}[section]
\newtheorem{lem}[thm]{Lemma}
\newtheorem{prop}[thm]{Proposizione}
\newtheorem{cor}[thm]{Corollario}
\newtheorem{esempio}[thm]{Esempio}
\theoremstyle{plain}
\newtheorem{definizione}[thm]{Definizione}
\theoremstyle{remark}
\newtheorem*{oss}{Osservazione}


%%%%%%%%%%%%%%%%%%%%%%%%%%%%%%%%%%%%%%%%%%%%
%%%%%%%%%%%%%%%%%%%%%%%%%%%%%%%%%%%%%%%%%%%%

\use{hyperref}
\hypersetup{%
        pdfauthor={Davide Peccioli},
        pdfsubject={},
        allcolors=black,
        citecolor=black,
%	colorlinks=true,
        bookmarksopen=true}
\setcounter{secnumdepth}{0} % rimuove i numeri di sezione senza rimuovere le ref
\renewcommand{\href}[2]{\textcolor{blue}{#2}} % disabilita il comando href
\use{enotez} %
\setenotez{%
 mark-format = \footnotemarkformatting % Mette i numeri tra parentesi quadre%
}\let\footnote=\endnote % rende tutte le note a pié pagina come delle note a fine file 


\let\olddocument\document % modifico l'ambiende documenti per non dover stampare \printendnote
\let\oldenddocument\enddocument
\renewenvironment{document}%
{%
  \olddocument
}{%
  \printendnotes\oldenddocument
}
\renewcommand{\thethm}{\arabic{thm}}

\usepackage[hyperref]{biblatex}
\addbibresource{~/Documents/org/roam/bib/master.bib}
\def\L{\operatorname{L}}
\def\Frvar{\operatorname{Frvar}}
\def\M{\mathcal{M}}
\def\NN{\mathcal{N}}
\def\Reg{\operatorname{Reg}}
\def\[[{\llbracket}
\def\]]{\rrbracket}
\def\defaultbooleanmodel{B}
\def\defaultmodel{\M}
\def\eval#1{\[[#1\]]_{\defaultbooleanmodel}^{\defaultmodel}}
\author{Matteo Viale}
\date{\today}
\title{Boolean valued semantics for infinitary logics [CORSO]}
\begin{document}

\maketitle

\href{https://www.mathphd.unito.it/do/corsi.pl/Show?\_id=o1lj}{See the campusnet webpage}, and the \href{https://drive.google.com/drive/folders/1EnAASHseIcdUT5CfzvPMmaQbQODLSTnt?usp=sharing}{course material}.
\section{Lecture 1 -  \textit{<2025-11-17 Mon>}}
\label{sec:orga8e1c54}

We work in ZFC

\begin{definizione}
Given a cardinal \(\lambda\), \(\L\) is a \(\lambda\)-signature if
\begin{equation*}
\L=\set{R_{i} \mid i \in I} \cup \set{f_{j} \mid j \in J} \cup \set{c_{k} \mid k \in K}
\end{equation*}
such that:
\begin{itemize}
\item \(R_{i}\) is an \(\alpha_{i}\)-ary predicat simbpol for some \(\alpha_{i}<\lambda\) for all \(i \in I\);
\item \(f_{j}\) is an \(\beta_{j}\)-ary function simbpol for some \(\beta_{j}<\lambda\) for all \(j \in J\);
\item \(c_{k}\) is a constant symbol for all \(k \in K\).
\end{itemize}
\end{definizione}

\uline{Note}: if \(\lambda =\omega\) a \(\lambda\)-signature is a first order signature.

\begin{definizione}
Let \(\L\) be a \(\lambda\)-signature. We fix \(\set{x_{\alpha} \mid \alpha<\lambda}\) a set of free variable.

The \(\L\)-terms are obtained as expected:
\begin{itemize}
\item if \(x\) is a variabile, it is a term;
\item if \(f_{j}\) is \(\beta_{j}\)-ary and \((t_{\beta} \mid \beta<\beta_{j})\) is a sequence of terms, than
\begin{equation*}
  f_{J}(t_{\beta} \mid \beta<\beta_{j})
\end{equation*}
is a term for all \(j \in J\);
\item \(c_{k}\) is a term for all \(k \in K\).
\end{itemize}
The \(\L\)-atomic formulae are obtained as expected:
\begin{itemize}
\item if \(R_{i}\) is an \(\alpha_{i}\)-ary relation symbol and \((t_{\alpha} \mid \alpha<\alpha_{i})\) is a sequence of terms, then
\begin{equation*}
  R_{i}(t_{\alpha} \mid \alpha<\alpha_{i})
\end{equation*}
is an atomic \(\L\)-formula.
\end{itemize}
\end{definizione}
\begin{oss}
In both formulae and terms only \(<\lambda\) variables can occure (check through induction).
\end{oss}

\begin{definizione}
Given \(\kappa, \lambda\) infinite regular cardinals and \(\L\) a \(\lambda\)-signature, the \(\L_{\kappa,\lambda}\)-formulae are defined as follows:
\begin{itemize}
\item every atomic \(\L\)-formula is an \(\L_{\kappa,\lambda}\)-formula
\item if \(\varphi\) is an \(\L_{\kappa,\lambda}\)-formula, then \(\lnot\varphi\) is;
\item if
\(U \in [\lambda]^{<\lambda} %
  \coloneqq \set{X \subseteq \lambda \mid \card{X}<\lambda}\)
and \(\varphi\) is a \(\L_{\kappa,\lambda}\)-formula, then
\begin{equation*}
  \forall \set{x_{\alpha} \mid \alpha \in U}\ \varphi,\qquad %
  \exists \set{x_{\alpha} \mid \alpha \in U}\ \varphi,
\end{equation*}
are \(\L_{\kappa,\lambda}\)-formulae;
\item if \(\card{Z}<\kappa\) and \(\set{\varphi_{z} \mid z \in Z} \subseteq \L_{\kappa,\lambda}\)-formulae and there is \(U \subseteq [\lambda]^{<\lambda}\) s.t. for each \(z \in Z\)
\begin{equation*}
  \Frvar(\varphi_{z}) \subseteq \set{x_{\alpha} \mid \alpha \in U}
\end{equation*}
then the following are \(\L_{\kappa,\lambda}\)-formulae:
\begin{equation*}
  \bigwedge_{z \in Z} \varphi_{z}, \qquad %
  \bigvee_{z \in Z} \varphi_{z}.
\end{equation*}
\end{itemize}
\end{definizione}

\textbf{\textbf{Definizione per induzione di \(\Frvar\)}}.

\begin{esempio}
In \(\L_{\omega_{1},\omega_{1}}\), consider \(\L_{\kappa,\lambda}=\set{R}\) with \(R\) a binary relation symbol. Consider these axioms:
\begin{itemize}
\item Order axioms of \(R\);
\item this axiom:
\begin{equation*}
  \forall (x_{n} \mid n <\omega)\ \left[%
  \left[\bigwedge_{n<m<\omega}(x_{n} \mathrel{R} x_{m})\right] \implies
  \left[\exists y\ \bigwedge_{n \in \omega} x_{n} \mathrel{R} y\right]\right]
\end{equation*}
\end{itemize}
One can check that a Tarski-Model has to be an uncountable linear order.
\end{esempio}

\begin{esempio}
In \(\L_{\omega_{1},\omega_{1}}\) consider
\begin{equation*}
\L=\set{R,c_{\alpha} \mid \alpha<\omega_{1}, d_{n} \mid n<\omega, X, Y}.
\end{equation*}
\begin{itemize}
\item \(X,Y\) are unary relations
\item \(R\) is the graph of a surjecting function \(X\to Y\), with axioms:
\begin{align*}
  \forall x,y,z\ \big(R(x,y) \land R(x,z)\big) &\implies (y=z);\\
  \forall x,y\ R(x,y) &\implies \big((x \in X) \land (y \in Y)\big);\\
  \forall x\ \bigg[X(x) &\iff \big(\exists y\ R(x,y)\big)\bigg];\\
  \forall y\ \bigg[Y(x) &\iff \big(\exists x\ R(x,y)\big)\bigg].
\end{align*}
\item we add two axioms
\begin{align*}
  \forall x\ \bigg(%
  X(x) &\iff \bigvee_{n \in \omega} (x=d_{n}) %
  \bigg);\\
  \forall y\ \bigg(%
  Y(y) &\iff \bigvee_{\alpha \in \omega_{1}} (y=c_{\alpha}) %
  \bigg);\\
\end{align*}
\item we add the last axiom
\begin{equation*}
  \bigwedge_{\substack{m,n \in\omega\\ m\neq n}} d_{n}\neq d_{m} \land \bigwedge_{\substack{\alpha,\beta \in\omega_{1}\\ \alpha\neq\beta}} c_{\alpha} \neq c_{\beta}.
\end{equation*}
\end{itemize}

So we have a \(R\) functional from \(X\) to \(Y\), total and surjective on \(Y\) s.t. \(X\) is countable and \(Y\) is uncountable. This theory can't have a Tarski-Model.
\label{esempio2VialePHD}
\end{esempio}

\textbf{From now on no function symbols in what is written, but reintroducing them is a standard exercise}.
\subsection{Tarski semantics for \(\L_{\kappa,\lambda}\)}
\label{sec:orge263167}

\begin{definizione}
Given a \(\lambda\)-signature \(\L = \set{R_{i} \mid i \in I, c_{k} \mid k \in K}\) (relational), \(\M\) is an \(\L\)-structure if
\begin{equation*}
\M=(M, R^{\M} \mid i \in I, c_{k}^{\M} \mid k \in K)
\end{equation*}
with
\begin{itemize}
\item \(R_{i}^{\M} \subseteq M^{\alpha_{i}}\) for all \(i \in I\) where \(\alpha_{i}\) is the arity of \(R_{i}\);
\item \(c_{k}^{\M} \in M\) for all \(k \in K\).
\end{itemize}
\end{definizione}

\(\M\) is a \uline{Tarski-model}.

We say that \(\M\vDash R_{i}(x_{j} \mid j \in U, c_{k} \in V)[m_{j}/x_{j} \mid j \in U]\) for \(m_{j} \in M\), where \(U\cup V = \alpha_{j}\) iff
\begin{equation*}
(m_{j} \mid j \in U, c_{k}^{\M} \mid k \in V) \in R_{i}^{\M}.
\end{equation*}
(we are discarding the order with this notation)

Given \(\varphi(x_{j} \mid j \in U)\) with displayed free variables
\begin{equation*}
\M \vDash \varphi(x_{j} \mid j \in U)[m_{j}/x_{j} \mid j \in U]
\end{equation*}
iff
\begin{itemize}
\item if \(\varphi\) is \(\lnot \psi\) and \(\M\not\vDash \psi(x_{j} \mid j \in U)[m_{j}/x_{j} \mid j \in U]\).
\item if \(\varphi\) is \(\exists V\ \psi\) and
\begin{equation*}
  \M\vDash \psi(x_{j} \mid j \in U, y_{\ell} \mid \ell \in V)%
  [m_{j}/x_{j} \mid j \in U, n_{\ell}/y_{\ell} \mid \ell \in V]
\end{equation*}
for some \(\set{n_{\ell} \mid \ell \in V} \subseteq M\).
\item if \(\varphi\) is
\begin{equation*}
  \bigwedge_{z \in Z} \psi_{z} (x_{j} \mid j \in U)
\end{equation*}
and for all \(z \in Z\) we have \(\M\vDash \psi_{z}(x_{j} \mid j \in U) [m_{j}/x_{j} \mid j \in U]\)
\item if \(\varphi\) is
\begin{equation*}
  \bigvee_{z \in Z} \psi_{z} (x_{j} \mid j \in U)
\end{equation*}
and for some \(z \in Z\) we have \(\M\vDash \psi_{z}(x_{j} \mid j \in U) [m_{j}/x_{j} \mid j \in U]\)
\end{itemize}

With this semantics
\begin{itemize}
\item the first example is in \(\L_{\omega_{1},\omega_{1}}\) and its Tarski models are linear orders of uncountable cofinality.
\item the second example is in \(\L_{\omega_{2},\omega}\) and has no Tarski models.
\end{itemize}
\subsection{Proof system}
\label{sec:org1f13686}

See page 9 of \autocite{santiagosuarezBooleanValuedSemantics2024}

\begin{thm}
(Soundness Theorem for Tarski Semantics).
The proof system is sound for Tarski Semantics.
\end{thm}

\uline{Fact}: the proof system is not complete for Tarski Semantics

\begin{esempio}
Recall the Example~\ref{esempio2VialePHD}. The conjunction of all the axioms is an \(\L_{\omega_{2},\omega}\)-sentence which is coherent for the proof system but Tarski-false. Hence its negation is Tarki-valid but not provable.
\end{esempio}

\begin{esempio}
\uline{Failure of completeness and interpolation for Tarski-semantics}.
Let
\begin{equation*}
\L_{0} = \set{\mathord{=}}, \quad %
\L_{1}= \set{\mathord{=}, c_{\alpha} \mid \alpha<\omega_{1}},\quad %
\L_{2} = \set{\mathord{=}, d_{n} \mid n <\omega}
\end{equation*}
and let
\begin{align*}
\psi_{1} &= \bigwedge_{\alpha<\beta<\omega_{1}} c_{\alpha} \neq c_{b}, &%
	\psi_{1} &\in \set{\L_{1}\text{-sentences}}\\
\psi_{2} &= \exists v\ \left(\bigwedge_{n \in \omega} v\neq c_{n}\right) \land \left(\bigwedge_{n<m<\omega} c_{n}\neq c_{m}\right) &%
	\psi_{2} &\in \set{\L_{2}\text{-sentences}}
\end{align*}
then \(\psi_{1}\mathrel{\vDash_{\text{TS}}}\psi_{2}\) but \(\psi_{1}\not\vdash \psi_{2}\).
\end{esempio}
\section{Lecture 2 - \textit{<2025-11-19 Wed>}}
\label{sec:org126a303}

A little recap/obs about last lecture.

\begin{oss}
\begin{enumerate}
\item \(\L_{\omega,\omega}\) is (equivalent to) first order logic.
\item \(\L_{\infty, \lambda} = \bigcup_{\kappa \in \Reg} \L_{\kappa, \lambda}\), where \(\Reg \subseteq \operatorname{Card}\) are the regular cardinals;
\(\L_{\infty,\infty} \bigcup_{\lambda \in \Reg} \L_{\infty, \lambda}\).

The \(\L_{\infty,\omega}\) is a ``good logic'', very tame with nice semantics, boolean valued, while \(\L_{\infty,\infty}\) is a ``bad logic''.
\end{enumerate}
\end{oss}

Lets write a proof of the following:
\begin{equation*}
\bigwedge_{\substack{i \in I\\ j \in J}} (\varphi_{i} \lor \psi_{j}) \vdash \bigg(\bigwedge_{i \in I} \varphi_{i}\bigg) \lor \bigg(\bigwedge_{j \in J} \psi_{j}\bigg).
\end{equation*}
which is the following, from the rules in \autocite{suarezBooleanCompactnessTheorem2025} in Section 3.2
\begin{prooftree}
\AxiomC{\(\varphi_{i} \vdash \varphi_{i}, \psi_{j}\)}
\AxiomC{\(\psi_{j} \vdash \varphi_{i}, \psi_{j}\)}
\AxiomC{\(\forall i \in I, j \in J\)}
\TrinaryInfC{%
	\(\varphi_{i}\lor\psi_{j} \vdash \varphi_{i}, \psi_{j},\qquad %
        \forall i \in I, j \in J \)}
\UnaryInfC{\(\bigwedge_{\substack{i \in I\\ j \in J}} (\varphi_{i}\lor\psi_{j}) \vdash \varphi_{i}, \psi_{j}, \qquad \forall i \in I, j \in J\)}
\UnaryInfC{\(\bigwedge_{\substack{i \in I\\ j \in J}} (\varphi_{i}\lor\psi_{j}) \vdash \varphi_{i}, \bigwedge_{j \in J} \psi_{j}, \qquad \forall i \in I\)}
\UnaryInfC{\(\bigwedge_{\substack{i \in I\\ j \in J}} (\varphi_{i}\lor\psi_{j}) \vdash \bigwedge_{i \in I}\varphi_{i}, \bigwedge_{j \in J} \psi_{j}\)}
\UnaryInfC{\(\bigwedge_{\substack{i \in I\\ j \in J}} (\varphi_{i}\lor\psi_{j}) \vdash \bigg(\bigwedge_{i \in I} \varphi_{i}\bigg) \lor \bigg(\bigwedge_{j \in J} \psi_{j}\bigg)\)}
\end{prooftree}


\begin{esempio}
If \((M,E)\vDash \mathrm{ZFC}\) as axiomatized in signature \(\set{\in}\), we have that:
\begin{enumerate}
\item \(\set{(\Gamma,\Delta) \in M^{2}\mid (M,E)\vDash\text{ there is a proof of }\Gamma\vdash\Delta}\) is a class which is uniformly \(\Delta_{1}\)-definable in \(M\) without parameters i.e. there are \(\Delta_{0}\)-formulae\footnote{All quantification is bounded} \(\theta_{0}(x,y,z)\) and \(\theta_{1}(x,y,w)\) s.t.
\begin{equation*}
 \mathrm{ZFC} \vdash
\end{equation*}
\end{enumerate}
{[}\ldots{}]
\end{esempio}

Completeness for the proof system fine for \(\omega\)-signatures. (first order signatures). But this is not important, since \(\L_{\infty,\omega}\) is the \uline{good} infinitary logic.
\subsection{Failure of interpolation for Tarski semantics on \(\L_{\infty,\omega}\) and \(\L_{\infty,\infty}\).}
\label{sec:org5d0040e}

In PHD THESIS (to cite):
\begin{enumerate}
\item There are \(\psi_{0},\psi_{1}\), \(\psi_{0} \in \L_{\omega_{2},\omega}\) and \(\psi_{1} \in \L_{\omega_{1},\omega}\) s.t. \(\psi_{0}\implies \psi_{1}\) is Tarski valid, but for no \(\theta\) in \(\L_{\infty,\omega}\), for \(\set{=} = \L(\psi_{0})\cap \L(\psi_{1})\) \(\psi_{0}\implies \theta\) and \(\theta\implies \psi_{1}\) are both Tarski-valid
\item for any \(\psi_{0}\implies \psi_{1}\) valod for Tarski semantics on \(\L_{\infty,\omega}\) there is a \(\theta\) in \(\L_{\infty,\infty}\) such that
\begin{equation*}
 \psi_{0}\implies \theta,\quad \theta\implies\psi_{1}
\end{equation*}
are tarski valid, and \(\theta\) is in \(\L(\psi_{0})\cap \L(\psi_{1})\).
\item \(\L_{\infty,\infty}\) interpolation fails. There is \(\varphi\implies\psi\) in \(\L_{\omega_{1},\omega_{1}}\) which is Tarski valid but has not interpolant \(\theta\) which is in \(\L_{\infty,\infty}\) which respect to Tarski semantics.
\end{enumerate}

\uline{Counterexample to interpolation for TS in \(\L_{\omega_{2},\omega}\)}:
Let \(\L_{0} = \set{c_{\alpha} \mid \alpha<\omega_{1}}\) and \(\L_{1} = \set{d_{n} \mid n \in \omega}\).
\begin{align*}
\psi_{0} &= \bigwedge_{\alpha<\beta<\omega_{1}}(c_{\alpha} \neq c_{\beta})\\
\psi_{1} &= \exists v\ \bigg(\bigwedge_{n<\omega} d_{n}\neq v \bigg)
\end{align*}
\(\psi_{0}\implies \psi_{1}\) is Tarski valid. In fact, if \(\M\) is a \(\L_{0}\cup\L_{1}\) structure s.t. \(\M\vDash \psi_{0}\), then \(M\) (the domain of \(\M\)) is uncountable. Then \(D=\set{d_{n}^{\M} \mid n \in \omega} \neq M\), so there is \(u \in M\) s.t. \(u \notin D\). \(u\) witness \(\M\vDash\psi_{1}\).

In \(\L = \set{=}\), structures are categorical in each infinite cardinality (obv). Then for \(\M\) an \(\L\)-structure of infinite cardinality, \(\operatorname{Th}(\M)\) is model-complete, in the sense that \(\M_{0} \sqsubseteq \M_{1}\) models of \(\operatorname{Th}(\M)\) are \(\M_{0} \preceq \M_{1}\).

Now, by way of contradiction, assume \(\theta\) is an \(\L\)-sentence for \(\L=\set{=}\) s.t. \(\psi_{0}\implies \theta\) and \(\theta\implies \psi_{1}\) are both Tarski-valid. Then
\begin{align*}
\lnot\psi_{1}\implies\lnot \theta,&\qquad%
\lnot\theta\implies\lnot\psi_{0}.\\
\lnot\psi_{0} &\equiv\bigvee_{\alpha<\beta<\omega_{1}} c_{\alpha}=c_{\beta}\\
\lnot\psi_{1} &\equiv\forall v \big(\bigvee_{n \in \omega} d_{n} = v\big)
\end{align*}
\(\lnot\psi_{1}\) is true only in a countable structure. On the other hand \(\lnot\psi_{0}\) can be true in structures of any infinite size.

In particular \(\lnot\theta\) is true in structures for \(\set{=}\) of any infinite cardinality, so it is true in all structure of infinity cardinality (by the Lemma~\ref{prossimolemma_vialelll}). Then \(\lnot\psi_{0}\) is true in an uncountable structure. Absurd.

\begin{lem}
Let \(A,B\) some infinite \(\set{=}\)-structures s.t. \(A \sqsubseteq B\), \(a_{1},\dots,a_{n} \in A\) and \(\sigma \in \L_{\infty,\omega}\),
\begin{equation*}
A\vDash\sigma(a_{1},\dots,a_{n}) \iff B\vDash \sigma(a_{1},\dots,a_{n}).
\end{equation*}
\label{prossimolemma_vialelll}
\end{lem}
\begin{proof}
Let \(B\vDash\exists x\ \psi(x,a_{1},\dots,a_{n})\), and let \(C=A\cup \set{c}\) s.t. \(B\vDash\psi(c,a_{1},\dots,a_{n})\).

Then \(C \sqsubseteq B\), \(C\vDash\psi(c,a_{1},\dots,a_{n})\) and \(C\cong A\) via an infinity morph \(f\) that fized \(a_{1},\dots,a_{n}\), so
\begin{equation*}
A\vDash \psi(f(c),a_{1},\dots,a_{n}).\qedhere
\end{equation*}
\end{proof}
\subsection{Boolean valid semantics}
\label{sec:orgf521a2c}

\begin{definizione}
Let \(\L\) be a relational \(\lambda\)-signature. \(\M\) is a \(B\)-valued model for
\(\L=\set{R_{i}\mid i \in I} \cup\set{c_{k}\mid k \in K}\) for \(B\) a boolean algebra if the following holds:
\begin{equation*}
\M=(M, R_{i}^{\M}, c_{k}^{\M})
\end{equation*}
\begin{enumerate}
\item There is a function
\begin{align*}
=^{\M}: M^{2} &\longrightarrow B\\
(\tau,\sigma) &\longmapsto \eval{\tau=\sigma}
\end{align*}
s.t.
\begin{align*}
 \eval{\tau=\tau} &= 1_{B}\\
 \eval{\tau=\sigma} &= \eval{\sigma=\tau}\\
 \eval{\tau=\sigma} \land \eval{\sigma = \nu} &\le \eval{\tau=\nu}.
\end{align*}
\item For each \(i \in I\), given \(n_{i}\) the arity of \(R_{i}\), there is a function
\begin{align*}
R_{i}^{\M}: M^{n_{i}} &\longrightarrow B\\
(\sigma_{1},\dots,\sigma_{n_{i}}) &\longmapsto \eval{R_{i}(\sigma_{1},\dots,\sigma_{n_{i}})}
\end{align*}
\end{enumerate}
\end{definizione}
\section{Lecture 3 - \textit{<2025-11-24 Mon>}}
\label{sec:orgea86b7f}

Last lecture:
\begin{enumerate}
\item Proof that interpolation fail for \(\L_{\infty,\omega}\).
\item Failure of completeness for Tarski semantics with regard to the Proof System in \autocite{suarezBooleanCompactnessTheorem2025}.
\item \(\M \mathrel{\vDash_{\L_{\infty,\omega}}^{\text{TS}}} \varphi\) is \(\Delta_{1}(\mathrm{ZFC})\).
\item \(\M \mathrel{\vDash_{\L_{\infty,\infty}}^{\text{TS}}} \varphi\) is at least \(\Pi_{1}(\mathrm{ZFC})\)

Note that 3. and 4. say that the concept of provability in \(\L_{\infty,\omega}\) is ``good'', since its absolute, while in \(\L_{\infty,\infty}\) is not.
\end{enumerate}

In this lecture we will show a sentence \(\psi\) and a structure \(\M\) s.t. \(\M\vDash\psi\) holds in \(V\) but not in \(V[G]\).

\begin{esempio}
We built a bijection between \(\omega\) and \(\omega_{1}\):
\begin{equation*}
\L=\set{R,X,Y} \cup \set{c_{\alpha} \mid \alpha < \omega_{1}^{V}} \cup \set{d_{n} \mid n \in \omega}
\end{equation*}
where \(R\) is binary and \(X,Y\) are unary.
\begin{itemize}
\item \(\psi\) is given by the conjunction of:
\begin{enumerate}
\item \(R:X\to Y\) is the graph of a bijection (first order sentence);
\item \(\displaystyle \forall x\ \bigg(X(x) \iff \bigwedge_{\alpha<\omega_{1}^{V}} x = c_{\alpha}\bigg) \land \bigwedge_{\alpha<\beta<\omega_{1}^{V}} (c_{\alpha} \neq c_{\beta})\);
\item \(\forall y\ \bigg(Y(y) \iff \bigwedge_{n \in \omega} y = d_{n}\bigg) \land \bigwedge_{n<m<\omega} (d_{n}\neq d_{m})\).
\end{enumerate}
Then \(\psi \mathrel{\vDash^{\text{TS}}} \bot\).
\end{itemize}

We want to show that there is no proof in \(V\) of \(\psi\vdash\bot\).

Assume ``\(P\) is a proof of \(\psi\vdash\bot\)'', \(P \in V\). But that is \(\Delta_{1}(\text{ZFC})\) in parameters \(P,\psi,\bot\).
Let \(G\) be \(V\)-generic for \(\operatorname{Coll}(\omega,\omega_{1}^{V})\), then
\begin{equation*}
V[G]\vDash\text{``}P\text{ is a proof of }\psi\vdash\bot\text{''}
\end{equation*}
but in
\begin{equation*}
V[G]\vDash \exists f\ \big[(f:\omega_{1}^{V}\to \omega)\text{ is a bijection}\big].
\end{equation*}
Then we can consider
\begin{equation*}
\M = (f,\omega_{1}^{V}, \omega, c_{\alpha}\mapsto\alpha, d_{n}\mapsto n)
\end{equation*}
and \(M\vDash\psi\). Given ZFC consistency, we have that there is noproof \(P\) of \(\psi\vdash\bot\) in \(V\).
\end{esempio}

\begin{oss}
Why \(\M \mathrel{\vDash_{\L_{\infty,\omega}}^{\text{TS}}} \varphi\) is \(\Delta_{1}\)? Because we need only to consider the structure \(\M^{<\omega}\) to understand what is true in \(\M\), where \(M\) is its domain.

In fact, if
\begin{equation}
\M\vDash \exists (x_{1},\dots,x_{n})\ \psi(x_{1},\dots,x_{n}) %
\label{jdsknhbvsafiudsjkfhieudn}
\end{equation}
the we know that for every \(a_{1},\dots,a_{n} \in M\)
\begin{equation*}
\M\vDash \psi(x_{1},\dots,x_{n})[a_{i}/x_{i} : i = 1,\dots,n]
\end{equation*}
is simple i.e. \(\Delta_{1}\), so \eqref{jdsknhbvsafiudsjkfhieudn} iff:
\begin{equation*}
V\vDash \exists  s \in M^{<\omega}\ \M\vDash \psi(x_{1},\dots,x_{n})[s_{i}/x_{i} \mid i<n]
\end{equation*}
simple in parameters \(M\) and \(\omega\).
\end{oss}
\begin{oss}
But if we consider:
\begin{equation*}
M\mathrel{\vDash_{\L_{\infty,\infty}}^{\text{TS}}} \exists (x_{n} \mid n<\omega)\ \psi(x_{n} \mid n<\omega)
\end{equation*}
which is equivalent to
\begin{equation*}
V\vDash \exists  f \in M^{\omega}\ \M\vDash\psi(x_{n} \mid n<\omega)[f(i)/x_{i} \mid i<\omega]
\end{equation*}
and so we have to know \(M^{\omega}\) to check whether \(\exists (x_{n} \mid n<\omega)\ \psi(x_{n} \mid n<\omega)\) is true  when I have to check it for \(\psi(x_{n}\mid n<\omega)\).
\end{oss}
\begin{esempio}
Let \(\L = \set{<}\) binary relation, and let \(\psi\) be the conjuction of
\begin{itemize}
\item Axioms of linear order
\item \(\forall (x_{n} \mid n <\omega) \ \exists y \bigg[\bigwedge_{n \in \omega} (x_{n}<y)\bigg]\).
\end{itemize}

In \(V\) we have that \(\langle \omega_{1}^{V},\in \rangle \vDash\psi\).
\end{esempio}

\uline{Quantifier elimination axiom}: given \(C=\set{c_{i}\mid i \in I}\) a set of constant in a signature \(\L\):
\begin{equation*}
\mathrm{QE}_{C} \coloneqq \qquad \forall x\ \bigg[\bigvee_{i \in I}(x=c_{i})\bigg]
\end{equation*}
and if \(\M\vDash \mathrm{QE}_{C}\) then, given \(M\) domain of \(\M\), we have \(\card{M}\le \card{I}\).

Inductively, one maps \(\psi\mapsto \psi_{c}\), by
\begin{align*}
\bigg(\bigwedge_{j \in K} \psi_{j}\bigg)_{C} &= \bigwedge_{j \in J} \big((\psi_{j})_{C}\big)\\
\bigg(\exists x\ \psi(x)\bigg)_{C} &= \bigwedge_{i \in I} \big(\psi(c_{i})\big)_{C}\\
(\lnot \psi)_{C} &= \lnot(\psi)_{C}\\
&\vdots
\end{align*}
\uline{Fact}: \(\psi \mathrel{\equiv_{\mathrm{QE}_{C}}^{\text{TS}}} \psi_{C}\) for \(\L_{\infty,\omega}\).

One can generalize this concept on \(\psi \mathrel{\equiv_{\mathrm{QE}_{C}}^{\text{TS}}} \psi_{C}\) for \(\L_{\infty,\infty}\).

\begin{esempio}
We give an example of a \(\psi \mathrel{\equiv_{\mathrm{QE}_{C}}} \varphi\) in \(V\) but not in some \(V[G]\). Given \(C=\set{c_{\alpha} \mid \alpha<\omega_{1}}\):
\begin{align*}
\psi: \qquad &\forall (x_{n}\mid n\in \omega) \ \exists y\ %
	 \bigg[\bigwedge_{n<\omega} (x_{n}<y)\bigg]\\
\varphi: \qquad &\bigwedge_{f \in (C^{\omega})^{V}} \bigvee_{\alpha<\omega_{1}^{V}} %
	\bigg[\bigwedge_{n<\omega} \big(f(n)<c_{\alpha}\big)\bigg]
\end{align*}
Then \(V\vDash \big(\psi \mathrel{\equiv_{\mathrm{QE}_{C}}^{\text{TS}}} \varphi\big)\).

If then \(G\) is  \(V\)-generic for \(\operatorname{Coll}(\omega,\omega_{1})\) then
\begin{align*}
\langle \omega_{1}^{V}, \in, c_{\alpha}\mapsto \alpha : \alpha<\omega_{1}^{V} \rangle &\vDash \psi\qquad \text{holds in \(V\)}\\
\langle \omega_{1}^{V}, \in, c_{\alpha}\mapsto \alpha : \alpha<\omega_{1}^{V} \rangle &\vDash \varphi\qquad \text{holds in \(V\)}\\
\langle \omega_{1}^{V}, \in, c_{\alpha}\mapsto \alpha : \alpha<\omega_{1}^{V} \rangle &\not\vDash \psi\qquad \text{holds in \(V[G]\)}\\
\langle \omega_{1}^{V}, \in, c_{\alpha}\mapsto \alpha : \alpha<\omega_{1}^{V} \rangle &\vDash \varphi \qquad \text{holds in \(V[G]\)}.
\end{align*}
\end{esempio}

Modulo \(\mathrm{QE}_{C}\), \(\L_{\infty,\infty}\) is the same as second order logic, at least for Tarski semantics.

In fact, if \(C=\set{c_{i} \mid i \in I}\), it is true in \(\L_{\infty, \card{I}^{+}}\). Given a typical second order formula:
\begin{equation*}
\forall  R\ \parentesi{\text{first order}}{R(x) \land \psi(x,y,z,\dots)}
\end{equation*}
we can write in \(\L_{\infty, \card{I}^{I}}\):
\begin{equation*}
\forall (x_{i} \mid i \in I) \ \bigwedge_{i \in I} R(x_{i}) \land \psi(x_{i}, y,z,\dots)
\end{equation*}
\textbf{TO BE ELABORATED NEXT LECTURE}.

\uline{Failure of compactness for \(\L_{\omega_{1},\omega}\) for Tarski semantics.}

Let
\begin{equation*}
T = \set{\bigwedge_{n \in \omega} c_{n} = c_{\omega}} \cup \set{c_{n} \neq c_{m} \mid n<m\le \omega}.
\end{equation*}
If \(T_{0} \subseteq T\) is finite, let \(K\) be s.t. all constant \(c_{i} \in \psi \in T_{0}\) are with \(i<K\) or \(i=\omega\). Then \(T_{0}\) holds in
\begin{equation*}
(K+1, c_{i}\mapsto i, c_{\omega}\mapsto k) \vDash T_{0}
\end{equation*}
while no \(\M\vDash T\).
\subsection{Boolean valued semantics}
\label{sec:orgca852a2}

\begin{definizione}
Let \(\L\) be a relational first order signature, \(\L=\set{R_{i}\mid i \in I} \cup\set{c_{k}\mid k \in K}\), and let \(B\) be a boolean algebra.

\(\M =(M, R_{i}^{\M}, c_{k}^{\M})\) is a \(B\)-valued model for
\(\L\) if the following holds:
\begin{enumerate}
\item There is a function
\begin{align*}
=^{\M}: M^{2} &\longrightarrow B\\
(\tau,\sigma) &\longmapsto \eval{\tau=\sigma}
\end{align*}
s.t.
\begin{align*}
 \eval{\tau=\tau} &= 1_{B}\\
 \eval{\tau=\sigma} &= \eval{\sigma=\tau}\\
 \eval{\tau=\sigma} \land \eval{\sigma = \eta} &\le \eval{\tau=\eta}.
\end{align*}
\item For each \(i \in I\), given \(n_{i}\) the arity of \(R_{i}\), \(R_{i}^{\M}\) is a function
\begin{align*}
R_{i}^{\M}: M^{n_{i}} &\longrightarrow B\\
(\sigma_{1},\dots,\sigma_{n_{i}}) &\longmapsto \eval{R_{i}(\sigma_{1},\dots,\sigma_{n_{i}})}
\end{align*}
s.t.
\begin{equation*}
 \eval{R_{i}(\tau_{1},\dotsm\tau_{n_{i}})} \land \bigwedge_{j=1}^{n_{i}} \eval{\tau_{j} = \sigma_{j}} \le \eval{R_{i}(\sigma_{1},\dots,\sigma_{n_{i}})}.
\end{equation*}

\item For each \(k \in K\), \(c_{k}^{\M} \in M\).
\end{enumerate}
\end{definizione}

If \(B=2 = \set{0,1}\), then the \(B\)-valued models are exactly the Tarski models. The rules are given by the equality axioms in Section 3.2 in \autocite{suarezBooleanCompactnessTheorem2025}.

\begin{definizione}
Let \(\M\) be a \(B\)-valued model and assume \(B\) is a complete boolean algebra (otherwise in what follows replace \(B\) by \(\operatorname{RO}(B^{+})\)).
\begin{quote}
Definition~3.4 of \autocite{suarezBooleanCompactnessTheorem2025}
\end{quote}
\end{definizione}
\section{Lecture 4 - \textit{<2025-11-26 Wed>}}
\label{sec:orgd38799f}


\def\defaultbooleanmodel{\parti{J}}
\def\defaultmodel{\NN}
\begin{esempio}
Let \(\L=\set{R_{i} \mid i \in I} \cup \set{c_{k} \mid k \in K}\)
be a \(\omega\)-signature, where \(R_{i}\) are \(n_{i}\)-ary relation symbols and \(c_{k}\) are constant symbols.

For each \(j \in J\) let
\begin{equation*}
\M_{j} = (M_{j}, R_{i}^{\M_{j}}, c_{k}^{\M_{j}})
\end{equation*}
be a Tarski \(\L\)-structure. Consider the \(\parti{J}\)-structure \(\NN\):
\begin{equation*}
\NN = \bigg(%
\prod_{j \in J} M_{j}, R_{i}^{\NN}, c_{k}^{\NN}
\bigg)
\end{equation*}
\begin{itemize}
\item for each \(k \in K\): \(c_{k}^{\N} = (c_{k}^{\M_{j}} \mid j \in J)\):
\begin{equation*}
  c_{k}^{\NN} : j \mapsto c_{k}^{\M_{j}}
\end{equation*}
\item for each \(i \in I\):
\begin{equation*}
  \eval{R_{i}^{\NN} (f_{1},\dots,f_{n_{i}})} = \set{j \in J \mid
  R_{i}^{\M_{j}} (f_{1},\dots,f_{n_{i}})
    }
\end{equation*}
\end{itemize}

{[}\ldots{}]: prove that this is in fact a \(\parti{J}\)-structure. This is the standard construction of the ultraproduct (before the equivalent relation).
\label{esempio:4.1}
\end{esempio}
\def\defaultbooleanmodel{\M}
\def\defaultmodel{\operatorname{Meas}}
\begin{esempio}
Let's consider
\begin{equation*}
L^{\infty}(\R) = \set{ f:\R\to \R \mid %
	f\text{ Lebesgue measurable}, \operatorname{supess}|f| < \infty %
}
\end{equation*}
where
\(\operatorname{supess}|f| = \sup \set{r \in \R \mid \mu(|f|>r) > 0}\).

The unit ball of \(L^{\infty}(\R)\) is
\begin{equation*}
B_{1}(L^{\infty}(\R)) \coloneqq %
\set{f \in L^{\infty}(\R) \mid \mu\big(\set{x \mid |f(x)| > 1}\big) = 0}
\end{equation*}
Let \(\M = B_{1}(L^{\infty}(\R))\) be a \(\operatorname{Meas}\)-model in a language \(\L = \set{+, \cdot, =}\), where \(+,\cdot\) are ternary relation symbols identifying the graphs of those function, and where
\begin{equation*}
\operatorname{Meas} = \set{X \subseteq \R \mid X\text{ is Lebesgue Measurable}}.
\end{equation*}
The evaluations are:
\begin{itemize}
\item \(\eval{f=g} = \set{x \in \R \mid f(x) = g(x)}\);
\item \(\eval{f+g=h} = \set{x \in \R\mid f(x) + g(x) = h(x)}\);
\item \(\eval{f\cdot g=h} = \set{x \in \R \mid f(x) \cdot g(x) = h(x)}\);
\end{itemize}
and this is a boolean model.

\def\defaultmodel{\operatorname{MALG}}

We can now consider the boolean algebra
\begin{equation*}
\operatorname{MALG} \coloneqq \operatorname{Meas} /\operatorname{Null} %
= \set{[A]_{\operatorname{Null}} \mid A \in \operatorname{Meas}},\qquad %
[A]=[B] \IFF \mu(A \mathrel{\triangle} B) = 0
\end{equation*}
Is \(L^{\infty}(\R)\) a \(\operatorname{MALG}\)-model? We could use the evaluations
\begin{itemize}
\item \(\eval{f=g} = \big[\set{x \in \R \mid f(x) = g(x)}\big]_{\operatorname{Null}}\);
\item \(\eval{f+g=h} = \big[\set{x \in \R\mid f(x) + g(x) = h(x)}\big]_{\operatorname{Null}}\);
\item \(\eval{f\cdot g=h} = \big[\set{x \in \R \mid f(x) \cdot g(x) = h(x)}\big]_{\operatorname{Null}}\).
\end{itemize}
\end{esempio}
\def\defaultbooleanmodel{\M}
\def\defaultmodel{B}

\begin{oss}
For any info about boolean algebras, see \autocite{vialeForcingMethodSet2024}, Chapters 3 and 4.
\end{oss}

\begin{definizione}
Let \(B\) a Boolean Algebra and \(F\) a filter on \(B\) (or dually \(I = \tilde{F} = \set{\lnot a \mid a \in F}\)):
\begin{equation*}
B/F \coloneqq \set{[a]_{F} \mid a \in B}, \qquad%
[a]_{F} = [b]_{F} \IFF (a \mathrel{\triangle} b) \in I
\end{equation*}
\end{definizione}

\(B/F\) is a boolean algebra with operations
\begin{align*}
[a]_{F} \lor [b]_{F} &= [a \lor b]_{F}\\
[a]_{F} \land [b]_{F} &= [a \land b]_{F}\\
\lnot[a]_{F}  &= [\lnot a]_{F}
\end{align*}

\begin{definizione}
Let \(\M\) be a \(B\)-valued model for \(\L\). We define the \(B/F\)-structure \(\M / F\) as follows: if
\begin{equation*}
\M = (M, R_{i}^{\M}, c_{k}^{\M})
\end{equation*}
then
\begin{equation*}
\M/F = (M/F, R_{i}^{\M/F}, c_{k}^{\M/F})
\end{equation*}
\begin{itemize}
\item \(M/F \coloneqq \set{[\tau]_{F} \mid \tau \in M}\) where
\begin{equation*}
  [\tau]_{F} \coloneqq \set{\sigma \mid \eval{\sigma=\tau} \in F}
\end{equation*}
\item we obviously have that
\def\defaultbooleanmodel{\M/F}
\def\defaultmodel{B/F}
\begin{equation*}
  \eval{[\tau]_{F} = [\tau]_{F}} = [1_{B}]_{F}
\end{equation*}
\item copy for relations, and well poseness
\end{itemize}
See Definition~3.5 of \autocite{suarezBooleanCompactnessTheorem2025}.
\end{definizione}
\def\defaultbooleanmodel{\M}
\def\defaultmodel{B}

\uline{Fact} if \(G\) is a ultrafilter of the boolean algebra \(B\), and \(\M\) is a \(B\)-valued model for \(\L\), we have that \(\M/G\) is a \(B/G\)-valued model.

But \(B/G \cong 2\) (since \(G\) is a ultrafilter), so \(\M/G\) is a Tarski model.

\begin{esempio}
Let \(\set{\M_{j} \mid j \in J}\) be \(\L\)-Tarski-structures, for \(\L=\set{R_{i} \mid i \in I} \cup \set{c_{k} \mid k \in K}\)
an and \(c_{k}\) are constant symbols, and
consider the \(\parti{J}\)-structure \(\NN\):
\begin{equation*}
\NN = \bigg(%
\prod_{j \in J} M_{j}, R_{i}^{\NN}, c_{k}^{\NN}
\bigg)
\end{equation*}
as in the Example~\ref{esempio:4.1}.

Let \(G\) be an ultrafilter of \(\parti{J}\), i.e. \(G\) is a ultrafilter on \(J\). Then \(\NN/G\) is a Tarski structure, and it is called the ultraproduct of \(\set{\M_{j} \mid j \in J}\) by \(G\).
\end{esempio}
\begin{thm}
(Los Theorem)
\(\NN/G \vDash \varphi(f_{1},..,f_{n})\) iff \(\set{j \in J \mid \M_{j} \vDash \varphi(f_{1}(j),\dots,f_{n}(j))} \in G\).
\end{thm}
\uline{Exercise}: for all \(\varphi(x_{1},\dots,x_{n})\) and \(f_{1},\dots,f_{n} \in \NN\)
\begin{equation*}
\eval{\varphi(f_{1},\dots,f_{n})} = \set{j \in J \mid \M_{j} \vDash \varphi(f_{1}(j),\dots,f_{n}(j))}
\end{equation*}

\begin{definizione}
Given a logic \(\L_{\star\star}\) among the \(\L_{\kappa,\lambda}\), we say that a \(B\)-valued model \(\M\) for \(\L\) is a \(\L_{\star\star}\)-fill if for \(\L_{\star\star}\)-formulae (in display free variables)
\begin{equation*}
\varphi(x_{i} \mid i \in I, y_{k} \mid k \in K)
\end{equation*}
we have that
\begin{equation*}
\eval{%
\exists (x_{i} \mid i \in I)\ \varphi (x_{i} \mid i \in I, y_{k} \mid k \in K)
} = ???
\end{equation*}
\end{definizione}

\uline{Fact}: The Los Theorem for \(\L_{\infty,\infty}\) fails for at least one theory \(T\). (De Bondt)
\begin{thm}
(Folklore; V. \& Pierobon; V. \& Santiago).
For any \(\L_{\infty,\omega}\) theory \(T\) TFAE:
\begin{enumerate}
\item \(T\) is Boolean consistent (so there is a \(B\)-valued model \(\M\) s.t. \(\eval{\bigwedge T}\));
\item \(T\) is provably consistent (for the proof system given);
\item \(T\) has a Full model for \(\L_{\infty, \omega}\).
\end{enumerate}
\end{thm}

\autocite{mansfieldCompletenessTheoremInfinitary1972a} proves \(1.\Leftrightarrow 2.\) for \(\L_{\infty,\infty}\).

\begin{thm}
Assume \(\M\) is full for \(\L_{\omega,\omega}\). Then for all \(\L_{\omega,\omega}\)-formula \(\varphi(x_{i} \mid i \in I)\) and \((\sigma_{i} \in i \in I) \in M^{I}\) and \(G\) ultrafilter on \(B\)
\begin{equation*}
\M / G \vDash \varphi\big([\sigma_{i}]_{G} \mid i \in I\big) %
\IFF%
\eval{\varphi(\sigma_{i} \mid i \in I)} \in G
\end{equation*}
\end{thm}

\def\defaultbooleanmodel{\operatorname{MALG}}
\def\defaultmodel{\M}
\uline{Counterexample to fullness}: take
\(C^{\omega}(\R) = \set{f: \R\to \R \mid f\text{ is analytic}}\). We make \(\M =C^{\omega}(\R)\) a \(\operatorname{MALG}\)-valued model for \(\L=\set{C,<}\), with \(C\) unary and \(<\) binary:
\begin{align*}
\eval{f=g} &= \big[\set{x \in \R \mid f(x) = g(x)}\big]\\
\eval{C(f)} &= \begin{cases}
[\R], & f \text{ is constant}\\
[\emptyset], & f \text{ is not constant}
\end{cases}\\
\eval{f<g} &= \big[\set{x \in \R \mid f(x) < g(x)}\big]-
\end{align*}
\(\M\) is not a full model for first order logic.
\section{Lecture 5 - \textit{<2025-12-01 Mon>}}
\label{sec:org0f8cb62}

Non ho preso appunti
\newpage
\printbibliography
\end{document}
