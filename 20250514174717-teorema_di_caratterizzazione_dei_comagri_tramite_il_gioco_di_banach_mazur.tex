% Created 2026-02-07 Sat 19:32
% Intended LaTeX compiler: pdflatex
\documentclass[10pt]{article}
%% CREATO CON ORG - EMACS
\newcommand{\use}[2][]{\usepackage[#1]{#2}}
% PACCHETTI FONDAMENTLAI
\use[utf8]{inputenc}
\use[T1]{fontenc}
\use{graphicx}
\use{longtable}
\use{wrapfig}
\use{rotating}
\use[normalem]{ulem}
\use{amsmath}
\use{amsthm}
\use{amssymb}

\use{eucal} % Cambia mathcal{...}

\use{capt-of}
\use[italian]{babel}
\use[babel]{csquotes}
% bib la TEX lo carica in automatico org-cite
\use{microtype}
\use{lmodern}
\use{subfig} % sottofigure
\use{multicol} % due colonne
\use{lipsum} % lorem ipsum
\use{color} % colori in latex
\use{parskip} % rimuove l'indentazione dei nuovi paragrafi %% Add parbox=false to all new tcolorbox
\use{centernot}
\use[outline]{contour}\contourlength{3pt}
\use{fancyhdr}
\use{layout}
\use[most]{tcolorbox} % Riquadri colorati
\use{ifthen} % IFTHEN
\use{geometry}

% pacchetti matematica
\use{yhmath}
\use{dsfont}
\use{mathrsfs}
\use{cancel} % semplificare
\use{polynom} %divisione tra polinomi
\use{forest} % grafi ad albero
\use{booktabs} % tabelle
\use{commath} %simboli e differenziali
\use{bm} %bold
\use[fulladjust]{marginnote} %to use marginnote for date notes
\use{arrayjobx}%array
\use[intlimits]{empheq} % Riquadri colorati attorno alle equazioni
\use{mathtools}
\use{circuitikz} % Disegnare i circuiti
\use{mathtools}
\use{stmaryrd} % [[ \llbracket ]] \rrbracket
\use{bussproofs} % dimostrazioni

%%%%%%%%%%%%%


%%%% QUIVER
\newcommand{\duepunti}{\,\mathchar\numexpr"6000+`:\relax\,}
% A TikZ style for curved arrows of a fixed height, due to AndréC.
\tikzset{curve/.style={settings={#1},to path={(\tikztostart)
    .. controls ($(\tikztostart)!\pv{pos}!(\tikztotarget)!\pv{height}!270:(\tikztotarget)$)
    and ($(\tikztostart)!1-\pv{pos}!(\tikztotarget)!\pv{height}!270:(\tikztotarget)$)
    .. (\tikztotarget)\tikztonodes}},
    settings/.code={\tikzset{quiver/.cd,#1}
        \def\pv##1{\pgfkeysvalueof{/tikz/quiver/##1}}},
    quiver/.cd,pos/.initial=0.35,height/.initial=0}

% TikZ arrowhead/tail styles.
\tikzset{tail reversed/.code={\pgfsetarrowsstart{tikzcd to}}}
\tikzset{2tail/.code={\pgfsetarrowsstart{Implies[reversed]}}}
\tikzset{2tail reversed/.code={\pgfsetarrowsstart{Implies}}}
% TikZ arrow styles.
\tikzset{no body/.style={/tikz/dash pattern=on 0 off 1mm}}
%%%%%%%%%%


%% DEFINIZIONI COMANDI MATEMATICI
\let\sin\relax %TOGLIE LA DEFINIZIONE SU "\sin"

% cambia la definizione di empty set
% ---
\let\oldemptyset\emptyset
% ---
% \let\emptyset\varnothing
% ---
% \let\emptyset\relax
% \newcommand{\emptyset}{\text{\textnormal{\O}}}
% ---

\DeclareMathOperator{\bounded}{bd}
\DeclareMathOperator{\sin}{sen}
\DeclareMathOperator{\epi}{Epi}
\DeclareMathOperator{\cl}{cl}
\DeclareMathOperator{\graph}{graph}
\DeclareMathOperator{\arcsec}{arcsec}
\DeclareMathOperator{\arccot}{arccot}
\DeclareMathOperator{\arccsc}{arccsc}
\DeclareMathOperator{\spettro}{Spettro}
\DeclareMathOperator{\nulls}{nullspace}
\DeclareMathOperator{\dom}{dom}
\DeclareMathOperator{\ar}{ar}
\DeclareMathOperator{\const}{Const}
\DeclareMathOperator{\fun}{Fun}
\DeclareMathOperator{\rel}{Rel}
\DeclareMathOperator{\altezza}{ht}
\let\det\relax %TOGLIE LA DEFINIZIONE SU "\det"
\DeclareMathOperator{\det}{det}
\DeclareMathOperator{\End}{End}
\DeclareMathOperator{\gl}{GL}
\def\Id{\mathrm{Id}}
\def\id{\mathrm{id}}
\DeclareMathOperator{\I}{\mathds{1}}
\DeclareMathOperator{\II}{II}
\DeclareMathOperator{\rank}{rank}
\DeclareMathOperator{\tr}{tr}
\DeclareMathOperator{\tc}{t.c.}
\DeclareMathOperator{\T}{T}
\DeclareMathOperator{\var}{Var}
\DeclareMathOperator{\cov}{Cov}
\DeclareMathOperator{\st}{st}
\DeclareMathOperator{\mon}{Mon}
\newcommand{\card}[1]{\left\vert #1 \right\vert}
\newcommand{\trasposta}[1]{\prescript{\text{T}}{}{#1}}
\newcommand{\1}{\mathds{1}}
\newcommand{\R}{\mathds{R}}
\newcommand{\diesis}{\#}
\newcommand{\bemolle}{\flat}
\newcommand{\nonstandard}[1]{\prescript{*}{}{#1}}
\newcommand{\starR}{\nonstandard{\R}}
\newcommand{\borel}{\mathscr{B}}
\newcommand{\lebesgue}[1]{\mathscr{L}\left(#1\right)}
\newcommand{\media}{\mathds{E}}
\newcommand{\K}{\mathds{K}}
\newcommand{\A}{\mathds{A}}
\newcommand{\Q}{\mathds{Q}}
\newcommand{\N}{\mathds{N}}
\newcommand{\C}{\mathds{C}}
\newcommand{\Z}{\mathds{Z}}
\newcommand{\qo}{\hspace{1em}\text{q.o.}\,}
\renewcommand{\tilde}[1]{\widetilde{#1}}
\renewcommand{\parallel}{\mathrel{/\mkern-5mu/}}
\newcommand{\parti}[2][]{\wp_{#1}(#2)}
\newcommand{\diff}[1]{\operatorname{d}_{#1}}
\let\oldvec\vec
\renewcommand{\vec}[1]{\overrightarrow{\vphantom{i}#1}}
\newcommand{\floor}[1]{\left\lfloor #1 \right\rfloor}
\newcommand{\cat}[1]{\mathbf{#1}}
\newcommand{\dfreccia}[1]{\xrightarrow{\ #1 \ }}
\newcommand{\sfreccia}[1]{\xleftarrow{\ #1 \ }}
\newcommand{\formalsum}[2]{{\sum_{#1}^{#2}}{\vphantom{\sum}}'}
\newcommand{\minim}[2]{\mu_{#1}\, \left(#2\right)}
\newcommand{\concat}{\null^{\frown}} % concatenazione di stringe
\newcommand{\godelcode}[1]{\langle\!\langle #1 \rangle\!\rangle}
\newcommand{\godeldec}[1]{(\!(#1)\!)}
\newcommand{\termcode}[1]{\ulcorner #1\urcorner}
\newcommand{\partialto}{\dashrightarrow}
\newcommand{\restricted}{\upharpoonright}
\newcommand{\embeds}{\precsim}
\newcommand{\surjects}{\twoheadrightarrow}
\newcommand{\equipotenti}{\asymp}
%% \newcommand{\dotplus}{\mathbin{\dot{+}}} %% A quanto pare esiste già
\newcommand{\bigdot}{\mathbin{\boldsymbol{\cdot}}}
\newcommand{\dotexp}[1]{^{.#1}}
\newcommand{\conv}{\mathbin{*}}
\newcommand{\convolution}[2]{(#1\conv #2)}
\newcommand{\nil}{\mathfrak{N}}
\newcommand{\divisore}{\mathrel{|}}
\newcommand{\simplesso}[1]{\mathrm{e}_{#1}}

\renewcommand{\iff}{\mathrel{\longleftrightarrow}} %% Notazione Logica.
\newcommand{\oldiff}{\mathrel{\Longleftrightarrow}}
\renewcommand{\implies}{\mathrel{\rightarrow}} %% Notazione Logica
\newcommand{\oldimplies}{\mathrel{\Longrightarrow}}
\renewcommand{\impliedby}{\mathrel{\leftarrow}} %% Notazione Logica
\newcommand{\oldimpliedby}{\mathrel{\Longleftarrow}}

\newcommand{\IFF}{\quad\Longleftrightarrow\quad}
\newcommand{\IMPLICA}{\quad\Longrightarrow\quad}


\renewcommand{\descriptionlabel}[1]{\hspace{\labelsep}\normalfont #1} % remove bold from description


%% Definizione di Divergenza di K-L

\DeclarePairedDelimiterX{\infdivx}[2]{(}{)}{%
  #1\;\delimsize\|\;#2%
}
\newcommand{\kldiv}{D_{KL}\infdivx}

%% Definizione di \dotminus

\makeatletter
\newcommand{\dotminus}{\mathbin{\text{\@dotminus}}}

\newcommand{\@dotminus}{%
  \ooalign{\hidewidth\raise1ex\hbox{.}\hidewidth\cr$\m@th-$\cr}%
}
\makeatother

%tramite i prossimi due comandi posso decidere come scrivere i logaritmi naturali in tutti i documenti: ho infatti eliminato qualsiasi differenza tra "ln" e "log": se si vuole qualcosa di diverso bisogna inserire manualmente il tutto
\let\ln\relax
\DeclareMathOperator{\ln}{ln}
\let\log\relax
\DeclareMathOperator{\log}{log}
%%%%%%

%% NUOVI COMANDI
\newcommand{\straniero}[1]{\textit{#1}} %parole straniere
\newcommand{\titolo}[1]{\textsc{#1}} %titoli
\newcommand{\qedd}{\tag*{$\blacksquare$}} %qed per ambienti matemastici
\renewcommand{\qedsymbol}{$\blacksquare$} %modifica colore qed
\newcommand{\ooverline}[1]{\overline{\overline{#1}}}
\newcommand{\circoletto}[1]{\left(#1\right)^{\text{o}}}
%
\newcommand{\qmatrice}[1]{\begin{pmatrix}
#1_{11} & \cdots & #1_{1n}\\
\vdots & \ddots & \vdots \\
#1_{m1} & \cdots & #1_{mn}
\end{pmatrix}}
%
\newcommand{\parentesi}[2]{%
\underset{#1}{\underbrace{#2}}%
}
%
\newcommand{\norma}[1]{% Norma
\left\lVert#1\right\rVert%
}
\newcommand{\scalare}[2]{% Scalare
\left\langle #1, #2\right\rangle
}
%%%%%

%% RESTRIZIONI
\newcommand{\referenze}[2]{
        \phantomsection{}#2\textsuperscript{\textcolor{blue}{\textbf{#1}}}
}

\let\restriction\relax

\def\restriction#1#2{\mathchoice
              {\setbox1\hbox{${\displaystyle #1}_{\scriptstyle #2}$}
              \restrictionaux{#1}{#2}}
              {\setbox1\hbox{${\textstyle #1}_{\scriptstyle #2}$}
              \restrictionaux{#1}{#2}}
              {\setbox1\hbox{${\scriptstyle #1}_{\scriptscriptstyle #2}$}
              \restrictionaux{#1}{#2}}
              {\setbox1\hbox{${\scriptscriptstyle #1}_{\scriptscriptstyle #2}$}
              \restrictionaux{#1}{#2}}}
\def\restrictionaux#1#2{{#1\,\smash{\vrule height .8\ht1 depth .85\dp1}}_{\,#2}}
%%%%%%%%%%%

%%% FORMATTAZIONE FOOTNOTEMARK

\def\footnotemarkformatting#1{[#1]}
\renewcommand{\thefootnote}{\footnotemarkformatting{\arabic{footnote}}}

%% SEZIONE GRAFICA
\use{tikz}
\usetikzlibrary{matrix, patterns, calc, decorations.pathreplacing, hobby, decorations.markings, decorations.pathmorphing, babel}
\use{tikz-3dplot}
\use{mathrsfs} %per geogebra
\use{tikz-cd}
\tikzset
{
  %surface/.style={fill=black!10, shading=ball,fill opacity=0.4},
  plane/.style={black,pattern=north east lines},
  curve/.style={black,line width=0.5mm},
  dritto/.style={decoration={markings,mark=at position 0.5 with {\arrow{Stealth}}}, postaction=decorate},
  rovescio/.style={decoration={markings,mark=at position 0.5 with {\arrow{Stealth[reversed]}}}, postaction=decorate}
}
\use{pgfplots} % stampare le funzioni
        \pgfplotsset{/pgf/number format/use comma,compat=1.15}
        %\pgfplotsset{compat=1.15} %per geogebra
        \usepgfplotslibrary{fillbetween, polar}
%%%%%%

%% CITAZIONI
\use{lineno}

\newcommand{\citazione}[1]{%
  \begin{quotation}
  \begin{linenumbers}
  \modulolinenumbers[5]
  \begingroup
  \setlength{\parindent}{0cm}
  \noindent #1
  \endgroup
  \end{linenumbers}
  \end{quotation}\setcounter{linenumber}{1}
  }
%%%%%%

%%%%%%%%%%%%%%%%%%%%%%%%%%%%%%%%%%%%%%%%%%%%
%%%%%%%%%%%%%%%%%%%%%%%%%%%%%%%%%%%%%%%%%%%%

%% AMS THM

\theoremstyle{definition}% default
\newtheorem{thm}{Teorema}[section]
\newtheorem{lem}[thm]{Lemma}
\newtheorem{prop}[thm]{Proposizione}
\newtheorem{cor}[thm]{Corollario}
\newtheorem{esempio}[thm]{Esempio}
\theoremstyle{plain}
\newtheorem{definizione}[thm]{Definizione}
\theoremstyle{remark}
\newtheorem*{oss}{Osservazione}


%%%%%%%%%%%%%%%%%%%%%%%%%%%%%%%%%%%%%%%%%%%%
%%%%%%%%%%%%%%%%%%%%%%%%%%%%%%%%%%%%%%%%%%%%

\use{hyperref}
\hypersetup{%
        pdfauthor={Davide Peccioli},
        pdfsubject={},
        allcolors=black,
        citecolor=black,
%	colorlinks=true,
        bookmarksopen=true}
\setcounter{secnumdepth}{0} % rimuove i numeri di sezione senza rimuovere le ref
\renewcommand{\href}[2]{\textcolor{blue}{#2}} % disabilita il comando href
\use{enotez} %
\setenotez{%
 mark-format = \footnotemarkformatting % Mette i numeri tra parentesi quadre%
}\let\footnote=\endnote % rende tutte le note a pié pagina come delle note a fine file 


\let\olddocument\document % modifico l'ambiende documenti per non dover stampare \printendnote
\let\oldenddocument\enddocument
\renewenvironment{document}%
{%
  \olddocument
}{%
  \printendnotes\oldenddocument
}
\renewcommand{\thethm}{\arabic{thm}}

\usepackage[hyperref]{biblatex}
\addbibresource{~/Documents/org/roam/bib/master.bib}
\author{Davide Peccioli}
\date{\today}
\title{Caratterizzazione di magri e comagri tramite il gioco di Banach-Mazur}
\begin{document}

\section{Teorema I}
\label{sec:org80b265e}

Sia \(X\) uno \href{20250103145124-topologia.org}{spazio topologico} \href{20250131161811-insieme_vuoto_mk.org}{non vuoto}, e sia \(A \subseteq X\) un \href{20250131155822-operazioni_insiemistiche_tra_classi_mk.org}{sottoinsieme} qualsiasi. Allora \(A\) è \href{20250419122752-insieme_magro.org}{comagro} se e solo se il giocatore II ha una \href{20250513171520-giochi_di_gale_stewart.org}{strategia vincente} nel \href{20250513111844-gioco_di_banach_mazur.org}{gioco di Banach-Mazur} \(G^{**}(A)\).
\subsection{Dimostrazione}
\label{sec:orgf564b78}

(\(\Rightarrow\)): Se \(A\) è comagro, allora esistono \((W_{n})_{n \in\omega}\) aperti densi di \(X\) tali che
\begin{equation*}
\bigcap_{n \in\omega} W_{n} \supseteq A.
\end{equation*}

Il giocatore II gioca \(V_{n} \coloneqq W_{n}\cap U_{n}\); questo è aperto, e inoltre è non vuoto poiché \(W_{n}\) è denso in \(X\).

(\(\Leftarrow\)): Sia \(\sigma\) una strategia vincente di II. Si costruisce \(\sigma' \subseteq \sigma\) albero potato e non vuoto per induzione sulla lunghezza delle stringhe.
\begin{itemize}
\item \(\emptyset \in \sigma'\).
\item Sia \(s=\langle U_{0},V_{0},\dots,U_{n}\rangle\). Allora esiste un unico \(V_{n} \subseteq U_{n}\) tale che \(s\concat V_{n} \in \sigma\). Si pone \(s\concat V_{n} \in \sigma'\).
\item Sia \(s = \langle U_{0},V_{0},\dots, U_{n}, V_{n}\rangle \in \sigma'\). Per ogni sottoinsieme aperto \(U \subseteq V_{n}\) si definisce \(U^{*}\) l'unico sottoinsieme di \(U\) tale che
\begin{equation*}
  s\concat \langle U,U^{*}\rangle \in \sigma
\end{equation*}

È possibile, tramite un'applicazione del Lemma di Zorn, garantire l'esistenza di una collezione massimale \(\mathcal{U}_{s}\) di aperti non vuoti \(U \subseteq V_{n}\) tale che la collezione \(\mathcal{V}_{s} \coloneqq \set{U^{*}\mid U \in \mathcal{U}_{s}}\) sia composta da insiemi a due a due disgiunti.

\begin{itemize}
\item Infatti, data una catena di collezioni di aperti che soddisfino la proprietà richiesta \((\mathcal{U}_{\alpha})_{\alpha}\) ordinata dall'inclusione, allora
\begin{equation*}
\mathcal{U}^{\star}\coloneqq \bigcup_{\alpha} \mathcal{U}_{\alpha}
\end{equation*}
è un maggiorante della catena, in quanto detto
\begin{equation*}
\mathcal{V}^{\star} \coloneqq \set{U^{*}\mid U \in \mathcal{U}^{\star}}
\end{equation*}
dati \(V,V' \in \mathcal{V}^{\star}\) allora esiste \(\mathcal{U}_{\alpha_{0}}\) ed esistono \(U_{0},U_{1} \in \mathcal{U}_{\alpha_{0}}\) tali che
\begin{equation*}
U_{0}^{*}=V,\quad U_{1}^{*} = V'
\end{equation*}
e pertanto \(V\cap V' =\emptyset\).
\end{itemize}

Dunque, per ogni \(U \in \mathcal{U}_{s}\), \(s\concat U \in \sigma'\).

Inoltre \(\bigcup \mathcal{V}_{s}\) è denso in \(V_{n}\). Infatti, se per assurdo esistesse \(B \subseteq V_{n}\) aperto tale che \(B\cap \bigcup \mathcal{V}_{s} = \emptyset\), allora \(\mathcal{U}_{s}\cup \set{B}\) viola la massimalità di \(\mathcal{U}_{s}\).
\end{itemize}

Sia ora, per ogni \(n \in\omega\):
\begin{equation*}
W_{n+1} \coloneqq \bigcup_{\substack{s \in \sigma'\\
\operatorname{lh}(s) = 2n}} \bigcup \mathcal{V}_{s} = \bigcup_{\langle U_{0},V_{0},\dots, U_{n+1}, V_{n+1}\rangle \in \sigma'} V_{n+1}
\end{equation*}

Per ogni \(n \in \omega\), \(W_{n+1} \subseteq X\) è denso.
\begin{itemize}
\item \(W_{1}\) è denso, poiché \(\mathcal{U}_{\emptyset}\) è una collezione di aperti di \(X\) tali che \(\mathcal{V}_{\emptyset}\) sia composta da insiemi a due a due disgiunti, e pertanto, se vi fosse \(B \subseteq X\) aperto tale che \(B\cap W_{1} = \emptyset\), allora \(\mathcal{U}_{\emptyset}\cup\set{B}\) viola la massimalità di \(\mathcal{U}_{\emptyset}\).
\item Se \(W_{n+1}\) è denso, allora lo è anche \(W_{n+2}\). Sia \(B \subseteq X\) aperto.

Siccome \(W_{n+1}\) è denso allora \(W_{n+1}\cap B\neq \emptyset\), ed esiste \(\tilde{s} = \langle U_{0},V_{0},\dots,U_{n}, V_{n}\rangle \in \sigma'\) tale che \(B\cap\bigcup \mathcal{V}_{\tilde{s}} \neq \emptyset\).

Quindi esistono \(V_{n}\supseteq U \supseteq V\) tali che \(\tilde{s}\concat \langle U, V\rangle \in \sigma'\), con \(V\cap B\neq \emptyset\). Infatti, se così non fosse, allora \(\mathcal{U}_{\tilde{s}}\cup\set{V_{n}\cap B}\) contraddice la massimalità di \(\mathcal{U}_{\tilde{s}}\).

Poiché \(\bigcup \mathcal{V}_{s\concat \langle U, V\rangle}\) è denso in \(V\), allora \(\bigcup \mathcal{V}_{s\concat \langle U, V\rangle} \cap B \neq \emptyset\), ed inoltre
\begin{equation*}
  \bigcup \mathcal{V}_{\tilde{s}\concat \langle U, V\rangle} \subseteq W_{n+2}
\end{equation*}
e pertanto \(W_{n+2}\cap B\neq \emptyset\).
\end{itemize}

Per finire, si dimostra che \(\bigcap_{n \in \omega} W_{n+1} \subseteq A\). Sia \(x \in \bigcap_{n \in \omega} W_{n+1}\).

Allora esiste \((U_{i}, V_{i})_{i \in \omega} \in [\sigma']\) tale che \(x \in V_{n}\) per ogni \(n\). Questa si costruisce per induzione.
\begin{itemize}
\item Poiché \(x \in W_{1}\), allora esiste \(\langle U_{0},V_{0},U_{1},V_{1}\rangle \in \sigma'\) tale che \(x \in V_{1}\).
\item Sia ora \(p=\langle U_{0},V_{0},\dots,U_{n}, V_{n}\rangle \in\sigma'\) tale che \(x \in V_{n}\).

Siccome \(x \in W_{n+1}\) allora esiste \(p' \in\sigma'\),
\begin{equation*}
  p'\coloneqq \langle U_{0}',V_{0}',\dots,U_{n+1}',V_{n+1}'\rangle
\end{equation*}
tale che \(x \in V_{n+1}\). Necessariamente \(p'\) estende \(p\).

Infatti, si supponga per assurdo che \(p \neq \langle U_{0}',V_{0}',\dots,U_{n}',V_{n}'\rangle\), e sia \(j\le n\) il primo indice tale che
\begin{equation*}
  \langle U_{j}, V_{j}\rangle \neq \langle U_{j}', V_{j}'\rangle.
\end{equation*}
Necessariamente allora \(U_{j}\neq U_{j}'\), poiché \(V_{j}\) e \(V_{j}'\) sono univocamente determinati dall'insieme precedente. In particolare, però:
\begin{equation*}
  U_{j}, U_{j}' \in \mathcal{U}_{\langle U_{0},V_{0},\dots,U_{j-1},V_{j-1}\rangle} = \mathcal{U}_{\langle U_{0}',V_{0}',\dots,U_{j-1}',V_{j-1}'\rangle}
\end{equation*}
e pertanto, per definizione, \(V_{j}\cap V_{j'} = \emptyset\). Assurdo, poiché \(x \in V_{j}\cap V_{j}'\).
\end{itemize}

Dunque \(\langle U_{0},V_{0},\dots,U_{n}, V_{n}, U_{n+1}', V_{n+1}'\rangle\) estende la sequenza iniziale.

In particolare, quindi \(x \in \bigcap_{n \in \omega} V_{n}\).

Poiché \(\sigma\) è una strategia vincente per il giocatore II, allora per ogni \((U_{i}, V_{i})_{i \in \omega} \in [\sigma'] \subseteq [\sigma]\),
\begin{equation*}
\bigcap_{i \in \omega} U_{i} = \bigcap_{i \in \omega} V_{i}\subseteq A
\end{equation*}
e dunque \(x \in A\).\qed
\section{Teorema II}
\label{sec:orgd10ad1b}
Se \(X\) è uno \href{20250103145124-topologia.org}{spazio topologico} \href{20250514174255-gioco_di_choquet.org}{di Choquet} non \href{20250131161811-insieme_vuoto_mk.org}{vuoto} ed esiste una \href{20250301193511-spazio_metrico.org}{distanza} \(d\) su \(X\) le cui \href{20250301193511-spazio_metrico.org}{palle aperte} sono aperti di \(X\), allora:

\(A\) è \href{20250419122752-insieme_magro.org}{magro} in un \href{20250103145124-topologia.org}{aperto} non vuoto se e solo se il giocatore I ha una \href{20250513171520-giochi_di_gale_stewart.org}{strategia vincente} nel \href{20250513111844-gioco_di_banach_mazur.org}{gioco di Banach-Mazur} \(G^{**}(A)\).
\subsection{Dimostrazione}
\label{sec:org8f28b5d}

(\(\Rightarrow\)): Se \(A\) è magro in \(Y \subseteq X\), sia per ogni \(n \in \omega\): \(W_{n} \subseteq Y\) aperti densi di \(Y\), con
\begin{equation*}
\bigcap_{n \in\omega} W_{n} \subseteq Y \setminus A.
\end{equation*}

\href{20250514174255-gioco_di_choquet.org}{Poiché} \(Y\) è uno \href{20250514174255-gioco_di_choquet.org}{spazio di Choquet}, allora nel \href{20250513171520-giochi_di_gale_stewart.org}{gioco}:
\begin{equation*}
\begin{tikzcd}[ampersand replacement=\&,cramped, sep=tiny]
	{\text{I}} \&\& {B_1} \&\& {B_2} \&\& \dots \\
	{\text{II}} \& {A_0} \&\& {A_1} \&\& \dots
\end{tikzcd}
\end{equation*}
con gli aperti non vuoti \(Y\supseteq V_{0}\supseteq U_{1}\supseteq V_{1}\supseteq \dots\) in cui I vince sse \(\bigcap_{n \in \omega}{B_{n}} \neq \emptyset\), I ha una \href{20250513171520-giochi_di_gale_stewart.org}{strategia vincente}. Questo infatti è un gioco di Choquet a giocatori invertiti.

Sia quindi \(\sigma\) la strategia vincente di I in questo gioco di Choquet.

Nel gioco \(G^{**}(A)\), il giocatore I pone \(U_{0} \coloneqq Y\). Si costruisce per induzione la strategia vincente per I.

Al passo \(n+1\)-esimo, sia \((U_{0},V_{0},\dots, U_{n}, V_{n})\) la sequenza di insiemi giocati. Si pone, per ogni \(i\le n\): \(V_{i}'\coloneqq V_{i}\cap W_{i}\), e si sceglie \(U_{n+1}\) come l'unico sottoinsieme aperto non vuoto di \(V_{n}\) tale che
\begin{equation*}
(V_{0}', U_{1}, V_{1}', U_{2},\dots, V_{n}', U_{n+1}) \in\sigma.
\end{equation*}

Allora \(\bigcap_{n \in \omega} U_{n}\neq\emptyset\) e inoltre
\begin{equation*}
\bigcap_{n \in\omega} U_{n} = \bigcap_{n \in\omega} V_{n}' \subseteq \bigcap_{n \in \omega} W_{n} \subseteq Y\setminus A
\end{equation*}
e dunque \(\bigcap_{n \in\omega} U_{n} \not\subseteq A\).

(\(\Leftarrow\)): Sia \(\sigma\) una strategia vincente per I in \(G^{**}(A)\), e sia \(U_{0}\) l'elemento di partenza per \(\sigma\).

Si costruisce una strategia \(\sigma'\) per I, vincente, e tale che l'insieme giocato al passo \(n\)-esimo \(U_{n}\) abbia diametro (rispetto alla metrica \(d\)):
\begin{equation*}
\operatorname{diam}(U_{n})<2^{-n}.
\end{equation*}

Al passo \(n+1\), sia \((U_{0},V_{0},\dots,U_{n}, V_{n})\) la sequenza di insiemi giocati, e sia \(v_{n} \in V_{n}\). Si definisce
\begin{equation*}
V_{n}'\coloneqq V_{n}\cap B_{d}(v_{n}, 2^{-n-1}), \qquad \operatorname{diam}(V_{n}) \le 2^{-n}
\end{equation*}
che è un aperto non vuoto. Si pone infine \(U_{n+1}\) come l'unico sottoinsieme aperto di \(V_{n}'\) tale che
\begin{equation*}
(U_{0},V_{0},\dots,U_{n}, V_{n}', U_{n+1}) \in \sigma.
\end{equation*}

Questo \(U_{n+1}\) è la risposta secondo la strategia \(\sigma'\), in quanto \(\operatorname{diam}(U_{n})\le \operatorname{diam}(V_{n}')\le 2^{-n}\).

Siccome \(\sigma'\) è una strategia vincente per I, allora
\begin{equation*}
\emptyset\neq\bigcap_{n \in \omega} U_{n}
\end{equation*}
e inoltre
\begin{equation*}
\operatorname{diam}\left(\bigcap_{n \in \omega} U_{n}\right) = 0
\end{equation*}
Segue che \(\bigcap_{n \in \omega} U_{n} = \set{x}\), con \(x \in U_{0}\setminus A\).

Sia quindi
\begin{equation*}
W\coloneqq \set{x \in U_{0}\mid
\exists\, (U_{i}, V_{i})_{i \in \omega} \in [\sigma']\ x \in \bigcap_{n \in \omega} U_{i}}
\end{equation*}
\begin{itemize}
\item \(W\) è denso in \(U_{0}\), poiché per ogni \(B \subseteq U_{0}\) esiste \(p = (U_{i}, V_{i})_{i \in \omega} \in [\sigma']\) tale che \(V_{0} = B\), e, siccome \(p \in [\sigma']\) allora
\begin{equation*}
  \bigcap_{n \in \omega} U_{i} = \set{x} \subseteq U_{1} \subseteq V_{0} = B
\end{equation*}
e dunque \(W\cap B\neq \emptyset\).
\item Inoltre \(W \subseteq U_{0}\setminus A\), per costruzione di \(\sigma'\).
\end{itemize}

Pertanto \(A\) è magro in \(U_{0}\).\qed
\end{document}
