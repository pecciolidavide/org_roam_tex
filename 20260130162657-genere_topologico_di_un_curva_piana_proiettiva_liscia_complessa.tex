% Created 2026-02-07 Sat 19:31
% Intended LaTeX compiler: pdflatex
\documentclass[10pt]{article}
%% CREATO CON ORG - EMACS
\newcommand{\use}[2][]{\usepackage[#1]{#2}}
% PACCHETTI FONDAMENTLAI
\use[utf8]{inputenc}
\use[T1]{fontenc}
\use{graphicx}
\use{longtable}
\use{wrapfig}
\use{rotating}
\use[normalem]{ulem}
\use{amsmath}
\use{amsthm}
\use{amssymb}

\use{eucal} % Cambia mathcal{...}

\use{capt-of}
\use[italian]{babel}
\use[babel]{csquotes}
% bib la TEX lo carica in automatico org-cite
\use{microtype}
\use{lmodern}
\use{subfig} % sottofigure
\use{multicol} % due colonne
\use{lipsum} % lorem ipsum
\use{color} % colori in latex
\use{parskip} % rimuove l'indentazione dei nuovi paragrafi %% Add parbox=false to all new tcolorbox
\use{centernot}
\use[outline]{contour}\contourlength{3pt}
\use{fancyhdr}
\use{layout}
\use[most]{tcolorbox} % Riquadri colorati
\use{ifthen} % IFTHEN
\use{geometry}

% pacchetti matematica
\use{yhmath}
\use{dsfont}
\use{mathrsfs}
\use{cancel} % semplificare
\use{polynom} %divisione tra polinomi
\use{forest} % grafi ad albero
\use{booktabs} % tabelle
\use{commath} %simboli e differenziali
\use{bm} %bold
\use[fulladjust]{marginnote} %to use marginnote for date notes
\use{arrayjobx}%array
\use[intlimits]{empheq} % Riquadri colorati attorno alle equazioni
\use{mathtools}
\use{circuitikz} % Disegnare i circuiti
\use{mathtools}
\use{stmaryrd} % [[ \llbracket ]] \rrbracket
\use{bussproofs} % dimostrazioni

%%%%%%%%%%%%%


%%%% QUIVER
\newcommand{\duepunti}{\,\mathchar\numexpr"6000+`:\relax\,}
% A TikZ style for curved arrows of a fixed height, due to AndréC.
\tikzset{curve/.style={settings={#1},to path={(\tikztostart)
    .. controls ($(\tikztostart)!\pv{pos}!(\tikztotarget)!\pv{height}!270:(\tikztotarget)$)
    and ($(\tikztostart)!1-\pv{pos}!(\tikztotarget)!\pv{height}!270:(\tikztotarget)$)
    .. (\tikztotarget)\tikztonodes}},
    settings/.code={\tikzset{quiver/.cd,#1}
        \def\pv##1{\pgfkeysvalueof{/tikz/quiver/##1}}},
    quiver/.cd,pos/.initial=0.35,height/.initial=0}

% TikZ arrowhead/tail styles.
\tikzset{tail reversed/.code={\pgfsetarrowsstart{tikzcd to}}}
\tikzset{2tail/.code={\pgfsetarrowsstart{Implies[reversed]}}}
\tikzset{2tail reversed/.code={\pgfsetarrowsstart{Implies}}}
% TikZ arrow styles.
\tikzset{no body/.style={/tikz/dash pattern=on 0 off 1mm}}
%%%%%%%%%%


%% DEFINIZIONI COMANDI MATEMATICI
\let\sin\relax %TOGLIE LA DEFINIZIONE SU "\sin"

% cambia la definizione di empty set
% ---
\let\oldemptyset\emptyset
% ---
% \let\emptyset\varnothing
% ---
% \let\emptyset\relax
% \newcommand{\emptyset}{\text{\textnormal{\O}}}
% ---

\DeclareMathOperator{\bounded}{bd}
\DeclareMathOperator{\sin}{sen}
\DeclareMathOperator{\epi}{Epi}
\DeclareMathOperator{\cl}{cl}
\DeclareMathOperator{\graph}{graph}
\DeclareMathOperator{\arcsec}{arcsec}
\DeclareMathOperator{\arccot}{arccot}
\DeclareMathOperator{\arccsc}{arccsc}
\DeclareMathOperator{\spettro}{Spettro}
\DeclareMathOperator{\nulls}{nullspace}
\DeclareMathOperator{\dom}{dom}
\DeclareMathOperator{\ar}{ar}
\DeclareMathOperator{\const}{Const}
\DeclareMathOperator{\fun}{Fun}
\DeclareMathOperator{\rel}{Rel}
\DeclareMathOperator{\altezza}{ht}
\let\det\relax %TOGLIE LA DEFINIZIONE SU "\det"
\DeclareMathOperator{\det}{det}
\DeclareMathOperator{\End}{End}
\DeclareMathOperator{\gl}{GL}
\def\Id{\mathrm{Id}}
\def\id{\mathrm{id}}
\DeclareMathOperator{\I}{\mathds{1}}
\DeclareMathOperator{\II}{II}
\DeclareMathOperator{\rank}{rank}
\DeclareMathOperator{\tr}{tr}
\DeclareMathOperator{\tc}{t.c.}
\DeclareMathOperator{\T}{T}
\DeclareMathOperator{\var}{Var}
\DeclareMathOperator{\cov}{Cov}
\DeclareMathOperator{\st}{st}
\DeclareMathOperator{\mon}{Mon}
\newcommand{\card}[1]{\left\vert #1 \right\vert}
\newcommand{\trasposta}[1]{\prescript{\text{T}}{}{#1}}
\newcommand{\1}{\mathds{1}}
\newcommand{\R}{\mathds{R}}
\newcommand{\diesis}{\#}
\newcommand{\bemolle}{\flat}
\newcommand{\nonstandard}[1]{\prescript{*}{}{#1}}
\newcommand{\starR}{\nonstandard{\R}}
\newcommand{\borel}{\mathscr{B}}
\newcommand{\lebesgue}[1]{\mathscr{L}\left(#1\right)}
\newcommand{\media}{\mathds{E}}
\newcommand{\K}{\mathds{K}}
\newcommand{\A}{\mathds{A}}
\newcommand{\Q}{\mathds{Q}}
\newcommand{\N}{\mathds{N}}
\newcommand{\C}{\mathds{C}}
\newcommand{\Z}{\mathds{Z}}
\newcommand{\qo}{\hspace{1em}\text{q.o.}\,}
\renewcommand{\tilde}[1]{\widetilde{#1}}
\renewcommand{\parallel}{\mathrel{/\mkern-5mu/}}
\newcommand{\parti}[2][]{\wp_{#1}(#2)}
\newcommand{\diff}[1]{\operatorname{d}_{#1}}
\let\oldvec\vec
\renewcommand{\vec}[1]{\overrightarrow{\vphantom{i}#1}}
\newcommand{\floor}[1]{\left\lfloor #1 \right\rfloor}
\newcommand{\cat}[1]{\mathbf{#1}}
\newcommand{\dfreccia}[1]{\xrightarrow{\ #1 \ }}
\newcommand{\sfreccia}[1]{\xleftarrow{\ #1 \ }}
\newcommand{\formalsum}[2]{{\sum_{#1}^{#2}}{\vphantom{\sum}}'}
\newcommand{\minim}[2]{\mu_{#1}\, \left(#2\right)}
\newcommand{\concat}{\null^{\frown}} % concatenazione di stringe
\newcommand{\godelcode}[1]{\langle\!\langle #1 \rangle\!\rangle}
\newcommand{\godeldec}[1]{(\!(#1)\!)}
\newcommand{\termcode}[1]{\ulcorner #1\urcorner}
\newcommand{\partialto}{\dashrightarrow}
\newcommand{\restricted}{\upharpoonright}
\newcommand{\embeds}{\precsim}
\newcommand{\surjects}{\twoheadrightarrow}
\newcommand{\equipotenti}{\asymp}
%% \newcommand{\dotplus}{\mathbin{\dot{+}}} %% A quanto pare esiste già
\newcommand{\bigdot}{\mathbin{\boldsymbol{\cdot}}}
\newcommand{\dotexp}[1]{^{.#1}}
\newcommand{\conv}{\mathbin{*}}
\newcommand{\convolution}[2]{(#1\conv #2)}
\newcommand{\nil}{\mathfrak{N}}
\newcommand{\divisore}{\mathrel{|}}
\newcommand{\simplesso}[1]{\mathrm{e}_{#1}}

\renewcommand{\iff}{\mathrel{\longleftrightarrow}} %% Notazione Logica.
\newcommand{\oldiff}{\mathrel{\Longleftrightarrow}}
\renewcommand{\implies}{\mathrel{\rightarrow}} %% Notazione Logica
\newcommand{\oldimplies}{\mathrel{\Longrightarrow}}
\renewcommand{\impliedby}{\mathrel{\leftarrow}} %% Notazione Logica
\newcommand{\oldimpliedby}{\mathrel{\Longleftarrow}}

\newcommand{\IFF}{\quad\Longleftrightarrow\quad}
\newcommand{\IMPLICA}{\quad\Longrightarrow\quad}


\renewcommand{\descriptionlabel}[1]{\hspace{\labelsep}\normalfont #1} % remove bold from description


%% Definizione di Divergenza di K-L

\DeclarePairedDelimiterX{\infdivx}[2]{(}{)}{%
  #1\;\delimsize\|\;#2%
}
\newcommand{\kldiv}{D_{KL}\infdivx}

%% Definizione di \dotminus

\makeatletter
\newcommand{\dotminus}{\mathbin{\text{\@dotminus}}}

\newcommand{\@dotminus}{%
  \ooalign{\hidewidth\raise1ex\hbox{.}\hidewidth\cr$\m@th-$\cr}%
}
\makeatother

%tramite i prossimi due comandi posso decidere come scrivere i logaritmi naturali in tutti i documenti: ho infatti eliminato qualsiasi differenza tra "ln" e "log": se si vuole qualcosa di diverso bisogna inserire manualmente il tutto
\let\ln\relax
\DeclareMathOperator{\ln}{ln}
\let\log\relax
\DeclareMathOperator{\log}{log}
%%%%%%

%% NUOVI COMANDI
\newcommand{\straniero}[1]{\textit{#1}} %parole straniere
\newcommand{\titolo}[1]{\textsc{#1}} %titoli
\newcommand{\qedd}{\tag*{$\blacksquare$}} %qed per ambienti matemastici
\renewcommand{\qedsymbol}{$\blacksquare$} %modifica colore qed
\newcommand{\ooverline}[1]{\overline{\overline{#1}}}
\newcommand{\circoletto}[1]{\left(#1\right)^{\text{o}}}
%
\newcommand{\qmatrice}[1]{\begin{pmatrix}
#1_{11} & \cdots & #1_{1n}\\
\vdots & \ddots & \vdots \\
#1_{m1} & \cdots & #1_{mn}
\end{pmatrix}}
%
\newcommand{\parentesi}[2]{%
\underset{#1}{\underbrace{#2}}%
}
%
\newcommand{\norma}[1]{% Norma
\left\lVert#1\right\rVert%
}
\newcommand{\scalare}[2]{% Scalare
\left\langle #1, #2\right\rangle
}
%%%%%

%% RESTRIZIONI
\newcommand{\referenze}[2]{
        \phantomsection{}#2\textsuperscript{\textcolor{blue}{\textbf{#1}}}
}

\let\restriction\relax

\def\restriction#1#2{\mathchoice
              {\setbox1\hbox{${\displaystyle #1}_{\scriptstyle #2}$}
              \restrictionaux{#1}{#2}}
              {\setbox1\hbox{${\textstyle #1}_{\scriptstyle #2}$}
              \restrictionaux{#1}{#2}}
              {\setbox1\hbox{${\scriptstyle #1}_{\scriptscriptstyle #2}$}
              \restrictionaux{#1}{#2}}
              {\setbox1\hbox{${\scriptscriptstyle #1}_{\scriptscriptstyle #2}$}
              \restrictionaux{#1}{#2}}}
\def\restrictionaux#1#2{{#1\,\smash{\vrule height .8\ht1 depth .85\dp1}}_{\,#2}}
%%%%%%%%%%%

%%% FORMATTAZIONE FOOTNOTEMARK

\def\footnotemarkformatting#1{[#1]}
\renewcommand{\thefootnote}{\footnotemarkformatting{\arabic{footnote}}}

%% SEZIONE GRAFICA
\use{tikz}
\usetikzlibrary{matrix, patterns, calc, decorations.pathreplacing, hobby, decorations.markings, decorations.pathmorphing, babel}
\use{tikz-3dplot}
\use{mathrsfs} %per geogebra
\use{tikz-cd}
\tikzset
{
  %surface/.style={fill=black!10, shading=ball,fill opacity=0.4},
  plane/.style={black,pattern=north east lines},
  curve/.style={black,line width=0.5mm},
  dritto/.style={decoration={markings,mark=at position 0.5 with {\arrow{Stealth}}}, postaction=decorate},
  rovescio/.style={decoration={markings,mark=at position 0.5 with {\arrow{Stealth[reversed]}}}, postaction=decorate}
}
\use{pgfplots} % stampare le funzioni
        \pgfplotsset{/pgf/number format/use comma,compat=1.15}
        %\pgfplotsset{compat=1.15} %per geogebra
        \usepgfplotslibrary{fillbetween, polar}
%%%%%%

%% CITAZIONI
\use{lineno}

\newcommand{\citazione}[1]{%
  \begin{quotation}
  \begin{linenumbers}
  \modulolinenumbers[5]
  \begingroup
  \setlength{\parindent}{0cm}
  \noindent #1
  \endgroup
  \end{linenumbers}
  \end{quotation}\setcounter{linenumber}{1}
  }
%%%%%%

%%%%%%%%%%%%%%%%%%%%%%%%%%%%%%%%%%%%%%%%%%%%
%%%%%%%%%%%%%%%%%%%%%%%%%%%%%%%%%%%%%%%%%%%%

%% AMS THM

\theoremstyle{definition}% default
\newtheorem{thm}{Teorema}[section]
\newtheorem{lem}[thm]{Lemma}
\newtheorem{prop}[thm]{Proposizione}
\newtheorem{cor}[thm]{Corollario}
\newtheorem{esempio}[thm]{Esempio}
\theoremstyle{plain}
\newtheorem{definizione}[thm]{Definizione}
\theoremstyle{remark}
\newtheorem*{oss}{Osservazione}


%%%%%%%%%%%%%%%%%%%%%%%%%%%%%%%%%%%%%%%%%%%%
%%%%%%%%%%%%%%%%%%%%%%%%%%%%%%%%%%%%%%%%%%%%

\use{hyperref}
\hypersetup{%
        pdfauthor={Davide Peccioli},
        pdfsubject={},
        allcolors=black,
        citecolor=black,
%	colorlinks=true,
        bookmarksopen=true}
\setcounter{secnumdepth}{0} % rimuove i numeri di sezione senza rimuovere le ref
\renewcommand{\href}[2]{\textcolor{blue}{#2}} % disabilita il comando href
\use{enotez} %
\setenotez{%
 mark-format = \footnotemarkformatting % Mette i numeri tra parentesi quadre%
}\let\footnote=\endnote % rende tutte le note a pié pagina come delle note a fine file 


\let\olddocument\document % modifico l'ambiende documenti per non dover stampare \printendnote
\let\oldenddocument\enddocument
\renewenvironment{document}%
{%
  \olddocument
}{%
  \printendnotes\oldenddocument
}
\renewcommand{\thethm}{\arabic{thm}}

\usepackage[hyperref]{biblatex}
\addbibresource{~/Documents/org/roam/bib/master.bib}
\def\mult#1{\mathrm{mult}_{#1}\,}
\author{Davide Peccioli}
\date{\today}
\title{}
\begin{document}

\section{Genere topologico di un curva piana proiettiva liscia complessa}
\label{sec:org3c40243}
Sia \(X \subseteq \mathds{P}^{2}_{\C}\)\footnote{\(\mathds{P}^{2}_{\C}\) indica il \href{20241231115051-spazio_proiettivo.org}{piano proiettivo complesso}.} una \href{20260129103947-curva_piana_proiettiva_liscia_complessa.org}{curva liscia di grado \(d\)}, e sia \(F(z_{0},z_{1},z_{2})\) \href{20241231121125-polinomi_omogenei.org}{polinomio omogeneo} (di grado \(d\)) di cui \(X\) è \href{20241231112823-radici_polinomiali.org}{luogo di zeri}.

\begin{prop}
Il \href{20260127112828-superficie_di_riemann.org}{genere topologico} di \(X\) è
\begin{equation*}
g(X) = \frac{(d-1)(d-2)}{2}
\end{equation*}
\end{prop}

\begin{proof}
Supponiamo che \((0:1:0) \notin X\)\footnote{A meno di proiettività è sempre vero. Le proiettività sono biolomorfismi.}. Allora
\begin{equation*}
F(0:1:0) \neq 0
\end{equation*}
e quindi \(z_{1}^{d}\) compare in \(F\). Consideriamo ora la funzione olomorfa e non costante:
\begin{align*}
\pi: X &\longrightarrow \mathds{P}^{1}\\
(z_{0}:z_{1}:z_{2}) &\longmapsto (z_{0}:z_{2}).
\end{align*}
Per la \href{20260129171832-formula_di_hurewicz_superfici_di_riemann.org}{Formula di Hurwitz}, detto \(g\) il genere di \(X\):\footnote{Si noti che \(-2 = 2 g(\mathds{P}^{1}) - 2\), e \href{20260128181135-sfera_di_riemann_biolomorfa_al_piano_proiettivo_complesso.org}{\(\mathds{P}^{1}\cong \C_{\infty}\)}, e pertanto \(g(\mathds{P}^{1}) = 0\).
Vedi anche:
\begin{itemize}
\item \href{20260127112905-sfera_di_riemann.org}{Sfera di Riemann}
\item \href{20260130155105-teorema_di_classificazione_delle_superfici_topologiche.org}{Teorema di classificazione delle superfici topologiche}
\item \href{20260129171444-teorema_del_grado_per_olomorfismi_tra_superfici_di_riemannn.org}{Grado di un olomorfismo tra superfici di Riemann}
\end{itemize}}
\begin{equation*}
2g-2 = \deg\pi \cdot (-2) + \parentesi{r\coloneqq}{\sum_{p \in X}\big(\mult{p}\pi -1\big)}.
\end{equation*}

È necessario calcolare \(\deg\pi\) e \(r\).
\begin{itemize}
\item Consideriamo \(U_{2} = \set{z_{2} \neq 0} \subseteq \mathds{P}^{2}_{\C}\), \(U_{2} \cong \C_{x,y}^{2}\) per mezzo della mappa \(x=\frac{z_{0}}{z_{2}}\), \(y=\frac{z_{1}}{z_{2}}\), e sia \(X_{2} \coloneqq X\cap U_{2} \cong X_{2}' \subseteq \C_{x,y}^{2}\). Allora
\begin{align*}
\pi_{2}: X_{2}' &\longrightarrow \C\\
(x,y) &\longmapsto x
\end{align*}
è scrittura in carte locali di \(\pi\), e pertanto per ogni \(p \in X_{2}\): \(\mult{p} \pi = \mult{p'} \pi_{2}\).

Inoltre, \(X_{2}' \subseteq \C_{x,y}^{2}\) è il luogo di zeri
\begin{equation*}
  f(x,y) = 0,\qquad f(x,y) \coloneqq F(x,y,1)
\end{equation*}
e pertanto è una \href{20260129103934-curva_piana_affine_liscia_complessa.org}{curva piana liscia affine}: detta
\begin{align*}
\varphi: \C^{2} &\longrightarrow \C\\
(x,y) &\longmapsto \pd{f}{y}(x,y)
\end{align*}
si ha che\footnote{Per la \href{20260129103934-curva_piana_affine_liscia_complessa.org}{molteplicità dei punti di una Curva Piana affine liscia complessa}.} per ogni \(p' \in X_{2}'\): \(\mult{p'}\pi_{2} = \mathrm{ord}_{p'}\, \varphi + 1\), e quindi
\begin{equation*}
  \boxed{\mult{p}\pi - 1 =  \mathrm{ord}_{p'}\, \varphi}
\end{equation*}

\item Si noti che \(f\) è un polinomio di grado al massimo \(d\), e \(y^{d}\) compare in \(f\), quindi per ogni \(x_{0} \in \C\): \(f(x_{0},y)\) è di grado \(d\).

Si ha che
\(\pi_{2}^{-1}(x_{0}) = \set{(x_{0},y) \in X_{2}' \mid f(x_{0},y) = 0}\),
e \href{20250102154204-teorema_fondamentale_dell_algebra.org}{quindi} \(\pi_{2}^{-1}(x_{0})\) contiene \(d\) zeri contati con molteplicità. Ma la molteplicità \(m\) come zero di \(f(x_{0},y)\) è esattamente l'\href{20260128124105-ordine_di_una_funzione_olomorfa.org}{ordine} di \(\pd{f}{y}(x_{0},y) + 1\), e pertanto
\begin{equation*}
  d = \sum_{p' \in \pi_{2}^{-1}(x_{0})} m_{p'} = \sum_{p' \in \pi_{2}^{-1}(x_{0})} (\mathrm{ord}_{p'}\, \varphi + 1) = \sum_{p' \in \pi_{2}^{-1}(x_{0})} \mult{p'}\pi_{2} = \deg \pi_{2} =\deg \pi
\end{equation*}
L'ultima uguaglianza si ha perché \(\pi_{2}\) è la scrittura in carte locali di \(\pi\). Segue che \(\deg \pi = d\).

\item Consideriamo adesso \(U_{0} = \set{z_{0} \neq 0} \subseteq \mathds{P}^{2}_{\C}\), \(U_{0} \cong \C^{2}_{u,v}\) per mezzo della mappa \(u = \frac{z_{1}}{z_{0}}\), \(v=\frac{z_{2}}{z_{0}}\), e sia \(X_{0}\coloneqq X\cap U_{0} \cong X_{0}' \subseteq \C^{2}_{u,v}\).

\(X_{0}' \subseteq \C^{2}_{u,v}\) è luogo di zeri:
\begin{equation*}
  g(u,v) = 0,\qquad g(u,v) = F(1,u,v).
\end{equation*}

La scrittura in carte locali di \(\pi\) è
\begin{align*}
\pi_{0}: X_{0}' &\longrightarrow \C\\
(u,v) &\longmapsto v
\end{align*}
e pertanto, detta
\begin{align*}
\psi: \C^{2} &\longrightarrow \C\\
(u,v) &\longmapsto \pd{g}{u}
\end{align*}
si ha che, per ogni \(p \in X_{0}\):
\begin{equation*}
  \mult{p} \pi -1  = \mathrm{ord}_{p'}\,\psi.
\end{equation*}

\item Le funzioni \(\varphi\) e \(\psi\) sono olomorfe, e si ha
\begin{align*}
  \varphi(x,y) &= \dpd{f}{y} = \restriction{\dpd{F}{z_{1}}}{(x:y:1)}\\[2ex]
  \psi(u,v) &= \dpd{g}{u} = \restriction{\dpd{F}{z_{1}}}{(1:u:v)}
\end{align*}

Sia ora \(p \in X\setminus X_{2}\): necessariamente \(p \in X_{0}\).

Consideriamo la scrittura di \(\varphi\) per \(p' \in X_{0}'\):
\begin{align*}
  \tilde{\varphi} (u, v) &= \varphi\left(\frac{1}{v},\frac{u}{v}\right) = \dpd{F}{z_{1}} \left(\frac{1}{v},\frac{u}{v}, 1\right)\\
  &= \left(\frac{1}{v}\right)^{d-1} \dpd{F}{z_{1}}(1,u,v) = \left(\frac{1}{v}\right)^{d-1} \psi(u,v).
\end{align*}
Quindi \(\tilde{\varphi}\) è \href{20260128144105-funzione_meromorfa_su_una_superficie_di_riemann.org}{meromorfa}, e per ogni \(p \in X\setminus X_{2}\): l'\href{20260128144433-ordine_di_una_funzione_meromorfa.org}{ordine}
\begin{align*}
  \mathrm{ord}_{p'}\,\varphi &= -(d-1)\mathrm{ord}_{p'} (v) + \mathrm{ord}_{p'} \psi \\
  &= (\mult{p}\pi - 1) - (d-1) \mult{p} \pi
\end{align*}

Quindi, per ogni \(p \in X\setminus X_{2} = \pi^{-1}(1:0)\):
\begin{equation*}
  \boxed{%
  	\mult{p} \pi - 1 = \mathrm{ord}_{p'} \varphi + (d-1)\, \mult{p}\pi %
  }
\end{equation*}
\end{itemize}

Calcolando infine \(r\), si ha
\begin{align*}
r &= \sum_{p \in X} (\mult{p}\pi -1) = \sum_{p \in X_{2}} (\mult{p}\pi-1) + \sum_{p \in \pi^{-1}(1:0)} (\mult{p}\pi - 1)\\
&= \sum_{p \in X_{2}} \mathrm{ord}_{p'}\, \varphi + \sum_{p \in \pi^{-1}(1:0)} \left(\mathrm{ord}_{p'} \varphi + (d-1)\, \mult{p}\pi\right)\\
&= \parentesi{= 0}{\sum_{p \in X} \mathrm{ord}_{p'}}\, \varphi + (d-1) \, \parentesi{\deg\pi=}{\sum_{p \in \pi^{-1}(1:0)} \mult{p}\pi}.
\end{align*}
dove lo zero è dato dal fatto che \href{20260129171638-somma_ordini_di_una_funzione_meromorfa_su_una_superficie_di_riemann_e_nulla.org}{la somma degli ordini è nulla}.

Pertanto \(r=d\cdot (d-1)\), e quindi
\begin{equation*}
2g-2 = -2 d  + d\cdot (d-1) \IMPLICA g = \frac{(d-2)\,(d-1)}{2}.
\qedhere
\end{equation*}
\end{proof}
\end{document}
