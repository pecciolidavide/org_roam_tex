% Created 2026-02-07 Sat 19:34
% Intended LaTeX compiler: pdflatex
\documentclass[10pt]{article}
%% CREATO CON ORG - EMACS
\newcommand{\use}[2][]{\usepackage[#1]{#2}}
% PACCHETTI FONDAMENTLAI
\use[utf8]{inputenc}
\use[T1]{fontenc}
\use{graphicx}
\use{longtable}
\use{wrapfig}
\use{rotating}
\use[normalem]{ulem}
\use{amsmath}
\use{amsthm}
\use{amssymb}

\use{eucal} % Cambia mathcal{...}

\use{capt-of}
\use[italian]{babel}
\use[babel]{csquotes}
% bib la TEX lo carica in automatico org-cite
\use{microtype}
\use{lmodern}
\use{subfig} % sottofigure
\use{multicol} % due colonne
\use{lipsum} % lorem ipsum
\use{color} % colori in latex
\use{parskip} % rimuove l'indentazione dei nuovi paragrafi %% Add parbox=false to all new tcolorbox
\use{centernot}
\use[outline]{contour}\contourlength{3pt}
\use{fancyhdr}
\use{layout}
\use[most]{tcolorbox} % Riquadri colorati
\use{ifthen} % IFTHEN
\use{geometry}

% pacchetti matematica
\use{yhmath}
\use{dsfont}
\use{mathrsfs}
\use{cancel} % semplificare
\use{polynom} %divisione tra polinomi
\use{forest} % grafi ad albero
\use{booktabs} % tabelle
\use{commath} %simboli e differenziali
\use{bm} %bold
\use[fulladjust]{marginnote} %to use marginnote for date notes
\use{arrayjobx}%array
\use[intlimits]{empheq} % Riquadri colorati attorno alle equazioni
\use{mathtools}
\use{circuitikz} % Disegnare i circuiti
\use{mathtools}
\use{stmaryrd} % [[ \llbracket ]] \rrbracket
\use{bussproofs} % dimostrazioni

%%%%%%%%%%%%%


%%%% QUIVER
\newcommand{\duepunti}{\,\mathchar\numexpr"6000+`:\relax\,}
% A TikZ style for curved arrows of a fixed height, due to AndréC.
\tikzset{curve/.style={settings={#1},to path={(\tikztostart)
    .. controls ($(\tikztostart)!\pv{pos}!(\tikztotarget)!\pv{height}!270:(\tikztotarget)$)
    and ($(\tikztostart)!1-\pv{pos}!(\tikztotarget)!\pv{height}!270:(\tikztotarget)$)
    .. (\tikztotarget)\tikztonodes}},
    settings/.code={\tikzset{quiver/.cd,#1}
        \def\pv##1{\pgfkeysvalueof{/tikz/quiver/##1}}},
    quiver/.cd,pos/.initial=0.35,height/.initial=0}

% TikZ arrowhead/tail styles.
\tikzset{tail reversed/.code={\pgfsetarrowsstart{tikzcd to}}}
\tikzset{2tail/.code={\pgfsetarrowsstart{Implies[reversed]}}}
\tikzset{2tail reversed/.code={\pgfsetarrowsstart{Implies}}}
% TikZ arrow styles.
\tikzset{no body/.style={/tikz/dash pattern=on 0 off 1mm}}
%%%%%%%%%%


%% DEFINIZIONI COMANDI MATEMATICI
\let\sin\relax %TOGLIE LA DEFINIZIONE SU "\sin"

% cambia la definizione di empty set
% ---
\let\oldemptyset\emptyset
% ---
% \let\emptyset\varnothing
% ---
% \let\emptyset\relax
% \newcommand{\emptyset}{\text{\textnormal{\O}}}
% ---

\DeclareMathOperator{\bounded}{bd}
\DeclareMathOperator{\sin}{sen}
\DeclareMathOperator{\epi}{Epi}
\DeclareMathOperator{\cl}{cl}
\DeclareMathOperator{\graph}{graph}
\DeclareMathOperator{\arcsec}{arcsec}
\DeclareMathOperator{\arccot}{arccot}
\DeclareMathOperator{\arccsc}{arccsc}
\DeclareMathOperator{\spettro}{Spettro}
\DeclareMathOperator{\nulls}{nullspace}
\DeclareMathOperator{\dom}{dom}
\DeclareMathOperator{\ar}{ar}
\DeclareMathOperator{\const}{Const}
\DeclareMathOperator{\fun}{Fun}
\DeclareMathOperator{\rel}{Rel}
\DeclareMathOperator{\altezza}{ht}
\let\det\relax %TOGLIE LA DEFINIZIONE SU "\det"
\DeclareMathOperator{\det}{det}
\DeclareMathOperator{\End}{End}
\DeclareMathOperator{\gl}{GL}
\def\Id{\mathrm{Id}}
\def\id{\mathrm{id}}
\DeclareMathOperator{\I}{\mathds{1}}
\DeclareMathOperator{\II}{II}
\DeclareMathOperator{\rank}{rank}
\DeclareMathOperator{\tr}{tr}
\DeclareMathOperator{\tc}{t.c.}
\DeclareMathOperator{\T}{T}
\DeclareMathOperator{\var}{Var}
\DeclareMathOperator{\cov}{Cov}
\DeclareMathOperator{\st}{st}
\DeclareMathOperator{\mon}{Mon}
\newcommand{\card}[1]{\left\vert #1 \right\vert}
\newcommand{\trasposta}[1]{\prescript{\text{T}}{}{#1}}
\newcommand{\1}{\mathds{1}}
\newcommand{\R}{\mathds{R}}
\newcommand{\diesis}{\#}
\newcommand{\bemolle}{\flat}
\newcommand{\nonstandard}[1]{\prescript{*}{}{#1}}
\newcommand{\starR}{\nonstandard{\R}}
\newcommand{\borel}{\mathscr{B}}
\newcommand{\lebesgue}[1]{\mathscr{L}\left(#1\right)}
\newcommand{\media}{\mathds{E}}
\newcommand{\K}{\mathds{K}}
\newcommand{\A}{\mathds{A}}
\newcommand{\Q}{\mathds{Q}}
\newcommand{\N}{\mathds{N}}
\newcommand{\C}{\mathds{C}}
\newcommand{\Z}{\mathds{Z}}
\newcommand{\qo}{\hspace{1em}\text{q.o.}\,}
\renewcommand{\tilde}[1]{\widetilde{#1}}
\renewcommand{\parallel}{\mathrel{/\mkern-5mu/}}
\newcommand{\parti}[2][]{\wp_{#1}(#2)}
\newcommand{\diff}[1]{\operatorname{d}_{#1}}
\let\oldvec\vec
\renewcommand{\vec}[1]{\overrightarrow{\vphantom{i}#1}}
\newcommand{\floor}[1]{\left\lfloor #1 \right\rfloor}
\newcommand{\cat}[1]{\mathbf{#1}}
\newcommand{\dfreccia}[1]{\xrightarrow{\ #1 \ }}
\newcommand{\sfreccia}[1]{\xleftarrow{\ #1 \ }}
\newcommand{\formalsum}[2]{{\sum_{#1}^{#2}}{\vphantom{\sum}}'}
\newcommand{\minim}[2]{\mu_{#1}\, \left(#2\right)}
\newcommand{\concat}{\null^{\frown}} % concatenazione di stringe
\newcommand{\godelcode}[1]{\langle\!\langle #1 \rangle\!\rangle}
\newcommand{\godeldec}[1]{(\!(#1)\!)}
\newcommand{\termcode}[1]{\ulcorner #1\urcorner}
\newcommand{\partialto}{\dashrightarrow}
\newcommand{\restricted}{\upharpoonright}
\newcommand{\embeds}{\precsim}
\newcommand{\surjects}{\twoheadrightarrow}
\newcommand{\equipotenti}{\asymp}
%% \newcommand{\dotplus}{\mathbin{\dot{+}}} %% A quanto pare esiste già
\newcommand{\bigdot}{\mathbin{\boldsymbol{\cdot}}}
\newcommand{\dotexp}[1]{^{.#1}}
\newcommand{\conv}{\mathbin{*}}
\newcommand{\convolution}[2]{(#1\conv #2)}
\newcommand{\nil}{\mathfrak{N}}
\newcommand{\divisore}{\mathrel{|}}
\newcommand{\simplesso}[1]{\mathrm{e}_{#1}}

\renewcommand{\iff}{\mathrel{\longleftrightarrow}} %% Notazione Logica.
\newcommand{\oldiff}{\mathrel{\Longleftrightarrow}}
\renewcommand{\implies}{\mathrel{\rightarrow}} %% Notazione Logica
\newcommand{\oldimplies}{\mathrel{\Longrightarrow}}
\renewcommand{\impliedby}{\mathrel{\leftarrow}} %% Notazione Logica
\newcommand{\oldimpliedby}{\mathrel{\Longleftarrow}}

\newcommand{\IFF}{\quad\Longleftrightarrow\quad}
\newcommand{\IMPLICA}{\quad\Longrightarrow\quad}


\renewcommand{\descriptionlabel}[1]{\hspace{\labelsep}\normalfont #1} % remove bold from description


%% Definizione di Divergenza di K-L

\DeclarePairedDelimiterX{\infdivx}[2]{(}{)}{%
  #1\;\delimsize\|\;#2%
}
\newcommand{\kldiv}{D_{KL}\infdivx}

%% Definizione di \dotminus

\makeatletter
\newcommand{\dotminus}{\mathbin{\text{\@dotminus}}}

\newcommand{\@dotminus}{%
  \ooalign{\hidewidth\raise1ex\hbox{.}\hidewidth\cr$\m@th-$\cr}%
}
\makeatother

%tramite i prossimi due comandi posso decidere come scrivere i logaritmi naturali in tutti i documenti: ho infatti eliminato qualsiasi differenza tra "ln" e "log": se si vuole qualcosa di diverso bisogna inserire manualmente il tutto
\let\ln\relax
\DeclareMathOperator{\ln}{ln}
\let\log\relax
\DeclareMathOperator{\log}{log}
%%%%%%

%% NUOVI COMANDI
\newcommand{\straniero}[1]{\textit{#1}} %parole straniere
\newcommand{\titolo}[1]{\textsc{#1}} %titoli
\newcommand{\qedd}{\tag*{$\blacksquare$}} %qed per ambienti matemastici
\renewcommand{\qedsymbol}{$\blacksquare$} %modifica colore qed
\newcommand{\ooverline}[1]{\overline{\overline{#1}}}
\newcommand{\circoletto}[1]{\left(#1\right)^{\text{o}}}
%
\newcommand{\qmatrice}[1]{\begin{pmatrix}
#1_{11} & \cdots & #1_{1n}\\
\vdots & \ddots & \vdots \\
#1_{m1} & \cdots & #1_{mn}
\end{pmatrix}}
%
\newcommand{\parentesi}[2]{%
\underset{#1}{\underbrace{#2}}%
}
%
\newcommand{\norma}[1]{% Norma
\left\lVert#1\right\rVert%
}
\newcommand{\scalare}[2]{% Scalare
\left\langle #1, #2\right\rangle
}
%%%%%

%% RESTRIZIONI
\newcommand{\referenze}[2]{
        \phantomsection{}#2\textsuperscript{\textcolor{blue}{\textbf{#1}}}
}

\let\restriction\relax

\def\restriction#1#2{\mathchoice
              {\setbox1\hbox{${\displaystyle #1}_{\scriptstyle #2}$}
              \restrictionaux{#1}{#2}}
              {\setbox1\hbox{${\textstyle #1}_{\scriptstyle #2}$}
              \restrictionaux{#1}{#2}}
              {\setbox1\hbox{${\scriptstyle #1}_{\scriptscriptstyle #2}$}
              \restrictionaux{#1}{#2}}
              {\setbox1\hbox{${\scriptscriptstyle #1}_{\scriptscriptstyle #2}$}
              \restrictionaux{#1}{#2}}}
\def\restrictionaux#1#2{{#1\,\smash{\vrule height .8\ht1 depth .85\dp1}}_{\,#2}}
%%%%%%%%%%%

%%% FORMATTAZIONE FOOTNOTEMARK

\def\footnotemarkformatting#1{[#1]}
\renewcommand{\thefootnote}{\footnotemarkformatting{\arabic{footnote}}}

%% SEZIONE GRAFICA
\use{tikz}
\usetikzlibrary{matrix, patterns, calc, decorations.pathreplacing, hobby, decorations.markings, decorations.pathmorphing, babel}
\use{tikz-3dplot}
\use{mathrsfs} %per geogebra
\use{tikz-cd}
\tikzset
{
  %surface/.style={fill=black!10, shading=ball,fill opacity=0.4},
  plane/.style={black,pattern=north east lines},
  curve/.style={black,line width=0.5mm},
  dritto/.style={decoration={markings,mark=at position 0.5 with {\arrow{Stealth}}}, postaction=decorate},
  rovescio/.style={decoration={markings,mark=at position 0.5 with {\arrow{Stealth[reversed]}}}, postaction=decorate}
}
\use{pgfplots} % stampare le funzioni
        \pgfplotsset{/pgf/number format/use comma,compat=1.15}
        %\pgfplotsset{compat=1.15} %per geogebra
        \usepgfplotslibrary{fillbetween, polar}
%%%%%%

%% CITAZIONI
\use{lineno}

\newcommand{\citazione}[1]{%
  \begin{quotation}
  \begin{linenumbers}
  \modulolinenumbers[5]
  \begingroup
  \setlength{\parindent}{0cm}
  \noindent #1
  \endgroup
  \end{linenumbers}
  \end{quotation}\setcounter{linenumber}{1}
  }
%%%%%%

%%%%%%%%%%%%%%%%%%%%%%%%%%%%%%%%%%%%%%%%%%%%
%%%%%%%%%%%%%%%%%%%%%%%%%%%%%%%%%%%%%%%%%%%%

%% AMS THM

\theoremstyle{definition}% default
\newtheorem{thm}{Teorema}[section]
\newtheorem{lem}[thm]{Lemma}
\newtheorem{prop}[thm]{Proposizione}
\newtheorem{cor}[thm]{Corollario}
\newtheorem{esempio}[thm]{Esempio}
\theoremstyle{plain}
\newtheorem{definizione}[thm]{Definizione}
\theoremstyle{remark}
\newtheorem*{oss}{Osservazione}


%%%%%%%%%%%%%%%%%%%%%%%%%%%%%%%%%%%%%%%%%%%%
%%%%%%%%%%%%%%%%%%%%%%%%%%%%%%%%%%%%%%%%%%%%

\use{hyperref}
\hypersetup{%
        pdfauthor={Davide Peccioli},
        pdfsubject={},
        allcolors=black,
        citecolor=black,
%	colorlinks=true,
        bookmarksopen=true}
\setcounter{secnumdepth}{0} % rimuove i numeri di sezione senza rimuovere le ref
\renewcommand{\href}[2]{\textcolor{blue}{#2}} % disabilita il comando href
\use{enotez} %
\setenotez{%
 mark-format = \footnotemarkformatting % Mette i numeri tra parentesi quadre%
}\let\footnote=\endnote % rende tutte le note a pié pagina come delle note a fine file 


\let\olddocument\document % modifico l'ambiende documenti per non dover stampare \printendnote
\let\oldenddocument\enddocument
\renewenvironment{document}%
{%
  \olddocument
}{%
  \printendnotes\oldenddocument
}
\renewcommand{\thethm}{\arabic{thm}}

\usepackage[hyperref]{biblatex}
\addbibresource{~/Documents/org/roam/bib/master.bib}
\author{Davide Peccioli}
\date{\today}
\title{Funzioni ricorsive}
\begin{document}

Vedi \href{20250202130045-insieme_dei_numeri_naturali_mk.org}{Insieme dei numeri naturali MK}.

Tutte le funzioni sono da considerarsi potenzialmente \href{20250213105339-funzione_parziale.org}{parziali}.
\section{Definizione}
\label{sec:org1b6e5a5}
Questa definizione può essere \href{20250216162850-funzioni_ricorsive_in_piu_dimensioni.org}{ampliata} a funzioni in più dimensioni.

L'\href{20250130104331-insieme_mk.org}{insieme} \(\mathcal{R}\) delle \textbf{\href{20250202170607-classe_relazione_binaria.org}{funzioni} ricorsive} è la più piccola \href{20250130104320-classe_mk.org}{classe} contenente:
\begin{itemize}
\item la funzione costante nulla,
\begin{equation*}
  c_{0}:\N\to \N: x\mapsto 0
\end{equation*}
\item la \href{20250202124648-successore_di_un_insieme_mk.org}{funzione successore}:
\begin{equation*}
  S:\N\to \N: x\mapsto x+1
\end{equation*}
\item le funzioni proiezione \(U_{i}^{k}\), per ogni \(k \in \N^{+}\) e per ogni \(0<i \le k\):
\begin{align*}
  U_{i}^{k}: \N^{k} &\longrightarrow \N\\
  (x_{1},\dots,x_{k})&\longmapsto x_{i}
\end{align*}
In particolare, la funzione \(U_{1}^{1}:\N\to \N\) è la funzione identità;
\end{itemize}

e chiusa per:
\subsection{Schema di Composizione di funzioni ricorsive}
\label{sec:org726ba86}
Siano \(k,\ell \in \N^{+}\). Se \(h: \N^{k}\to \N\) e per ogni \(1\le i\le k\): \(g_{i}: \N^{\ell}\to \N\) sono funzioni ricorsive, allora la funzione \(f:\N^{\ell}\to \N\):
\begin{equation*}
f(x_{1},\dots,x_{\ell}) \coloneqq h\left(g_{1}(x_{1},\dots,x_{\ell}),\dots,g_{k}(x_{1},\dots,x_{\ell})\right)
\end{equation*}
è ricorsiva. In particolare, il \href{20250202173528-dominio_range_e_campo_di_una_classe_relazione.org}{dominio} di \(f\) è composto di tutti e soli gli \((x_{1},\dots,x_{\ell}) \in \N^{\ell}\) tali che:
\begin{enumerate}
\item \(\forall\, i\), \((x_{1},\dots,x_{\ell}) \in \dom g_{i}\);
\item \(\left(g_{1}(x_{1},\dots,x_{\ell}), \dots, g_{k}(x_{1},\dots,x_{\ell})\right) \in \dom h\).
\end{enumerate}
\subsection{Schema di Ricorsione di funzioni ricorsive}
\label{sec:org3e59b4d}
Sia \(k \in \N^{+}\). Se \(h:\N^{n +2}\to \N\) e \(g:\N^{ k}\to \N\) sono funzioni ricorsive allora la funzione \(f:\N^{k+1}\to \N\) definita dalle condizioni
\begin{equation*}
\begin{cases}
f(x_{1},\dots,x_{k},0) = g(x_{1},\dots,x_{k})\\
f(x_{1},\dots,x_{n},y+1) = h\left(x_{1},\dots,x_{k}, y, f(x_{1},\dots,x_{n},y)\right)
\end{cases}
\end{equation*}
è ricorsiva.

Notiamo che questa funzione esiste per il \href{20250207121906-teorema_di_ricorsione.org}{Teorema di Ricorsione}.

In particolare, il \href{20250202173528-dominio_range_e_campo_di_una_classe_relazione.org}{dominio} di \(f\) è composto di tutti e soli gli \((x_{1},\dots,x_{k},y) \in \N^{k+1}\) tali che
\begin{enumerate}
\item \((x_{1},\dots,x_{k}, 0) \in \dom g\);
\item \(\forall\, z< y\) si ha che \(\left(x_{1},\dots,x_{k}, z, f(x_{1},\dots,x_{k}, z)\right) \in\dom h\)
\end{enumerate}

Vedi la \href{20250601161456-generalizzazione_schema_di_ricorsione.org}{generalizzazione}.
\subsection{Schema di minimizzazione}
\label{sec:orge2a7c46}
Se \(h:\N^{k+1}\to \N\) è una funzione ricorsiva allora applicando l'\href{20250215151440-operatore_di_minimizzazione_non_limitato.org}{operatore} \(\mu\) si ottiene una funzione ricorsiva \(f:\N^{k}\to \N\):
\begin{equation*}
f(x_{1},\dots,x_{k}) = \minim{z}{h(x_{1},\dots,x_{k}, z) = 0}
\end{equation*}
che ha per \href{20250202173528-dominio_range_e_campo_di_una_classe_relazione.org}{dominio} tutti quei punti \((x_{1},\dots,x_{k}) \in \N^{k}\) per cui esiste \(z \in \N\) tale che
\begin{enumerate}
\item per ogni \(u\le z\), \((x_{1},\dots,x_{k}, u) \in \dom h\);
\item \(h(x_{1},\dots,x_{k},z) = 0\).
\end{enumerate}
\begin{oss}
Notiamo che se esiste un unico \(\tilde{z}\) tale che \(h(x_{1},\dots,x_{k}, \tilde{z}) = 0\), allora
\begin{equation*}
\minim{z}{h(x_{1},\dots,x_{k},z)=0} = \tilde{z}
\end{equation*}
\end{oss}
\begin{oss}
Ogni \href{20250215141024-funzioni_primitive_ricordive.org}{funzione ricorsiva primitiva} è ricorsiva: \(\mathcal{P} \subsetneqq \mathcal{R}\).
\end{oss}
\section{Definizione equivalente di funzioni ricorsive}
\label{sec:orgbcc7f2e}
Sia \(\mathcal{F}\) la più piccola famiglia di funzioni \(f:\N^{k}\to \N\) contenente le funzioni \(U_{i}^{k}, +,\cdot,\chi_{\le}\) e chiusa per composizione e applicazioni dell'\href{20250215151440-operatore_di_minimizzazione_non_limitato.org}{operatore di minimizzazione} a funzioni \href{20250213105339-funzione_parziale.org}{totali}.
Allora
\begin{equation*}
\mathcal{F} = \mathcal{R}.
\end{equation*}

La dimostrazione si articola tramite una serie di lemmi.

In questa sezione si identificano i \href{20250131155822-operazioni_insiemistiche_tra_classi_mk.org}{sottinsiemi} di \(\N^{k}\) (vedi \href{20250202130045-insieme_dei_numeri_naturali_mk.org}{Insieme dei numeri naturali MK}) con i \href{20250131103317-formula_del_prim_ordine.org}{predicati} \(k\)-ari (ovvero con \(k\) \href{20250131103429-variabile_libera_di_una_formula.org}{variabili libere}), per mezzo degli \href{20250131122913-soddisfazione_di_una_formula.org}{insiemi di verità}.

Scriveremo indifferentemente \((x_{1},\dots,x_{k}) \in P\) oppure \(P(x_{1},\dots,x_{k})\) per dire che \(\N\vDash P(x_{1},\dots,x_{k})\).
\subsection{Lemma 1}
\label{sec:org4447d08}
Le funzioni \(\operatorname{sgn}\) e \(\overline{\operatorname{sgn}}\) appartengono ad \(\mathcal{F}\). (Vedi ``\href{20250215141024-funzioni_primitive_ricordive.org}{Esempi di funzioni primitive ricorsive}'' per la definizione di queste funzioni).
\subsection{Lemma 2}
\label{sec:orgdad26cf}
La collezione dei predicati le cui funzioni caratteristiche sono in \(\mathcal{F}\) (ovvero gli \(\mathcal{F}\)-predicati) è chiusa per intersezioni, unioni e complementi. In particolare \(\le,\ge,=,\neq,<,>\) sono \(\mathcal{F}\)-predicati.
\subsection{Lemma 3}
\label{sec:orgea4f658}
Le funzioni costanti \(c_{n}\) e la funzione successore \(S\) appartengono ad \(\mathcal{F}\).
\subsection{Lemma 4}
\label{sec:org22ef937}
Gli \(\mathcal{F}\)-predicati sono chusi per quantificazioni limitate. Se \(P \subseteq \N^{k+1}\) è un \(\mathcal{F}\)-predicato allora lo sono anche \(Q_{1},Q_{1} \subseteq \N^{k+1}\) definiti da:
\begin{align*}
Q_{1}(\bm{x},y)\quad &\iff\quad \exists\,z\le y\ P(\bm{x},y)\\
Q_{2}(\bm{x},y)\quad &\iff\quad \forall\,z\le y\ P(\bm{x},y).
\end{align*}
\subsection{Lemma 5}
\label{sec:org1595a5b}
Le seguenti funzioni appartengono ad \(\mathcal{F}\):
\begin{itemize}
\item \(\bm{J}, (\cdot)_{0}, (\cdot)_{1}\) (vedi ``\href{20250215151413-biiezione_canonica_tra_n_e_n2.org}{Biiezione canonica tra N e prodotti cartesiani di N}'');
\item \(\operatorname{Res}\) (vedi ``\href{20250215171731-quoziente_e_resto_sono_funzioni_ricorsive_primitive.org}{Quoziente, resto, MCD e mcm sono funzioni ricorsive primitive}'' per una definizione);
\item \(\beta,\ell,\godeldec{\cdot,\cdot}\) (vedi ``\href{20250531110737-codifica_delle_sequenze_finite_tramite_beta_di_godel.org}{Codifica delle sequenze finite tramite beta di Godel}'')
\end{itemize}
\subsection{Lemma 6}
\label{sec:org2a4c9e8}
Le \href{20250215141024-funzioni_primitive_ricordive.org}{funzioni primitive ricorsive} sono in \(\mathcal{F}\): \(\mathcal{P} \subseteq \mathcal{F}\). In particolare, i \href{20250216174510-insieme_ricorsivo_primitivo.org}{predicati primitivi ricorsivi} sono \(\mathcal{F}\)-predicati.
\end{document}
