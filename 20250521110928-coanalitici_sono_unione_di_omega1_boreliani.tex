% Created 2026-02-07 Sat 19:30
% Intended LaTeX compiler: pdflatex
\documentclass[10pt]{article}
%% CREATO CON ORG - EMACS
\newcommand{\use}[2][]{\usepackage[#1]{#2}}
% PACCHETTI FONDAMENTLAI
\use[utf8]{inputenc}
\use[T1]{fontenc}
\use{graphicx}
\use{longtable}
\use{wrapfig}
\use{rotating}
\use[normalem]{ulem}
\use{amsmath}
\use{amsthm}
\use{amssymb}

\use{eucal} % Cambia mathcal{...}

\use{capt-of}
\use[italian]{babel}
\use[babel]{csquotes}
% bib la TEX lo carica in automatico org-cite
\use{microtype}
\use{lmodern}
\use{subfig} % sottofigure
\use{multicol} % due colonne
\use{lipsum} % lorem ipsum
\use{color} % colori in latex
\use{parskip} % rimuove l'indentazione dei nuovi paragrafi %% Add parbox=false to all new tcolorbox
\use{centernot}
\use[outline]{contour}\contourlength{3pt}
\use{fancyhdr}
\use{layout}
\use[most]{tcolorbox} % Riquadri colorati
\use{ifthen} % IFTHEN
\use{geometry}

% pacchetti matematica
\use{yhmath}
\use{dsfont}
\use{mathrsfs}
\use{cancel} % semplificare
\use{polynom} %divisione tra polinomi
\use{forest} % grafi ad albero
\use{booktabs} % tabelle
\use{commath} %simboli e differenziali
\use{bm} %bold
\use[fulladjust]{marginnote} %to use marginnote for date notes
\use{arrayjobx}%array
\use[intlimits]{empheq} % Riquadri colorati attorno alle equazioni
\use{mathtools}
\use{circuitikz} % Disegnare i circuiti
\use{mathtools}
\use{stmaryrd} % [[ \llbracket ]] \rrbracket
\use{bussproofs} % dimostrazioni

%%%%%%%%%%%%%


%%%% QUIVER
\newcommand{\duepunti}{\,\mathchar\numexpr"6000+`:\relax\,}
% A TikZ style for curved arrows of a fixed height, due to AndréC.
\tikzset{curve/.style={settings={#1},to path={(\tikztostart)
    .. controls ($(\tikztostart)!\pv{pos}!(\tikztotarget)!\pv{height}!270:(\tikztotarget)$)
    and ($(\tikztostart)!1-\pv{pos}!(\tikztotarget)!\pv{height}!270:(\tikztotarget)$)
    .. (\tikztotarget)\tikztonodes}},
    settings/.code={\tikzset{quiver/.cd,#1}
        \def\pv##1{\pgfkeysvalueof{/tikz/quiver/##1}}},
    quiver/.cd,pos/.initial=0.35,height/.initial=0}

% TikZ arrowhead/tail styles.
\tikzset{tail reversed/.code={\pgfsetarrowsstart{tikzcd to}}}
\tikzset{2tail/.code={\pgfsetarrowsstart{Implies[reversed]}}}
\tikzset{2tail reversed/.code={\pgfsetarrowsstart{Implies}}}
% TikZ arrow styles.
\tikzset{no body/.style={/tikz/dash pattern=on 0 off 1mm}}
%%%%%%%%%%


%% DEFINIZIONI COMANDI MATEMATICI
\let\sin\relax %TOGLIE LA DEFINIZIONE SU "\sin"

% cambia la definizione di empty set
% ---
\let\oldemptyset\emptyset
% ---
% \let\emptyset\varnothing
% ---
% \let\emptyset\relax
% \newcommand{\emptyset}{\text{\textnormal{\O}}}
% ---

\DeclareMathOperator{\bounded}{bd}
\DeclareMathOperator{\sin}{sen}
\DeclareMathOperator{\epi}{Epi}
\DeclareMathOperator{\cl}{cl}
\DeclareMathOperator{\graph}{graph}
\DeclareMathOperator{\arcsec}{arcsec}
\DeclareMathOperator{\arccot}{arccot}
\DeclareMathOperator{\arccsc}{arccsc}
\DeclareMathOperator{\spettro}{Spettro}
\DeclareMathOperator{\nulls}{nullspace}
\DeclareMathOperator{\dom}{dom}
\DeclareMathOperator{\ar}{ar}
\DeclareMathOperator{\const}{Const}
\DeclareMathOperator{\fun}{Fun}
\DeclareMathOperator{\rel}{Rel}
\DeclareMathOperator{\altezza}{ht}
\let\det\relax %TOGLIE LA DEFINIZIONE SU "\det"
\DeclareMathOperator{\det}{det}
\DeclareMathOperator{\End}{End}
\DeclareMathOperator{\gl}{GL}
\def\Id{\mathrm{Id}}
\def\id{\mathrm{id}}
\DeclareMathOperator{\I}{\mathds{1}}
\DeclareMathOperator{\II}{II}
\DeclareMathOperator{\rank}{rank}
\DeclareMathOperator{\tr}{tr}
\DeclareMathOperator{\tc}{t.c.}
\DeclareMathOperator{\T}{T}
\DeclareMathOperator{\var}{Var}
\DeclareMathOperator{\cov}{Cov}
\DeclareMathOperator{\st}{st}
\DeclareMathOperator{\mon}{Mon}
\newcommand{\card}[1]{\left\vert #1 \right\vert}
\newcommand{\trasposta}[1]{\prescript{\text{T}}{}{#1}}
\newcommand{\1}{\mathds{1}}
\newcommand{\R}{\mathds{R}}
\newcommand{\diesis}{\#}
\newcommand{\bemolle}{\flat}
\newcommand{\nonstandard}[1]{\prescript{*}{}{#1}}
\newcommand{\starR}{\nonstandard{\R}}
\newcommand{\borel}{\mathscr{B}}
\newcommand{\lebesgue}[1]{\mathscr{L}\left(#1\right)}
\newcommand{\media}{\mathds{E}}
\newcommand{\K}{\mathds{K}}
\newcommand{\A}{\mathds{A}}
\newcommand{\Q}{\mathds{Q}}
\newcommand{\N}{\mathds{N}}
\newcommand{\C}{\mathds{C}}
\newcommand{\Z}{\mathds{Z}}
\newcommand{\qo}{\hspace{1em}\text{q.o.}\,}
\renewcommand{\tilde}[1]{\widetilde{#1}}
\renewcommand{\parallel}{\mathrel{/\mkern-5mu/}}
\newcommand{\parti}[2][]{\wp_{#1}(#2)}
\newcommand{\diff}[1]{\operatorname{d}_{#1}}
\let\oldvec\vec
\renewcommand{\vec}[1]{\overrightarrow{\vphantom{i}#1}}
\newcommand{\floor}[1]{\left\lfloor #1 \right\rfloor}
\newcommand{\cat}[1]{\mathbf{#1}}
\newcommand{\dfreccia}[1]{\xrightarrow{\ #1 \ }}
\newcommand{\sfreccia}[1]{\xleftarrow{\ #1 \ }}
\newcommand{\formalsum}[2]{{\sum_{#1}^{#2}}{\vphantom{\sum}}'}
\newcommand{\minim}[2]{\mu_{#1}\, \left(#2\right)}
\newcommand{\concat}{\null^{\frown}} % concatenazione di stringe
\newcommand{\godelcode}[1]{\langle\!\langle #1 \rangle\!\rangle}
\newcommand{\godeldec}[1]{(\!(#1)\!)}
\newcommand{\termcode}[1]{\ulcorner #1\urcorner}
\newcommand{\partialto}{\dashrightarrow}
\newcommand{\restricted}{\upharpoonright}
\newcommand{\embeds}{\precsim}
\newcommand{\surjects}{\twoheadrightarrow}
\newcommand{\equipotenti}{\asymp}
%% \newcommand{\dotplus}{\mathbin{\dot{+}}} %% A quanto pare esiste già
\newcommand{\bigdot}{\mathbin{\boldsymbol{\cdot}}}
\newcommand{\dotexp}[1]{^{.#1}}
\newcommand{\conv}{\mathbin{*}}
\newcommand{\convolution}[2]{(#1\conv #2)}
\newcommand{\nil}{\mathfrak{N}}
\newcommand{\divisore}{\mathrel{|}}
\newcommand{\simplesso}[1]{\mathrm{e}_{#1}}

\renewcommand{\iff}{\mathrel{\longleftrightarrow}} %% Notazione Logica.
\newcommand{\oldiff}{\mathrel{\Longleftrightarrow}}
\renewcommand{\implies}{\mathrel{\rightarrow}} %% Notazione Logica
\newcommand{\oldimplies}{\mathrel{\Longrightarrow}}
\renewcommand{\impliedby}{\mathrel{\leftarrow}} %% Notazione Logica
\newcommand{\oldimpliedby}{\mathrel{\Longleftarrow}}

\newcommand{\IFF}{\quad\Longleftrightarrow\quad}
\newcommand{\IMPLICA}{\quad\Longrightarrow\quad}


\renewcommand{\descriptionlabel}[1]{\hspace{\labelsep}\normalfont #1} % remove bold from description


%% Definizione di Divergenza di K-L

\DeclarePairedDelimiterX{\infdivx}[2]{(}{)}{%
  #1\;\delimsize\|\;#2%
}
\newcommand{\kldiv}{D_{KL}\infdivx}

%% Definizione di \dotminus

\makeatletter
\newcommand{\dotminus}{\mathbin{\text{\@dotminus}}}

\newcommand{\@dotminus}{%
  \ooalign{\hidewidth\raise1ex\hbox{.}\hidewidth\cr$\m@th-$\cr}%
}
\makeatother

%tramite i prossimi due comandi posso decidere come scrivere i logaritmi naturali in tutti i documenti: ho infatti eliminato qualsiasi differenza tra "ln" e "log": se si vuole qualcosa di diverso bisogna inserire manualmente il tutto
\let\ln\relax
\DeclareMathOperator{\ln}{ln}
\let\log\relax
\DeclareMathOperator{\log}{log}
%%%%%%

%% NUOVI COMANDI
\newcommand{\straniero}[1]{\textit{#1}} %parole straniere
\newcommand{\titolo}[1]{\textsc{#1}} %titoli
\newcommand{\qedd}{\tag*{$\blacksquare$}} %qed per ambienti matemastici
\renewcommand{\qedsymbol}{$\blacksquare$} %modifica colore qed
\newcommand{\ooverline}[1]{\overline{\overline{#1}}}
\newcommand{\circoletto}[1]{\left(#1\right)^{\text{o}}}
%
\newcommand{\qmatrice}[1]{\begin{pmatrix}
#1_{11} & \cdots & #1_{1n}\\
\vdots & \ddots & \vdots \\
#1_{m1} & \cdots & #1_{mn}
\end{pmatrix}}
%
\newcommand{\parentesi}[2]{%
\underset{#1}{\underbrace{#2}}%
}
%
\newcommand{\norma}[1]{% Norma
\left\lVert#1\right\rVert%
}
\newcommand{\scalare}[2]{% Scalare
\left\langle #1, #2\right\rangle
}
%%%%%

%% RESTRIZIONI
\newcommand{\referenze}[2]{
        \phantomsection{}#2\textsuperscript{\textcolor{blue}{\textbf{#1}}}
}

\let\restriction\relax

\def\restriction#1#2{\mathchoice
              {\setbox1\hbox{${\displaystyle #1}_{\scriptstyle #2}$}
              \restrictionaux{#1}{#2}}
              {\setbox1\hbox{${\textstyle #1}_{\scriptstyle #2}$}
              \restrictionaux{#1}{#2}}
              {\setbox1\hbox{${\scriptstyle #1}_{\scriptscriptstyle #2}$}
              \restrictionaux{#1}{#2}}
              {\setbox1\hbox{${\scriptscriptstyle #1}_{\scriptscriptstyle #2}$}
              \restrictionaux{#1}{#2}}}
\def\restrictionaux#1#2{{#1\,\smash{\vrule height .8\ht1 depth .85\dp1}}_{\,#2}}
%%%%%%%%%%%

%%% FORMATTAZIONE FOOTNOTEMARK

\def\footnotemarkformatting#1{[#1]}
\renewcommand{\thefootnote}{\footnotemarkformatting{\arabic{footnote}}}

%% SEZIONE GRAFICA
\use{tikz}
\usetikzlibrary{matrix, patterns, calc, decorations.pathreplacing, hobby, decorations.markings, decorations.pathmorphing, babel}
\use{tikz-3dplot}
\use{mathrsfs} %per geogebra
\use{tikz-cd}
\tikzset
{
  %surface/.style={fill=black!10, shading=ball,fill opacity=0.4},
  plane/.style={black,pattern=north east lines},
  curve/.style={black,line width=0.5mm},
  dritto/.style={decoration={markings,mark=at position 0.5 with {\arrow{Stealth}}}, postaction=decorate},
  rovescio/.style={decoration={markings,mark=at position 0.5 with {\arrow{Stealth[reversed]}}}, postaction=decorate}
}
\use{pgfplots} % stampare le funzioni
        \pgfplotsset{/pgf/number format/use comma,compat=1.15}
        %\pgfplotsset{compat=1.15} %per geogebra
        \usepgfplotslibrary{fillbetween, polar}
%%%%%%

%% CITAZIONI
\use{lineno}

\newcommand{\citazione}[1]{%
  \begin{quotation}
  \begin{linenumbers}
  \modulolinenumbers[5]
  \begingroup
  \setlength{\parindent}{0cm}
  \noindent #1
  \endgroup
  \end{linenumbers}
  \end{quotation}\setcounter{linenumber}{1}
  }
%%%%%%

%%%%%%%%%%%%%%%%%%%%%%%%%%%%%%%%%%%%%%%%%%%%
%%%%%%%%%%%%%%%%%%%%%%%%%%%%%%%%%%%%%%%%%%%%

%% AMS THM

\theoremstyle{definition}% default
\newtheorem{thm}{Teorema}[section]
\newtheorem{lem}[thm]{Lemma}
\newtheorem{prop}[thm]{Proposizione}
\newtheorem{cor}[thm]{Corollario}
\newtheorem{esempio}[thm]{Esempio}
\theoremstyle{plain}
\newtheorem{definizione}[thm]{Definizione}
\theoremstyle{remark}
\newtheorem*{oss}{Osservazione}


%%%%%%%%%%%%%%%%%%%%%%%%%%%%%%%%%%%%%%%%%%%%
%%%%%%%%%%%%%%%%%%%%%%%%%%%%%%%%%%%%%%%%%%%%

\use{hyperref}
\hypersetup{%
        pdfauthor={Davide Peccioli},
        pdfsubject={},
        allcolors=black,
        citecolor=black,
%	colorlinks=true,
        bookmarksopen=true}
\setcounter{secnumdepth}{0} % rimuove i numeri di sezione senza rimuovere le ref
\renewcommand{\href}[2]{\textcolor{blue}{#2}} % disabilita il comando href
\use{enotez} %
\setenotez{%
 mark-format = \footnotemarkformatting % Mette i numeri tra parentesi quadre%
}\let\footnote=\endnote % rende tutte le note a pié pagina come delle note a fine file 


\let\olddocument\document % modifico l'ambiende documenti per non dover stampare \printendnote
\let\oldenddocument\enddocument
\renewenvironment{document}%
{%
  \olddocument
}{%
  \printendnotes\oldenddocument
}
\renewcommand{\thethm}{\arabic{thm}}

\usepackage[hyperref]{biblatex}
\addbibresource{~/Documents/org/roam/bib/master.bib}
\author{Davide Peccioli}
\date{\today}
\title{Coanalitici sono unione di omega1 boreliani}
\begin{document}

\section{Esercizio 4}
\label{sec:org6bb096c}

Prove the following theorem:
\begin{quote}
Let \(X\) be a Polish space. Then every \(A \in \bm{\Pi}^1_1(X)\) can be written as \(A = \bigcup_{\xi < \omega_1} A_\xi\), where \(A_\xi\) is Borel for every \(\xi < \omega_1\).
\end{quote}
by completing the details of the following steps:

\begin{enumerate}
\item First prove the theorem for \(X = \mathrm{LO}\) and \(A = \mathrm{WO}\) as follows:
\begin{itemize}
\item Given \(\omega \leq \xi < \omega_1\), let \(\mathrm{WO}_\xi\) be the set of codes for well-orders of \(\omega\) with order type \(\leq \xi\). Show that each \(\mathrm{WO}_\xi\) is analytic.
\item Argue that there is a Borel set \(A_\xi\) such that \(\mathrm{WO}_\xi \subseteq A_\xi \subseteq \mathrm{WO}\).

\emph{Optional}: Show that \(\mathrm{WO}_\xi\) itself is Borel by showing that its complement is analytic as well.
\item Conclude that \(\mathrm{WO} = \bigcup_{\xi < \omega_1} A_\xi\).
\end{itemize}

\item Use the fact that \(\mathrm{WO}\) is \(\bm{\Pi}^1_1\)-complete to prove the theorem for \(X = \omega^\omega\) and an arbitrary \(A \in \bm{\Pi}^1_1(\omega^\omega)\).

\item Use the Borel isomorphism theorem for Polish spaces to transfer the result to an arbitrary uncountable Polish space \(X\).

\item What happens if \(X\) is a countable Polish space?
\end{enumerate}
\subsection{Soluzione}
\label{sec:orge0a92be}

\subsubsection{Parte a.}
\label{sec:org2c25ba1}

Si consideri lo spazio polacco \(X\coloneqq\mathrm{LO} \subseteq 2^{\omega\times\omega}\) e si adotti la notazione dell'Esempio 3.1.8: l'insieme \(\mathrm{NWO}\) è analitico, mentre l'insieme \(\mathrm{WO}\) è coanalitico. È dunque possibile porre
\begin{equation*}
A\coloneqq \mathrm{WO} \in \bm{\Pi}_{1}^{1}(\mathrm{LO}).
\end{equation*}


\begin{itemize}
\item Sia \(\omega\le\xi< \omega_{1}\) fissato. Sia \(\mathrm{WO}_{\xi}\) l'insieme di tutti gli elementi di \(\mathrm{WO}\) con order type \(\le \xi\): un buon ordine \(\langle A, \preceq\rangle\) ha order type \(\xi'\) se e solo se esiste una biiezione \(f:A\to \xi'\) tale che, per ogni \(a,b \in A\)
\begin{equation*}
  	a\preceq b\quad \iff\quad f(a)< f(b)
\end{equation*}

Dunque \(x \in \mathrm{WO}\) ha order type \(\xi'\) se e solo se esiste una funzione biiettiva \(f:\omega \to\xi'\) tale che per ogni \(m,n \in \omega\):
\begin{equation*}
  	x(m,n) = 1\quad\iff\quad f(m)< f(n)
\end{equation*}

Si consideri quindi \(\mathrm{WO}^{=\xi'}\) l'insieme di tutti gli elementi di \(\mathrm{WO}\) con order type \uline{esattamente} \(\xi'\): per ogni \(x \in \mathrm{WO}\):
\begin{equation*}
  	x \in \mathrm{WO}^{=\xi'} \quad \iff \quad\exists\, f \in (\xi')^{\omega}\text{ biiettiva}\ \forall\, m,n \in\omega\ \left(x(m,n)=1\,\iff\, f(m)<f(n)\right).
\end{equation*}

Inoltre, se \(x \in \mathrm{LO}\), la condizione di destra garantisce che \(x \in \mathrm{WO}\), poiché la biiezione \(f\) è un isomorfismo di ordini e \(\xi'\) è ben ordinato (in quanto ordinale). Pertanto, per ogni \(x \in \mathrm{LO}\):
\begin{equation*}
  	x \in \mathrm{WO}^{=\xi'} \quad \iff \quad\exists\, f \in (\xi')^{\omega}\text{ biiettiva}\ \forall\, m,n \in\omega\ \left(x(m,n)=1\,\iff\, f(m)<f(n)\right).
\end{equation*}

\uline{Osservazione 1}: per ogni \(\xi' < \omega_{1}=\omega^{+}\), si ha che \(\card{\xi} =\aleph_{0}\), e pertanto \(\xi'\) è numerabile.

\uline{Osservazione 2}: per ogni \(\xi'<\omega_{1}\), \(\xi'\) è uno spazio polacco; infatti ogni ordinale numerabile è omeomorfo ad un sottoinsieme chiuso e numerabile di \(\R\) e pertanto è polacco. Quindi \((\xi')^{\omega}\) è ancora uno spazio polacco.

Si definisce quindi:
\begin{equation*}
  	A_{m,n} \coloneqq \set{(x, f) \in \mathrm{LO}\times (\xi')^{\omega }\mid \left(x(m,n)=1 \,\iff\, f(m)<f(n)\right) \,\land\, f\text{ biiettiva}}
\end{equation*}
Questo è un insieme \(\bm{{\operatorname{Bor}}}\left(\mathrm{LO}\times(\xi')^{\omega}\right)\), poiché tutte le condizioni sono Boreliane:
\begin{align*}
  	(x,f) \in A_{m,n}\quad \iff\quad &\left[x(m,n)=1 \,\iff\, f(m)<f(n)\right] \,\land\\
  	&\land\, \left[\forall\, \lambda,\mu \in \omega\ \left(f(\lambda)= f(\mu)\right) \,\implies\,(\lambda = \mu)\right] \,\land\\
  	&\land\, \left[\forall\,\lambda<\xi'\ \exists\, k \in \omega\ \left(f(k)=\lambda\right)\right]
\end{align*}
Le quantificazioni sono tutte numerabili in virtù dell'Osservazione 1.

Pertanto
\begin{equation*}
A_{m,n} \in \bm{{\operatorname{Bor}}}\left(\mathrm{LO}\times(\xi')^{\omega}\right) \subseteq \bm{\Sigma}_{1}^{1}\left(\mathrm{LO}\times(\xi')^{\omega}\right),
\end{equation*}
e dunque anche \(\bigcap_{m,n \in \omega} A_{m,n}\) è \(\bm{\Sigma}_{1}^{1}\left(\mathrm{LO}\times(\xi')^{\omega}\right)\).

Definita
\begin{equation*}
  	\pi_{\mathrm{LO}}: \mathrm{LO} \times (\xi')^{\omega} \to \mathrm{LO}
\end{equation*}
la proiezione sul primo fattore, allora
\begin{equation*}
  	\mathrm{WO}^{=\xi'} = \pi_{\mathrm{LO}}\left(\bigcap_{m,n \in \omega} A_{m,n}\right).
\end{equation*}
Dunque applicando la Proposizione 3.1.5 (per l'osservazione precedente \((\xi')^{\omega}\) è Polacco) si ottiene che \(\mathrm{WO}^{=\xi'}\) è \(\bm{\Sigma}_{1}^{1}(\mathrm{LO})\).

Inoltre,
\begin{equation*}
  	\mathrm{WO}_{\xi} = \bigcup_{\xi'\le \xi} \mathrm{WO}^{=\xi'}
\end{equation*}
e pertanto \uline{questo dimostra che \(\mathrm{WO}_{\xi} \in \bm{\Sigma}_{1}^{1}(\mathrm{LO})\)}, poiché \(\bm{\Sigma}_{1}^{1}\) è chiuso per unioni numerabili (per la Proposizione 3.1.5).

\item Sia \(\omega\le\xi< \omega_{1}\) fissato. È possibile applicare il Teorema 3.2.1 a \(\mathrm{WO}_{\xi}\)  e \(\mathrm{NWO}\) (infatti sono entrambi analitici e \(\mathrm{WO}_{\xi} \cap \mathrm{NWO} \subseteq \mathrm{WO} \cap \mathrm{NWO} =\emptyset\)): esiste \(A_{\xi}\) \uline{Boreliano} tale che:
\begin{equation*}
  \mathrm{WO}_{\xi} \subseteq A_{\xi}, \qquad A_{\xi} \cap \mathrm{NWO} = \emptyset
\end{equation*}
Siccome \(\mathrm{NWO} = X\setminus\mathrm{WO}\) si ha che \(A_{\xi} \subseteq \mathrm{WO}\):
\begin{equation*}
  \mathrm{WO}_{\xi} \subseteq A_{\xi} \subseteq \mathrm{WO}.
\end{equation*}

Per ogni \(\xi<\omega\) si pone \(A_{\xi} =\emptyset \in \bm{{\operatorname{Bor}}}(\mathrm{LO})\).
\item Vale la seguente uguaglianza: \(\mathrm{WO} = \bigcup_{\omega\le \xi<\omega_{1}} \mathrm{WO}_{\xi}\). (\(\supseteq\)): è ovvio, poiché per ogni \(\omega\le\xi<\omega_{1}\) si ha \(\mathrm{WO}_{\xi} \subseteq \mathrm{WO}\). (\(\subseteq\)): ciascun buon ordine lineare ha order type minore di \(\omega_{1}\), e pertanto se \(x \in \mathrm{WO}\) allora esiste \(\xi<\omega_{1}\) tale che \(x \in \mathrm{WO}_{\xi}\).

Pertanto si ha che
\begin{equation*}
  	\mathrm{WO} = \bigcup_{\omega\le \xi<\omega_{1}} \mathrm{WO}_{\xi} \subseteq \bigcup_{\omega\le \xi<\omega_{1}} A_{\xi} = \bigcup_{\xi<\omega_{1}} A_{\xi}
\end{equation*}
ed inoltre, per ogni \(\xi<\omega_{1}\), \(A_{\xi} \subseteq \mathrm{WO}\) e dunque
\begin{equation*}
  	\bigcup_{\xi<\omega_{1}} A_{\xi} \subseteq \mathrm{WO}
\end{equation*}

Per doppia inclusione si ha proprio \(\mathrm{WO} = \bigcup_{\xi<\omega_{1}} A_{\xi}\).
\end{itemize}
\subsubsection{Parte b.}
\label{sec:org42f3640}

Sia \(X\coloneqq\omega^{\omega}\) e \(A \in \bm{\Pi}_{1}^{1}(X)\).

Siccome \(\mathrm{WO}\) è \(\bm{\Pi}_{1}^{1}\)-completo, allora esiste una funzione continua
\begin{equation*}
f: \omega^{\omega}\to \mathrm{LO}
\end{equation*}
tale che \(f^{-1}(\mathrm{WO}) = A\).

Per il punto precedente è possibile scrivere \(\mathrm{WO} = \bigcup_{\xi<\omega_{1}} B_{\xi}\) con \(B_{\xi} \in \bm{{\operatorname{Bor}}}(\mathrm{LO})\), e quindi
\begin{equation*}
A = f^{-1}(\mathrm{WO}) = f^{-1}\left(\bigcup_{\xi<\omega_{1}} B_{\xi}\right) = \bigcup_{\xi<\omega_{1}} f^{-1}(B_{\xi}).
\end{equation*}
Posto \(A_{\xi}\coloneqq f^{-1}(B_{\xi})\), si ha che \(A_{\xi} \in \bm{{\operatorname{Bor}}}(X)\) poiché \(B_{\xi} \in \bm{{\operatorname{Bor}}}(\mathrm{LO})\) e \(f\) continua. Pertanto
\begin{equation*}
A=\bigcup_{\xi<\omega_{1}} A_{\xi}
\end{equation*}
con \(A_{\xi}\) boreliani.
\subsubsection{Parte c.}
\label{sec:orgd23ce07}

Sia \(X\) uno spazio polacco non numerabile, e sia \(A \in \bm{\Pi}_{1}^1(X)\). Per il Teorema 3.2.9 esiste un isomorfismo Boreliano:
\begin{equation*}
F: \omega^{\omega}\to X
\end{equation*}

In particolare \(B\coloneqq F^{-1}(A) \in \bm{\Pi}_{1}^{1}(X)\) per il Corollario 3.1.16, poiché \(F\) è Boreliana. Per il punto precedente,
\begin{equation*}
B=\bigcup_{\xi<\omega_{1}} B_{\xi}
\end{equation*}
con \(B_{\xi} \in \bm{{\operatorname{Bor}}}(\omega^{\omega})\)

Siccome \(F\) è una biiezione, allora \(A=F(B)\):
\begin{equation*}
A= F(B) = F\left(\bigcup_{\xi<\omega_{1}} B_{\xi}\right) = \bigcup_{\xi<\omega_{1}} F(B_{\xi}).
\end{equation*}

Posto ora \(A_{\xi} \coloneqq F(B_{\xi})\), questi sono Boreliani per il Corollario 3.2.7, poiché \(F\) Boreliana iniettiva e \(B_{\xi}\) Boreliano.
\subsubsection{Parte d.}
\label{sec:orgde9dfd9}

Se \(X\) è numerabile allora il teorema è banale: ogni sottoinsieme di \(X\) è unione numerabile di singoletti, che sono chiusi, e pertanto ogni sottoinsieme di \(X\) è un Boreliano.\qed
\end{document}
