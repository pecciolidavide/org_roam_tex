% Created 2026-02-07 Sat 19:30
% Intended LaTeX compiler: pdflatex
\documentclass[10pt]{article}
%% CREATO CON ORG - EMACS
\newcommand{\use}[2][]{\usepackage[#1]{#2}}
% PACCHETTI FONDAMENTLAI
\use[utf8]{inputenc}
\use[T1]{fontenc}
\use{graphicx}
\use{longtable}
\use{wrapfig}
\use{rotating}
\use[normalem]{ulem}
\use{amsmath}
\use{amsthm}
\use{amssymb}

\use{eucal} % Cambia mathcal{...}

\use{capt-of}
\use[italian]{babel}
\use[babel]{csquotes}
% bib la TEX lo carica in automatico org-cite
\use{microtype}
\use{lmodern}
\use{subfig} % sottofigure
\use{multicol} % due colonne
\use{lipsum} % lorem ipsum
\use{color} % colori in latex
\use{parskip} % rimuove l'indentazione dei nuovi paragrafi %% Add parbox=false to all new tcolorbox
\use{centernot}
\use[outline]{contour}\contourlength{3pt}
\use{fancyhdr}
\use{layout}
\use[most]{tcolorbox} % Riquadri colorati
\use{ifthen} % IFTHEN
\use{geometry}

% pacchetti matematica
\use{yhmath}
\use{dsfont}
\use{mathrsfs}
\use{cancel} % semplificare
\use{polynom} %divisione tra polinomi
\use{forest} % grafi ad albero
\use{booktabs} % tabelle
\use{commath} %simboli e differenziali
\use{bm} %bold
\use[fulladjust]{marginnote} %to use marginnote for date notes
\use{arrayjobx}%array
\use[intlimits]{empheq} % Riquadri colorati attorno alle equazioni
\use{mathtools}
\use{circuitikz} % Disegnare i circuiti
\use{mathtools}
\use{stmaryrd} % [[ \llbracket ]] \rrbracket
\use{bussproofs} % dimostrazioni

%%%%%%%%%%%%%


%%%% QUIVER
\newcommand{\duepunti}{\,\mathchar\numexpr"6000+`:\relax\,}
% A TikZ style for curved arrows of a fixed height, due to AndréC.
\tikzset{curve/.style={settings={#1},to path={(\tikztostart)
    .. controls ($(\tikztostart)!\pv{pos}!(\tikztotarget)!\pv{height}!270:(\tikztotarget)$)
    and ($(\tikztostart)!1-\pv{pos}!(\tikztotarget)!\pv{height}!270:(\tikztotarget)$)
    .. (\tikztotarget)\tikztonodes}},
    settings/.code={\tikzset{quiver/.cd,#1}
        \def\pv##1{\pgfkeysvalueof{/tikz/quiver/##1}}},
    quiver/.cd,pos/.initial=0.35,height/.initial=0}

% TikZ arrowhead/tail styles.
\tikzset{tail reversed/.code={\pgfsetarrowsstart{tikzcd to}}}
\tikzset{2tail/.code={\pgfsetarrowsstart{Implies[reversed]}}}
\tikzset{2tail reversed/.code={\pgfsetarrowsstart{Implies}}}
% TikZ arrow styles.
\tikzset{no body/.style={/tikz/dash pattern=on 0 off 1mm}}
%%%%%%%%%%


%% DEFINIZIONI COMANDI MATEMATICI
\let\sin\relax %TOGLIE LA DEFINIZIONE SU "\sin"

% cambia la definizione di empty set
% ---
\let\oldemptyset\emptyset
% ---
% \let\emptyset\varnothing
% ---
% \let\emptyset\relax
% \newcommand{\emptyset}{\text{\textnormal{\O}}}
% ---

\DeclareMathOperator{\bounded}{bd}
\DeclareMathOperator{\sin}{sen}
\DeclareMathOperator{\epi}{Epi}
\DeclareMathOperator{\cl}{cl}
\DeclareMathOperator{\graph}{graph}
\DeclareMathOperator{\arcsec}{arcsec}
\DeclareMathOperator{\arccot}{arccot}
\DeclareMathOperator{\arccsc}{arccsc}
\DeclareMathOperator{\spettro}{Spettro}
\DeclareMathOperator{\nulls}{nullspace}
\DeclareMathOperator{\dom}{dom}
\DeclareMathOperator{\ar}{ar}
\DeclareMathOperator{\const}{Const}
\DeclareMathOperator{\fun}{Fun}
\DeclareMathOperator{\rel}{Rel}
\DeclareMathOperator{\altezza}{ht}
\let\det\relax %TOGLIE LA DEFINIZIONE SU "\det"
\DeclareMathOperator{\det}{det}
\DeclareMathOperator{\End}{End}
\DeclareMathOperator{\gl}{GL}
\def\Id{\mathrm{Id}}
\def\id{\mathrm{id}}
\DeclareMathOperator{\I}{\mathds{1}}
\DeclareMathOperator{\II}{II}
\DeclareMathOperator{\rank}{rank}
\DeclareMathOperator{\tr}{tr}
\DeclareMathOperator{\tc}{t.c.}
\DeclareMathOperator{\T}{T}
\DeclareMathOperator{\var}{Var}
\DeclareMathOperator{\cov}{Cov}
\DeclareMathOperator{\st}{st}
\DeclareMathOperator{\mon}{Mon}
\newcommand{\card}[1]{\left\vert #1 \right\vert}
\newcommand{\trasposta}[1]{\prescript{\text{T}}{}{#1}}
\newcommand{\1}{\mathds{1}}
\newcommand{\R}{\mathds{R}}
\newcommand{\diesis}{\#}
\newcommand{\bemolle}{\flat}
\newcommand{\nonstandard}[1]{\prescript{*}{}{#1}}
\newcommand{\starR}{\nonstandard{\R}}
\newcommand{\borel}{\mathscr{B}}
\newcommand{\lebesgue}[1]{\mathscr{L}\left(#1\right)}
\newcommand{\media}{\mathds{E}}
\newcommand{\K}{\mathds{K}}
\newcommand{\A}{\mathds{A}}
\newcommand{\Q}{\mathds{Q}}
\newcommand{\N}{\mathds{N}}
\newcommand{\C}{\mathds{C}}
\newcommand{\Z}{\mathds{Z}}
\newcommand{\qo}{\hspace{1em}\text{q.o.}\,}
\renewcommand{\tilde}[1]{\widetilde{#1}}
\renewcommand{\parallel}{\mathrel{/\mkern-5mu/}}
\newcommand{\parti}[2][]{\wp_{#1}(#2)}
\newcommand{\diff}[1]{\operatorname{d}_{#1}}
\let\oldvec\vec
\renewcommand{\vec}[1]{\overrightarrow{\vphantom{i}#1}}
\newcommand{\floor}[1]{\left\lfloor #1 \right\rfloor}
\newcommand{\cat}[1]{\mathbf{#1}}
\newcommand{\dfreccia}[1]{\xrightarrow{\ #1 \ }}
\newcommand{\sfreccia}[1]{\xleftarrow{\ #1 \ }}
\newcommand{\formalsum}[2]{{\sum_{#1}^{#2}}{\vphantom{\sum}}'}
\newcommand{\minim}[2]{\mu_{#1}\, \left(#2\right)}
\newcommand{\concat}{\null^{\frown}} % concatenazione di stringe
\newcommand{\godelcode}[1]{\langle\!\langle #1 \rangle\!\rangle}
\newcommand{\godeldec}[1]{(\!(#1)\!)}
\newcommand{\termcode}[1]{\ulcorner #1\urcorner}
\newcommand{\partialto}{\dashrightarrow}
\newcommand{\restricted}{\upharpoonright}
\newcommand{\embeds}{\precsim}
\newcommand{\surjects}{\twoheadrightarrow}
\newcommand{\equipotenti}{\asymp}
%% \newcommand{\dotplus}{\mathbin{\dot{+}}} %% A quanto pare esiste già
\newcommand{\bigdot}{\mathbin{\boldsymbol{\cdot}}}
\newcommand{\dotexp}[1]{^{.#1}}
\newcommand{\conv}{\mathbin{*}}
\newcommand{\convolution}[2]{(#1\conv #2)}
\newcommand{\nil}{\mathfrak{N}}
\newcommand{\divisore}{\mathrel{|}}
\newcommand{\simplesso}[1]{\mathrm{e}_{#1}}

\renewcommand{\iff}{\mathrel{\longleftrightarrow}} %% Notazione Logica.
\newcommand{\oldiff}{\mathrel{\Longleftrightarrow}}
\renewcommand{\implies}{\mathrel{\rightarrow}} %% Notazione Logica
\newcommand{\oldimplies}{\mathrel{\Longrightarrow}}
\renewcommand{\impliedby}{\mathrel{\leftarrow}} %% Notazione Logica
\newcommand{\oldimpliedby}{\mathrel{\Longleftarrow}}

\newcommand{\IFF}{\quad\Longleftrightarrow\quad}
\newcommand{\IMPLICA}{\quad\Longrightarrow\quad}


\renewcommand{\descriptionlabel}[1]{\hspace{\labelsep}\normalfont #1} % remove bold from description


%% Definizione di Divergenza di K-L

\DeclarePairedDelimiterX{\infdivx}[2]{(}{)}{%
  #1\;\delimsize\|\;#2%
}
\newcommand{\kldiv}{D_{KL}\infdivx}

%% Definizione di \dotminus

\makeatletter
\newcommand{\dotminus}{\mathbin{\text{\@dotminus}}}

\newcommand{\@dotminus}{%
  \ooalign{\hidewidth\raise1ex\hbox{.}\hidewidth\cr$\m@th-$\cr}%
}
\makeatother

%tramite i prossimi due comandi posso decidere come scrivere i logaritmi naturali in tutti i documenti: ho infatti eliminato qualsiasi differenza tra "ln" e "log": se si vuole qualcosa di diverso bisogna inserire manualmente il tutto
\let\ln\relax
\DeclareMathOperator{\ln}{ln}
\let\log\relax
\DeclareMathOperator{\log}{log}
%%%%%%

%% NUOVI COMANDI
\newcommand{\straniero}[1]{\textit{#1}} %parole straniere
\newcommand{\titolo}[1]{\textsc{#1}} %titoli
\newcommand{\qedd}{\tag*{$\blacksquare$}} %qed per ambienti matemastici
\renewcommand{\qedsymbol}{$\blacksquare$} %modifica colore qed
\newcommand{\ooverline}[1]{\overline{\overline{#1}}}
\newcommand{\circoletto}[1]{\left(#1\right)^{\text{o}}}
%
\newcommand{\qmatrice}[1]{\begin{pmatrix}
#1_{11} & \cdots & #1_{1n}\\
\vdots & \ddots & \vdots \\
#1_{m1} & \cdots & #1_{mn}
\end{pmatrix}}
%
\newcommand{\parentesi}[2]{%
\underset{#1}{\underbrace{#2}}%
}
%
\newcommand{\norma}[1]{% Norma
\left\lVert#1\right\rVert%
}
\newcommand{\scalare}[2]{% Scalare
\left\langle #1, #2\right\rangle
}
%%%%%

%% RESTRIZIONI
\newcommand{\referenze}[2]{
        \phantomsection{}#2\textsuperscript{\textcolor{blue}{\textbf{#1}}}
}

\let\restriction\relax

\def\restriction#1#2{\mathchoice
              {\setbox1\hbox{${\displaystyle #1}_{\scriptstyle #2}$}
              \restrictionaux{#1}{#2}}
              {\setbox1\hbox{${\textstyle #1}_{\scriptstyle #2}$}
              \restrictionaux{#1}{#2}}
              {\setbox1\hbox{${\scriptstyle #1}_{\scriptscriptstyle #2}$}
              \restrictionaux{#1}{#2}}
              {\setbox1\hbox{${\scriptscriptstyle #1}_{\scriptscriptstyle #2}$}
              \restrictionaux{#1}{#2}}}
\def\restrictionaux#1#2{{#1\,\smash{\vrule height .8\ht1 depth .85\dp1}}_{\,#2}}
%%%%%%%%%%%

%%% FORMATTAZIONE FOOTNOTEMARK

\def\footnotemarkformatting#1{[#1]}
\renewcommand{\thefootnote}{\footnotemarkformatting{\arabic{footnote}}}

%% SEZIONE GRAFICA
\use{tikz}
\usetikzlibrary{matrix, patterns, calc, decorations.pathreplacing, hobby, decorations.markings, decorations.pathmorphing, babel}
\use{tikz-3dplot}
\use{mathrsfs} %per geogebra
\use{tikz-cd}
\tikzset
{
  %surface/.style={fill=black!10, shading=ball,fill opacity=0.4},
  plane/.style={black,pattern=north east lines},
  curve/.style={black,line width=0.5mm},
  dritto/.style={decoration={markings,mark=at position 0.5 with {\arrow{Stealth}}}, postaction=decorate},
  rovescio/.style={decoration={markings,mark=at position 0.5 with {\arrow{Stealth[reversed]}}}, postaction=decorate}
}
\use{pgfplots} % stampare le funzioni
        \pgfplotsset{/pgf/number format/use comma,compat=1.15}
        %\pgfplotsset{compat=1.15} %per geogebra
        \usepgfplotslibrary{fillbetween, polar}
%%%%%%

%% CITAZIONI
\use{lineno}

\newcommand{\citazione}[1]{%
  \begin{quotation}
  \begin{linenumbers}
  \modulolinenumbers[5]
  \begingroup
  \setlength{\parindent}{0cm}
  \noindent #1
  \endgroup
  \end{linenumbers}
  \end{quotation}\setcounter{linenumber}{1}
  }
%%%%%%

%%%%%%%%%%%%%%%%%%%%%%%%%%%%%%%%%%%%%%%%%%%%
%%%%%%%%%%%%%%%%%%%%%%%%%%%%%%%%%%%%%%%%%%%%

%% AMS THM

\theoremstyle{definition}% default
\newtheorem{thm}{Teorema}[section]
\newtheorem{lem}[thm]{Lemma}
\newtheorem{prop}[thm]{Proposizione}
\newtheorem{cor}[thm]{Corollario}
\newtheorem{esempio}[thm]{Esempio}
\theoremstyle{plain}
\newtheorem{definizione}[thm]{Definizione}
\theoremstyle{remark}
\newtheorem*{oss}{Osservazione}


%%%%%%%%%%%%%%%%%%%%%%%%%%%%%%%%%%%%%%%%%%%%
%%%%%%%%%%%%%%%%%%%%%%%%%%%%%%%%%%%%%%%%%%%%

\use{hyperref}
\hypersetup{%
        pdfauthor={Davide Peccioli},
        pdfsubject={},
        allcolors=black,
        citecolor=black,
%	colorlinks=true,
        bookmarksopen=true}
\setcounter{secnumdepth}{0} % rimuove i numeri di sezione senza rimuovere le ref
\renewcommand{\href}[2]{\textcolor{blue}{#2}} % disabilita il comando href
\use{enotez} %
\setenotez{%
 mark-format = \footnotemarkformatting % Mette i numeri tra parentesi quadre%
}\let\footnote=\endnote % rende tutte le note a pié pagina come delle note a fine file 


\let\olddocument\document % modifico l'ambiende documenti per non dover stampare \printendnote
\let\oldenddocument\enddocument
\renewenvironment{document}%
{%
  \olddocument
}{%
  \printendnotes\oldenddocument
}
\renewcommand{\thethm}{\arabic{thm}}

\usepackage[hyperref]{biblatex}
\addbibresource{~/Documents/org/roam/bib/master.bib}
\author{Davide Peccioli}
\date{\today}
\title{Rango di Cantor-Bendixson}
\begin{document}

\section{Definizione}
\label{sec:org9c31f39}

\subsection{Osservazione}
\label{sec:org4a59086}
\(x \notin X'\) se e solo se \(\set{x}\cap X\) è un aperto di \(X\).
\section{Proprietà di derivata e rango di Cantor-Bendixson per spazi polacchi}
\label{sec:org899bbd3}
Sia \(\alpha\) un ordinale limite, siano \((X_{\beta})_{\beta<\alpha}\) una famiglia di spazi polacchi e sia
\begin{equation*}
X\coloneqq \coprod_{\beta<\alpha} X_{\beta}.
\end{equation*}
Allora, per ogni ordinale \(\lambda\):
\begin{equation*}
X^{(\lambda)} = \coprod_{\beta<\alpha} X_{\beta}^{(\lambda)}
\end{equation*}
\subsection{Dimostrazione}
\label{sec:orgdb4ce1f}
Si ricorda la topologia dell'unione disgiunta: \(U \subseteq \coprod_{\beta<\alpha} X_{\beta}\) è aperto se e solo se, detta
\begin{equation*}
\varphi_{\beta_{i}}: X_{\beta_{i}}\to \coprod_{\beta<\alpha} X_{\beta}
\end{equation*}
l'iniezione canonica, per ogni \(\beta_{i}<\alpha\) l'insieme \(\varphi^{-1}_{\beta_{i}} \subseteq X_{\beta_{i}}\) è aperto.

Per induzione su \(\lambda\).
\begin{itemize}
\item Caso base: \(\lambda = 0\): banale.
\item Caso base: \(\lambda = 1\): bisogna dimostrare che
\begin{equation*}
  \left(\coprod_{\beta<\alpha} X_{\beta}\right)' = \coprod_{\beta<\alpha} X_{\beta}'
\end{equation*}

Sia \(x \in \coprod_{\beta<\alpha} X_{\beta}\). Allora esiste un unico \(\beta_{0}\) tale che \(x \in X_{\beta_{0}}\).

Dunque, se \(x \notin \left(\coprod_{\beta<\alpha} X_{\beta}\right)'\) allora:
\begin{itemize}
\item per ogni \(\beta\neq \beta_{0}\), \(x\notin X_{\beta}\) e quindi \(x\notin X_{\beta}' \subseteq X_{\beta}\);
\item per \(\beta_{0}\), \(\set{x} \subseteq X_{\beta_{0}}\) è aperto, e quindi \(x\notin X_{\beta_{0}}'\);
\end{itemize}
pertanto \(x\notin \coprod_{\beta<\alpha} X_{\beta}'\).

Viceversa, se \(x \notin \coprod_{\beta<\alpha} X_{\beta}'\) significa che \(\set{x} \subseteq X_{\beta_{0}}\) è aperto e pertanto per ogni \(\beta<\alpha\) l'insieme \(\varphi_{\beta}^{-1}\left(\set{x}\right)\) è aperto (poiché uguale a \(\emptyset\) se \(\beta\neq \beta_{0}\) e uguale a \(\set{x}\) se \(\beta=\beta_{0}\)). Pertanto \(\set{x} \subseteq \coprod_{\beta<\alpha} X_{\beta}\) è aperto, e dunque
\begin{equation*}
  x\notin \left(\coprod_{\beta<\alpha} X_{\beta}\right)'
\end{equation*}
\end{itemize}


\begin{itemize}
\item Ordinale successore: \(\lambda=\gamma+1\).
\begin{align*}
  X^{(\lambda)} &= \left(\coprod_{\beta<\alpha} X_{\beta}\right)^{(\lambda)} = \left(\left(\coprod_{\beta<\alpha} X_{\beta}\right)^{(\gamma)}\right)'\\
  &= \left(\coprod_{\beta<\alpha} X_{\beta}^{(\gamma)}\right)' = \coprod_{\beta<\alpha} (X_{\beta}^{(\gamma)})'\\
  &= \coprod_{\beta<\alpha} (X_{\beta}^{(\gamma+1)}) = \coprod_{\beta<\alpha} X_{\beta}^{(\lambda)}.
\end{align*}

\item Ordinale limite \(\lambda\):
\begin{align*}
  X^{(\lambda)} &= \bigcap_{\gamma<\lambda} X^{(\gamma)} = \bigcap_{\gamma<\lambda} \left(\coprod_{\beta<\alpha} X_{\beta}\right)^{(\gamma)}\\
  &= \bigcap_{\gamma<\lambda}\left(\coprod_{\beta<\alpha} X_{\beta}^{(\gamma)}\right) = \coprod_{\beta<\alpha} \bigcap_{\gamma<\lambda} X_{\beta}^{(\gamma)}\\
  &= \coprod_{\beta<\alpha} X_{\beta}^{(\lambda)}
\end{align*}
\end{itemize}
\section{Esempi di spazi polacchi con rango di Cantor-Bendixson arbitrario}
\label{sec:org73b8341}
Recall the notion of \href{20250403093420-rango_di_cantor_bendixson.org}{Cantor-Bendixson rank} of a \href{20250301194013-spazio_polacco.org}{Polish space} from Section 1.4 in the notes for the course. For each \href{20250203111003-ordinali.org}{ordinal} \(\alpha<\omega_1\), provide an example of a Polish space \(X\) with Cantor-Bendixson rank \(\alpha\). (Optional: show that such an \(X\) can always be taken as a countable space, and that if \(\alpha\) is a successor ordinal than \(X\) can be taken to be compact.)

{[}Hint. To geometrically visualize the problem it is easier to work in \(\R^2\). Use a construction by transfinite recursion over \(\alpha\). The cases \(\alpha=0,1\) are easy. For \(\alpha=2\) consider \(X=\{x\} \cup\left\{x_n \mid n \in \omega\right\}\) with x\textsubscript{n} \(\rightarrow\) x\$ and all \(x_{n}\) isolated. This suggest the strategy when \(\alpha=\beta+1\) is successor: consider a sequence of spaces of Cantor-Bendixson rank \(\beta\) and construe them as a sequence of spaces accumulating towards a point. For limit cases, consider the (disjoint) sum of spaces with Cantor-Bendixson rank cofinal in \(\alpha\).]
\subsection{Soluzione}
\label{sec:orgfabf20b}

Si costruiscono, per ricorsione, spazi polacchi \(X_{\alpha} \subseteq \R\) con rango di Cantor-Bendixson \(\alpha\) e tali che \(X^{\infty}_{\alpha} = \emptyset\).

Questo garantisce che ciascun \(X_{\alpha}\) sia uno spazio polacco \href{20250111143651-insieme_numerabile.org}{numerabile}.
\subsubsection{Caso base}
\label{sec:orgbc3e0b3}

Per \(\alpha=0\) deve valere che \(X_{0}^{(0)} = X_{0} = \emptyset\). Pertanto si pone \(X_{0}= \emptyset\).

Per \(\alpha=1\) deve valere che \(X^{(1)}_{1} = X_{1}' = \emptyset\). Pertanto si pone \(X_{1}= \omega \subseteq \R\).
\subsubsection{Ordinale successore}
\label{sec:org50a181d}

Sia \(\alpha=\beta+1\) un ordinale successore, e sia \(X_{\beta} \subseteq \R\) uno spazio polacco con rango di Cantor-Bendixson \(\beta\) e tale che \(X_{\beta}^{\infty} = \emptyset\).

Sia \(\set{y_{n}\mid n \in \omega} \subseteq \R\) una successione convergente ad \(y \in \R\), composta da punti isolati tali che per ogni \(n \in \omega\): \(y_{n}< y\). Per ciascun \(n \in \omega\) sia \(U_{n} \subseteq \R\) un intervallo aperto tale che \(y_{n} \subseteq U_{n}\) e che \(\forall\, m\neq n\): \(\operatorname{Cl}(U_{n})\cap \operatorname{Cl}(U_{m}) = \emptyset\).

Sia ora, per ogni \(n \in\omega\), \(\Phi_{n}: \R\to U_{n}\) un omeomorfismo (è sufficiente considerare una contrazione dell'arco tangente). Siano \(X_{n}\) le immagini di \(X_{\beta}\) tramite \(\Phi_{n}\):
\begin{equation*}
X_{n} \coloneqq \Phi_{n}(X_{\beta}) \subseteq U_{n}.
\end{equation*}

Si definisce \(X_{\alpha} \coloneqq \set{y}\cup \bigcup_{n \in \omega} X_{n}\). Questo è spazio polacco in quanto unione numerabile di spazi polacchi.

Per induzione su \(\lambda<\alpha\):
\begin{equation*}
X_{\alpha}^{(\lambda)} = \set{y} \cup \bigcup_{n \in \omega} X_{n}^{(\lambda)}
\end{equation*}
\begin{itemize}
\item Caso base: per \(\lambda=0\) è banale.
\item Ordinale successore: sia \(\lambda<\alpha\), \(\lambda = \gamma+1\).
Si dimostra che
\begin{equation*}
  	\set{y}\cup \bigcup_{n \in \omega} (X_{n}^{(\gamma)})' = \left(\set{y}\cup \bigcup_{n \in \omega} X_{n}^{(\gamma)}\right)'
\end{equation*}

Si consideri \(x \notin \left(\set{y}\cup \bigcup_{n \in \omega} X_{n}^{(\gamma)}\right)'\), \(x\neq y\).

Allora \(\set{x} \subseteq\set{y}\cup \bigcup_{n \in \omega} X_{n}^{(\gamma)}\) è aperto: per ogni \(n \in\omega\) si ha che \(\set{x}\cap X_{n}^{(\gamma)}\) è aperto in \(X_{n}^{(\gamma)}\)   e quindi per ogni \(n \in \omega\): \(x\notin (X_{n}^{(\gamma)})'\), ovvero
\begin{equation*}
  	x \notin \set{y}\cup \bigcup_{n \in \omega} \left(X_{n}^{(\gamma)}\right)'.
  \end{equation*}

Se invece per assurdo \(y\notin \left(\set{y}\cup \bigcup_{n \in \omega} X_{n}^{(\gamma)}\right)'\) allora $\backslash$\(\set{y} \subseteq \set{y} \bigcup_{n \in \omega} X_{n}^{(\gamma)}\) è aperto e quindi esiste \(\varepsilon>0\) tale che
\begin{equation*}
  	(y-\varepsilon, y+\varepsilon) \cap \left(\set{y}\cup\bigcup_{n \in \omega} X_{n}^{(\gamma)}\right) = \set{y}
\end{equation*}
Siano ora \(y_{n_{0}}, y_{n_{1}}, y_{n_{2}} \in (y-\varepsilon, y+\varepsilon)\) (che esistono poiché \(y_{n}\to y\)), con \(y_{n_{0}}< y_{n_{1}} < y_{n_{2}}\). Allora, siccome \(U_{n_{1}} = (a,b)\) per certi \(a,b \in \R\) tali che \(y_{n_{0}} < a\) e \(b< y_{n_{1}} < y\), si ha:
\begin{equation*}
  	U_{n_{1}} \subseteq (y-\varepsilon, y).
\end{equation*}
Siccome \(\lambda<\alpha\) ovvero \(\gamma+1<\beta+1\) allora \(\gamma<\beta\) e pertanto, per ogni \(n \in \omega\): \(X_{n}^{(\gamma)}\neq \emptyset\). Quindi
\begin{equation*}
  	\emptyset\neq\Phi_{n_{1}}(X_{\beta}^{(\gamma)}) = X_{n_{1}}^{(\gamma)} \subseteq U_{n_{1}} \subseteq (y-\varepsilon, y)
\end{equation*}
e pertanto
\begin{equation*}
  	(y-\varepsilon, y+\varepsilon) \cap\left(\set{y}\cup\bigcup_{n \in \omega} X_{n}^{(\gamma)}\right) \supseteq \set{y}\cup \Phi_{n_{1}}(X_{\beta}^{(\gamma)})\supsetneqq \set{y}
\end{equation*}
Assurdo. Quindi \(y \in \left(\set{y}\cup\bigcup_{n \in \omega} X_{n}^{(\gamma)}\right)'\)

Viceversa, se \(x \notin \set{y}\cup \bigcup_{n \in \omega}(X_{n}^{(\gamma)})'\) allora per ogni \(n \in \omega\):
\begin{equation*}
  	x \notin (X_{n}^{(\gamma)})'
\end{equation*}
e pertanto \(\set{x} \subseteq X_{n}^{(\gamma)}\) è aperto. Ma \(X_{n}^{(\gamma)}\) è aperto di \(\set{y}\cup \bigcup_{n \in \omega} X_{n}^{(\gamma)}\) e quindi anche \(\set{x}\) lo è:
\begin{equation*}
  	x \notin \left(\set{y}\cup\bigcup_{n \in \omega} X_{n}^{(\lambda)}\right)'.
\end{equation*}
Si noti che per ogni \(n \in \omega\) si ha che \(X_{n}^{(\gamma)}\) è aperto di \(\set{y}\cup \bigcup_{n \in \omega} X_{n}^{(\gamma)}\) poiché
\begin{equation*}
  	X_{n}^{(\gamma)} = U_{n}\cap \left(\set{y}\cup\bigcup_{n \in \omega} X_{n}^{(\gamma)}\right)
\end{equation*}
dove \(U_{n}\) è un aperto di \(\R\).

Pertanto si ha che
\begin{align*}
  	X_{\alpha}^{(\lambda)} &= \left(X_{\alpha}^{(\gamma)}\right)'\\
  	&= \set{y}\cup \bigcup_{n \in \omega} (X_{n}^{\gamma})' = \set{y}\cup\bigcup_{n \in \omega} X_{n}^{(\lambda)}.
\end{align*}
\end{itemize}


\begin{itemize}
\item Ordinale limite: sia \(\lambda <\alpha\) un ordinale limite. Allora
\begin{align*}
  	X_{\alpha}^{(\lambda)} &= \bigcap_{\gamma<\lambda} X_{\alpha}^{(\gamma)} = \bigcap_{\gamma<\lambda}\left(\set{y}\cup \bigcup_{n \in\omega} X_{n}^{(\gamma)}\right)\\
  	&= \set{y}\cup \bigcap_{\gamma<\lambda} \bigcup_{n \in \omega} X_{n}^{(\gamma)}\\
  	&= \set{y}\cup \bigcup_{n \in \omega} \bigcap_{\gamma<\lambda} X_{n}^{(\gamma)} = \set{y}\cup\bigcup_{n \in\omega} X_{n}^{(\lambda)}.
\end{align*}
\end{itemize}

Pertanto \(X_{\alpha}^{(\beta)} = \set{y}\) e \(X_{\alpha}^{(\alpha)} = \emptyset\).
\subsubsection{Ordinale limite}
\label{sec:org92cf726}
Sia \(\alpha<\omega_{1}\) un ordinale limite, e sia per ogni \(\beta<\alpha\): \(X_{\beta}\) uno spazio polacco con rango di Cantor-Bendixson \(\beta\) e tale che \(X_{\beta}^{\infty} = \emptyset\).

Sia \((\beta_{n})_{n \in \omega}\) una successione di ordinali cofinale in \(\alpha\). Senza perdita di generalità è possibile considerare ciascun \(X_{\beta_{n}}\) contenuto nell'intervallo \(I_{n}\coloneqq(n-1/2,n+1/2)\), per mezzo di un omeomorfismo \(\R\to I_{n}\).

Allora si pone \(X_{\alpha}\coloneqq \coprod_{n<\omega} X_{\beta_{n}}\), \(X_{\alpha} \subseteq \R\) è uno spazio polacco in quanto unione numerabile di spazi polacchi.

Inoltre \(X^{\infty}_{\alpha} = \emptyset\) e \(X\) ha rango di Cantor-Bendixson \(\alpha\). Infatti, per ogni \(\lambda<\alpha\) esiste \(n_{0} \in \omega\) tale che \(\beta_{n_{0}} >\lambda\) per cofinalità di \((\beta_{n})_{n \in \omega}\) e pertanto:
\begin{equation*}
X^{(\lambda)}_{\alpha} = \coprod_{n<\omega} X_{\beta_{n}}^{(\lambda)} \supseteq X_{\beta_{n_{0}}}^{(\lambda)} \neq\emptyset
\end{equation*}
mentre
\begin{equation*}
X^{(\alpha)}_{\alpha} = \coprod_{n<\omega} X_{\beta_{n}}^{(\alpha)} = \coprod_{n<\omega}\emptyset = \emptyset.\qedd
\end{equation*}
\end{document}
