% Created 2026-02-07 Sat 19:32
% Intended LaTeX compiler: pdflatex
\documentclass[10pt]{article}
%% CREATO CON ORG - EMACS
\newcommand{\use}[2][]{\usepackage[#1]{#2}}
% PACCHETTI FONDAMENTLAI
\use[utf8]{inputenc}
\use[T1]{fontenc}
\use{graphicx}
\use{longtable}
\use{wrapfig}
\use{rotating}
\use[normalem]{ulem}
\use{amsmath}
\use{amsthm}
\use{amssymb}

\use{eucal} % Cambia mathcal{...}

\use{capt-of}
\use[italian]{babel}
\use[babel]{csquotes}
% bib la TEX lo carica in automatico org-cite
\use{microtype}
\use{lmodern}
\use{subfig} % sottofigure
\use{multicol} % due colonne
\use{lipsum} % lorem ipsum
\use{color} % colori in latex
\use{parskip} % rimuove l'indentazione dei nuovi paragrafi %% Add parbox=false to all new tcolorbox
\use{centernot}
\use[outline]{contour}\contourlength{3pt}
\use{fancyhdr}
\use{layout}
\use[most]{tcolorbox} % Riquadri colorati
\use{ifthen} % IFTHEN
\use{geometry}

% pacchetti matematica
\use{yhmath}
\use{dsfont}
\use{mathrsfs}
\use{cancel} % semplificare
\use{polynom} %divisione tra polinomi
\use{forest} % grafi ad albero
\use{booktabs} % tabelle
\use{commath} %simboli e differenziali
\use{bm} %bold
\use[fulladjust]{marginnote} %to use marginnote for date notes
\use{arrayjobx}%array
\use[intlimits]{empheq} % Riquadri colorati attorno alle equazioni
\use{mathtools}
\use{circuitikz} % Disegnare i circuiti
\use{mathtools}
\use{stmaryrd} % [[ \llbracket ]] \rrbracket
\use{bussproofs} % dimostrazioni

%%%%%%%%%%%%%


%%%% QUIVER
\newcommand{\duepunti}{\,\mathchar\numexpr"6000+`:\relax\,}
% A TikZ style for curved arrows of a fixed height, due to AndréC.
\tikzset{curve/.style={settings={#1},to path={(\tikztostart)
    .. controls ($(\tikztostart)!\pv{pos}!(\tikztotarget)!\pv{height}!270:(\tikztotarget)$)
    and ($(\tikztostart)!1-\pv{pos}!(\tikztotarget)!\pv{height}!270:(\tikztotarget)$)
    .. (\tikztotarget)\tikztonodes}},
    settings/.code={\tikzset{quiver/.cd,#1}
        \def\pv##1{\pgfkeysvalueof{/tikz/quiver/##1}}},
    quiver/.cd,pos/.initial=0.35,height/.initial=0}

% TikZ arrowhead/tail styles.
\tikzset{tail reversed/.code={\pgfsetarrowsstart{tikzcd to}}}
\tikzset{2tail/.code={\pgfsetarrowsstart{Implies[reversed]}}}
\tikzset{2tail reversed/.code={\pgfsetarrowsstart{Implies}}}
% TikZ arrow styles.
\tikzset{no body/.style={/tikz/dash pattern=on 0 off 1mm}}
%%%%%%%%%%


%% DEFINIZIONI COMANDI MATEMATICI
\let\sin\relax %TOGLIE LA DEFINIZIONE SU "\sin"

% cambia la definizione di empty set
% ---
\let\oldemptyset\emptyset
% ---
% \let\emptyset\varnothing
% ---
% \let\emptyset\relax
% \newcommand{\emptyset}{\text{\textnormal{\O}}}
% ---

\DeclareMathOperator{\bounded}{bd}
\DeclareMathOperator{\sin}{sen}
\DeclareMathOperator{\epi}{Epi}
\DeclareMathOperator{\cl}{cl}
\DeclareMathOperator{\graph}{graph}
\DeclareMathOperator{\arcsec}{arcsec}
\DeclareMathOperator{\arccot}{arccot}
\DeclareMathOperator{\arccsc}{arccsc}
\DeclareMathOperator{\spettro}{Spettro}
\DeclareMathOperator{\nulls}{nullspace}
\DeclareMathOperator{\dom}{dom}
\DeclareMathOperator{\ar}{ar}
\DeclareMathOperator{\const}{Const}
\DeclareMathOperator{\fun}{Fun}
\DeclareMathOperator{\rel}{Rel}
\DeclareMathOperator{\altezza}{ht}
\let\det\relax %TOGLIE LA DEFINIZIONE SU "\det"
\DeclareMathOperator{\det}{det}
\DeclareMathOperator{\End}{End}
\DeclareMathOperator{\gl}{GL}
\def\Id{\mathrm{Id}}
\def\id{\mathrm{id}}
\DeclareMathOperator{\I}{\mathds{1}}
\DeclareMathOperator{\II}{II}
\DeclareMathOperator{\rank}{rank}
\DeclareMathOperator{\tr}{tr}
\DeclareMathOperator{\tc}{t.c.}
\DeclareMathOperator{\T}{T}
\DeclareMathOperator{\var}{Var}
\DeclareMathOperator{\cov}{Cov}
\DeclareMathOperator{\st}{st}
\DeclareMathOperator{\mon}{Mon}
\newcommand{\card}[1]{\left\vert #1 \right\vert}
\newcommand{\trasposta}[1]{\prescript{\text{T}}{}{#1}}
\newcommand{\1}{\mathds{1}}
\newcommand{\R}{\mathds{R}}
\newcommand{\diesis}{\#}
\newcommand{\bemolle}{\flat}
\newcommand{\nonstandard}[1]{\prescript{*}{}{#1}}
\newcommand{\starR}{\nonstandard{\R}}
\newcommand{\borel}{\mathscr{B}}
\newcommand{\lebesgue}[1]{\mathscr{L}\left(#1\right)}
\newcommand{\media}{\mathds{E}}
\newcommand{\K}{\mathds{K}}
\newcommand{\A}{\mathds{A}}
\newcommand{\Q}{\mathds{Q}}
\newcommand{\N}{\mathds{N}}
\newcommand{\C}{\mathds{C}}
\newcommand{\Z}{\mathds{Z}}
\newcommand{\qo}{\hspace{1em}\text{q.o.}\,}
\renewcommand{\tilde}[1]{\widetilde{#1}}
\renewcommand{\parallel}{\mathrel{/\mkern-5mu/}}
\newcommand{\parti}[2][]{\wp_{#1}(#2)}
\newcommand{\diff}[1]{\operatorname{d}_{#1}}
\let\oldvec\vec
\renewcommand{\vec}[1]{\overrightarrow{\vphantom{i}#1}}
\newcommand{\floor}[1]{\left\lfloor #1 \right\rfloor}
\newcommand{\cat}[1]{\mathbf{#1}}
\newcommand{\dfreccia}[1]{\xrightarrow{\ #1 \ }}
\newcommand{\sfreccia}[1]{\xleftarrow{\ #1 \ }}
\newcommand{\formalsum}[2]{{\sum_{#1}^{#2}}{\vphantom{\sum}}'}
\newcommand{\minim}[2]{\mu_{#1}\, \left(#2\right)}
\newcommand{\concat}{\null^{\frown}} % concatenazione di stringe
\newcommand{\godelcode}[1]{\langle\!\langle #1 \rangle\!\rangle}
\newcommand{\godeldec}[1]{(\!(#1)\!)}
\newcommand{\termcode}[1]{\ulcorner #1\urcorner}
\newcommand{\partialto}{\dashrightarrow}
\newcommand{\restricted}{\upharpoonright}
\newcommand{\embeds}{\precsim}
\newcommand{\surjects}{\twoheadrightarrow}
\newcommand{\equipotenti}{\asymp}
%% \newcommand{\dotplus}{\mathbin{\dot{+}}} %% A quanto pare esiste già
\newcommand{\bigdot}{\mathbin{\boldsymbol{\cdot}}}
\newcommand{\dotexp}[1]{^{.#1}}
\newcommand{\conv}{\mathbin{*}}
\newcommand{\convolution}[2]{(#1\conv #2)}
\newcommand{\nil}{\mathfrak{N}}
\newcommand{\divisore}{\mathrel{|}}
\newcommand{\simplesso}[1]{\mathrm{e}_{#1}}

\renewcommand{\iff}{\mathrel{\longleftrightarrow}} %% Notazione Logica.
\newcommand{\oldiff}{\mathrel{\Longleftrightarrow}}
\renewcommand{\implies}{\mathrel{\rightarrow}} %% Notazione Logica
\newcommand{\oldimplies}{\mathrel{\Longrightarrow}}
\renewcommand{\impliedby}{\mathrel{\leftarrow}} %% Notazione Logica
\newcommand{\oldimpliedby}{\mathrel{\Longleftarrow}}

\newcommand{\IFF}{\quad\Longleftrightarrow\quad}
\newcommand{\IMPLICA}{\quad\Longrightarrow\quad}


\renewcommand{\descriptionlabel}[1]{\hspace{\labelsep}\normalfont #1} % remove bold from description


%% Definizione di Divergenza di K-L

\DeclarePairedDelimiterX{\infdivx}[2]{(}{)}{%
  #1\;\delimsize\|\;#2%
}
\newcommand{\kldiv}{D_{KL}\infdivx}

%% Definizione di \dotminus

\makeatletter
\newcommand{\dotminus}{\mathbin{\text{\@dotminus}}}

\newcommand{\@dotminus}{%
  \ooalign{\hidewidth\raise1ex\hbox{.}\hidewidth\cr$\m@th-$\cr}%
}
\makeatother

%tramite i prossimi due comandi posso decidere come scrivere i logaritmi naturali in tutti i documenti: ho infatti eliminato qualsiasi differenza tra "ln" e "log": se si vuole qualcosa di diverso bisogna inserire manualmente il tutto
\let\ln\relax
\DeclareMathOperator{\ln}{ln}
\let\log\relax
\DeclareMathOperator{\log}{log}
%%%%%%

%% NUOVI COMANDI
\newcommand{\straniero}[1]{\textit{#1}} %parole straniere
\newcommand{\titolo}[1]{\textsc{#1}} %titoli
\newcommand{\qedd}{\tag*{$\blacksquare$}} %qed per ambienti matemastici
\renewcommand{\qedsymbol}{$\blacksquare$} %modifica colore qed
\newcommand{\ooverline}[1]{\overline{\overline{#1}}}
\newcommand{\circoletto}[1]{\left(#1\right)^{\text{o}}}
%
\newcommand{\qmatrice}[1]{\begin{pmatrix}
#1_{11} & \cdots & #1_{1n}\\
\vdots & \ddots & \vdots \\
#1_{m1} & \cdots & #1_{mn}
\end{pmatrix}}
%
\newcommand{\parentesi}[2]{%
\underset{#1}{\underbrace{#2}}%
}
%
\newcommand{\norma}[1]{% Norma
\left\lVert#1\right\rVert%
}
\newcommand{\scalare}[2]{% Scalare
\left\langle #1, #2\right\rangle
}
%%%%%

%% RESTRIZIONI
\newcommand{\referenze}[2]{
        \phantomsection{}#2\textsuperscript{\textcolor{blue}{\textbf{#1}}}
}

\let\restriction\relax

\def\restriction#1#2{\mathchoice
              {\setbox1\hbox{${\displaystyle #1}_{\scriptstyle #2}$}
              \restrictionaux{#1}{#2}}
              {\setbox1\hbox{${\textstyle #1}_{\scriptstyle #2}$}
              \restrictionaux{#1}{#2}}
              {\setbox1\hbox{${\scriptstyle #1}_{\scriptscriptstyle #2}$}
              \restrictionaux{#1}{#2}}
              {\setbox1\hbox{${\scriptscriptstyle #1}_{\scriptscriptstyle #2}$}
              \restrictionaux{#1}{#2}}}
\def\restrictionaux#1#2{{#1\,\smash{\vrule height .8\ht1 depth .85\dp1}}_{\,#2}}
%%%%%%%%%%%

%%% FORMATTAZIONE FOOTNOTEMARK

\def\footnotemarkformatting#1{[#1]}
\renewcommand{\thefootnote}{\footnotemarkformatting{\arabic{footnote}}}

%% SEZIONE GRAFICA
\use{tikz}
\usetikzlibrary{matrix, patterns, calc, decorations.pathreplacing, hobby, decorations.markings, decorations.pathmorphing, babel}
\use{tikz-3dplot}
\use{mathrsfs} %per geogebra
\use{tikz-cd}
\tikzset
{
  %surface/.style={fill=black!10, shading=ball,fill opacity=0.4},
  plane/.style={black,pattern=north east lines},
  curve/.style={black,line width=0.5mm},
  dritto/.style={decoration={markings,mark=at position 0.5 with {\arrow{Stealth}}}, postaction=decorate},
  rovescio/.style={decoration={markings,mark=at position 0.5 with {\arrow{Stealth[reversed]}}}, postaction=decorate}
}
\use{pgfplots} % stampare le funzioni
        \pgfplotsset{/pgf/number format/use comma,compat=1.15}
        %\pgfplotsset{compat=1.15} %per geogebra
        \usepgfplotslibrary{fillbetween, polar}
%%%%%%

%% CITAZIONI
\use{lineno}

\newcommand{\citazione}[1]{%
  \begin{quotation}
  \begin{linenumbers}
  \modulolinenumbers[5]
  \begingroup
  \setlength{\parindent}{0cm}
  \noindent #1
  \endgroup
  \end{linenumbers}
  \end{quotation}\setcounter{linenumber}{1}
  }
%%%%%%

%%%%%%%%%%%%%%%%%%%%%%%%%%%%%%%%%%%%%%%%%%%%
%%%%%%%%%%%%%%%%%%%%%%%%%%%%%%%%%%%%%%%%%%%%

%% AMS THM

\theoremstyle{definition}% default
\newtheorem{thm}{Teorema}[section]
\newtheorem{lem}[thm]{Lemma}
\newtheorem{prop}[thm]{Proposizione}
\newtheorem{cor}[thm]{Corollario}
\newtheorem{esempio}[thm]{Esempio}
\theoremstyle{plain}
\newtheorem{definizione}[thm]{Definizione}
\theoremstyle{remark}
\newtheorem*{oss}{Osservazione}


%%%%%%%%%%%%%%%%%%%%%%%%%%%%%%%%%%%%%%%%%%%%
%%%%%%%%%%%%%%%%%%%%%%%%%%%%%%%%%%%%%%%%%%%%

\use{hyperref}
\hypersetup{%
        pdfauthor={Davide Peccioli},
        pdfsubject={},
        allcolors=black,
        citecolor=black,
%	colorlinks=true,
        bookmarksopen=true}
\setcounter{secnumdepth}{0} % rimuove i numeri di sezione senza rimuovere le ref
\renewcommand{\href}[2]{\textcolor{blue}{#2}} % disabilita il comando href
\use{enotez} %
\setenotez{%
 mark-format = \footnotemarkformatting % Mette i numeri tra parentesi quadre%
}\let\footnote=\endnote % rende tutte le note a pié pagina come delle note a fine file 


\let\olddocument\document % modifico l'ambiende documenti per non dover stampare \printendnote
\let\oldenddocument\enddocument
\renewenvironment{document}%
{%
  \olddocument
}{%
  \printendnotes\oldenddocument
}
\renewcommand{\thethm}{\arabic{thm}}

\usepackage[hyperref]{biblatex}
\addbibresource{~/Documents/org/roam/bib/master.bib}
\author{Davide Peccioli}
\date{\today}
\title{}
\begin{document}

\section{Teorema di Spezzamento SEC}
\label{sec:org0566f75}
Sia \(R\) un \href{20241205141119-anello.org}{anello} commutativo con unità.

\begin{thm}
Si consideri la \href{20250120131527-sec.org}{SEC} di \href{20241205141053-r_moduli.org}{\(R\)-moduli}:
\begin{equation*}
\begin{tikzcd}[ampersand replacement=\&]
	0 \& M \& N \& P \& 0
	\arrow[from=1-1, to=1-2]
	\arrow["f", from=1-2, to=1-3]
	\arrow["\varphi", color={rgb,255:red,214;green,92;blue,92}, bend left=24, dashed, from=1-3, to=1-2]
	\arrow["g", from=1-3, to=1-4]
	\arrow["\psi", color={rgb,255:red,214;green,92;blue,92},  bend left=24, dashed, from=1-4, to=1-3]
	\arrow[from=1-4, to=1-5]
\end{tikzcd}
\end{equation*}
Sono fatti equivalenti:
\begin{enumerate}
\item esiste un \href{20241206115416-morfismi_r_moduli.org}{morfismo} \(\psi:P\longrightarrow N\) tale che \(g\circ\psi =\operatorname{id}_{P}\) (e \(\psi\) si dice \textbf{sezione});
\item esiste un \href{20241206142802-sottomoduli.org}{sottomodulo} \(N' \mathrel{\subseteq_{R}} N\) tale che\footnote{Si veda:
\begin{itemize}
\item \href{20241206142802-sottomoduli.org}{Somma di sottomoduli}
\item \href{20260112124147-sottomoduli_in_somma_diretta.org}{Sottomoduli in somma diretta}
\item \href{20250202190147-immagine_punto_a_punto_di_due_classi.org}{Immagine e retroimmagine tramite una funzione}
\item \href{20241213095808-somma_diretta.org}{Somma Diretta}
\item \href{20250131155822-operazioni_insiemistiche_tra_classi_mk.org}{Intersezione}
\end{itemize}}
\begin{equation*}
 N \cong N'\oplus \operatorname{Im}f
\end{equation*}
e tale che \(\restriction{g}{N'}:N' \longrightarrow P\) sia un \href{20241206115416-morfismi_r_moduli.org}{isomorfismo} (ovvero \(N \cong M \oplus P\)).
\item esiste un \href{20241206115416-morfismi_r_moduli.org}{morfismo} \(\varphi:N\longrightarrow M\) tale che \(\varphi\circ f = \operatorname{Id}_{M}\) (e \(\varphi\) si dice \textbf{retrazione})
\end{enumerate}
\end{thm}
\begin{definizione}
Se vale una delle condizioni di cui sopra, la SEC si dice \textbf{spezzante}.
\end{definizione}
\begin{proof}
\begin{description}
\item[{(\(2.\Rightarrow 3.\)):}] Per ogni \(n \in N\), esistono \(n' \in N'\) e \(m \in M\) tali che
\begin{equation*}
n = n'+f(m)
\end{equation*}
Inoltre questi sono unici poiché \(f\) è iniettiva e \(N=N'+\operatorname{Im}f\).

Si definisce pertanto
\begin{equation*}
\varphi(n) = m
\end{equation*}
che è ben definita per l'unicità della scrittura di cui sopra. Inoltre, questa è un morfismo (lasciato per esercizio).
\begin{equation*}
\varphi\circ f(m) = \varphi\left(0+ f(m)\right) = m
\end{equation*}

\item[{(\(1.\Rightarrow 2.\)):}] Supponiamo esista un morfismo \(\psi:P\longrightarrow N\) tale che \(g\circ \psi =\operatorname{Id}_{P}\).
Si definisca \(N' \coloneqq \operatorname{Im}\psi\). Inoltre, \(N' \mathrel{\subseteq_{R}} N\).

\begin{itemize}
\item \textbf{Claim 1:} \(\operatorname{Im}f\cap N' = \set{0}\)

Sia \(n \in \operatorname{Im}f \cap N'\).
\begin{itemize}
\item Poiché la catena è esatta, allora \(\operatorname{Im}f=\operatorname{ker}g\), e dunque \(n \in \operatorname{ker}g\), e quindi \(g(n)=0\).
\item Inoltre \(n \in N' = \operatorname{Im}\psi\), e dunque esiste \(p \in P\) tale che \(\psi(p) = n\).
\end{itemize}

Dunque
\begin{equation*}
0 = g(n)=g\circ\psi(p) = p
\end{equation*}
e pertanto \(n=\psi(0)\). Siccome \(\psi\) è morfismo, allora \(n=0\).

\item \textbf{Claim 2:} \(\restriction{g}{N'}:N'\longrightarrow P\) iniettiva e suriettiva

Sia \(n' \in N'\) tale che \(g(n')=0\). Allora \(n' \in \operatorname{ker}g \cap N' = \operatorname{Im}f\cap N' = \set{0}\) e dunque \(n'=0\). Pertanto \(\restriction{g}{N'}\) è iniettiva.

Sia ora \(p \in P\). Allora \(g\left(\psi(p)\right) = p\) e \(\psi(p) \in N'\).

\item \textbf{Claim 3}: \(\operatorname{Im}f+N' = N\)

Sia \(n \in N\). Allora \(g(n) \in P\) e \(\psi\left(g(n)\right) \in N' = \operatorname{Im}\psi\). Sia \(n'\coloneqq \psi\left(g(n)\right)\).
\begin{align*}
g(n-n') &= g(n)-g(n')\\
&= g(n) - g\left(\psi\left(g(n)\right)\right)\\
&= g(n) - g\circ\psi\left(g(n)\right)\\
&= g(n)-g(n) = 0
\end{align*}
e pertanto \((n-n') \in \operatorname{ker}g = \operatorname{Im}f\). Dunque esiste \(m \in M\) tale che \(n-n'=f(m)\) e
\begin{equation*}
n = n' + f(m).
\end{equation*}
\end{itemize}

\item[{(\(3.\Rightarrow 1.\)):}] Supponiamo che esista il morfismo \(\varphi:N\longrightarrow M\). Voglio costruire \(\psi:P\longrightarrow N\).

Per ogni \(p \in P\), si scelga \(n_{p} \in g^{-1}(p)\). Allora
\begin{equation*}
\psi(p) \coloneqq n_{p} - f\circ\varphi(n_{p})
\end{equation*}

\begin{itemize}
\item \textbf{Claim 1}: \(\psi\) è ben definita

Fissato \(p \in P\), siano \(n,n' \in g^{-1}(p)\).

Notiamo che \(g(n-n') = g(n)-g(n') = p-p = 0\), pertanto \(n-n' \in \operatorname{ker}g=\operatorname{Im}f\) e pertanto esiste \(m \in M\) tale che
\begin{equation*}
n'=n-f(m)
\end{equation*}

Dunque
\begin{align*}
n'-f\circ\varphi(n' ) &= n-f(m)-f\left(\varphi\left(n-f(m)\right)\right)\\
&= n-f(m)-f\left(\varphi(n)-\varphi\circ f(m)\right)\\
&= n - f(m) - f\varphi(n) + f \varphi f(m)\\
&= n - f\varphi(m) = n - f\circ\varphi(n)
\end{align*}

\item \textbf{Claim 2}: \(g\circ\psi = \operatorname{Id}_{P}\)

Sia \(p \in P\) e sia \(n_{p} \in g^{-1}(p)\).

\begin{equation*}
g\left(\psi(p)\right) = g\left(n_{p} - f\varphi (n)\right) = g(n_{p}) - gf\varphi(n) = p
\end{equation*}
poiché \(g(n_{p}) = p\) e \(\operatorname{ker}g = \operatorname{Im}f\).

\item \textbf{Claim 3}: \(\psi\) è un morfismo
\qedhere
\end{itemize}
\end{description}
\end{proof}
\section{Spezzamento SEC con modulo finale libero}
\label{sec:org4271be6}
Sia \(R\) un \href{20241205141119-anello.org}{anello} commutativo con unità.

\begin{prop}
Si consideri la \href{20250120131527-sec.org}{SEC}:
\begin{equation*}
\begin{tikzcd}[ampersand replacement=\&]
	0 \& M \& N \& P \& 0
	\arrow[from=1-1, to=1-2]
	\arrow["f", from=1-2, to=1-3]
	\arrow["g", from=1-3, to=1-4]
	\arrow["\psi", color={rgb,255:red,214;green,92;blue,92},  bend left=23, dashed, from=1-4, to=1-3]
	\arrow[from=1-4, to=1-5]
\end{tikzcd}
\end{equation*}

Se \(P\) è libero, allora \(N\cong M\oplus P\)\footnote{Vedi ``\href{20241206115416-morfismi_r_moduli.org}{Isomorfismo tra R-Moduli}'' e ``\href{20241213095808-somma_diretta.org}{Somma-Diretta}''.}.
\end{prop}

\begin{proof}
Sia \(E \subseteq P\) una \href{20241213094625-modulo_libero.org}{base}. Per ogni \(e_{i} \in E\) si fissi \(\overline{e_{i}} \in g^{-1}(e_{i})\).
Si definisce
\begin{equation*}
\psi\left(\sum a_{i}\ e_{i}\right)\coloneqq \sum a_{i}\ \overline{e_{i}}
\end{equation*}
che è \href{20241206115416-morfismi_r_moduli.org}{morfismo} per definizione, \(\psi: P\longrightarrow N\).

Inoltre \(g\circ\psi = \operatorname{Id}_{P}\). Pertanto, per il \hyperref[sec:org0566f75]{teorema di spezzamento},
\begin{equation*}
N\cong M\oplus P.%
\qedhere
\end{equation*}
\end{proof}
\end{document}
