% Created 2026-02-07 Sat 19:31
% Intended LaTeX compiler: pdflatex
\documentclass[10pt]{article}
%% CREATO CON ORG - EMACS
\newcommand{\use}[2][]{\usepackage[#1]{#2}}
% PACCHETTI FONDAMENTLAI
\use[utf8]{inputenc}
\use[T1]{fontenc}
\use{graphicx}
\use{longtable}
\use{wrapfig}
\use{rotating}
\use[normalem]{ulem}
\use{amsmath}
\use{amsthm}
\use{amssymb}

\use{eucal} % Cambia mathcal{...}

\use{capt-of}
\use[italian]{babel}
\use[babel]{csquotes}
% bib la TEX lo carica in automatico org-cite
\use{microtype}
\use{lmodern}
\use{subfig} % sottofigure
\use{multicol} % due colonne
\use{lipsum} % lorem ipsum
\use{color} % colori in latex
\use{parskip} % rimuove l'indentazione dei nuovi paragrafi %% Add parbox=false to all new tcolorbox
\use{centernot}
\use[outline]{contour}\contourlength{3pt}
\use{fancyhdr}
\use{layout}
\use[most]{tcolorbox} % Riquadri colorati
\use{ifthen} % IFTHEN
\use{geometry}

% pacchetti matematica
\use{yhmath}
\use{dsfont}
\use{mathrsfs}
\use{cancel} % semplificare
\use{polynom} %divisione tra polinomi
\use{forest} % grafi ad albero
\use{booktabs} % tabelle
\use{commath} %simboli e differenziali
\use{bm} %bold
\use[fulladjust]{marginnote} %to use marginnote for date notes
\use{arrayjobx}%array
\use[intlimits]{empheq} % Riquadri colorati attorno alle equazioni
\use{mathtools}
\use{circuitikz} % Disegnare i circuiti
\use{mathtools}
\use{stmaryrd} % [[ \llbracket ]] \rrbracket
\use{bussproofs} % dimostrazioni

%%%%%%%%%%%%%


%%%% QUIVER
\newcommand{\duepunti}{\,\mathchar\numexpr"6000+`:\relax\,}
% A TikZ style for curved arrows of a fixed height, due to AndréC.
\tikzset{curve/.style={settings={#1},to path={(\tikztostart)
    .. controls ($(\tikztostart)!\pv{pos}!(\tikztotarget)!\pv{height}!270:(\tikztotarget)$)
    and ($(\tikztostart)!1-\pv{pos}!(\tikztotarget)!\pv{height}!270:(\tikztotarget)$)
    .. (\tikztotarget)\tikztonodes}},
    settings/.code={\tikzset{quiver/.cd,#1}
        \def\pv##1{\pgfkeysvalueof{/tikz/quiver/##1}}},
    quiver/.cd,pos/.initial=0.35,height/.initial=0}

% TikZ arrowhead/tail styles.
\tikzset{tail reversed/.code={\pgfsetarrowsstart{tikzcd to}}}
\tikzset{2tail/.code={\pgfsetarrowsstart{Implies[reversed]}}}
\tikzset{2tail reversed/.code={\pgfsetarrowsstart{Implies}}}
% TikZ arrow styles.
\tikzset{no body/.style={/tikz/dash pattern=on 0 off 1mm}}
%%%%%%%%%%


%% DEFINIZIONI COMANDI MATEMATICI
\let\sin\relax %TOGLIE LA DEFINIZIONE SU "\sin"

% cambia la definizione di empty set
% ---
\let\oldemptyset\emptyset
% ---
% \let\emptyset\varnothing
% ---
% \let\emptyset\relax
% \newcommand{\emptyset}{\text{\textnormal{\O}}}
% ---

\DeclareMathOperator{\bounded}{bd}
\DeclareMathOperator{\sin}{sen}
\DeclareMathOperator{\epi}{Epi}
\DeclareMathOperator{\cl}{cl}
\DeclareMathOperator{\graph}{graph}
\DeclareMathOperator{\arcsec}{arcsec}
\DeclareMathOperator{\arccot}{arccot}
\DeclareMathOperator{\arccsc}{arccsc}
\DeclareMathOperator{\spettro}{Spettro}
\DeclareMathOperator{\nulls}{nullspace}
\DeclareMathOperator{\dom}{dom}
\DeclareMathOperator{\ar}{ar}
\DeclareMathOperator{\const}{Const}
\DeclareMathOperator{\fun}{Fun}
\DeclareMathOperator{\rel}{Rel}
\DeclareMathOperator{\altezza}{ht}
\let\det\relax %TOGLIE LA DEFINIZIONE SU "\det"
\DeclareMathOperator{\det}{det}
\DeclareMathOperator{\End}{End}
\DeclareMathOperator{\gl}{GL}
\def\Id{\mathrm{Id}}
\def\id{\mathrm{id}}
\DeclareMathOperator{\I}{\mathds{1}}
\DeclareMathOperator{\II}{II}
\DeclareMathOperator{\rank}{rank}
\DeclareMathOperator{\tr}{tr}
\DeclareMathOperator{\tc}{t.c.}
\DeclareMathOperator{\T}{T}
\DeclareMathOperator{\var}{Var}
\DeclareMathOperator{\cov}{Cov}
\DeclareMathOperator{\st}{st}
\DeclareMathOperator{\mon}{Mon}
\newcommand{\card}[1]{\left\vert #1 \right\vert}
\newcommand{\trasposta}[1]{\prescript{\text{T}}{}{#1}}
\newcommand{\1}{\mathds{1}}
\newcommand{\R}{\mathds{R}}
\newcommand{\diesis}{\#}
\newcommand{\bemolle}{\flat}
\newcommand{\nonstandard}[1]{\prescript{*}{}{#1}}
\newcommand{\starR}{\nonstandard{\R}}
\newcommand{\borel}{\mathscr{B}}
\newcommand{\lebesgue}[1]{\mathscr{L}\left(#1\right)}
\newcommand{\media}{\mathds{E}}
\newcommand{\K}{\mathds{K}}
\newcommand{\A}{\mathds{A}}
\newcommand{\Q}{\mathds{Q}}
\newcommand{\N}{\mathds{N}}
\newcommand{\C}{\mathds{C}}
\newcommand{\Z}{\mathds{Z}}
\newcommand{\qo}{\hspace{1em}\text{q.o.}\,}
\renewcommand{\tilde}[1]{\widetilde{#1}}
\renewcommand{\parallel}{\mathrel{/\mkern-5mu/}}
\newcommand{\parti}[2][]{\wp_{#1}(#2)}
\newcommand{\diff}[1]{\operatorname{d}_{#1}}
\let\oldvec\vec
\renewcommand{\vec}[1]{\overrightarrow{\vphantom{i}#1}}
\newcommand{\floor}[1]{\left\lfloor #1 \right\rfloor}
\newcommand{\cat}[1]{\mathbf{#1}}
\newcommand{\dfreccia}[1]{\xrightarrow{\ #1 \ }}
\newcommand{\sfreccia}[1]{\xleftarrow{\ #1 \ }}
\newcommand{\formalsum}[2]{{\sum_{#1}^{#2}}{\vphantom{\sum}}'}
\newcommand{\minim}[2]{\mu_{#1}\, \left(#2\right)}
\newcommand{\concat}{\null^{\frown}} % concatenazione di stringe
\newcommand{\godelcode}[1]{\langle\!\langle #1 \rangle\!\rangle}
\newcommand{\godeldec}[1]{(\!(#1)\!)}
\newcommand{\termcode}[1]{\ulcorner #1\urcorner}
\newcommand{\partialto}{\dashrightarrow}
\newcommand{\restricted}{\upharpoonright}
\newcommand{\embeds}{\precsim}
\newcommand{\surjects}{\twoheadrightarrow}
\newcommand{\equipotenti}{\asymp}
%% \newcommand{\dotplus}{\mathbin{\dot{+}}} %% A quanto pare esiste già
\newcommand{\bigdot}{\mathbin{\boldsymbol{\cdot}}}
\newcommand{\dotexp}[1]{^{.#1}}
\newcommand{\conv}{\mathbin{*}}
\newcommand{\convolution}[2]{(#1\conv #2)}
\newcommand{\nil}{\mathfrak{N}}
\newcommand{\divisore}{\mathrel{|}}
\newcommand{\simplesso}[1]{\mathrm{e}_{#1}}

\renewcommand{\iff}{\mathrel{\longleftrightarrow}} %% Notazione Logica.
\newcommand{\oldiff}{\mathrel{\Longleftrightarrow}}
\renewcommand{\implies}{\mathrel{\rightarrow}} %% Notazione Logica
\newcommand{\oldimplies}{\mathrel{\Longrightarrow}}
\renewcommand{\impliedby}{\mathrel{\leftarrow}} %% Notazione Logica
\newcommand{\oldimpliedby}{\mathrel{\Longleftarrow}}

\newcommand{\IFF}{\quad\Longleftrightarrow\quad}
\newcommand{\IMPLICA}{\quad\Longrightarrow\quad}


\renewcommand{\descriptionlabel}[1]{\hspace{\labelsep}\normalfont #1} % remove bold from description


%% Definizione di Divergenza di K-L

\DeclarePairedDelimiterX{\infdivx}[2]{(}{)}{%
  #1\;\delimsize\|\;#2%
}
\newcommand{\kldiv}{D_{KL}\infdivx}

%% Definizione di \dotminus

\makeatletter
\newcommand{\dotminus}{\mathbin{\text{\@dotminus}}}

\newcommand{\@dotminus}{%
  \ooalign{\hidewidth\raise1ex\hbox{.}\hidewidth\cr$\m@th-$\cr}%
}
\makeatother

%tramite i prossimi due comandi posso decidere come scrivere i logaritmi naturali in tutti i documenti: ho infatti eliminato qualsiasi differenza tra "ln" e "log": se si vuole qualcosa di diverso bisogna inserire manualmente il tutto
\let\ln\relax
\DeclareMathOperator{\ln}{ln}
\let\log\relax
\DeclareMathOperator{\log}{log}
%%%%%%

%% NUOVI COMANDI
\newcommand{\straniero}[1]{\textit{#1}} %parole straniere
\newcommand{\titolo}[1]{\textsc{#1}} %titoli
\newcommand{\qedd}{\tag*{$\blacksquare$}} %qed per ambienti matemastici
\renewcommand{\qedsymbol}{$\blacksquare$} %modifica colore qed
\newcommand{\ooverline}[1]{\overline{\overline{#1}}}
\newcommand{\circoletto}[1]{\left(#1\right)^{\text{o}}}
%
\newcommand{\qmatrice}[1]{\begin{pmatrix}
#1_{11} & \cdots & #1_{1n}\\
\vdots & \ddots & \vdots \\
#1_{m1} & \cdots & #1_{mn}
\end{pmatrix}}
%
\newcommand{\parentesi}[2]{%
\underset{#1}{\underbrace{#2}}%
}
%
\newcommand{\norma}[1]{% Norma
\left\lVert#1\right\rVert%
}
\newcommand{\scalare}[2]{% Scalare
\left\langle #1, #2\right\rangle
}
%%%%%

%% RESTRIZIONI
\newcommand{\referenze}[2]{
        \phantomsection{}#2\textsuperscript{\textcolor{blue}{\textbf{#1}}}
}

\let\restriction\relax

\def\restriction#1#2{\mathchoice
              {\setbox1\hbox{${\displaystyle #1}_{\scriptstyle #2}$}
              \restrictionaux{#1}{#2}}
              {\setbox1\hbox{${\textstyle #1}_{\scriptstyle #2}$}
              \restrictionaux{#1}{#2}}
              {\setbox1\hbox{${\scriptstyle #1}_{\scriptscriptstyle #2}$}
              \restrictionaux{#1}{#2}}
              {\setbox1\hbox{${\scriptscriptstyle #1}_{\scriptscriptstyle #2}$}
              \restrictionaux{#1}{#2}}}
\def\restrictionaux#1#2{{#1\,\smash{\vrule height .8\ht1 depth .85\dp1}}_{\,#2}}
%%%%%%%%%%%

%%% FORMATTAZIONE FOOTNOTEMARK

\def\footnotemarkformatting#1{[#1]}
\renewcommand{\thefootnote}{\footnotemarkformatting{\arabic{footnote}}}

%% SEZIONE GRAFICA
\use{tikz}
\usetikzlibrary{matrix, patterns, calc, decorations.pathreplacing, hobby, decorations.markings, decorations.pathmorphing, babel}
\use{tikz-3dplot}
\use{mathrsfs} %per geogebra
\use{tikz-cd}
\tikzset
{
  %surface/.style={fill=black!10, shading=ball,fill opacity=0.4},
  plane/.style={black,pattern=north east lines},
  curve/.style={black,line width=0.5mm},
  dritto/.style={decoration={markings,mark=at position 0.5 with {\arrow{Stealth}}}, postaction=decorate},
  rovescio/.style={decoration={markings,mark=at position 0.5 with {\arrow{Stealth[reversed]}}}, postaction=decorate}
}
\use{pgfplots} % stampare le funzioni
        \pgfplotsset{/pgf/number format/use comma,compat=1.15}
        %\pgfplotsset{compat=1.15} %per geogebra
        \usepgfplotslibrary{fillbetween, polar}
%%%%%%

%% CITAZIONI
\use{lineno}

\newcommand{\citazione}[1]{%
  \begin{quotation}
  \begin{linenumbers}
  \modulolinenumbers[5]
  \begingroup
  \setlength{\parindent}{0cm}
  \noindent #1
  \endgroup
  \end{linenumbers}
  \end{quotation}\setcounter{linenumber}{1}
  }
%%%%%%

%%%%%%%%%%%%%%%%%%%%%%%%%%%%%%%%%%%%%%%%%%%%
%%%%%%%%%%%%%%%%%%%%%%%%%%%%%%%%%%%%%%%%%%%%

%% AMS THM

\theoremstyle{definition}% default
\newtheorem{thm}{Teorema}[section]
\newtheorem{lem}[thm]{Lemma}
\newtheorem{prop}[thm]{Proposizione}
\newtheorem{cor}[thm]{Corollario}
\newtheorem{esempio}[thm]{Esempio}
\theoremstyle{plain}
\newtheorem{definizione}[thm]{Definizione}
\theoremstyle{remark}
\newtheorem*{oss}{Osservazione}


%%%%%%%%%%%%%%%%%%%%%%%%%%%%%%%%%%%%%%%%%%%%
%%%%%%%%%%%%%%%%%%%%%%%%%%%%%%%%%%%%%%%%%%%%

\use{hyperref}
\hypersetup{%
        pdfauthor={Davide Peccioli},
        pdfsubject={},
        allcolors=black,
        citecolor=black,
%	colorlinks=true,
        bookmarksopen=true}
\setcounter{secnumdepth}{0} % rimuove i numeri di sezione senza rimuovere le ref
\renewcommand{\href}[2]{\textcolor{blue}{#2}} % disabilita il comando href
\use{enotez} %
\setenotez{%
 mark-format = \footnotemarkformatting % Mette i numeri tra parentesi quadre%
}\let\footnote=\endnote % rende tutte le note a pié pagina come delle note a fine file 


\let\olddocument\document % modifico l'ambiende documenti per non dover stampare \printendnote
\let\oldenddocument\enddocument
\renewenvironment{document}%
{%
  \olddocument
}{%
  \printendnotes\oldenddocument
}
\renewcommand{\thethm}{\arabic{thm}}

\usepackage[hyperref]{biblatex}
\addbibresource{~/Documents/org/roam/bib/master.bib}
\author{Davide Peccioli}
\date{\today}
\title{(bib) FIPSAS - ``P1 - Lezione T5 - Apparato respiratorio e circolatorio''}
\begin{document}

\section{Approfondimento SLIDES}
\label{sec:org779c8ef}

Approfondimento fatto tramite \autocites{fipsasManualeCorso2deg2017}[][]{ManP1_250621}
\subsection{Slide 1: FIPSAS - 1\textsuperscript{o} GRADO AR: ARGOMENTI DEL CORSO}
\label{sec:org5b4b3c0}
\subsubsection{Contenuto della Slide (Riepilogo)}
\label{sec:org44839ab}
La slide elenca gli argomenti principali del corso \textbf{\textbf{FIPSAS 1\textsuperscript{o} Grado AR}} (Autorespiratore ad Aria). Gli argomenti includono: Attrezzatura di base, Cenni di Fisica per il subacqueo, Apparato uditivo e compensazione, La vista in immersione, Attrezzatura ARA, Apparato respiratorio e circolatorio, Assorbimento e rilascio di gas in immersione, Comportamento e tecnica d’immersione, Tabelle d’immersione, e Pianificazione dell’immersione.
\subsubsection{Approfondimento dal Manuale P1}
\label{sec:orgb46ea02}
Il corso di \textbf{\textbf{1\textsuperscript{o} Grado AR (P1)}} abilita il subacqueo a immergersi fino a una profondità massima di \textbf{\textbf{18 metri}}, sempre accompagnato da almeno un compagno. Il percorso didattico completo della FIPSAS per gli autorespiratori si articola in tre livelli base (1\textsuperscript{o}, 2\textsuperscript{o} e 3\textsuperscript{o} grado AR), che abilitano a profondità crescenti: 18 m, 30 m e 42 m.

Il \textbf{\textbf{Manuale P1}} funge da strumento di studio fondamentale, contenendo in forma estesa tutte le informazioni che l'istruttore illustrerà in aula. La struttura teorica del corso si articola in 12 capitoli che riflettono gli argomenti elencati nella slide, aggiungendo in particolare i cenni sull'ambiente marino.

Il corso include una parte teorica e una pratica in acqua. L'addestramento in piscina (o altro bacino delimitato) è fondamentale per acquisire l'acquaticità e sviluppare le abilità richieste in immersione in un ambiente sicuro e a bassa profondità. L'addestramento in acque libere (mare o lago) è indispensabile per \textbf{\textbf{consolidare le abilità}} nell'ambiente effettivo di immersione, sempre sotto la continua supervisione dell'istruttore, per diventare subacquei consapevoli e immergersi in sicurezza.
\subsection{Slide 2: CENNI DI ANATOMIA E FISIOLOGIA - Cellule (p. 71)}
\label{sec:org3277fe0}
\subsubsection{Contenuto della Slide (Riepilogo)}
\label{sec:orgb3d30da}
Introduzione al concetto di \textbf{\textbf{Cellule}} come unità base degli organismi e la loro funzione vitale (nascere, nutrirsi, riprodursi, morire). Viene definita la formula del \textbf{\textbf{Metabolismo cellulare}}: \(\text{O}_2 + \text{nutrienti} \rightarrow \text{CO}_2 + \text{scorie} + \text{energia}\).
\subsubsection{Approfondimento dal Manuale P1 / Manuale Federale d’Immersione}
\label{sec:org8d95d39}
Le cellule sono l'unità funzionale della vita. Il \textbf{\textbf{Metabolismo cellulare}} è il processo di ``combustione'' di sostanze energetiche (come \textbf{\textbf{zuccheri e grassi}}) all'interno della cellula per ricavare l'\textbf{\textbf{energia}} indispensabile.

Questo processo vitale genera calore e i sottoprodotti di scarto: \textbf{\textbf{anidride carbonica (\(\text{CO}_2\)) e scorie}}. Il metabolismo stabilisce la necessità fisiologica di un costante apporto di \textbf{\textbf{Ossigeno (\(\text{O}_2\))}} e di una rimozione efficace della \(\text{CO}_2\).

L'azoto (\(\text{N}_2\)), che costituisce circa il 78\% dell'aria respirata, è un \textbf{\textbf{gas inerte}} e \textbf{\textbf{non interviene nei processi metabolici}}. L'Ossigeno, invece, rappresenta circa il 21\% dell'aria inspirata e viene utilizzato nei processi metabolici.
\subsection{Slide 3: CENNI DI ANATOMIA E FISIOLOGIA - Tessuti, Organi, Apparati}
\label{sec:orgcac229c}
\subsubsection{Contenuto della Slide (Riepilogo)}
\label{sec:orgccadd95}
Viene presentata la gerarchia strutturale del corpo:
\begin{itemize}
\item \textbf{\textbf{Tessuti:}} Insieme di cellule (es. Sangue, Adiposo, Nervoso, Muscolare, Cartilagineo, Osseo).
\item \textbf{\textbf{Organi:}} Insieme di tessuti (es. Occhio, Orecchio, Polmone, Cuore).
\item \textbf{\textbf{Apparati:}} Insiemi di organi (es. Visivo, Uditivo, Respiratorio, Circolatorio).
\end{itemize}
\subsubsection{Approfondimento dal Manuale P1}
\label{sec:org90ece2c}
Sebbene il Manuale P1 non approfondisca specificamente la definizione di tessuti o apparati in questo punto, esso tratta in dettaglio gli apparati cruciali per il subacqueo, come evidenziato dalla struttura del corso:
\begin{enumerate}
\item \textbf{\textbf{Orecchio e compensazione (T3):}} L'orecchio, in particolare l'orecchio medio, è trattato ampiamente in relazione agli effetti della pressione.
\item \textbf{\textbf{Vista in immersione (T4):}} Tratta le alterazioni percettive causate dall'ambiente subacqueo (distorsione delle dimensioni e delle distanze, riduzione del contrasto e del rilievo cromatico) e la funzione della maschera.
\item \textbf{\textbf{Apparato respiratorio e cardiocircolatorio (T6):}} Questi sistemi sono fondamentali per sostenere il metabolismo cellulare, garantendo lo scambio \(\text{O}_2\)/\(\text{CO}_2\) e il trasporto nel corpo.
\end{enumerate}
\subsection{Slide 4: APPARATO RESPIRATORIO - Funzione e Costituzione (p. 71)}
\label{sec:org4e8c51f}
\subsubsection{Contenuto della Slide (Riepilogo)}
\label{sec:orge765026}
\textbf{\textbf{Funzione:}} Fornire \(\text{O}_2\) al sangue ed eliminare la \(\text{CO}_2\).
\textbf{\textbf{Costituzione - Vie aeree superiori:}} Naso, seni paranasali, bocca, faringe, laringe.
\textbf{\textbf{Costituzione - Vie aeree inferiori:}} Trachea, bronchi, polmoni.
\subsubsection{Approfondimento dal Manuale Federale d’Immersione}
\label{sec:orgc88af91}
L'apparato respiratorio ha il compito di mettere in atto la \textbf{\textbf{fase meccanica}} della respirazione (richiamo dell'aria) e lo \textbf{\textbf{scambio gassoso}} (captazione di \(\text{O}_2\) ed eliminazione di \(\text{CO}_2\)).

La respirazione è guidata primariamente dalla necessità di smaltire la \(\text{CO}_2\). Lo \textbf{\textbf{stimolo respiratorio}} è dato principalmente dall’incremento della \textbf{\textbf{pressione parziale dell’anidride carbonica}} nel sangue. Specifiche cellule nervose nel midollo allungato (nuclei chemiotattici) sono sensibili all’eccesso di \(\text{CO}_2\) (ipercapnia) e, in misura minore, alla mancanza di \(\text{O}_2\) (ipossia), e da qui partono i comandi per i muscoli respiratori.
\subsection{Slide 5: APPARATO RESPIRATORIO - Vie Aeree Inferiori (Dettaglio)}
\label{sec:org6edc573}
\subsubsection{Contenuto della Slide (Riepilogo)}
\label{sec:orgd12adc3}
Viene fornita una lista sequenziale delle vie aeree inferiori, culminanti negli \textbf{\textbf{alveoli}} dove avvengono gli scambi, e l'elenco delle strutture associate:
\begin{itemize}
\item Trachea (1)
\item Bronchi (2)
\item Polmoni (3)
\item Bronchioli (4)
\item Lobuli (5)
\item Alveoli (6)
\item Pleure (7) e Diaframma (8).
\end{itemize}
\subsubsection{Approfondimento dal Manuale P1 / Manuale Federale d’Immersione}
\label{sec:orge624cb4}
Le vie aeree inferiori (trachea, bronchi, bronchioli e alveoli) contengono l'\textbf{\textbf{aria ventilata a vuoto}} che non partecipa agli scambi gassosi, nota come \textbf{\textbf{Spazio morto bronco tracheale}}.

Gli alveoli sono il sito primario dello scambio gassoso. Il \textbf{\textbf{diaframma}} è un muscolo fondamentale per la respirazione.
\subsection{Slide 6: APPARATO RESPIRATORIO - Respirazione: Inspirazione (Fase attiva)}
\label{sec:orge806450}
\subsubsection{Contenuto della Slide (Riepilogo)}
\label{sec:org3112366}
L'\textbf{\textbf{Inspirazione}} è la \textbf{\textbf{fase attiva}}. Si verifica quando il \textbf{\textbf{diaframma si contrae e si abbassa}}, espandendo la gabbia toracica e permettendo all’aria di entrare nei polmoni.
\subsubsection{Approfondimento dal Manuale P1}
\label{sec:org00f2ba5}
(Non si trovano dettagli aggiuntivi sul processo meccanico dell'inspirazione oltre quanto già riportato).
\subsection{Slide 7: APPARATO RESPIRATORIO - Respirazione: Espirazione (Fase passiva)}
\label{sec:org604f8aa}
\subsubsection{Contenuto della Slide (Riepilogo)}
\label{sec:orgc426c34}
L'\textbf{\textbf{Espirazione}} è la \textbf{\textbf{fase passiva}}. Il diaframma si rilassa e si alza; la cassa toracica si rilassa (o si comprime).
\subsubsection{Frequenza respiratoria media: 12-18 atti al minuto.}
\label{sec:org6a7168c}
\subsubsection{Volume di ventilazione totale: 7-8 litri al minuto.}
\label{sec:org885354b}
\subsubsection{Approfondimento dal Manuale P1}
\label{sec:org8ea6fb8}
Il manuale sottolinea che la respirazione deve essere \textbf{\textbf{regolare e continua}} e, per il subacqueo, è raccomandato che l'\textbf{\textbf{espirazione sia più lunga dell’inspirazione}}.

In caso di affanno (condizione in cui la \(\text{CO}_2\) ristagna nella parte inferiore dei polmoni), si deve \textbf{\textbf{interrompere qualsiasi attività}}, assumere una posizione orizzontale, e se possibile, facilitare la respirazione utilizzando il pulsante del secondo stadio dell'erogatore.
\subsection{Slide 8: APPARATO RESPIRATORIO - Volumi e Capacità Polmonari}
\label{sec:orgf5316fa}
\subsubsection{Contenuto della Slide (Riepilogo)}
\label{sec:orgaf07756}
\begin{itemize}
\item \textbf{\textbf{Volume polmonare medio totale:}} 5500 cc (5.5 litri).
\item \textbf{\textbf{Capacità vitale:}} 4000 cc (Volume di ventilazione 500 cc + Volume di riserva inspiratoria 2500 cc + Volume di riserva espiratoria 1000 cc).
\item \textbf{\textbf{Volume residuo:}} 1350 cc.
\item \textbf{\textbf{Spazio morto bronco tracheale:}} 150 cc.
\end{itemize}
\subsubsection{Approfondimento dal Manuale P1 / Manuale Federale d’Immersione}
\label{sec:orgdb281c5}
Il \textbf{\textbf{volume dei polmoni (Capacità polmonare)}} è stimato in circa \textbf{\textbf{5500 cc (5,5 litri)}}.

Il concetto di \textbf{\textbf{Spazio morto bronco tracheale}} (150 cc) è importante perché questo è il volume d’aria contenuto nelle vie aeree che \textbf{\textbf{non partecipa agli scambi gassosi}} e viene ventilato a vuoto ad ogni atto respiratorio.
\subsection{Slide 9: APPARATO RESPIRATORIO - Composizione Aria e Ruolo dell'Azoto}
\label{sec:org90b17c9}
\subsubsection{Contenuto della Slide (Riepilogo)}
\label{sec:org5bbd546}
\begin{itemize}
\item \textbf{\textbf{Aria inspirata:}} N2 (Azoto) 79\%, O2 (Ossigeno) 21\%, CO2 trascurabile.
\item \textbf{\textbf{Aria espirata:}} O2 16\%, CO2 5\%.
\item \textbf{\textbf{Azoto (N2):}} Gas inerte, non partecipa al metabolismo cellulare.
\item \textbf{\textbf{Stimolo respiratorio:}} Dato dal livello di \(\text{CO}_2\). L'eccesso di \(\text{CO}_2\) (ipercapnia) produce affanno.
\end{itemize}
\subsubsection{Approfondimento dal Manuale P1 / Manuale Federale d’Immersione}
\label{sec:orged918fb}
L'aria che respiriamo è composta da circa \textbf{\textbf{78\% di Azoto (\(\text{N}_2\))}} e \textbf{\textbf{21\% di Ossigeno (\(\text{O}_2\))}}. L'azoto è un gas inerte, ma la sua esposizione a pressioni superiori a quella atmosferica durante l'immersione fa sì che venga assorbito dai tessuti.

La \(\text{CO}_2\) è il principale regolatore della respirazione. Un aumento della concentrazione di anidride carbonica nel sangue determina l'aumento della frequenza respiratoria per eliminarla. Il subacqueo deve stare attento all'\textbf{\textbf{affanno}}, una condizione in cui la \(\text{CO}_2\) ristagna nella parte inferiore dei polmoni, spesso causata da una respirazione tecnicamente non corretta.

La tossicità dei gas è legata alla loro pressione parziale (\(\text{p}\)), secondo la \textbf{\textbf{Legge di Dalton}} (\(\text{p} = \text{f} \times \text{P}\)):
\begin{itemize}
\item \textbf{\textbf{Ossigeno:}} Diventa tossico ad elevate pressioni parziali. Il limite universalmente accettato è \textbf{\textbf{1.6 atm}} per 45 minuti. Se respiriamo aria (f = 0.21), una \(\text{pO}_2\) di 1.6 atm si raggiunge teoricamente a \textbf{\textbf{66 metri}}.
\item \textbf{\textbf{Azoto:}} Ad elevate pressioni parziali causa la \textbf{\textbf{narcosi da azoto}} (o ebbrezza da profondità o ``martini effect''). La pressione parziale massima sicura per l'azoto in immersione sportiva è circa \textbf{\textbf{4.3 atm}}, corrispondente a circa \textbf{\textbf{45 metri}} di profondità in aria.
\end{itemize}
\subsection{Slide 10: APPARATO RESPIRATORIO - Respirazione in Immersione}
\label{sec:org004a0c5}
\subsubsection{Contenuto della Slide (Riepilogo)}
\label{sec:orgaae35b8}
\textbf{\textbf{In immersione la respirazione deve essere:}}
\begin{enumerate}
\item Regolare e continua.
\item Senza pause pronunciate.
\item L'espirazione deve essere più lunga dell’inspirazione (idealmente 1 volta e \(\frac{1}{2}\)).
\item Consigliata la respirazione a ``dente di sega''.
\end{enumerate}
\subsubsection{Approfondimento dal Manuale P1}
\label{sec:orgc52ce73}
La regola di non trattenere il respiro, specialmente in risalita, è fondamentale per la sicurezza.

Se il subacqueo con autorespiratori non espira durante la risalita, l'aria contenuta nei polmoni si espande a causa della diminuzione della pressione ambiente (Legge di Boyle-Mariotte). Questa espansione forzata (sovradistensione polmonare), non trovando via d'uscita, può \textbf{\textbf{lacerare il tessuto polmonare}}, causando l'ingresso d'aria nel circolo sanguigno arterioso e portando all'\textbf{\textbf{Embolia Gassosa Arteriosa (EGA)}}. Questo incidente, tra i più gravi, può teoricamente verificarsi anche risalendo da poco più di un metro di profondità.
\subsection{Slide 11: APPARATO CIRCOLATORIO - Funzione e Costituzione (p. 75)}
\label{sec:orgc0a2f7f}
\subsubsection{Contenuto della Slide (Riepilogo)}
\label{sec:orgaba68eb}
\textbf{\textbf{Funzione:}} Distribuire l'\(\text{O}_2\) al corpo e portare la \(\text{CO}_2\) ai polmoni.
\textbf{\textbf{Costituzione:}}
\begin{enumerate}
\item Cuore (pompa).
\item Arterie (vasi che partono dal cuore).
\item Vene (vasi che arrivano al cuore).
\item Sangue (arterioso ricco di \(\text{O}_2\), venoso ricco di \(\text{CO}_2\)).
\end{enumerate}
\subsubsection{Approfondimento dal Manuale P1 / Manuale Federale d’Immersione}
\label{sec:org1853b32}
La funzione del sangue è cruciale. Oltre ai globuli rossi, il sangue contiene \textbf{\textbf{piastrine e globuli bianchi}}. Le piastrine, che servono a far coagulare il sangue, e i globuli bianchi, coinvolti nella risposta immunitaria, hanno un ruolo nell’evoluzione della \textbf{\textbf{Malattia da Decompressione (MDD)}}.

L'apparato circolatorio è l'intermediario tra lo scambio gassoso nei polmoni e l'uso dei gas nelle cellule (metabolismo cellulare).
\subsection{Slide 12: APPARATO CIRCOLATORIO - Grande e Piccolo Circolo}
\label{sec:orgbca3872}
\subsubsection{Contenuto della Slide (Riepilogo)}
\label{sec:orgafcda5a}
\begin{enumerate}
\item \textbf{\textbf{Piccolo Circolo (Circolazione polmonare):}} VD \(\rightarrow\) Polmoni (scambio gas) \(\rightarrow\) AS.
\item \textbf{\textbf{Grande Circolo (Circolazione sistemica):}} VS \(\rightarrow\) Corpo (nutrimento cellule) \(\rightarrow\) AD.
\end{enumerate}
\subsubsection{Approfondimento dal Manuale P1 / Manuale Federale d’Immersione}
\label{sec:orgfa40938}
Il \textbf{\textbf{Piccolo Circolo}} è l'insieme dei vasi sanguigni tra il ventricolo destro (VD) e l'atrio sinistro (AS). Un aspetto interessante è che in questo circolo \textbf{\textbf{le vene contengono sangue arterioso e le arterie contengono sangue venoso}}.

Le valvole presenti nel cuore e nei vasi sanguigni assicurano che il \textbf{\textbf{verso della circolazione non possa invertirsi}}.
\subsection{Appendice: Principi Fisici e Fisiologici Correlati al Contenuto delle Slide}
\label{sec:org026cb05}

Sebbene le slide si concentrino sull'anatomia e fisiologia di base, per un subacqueo tali concetti sono indissolubilmente legati alle leggi fisiche e agli effetti iperbarici:

\textbf{\textbf{Compensazione e Boyle-Mariotte:}}
La legge di Boyle-Mariotte (\(\text{P} \times \text{V} = \text{k}\)) spiega che all'aumentare della pressione (in discesa), il volume dei gas diminuisce. Questa legge è fondamentale per la sicurezza perché spiega la necessità di:
\begin{enumerate}
\item \textbf{\textbf{Compensazione dell'orecchio medio:}} Scendendo, il volume d'aria nell'orecchio medio (normalmente chiuso dalla tuba di Eustachio) diminuisce, causando introflessione del timpano, fastidio e potenziale dolore/lacerazione. La manovra di compensazione (es. Valsalva o Marcante-Odaglia) introduce aria nell'orecchio medio attraverso la tuba per ristabilire l'equilibrio pressorio.
\item \textbf{\textbf{Compensazione della maschera:}} La maschera racchiude il naso, permettendo la compensazione di questo spazio, che altrimenti si schiaccerebbe contro il viso.
\end{enumerate}

\textbf{\textbf{Assorbimento dei Gas e Legge di Henry:}}
La \textbf{\textbf{Legge di Henry}} afferma che, a temperatura costante, la quantità di gas che si scioglie in un liquido è \textbf{\textbf{direttamente proporzionale alla pressione}} che il gas esercita sul liquido. Questa legge governa l'assorbimento dell'azoto nei tessuti corporei durante l'immersione:
\begin{itemize}
\item In immersione, l'organismo assorbe azoto in quantità proporzionale alla pressione.
\item Durante la risalita, l'azoto assorbito deve essere rilasciato in modo controllato, altrimenti può portare alla \textbf{\textbf{formazione di bolle gassose}} (MDD - Malattia da Decompressione).
\item L'azoto viene assorbito dai tessuti con velocità proporzionale al loro \textbf{\textbf{grado di vascolarizzazione}}. La cessione (desaturazione) avviene secondo il percorso inverso.
\item Per prevenire la MDD, è fondamentale rispettare una \textbf{\textbf{corretta velocità di risalita}} (9 m/min) e le \textbf{\textbf{soste di decompressione}} (se fuori curva).
\end{itemize}
\printbibliography
\end{document}
