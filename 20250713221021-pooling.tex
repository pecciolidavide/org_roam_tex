% Created 2026-02-07 Sat 19:31
% Intended LaTeX compiler: pdflatex
\documentclass[10pt]{article}
%% CREATO CON ORG - EMACS
\newcommand{\use}[2][]{\usepackage[#1]{#2}}
% PACCHETTI FONDAMENTLAI
\use[utf8]{inputenc}
\use[T1]{fontenc}
\use{graphicx}
\use{longtable}
\use{wrapfig}
\use{rotating}
\use[normalem]{ulem}
\use{amsmath}
\use{amsthm}
\use{amssymb}

\use{eucal} % Cambia mathcal{...}

\use{capt-of}
\use[italian]{babel}
\use[babel]{csquotes}
% bib la TEX lo carica in automatico org-cite
\use{microtype}
\use{lmodern}
\use{subfig} % sottofigure
\use{multicol} % due colonne
\use{lipsum} % lorem ipsum
\use{color} % colori in latex
\use{parskip} % rimuove l'indentazione dei nuovi paragrafi %% Add parbox=false to all new tcolorbox
\use{centernot}
\use[outline]{contour}\contourlength{3pt}
\use{fancyhdr}
\use{layout}
\use[most]{tcolorbox} % Riquadri colorati
\use{ifthen} % IFTHEN
\use{geometry}

% pacchetti matematica
\use{yhmath}
\use{dsfont}
\use{mathrsfs}
\use{cancel} % semplificare
\use{polynom} %divisione tra polinomi
\use{forest} % grafi ad albero
\use{booktabs} % tabelle
\use{commath} %simboli e differenziali
\use{bm} %bold
\use[fulladjust]{marginnote} %to use marginnote for date notes
\use{arrayjobx}%array
\use[intlimits]{empheq} % Riquadri colorati attorno alle equazioni
\use{mathtools}
\use{circuitikz} % Disegnare i circuiti
\use{mathtools}
\use{stmaryrd} % [[ \llbracket ]] \rrbracket
\use{bussproofs} % dimostrazioni

%%%%%%%%%%%%%


%%%% QUIVER
\newcommand{\duepunti}{\,\mathchar\numexpr"6000+`:\relax\,}
% A TikZ style for curved arrows of a fixed height, due to AndréC.
\tikzset{curve/.style={settings={#1},to path={(\tikztostart)
    .. controls ($(\tikztostart)!\pv{pos}!(\tikztotarget)!\pv{height}!270:(\tikztotarget)$)
    and ($(\tikztostart)!1-\pv{pos}!(\tikztotarget)!\pv{height}!270:(\tikztotarget)$)
    .. (\tikztotarget)\tikztonodes}},
    settings/.code={\tikzset{quiver/.cd,#1}
        \def\pv##1{\pgfkeysvalueof{/tikz/quiver/##1}}},
    quiver/.cd,pos/.initial=0.35,height/.initial=0}

% TikZ arrowhead/tail styles.
\tikzset{tail reversed/.code={\pgfsetarrowsstart{tikzcd to}}}
\tikzset{2tail/.code={\pgfsetarrowsstart{Implies[reversed]}}}
\tikzset{2tail reversed/.code={\pgfsetarrowsstart{Implies}}}
% TikZ arrow styles.
\tikzset{no body/.style={/tikz/dash pattern=on 0 off 1mm}}
%%%%%%%%%%


%% DEFINIZIONI COMANDI MATEMATICI
\let\sin\relax %TOGLIE LA DEFINIZIONE SU "\sin"

% cambia la definizione di empty set
% ---
\let\oldemptyset\emptyset
% ---
% \let\emptyset\varnothing
% ---
% \let\emptyset\relax
% \newcommand{\emptyset}{\text{\textnormal{\O}}}
% ---

\DeclareMathOperator{\bounded}{bd}
\DeclareMathOperator{\sin}{sen}
\DeclareMathOperator{\epi}{Epi}
\DeclareMathOperator{\cl}{cl}
\DeclareMathOperator{\graph}{graph}
\DeclareMathOperator{\arcsec}{arcsec}
\DeclareMathOperator{\arccot}{arccot}
\DeclareMathOperator{\arccsc}{arccsc}
\DeclareMathOperator{\spettro}{Spettro}
\DeclareMathOperator{\nulls}{nullspace}
\DeclareMathOperator{\dom}{dom}
\DeclareMathOperator{\ar}{ar}
\DeclareMathOperator{\const}{Const}
\DeclareMathOperator{\fun}{Fun}
\DeclareMathOperator{\rel}{Rel}
\DeclareMathOperator{\altezza}{ht}
\let\det\relax %TOGLIE LA DEFINIZIONE SU "\det"
\DeclareMathOperator{\det}{det}
\DeclareMathOperator{\End}{End}
\DeclareMathOperator{\gl}{GL}
\def\Id{\mathrm{Id}}
\def\id{\mathrm{id}}
\DeclareMathOperator{\I}{\mathds{1}}
\DeclareMathOperator{\II}{II}
\DeclareMathOperator{\rank}{rank}
\DeclareMathOperator{\tr}{tr}
\DeclareMathOperator{\tc}{t.c.}
\DeclareMathOperator{\T}{T}
\DeclareMathOperator{\var}{Var}
\DeclareMathOperator{\cov}{Cov}
\DeclareMathOperator{\st}{st}
\DeclareMathOperator{\mon}{Mon}
\newcommand{\card}[1]{\left\vert #1 \right\vert}
\newcommand{\trasposta}[1]{\prescript{\text{T}}{}{#1}}
\newcommand{\1}{\mathds{1}}
\newcommand{\R}{\mathds{R}}
\newcommand{\diesis}{\#}
\newcommand{\bemolle}{\flat}
\newcommand{\nonstandard}[1]{\prescript{*}{}{#1}}
\newcommand{\starR}{\nonstandard{\R}}
\newcommand{\borel}{\mathscr{B}}
\newcommand{\lebesgue}[1]{\mathscr{L}\left(#1\right)}
\newcommand{\media}{\mathds{E}}
\newcommand{\K}{\mathds{K}}
\newcommand{\A}{\mathds{A}}
\newcommand{\Q}{\mathds{Q}}
\newcommand{\N}{\mathds{N}}
\newcommand{\C}{\mathds{C}}
\newcommand{\Z}{\mathds{Z}}
\newcommand{\qo}{\hspace{1em}\text{q.o.}\,}
\renewcommand{\tilde}[1]{\widetilde{#1}}
\renewcommand{\parallel}{\mathrel{/\mkern-5mu/}}
\newcommand{\parti}[2][]{\wp_{#1}(#2)}
\newcommand{\diff}[1]{\operatorname{d}_{#1}}
\let\oldvec\vec
\renewcommand{\vec}[1]{\overrightarrow{\vphantom{i}#1}}
\newcommand{\floor}[1]{\left\lfloor #1 \right\rfloor}
\newcommand{\cat}[1]{\mathbf{#1}}
\newcommand{\dfreccia}[1]{\xrightarrow{\ #1 \ }}
\newcommand{\sfreccia}[1]{\xleftarrow{\ #1 \ }}
\newcommand{\formalsum}[2]{{\sum_{#1}^{#2}}{\vphantom{\sum}}'}
\newcommand{\minim}[2]{\mu_{#1}\, \left(#2\right)}
\newcommand{\concat}{\null^{\frown}} % concatenazione di stringe
\newcommand{\godelcode}[1]{\langle\!\langle #1 \rangle\!\rangle}
\newcommand{\godeldec}[1]{(\!(#1)\!)}
\newcommand{\termcode}[1]{\ulcorner #1\urcorner}
\newcommand{\partialto}{\dashrightarrow}
\newcommand{\restricted}{\upharpoonright}
\newcommand{\embeds}{\precsim}
\newcommand{\surjects}{\twoheadrightarrow}
\newcommand{\equipotenti}{\asymp}
%% \newcommand{\dotplus}{\mathbin{\dot{+}}} %% A quanto pare esiste già
\newcommand{\bigdot}{\mathbin{\boldsymbol{\cdot}}}
\newcommand{\dotexp}[1]{^{.#1}}
\newcommand{\conv}{\mathbin{*}}
\newcommand{\convolution}[2]{(#1\conv #2)}
\newcommand{\nil}{\mathfrak{N}}
\newcommand{\divisore}{\mathrel{|}}
\newcommand{\simplesso}[1]{\mathrm{e}_{#1}}

\renewcommand{\iff}{\mathrel{\longleftrightarrow}} %% Notazione Logica.
\newcommand{\oldiff}{\mathrel{\Longleftrightarrow}}
\renewcommand{\implies}{\mathrel{\rightarrow}} %% Notazione Logica
\newcommand{\oldimplies}{\mathrel{\Longrightarrow}}
\renewcommand{\impliedby}{\mathrel{\leftarrow}} %% Notazione Logica
\newcommand{\oldimpliedby}{\mathrel{\Longleftarrow}}

\newcommand{\IFF}{\quad\Longleftrightarrow\quad}
\newcommand{\IMPLICA}{\quad\Longrightarrow\quad}


\renewcommand{\descriptionlabel}[1]{\hspace{\labelsep}\normalfont #1} % remove bold from description


%% Definizione di Divergenza di K-L

\DeclarePairedDelimiterX{\infdivx}[2]{(}{)}{%
  #1\;\delimsize\|\;#2%
}
\newcommand{\kldiv}{D_{KL}\infdivx}

%% Definizione di \dotminus

\makeatletter
\newcommand{\dotminus}{\mathbin{\text{\@dotminus}}}

\newcommand{\@dotminus}{%
  \ooalign{\hidewidth\raise1ex\hbox{.}\hidewidth\cr$\m@th-$\cr}%
}
\makeatother

%tramite i prossimi due comandi posso decidere come scrivere i logaritmi naturali in tutti i documenti: ho infatti eliminato qualsiasi differenza tra "ln" e "log": se si vuole qualcosa di diverso bisogna inserire manualmente il tutto
\let\ln\relax
\DeclareMathOperator{\ln}{ln}
\let\log\relax
\DeclareMathOperator{\log}{log}
%%%%%%

%% NUOVI COMANDI
\newcommand{\straniero}[1]{\textit{#1}} %parole straniere
\newcommand{\titolo}[1]{\textsc{#1}} %titoli
\newcommand{\qedd}{\tag*{$\blacksquare$}} %qed per ambienti matemastici
\renewcommand{\qedsymbol}{$\blacksquare$} %modifica colore qed
\newcommand{\ooverline}[1]{\overline{\overline{#1}}}
\newcommand{\circoletto}[1]{\left(#1\right)^{\text{o}}}
%
\newcommand{\qmatrice}[1]{\begin{pmatrix}
#1_{11} & \cdots & #1_{1n}\\
\vdots & \ddots & \vdots \\
#1_{m1} & \cdots & #1_{mn}
\end{pmatrix}}
%
\newcommand{\parentesi}[2]{%
\underset{#1}{\underbrace{#2}}%
}
%
\newcommand{\norma}[1]{% Norma
\left\lVert#1\right\rVert%
}
\newcommand{\scalare}[2]{% Scalare
\left\langle #1, #2\right\rangle
}
%%%%%

%% RESTRIZIONI
\newcommand{\referenze}[2]{
        \phantomsection{}#2\textsuperscript{\textcolor{blue}{\textbf{#1}}}
}

\let\restriction\relax

\def\restriction#1#2{\mathchoice
              {\setbox1\hbox{${\displaystyle #1}_{\scriptstyle #2}$}
              \restrictionaux{#1}{#2}}
              {\setbox1\hbox{${\textstyle #1}_{\scriptstyle #2}$}
              \restrictionaux{#1}{#2}}
              {\setbox1\hbox{${\scriptstyle #1}_{\scriptscriptstyle #2}$}
              \restrictionaux{#1}{#2}}
              {\setbox1\hbox{${\scriptscriptstyle #1}_{\scriptscriptstyle #2}$}
              \restrictionaux{#1}{#2}}}
\def\restrictionaux#1#2{{#1\,\smash{\vrule height .8\ht1 depth .85\dp1}}_{\,#2}}
%%%%%%%%%%%

%%% FORMATTAZIONE FOOTNOTEMARK

\def\footnotemarkformatting#1{[#1]}
\renewcommand{\thefootnote}{\footnotemarkformatting{\arabic{footnote}}}

%% SEZIONE GRAFICA
\use{tikz}
\usetikzlibrary{matrix, patterns, calc, decorations.pathreplacing, hobby, decorations.markings, decorations.pathmorphing, babel}
\use{tikz-3dplot}
\use{mathrsfs} %per geogebra
\use{tikz-cd}
\tikzset
{
  %surface/.style={fill=black!10, shading=ball,fill opacity=0.4},
  plane/.style={black,pattern=north east lines},
  curve/.style={black,line width=0.5mm},
  dritto/.style={decoration={markings,mark=at position 0.5 with {\arrow{Stealth}}}, postaction=decorate},
  rovescio/.style={decoration={markings,mark=at position 0.5 with {\arrow{Stealth[reversed]}}}, postaction=decorate}
}
\use{pgfplots} % stampare le funzioni
        \pgfplotsset{/pgf/number format/use comma,compat=1.15}
        %\pgfplotsset{compat=1.15} %per geogebra
        \usepgfplotslibrary{fillbetween, polar}
%%%%%%

%% CITAZIONI
\use{lineno}

\newcommand{\citazione}[1]{%
  \begin{quotation}
  \begin{linenumbers}
  \modulolinenumbers[5]
  \begingroup
  \setlength{\parindent}{0cm}
  \noindent #1
  \endgroup
  \end{linenumbers}
  \end{quotation}\setcounter{linenumber}{1}
  }
%%%%%%

%%%%%%%%%%%%%%%%%%%%%%%%%%%%%%%%%%%%%%%%%%%%
%%%%%%%%%%%%%%%%%%%%%%%%%%%%%%%%%%%%%%%%%%%%

%% AMS THM

\theoremstyle{definition}% default
\newtheorem{thm}{Teorema}[section]
\newtheorem{lem}[thm]{Lemma}
\newtheorem{prop}[thm]{Proposizione}
\newtheorem{cor}[thm]{Corollario}
\newtheorem{esempio}[thm]{Esempio}
\theoremstyle{plain}
\newtheorem{definizione}[thm]{Definizione}
\theoremstyle{remark}
\newtheorem*{oss}{Osservazione}


%%%%%%%%%%%%%%%%%%%%%%%%%%%%%%%%%%%%%%%%%%%%
%%%%%%%%%%%%%%%%%%%%%%%%%%%%%%%%%%%%%%%%%%%%

\use{hyperref}
\hypersetup{%
        pdfauthor={Davide Peccioli},
        pdfsubject={},
        allcolors=black,
        citecolor=black,
%	colorlinks=true,
        bookmarksopen=true}
\setcounter{secnumdepth}{0} % rimuove i numeri di sezione senza rimuovere le ref
\renewcommand{\href}[2]{\textcolor{blue}{#2}} % disabilita il comando href
\use{enotez} %
\setenotez{%
 mark-format = \footnotemarkformatting % Mette i numeri tra parentesi quadre%
}\let\footnote=\endnote % rende tutte le note a pié pagina come delle note a fine file 


\let\olddocument\document % modifico l'ambiende documenti per non dover stampare \printendnote
\let\oldenddocument\enddocument
\renewenvironment{document}%
{%
  \olddocument
}{%
  \printendnotes\oldenddocument
}
\renewcommand{\thethm}{\arabic{thm}}

\usepackage[hyperref]{biblatex}
\addbibresource{~/Documents/org/roam/bib/master.bib}
\author{Davide Peccioli}
\date{\today}
\title{}
\begin{document}

\section{Pooling}
\label{sec:org6852dba}
Sia \(f:[a,b]\to \R\) una funzione continua, e sia
\begin{equation*}
a=x_{0}<x_{1}<\dots<x_{n}=b
\end{equation*}
la partizione di \([a,b]\) di ampiezza \((b-a)/n\).

\begin{definizione}
Il \uline{max-pooling} di \(f\) è la \href{20250630171950-funzione_semplice.org}{funzione semplice}
\begin{equation*}
S_{n}(x) = \sum_{i} M_{i}\,\mathds{1}_{[x_{i-1},x_{i})}(x)
\end{equation*}
dove \(M_{i}\coloneqq \max_{x \in[x_{i-1},x_{i}]}f(x)\).
\end{definizione}

\begin{definizione}
Il \uline{min-pooling} di \(f\) è la \href{20250630171950-funzione_semplice.org}{funzione semplice}
\begin{equation*}
s_{n}(x) = \sum_{i} m_{i}\,\mathds{1}_{[x_{i-1},x_{i})}(x)
\end{equation*}
dove \(m_{i}\coloneqq \min_{x \in[x_{i-1},x_{i}]}f(x)\).
\end{definizione}

\begin{definizione}
L'\uline{average-pooling} di \(f\) è la \href{20250630171950-funzione_semplice.org}{funzione semplice}
\begin{equation*}
A_{n}(x) = \sum_{i} a_{i}\,\mathds{1}_{[x_{i-1},x_{i})}(x)
\end{equation*}
dove \(a_{i}\coloneqq \frac{n}{b-a}\int_{x_{i-1}}^{x_{i}} f(x)\dif x = \frac{1}{x_{i}-x_{i-1}}\int_{x_{i-1}}^{x_{i}} f(x)\dif x\).
\end{definizione}
\begin{thm}
Se \(f:[a,b]\to \R\) è \href{20250103103252-funzione_continua.org}{continua}, allora \(S_{n}, s_{n}, A_{n}\) \href{20250629105745-convergenza_uniforme.org}{convergono uniformemente} ad \(f\) su \([a,b]\) per \(n\to\infty\).
\end{thm}
Si denoti ora con \(\mathcal{P}_{n}\) l'operatore che a \(g(x)\) associa il corrispondente max/min/avg pooling: \(s_{n}\), \(S_{n}\), \(A_{n}\); sia
\begin{equation*}
T_{a}\circ g(x) \coloneqq g(x-a).
\end{equation*}

\begin{prop}
Se \(f:[a,b]\to \R\) è continua, allora esiste \(n \in\N\) ed esiste \(\varepsilon_{n} >0\) tali che
\begin{equation*}
\forall |a|<\varepsilon_{n}\qquad\mathcal{P}_{n}(T_{a}\circ f) = \mathcal{P}_{n}(f).
\end{equation*}
\end{prop}

The previous property provides stability of the pooling under small input variations.
\subsection{Pooling Layer in una rete neurale}
\label{sec:orgda31fb3}
\begin{definizione}
Si consideri una \href{20250624155858-neurone_artificiale.org}{rete neurale} \href{20250624155858-neurone_artificiale.org}{feedforward}. Il layer \(\ell\) è detto \uline{di pooling} se:
\begin{enumerate}
\item il layer \(\ell-1\) è diviso in \(N\) gruppi;
\item i neuroni del layer \(\ell-1\) che appartengono alla stessa classe sono collegati ad un unico neurone nel layer \(\ell\);
\item ci sono \(N\) neuroni nel layer \(\ell\), ciascuno dei quali ha funzione di attivazione \(\max\).
\end{enumerate}
\end{definizione}

\begin{figure}
\begin{equation*}
\begin{tikzcd}[ampersand replacement=\&,cramped,row sep=small]
	{(\ell-1)} \&\&\& {(\ell)} \\
	\\
	{x_{1,1}} \\
	\vdots \&\& \bullet \& {y_1=\max\set{x_{1,1},\dots,x_{1,p}}} \\
	{x_{1, p}} \\
	\\
	{x_{2,1}} \\
	\vdots \&\& \bullet \& {y_2=\max\set{x_{2,1},\dots,x_{2,q}}} \\
	{x_{2,q}} \\
	\\
	\vdots \\
	\\
	{x_{N, 1}} \\
	\vdots \&\& \bullet \& {y_N=\max\set{x_{N,1},\dots,x_{N,r}}} \\
	{x_{N,r}}
	\arrow[from=3-1, to=4-3]
	\arrow[from=4-1, to=4-3]
	\arrow[from=4-3, to=4-4]
	\arrow[from=5-1, to=4-3]
	\arrow[from=7-1, to=8-3]
	\arrow[from=8-1, to=8-3]
	\arrow[from=8-3, to=8-4]
	\arrow[from=9-1, to=8-3]
	\arrow[from=13-1, to=14-3]
	\arrow[from=14-1, to=14-3]
	\arrow[from=14-3, to=14-4]
	\arrow[from=15-1, to=14-3]
\end{tikzcd}
\end{equation*}
\caption{Un layer di pooling}
\end{figure}

Si vuole studiare l'\href{20250714154153-sigma_algebra_come_campo_di_informazione.org}{informazione contenuta nel Layer di Pooling}. Se i neuroni del layer \(\ell-1\) producono come output delle \href{20250711175937-variabile_aleatoria.org}{variabili aleatorie}
\begin{equation*}
X_{11},\dots,X_{1p},\quad X_{21},\dots,X_{2p},\quad\dots\quad X_{N{1}},\dots, X_{Np}
\end{equation*}
(si è supposto senza perdita di generalità che ogni gruppo contenga lo stesso numero \(p\) di neuroni), allora i neuroni del layer di pooling producono
\begin{equation*}
Y_{i} = \max\set{X_{i{1}},\dots,X_{ip}}
\end{equation*}

L'informazione contenuta nel layer \(\ell\) è\footnote{Con \(\sigma(X)\) si intende la \href{20250714154501-sigma_algebra_generata_da_una_variabile_aleatoria.org}{\(\sigma\)-algebra generata da \(X\)}.}
\begin{equation*}
\sigma(Y) \coloneqq \sigma(Y_{1},\dots,Y_{N}) = \sigma\left[\bigcup\sigma(Y_{i})\right]
\end{equation*}
Siccome \href{20250714162717-sigma_algebra_generata_dal_massimo_di_variabili_aleatorie.org}{si è dimostrato} che \(\sigma(Y_{i}) \subseteq \bigcap \sigma(X_{ij})\)
\begin{align*}
\sigma(Y) &= \sigma\left[\bigcup\sigma(Y_{i})\right]\subseteq\sigma\left[\bigcup_{i=1}^{N}\bigcap_{j=1}^{p} \sigma(X_{ij})\right] = \\
&=\sigma\left[\bigcap_{j=1}^{p}\bigcup_{i=1}^{N}\sigma(X_{ij})\right] \subseteq \bigcap_{j=1}^{p} \sigma\left[\bigcup_{i=1}^{N}\sigma(X_{ij})\right] = \bigcap_{j=1}^{p} \sigma(X_{1j},\dots,X_{Nj})
\end{align*}

Dunque, se per ciascun gruppo del \((\ell-1)\)-layer prendo un neurone qualsiasi \(X_{ij_{i}}\), allora
\begin{equation*}
\sigma(Y) \subseteq \sigma(X_{1j_{1}},\dots,X_{Nj_{N}}).
\end{equation*}
La cosa è significativa, poiché consente di eliminare dei neuroni superflui ogni qualvolta che si presenta un layer di pooling.
\end{document}
